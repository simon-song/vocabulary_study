\mychapter{Webster's Dictionary of Synonyms}

\setitemize{nosep}  % set no itemsep for itemize lists
\leftmargini=7mm  % controls the relative spacing of description list

%%%%%%%%%%%%%%%%%%%%%%%%%%%%%%%%%%%%%%%%%%%%%%%%%%%%%%%%%%%%%%%%%%


%\begin{description}
\begin{description}[style=unboxed] % this can handle multiple line items

%%%%%%%%%%%%%%%%%%%%%
\synitem{abandon, desert, forsake}{abandon etc.} v.
\begin{mynewitemize}
\item \textbf{Abandon} implies surrender of control or possession often with the implication
  that the thing abandoned is left to the mercy of someone or something else.
\item \textbf{Desert} commonly implies previous occupation, companionship, or guardianship
  and often connotes desolation.
\item \textbf{Forsake} often retains connotation of repudiation, frequently
  suggests renunciation, and stresses the breaking off of an assocaiation with
  someone or something.
\item[\P] MARY AUSTIN, The ghost of grandeur that lingers between the walls of 
  \textbf{abandoned} haciendas in New Mexico. 
\item[\P] HEISER, In the frantic rush to escape the insane had usually been forgotten
  and \textbf{abandoned} to horrible death.
\item[\P] \textbf{deserted} farms growing up to bush
\item[\P] \textbf{forsake} the world and all its pleasure
\item[\P] DELAND She was \textbf{forsaken} at the altar.
\end{mynewitemize}

%%%%%%%%%%%%%%%%%%%%%%%%%%%%%%%%%%%%%%%%
\synitem{adduce,advance, allege,cite}{adduce etc.} v.

They may be used interchangeably to mean to bring forward by way of explanation, 
proof, illustration, or demonstration; however, they usually are clearly distinguishable 
in their implications and in their idiomatic associations.

\begin{mynewitemize}
\item One \textbf{adduces} facts, evidence, instances, passages, reasons, arguments
when one presents these in support of a contention.
\item One \textbf{advances} something (as a theory, a proposal, a claim, an argument)
that is in itself contentious when one presents it for acceptance or consideration.
\item \textbf{Allege} may indicate a bringing forward or stating as if needing no proof.
It may on the other hand stress doubt about an assertion or convey a warning about
or a disclaimer of responsibility for the truth of matter under discussion. Its
participial adjective \textit{alleged}, especially, often serves as a disclaimer
of responsibility for the assertion.
\item One \textbf{cites} only something concrete and specific (as a passage from 
a book or a definite instance) when one adduces it in support of a contention;
one \textit{cites} by quoting a passage to give an authority; one \textit{cites}
an instance that serves as a precedent or illustration; one \textit{cites} an
instance that serves as a precedent or illustration; one \textit{cites} define facts
in support of something (as a claim or proposal) advanced.
\end{mynewitemize}

%%%%%%%%%%%%%%%%%%%%%%%%%%%%%%%%%%%%%%%%
\synitem{impertinent, officious, meddlesome, intrusive, obtrusive}{impertinent etc.} a.

\begin{mynewitemize}

\item They applied to persons and their acts and utterances and mean exceeding or tending
to exceed the bounds of propriety regarding the interposition of oneself in
another person's affairs.

\item \textbf{Impertinent} implies concerning oneself more or less offensively with things
which are another's business or, at least, not in any sense one's own business.

\item \textbf{Officious} implies the offering, often well-meant, of services, attentions,
or assitance that are not needed or that are unwelcome or offensive

\item \textbf{Meddlesome} carries a stronger implication of annoying interference in other
people's affairs; it usually also connotes a prying or inquisitive nature.

\item \textbf{Intrusive} applies largely to persons, actions, or words that reveal a
disposition to thrust oneself into other people's affairs or society or to be
unduly curious about what is not one's concern

\item \textbf{Obtrusive} is similar to \textbf{intrusive}. it also connotes objectionable actions
more than an objectionable disposition and so stresses a thrusting forward of
oneself, as into a position where one can harm more often than help or where one
is unduly or imporperly conspicuous.
\end{mynewitemize}


%%%%%%%%%%%%%%%%%%%%%%%%%%%%%%%%%%%%%%%%
\synitem{abandoned, reprobate, profligate, dissolute}{abandoned etc.} a.

\begin{mynewitemize}
\item They fundamentally mean utterly depraved. 

\item \textit{Abandoned} and \textit{reprobate} were originally applied to 
sinners and to their acts.

\item One who is \textbf{abandoned} by his complete surrender to a life of sin
seems spiritually lost or morally irreclaimable.

\item One who is \textbf{reprobate} is abandoned and therefore rejected by God
or by his fellows; \textit{reprobate} implies ostrcism by or exclusion from a 
social group for a serious offense against its code.

\item \textit{Profligate} and \textit{dissolute} convey little if any suggestion of divine
or social condemnation but both imply complete moral breakdown and self-indulgence
to such an extreme that all standards of morality and prudence are disregarded.

\item  One who is \textbf{profligate} openly and shamelessly flouts all the decencies 
and wastes his substance in dissipation.

\item One who is \textbf{dissolute} has completely thrown off all moral and 
prudential restraints on the indulgence of his appetites.

\end{mynewitemize}


%%%%%%%%%%%%%%%%%%%%%%%%%%%%%%%%%%%%%%%%
\synitem{irritable,fractious,peevish,snappish,waspish,petulant,
        pettish,huffy,fretful,querulous}{irritable etc.} a.

\begin{mynewitemize}
\item They apply to persons or to their moods or dispositions in the sense of
showing impatience or anger without due or sufficient cause.

\item \textbf{Irritable} implies extreme excitability of temperament, often
associated with or arising from fatigue or physical or mental distress, that 
makes one exceedingly easy to annoy or difficult to please.

\item \textbf{Fractious} carries a stronger implication of willfulness or of 
unruliness than \textit{irritable}, and although it also implies extreme
excitability, it suggests even greater loss of self-control; the term is
often applied to animals as well as to persons.

\item \textbf{Peevish} implies childish irritability and a tendency to give
expression to petty complaints or ill-humored trivial critisms.

\item \textbf{Snappish} implies irritability or sometimes peevishness that
manifests itself in sharp, cutting questions, comments, or objections that
discourage conversation or sociability.

\item \textbf{Waspish} stresses testiness rather than irritability,but it 
implies a readiness to sting or hurt others without warrant or without sufficient
warrant.

\item \textbf{Petulant} usually suggests the sulkiness of a spoiled child as
well as peevishness and capricious impatience.

\item \textbf{Pettish} implies sulky or childish ill humor (as of one who
is slighted or offended).

\item \textbf{Huffy} also implies a tendency to take offense without due cause,
but it suggests more of a display of injured pride than \textit{pettish}.

\item \textbf{Fretful} implies irritability and restlessness that may manifest
itself in complaints or in a complaining tone of voice, but often is merely
suggested by a lack of ease and repose.

\item \textbf{Querulous} implies an often habitual discontent that manifests
itself in whining complaints or in fretfulness of temper; it often also suggests
petulance.

%\item[\P]
\end{mynewitemize}


%%%%%%%%%%%%%%%%%%%%%%%%%%%%%%%%%%%%%%%%
%\synitem{sullen,glum,morose,surly,sulky,crabbed,saturnine,\\dour,gloomy}{sullen etc.} v.
\synitem{sullen, glum, morose, surly, sulky, crabbed, saturnine, dour, gloomy}
        {sullen etc.} a.

\begin{mynewitemize}
\item They can mean governed by or showing, especially in one's aspect, a 
forbidding or disagreeable mood or disposition.

\item One is \textbf{sullen} who is, often by disposition, gloomy, silent, and 
ill-humored and who refuses to be sociable, cooperative, or responsive.

\item One is \textbf{glum} who is dismally silent either because of low spirits
or depressing circumstances.

\item One is \textbf{morose} who is austerely sour or bitter and inclined 
to glumness.

\item One is \textbf{surly} who adds churlishness or gruffness of speech and 
manner to sullenness or moroseness.

\item One is \textbf{sulky} who manifests displeasure, discontent, or resentment
by giving way childishly to a fit of peevish sullenness.

\item One is \textbf{crabbed} who is actually or seemingly ill-natured, harsh,
and forbidding. The term often refers to one's aspect and manner of speaking 
and usually implies a sour or morose disposition or a settled crossness.

\item One is \textbf{saturnine} who presents a heavy, forbidding, taciturn 
gloom, but \textit{saturnine} may come close to \textit{sardonic} and then
suggests less a depressing heaviness and gloom than a wry mocking disdain and 
skepticism that is often at least superficially attractive.

\item One is \textbf{dour} who gives a sometimes superficial effect of severity,
obstinacy, and grim bitterness of disposition.

\item One is \textbf{gloomy} who is so depressed by events or conditions
or so oppressed by melancholy that all signs of cheerfulness or optimism
are obscured, so that he appears sullen, glum, or dour as well as
low-spirited.

%\item[\P]
\end{mynewitemize}

%%%%%%%%%%%%%%%%%%%%%%%%%%%%%%%%%%%%%%%%
\synitem{irascible, choleric, splenetic, testy, touchy, cranky, cross}{irascible etc.} a.
\begin{mynewitemize}
\item They mean easily angered or enraged.

\item \textbf{Irascible} implies the possession of a fiery or inflammable temper
or a tendency to flare up at the slightest provocation.

\item \textbf{Choleric} implies excitability of temper, unreasonableness
in anger, and usually an impatient and uniformly irritable frame of mind.

\item \textbf{Splenetic} implies a similar temperament, but one especially 
given to moroseness and fits of bad temper which exhibit themselves in angry,
sullen, or intensely peevish moods, words, or acts.

\item \textbf{Testy} implies irascibility occasioned by small annoyances.

\item \textbf{Touchy} suggests readiness to take offense; it often connotes
undue irritability or oversensitiveness.

\item \textit{Cranky} and \textit{cross} often mean little more than irritable
and difficult to please.

\item \textbf{Cranky} may carry an implication of the possession of set notions, 
fixed ideas, or unvarying standards which predispose one to anger or a show
of temper when others (as in their speech, conduct, requests, or work) do not 
conform to these standards.

\item \textbf{Cross} may imply a being out of sorts that results in irascibility
or irritability but only for the duration of one's mood.

%\item[\P]
\end{mynewitemize}

%%%%%%%%%%%%%%%%%%%%%%%%%%%%%%%%%%%%%%%%
\synitem{abusive, opprobrious, vituperative, contumelious, scurrilous}{abusive etc.} a.
\begin{mynewitemize}
\item They apply chiefly to language or utterances and to persons as they
emply such language: the words agree in meaning coarse, insulting, and contemptuous 
in character or utterance.
\item \textbf{Abusive} means little more than this, all the other terms carry
specific and distinctive implications.
\item \textbf{Opprobrious} suggests the imputation of disgraceful actions or
of shameful conduct: it implies not only abusiveness but also severe, often
unjust, condemnation.
\item \textbf{Vituperative} implies indulgence in a stream of insulting language
especially in attacking an opponent.
\item \textbf{Contumelious} adds to \textit{opprobrious} the implications of
insolence and extreme disrespect and usually connotes the bitter humiliation
of its victim.
\item \textbf{Scurrilous} often approaches \textit{vituperative} in suggesting
attack and abuse but it always implies gross, vulgar, often obscenely ribald 
language.
%\item[\P]
\end{mynewitemize}

%%%%%%%%%%%%%%%%%%%%%%%%%%%%%%%%%%%%%%%%
\synitem{incite, instigate, abet, foment}{incite etc.} v.
\begin{mynewitemize}
\item They are comparable when they mean to spur on to action or to excite
into activity.
\item \textbf{Incite} stresses stirring up and urging on; frequently it 
implies active prompting.
\item \textbf{Instigate}, in contrast with \textit{incite}, unequivocally
implies prompting and responsibility for the initiation of the action; it
also commonly connotes underhandedness and evil intention; thus, one may be 
\textit{incited} but not \textit{instigated} to the performance of a good 
act; one may be \textit{incited} or \textit{instigated} to the commision
of a crime.
\item \textbf{Abet} tends to lose its original implication of baiting or
hounding on and to emphasize its acquired implications of seconding, supporting,
and encouraging.
\item \textbf{Foment} stresses persistence in goading; thus, one who 
\textit{incites} rebellion may provide only the initial stimulus; one
who \textit{foments} rebellion keeps the rebellious spirit alive by 
supplying fresh incitements.
%\item[\P]
\end{mynewitemize}

%%%%%%%%%%%%%%%%%%%%%%%%%%%%%%%%%%%%%%%%
\synitem{vociferous, clamorous, blatant, strident, boisterous, obstreperous}
{vociferous etc.} a.
\begin{mynewitemize}
\item They are comparable when they mean so loud and noisy, especially vocally,
as to compel attention, often unwilling attention.
\item \textbf{Vociferous} implies both loud and vehement cries or shouts; it
often suggests also a deafening quality.
\item \textbf{Clamorous} can imply insistency as well as vociferousness in 
demanding or protesting, but as often it stresses the notion of sustained 
din or confused turbulence.
\item \textbf{Blatant} implies a tendency to bellow or be conspicuously, 
offensively, or vulgarly noisy or clamourous.
\item \textbf{Strident} basically implies a harsh and discordant quality 
characteristic of some noises that are peculiarly distressing to the ear;
it is applied not only to loud, harsh sounds but also to things which, like 
these, irresistibly and against one's will force themselves upon the attention.
\item \textbf{Boisterous} has usually an implication or rowdy high spirits
and flouting of customary order and is applied to persons or things that are 
extremely noisy and turbulent, as though let loose from all restraint.
\item \textbf{Obstreperous} suggests unruly and aggressive noiseness, typically
occurring in resistance to or defiance of authority or restraining influences.
%\item[\P]
\end{mynewitemize}

%%%%%%%%%%%%%%%%%%%%%%%%%%%%%%%%%%%%%%%%
\synitem{caprice, freak, fancy, whim, whimsy, conceit, vagary, crotchet}
        {caprice etc.} n.
\begin{mynewitemize}
\item These words are comparable when denoting an arbitrary notion that
usually lacks a logical basis and therefore may be unsound, impractical,
or even irrational.
\item \textbf{Caprice} emphasizes the lack of apparent motivation and 
implies a certain willfulness or wantonness.
\item \textbf{Freak} suggests an impulsive, seemingly causeless change of
mind, like that of a child or a lunatic.
\item \textbf{Fancy} stresses casualness an lack of reflection in forming
an idea and may sometimes suggest a kind of harmless perverseness in the
idea formed.
\item \textbf{Whim} and \textbf{whimsy} suggest not so much a sudden as
a quaint, fantastic, or humorous turn or inclination, but \textit{whim}
often stresses capriciousness, and \textit{whimsy} fancifulness.
\item \textbf{Conceit} suggests more strongly than \textit{whim} or
\textit{whimsy} the quaint, fantastic, or erratic character of the notion
formed and also may suggest the firmness and persistence with which it
is held.
\item \textbf{Vagary} suggests still more strongly the erratic, extravagant,
or irresponsible character of the notion or fancy.
\item \textbf{Crotchet} implies even more perversity of temper or more 
indifference to right reason than \textit{vagary}; it often is applied 
to a capriciously heretical opinion on some frequently unimportant or
trivial point.
\end{mynewitemize}

%%%%%%%%%%%%%%%%%%%%%%%%%%%%%%%%%%%%%%%%
\synitem{melancholy, dolorous, dolefule, lugubrious, rueful, plantive}
{melancholy etc.} a.
\begin{mynewitemize}
\item They are comparable when they mean expressing or suggesting sorrow
or mourning. All of these words have, to a greater or less extent, weakened
from their original meaning and are often used with a half-humorous
connotation.
\item \textbf{Melancholy} may stress a quality that inspires pensiveness
or sad reflection or awakens mournful thoughts or recollections which are
not only not necessarily painful or disagreeable, but often agreeable, 
especially to the poetic or thoughtful mind.
\item \textbf{Dolorous} describes what is lamentable in its gloom or
dismalness or is exaggerately dismall.
\item \textit{Doleful} and \textit{lugubrious} are also frequently 
applied to what is exaggeratedly dismal or dreary, but \textbf{doleful}
connotes a weight of woe, and \textbf{lugubrious}, an undue, and often
an affected, heaviness or solemnity.
\item \textbf{Rueful} implies sorrow and regret but it often suggests a
quizzical attitude.
\item \textbf{Plaintive} applies chiefly to tones, sounds, utterances, or 
rhythms that suggest complaint or mourning or that excite pity or compassion.
%\item[\P]
\end{mynewitemize}

%%%%%%%%%%%%%%%%%%%%%%%%%%%%%%%%%%%%%%%%
\synitem{inflexible, inexorable, obdurate, adamant, adamantine}
{inflexible etc.} a.
\begin{mynewitemize}
\item They mean not to be moved from or changed in a predetermined course
or purpose. All are applicable to persons, decisions, laws, and principles;
otherwise, they vary in their applications.
\item \textbf{Inflexible} usually implies firmly established principles
rigidly adhered to; sometimes it connotes resolute steadfastness, sometimes 
slavish conformity, sometimes more pigheadedness.
\item \textbf{Inexorable}, when applied to persons, stresses deafness to 
entreaty. When applied to decisions, rules, laws, and their enforcement,
it often connotes relentlessness, ruthlessness, and finality beyond
question.
\item \textbf{Obdurate} is applicable chiefly to persons and almost 
invariably implies hardness of heart or insensitiveness to such external
influences as divine grace or to appeals for mercy, forgiveness, or
assistance.
\item \textbf{Adamant} and \textbf{adamantine} usually imply extraordinary
strength of will or impenetrability to temptation or entreaty.
%\item[\P]
\end{mynewitemize}

%%%%%%%%%%%%%%%%%%%%%%%%%%%%%%%%%%%%%%%%
\synitem{infuse, suffuse, imbue, ingrain, inoculate, leaven}{infuse etc.} v.
\begin{mynewitemize}
\item They can all mean to introduce one thing into anther so as to affect
it throughout.
\item \textbf{Infuse} implies a permeating like that of infiltering fluid,
usually of something which imbues the recipient with new spirit, life, or
vigor or gives it or him a new cast or new significance.
\item \textbf{Suffuse} implies an overspreading of a surface by or a 
spreading through an extent of something that gives the thing affected
a distinctive or unusual color, aspect, texture, or quality.
\item \textbf{Imbue} implies the introduction of something that enters
so deeply and so exensively into the thing's substance or nature that
no part is left untouched or unaffected; unlike \textit{infuse}, which it
otherwise closely resembles, \textit{imbue} takes as its object the person
or thing affected, not the thing that is introduced.
\item \textbf{Ingrain} is found in the past participle or passive forms 
only; like \textit{imbue}, it implies an incorporation of something
comparable to a pervading dye with the body, substance, or nature of
whatever is affected, but unlike \textit{imbue}, it takes for its object
or, when the verb is passive, as its subject the thing introduced rather
than the person or thing affected.
\item \textbf{Inoculate} implies imbuing a person with something that 
alters him in a manner suggestive of a disease germ or an antigen. Often,
the term implies an introduction of an idea, a doctrine, an emotion, or
a taste by highly surreptitious or artificial means, in order to achieve 
a desired end; less often, it additionally implies an evil and destructive
quality in what is introduced.
\item \textbf{Leaven} implies a transforming or tempering of a body or
mass by the introduction of something which enlivens, elevates, exalts, 
or, occasionally, causes disturbance, agitation, or corruption.
\end{mynewitemize}

%%%%%%%%%%%%%%%%%%%%%%%%%%%%%%%%%%%%%%%%
\synitem{gaudy, tawdry, garish, flashy, meretricious}{gaudy etc.} a.
\begin{mynewitemize}
\item They are comparable when meaning vulgar or cheap in its showiness.
\item Something is \textbf{gaudy} which uses gay colors and conspicuous
ornaments or ornamentation lavishly, ostentatiously, and tastelessly.
\item Something is \textbf{tawdry} which is not only gaudy but cheap and sleazy.
\item Something is \textbf{garish} which is distressingly or offensively bright.
\item Something is \textbf{flashy} which dazzles for a moment but then reveals 
itself as shallow or vulgar display.
\item Something is \textbf{meretricious} which allures by false or deceitful
show (as of worth, value, or brilliancy).
\end{mynewitemize}

%%%%%%%%%%%%%%%%%%%%%%%%%%%%%%%%%%%%%%%%
\synitem{limp, floppy, flaccid, flappy, flimsy, sleazy}{limp etc.} a.
\begin{mynewitemize}
\item They mean deficient in firmness of texture, substance, or structure
and therefore unable to keep a shape or in shape.
\item \textbf{Limp} applies to something that lacks or has lost the stiffness
or firmness necessary to keep it from drooping or losing its original
sturdiness or freshness.
\item \textbf{Floopy} applies to something that sags or hangs limply.
\item \textbf{Flaccid} implies a loss or lack of elasticity or resilience and
therefore an incapacity to return to an original shape or condition or to 
keep a desired shape; the term applies primarily to flesh and other living tissues.
In extendend use the term implies lack of force or energy or substance.
\item \textbf{Flabby} applies to something that is so soft that it yields
readily to the touch or is easily shaken. In extended use the term implies
the loss or lack of what keeps a thing up or in good sound condition; it
often carries suggestions of spinelessness, spiritlessness, or lethargy.
\item \textbf{Flimsy} applies to something that by its looseness of structure
or insubstantiality of texture cannot hold up under use or strain. In extended
use the term applies to whatever is so frail or slight as to be without 
value or endurance.
\item \textbf{Sleazy} applies especially to flimsy textiles, but it often
suggests, as \textit{flimsy} need not, fraud or carelessness in its manufacture.
In extended use the term may stress lack or inferiority of standards or inferiority
of the resultant product but often its suggestion is one of cheap shabby inferiority.
\end{mynewitemize}

%%%%%%%%%%%%%%%%%%%%%%%%%%%%%%%%%%%%%%%%
\synitem{ominous, portentous, fateful, inauspicious, unpropitious}{ominous etc.} a.
\begin{mynewitemize}
\item They basically mean having a menacing or threatening character or quality. 
\item What is \textbf{ominous} has or seems to have the character of an omen,
especially of an omen forecasting evil; the term commonly suggests a frightening
or alarming quality that bodes no good, and it may imply impending disaster.
\item What is \textbf{portentous} has or seems to have the character of a portent;
\textit{portentous}, however, less often than \textit{ominous} suggests a threatening
character; it usually means little more than prodigious, monstrous, or almost
frighteningly marvelous, solemn, or impressive.
\item What is \textbf{fateful} has or seems to have the quality, character, or
importance decreed for it by fate or suggests inevitability, but the term often 
means little more than momentous or appallingly decisive.
\item What is \textbf{inauspicious} is or seems to be attended by signs that are
distinctly unfavorable. But \textit{inauspicious} usually means nothing more than
unlucky, unfortunate, or unlikely to succeed.
\item What is \textbf{unpropitious} carries or seems to carry no sign of favoring
one's ends or intentions. In its more common extended sense the term means merely
unfavorable, discouraging, or harmful.
%\item[\P]
\end{mynewitemize}

%%%%%%%%%%%%%%%%%%%%%%%%%%%%%%%%%%%%%%%%
\synitem{licentious, libertine, lewd, wanton, lustful, lascivious, libidinous, lecherous}
{licentious etc.} a.
\begin{mynewitemize}
\item They all suggest unchaste habits, especially in being given to or indicative
of imorality in sex relations.
\item \textbf{Licentious} basically implies disregard of the restrains imposed
by law or custom for th enforcement of chastity; the term stresses looseness of 
life and of habits rather than the imperiousness of one's desires.
\item \textbf{Libertine} suggests a more open and a more habitual disregard of 
moral laws, especially those pertaining to the sex relations of men and women.
\item \textbf{Lewd} often carries strong connotation of grossness, vileness, and
vulgarity which color its other implications of sensuality, dissoluteness, and
unconcern for chastity. As a result it is applied less often than the preceeding
terms to persons, or to the manners, thoughts, and acts of persons, who retain
in their immorality evidence of breeding, refinement, or gentility.
\item \textbf{Wanton} implies moral irresponsibility or a disposition or way
of life marked by indifference to moral restraints; it often suggests freedom
from restraint comparable to that of animals, thereby connoting lightness,
incapability for faithfulness or seriousness, or a generally unmoral attitude.
\item \textbf{Lustful} implies the influence or the frequent incitement of
desires, especially of strong and often unlawful sexual desires.
\item \textbf{Lascivious}, like \textit{lewd}, definitely suggests sensuality,
but it carries a clearer implication of an inclination to lustfulness or of a
capacity for inciting lust.
\item \textit{Libidinous} and \textit{lecherous} are the strongest of all 
these terms in their implications of deeply ingrained lustfulness and of
debauchery.
\item \textbf{Libidinous} distinctively suggests a complete surrender to one's
sexual desires.
\item \textbf{Lecherous} clearly implies habitual indulgence of one's lust,
the term often being used when any of the others would seem too weak to 
express one's contempt.
\end{mynewitemize}

%%%%%%%%%%%%%%%%%%%%%%%%%%%%%%%%%%%%%%%%
\synitem{supererogatory, gratuitous, uncalled-for, wanton}{supererogatory etc.}
a.
\begin{mynewitemize}
\item They are comparable when they mean given or done freely and without
compulsion or provocation or without warrant or justification.
\item \textbf{Supererogatory} basically implies a giving above or beyond what is
required or is laid down in the laws or rules; the word then suggests a devotion
or loyalty that is not satisfied merely with the doing of what is required and
that finds expression in the performance of additional labors, works, or
services beyond those expected or demanded. In other usage the term is
definitely depreciative in that it implies not a giving freely over and above
what is required but a giving or adding of what is not needed or wanted and is
therefore an embarrassment or encumbrance.
\item \textbf{Gratuituous} may apply to a giving voluntarily without expectation
of recompense, reward, or compensation, but it often stresses a giving without
provocation of something disagreeable, offensive, troublesome, or painful.
\item \textit{Gratuituous} often means little more than \textbf{uncalled-for},
which suggests not only a lack of provocation but a lack of need or
justification and therefore implies impertinence or absurdity, often logical
absurdity.
\item \textbf{Wanton} also implies want of provocation, but it stresses
capriciousness and the absence of a motive except reckless sportiveness or
arbitrariness or pure malice.
%\item[\P]
\end{mynewitemize}

%%%%%%%%%%%%%%%%%%%%%%%%%%%%%%%%%%%%%%%%
\synitem{proud, arrogant, haughty, lordly, insolent, overbearing, supercilious,
disdainful}{proud etc.} v.
\begin{mynewitemize}
\item They can mean in common filled with or showing a sense of one's
superiority and scorn for what one regards as in some way inferior.
\item \textbf{Proud} usually connotes a lofty of imposing manner, attitude, or
appearance that may be interpreted as dignified, elevated, spirited, imperious,
satisfied, contemptuous, or inordinately conceited according to the
circumstances.
\item \textbf{Arrogant} implies a disposition to claim for oneself, often
domineeringly or aggressively, more consideration or importance than is
warranted or justly due.
\item \textbf{Haughty} implies a strong consciousness of exalted birth, station,
or character, and a more or less obvious scorn of those who are regarded as
beneath one.
\item \textbf{Lordly} usually suggests pomposity, strutting, or an arrogant
display of power or magnificence.
\item \textbf{Insolent} implies both haughtiness and extreme contemptuousness;
it carries a stronger implication than the preceding words of a will to insult
or affront the person so treated.
\item \textbf{Overbearing} suggests a bullying or tyrannical disposition, or
intolerable insolence.
\item \textbf{Supercilious} stresses such superficial aspects of haughtiness as
a lofty patronizing manner intended to repel advances. It refers to one's
behavior to others rather than to one's conceit of onself, though the latter is
always implied; often it suggests not only scorn but also incivility.
\item \textbf{Disdainful} implies a more passionate scorn for what is beneath
one than does \textit{supercilious}; it as often as not suggests justificable
pride or justificable scorn.
%\item[\P]
\end{mynewitemize}

%%%%%%%%%%%%%%%%%%%%%%%%%%%%%%%%%%%%%%%%
\synitem{wander, stray, roam, ramble, rove, range, prowl, gad, gallivant,
traipse, meander}{wander etc.} v.
\begin{mynewitemize}
\item They can mean to move about more or less aimlessly or without a plan from
place to place or from point to point. Most of these verbs may imply walking,
but most are not restricted in their reference to human beings or to any
particular means of locomotion.
\item \textbf{Wander} implies the absense of a fixed course or more or less
indifferent to a course that has been fixed or otherwise indicated; the term may
imply the movement of a walker whether human or animal or of any traveler, but
it may be used of anything capable of direction or control that is permitted to
move aimlessly.
\item \textbf{Roam} carries a stronger suggestion of freedom and ofscope than
\textit{wander}; it usually carries no implication of a definite object or goal,
but it seldom suggests futility or fruitlessness and often connotes delight or
enjoyment.
\item \textbf{Ramble}, in contrast, suggests carelessness in wandering and more
or less indifference to one's path or goal. It often, especially in its extended
uses, implies a straying beyond bounds, an inattention to details that ought to
serve as guides, or an inability to proceed directly or under proper
restrictions.
\item \textbf{Rove} comes close to \textit{roam} in its implication of wandering
over extensive territory, but it usually carries a suggestion of zest in the
activity, and does not preclude the possibility of a definite end or purpose.
\item \textbf{Range} is often used in place of \textit{rove} without loss; it
may be preferred when literal wandering is not implied or when the stress is on
the sweep of territory covered rather than on the form of locomotion involved.
\item \textbf{Prowl} implies a stealthy of furtive roving, especially in search
of prey or booty. It is used not only of animals but often also human beings
intent on marauding, but it is also applied with little or no connotation of an
evil intention to persons, especially those of a restless or vagabond
temperament, who rove, often singly, through the strees or the fields in a quiet
and leisurely manner.
\item \textit{Gad} and \textit{gallivant} imply a wandering or roving especially
by those who ought to be under restrictions (as servants, children, husbands or
wives, and persons who have not much strength or enough money).
\item \textbf{Gad}, usually with \textit{about}, may suggest a bustling from
place to place idly or for the most trivial ends and often to the detriment of
one's actual duties.
\item \textbf{Gallivant} adds to \textit{gad} the implication of a search for
pleasure or amusement or the use of an opportunity to display one's finery.
\item \textbf{Traipse}, which commonly suggests more vigor in movement and less
aimlessness in intent than the remaining terms, may come close to \textit{come,
go}, or \textit{travel} in meaning. Even when used with reference to an erratic
course \textit{traipse} ordinarily implies a positive purpose, or stresses a
bustling activity, or a wearying expenditure of energy. Sometimes the term loses
most of its reference to a course and then stresses a dashing or flaunting
manner of going.
\item \textbf{Meander} may be used in reference to persons and animals but more
characteristically in reference to things (as streams, paths, or roads) that
follow a winding or intricate course in such a way as to suggest aimless or
listless wandering.
%\item[\P]
\end{mynewitemize}

%%%%%%%%%%%%%%%%%%%%%%%%%%%%%%%%%%%%%%%%
\synitem{boast, brag, vaunt, crow, gasconade}{boast etc.} v.
\begin{mynewitemize}
\item
\item \textbf{}
%\item[\P]
\end{mynewitemize}



%%%%%%%%%%%%%%%%%%%%%%%%%%%%%%%%%%%%%%%%
%\synitem{ }{ } v.
%\begin{mynewitemize}
%\item
%\item \textbf{}
%%\item[\P]
%\end{mynewitemize}


% %%%%%%%%%%%%%%%
% \myitem{ }
% \begin{myitemize}
% \item{[ ]}
% \item
% \item Burke, A Vindication of Natural Society,
%    \textit{ }
% \end{myitemize}




\end{description}
