\chapter*{Webster}
\markright{OED: Webster}{}
\addcontentsline{toc}{chapter}{OED: Webster}%

%\setitemize{nosep}  % set no itemsep for itemize lists
%\leftmargini=7mm  % controls the relative spacing of description list

%%%%%%%%%%%%%%%%%%%%%%%%%%%%%%%%%%%%%%%%%%%%%%%%%%%%%%%%%%%%%%%%%%
\noindent The list of words is selected from \textit{Webster Dictionary of Synonyms.}

\begin{description}[wide, labelwidth=!, labelindent=0pt] % noindent

%%%%%%%%%%%%%%%%%%%%%%%%%%%%%%%%
\myitem{hoard} v.

\noindent \phonetic{(hɔəd)}

\noindent [OE. hordian, f. hord hoard n.1 (Cf. Goth. huzdjan, OHG. gihurten, MHG. gehürten, MG. gehorden, which belong to a different conjugation.)]
\vspace{-0.3cm}

\begin{myenumerate}

\itembf{1.} trans. To amass and put away (anything valuable) for preservation, security, or future use; to treasure up: esp. money or wealth.

\P c1000 ÆLFRIC  \textit{Hom.} II. 104 Hordiað eowerne goldhord on heofenum.
\P c1200 ORMIN  12281 Grediȝliȝ to sammnenn all \& hordenn þatt tu winnesst.    
\P 1526  \textit{Pilgr. Perf.} (W. de W. 1531) 98 b, To helpe other with them, and not inordynately to hoorde \& kepe them.    
\P 1530 PALSGR. 588/2, I hourde, je amasse. Declared in ‘I hoorde’.    
\P 1535 COVERDALE  \textit{Prov.} xi. 26 Who so hoordeth vp his corne, shalbe cursed amonge the people.    
\P 1548 UDALL, etc. \textit{Erasm. Par. Matt.} v. 36 Whorded and heaped up.    ? a 
\P 1550 in  \textit{Dunbar's Poems} (1893) 306 Gif thow hes a benefice, Preiss nevir to hurde the kirkis gude.    
\P 1573 G. HARVEY  \textit{Letter-bk.} (Camden) 8 He did not wel to hord it up.    
\P 1583 STANYHURST  \textit{Æneis} ii. (Arb.) 68 Theere Troian treasur is hurded.    
\P 1615 G. SANDYS  \textit{Trav.} 136 The Granaries of Joseph: wherein he hoorded corne.    
\P 1635 A. STAFFORD  \textit{Fem. Glory} (1869) 124 Whereof the Rich hide and hoard up their wealth.    
\P 1702 ADDISON  \textit{Dial. Medals} (1727) 25 Hoording up such pieces of money.    
\P 1840 HOOD  \textit{Kilmansegg, Moral}, Gold! Gold! Gold! Gold!‥Hoarded, barter'd, bought and sold.    
\P 1878 JEVONS  \textit{Prim. Pol. Econ.} 22 If the rich man actually hoards up his money in the form of gold or silver, he gets no advantage from it.

\itembf{b.} absol.

\P 1000 ÆLFRIC  \textit{Hom.} I. 66 \phonetic{Seðe hordað, and nat hwam he hit ᴁegadarað.}
\P a1300 E.E.  \textit{Psalter} xxxviii. 7 [xxxix. 6] He hordes, and he wate noght To wham þat he samenes oght.    
\P 1590 SPENSER  \textit{F.Q.} i. x. 38 He‥Ne car'd to hoord for those whom he did breede.    
\P 1842 TENNYSON  \textit{Ulysses} 5 A savage race, That hoard, and sleep, and feed, and know not me.    
\P 1860 EMERSON  \textit{Cond. Life, Wealth} Wks. (Bohn) II. 349 They should own who can administer; not they who hoard and conceal.

\itembf{2.} fig. and transf. To keep in store, cherish, treasure up, conceal (e.g. in the heart).

\P 1340  \textit{Ayenb.} 182 Þet greate lost þet god hordeþ and wyteþ to ham þet ouercomeþ þe aduersetes of þise wordle.    c 
\P 1380 WYCLIF  \textit{Wks.} (1880) 321 Crist‥lokyng on þe citee‥wepte þer upon for greet synne þat it hoordede.    
\P 1596 SPENSER  \textit{F.Q.} iv. xi. 43 The goodly Barow which doth hoord Great heapes of salmons in his deepe bosome.    
\P 1699 DRYDEN  \textit{Ep. to J. Driden} 117 You hoard not health for your own private use; But on the public spend the rich produce.    
\P 1789 BURKE  \textit{Corr.} (1844) III. 119 Revenge will be smothered and hoarded.    
\P 1821 B. CORNWALL  \textit{Mirandola} iv. i, Half of the ills we hoard within our hearts Are ills because we hoard them.    
\P 1870 MORRIS  \textit{Earthly Par.} I. i. 370.

\itembf{3.} intr. in reflexive or passive sense: To lie treasured up, lie hid. Obs. rare.

\P 1567 TURBERV.  \textit{Epit. \& Sonn.} Wks. (1837) 300 In common weales what beares a greater sway Than hidden hate that hoordes in haughtie brest?
\end{myenumerate}


%%%%%%%%%%%%%%%%%%%%%%%%%%%%%%%%
\myitem{inculpate} v.

\noindent \phonetic{(ˈɪnkʌlpeɪt, ɪnˈkʌlpeɪt)}

\noindent [f. med. L. inculpāt-, ppl. stem of inculpāre, f. in- (in-2) + culpāre to blame; cf. exculpate. As to the pronunciation, see contemplate.]
\vspace{-0.3cm}

\begin{myenumerate}

\itembf{1.} trans. To bring a charge against; to accuse; to blame, find fault with.

\P 1799 S. TURNER  \textit{Anglo-Sax.} I. iii. iii. 173 Gildas inculpates him for having destroyed his uncle.    
\P 1833 I. TAYLOR  \textit{Fanat.} vi. 185 We should be slow to inculpate motives.    
\P 1846 DE QUINCEY  \textit{Glance Wks. Mackintosh} Wks. XIII. 65 The poor lady could have had no rational motive for inculpating herself.

\itembf{2.} To involve in a charge; to incriminate.

\P 1897 M. KINGSLEY  \textit{W. Africa} 427 Attempting to exculpate himself and inculpate Dr. Nassau for not having told him one was necessary.

\noindent Hence \textbf{inculpated}, \textbf{inculpating} ppl. adjs.

\P 1837 CARLYLE  \textit{Fr. Rev.} III. iii. ix, Will not perhaps the inculpated Deputies consent to withdraw voluntarily?    
\P 1864  \textit{Daily Tel.} 8 June, Major-General Dix‥was‥ordered forthwith to stop the further publication of the inculpated newspapers.    
\P 1892  \textit{Pall Mall G.} 15 Mar. 2/3, I think it is generally felt that the inculpating lie is more serious than the exculpating falsehood.
\end{myenumerate}


%%%%%%%%%%%%%%%%%%%%%%%%%%%%%%%%
\myitem{incriminate} v.

\noindent \phonetic{(ɪnˈkrɪmɪneɪt)}

\noindent [f. ppl. stem of med.L. incrīmināre to accuse, f. in- (in-2) + crīmināre to criminate; perh. partly due to F. incriminer (1791 in Hatz.-Darm.).]
%\vspace{-0.3cm}

\noindent trans. To charge with a crime; to involve in an accusation or charge.

\P 1730-6 BAILEY (folio), To incriminate, to recriminate.
\P 1828 WEBSTER,  \textit{Incriminate}, to accuse; to charge with a crime or fault.    
\P 1862 WRAXALL  \textit{Hugo's ‘Misérables’} v. xi, Their theory is incriminated.    
\P 1874 SYMONDS  \textit{Sk. Italy \& Greece} (1898) I. xi. 220 It would be wrong to incriminate the Order of S. Francis by any suspicion.    
\P 1885 \textit{Manch.  Exam.} 6 June 5/3 Evidence which will incriminate others while it clears themselves.

\noindent Hence \textbf{incriminated, incriminating} ppl. adjs.

\P 1858  \textit{Times} 27 Nov. 8/2 Any incriminated phrase of Montalembert's pamphlet.    
\P 1863 KINGLAKE  \textit{Crimea} I. xiv. 231 This Maupas, or de Maupas‥deliberately offered to arrange that incriminating papers‥should be secretly placed in the houses of the men whom he wanted to have accused.    
\P 1875 STUBBS  \textit{Const. Hist.} III. xix. 343 If the ordinary claimed the incriminated clerk.    
\P 1882 TRAILL  \textit{Sterne} iv. 40 An excuse for the incriminated passage.


%%%%%%%%%%%%%%%%%%%%%%%%%%%%%%%%
\myitem{arraign} v.

\noindent \phonetic{(əˈreɪn)}

\noindent [a. AF. araine-r, areine-r, arene-r, OF. arais-, areis-, aresnier:—L. adratiōnāre, f. ad to + ratiōnāre to reason, talk reasonably, talk, f. ratiōn-em reason, reasoning, discourse. The later F. araisonner was adopted in Eng. as areason.]
%\vspace{-0.3cm}

\begin{myenumerate}

\itembf{1.} trans. To call (a person) to account, or to answer for himself; to interrogate, examine. Obs.

\P 1325  \textit{E.E. Allit.} P. C. 191 Arayned hym [Jonah] ful runyschly what raysoun he hade‥to slepe so faste.    c 
\P 1360 MERCY 85 in  \textit{E.E.P.} (1862) 121 Þeose are þe werkes of Merci, Of whuche crist wol vs areyne.    
\P 1387 TREVISA  \textit{Higden} Rolls Ser. IV. 303 Augustus areyned [interrogavit] hym and seide.    
\P 1447 O. BOKENHAM  \textit{Lyvys of Seyntys} 15 He hyr thus areynyd wyth a pale faas.

\itembf{2.} esp. To call upon one to answer for himself on a criminal charge; to indict before a tribunal. Hence gen. To accuse, charge with fault.

\P 1400  \textit{Leg. Rood} 147 To a-rene Wrecches þat wraþþe þi chylde.    
\P 1450 SOMNER in  \textit{4 C. Eng. Lett.} 4 He was arreyned upon the appechements and fonde gylty.    c 
\P 1450 HENRYSON  \textit{Mor. Fa.} 42 The Sheepe againe before the Wolfe arenȝied.    
\P 1528 MORE  \textit{Heresyes} iii. Wks. 212/2 Yt were arreygned for a felonye.    
\P 1542 BRINKLOW  \textit{Complaynt} v. (1874) 18 The day whan ye shal be reygned at the iudgement seate of God.    
\P 1611 SHAKES.  \textit{Wint. T.} iii. ii. 14 Thou art here accused and arraigned of High Treason.    
\P 1722 DE FOE  \textit{Moll Fl.} (1840) 310, I was carried down to the Sessions house, where I was arraigned.    
\P 1754 RICHARDSON  \textit{Grandison} IV. xxiv. 177 Lady Olivia is grieved‥and arraigns herself and her wicked passion.    
\P 1876 FREEMAN  \textit{Norm. Conq.} IV. xviii. 129 For that crime he was arraigned‥before the King and his Witan.

\itembf{3.} To accuse of some fault or imperfection, impeach, call in question, find fault with (actions, measures, statements, opinions).

\P 1672 DRYDEN  \textit{Conq. Granada} i. i, Judge-like thou sit'st, to praise or to arraign The flying Skirmish of the darted Cane.    
\P 1772 JUNIUS  \textit{Lett.} Pref. 10 They arraign the goodness of Providence.    
\P 1776 GIBBON  \textit{Decl. \& F.} I. xxiv. 681 He boldly arraigned the abuses of public and private life.    
\P 1820 BYRON  \textit{Mar. Fal.} v. i. 269 You do not then‥arraign our equity?

\itembf{b.} absol.

\P 1746 SMOLLETT  \textit{Reproof} 202 And let me still the sentiment disdain Of him, who never speaks but to arraign.

\itembf{4.} To try, judge. Obs. rare.

\P 1623 HEMING \& COND.  in \textit{Shaks. C. Praise} 145 Though you be a Magistrate of wit, and sit on the Stage‥to arraigne Playes dailie.

\itembf{5.} To sentence, condemn. Obs. rare.

\P 1658 J. ROWLAND  \textit{Mouffet's Theat. Ins.} 1102 When  they finde they are arraigned to die.
\end{myenumerate}


%%%%%%%%%%%%%%%%%%%%%%%%%%%%%%%%
\myitem{piquant} a. (n.)

\noindent \phonetic{(ˈpiːkənt)}

\noindent [a. F. piquant (picquant), pr. pple. of piquer to prick, sting: see 
pick v.1, pique v.1 The form piccant was ad. It. piccante. In 19th c. authors, 
piquante (\phonetic{piˈkɑːnt}) usually represents the Fr. fem. piquante 
(\phonetic{pikɑ̃t}).]
\vspace{-0.3cm}

\begin{myenumerate}

\itembf{A.} adj.

\itembf{1.} That pierces or stings; esp. sharp or stinging to the feelings; keen, trenchant; severe, bitter. Chiefly fig. Obs. or arch.

\P 1521 WOLSEY in  \textit{St. Papers Hen. VIII}, I. 43 Notwith$\sim$standing the pickande wordes conteigned in thEmperours letters.    
\P 1549 CHALONER  \textit{Erasm. on Folly} M iij, Who is he so blunt and restiue, that could not with theyr pickant spurres be quickened?    
\P 1591 CONINGSBY  \textit{Siege Rouen} in \textit{Camden Misc.} (1847) I. 29 This daie the marshall wrote a letter‥a lytle pickante.    
\P 1651  \textit{Life Father Sarpi} (1676) 32 By some picquant words or arguteness to put them into choler.    
\P 1654 tr.  \textit{Scudery's Curia Pol.} 6 The pangs of the Gout are so sharpe and picquant.    
\P 1789 E. DARWIN  \textit{Let.} in \textit{Life} (1879) 37 Never to make any piquant or angry answer.    
\P 1868 LANIER  \textit{Jacquerie} i. 131 Urged him on With piquant spur.

\itembf{b.} Sharp-pointed, peaked. Obs. rare.

\P 1650 BULWER  \textit{Anthropomet.} 261 When sharp piquant Toes were altogether in request.

\itembf{2.} Agreeably pungent or sharp of taste; sharp, stinging, biting; stimulating or whetting to the appetite; appetizing.

\P 1645 HOWELL  \textit{Lett.} I. v. xxxviii, [A cook] excellent for a pickant sawce and the haugou.    
\P 1656 STANLEY  \textit{Hist. Philos.} v. II. 78 The differences of Sapours are seven; sweet, sharp, sowre, picqueant, salt, acid, bitter.    
\P 1704 ADDISON  \textit{Italy} (1733) 301 As piquant to the Tongue as Salt it self.    
\P 1827 DISRAELI  \textit{Viv. Grey} v. xiii, As piquant as an anchovy toast.    
\P 1840 THACKERAY  \textit{Paris Sk.-bk.} (1872) 227 A piquant sauce for supper.

\itembf{3.} fig. That acts upon the mind as a piquant sauce, or the like, upon the palate; that stimulates or excites keen interest or curiosity; pleasantly stimulating or disquieting.

\P 1695 \textit{Whether  Parlt. be not in Law dissolved}, etc. 47 It falls below being piquant, and keeps within the Limits and Precincts of Modesty.    
\P 1706 \textit{Art  of Painting} 319 He [Rembrandt] design'd an infinite Number of Thoughts, that were as sensible and as Picquant as the Productions of the best Masters.    
\P 1792 M. WOLLSTONECRAFT  \textit{Rights Wom.} iv. 144 Their husbands‥leave home to seek for a more agreeable—may I be allowed to use a significant French word?—piquant society.    
\P 1819 J. W. CROKER in \textit{C. Papers} 24 Aug., Your notices of literary works should be short, light, and piquant.    
\P 1849 C. BRONTË  \textit{Shirley} vi, She disapproved entirely of the piquant neatness of Caroline's costume.    
\P 1879 TOURGEE  \textit{Fool's Err.} xxxv. 235 These charms combined to render her an exceedingly piquant and charming maiden.    
\P 1885 MABEL  \textit{Collins Prettiest Woman} xv, This lovely girl had not Wanda's piquant, pretty face.

\itembf{b.} After F. \textit{piquante} fem.

\P 1823 SCOTT  \textit{Peveril} xxxix, The monkey has a turn for satire, too, by all that is piquante.    
\P 1850 SMEDLEY  \textit{F. Fairlegh} (1894) 52 Lucy's‥what you call piquante.    
\P 1873 SMILES  \textit{Huguenots Fr.} i. i. (1881) 3 That picquante letter-writer, Madame de Sévigné.    
\P 1898 RIDER  \textit{Haggard Dr. Therne} i. 15 The face of a rather piquante and pretty girl.

\itembf{B.} n. rare. That which is piquant. \textbf{a.} A hedgehog's prickle; \textbf{b.} A piquant dish; a whet.

\P 1835 KIRBY  \textit{Hab. Anim.} II. xvii. 213 The two most remarkable animals in the insectivorous tribe‥are the mole, and the hedgehog,‥the latter for its piquants, and the former for its hand turned outwards.    
\P 1843 \textit{P. Parley's  Ann.} IV. 239 He pined for the piquants—he had dreams of the savouries.

\noindent Hence \textbf{piquantly} adv., in a piquant manner; \textbf{piquantness} (rare), piquancy.

\P 1697 POTTER  \textit{Antiq. Greece} i. xxvi. (1715) 158 If an Orator‥hath been piquantly Censorious.    
\P 1703 \textit{Art \& Myst.  Vintners} 17 Claret loseth much of its Briskness and Picquantness.    
\P 1727 BAILEY  vol. II, \textit{Piquantness}, ‥ sharpness, bitingness.    
\P 1882 W. H. BISHOP in \textit{Harper's Mag.} Dec. 54/2 The village is piquantly foreign.    
\P 1922 JOYCE  \textit{Ulysses} 399 Blushing piquantly and whispering in my ear.    
\P 1955  \textit{Times} 10 May 3/7 M. Claude Barma's production was most piquantly revealing.    
\P 1971  \textit{Daily Tel.} 16 July 11/8 With a cast of two, it presents a piquantly rounded theme.
\end{myenumerate}


%%%%%%%%%%%%%%%%%%%%%%%%%%%%%%%%
\myitem{suave} a. (adv.)

\noindent \phonetic{(swɑːv, formerly also sweɪv)}

\noindent [a. F. suave (16th cent.), a ‘learned’ formation which took the place of the ‘popular’ OF. soef, suef (suaif):—L. suāvis sweet, agreeable:—*swādwis, f. swād- (see sweet a.).]
\vspace{-0.3cm}

\begin{myenumerate}

\itembf{1.} Pleasing or agreeable to the senses or the mind; sweet.

\P 1560 A. SCOTT  \textit{Poems} (S.T.S.) vii. 29 Adew þe fragrant balme suaif, And lamp of ladeis lustiest!    
\P 1598 QUEEN ELIZABETH  \textit{Plutarch} ix. 3 The suafes thing that Silence dothe Expres.    
\P 1694 MOTTEUX  \textit{Rabelais} v. Epist. 251 These Times‥alterate the suavest Pulchritude.    
\P 1849 C. BRONTË  \textit{Shirley} xxvi, To whom the husky oat-cake was from custom suave as manna.    
\P 1859 MISS MULOCK  \textit{Life for a Life} xvii, To break the suave harmony of things.    
\P 1878 H. S. WILSON  \textit{Alpine Ascents} iii. 99 The suaver white hoods of snow summits.

\itembf{2.} Gracious, kindly. Also advb. Sc. Obs.

\P 1501 DOUGLAS  \textit{Pal. Hon.} iii. ii, Thir musis gudelie and suaue.    c 
\P 1550 ROLLAND  \textit{Crt. Venus} ii. 76 The nine Musis sweit and swaue.    c 
\P 1560 A. SCOTT  \textit{Poems} (S.T.S.) i. 214 Resaif swaif, and haif ingraif it heir.    Ibid. xxxvi. 73 Sweit Lord, to Syon be suave.

\itembf{3.} Of persons, their manner: Blandly polite or urbane; soothingly agreeable. (Cf. suavity 4.)

\P 1831 F. REYNOLDS  \textit{Playwright's Adventures} iv. 63 St Alm was anything but suave.    
\P 1847 C. BRONTË  \textit{J. Eyre} xiv, He‥showed a solid enough mass of intellectual organs, but an abrupt deficiency where the suave sign of benevolence should have risen.    
\P 1853    \textit{Villette} xxi, The rare passion of the constitutionally suave, and serene, is not a pleasant spectacle.    
\P 1853 LYTTON  \textit{My Novel} iii. xxvi, A slight disturbance of his ordinary suave and well-bred equanimity.    
\P 1863 GEO. ELIOT  \textit{Romola} xxxi, Doubtless the suave secretary had his own ends to serve.    
\P 1898 J. A. OWEN  \textit{Hawaii} iii. 55 Oahumi was quite captivated by the plausible, suave manners of the ingratiating southern chief.

Comb. \P 1894 MAX O'RELL \textit{J. Bull \& Co.} 30 These suave-looking people, far away in the Pacific Ocean.
\end{myenumerate}


%%%%%%%%%%%%%%%%%%%%%%%%%%%%%%%%
\myitem{bland} a.

\noindent \phonetic{(blænd)}

\noindent [ad. L. bland-us soft, smooth, caressing.]
\vspace{-0.3cm}

\begin{myenumerate}

\itembf{1.} Of persons, their actions, etc.: Smooth and suave in manner; mildly soothing or coaxing: gentle.

\P 1661 PEPYS  \textit{Diary} 12 Sept., With some bland counsel of his.    
\P 1667 MILTON  \textit{P.L.} ix. 855 With bland words at will.    
\P 1774 GOLDSM.  \textit{Retal.} 140 His manners were gentle, complying, \& bland.    
\P 1801 SOUTHEY  \textit{Garci Ferrand.} ii. iii, Winning eye and action bland.    
\P 1828 CARLYLE  \textit{Misc.} (1857) I. 93 Bland satire on his friends.    
\P 1855 MACAULAY  \textit{Hist. Eng.} III. 439 A bland temper and winning manners.    
\P 1878 BLACK  \textit{Green Past.} xv. 120 A bland and benevolent face.

\itembf{2.} Of things: Soft, mild, pleasing to the senses; gentle, genial, balmy, soothing.

\P 1667 MILTON  \textit{P.L.} v. 5 Temperat vapours bland.    
\P 1820 KEATS  \textit{St. Agnes} xi, The sound of merriment and chorus bland.    
\P 1872 C. KING  \textit{Sierra Nev.} vi. 122 The air was bland, the heavens cloudless.

\itembf{b.} Of medicines: Mild, unirritating. Of food: Not stimulating. (Cf. quot. 1667 in 2.)

\P 1836 TODD  \textit{Cycl. Anat. \& Phys.} I. 671/2 A very small force only is requisite to cause bland fluids to follow the course of blood.    
\P 1876 DUHRING  \textit{Dis. Skin} 92 Bland oils are serviceable in softening scales and crusts.    
\P 1878 HOLBROOK  \textit{Hyg. Brain} 111 The food should be bland.

\noindent quasi-advb. (in poetry).

\P 1596 SPENSER  \textit{Hymn to Beauty} 171 That base affection, which your eares would bland Commend to you by Loves abused name.    
\P 1850 MRS. BROWNING  \textit{Poet's Vow} ii, They clasping bland his gift.
\end{myenumerate}

%%%%%%%%%%%%%%%%%%%%%%%%%%%%%%%%
\myitem{divulge} v.

\noindent \phonetic{(dɪˈvʌldʒ, daɪ-)}

\noindent [ad. L. dīvulgā-re to spread abroad among the people, make common, f. dī-, dis- 1 + vulgāre to make common, publish; cf. F. divulguer (14th c.), but the palatalized g in English is abnormal.]
\vspace{-0.3cm}

\begin{myenumerate}

\itembf{1.} trans. To make publicly known, to publish abroad (a statement, etc.). Obs.

\P 1460 J. CAPGRAVE  \textit{Chron.} 1 It is somewhat divulgid in this lond, that I have aftir my possibilitie be occupied in wryting.    
\P 1490 CAXTON  \textit{Eneydos} vi. 25 Fame of his ouurages hath ben dyuulged.    
\P 1548 HALL  \textit{Chron., Hen. IV} (an. 3) 20 Whiche fraude the Kyng caused openly to be published and divulged.    
\P 1669 GALE  \textit{Crt. Gentiles} i. ii. i. 4 Their fables they divulge, first by Hymns and Songs.    
\P 1768 H. WALPOLE  \textit{Hist. Doubts} 14 It is impossible to believe the account as fabricated and divulged by Henry the Seventh.    
\P 1791 COWPER  \textit{Iliad} i. 133 Among the Danai thy dreams Divulging.

\itembf{b.} To proclaim (a person, etc.) publicly. Obs.

\P 1598 SHAKES.  \textit{Merry W.} iii. ii. 42, I will divulge Page himselfe for a secure and wilfull Acteon.    
\P 1671 MILTON  \textit{P.R.} iii. 60 When God‥with approbation marks The just man, and divulges him through Heaven To all his angels.

\itembf{c.} To publish (a book or treatise). Obs.

\P 1566 in  \textit{Strype Ann. Ref.} I. xlviii. 517 That treatise‥so publickly by print divulged and dispersed.    
\P 1644 MILTON  \textit{Areop.} (Arb) 53 Ye must repeal and proscribe all scandalous and unlicenc't books already printed and divulg'd.    
\P 1709 STRYPE  \textit{Ann. Ref.} I. lvii. 629 Divers other articles‥propounded and divulged abroad by the said Cartwright.

\itembf{2.} To declare or tell openly (something private or secret); to disclose, reveal.

\P 1602 MARSTON  \textit{Ant. \& Mel. Induct.} Wks. 1856 I. 4, I  will ding his spirit to the verge of hell, that dares divulge a ladies prejudice.    
\P 1671 MILTON  \textit{Samson} 201 Who‥have divulg'd the secret gift of God To a deceitful woman.    
\P 1797 MRS. RADCLIFFE  \textit{Italian} xxvi, Command him to divulge the crimes confessed to him.    
\P 1849 MACAULAY  \textit{Hist. Eng.} I. ii. 268 Cowardly traitors hastened to save themselves, by divulging all‥that had passed in the deliberations of the party.

\itembf{3.} transf. To make common, impart generally. [A Latinism.] Obs. rare.

\P 1667 MILTON  \textit{P.L.} viii. 583 The sense of touch‥would not be To them made common \& divulg'd.

\itembf{4.} intr. (for refl.) To become publicly known. rare.

\P 1602 SHAKES.  \textit{Ham.} iv. i. 22 To keepe it [a disease] from divulging, let's it feede Euen on the pith of life.    
\P 1890 CHILD  \textit{Ballads} vii. cxciv. 29 Nothing seems to have been done to keep the murder from divulging.

\noindent Hence \textbf{divulged} ppl. a.; \textbf{divulging} vbl. n. and ppl. a.

\P 1601 SHAKES.  \textit{All's Well} ii. i. 174 A divulged shame Traduc'd by odious ballads.    
\P 1604 \textit{St. Trials,  Hampton Crt. Confer.} (R.), There is no such licencious divulging of these books.    
\P 1607 TOPSELL  \textit{Four-f. Beasts} (1658) 555 That which divulged fame doth perswade the believers.    
\P 1614 T. ADAMS  \textit{Devil's Banquet} 338 Cease your obstreperous clamours, and divulging slanders.    
\P 1883  \textit{Daily News} 20 July 6/2 An action brought for alleged divulging of telegrams.
\end{myenumerate}


%%%%%%%%%%%%%%%%%%%%%%%%%%%%%%%%
\myitem{gainsay} v.

\noindent \phonetic{(ˈgeɪnseɪ, ˌgeɪnˈseɪ)}

\noindent [f. gain- prefix 1 + say v. Now a purely literary word, and slightly arch. The stress is even or variable; the vbl. n. is commonly \phonetic{ˈgainsaying}. In gainsaid the last syllable is usually (\phonetic{-sɛd}).]
\vspace{-0.3cm}

\begin{myenumerate}

\itembf{1.} trans. To deny.

\P 1300  \textit{Cursor M.} 883 (Gött.) All þis may scho noght gain say.    c 
\P 1330 R. BRUNNE  \textit{Chron.} (1810) 154 If he it geynsay, I wille proue it on him.    
\P 1489 CAXTON  \textit{Faytes of A.} i. i. 8 Yf it happene that ye said aduersarye delyuer deffences \& wyll gaynsaye it.    
\P 1530 PALSGR. 560/1 If I have sayd it I wyll nat gayne saye it.    c 
\P 1570 \textit{Pride \& Lowl.}  (1841) 22 That this is true and may not be denyed, I wyll averre, and yf he it gayne say, I am content by verdict it be tryed.    a 
\P 1619 M. FOTHERBY  \textit{Atheom.} i. viii. § 1. (1622) 55 He, which dare gain-say a thing so generally received.    
\P 1682 BUNYAN  \textit{Holy War} 113 He that gainsays the truth of this must lie against his Soul.    
\P 1728 T. SHERIDAN  \textit{Persius} vi. (1739) 91 Gainsay it if you dare.    
\P 1826 E. IRVING  \textit{Babylon} II. vii. 168 Whether he will in person appear‥we dare neither say nor gainsay.    
\P 1867 FREEMAN  \textit{Norm. Conq.} (1876) I. vi. 498 Facts which cannot be gainsayed.    
\P 1874 G. W. DASENT  \textit{Tales fr. Fjeld} 350 So when the Sheriff asked him Matt did not gainsay that he had slain the parson.

\itembf{2.} To speak against, contradict.

\P 1340  \textit{Cursor M.} 14817 (Fairf.) Nane man may him gaine-sagh. [The other texts have n.]    c 
\P 1450 \textit{St. Cuthbert}  (Surtees) 2086, Bot oft tymes schortely him gainsayed.    
\P 1581 J. BELL  \textit{Haddon's Answ. Osor.} 506 Not we onelye do gaynesay you, but the whole authority of Gods Testament doth determine agaynst you.    
\P 1689–92 LOCKE \textit{Toleration} iii. x. Wks. 1727 II. 463  And that certainly you may think safely, and without fear of being gain-said.    
\P 1742 R. BLAIR  \textit{Grave} 230 The Grave gainsays the smooth-complexion'd Flattery, And with blunt Truth acquaints us what we are.    
\P 1874 CARPENTER  \textit{Ment. Phys.} i. viii. (1879) 374 We have evidence that can scarcely be gainsaid.

\itembf{3.} To speak or act against, oppose, hinder.

\P 1340  \textit{Cursor M.} 5769 (Trin.) Þat þei not ȝein seye [earlier texts say again] my sonde wiþ my tokenes þou shalt hem fonde.    c 
\P 1440 \textit{York  Myst.} x. 198 My lord god will I noght gayne-saye.    c 
\P 1489 CAXTON  \textit{Blanchardyn} xxxviii. 143 That wold hem lete or gaynsey thentre therof.    
\P 1550 CROWLEY  \textit{Way to Wealth} B iv, No man durste gaine saye your doinges for feare of displeasure.    
\P 1601 R. JOHNSON  \textit{Kingd. \& Commw.} (1603) 34 The waters‥gainsaid and put a period to their further progresses.    
\P 1667 MILTON  \textit{P.L.} ix. 1158 Too facil then thou didst not much gainsay, Nay didst permit, approve, and fair dismiss.    
\P 1768 BEATTIE  \textit{Minstr.} i. xlix, Or shall frail man heaven's dread decree gainsay.    
\P 1826 SCOTT  \textit{Woodst.} ii, ‘Yet be ruled, dearest father, and submit to that which we cannot gainsay.’    
\P 1852 M. ARNOLD  \textit{Empedocles on Etna} i. ii, Why is it, that still Man‥believes Nature outraged if his will's gainsaid?

\itembf{4.} To refuse. rare.

\P 1330 R. BRUNNE  \textit{Chron.} (1810) 9 Kynewolf‥toke þe feaute of þe kynges alle‥Bot of Kent and Lyndesay and Northumberland. Þise þre kynges geynsaid it hym.    c 
\P 1532 G. DU WES  \textit{Introd. Fr.} in \textit{Palsgr.} 923 To be gainsayeng and refusyng good counsayle.    
\P 1575 R. B. \textit{Appius  \& Virg.} in Hazl. \textit{Dodsley} IV. 126 Would I gainsay her tender skin to bathe, where I do wash?    
\P 1667 PEPYS  \textit{Diary} (1879) IV. 310 It is not in his nature to gainsay anything that relates to his pleasures.
\end{myenumerate}


%%%%%%%%%%%%%%%%%%%%%%%%%%%%%%%%
\myitem{impugn} v.

\noindent \phonetic{(ɪmˈpjuːn)}

\noindent [a. F. impugner (1363 in Godefroy) = Pr. im-, enpugnar, Sp. impugnar, It. impugnare, ad. L. impugnāre to attack, assail, f. im- (im-1) + pugnāre to fight.]
\vspace{-0.3cm}

\begin{myenumerate}

\itembf{1.} trans. To fight against: to attack, assail, assault (a person, city, etc.). Obs.

\P 1382 WYCLIF  \textit{1 Macc.} xi. 41 Thei inpungneden Yrael.    
\P 1388 \textit{Judg.} ix. 44 He roos‥and enpugnyde [1382 aȝenfiȝtynge]  and bisegide the citee.    c 
\P 1450 tr.  \textit{De Imitatione} iii. xl. 110 Þou dwellist amonge enemyes, þou art impugned on þe riȝt honde \& on þe lifte honde.    
\P 1553 BECON  \textit{Reliques of Rome} (1563) 264 We are set in a slipperye place, and are impugned of deuills.    
\P 1603 KNOLLES  \textit{Hist. Turks} (1621) 35 He‥laid siege unto Damascus‥which he so notably impugned, that [etc.].

fig. \P 1651 HOBBES  \textit{Leviath.} Ded., The Outworks of the Enemy, from whence they impugne the Civill Power.

\itembf{b.} To fight in resistance against; to withstand, resist, oppose. Obs.

\P 1577 HANMER  \textit{Anc. Eccl. Hist.} (1619) 43 Josephus‥which himselfe also at the first impugned the Romaines.    
\P 1591  \textit{Troub. Raigne K. John} ii. (1611) 107 Only the heart impugnes with faint resist The fierce inuade of him that conquers Kings.    
\P 1611 SPEED  \textit{Hist. Gt. Brit.} ix. v. §25 God‥will not leaue vs succourlesse, whiles in a just cause, we impugne a most vnjust Intruder.    
\P 1660 F. BROOKE tr. \textit{Le Blanc's Trav.} 223 To impugn with all his power the Moores, Jews, and Idolaters.

transf. \P 1646 SIR T. BROWNE  \textit{Pseud. Ep.} vi. v. 291 The defect of alternation would utterly impugne the generation of all things.

\itembf{2.} To assail (an opinion, statement, document, action, etc.) by word or argument; to call in question; to dispute the truth, validity, or correctness of; to oppose as false or erroneous.

\P 1362 LANGL.  \textit{P. Pl.} A. viii. 155 Al þis makeþ me‥to þenken‥On Pers þe plouhmon and whuch a pardoun he hedde, And hou þe preost inpugnede hit.    c 
\P 1380 WYCLIF  \textit{Sel. Wks.} III. 350 Þes sectis inpungnen þe gospel, and also þe olde lawe.    
\P 1415 HOCCLEVE  \textit{To Sir J. Oldcastle} 172 No man wolde Impugne hir right.    
\P 1494 FABYAN \textit{Chron.} ii. xliii. 29 This sayinge contraryeth and enpugnyth myne Auctor Gaufride.    
\P 1549  \textit{Compl. Scot. To Rdr.} 12 Detractione‥reddy to suppedit \& tyl impung ane verteous verk.    a 
\P 1614 DONNE  \textit{Βιαθανατος} (1644) 124 No man hath as yet, to my knowledge, impugned this custome of ours.    
\P 1678 CUDWORTH  \textit{Intell. Syst.} i. v. 642 It cannot be accounted less than extreme sottishness and stupidity of mind‥thus to impugn a Deity.    
\P 1777 WATSON  \textit{Philip II} (1793) I. v. 181 An opinion which in France had always been impugned and rejected.    
\P 1847 DISRAELI  \textit{Tancred} i. v, The saint was scarcely canonised, before his claims to beatitude were impugned.

\itembf{b.} To assail the actions, question the statements, etc. of (a person); to find fault with, accuse. Now rare.

\P 1377 LANGL.  \textit{P. Pl.} B. xiii. 123 One Pieres þe ploughman hath inpugned vs alle, And sette alle sciences at a soppe, saue loue one.    
\P 1491 CAXTON  \textit{Vitas Patr.} (W. de W. 1495) iii. iii. 318 b/1 Many hated hym \& specyally theretykes; for he cessed not to enpugne \& repreef theym.    
\P 1530 LYNDESAY  \textit{Test. Papyngo} 13 Quho dar presume thir Poetis tyll Impung, Quhose sweit sentence throuch Albione bene sung?    
\P 1596 SHAKES.  \textit{Merch. V.} iv. i. 179 Yet in such rule, that the Venetian Law Cannot impugne you as you do proceed.    
\P 1879 FARRAR  \textit{St. Paul} xl. II. 323 note, The Law, for the supposed apostasy from which he was impugned.

\noindent Hence \textbf{impugned} ppl. a.; \textbf{impugning} vbl. n. and ppl. a.

\P 1400  \textit{Apol. Loll.} 73 Inpungning of þe law of God.    c 
\P 1440 \textit{Jacob's  Well} (E.E.T.S.) 276 It techyth þe‥to defende þi feyth wyth resouns fro inpugnyng of heretykes.    
\P 1599 SANDYS  \textit{Europæ Spec.} (1632) 94 For defence of impugned truth.    
\P 1802–12 BENTHAM \textit{Rat. Judic. Evid.} (1827) III. 204 It should be allowable‥to call upon the impugning witness‥to declare [etc.].    
\P 1860  \textit{Sat. Rev.} IX. 145/2 The impugned department will send down‥a cohort of witnesses.
\end{myenumerate}

%%%%%%%%%%%%%%%%%%%%%%%%%%%%%%%%
\myitem{repudiate} v.

\noindent \phonetic{(rɪˈpjuːdɪeɪt)}

\noindent [f. L. repudiāt-, ppl. stem of repudiāre to divorce, reject, etc., f. repudium repudy n.]
\vspace{-0.3cm}

\begin{myenumerate}

\itembf{1.} trans. \itembf{a.} Of a husband: To put away or cast off (his wife); to divorce, dismiss.

\P 1545 JOYE  \textit{Exp. Dan.} xi. 185 This Antiochus repudiated his own wyfe called Laodice.    
\P 1597 BEARD  \textit{Theatre God's Judgem.} (1612) 414 Hugh Spencer‥was he that first persuaded the king to forsake and repudiate the queene his wife.    
\P 1663 H. COGAN tr. \textit{Pinto's Trav.} lx. 245 He had repudiated a daughter of his, which he had married three years before.    
\P 1716 BOLINGBROKE  \textit{Refl. upon Exile} Wks. 1754 I. 112  His separation from Terentia, whom he repudiated not long afterward, was perhaps an affliction to him at this time.    
\P 1850 W. IRVING  \textit{Mahomet} vii. (1853) 37 Abu Labab and his wife‥compelled their son, Otha, to repudiate his wife.    
\P 1870 EDGAR  \textit{Runnymede} xxxv. 202 The pope forced her husband to repudiate her.

\itembf{b.} To cast off, disown (a person or thing).

\P 1699 BENTLEY  \textit{Phal.} 316 Other Writers; who being Dorians born, repudiated their vernacular Idiom for that of the Athenians.    
\P 1844 DICKENS  \textit{Mart. Chuz.} xvi, He felt it necessary‥to repudiate and denounce his father.    
\P 1855 PRESCOTT  \textit{Philip II}, I. i. iii. 31 England, after repudiating her heresies, was received into the fold of the Roman Catholic Church.    
\P 1873  \textit{Daily News} 12 Sept. 4/4 M. de Mahy‥called upon the Ministers to repudiate the document.

\itembf{2.} To reject; to refuse to accept or entertain (a thing) or to have dealings with (a person).

\P 1548 HALL  \textit{Chron., Hen. VII} 1 b, The damosell dyd not alonly disagre and repudiate that matrimony, but abhorred‥his‥desyre.    
\P 1674  \textit{Govt. Tongue} 100 O let not those that have repudiated the more inviting sins, show themselves philtr'd and bewitch'd by this.    
\P 1837 LOFFT  \textit{Self-form.} II. 63 Gladly would we have repudiated the property‥so heavily bestowed upon us.    
\P 1862 BEVERIDGE  \textit{Hist. India} II. vi. viii. 802 If they repudiated the empire placed within their reach, some other power would certainly seize it.    
\P 1879 M. ARNOLD  \textit{Mixed Ess.} 32 Not only did the whole repudiate the physician, but also those who were sick.

\itembf{b.} To reject (opinions, conduct, etc.) with condemnation or abhorrence.

\P 1824–9 LANDOR \textit{Imag. Conv., Lucian \& Timotheus}, You have acknowledged his eloquence, while you‥repudiated his morals. 
\P 1840 HERSCHEL  \textit{Ess.} (1857) 109 A doctrine which‥we must repudiate.    
\P 1865 R. W. DALE  \textit{Jew. Temp.} viii. (1877) 85, I repudiate the dreams of Pantheism.

\itembf{c.} To reject (a charge, etc.) with denial, as being quite unfounded or inapplicable.

\P 1865 DICKENS  \textit{Mut. Fr.} iii. i, The old man shook his head, gently repudiating the imputation.    
\P 1874 GREEN  \textit{Short Hist.} viii. §6. 525 Politically it repudiated the taunt of revolutionary aims.

\itembf{3.} To reject as unauthorized or as having no authority or binding force on one.

\P 1646 SIR T. BROWNE  \textit{Pseud. Ep.} 42 He hath obtained with some to repudiate the books of Moses.    
\P 1692 BENTLEY  \textit{Boyle Lect.} ix. 304 Repudiating at once the whole Authority of Revelation.    
\P 1837 LOFFT  \textit{Self-form.} II. 174, I had repudiated the second hand faculty as vain‥and delusive.    
\P 1852 H. ROGERS  \textit{Ecl. Faith} (1853) 74 You would repudiate at once his claims‥to be your infallible guide.    
\P 1879 FROUDE  \textit{Short Stud.} (1883) IV. v. 350 They were ready‥to repudiate the authority of the Pope.

\itembf{b.} To refuse to discharge or acknowledge (a debt or other obligation). Chiefly of (American) states disowning a public debt, and freq. absol.

\P 1837 LOFFT  \textit{Self-form.} I. 249 If a man‥repudiate the care of his wife or children, villain is a word not villanous enough for him.    
\P 1847 WEBSTER  s.v., The state has repudiated its debts.    
\P 1863 H. SPENCER  \textit{Ess.} II. 228 Sir Robert Inglis‥hinted that the national debt would not improbably be repudiated if the proposed measure became law.

absol. \P 1843 SYD. SMITH  \textit{Wks.} (1859) II. 331/2, I am accused of applying the epithet repudiation to States which have not repudiated.    
\P 1862 J. SPENCE  \textit{Amer.} 74 In each of the States that has repudiated there was a large majority of men thoroughly honourable in their private affairs.

\noindent Hence \textbf{repudiated} ppl. a., \textbf{repudiating} vbl. n. and ppl. a.

\P 1635 J. HAYWARD tr. \textit{Biondi's Banish'd Virg.} 143 My first businesse was to hasten the repudiating of the Queene.    
\P 1788 H. WALPOLE  \textit{Remin.} ii. 24 Eldest daughter‥of the repudiated wife of the earl of Macclesfield.    
\P 1843 SYD. SMITH  \textit{Wks.} (1859) II. 328/1 Persons who‥are inclined to consider the abominable conduct of the repudiating States to proceed from exhaustion.    Ibid. 329/1 This swamp we gained‥by the repudiated loan of 1828.    
\P 1880 DIXON  \textit{Windsor} III. xiii. 124 Henry allowed her to live with his repudiated daughter.
\end{myenumerate}


%%%%%%%%%%%%%%%%%%%%%%%%%%%%%%%%
\myitem{spurn} v.1

\noindent \phonetic{(spɜːn)}

\noindent [OE. spurnan, spornan strong v. (pa. tense spearn, pa. pple. -spornen), = OS. spurnan, ON. *sporna (pa. tense sparn), related to the weak vbs. OHG. spornôn, ON. sporna, OHG. spurnan, -en, ON. spyrna, and OHG. (fir)spirnen, ON. sperna, MSw. and Sw. spjärna. The stem is prob. that of spur n.1 In OE. the simple verb is less frequent than the compound ætspurnan.]
\vspace{-0.3cm}

\begin{myenumerate}

\itembf{I.} intr.

\itembf{1.} To strike against something with the foot; to trip or stumble. Also fig. Obs.

\P 1000 \textit{Ags.  Psalter} (Thorpe) xc. 12 Þe læs þu fræcne on stan fote spurne.    a 
\P 1225  \textit{Ancr. R.} 186 A child, ȝif hit spurneð o summe þing, oðer hurteð him, me bet þet þing þet hit hurteð on.    
\P 1297 R. GLOUC.  (Rolls) 7710 As he rod an honteþ \& par auntre is hors spurnde.    a 
\P 1300  \textit{Cursor M.} 3575 Quen þat [a man] sua bicums ald,‥þan es eth þe fote to spurn.    
\P 1388 WYCLIF  \textit{Jer.} xxxi. 9 Y schal brynge them‥in a riȝtful, weie, thei shulen not spurne therynne.    c 
\P 1400  \textit{Beryn} 2862, I shall make hem spurn, \& have a sore falle.    c 
\P 1449 PECOCK  \textit{Repr.} v. viii. 525 Lest if‥the hors where left to his fredom‥he schulde be in perel forto the oftir spurne.    
\P 1549–62 STERNHOLD \& H. \textit{Ps.} xci. 12 So that thy foote shall never chaunce to spurne at any stone.    
\P 1603 \textit{Proph.  of T. Rymour} (Bann. Cl.) 12 Where the water runnes bright and sheene Thair shal many steides spurne.    
\P 1639 FULLER  \textit{Holy War} iv. xxi. (1840) 218 And their legs so stand in men's way that few can go by them without spurning at them.    
\P 1714 GAY  \textit{Trivia} ii. 211 How can ye Laugh, to see the Damsel spurn, Sink in your Frauds and her green Stocking mourn?    
\P 1734 ARBUTHNOT, etc. \textit{Mart. Scriblerus} viii. (1756) 39 The maid‥ran up stairs, but spurning at the dead body, fell upon it in a swoon.

\itembf{b.} In proverbial contrast with speed. Chiefly Sc.

\P 1423 JAS. I  \textit{Kingis Q.} clxxxi, Quhen thai wald faynest speid, that thai may spurn.    c 
\P 1440  \textit{York Myst.} xxxix. 15, I sporne þer I was wonte to spede.    a 
\P 1500 \textit{Ratis  Raving} ii. 362 That garris thaim spwrn quhen thai suld speid.    
\P 1535 STEWART  \textit{Cron. Scot.} III. 226 Quha spurnis airlie cumis lidder speid.

\itembf{2.} To strike or thrust with the foot; to kick (at something). Obs.

\P 1400 LYDG.  \textit{Æsop's Fab.} i. 52 [The cock] On a smal dunghill‥Gan to scrape and sporn.    15‥ Smith \& his Dame 301 in Hazl. E.P.P. III. 212 Than she spvrned at hym so, That hys shynnes bothe two In sonder she there brake.    
\P 15‥ \textit{Smith \& his Dame} 301 in Hazl. E.P.P. III. 212 Than she spvrned at hym so, That hys shynnes bothe two In sonder she there brake.    1592 Nashe P. Penilesse (ed. 2) 3 b, Who spurneth not at a dead dogge?
\P 1592 NASHE  \textit{P. Penilesse} (ed. 2) 3 b, Who spurneth not at a dead dogge?    
\P 1598 MUCEDORUS  \textit{Induct.} 32 Where I may see them wallow in there blood, To spurne at armes and legges quite shiuered off [etc.].    
\P 1690 [See  SPRUNT v.].    
\P 1740 SOMERVILLE  \textit{Hobbinolia} ii. 295 His Iron Fist descending crush'd his Skull, And left him spurning on the bloody Floor.

fig. \P a1548 HALL  \textit{Chron., Hen. V}, 81 This prince was a capitaine against whome fortune never frowned nor mischance once spurned.

\itembf{b.} In allusive phrases. Obs. (Cf. KICK v. 1 c.)

\P 1390 CHAUCER  \textit{Truth} 11 Bywar þerfore to spurne aȝeyns an al.    c 
\P 1480 HENRYSON  \textit{Test. Cres.} 475 Quhy spurnis thow aganis the Wall?    
\P 1483 \textit{Vulgaria}  26 It is a foly to sporn ageyns the pryk.    
\P 1513 MORE  \textit{Rich. III}, Wks. 70/2, I purpose not to spurne againste a prick.    
\P 1562 HEYWOOD  \textit{Prov. \& Epigr.} (1867) 116 Folly to spurne or kycke against the harde wall.    
\P 1573 TUSSER  \textit{Husb.} (1878) 205 What profit then‥Against the prick to seeme to spurne?    
\P 1605 CAMDEN  \textit{Rem.} (1623) 268 Folly it is to spurne against a pricke.    [
\P 1816 SCOTT  \textit{Old Mort.} Introd., Waste not your strength by spurning against a castle wall.]

\itembf{c.} To strike at with a weapon. Obs.

\P 1400  \textit{Destr. Troy} 4744 The grekes‥With speris full dispitiously spurnit at the yates.

\itembf{d.} To dash; to drive quickly. Obs.

\P a1400–50 \textit{Alexander} 786 Now aithire stoure on þar stedis strikis to-gedire, Spurnes out spakly with speris in hand. 
\P 1400 \textit{St. Cuthbert}  (Surtees) 4706 Thre grete wawes in spurned.    Ibid. 6796 Þe shipp agayn to land spurned.

\itembf{3.} fig. To kick against or at something disliked or despised; to manifest opposition or antipathy, esp. in a scornful or disdainful manner.

(a) \P 1526  \textit{Pilgr. Perf.} (W. de W. 1531) 17 b, Than they wyll sporne agaynst god,‥and vtterly refuse and forsake the batayle of vertue.    
\P 1559 \textit{Mirr.  Mag., Owen Glendour} xiii, Was none so bold durst once agaynst me spurne.    
\P 1605 STOW  \textit{Ann.} (ed. 2) 683 Wel knowing that the Queene would spurne against the conclusions.    
\P 1633 BP. HALL  \textit{Hard Texts, N.T.} 145 It is no boot for thee to struggle and spurne against my almighty power.

(b) \P 1549 LATIMER  \textit{3rd Serm. bef. Edw. VI}, G vi, They that be good wyl beare, and not spourne at the preachers; they that be faultye‥must amende, and neyther spourne, nor wynse, nor whyne.    
\P 1594 SHAKES.  \textit{Rich. III,} i. iv. 203 Will you then Spurne at his Edict, and fulfill a Mans?    
\P 1603 KNOLLES  \textit{Hist. Turks} (1621) 1321 Spurning  at their bread and rice which was given them for their daily entertainement.    
\P 1660 \textit{Extr.  State Papers rel. Friends Ser.} ii. (1911) 120 Anabaptists‥will make advantage of the first opportunity to fly out, and spurne att his Maiesties Gouerment.    
\P 1753 H. WALPOLE in  \textit{World} No. 10, One must be an infidel indeed to spurn at such authority.    a 
\P 1781 R. WATSON  \textit{Philip III} (1839) 119 They spurned at danger, and made several vigorous sallies on the enemy.    
\P 1839 T. MITCHELL  \textit{Frogs of Aristoph.} Introd. p. cxi, That parent required sacrifices of him, at which his genius evidently spurned.

\itembf{II.} trans.

\itembf{4.} To strike (the foot) against something. Obs.

\P 1300 E.E.  \textit{Ps.} xc. 12 Þat thurgh hap þou ne spurn þi fote til stane.    c 
\P 1430 \textit{Hymns  Virgin} (1867) 43 Lest þou spurne þi foot at a stoon.

\itembf{5.} To strike or tread (something) with the foot; to trample or kick.

   In later use freq. with implication of contempt.

\P 1390 GOWER  \textit{Conf.} II. 72 The ground he sporneth and he tranceth.    a 
\P 1500 LYTTEL  \textit{Geste of Robyn Hode} iii. clxi, He sporned the dore with his fote.    
\P 1560 J. DAUS tr. \textit{Sleidane's Comm.} 295 The people came running to it, jobbed it in with their daggers, and spurned it with their fete.    
\P 1609 HOLLAND  \textit{Amm. Marcell.} xiv. vii. 15 The foresaid governour‥they layed at and spurned with their heeles.    
\P 1634 SIR T. HERBERT  \textit{Trav.} 20 With their Feet they spurne the yeelding sands.    
\P 1735 SOMERVILLE  \textit{Chase} iii. 335 Wounded, he rears aloft,‥then bleeding spurns the Ground.    
\P 1743 FRANCIS tr.  \textit{Hor., Odes} iii. v. 36 When‥the hind shall turn Fierce on her hunters, he the prostrate foe may spurn In second fight.    
\P 1810 SCOTT  \textit{Lady of L.} i. v, With flying foot the heath he spurned.    
\P 1848 A. JAMESON  \textit{Sacr. \& Leg. Art} 219 Mary is spurning with her feet a casket of jewels.    
\P 1875 LONGFELLOW  \textit{Masque of Pandora} iv, With one touch of my‥feet, I spurn the solid Earth.

\itembf{b.} With advs. or advb. phrases, as away, down, off, up, etc. Also fig.

\P 1386 CHAUCER  \textit{Sqr.'s T.} 608 He with his feet wol spurne adoun his cuppe.    c 
\P 1450  \textit{Merlin} xiii. 199 Galashin with his fote spurned his body to grounde.    
\P 1526  \textit{Pilgr. Perf.} (W. de W. 1531) 264 Auaunce thy spirituall courage, and sporne away all dulnesse \& slouth.    
\P 1590 SHAKES.  \textit{Com. Err.} ii. i. 83 You spurne me hence, and he will spurne me hither.    
\P 1609 ROWLANDS  \textit{Knaue of Clubbes} (Hunterian Cl.) 6 Then with her feete she spurn'd them out of bed.    
\P 1642 D. ROGERS  \textit{Naaman} 30 The Pope treading on his necke, and spurning off his Crowne with his foot.    
\P 1700 DRYDEN  \textit{Cock \& Fox} 85 If, spurning up the Ground, he sprung a Corn.    
\P 1727 SWIFT  \textit{Country Post} Wks. 1751 III.  i. 178 The grave-stones of John Fry, Peter How, and Mary d'Urfey were spurned down.    
\P 1793 T. BEDDOES  \textit{Demonstr. Evid.} 110 It is said, that the statesman‥is apt to spurn away the ladder by which he has mounted to power.    
\P 1836 H. ROGERS  \textit{J. Howe} ii. 30 There is no barrier to such inter-communion,‥which the genuine spirit of charity will not spurn down.    
\P 1855 MACAULAY  \textit{Hist. Eng.} xiii. III. 360 The few who were so luxurious as to wear rude socks of untanned hide spurned them away.    
\P 1878 BROWNING  \textit{Poets Croisic} lii, To learn‥how fate could puff Heaven-high‥, then spurn To suds so big a bubble in some huff.

\itembf{6.} To reject with contempt or disdain; to treat contemptuously; to scorn or despise.

\P 1000 ÆLFRIC  \textit{Saints' Lives} vii. 64 \phonetic{Æfter þæs mædenes spræce þe hine spearn mid wordum.    a 1400–50 Alexander 3533 We sall neuer spise ȝow ne sporne in speche ne in dede.}    
\P 1435 MISYN  \textit{Fire of Love} 44 Þat, vanite spisyd \& spurnyd, to trewth vnpartyngly we draw.    
\P 1501 PLUMPTON  \textit{Corr.} (Camden) 155 He‥wyll abyde by yt for his dede,‥\& so will shew to all men that spurns him any wher.    a 
\P 1548 HALL  \textit{Chron., Hen.} VI, 98 b, Well knowyng, that the Quene would spurne and impugne the conclusions.    
\P 1591 SHAKES.  \textit{Two Gent.} iv. ii. 14 The more she spurnes my loue, The more it growes.    
\P 1635 QUARLES  \textit{Embl.} v. 13 O how my soul would spurn this ball of clay, And loathe the dainties of earth's painful pleasure.    
\P 1697 DRYDEN  \textit{Virg. Georg.} iv. 339 The pleasing Pleiades appear, And springing upward spurn the briny Seas.    
\P 1791 BOSWELL  \textit{Johnson} II. 117 When he suspected that he was invited to be exhibited, he constantly spurned the invitation.    
\P 1848 DICKENS  \textit{Dombey} liii, I came back, weary and lame, to spurn your gift.    
\P 1868 FREEMAN  \textit{Norm. Conq.} (1877) II. 144 Every offer tending to conciliation had been spurned.

\noindent Hence \textbf{spurned} ppl. a.

\P 1805 WORDSW.  \textit{Prelude} v. 278 He‥draws‥sweet honey out of spurned or dreaded weeds.



\end{myenumerate}


%%%%%%%%%%%%%%%%%%%%%%%%%%%%%%%%
\myitem{conspicuous} a.

\noindent \phonetic{(kənˈspɪkjuːəs)}

\noindent [f. L. conspicu-us visible, striking + -ous.]
\vspace{-0.3cm}

\begin{myenumerate}

\itembf{1.} Clearly visible, easy to be seen, obvious or striking to the eye.

\P 1545 T. RAYNALDE  \textit{Byrthe Mankynde} Hh vij, These vaynes doo appeare more conspicuous and notable to the eyes.    
\P 1592 R. D. tr.  \textit{Hypnerotomachia} 97 Hils couered ouer with green trees of a conspicuous thicknes.    
\P 1667 PEPYS  \textit{Diary} (1879) IV. 415 These Rogues‥ to be hung in some conspicuous place in the town, for an example.    
\P 1667 MILTON  \textit{P.L.} iv. 545 A Rock Of Alablaster, pil'd up to the Clouds, Conspicuous farr.    
\P 1808 SCOTT  \textit{Marm.} ii. xi, Conspicuous by her veil and hood.    
\P 1840 MACAULAY  \textit{Clive} 47 Conspicuous in the ranks of the little army.

\itembf{2. a.} Obvious to the mental eye, plainly evident; attracting notice or attention, striking; hence, eminent, remarkable, noteworthy.

\P 1613 R. C. TABLE  \textit{Alph.} (ed. 3), Conspicuous, easie to be seene, excellent.    
\P 1651 HOBBES  \textit{Leviath.} i. x. 44 To be Conspicuous, that is to say, to be known for Wealth‥or any eminent Good, is Honourable.    
\P 1845 S. AUSTIN  \textit{Ranke's Hist. Ref.} III. 209 Frankfurt—a city so conspicuous for its loyalty to the imperial house.    
\P 1876 J. H. NEWMAN  \textit{Hist. Sk.} I. i. iii. 131 Sultan Soliman, who plays so conspicuous a part in Tasso's celebrated Poem.

\itembf{b.} Phr. conspicuous by its absence.

\P 1859 LD. J. RUSSELL  \textit{Addr. Electors of Lond.}, Among the defects of the Bill, which were numerous, one provision was conspicuous by its presence, and one by its absence.    
\P 1859 \textit{Sp. at Lond. Tavern} 15 Apr., I alluded to it as ‘a provision conspicuous by its absence,’ a turn of phraseology which is not an original expression of mine, but is taken from one of the greatest historians of antiquity. [Tacitus Ann. iii. 76.]    
\P 1875 BRYCE  \textit{Holy Rom. Emp.} xv. (ed. 5) 287 Those monuments which do exist are just sufficient to make the absence of all others more conspicuous.    
\P 1878 W. A. WRIGHT  \textit{Note on Shaks.} Jul. C. ii. i. 70 Cassius had married Junia, Brutus' sister‥At her funeral in a.d. 22 the images of Brutus and Cassius were conspicuous by their absence, or as Tacitus (Ann. iii. 76) puts it, ‘sed praefulgebant‥eo ipso quod effigies eorum non visebantur’.

\itembf{3.} Designating expenditure on or consumption of luxuries on a lavish scale in an attempt to enhance one's prestige.

\P 1899 T. VEBLEN  \textit{Theory of Leisure Class} iv. 75 Conspicuous consumption of valuable goods is a means of reputability to the gentleman of leisure.    Ibid. iv. 96 Throughout the entire evolution of conspicuous expenditure, whether of goods or of services or human life, runs the obvious implication that in order to effectually mend the consumer's good fame it must be an expenditure of superfluities.    
\P 1926 B. WEBB  \textit{My Apprent.} i. 53 Competition in conspicuous expenditure on clothes, food, wine and flowers.    
\P 1962 E. GODFREY  \textit{Retail Selling} xxi. 214 In the past ‘conspicuous’ consumption, of the swimming-pool, cabin cruiser, high-powered sports car variety, was confined to‥the idle rich.
\end{myenumerate}


%%%%%%%%%%%%%%%%%%%%%%%%%%%%%%%%
\myitem{sundry} a.

\noindent \phonetic{(ˈsʌndrɪ)}

\noindent [OE. \phonetic{syndriᴁ} separate, special, private, exceptional, corresp. to MLG. sunder(i)ch single, special, LG. sunderig, OHG. sunt(a)rîc, sund(i)rîc, -erîg special (MHG. sunderig, -ic); f. sunder sunder a.: see -y1.]
\vspace{-0.3cm}

\begin{myenumerate}

\itembf{1.} Having an existence, position, or status apart; separate, distinct. Obs. exc. dial.

\P 1000 ÆLFRIC  \textit{Judg.} Epil. (Gr.) 263 \phonetic{Þa senatores‥dæᴁ$\sim$hwanlice smeadon on anum sindrian huse embe ealles folces þearfe.}
\P c1000 \textit{Ags.  Ps.} cxl. 12 (Gr.) Ic me \phonetic{syndriᴁ} eom.    c 
\P 1250 \textit{Gen. \& Ex.} 1985 Ðor was in  helle a sundri stede, wor ðe seli folc reste dede.    a 
\P 1300  \textit{Cursor M.} 332 Þis wright [sc. God]‥Fra al oþer, sundri [Fairf. ys sundre] and sere.    Ibid. 16094 Þe pretori, þat was a sundri stede.    
\P 1393 LANGL.  \textit{P. Pl.} C. xix. 192 Þre persones in o pensel‥departable from oþer‥And sondry to seo vpon.    
\P 1533 N. UDALL  \textit{Coronat. Anne Boleyn} in \textit{Arb. Garner} II. 58 The fourth Lady‥peerless in riches, wit, and beauty; Which are but sundry qualities in yon three [sc. Juno, Pallas, and Venus].    
\P 1549 COVERDALE, etc. \textit{Erasm. Par. 1 Pet.} 9 Let not age, estate, condicion or sondry being in diuerse countres disseuer you a sondre.    
\P 1790 MRS. WHEELER  \textit{Westmld. Dial.} (1802) 114 She ligs in a sendry kaw boose.

\itembf{2.} Belonging or assigned distributively to certain individuals; distinct or different for each respectively. Obs.

\P a900 tr. \textit{Bæda's Hist.} iv. xxiii. [xxii.] (1890) 328 Þurh \phonetic{syndriᴁe} þine ondsware [orig. per singula tua responsa] ic \phonetic{onᴁet} \& oncneow, þæt [etc.].    
\P \textit{Ibid.} v. xxiii. (1899) 697/1 On septem Epistolas Canonicas [ic sette] syndrie bec.    
\P 1000 ÆLFRIC  \textit{Deut.} xxxiii. 5 \phonetic{Moyses þa ᴁebletsode$\sim$þa twelf mæᴁða ælce mid sindriᴁre bletsunge.}    
\P c1205 LAY.  2688 He hefde on liue tuenti sunen and alc hefde sindri moder.    a 
\P 1300  \textit{Cursor M.} 9533 Ilkan sum-dri gift he gaue.    
\P 1375 BARBOUR  \textit{Bruce} x. 731 His men, in-to syndry plas, Clam our the wall.    
\P 1430-40 LYDG.  \textit{Bochas} i. ii. (MS. Bodl. 263) 17/1 The contre off Sennar thei forsook And ech off hem a sondri contre took.    a 
\P 1548 HALL  \textit{Chron., Hen. VIII} 70, iiii. hed peces called Armites, euery pece beyng of a sundery deuice.    
\P 1549  \textit{Compl. Scot.} vi. 65 Ilk ane of them hed ane syndry instrament to play to the laif, the fyrst hed ane drone bag pipe, the nyxt hed ane pipe maid of ane bleddir and of ane reid, the thrid playit on ane trump [etc.].    
\P 1592 GREENE  \textit{Conny Catching} Wks. (Grosart) XI. 84 Those Amarosos here in England‥that‥wil haue in euery shire in England a sundry wife.    a 
\P 1700 DRYDEN  \textit{Ovid's Art Love} i. 863 Experience finds That sundry Women are of sundry Minds.    
\P 1715 PENNECUIK  \textit{Truth's Trav.} 114 Ilk an ran a sindrie gait.    
\P 1738 WESLEY  \textit{Ps.} civ. iv, His Ministers Heav'n's Palace fill, To have their sundry Tasks assign'd.

\itembf{3.} Individually separate; that is one of a number of individuals of a class or group. Usually with pl. n. or sing. n. in pl. sense: Various, (many) different. Obs. (or merged in 5).

\P 1250 \textit{Gen. \& Ex.}  665 Al was on speche ðor bi-foren, ðor woren sundri speches boren.    
\P 1375 BARBOUR  \textit{Bruce} v. 7 For to mak in thair synging Syndry notis, and soundis sere.    
\P 14‥ \textit{Sir Beues} (MS. E.) 4313 + 46 He hadde wunnen in to hys hond Many a batayle in sundry lond.    c 
\P 1470 HENRY  \textit{Wallace} i. 29 Elrisle‥Auchinbothe, and othir syndry place.    
\P 1551 RECORDE  \textit{Pathw. Knowl.} i. xvii, Diligently behold how these sundry figures be turned into triangles.    
\P 1561 T. HOBY tr. \textit{Castiglione's Courtyer} i. (1577) D vij b, In learning to handle sundrie kinde of weapons.    
\P 1596 \textit{Edw.  III}, iii. i. 69 Like to a meddow full of sundry flowers.    
\P 1603 OWEN  \textit{Pembrokeshire} (1892) 269 The seuerall sortes of fowle‥and‥the sondrey kindes of takeinge of them.    
\P 1677 in  \textit{Verney Mem.} (1907) II. 327 There are sundry sorts of Habits becomming Souldiers in particular.    
\P 1754 SHERLOCK  \textit{Disc.} vii. (1759) I. 215 The Prophets of old were‥destroyed by sundry Kinds of Death.

\itembf{b.} Preceded (rarely followed) by an adj. of number or plurality (esp. many). See also 6 e. Obs.

\P 1377 LANGL.  \textit{P. Pl.} B. xiii. 38 Þanne cam scripture And serued hem‥of sondry metes manye.    
\P 1390 GOWER  \textit{Conf.} II. 359 Thei bede‥Tuo sondri beddes to be dyht.    
\P 1474 CAXTON  \textit{Chesse} iv. v. (1883) 176 Whan he is in the myddes of the tabler he may goo in to viii. places sondry.    
\P 1500-20 DUNBAR  \textit{Poems} xxvi. 26 Heilie harlottis‥Come in with mony sindrie gyiss.    15‥ Adam Bel 470 in Hazl. E.P.P. II. 158 We haue slaie your fat falow der In many a sondry place.    
\P 1570 FOXE  \textit{A. \& M.} (ed. 2) 1362/2 In those dayes there were ij. sundry Bibles in Englishe.    
\P 1570 SATIR.  \textit{Poems Reform.} xiii. 17 And this he vsis mony sindrie sortis.    1570-6 Lambarde Peramb. Kent (1826) 198 The third Brooke‥being crossed in the way by seven other sundry bridges.    
\P 1617 MORYSON  \textit{Itin.} i. 231 Nine sundry Sects of Christians haue their Monasteries within this City.    
\P 1678 R. BARCLAY  \textit{Apol. Quakers} v. §20. 157 This Parable, repeated in three sundry Evangelists.

\itembf{c.} Comb., as sundry-coloured, sundry-shaped adjs.

\P 1587 GOLDING  \textit{De Mornay} vi. (1592) 62 Afore making this sundrishaped world, God had conceiued an incorruptible paterne thereof.    
\P 1593 DRAYTON  \textit{Ecl.} i. 14 His sundrie coloured Coat.    a 
\P 1700 EVELYN  \textit{Diary} June 1645, The quire, wall'd‥with sundry colour'd stone halfe relievo.

\itembf{4.} Different, other. (Const. from.) With pl. n. or sing. n. in pl. sense: Diverse, manifold. Obs.

\P 13‥ \textit{Cursor M.} 4246 (Gött.) Putyfar‥held ioseph in mensk and lare Al þou þair treuthes sundri ware. 
\P 1400  \textit{Rom. Rose} 5184 If I may lere Of sondry loves the manere.    c 
\P 1470 HENRY  \textit{Wallace} x. 708 The king changyt on syndry hors off Spayn.    
\P 1509 HAWES  \textit{Past. Pleas.} iv. (Percy Soc.) 19 A venemous beast of sundry likenes.    
\P 1535 COVERDALE  \textit{Bible Prol. to Rdr.} \phonetic{⁋}2 Euery church allmost had ye Byble of a sondrye translacion.     
\P 1548 TURNER  \textit{Names Herbes} (E.D.S.) 23 Carduus‥is a sundry herbe from Cinara.     
\P 1551 \textit{Herbal} i. E iij, Dioscorides descrybeth thes herbes seuerally, \& so maketh them sondry herbes.     
\P 1586 DAY  \textit{Engl. Secretorie} i. (1625) 132 How many, and how sundry are the euils wherewith our mortall state is endangered.     
\P 1614 W. B. \textit{Philos.  Banquet} (ed. 2) 113 The sundryest kindes of extremities.     
\P 1639 FULLER  \textit{Holy War} iv. vi. (1647) 176 A sundry dialect maketh not a severall language.     
\P 1668 CULPEPPER \& COLE  \textit{Barthol. Anat.} iii. xi. 152 The external parts about the mouth are sundry.

\itembf{b.} (a) Consisting of different elements, of mixed composition. Obs. rare.

\P 1594 HOOKER  \textit{Eccl. Pol.} iv. vi. §3 Forbidding them [sc. the Jews] to put on garments of sundry stuffe.    
\P 1600 SHAKES.  \textit{A.Y.L.} iv. i. 17 A melancholy of mine owne, compounded of many simples, extracted from many obiects, and indeed the sundrie contemplation of my trauells, in which my often rumination, wraps me in a most humorous sadnesse.

(b) Consisting of miscellaneous items: cf. sundries.

\P 1790 BEATSON  \textit{Nav. \& Mil. Mem.} II. 187, 75 tons of sundry wood.    
\P 1870 RAYMOND  \textit{Statist. Mines \& Mining} (1872) 98 The assets of the company [include] Cash in Bank of California $119,609.‥ Sundry open accounts $2,863.    
\P 1913  \textit{Times} 9 Aug. 19/2 Yield, including sundry revenue, £4,855.

\itembf{5.} As an indefinite numeral: A number of, several. (The prevailing use.)

\noindent   Occas. with poss. as \textbf{sundry his} = several of his.

\P 1375 \textit{Sc. Leg.  Saints} ii. (Paulus) 26 In parelis wes he stad sindry.    
\P 1390 GOWER  \textit{Conf.} I. 209 This Emperour‥Withinne a ten mile enviroun‥Hath sondry places forto reste.    
\P 1456 SIR G. HAYE  \textit{Law Arms} (S.T.S.) 107 And ȝit is thare sindry othir realmes that obeyis nocht to the Emperoure.    
\P 1542 UDALL  \textit{Erasm. Apoph.} 321 Whom Cicero veray often tymes citeth in soondrie his werkes.    
\P 1552  \textit{Bk. Com. Prayer, Morn. Prayer, Exh.}, The scripture moueth vs in sondrye places, to acknowledge and confesse our manyfolde synnes and wyckednesse.    
\P 1605 SHAKES.  \textit{Macb.} iv. iii. 158 Sundry Blessings hang about his Throne, That speake him full of Grace.    
\P 1630 PRYNNE  \textit{Anti-Armin.} 118 Subiecting it to sundry alterations, periods, and changes at our pleasure.    
\P 1782 F. BURNEY  \textit{Cecilia} ii. ii, [She] was then ushered with great pomp through sundry apartments.    
\P 1794 \textit{Bloomfield's  Reports} 13 The Court having heard‥sundry affidavits read.    
\P 1843 JAMES  \textit{Forest Days} i, These benches formed the favourite resting-place of sundry old men.    
\P 1870 A. R. HOPE  \textit{My Schoolboy Fr.} xi. 149 Disturbing the placid repast of sundry forlorn cows.    
\P 1913  \textit{Oxf. Univ. Gaz.} 19 Feb. 493/2 Having built some proper out-houses to replace sundry untidy wooden hen-roosts.

\itembf{b.} In collocations, as sundry (and) divers, divers (and) sundry, sundry (and) several. Obs.

\P 1420 ? LYDG. \textit{Assembly of Gods} 321 Chaungeable of sondry dyuerse colowres.    
\P 1483 \textit{Rolls  of Parlt.} VI. 245/1 Sundrie and diverse false and traiterous proclamacions.    
\P 1495 \textit{Naval  Acc. Hen. VII} (1896) 138 Diverse \& soundrie shippes.    a 
\P 1548 HALL  \textit{Chron., Edw. IV} 222 At sondry and seuerall tymes (and not all at one tyme).    
\P 1574 in  \textit{10th Rep. Hist. MSS. Comm. App.} v. 424 For dyverse and sondrye good occations.    
\P 1590 L. LLOYD  \textit{Diall Daies} 76 At sundrie severall times.

\itembf{c.} ellipt. and (chiefly Sc.) absol. (Cf. several a. 4 c.)

\P 1470 HENRY  \textit{Wallace} i. 199 Syndry wayntyt, bot nane wyst be quhat way.    
\P 1575 in  \textit{Maitl. Club Misc.} I. 115 Syndery boyith of the citie and gentillmen upaland.    a 
\P 1629 HINDE  \textit{J. Bruen} xlvi. (1641) 146 Divers and sundry of the workes of the Lord.    
\P 1680 H. MORE  \textit{Apocal. Apoc.} 123 The not understanding of which has made sundry in vain attempt to predict events foretold in the Apocalypse.    a 
\P 1796 BURNS  \textit{Katherine Jaffray} iii, He's tell'd her father and mother baith, As I hear sindry say, O.    
\P 1825 T. HOOK  \textit{Sayings Ser.} ii. Doubts \& F. i. II. 84 Sundry of those little hemmings and coughings.    
\P 1875 WHITNEY  \textit{Life Lang.} vii. 115 Sundry of the modern European languages.

\itembf{6.} Phr. \textbf{a.} on sundry, in sundry, a sundry: alteration of on-, in-sunder (see sunder B), asunder. 
\textbf{b.} by sundries: individually. 
\textbf{c.} in or on sundry wise (occas. sundry wises), later sundry wise: in various or different ways; variously, diversely. 
\textbf{d.} (in) sundry ways (in the same sense). 
\textbf{e.} all and sundry, occas. all sundry: every individual, every single; now only absol. (occas. all and sundries) = everybody of all classes, one and all. (orig. and chiefly Sc. = L. omnes et singuli.)

a. \P 1250 \textit{Gen. \& Ex.}  393 On sundri ðhenken he to ben.    
\P 13‥ \textit{Cursor M.} 14665 (Gött.) We er all ane,‥Sua þat we thoru nane-kin art Ne man be made in sundri [Cott. in sundre] part.    c 
\P 1330 \textit{Amis \& Amil.}  309 Now we asondri schal wende.    a 
\P 1400 \textit{Parlt.  3 Ages} (Roxb.) 90, I‥choppede of the nekke And þe hede and the haulse homelyde in sondree.    c 
\P 1420 ?LYDG.  \textit{Assembly of Gods} 1765 Whyche  iii tymes, a sondry deuydyd, Mayst thow here see.

b. \P 1400-50 \textit{Wars  Alex.} 3909 Þai seke out be sundres sexti to-gedire.

c. \P 1375 \textit{Sc. Leg.  Saints} v. (Johannes) 558 He taucht þam in syndry vyis.    
\P 1375 BARBOUR  \textit{Bruce} ix. 441 The laif‥Sesit‥Men, armyng, and marchandiss, And othir gudis on syndri viss.    14‥ Chaucer's Friar's T. 172 (Harl. MS. 7334) Why‥ryde ȝe þan or goon, In sondry wyse [v.r. shape] and nouȝt alway in oon?    
\P 1484 in  \textit{Lett. Rich. III \& Hen. VII} (Rolls) I. 88 Feithful services to us in sundry wises doon.    
\P 1549 COVERDALE, etc. \textit{Erasm. Par. Rom.} 33 God doeth in sondry wyse bestow his giftes.    
\P 1577 B. GOOGE  \textit{Heresbach's Husb.} i. (1586) 3 b, The fruitefull Earth that tyld in sundry wyse, Vnto the eye her goodly fruites dooth yeelde.    
\P 1591 R. TURNBULL  \textit{St. James} 149 b, Men fall and sinne‥three waies‥and there is no man which doeth not fall through euerie one of these, sundriwise.    
\P 1818 SCOTT  \textit{Hrt. Midl.} xlvii, Twa precious saints might pu' sundry wise, like twa cows riving at the same hay-band.

d. \P 1578 LINDESAY  (Pitscottie) \textit{Chron. Scot.} (S.T.S.) I. 3 Ingyne of man be Inclinatioun in sindrie wayes is giwin.    
\P 1592 TIMME  \textit{Ten Engl. Lepers} E 4 b, This leprosie of pride dooth sundrie waies lay holde upon men.    
\P 1605 SHAKES.  \textit{Macb.} iv. iii. 48 Yet my poore Country Shall‥More suffer, and more sundry wayes then euer.    
\P 1609 SKENE  \textit{Reg. Maj. Table} 61 He quha being lawfullie summoned, is absent,‥is sindrie wayes vnlawed according to the diversitie of the courts.    
\P 1697 DRYDEN  \textit{Virg. Georg.} iii. 187 To breed him, break him, back him, are requir'd Experienc'd Masters; and in sundry Ways: Their Labours equal, and alike their Praise.    
\P 1743 BULKELEY \& CUMMINS  \textit{Voy. S. Seas} 36 There have died sundry ways since the Ship first struck forty-five Men.

e. \P 1389 in  \textit{Sir W. Fraser Wemyss of W.} (1888) II. 24 Til there thyngys al and syndry lelily and fermly to be fulfyllyt and yhemmyt.    
\P 1480 in  \textit{Exch. Rolls Scot.} IX. 120 note, All and sendri oure liegis and subditis.    
\P 1552 ABP. HAMILTON  \textit{Catech.} (1884) 3 Till all and sindry personis.    
\P 1562 A. SCOTT  \textit{Poems} (S.T.S.) i. 95 To ceis all sindrye sectis of hereseis.    
\P 1597  \textit{Reg. Mag. Sig. Scot.} 303/2 Togidder with all and sindrie the teindscheves.    
\P 1682  \textit{Lond. Gaz.} No. 1682/1 To have forfault‥all and sundry his Lands, Heretages, Liffrents, and Rents.

absol. \P 1428 \textit{Munim.  de Melros} (Bann.) 519 Till all \& syndry to quham þe knawlage of þir presentz lettris sall to cum.    
\P 1442 in  \textit{Reg. Mag. Sig. Scot.} 63/2 Till al and sindri that thir presentez lettrez sall here or see.    
\P 1783 W. GORDON tr. \textit{Livy's Rom. Hist.} iv. ii. 310 Sedition never failed to procure honour and respect to all and sundries, its authors and abettors.    
\P 1818 SCOTT  \textit{Hrt. Midl.} lii, Join wi' Rob Roy‥and revenge Donacha's death on all and sundry.    
\P 1837-42 HAWTHORNE  \textit{Twice-told T.} (1851) I. x. 171, I cry aloud to all and sundry, in my plainest accents.    
\P 1901  \textit{Scotsman} 13 Mar. 12/2 The city must advertise for estimates from all and sundry.

\itembf{7.} That sunders or separates; dividing; discriminating. Obs. rare.

\P 1564 HARDING  \textit{Answ. to Jewel's Chalenge} 133 b, They must vse a discretion, and a sundry iudgement betwen the thinges they write agonisticῶς,‥and the thinges they vtter dogmaticῶς.    
\P 1593 A. CHUTE  \textit{Beautie Dishonoured} (1908) 111 Thus life, and death, in unitie agreeing Dated the tenor of their sonderie strife.

\noindent Hence \textbf{sundryfold} a., manifold; \textbf{sundryhead}, diversity, variety; 
\textbf{sundrywhere} adv., in various places.

\P 1430 LYDG.  \textit{Minor Poems} (Percy Soc.) 194 Complexionat of *sondryfold coloures.    
\P 1557 T. PHAER  \textit{Æneid} v. M iv b, Skant yemen twayn‥the same coud beare, So sondriefolde it was.

\P 1395 HYLTON  \textit{Scala Perf.} (W. de W. 1494) ii. xlvi, Þe *soundryhede of orders [of angels].

\P 1548 PATTEN  \textit{Exped. Scot.} M vij b, His valiaunce *sundry whear tried.    
\P 1568 T. HOWELL  \textit{Arb. Amitie Poems} (1879) 35 The fethred foule‥sundrie where his fostring foode, With chirping bill he peekes.
\end{myenumerate}

%%%%%%%%%%%%%%%%%%%%%%%%%%%%%%%%
\myitem{mundane} a. (n.)

\noindent \phonetic{(ˈmʌndeɪn, mʌnˈdeɪn)}

\noindent [a. F. mondain (12-13th c.), ad. L. mundān-us, f. mundus world.]
\vspace{-0.3cm}

\begin{myenumerate}

\itembf{1.} Belonging to this world (i.e. the earth as contrasted with heaven); worldly; earthly.

   In early use (till 1550) often following its n., and sometimes taking s in the pl.

\P 1475  \textit{Bk. Noblesse} (Roxb.) 70 He saide that fortune and felicite mondeyne was joyned and knyt withe his vertue and noblesse roiall.    
\P 1509 BARCLAY  \textit{Shyp of Folys} 67 b, Alas oft goddes goodes‥Of suche folys is wastyd‥In great folyes mundaynes and outrage.
\P a1652 J. SMITH  \textit{Sel. Disc.} i. 21 Entangled with the birdlime of fleshly passions and mundane vanity.
\P a1720 SEWEL  \textit{Hist. Quakers} (1795) I. ii. 146 By a singular and very strange turn of mundane affairs.    
\P 1869 MOZLEY  \textit{Univ. Serm.} ii. (1876) 50 Not like the goodness which feeds upon mundane motives and is weak and sickly.

\itembf{b.} Belonging to the ‘world’ as distinguished from the church. Of literature: Secular. rare.

\P 1848 W. K. KELLY tr. \textit{L. Blanc's Hist. Ten Y.} II. 532 It [Talleyrand's reconciliation to the church] was matter of inexpressible surprise and pain to the more mundane portion of the prince's intimate acquaintances.    
\P 1865 M. ARNOLD  \textit{Ess. Crit.} vi. (1875) 245 The beginnings of the mundane poetry of the Italians are in Sicily.

\itembf{c.} Belonging to the world of fashion. [= F. mondain.]

\P 1904  \textit{Edin. Rev.} Oct. 298 The Athénée and the Nouveautés‥the favourite resorts of ‘mundane’ pleasure-seekers.

\textit{d.} In weakened use: everyday, ordinary, commonplace; hence, banal, prosaic, dull; routine, trite.

\P 1898  \textit{Westm. Gaz.} 28 July 2/3 To consider‥more mundane matters, such as the number and characters of transmigrating households.    
\P 1938 R. NARAYAN  \textit{Dark Room} iii. 29 The whole picture swept her mind clear of mundane debris.    
\P 1965 A. J. P. TAYLOR  \textit{Eng. Hist. 1914-1945} X. 322  There were also more mundane calculations. The Conservatives were confident they could win an election on the National cry.    
\P 1976 G. GORDON  \textit{100 Scenes from Married Life} 118 Inject a spot of excitement into our mundane and self-satisfied lives.    
\P 1986 P. READING  \textit{Essential Reading} 85 At least this would avoid your having to employ your pen on such mundane matters when it could be used to such good effect elsewhere!

\itembf{2.} Pertaining to the cosmos or universe; cosmic.

   \textit{mundane soul, spirit}: the anima mundi of the Platonists (ἡ τοῦ κόσµου ψυχή, ἡ κοσµικὴ ψυχή in Proclus).

\P 1642 H. MORE  \textit{Song of Soul} ii. iii. i. 18 We have the sight Of what the Mundane spirit suffereth By colours, figures, or inherent light.    
\P 1665 GLANVILL  \textit{Scepsis Sci.} xxiv. 147 The Platonicall Hypothesis of a Mundane Soul.    
\P 1692 BENTLEY  \textit{Boyle Lect.} vii. (1693) 7 The Atoms or Particles which now constitute Heaven and Earth, being once separate and diffused in the Mundane Space, like the supposed Chaos, could never [etc.].    
\P 1872 MOZLEY  \textit{Mirac.} (ed. 3) Pref. 24 The idea of God as the Supreme Mundane Being.

\itembf{b.} \textit{mundane egg}: in Indian and other cosmogonies, a primordial egg from which the world was hatched.

\P 1789 [see ORPHIC 1 b].    
\P 1841 ELPHINSTONE  \textit{Hist. Ind.} I. i. iv. 75 From this seed sprung the mundane egg, in which the Supreme Being was himself born in the form of Brahmá.

\itembf{c.} mundane era, an era reckoned from the time of the creation of the world.

\P 1892 E. M. THOMPSON  \textit{Gr. \& Lat. Palæogr.} Add. 323 To reduce the Mundane era of Constantinople to the Christian era.

\itembf{3.} Astrol. Pertaining to the horizon and not to the ecliptic or zodiac; chiefly in mundane aspect, mundane parallel.

\P 1687 J. BISHOP  \textit{Marrow Astrol.} ii. 33 At which time the \phonetic{☽} was directed to a mundane parallel of \phonetic{♂}.    Ibid. 76 Narrowly observe all the Aspects, as well those in the World, as those in the Zodiack, for many times a Zodiacal Aspect may promise good in the Business, when there may be a Mundane Aspect will frustrate the good promised by the other.    
\P 1819 J. WILSON  \textit{Dict. Astrol.} 295 Mundane Aspects, distances in the world measured by the semiarc wholly independent of the zodiac.

\itembf{4.} Nat. Hist. Used by Darwin for: Found in all parts of the world, widely distributed.

\P 1844 DARWIN in  \textit{Life \& Lett.} (1887) II. 25 The Owl is mundane, and many of the species have very wide ranges.

\itembf{5.} n. A dweller in this world. Obs. rare—1.

\P 1517 H. WATSON  \textit{Ship of Fools Prol.} A ij b, By the shyppe we maye vnderstande ye folyes and erroures that the mondaynes are in, by the se this present worlde.

\noindent Hence \textbf{mundanely} adv., \textbf{mundaneness}.

\P 1727 BAILEY  vol. II, Mundaneness, worldliness.    
\P 1824 LANDOR  \textit{Imag. Conv.} ii. Wks. 1846 I. 46 The  greatest of stakes, mundanely speaking, is the stake of reputation.    
\P 1886 MYERS in  Gurney, etc. \textit{Phantasms of Living} II. 294 This very mundaneness of the apparition is precisely what was to be expected.




\end{myenumerate}

%%%%%%%%%%%%%%%%%%%%%%%%%%%%%%%%
\myitem{precocious} a.

\noindent \phonetic{(prɪˈkəʊʃəs)}

\noindent [f. L. præcox, -cocem (precoce): see -ious.]
\vspace{-0.3cm}

\begin{myenumerate}

\itembf{1.} Of a plant: Flowering or fruiting early; spec. bearing blossom before the leaves; also said of the blossoms or fruit.

\P 1650 SIR T. BROWNE  \textit{Pseud. Ep.} ii. vi. (ed. 2) 79 Many precocious trees, and such as have their spring in the winter, may be found in most parts of Europe.    a 
\P 1682 \textit{Tracts} (1684) 72 That there were precocious and early bearing Trees in Judæa, may be illustrated from some expressions in Scripture concerning precocious Figgs.    
\P 1872 OLIVER  \textit{Elem. Bot.} ii. 234 A‥tree, with‥precocious hermaphrodite flowers.

\itembf{2. a.} fig. Of persons: Prematurely developed in some faculty or proclivity.

\P 1678 CUDWORTH  \textit{Intell. Syst.} i. iv. §21. 388 However it hath been of late so much decried‥by‥precocious and conceited wits also, as non-sence and impossibility.    
\P 1819 BYRON  \textit{Juan} i. liv, To be precocious Was in her eyes a thing the most atrocious.    
\P 1829 LYTTON  \textit{Devereux}, i. v, We were all three‥precocious geniuses.    
\P 1868 E. EDWARDS  \textit{Ralegh} I. xv. 299 She was somewhat precocious in love matters.

\itembf{b.} Of, pertaining to, or indicative of precocity or premature development.

\P 1672 SIR T. BROWNE  \textit{Let. Friend} §28 'Tis superfluous to live unto gray Hairs, when in a precocious Temper we anticipate the Virtues of them.    
\P 1827 MACAULAY  \textit{Machiavelli Ess.} (1887) 36 Untimely decrepitude was the penalty of precocious maturity.
\P a1863 THACKERAY  \textit{Christmas Bks.} (1872) 19 His ‘Love Lays’‥were pronounced to be wonderfully precocious for a young gentleman then only thirteen.

\itembf{c.} Of things: Of early development.

\P 1838 DICKENS  \textit{Nich. Nick.} xx, Youthful misery stalks precocious.    
\P 1899 \textit{Allbutt's  Syst. Med.} VII. 668 ‘Specific’ phenomena are more commonly observed within a comparatively short time from the date of infection in which case they are not rightly regarded as ‘precocious’ symptoms.

\itembf{3. a.} Zool. (See quot.) Contrasted with serotinous.

\P 1900 QUEKETT  \textit{Microsc. Club Jrnl. Ser.} ii. VII. 260 All the social or colonial Radiolarians (Polycyttaria) and most of the Acantharia are precocious, for in them the nucleus divides early in the life history of the cell.

\itembf{b.} = PRÆCOCIAL a.

\P 1897 PARKER \& HASWELL  \textit{Text-bk. Zool.} II. xiii. 382 The newly-hatched young may be‥well covered with down and able to run or swim and to obtain their own food, in which case they are said to be precocious.    
\P 1970 R. A. \&  B. M. Maier \textit{Compar. Animal Behavior} ix. 193 Domestic chicks are precocious (well developed at hatching).
\end{myenumerate}

%%%%%%%%%%%%%%%%%%%%%%%%%%%%%%%%
\myitem{meridian} n.

\noindent \phonetic{(məˈrɪdɪən)}

\noindent [From various elliptical uses of meridian a., chiefly adopted from OF. or med.L.

   Cf. L. merīdiānum (sc. tempus), noon; merīdiānum, the south; med.L. meridiāna (OF. méridiane, earlier meriene; mod.F. méridienne), noon, midday rest, siesta; F. méridien = sense 4 below; méridienne (= ligne m.), a meridian line.]
\vspace{-0.3cm}

\begin{myenumerate}
\itembf{1.} Mid-day, noon. Obs. exc. in humorously pedantic use.

\P 1380 ST. Augustine 1673 in  \textit{Horstm. Altengl. Leg.} (1878) 90 Vppon a day aftur þe meridien Austin apeered to him þen.    c 
\P 1391 CHAUCER  \textit{Astrol.} ii. §44 Adde hit [to-geder], and þat is thy mene mote, for the laste meridian of the december, for the same ȝere wyche þat þou [hast] purposid.    
\P 1637 HEYWOOD  \textit{Lond. Mirrour} Wks. 1874 IV. 311  The very day that doth afford him light, Is Morning, the Meridian, Evening, Night.    
\P 1871 G. MEREDITH  \textit{H. Richmond} xlii, If any thing fresh occurred between meridian and six o'clock, he should be glad, he said, to have word of it by messenger.

\itembf{b.} night's meridian: ‘the noon of night’, midnight. nonce-use.

\P 1826 CARRINGTON  \textit{Dartmoor} 62 A fearful gloom, deep'ning and deep'ning, till 'Twas dark as night's meridian.

\itembf{c.} Hist. A mid-day rest or siesta. [tr. med.L. meridiana; cf. F. méridienne, OF. merien(n)e.]

\P 1801 J. MILNER  \textit{Hist. Winchester} II. 101 There was now a vacant space of an hour or an hour and an half, during part of which those [monks] who were fatigued were at liberty to take their repose,‥which was called from the time of day when it was taken, The Meridian.    
\P 1820 SCOTT  \textit{Monast.} xix [Abbot loq.], As we have‥in the course of this our toilsome journey, lost our meridian, indulgence shall be given [etc.].

\itembf{d.} Sc. A mid-day dram. (See also E.D.D.)

\P 1818 SCOTT  \textit{Hrt. Midl.} iv, Plumdamas joined the other two gentlemen in drinking their meridian (a bumper-dram of brandy).    
\P 1825 CHAMBERS  \textit{Trad. Edinb.} II. 243 It was then [18th c.] the custom of all the shop-keepers in Edinburgh to drink what they called their meridian. This was a very moderate debauch,—consisting only in a glass of usquebaugh and a draught of small ale.

\itembf{2.} The point at which the sun or a star attains its highest altitude.

\P 1450 LYDG.  \textit{Secrees} 347 Phebus‥In merydien fervent as the glede.    
\P 1647 CRASHAW  \textit{Poems} 130 Sharp-sighted as the eagle's eye, that can Outstare the broad-beam'd day's meridian.    a 
\P 1667 COWLEY  \textit{Ess.}, Greatness, There is in truth no Rising or Meridian of the Sun, but only in respect to several places.    
\P 1728 POPE  \textit{Dunc.} iii. 195 note, The device, A Star rising to the Meridian, with this Motto, Ad Summa.    
\P 1843 JAMES  \textit{Forest Days} viii, The sun had declined about two hours and a half from the meridian.

\itembf{b.} fig. The point or period of highest development or perfection, after which decline sets in; culmination, full splendour.

\P 1613 SHAKES.  \textit{Hen. VIII}, iii. ii. 224 And from that full Meridian of my Glory, I haste now to my Setting.    
\P 1638 SIR T. HERBERT  \textit{Trav.} (ed. 2) 93 Yet in the meridian of his hopes [he] is dejected by valiant Rustang.    c 
\P 1645 HOWELL  \textit{Lett.} (1655) III. ix. 17 Naturall human knowledg is not yet mounted to its Meridian, and highest point of elevation.    
\P 1673 TEMPLE  \textit{United Prov.} Wks. 1731 I. 67,  I am of Opinion, That Trade has, for some Years ago, pass'd its Meridian, and begun sensibly to decay among them.    
\P 1700 DRYDEN  \textit{Fables} Pref. *Bb, Ovid liv'd when the Roman Tongue was in its Meridian; Chaucer, in the Dawning of our Language.    a 
\P 1761 CAWTHORN  \textit{Poems} (1771) 61 My merit in its full meridian shone.    a 
\P 1859 MACAULAY  \textit{Hist. Eng.} xxiii. (1861) V. 67 This was the moment at which the fortunes of Montague reached the meridian. The decline was close at hand.    
\P 1893 G. HILL  \textit{Hist. Eng. Dress} II, 268 Dress was in its meridian of ugliness.

\itembf{c.} The middle period of a man's life, when his powers are at the full.

\P 1645 HOWELL  \textit{Lett.} i. vi. lx. (1655) 307 You seem to marvell I do not marry all this while, considering that I am past the Meridian of my age.    
\P 1703 E. WARD  \textit{Lond. Spy} xvii. (1706) 406 As for her Age, I believe she was near upon the Meridian.    
\P 1795 MASON  \textit{Ch. Mus.} ii. 133 When Purcel was in the meridian of his short life.    
\P 1864 H. AINSWORTH  \textit{John Law} Prol. iii. (1881) 19 Though long past his meridian, and derided as an antiquated beau by the fops of the day.    
\P 1873 HAMERTON  \textit{Intell. Life} iv. ii. (1875) 143 Any person who has passed the meridian of life.

\itembf{3.} The south. Obs. [So L. \textit{meridianum}.]

\P 1430-40 LYDG.  \textit{Bochas} vi. i. (1494) t ij b, Nowe in the west, nowe in the oryent, To sech stories north and meredien Of worthy princes that here to fore haue ben.    
\P 1432-50 tr.  \textit{Higden} (Rolls) I. 47 Asia‥whiche goenge from the meridien or sowthe by the este vn to the northe, is compassede on euery syde with the occean.    Ibid. VI. 41 Machomete made an ydole‥havynge the face of hit towarde the meridien.    
\P 1601 HOLLAND  \textit{Pliny} I. 34 With vs the stars about the North Pole neuer go downe, and those contrariwise about the Meridian neuer rise.    Ibid. 48 From the Meridian or South-point to the North.

\itembf{4.} [Ellipt. for meridian circle or line.] \textbf{a.} Astr. (More explicitly celestial m.) The great circle (of the celestial sphere) which passes through the celestial poles and the zenith of any place on the earth's surface. \itembf{b.} (More explicitly terrestrial m.) The great circle (of the earth) which lies in the plane of the celestial meridian of a place, and which passes through the place and the poles; also often applied to that half of this circle that extends from pole to pole through the place.

   So named because the sun crosses it at noon. A terrestrial globe, or a map of the earth or part of it, has usually a number of meridians drawn upon it at convenient distances, marked with figures indicating their respective longitude or angular distance on a parallel from the first meridian, i.e. the meridian (in British maps that of Greenwich) conventionally determined to be of longitude 0°.

\P 1391 CHAUCER  \textit{Astrol.} ii. §39 And [yf] so be þat two townes haue illike Meridian, or on Meridian, than is the distance of hem bothe ylike fer fro the Est.    
\P 1549  \textit{Compl. Scot.} vi. 51 Quhen the sune rysis at our est orizon, than it ascendis quhil it cum til our meridian.    
\P 1555 EDEN  \textit{Decades} 243 And commaunded a line or meridian to bee drawen Northe and south.    
\P 1594 BLUNDEVIL  \textit{Exerc.} iv. xviii. (1636) 461 Whereas the Terrestriall Globe is traced with 12 Meridians,‥The Celestiall Globe is only traced with 6 Meridians.    
\P 1669 STURMY  \textit{Mariner's Mag.} ii. 93 You must wait‥till the Sun is upon the Meridian.    
\P 1678 HOBBES  \textit{Decam.} viii. 101 It will turn it self till it lye in a Meridian, that is to say, with one and the same Line still North and South.    
\P 1698 J. KEILL  \textit{Exam. Th. Earth} (1734) 231 All those who live under the same Meridian have twelve of the Clock at the same time.    
\P 1715 tr.  \textit{Gregory's Astron.} I. 211 Any such Secondary Circle drawn thro' any Place upon the Earth, is called the Meridian of that Place.    Ibid. 212 They feigned therefore a first Meridian passing thro' the most Western Place of the Earth, that was then known.    
\P 1839 PENNY  \textit{Cycl.} XV. 110/1 The terrestrial meridian is the section of the earth made by the plane of the celestial meridian.    
\P 1841 ELPHINSTONE  \textit{Hist. Ind.} II. 177 These two rájas soon reduced the Mussulman frontier to the Kishna on the south, and the meridian of Heiderábád on the east.

\itembf{c.} transf. (a) Geom. Occasionally applied to any great circle of a sphere that passes through the poles, or to a line, on a surface of revolution, that is in a plane with its axis. (b) magnetic meridian: the great circle of the earth that passes through any point on its surface and the magnetic poles.

\P 1704 J. HARRIS  \textit{Lex. Techn.} I, Meridian Magnetical, is a Great Circle passing through or by the Magnetical Poles.    a 
\P 1721 J. KEILL  \textit{Maupertuis' Diss.} (1734) 47 The Meridians of the Spheroids are continually Algebraic Curves.    
\P 1832 \textit{Nat.  Philos.} II. Magnet. iii. 23 (Usef. Knowl. Soc.), The magnetic meridian.    
\P 1837 BREWSTER  \textit{Magnet.} 11 He‥made numerous experiments with bars of iron and steel placed in the magnetic meridian.

\itembf{d.} meridian of a globe or brass meridian: a graduated ring (sometimes a semicircle only) of brass in which an artificial globe is suspended and revolves concentrically.

\P 1633 G. HERBERT  \textit{Temple, Size} viii, An earthly globe, On whose meridian was engraven, These seas are tears, and heav'n the haven.    
\P 1727-51 CHAMBERS  \textit{Cycl.} s.v. Globe, The globe itself thus finished, they hang it in a brass meridian.

\itembf{e.} attrib. in meridian circle (see also meridian a. 3), an astronomical instrument consisting of a telescope carrying a large graduated circle, by which the right ascension and declination of a star may be determined; a transit-circle; meridian-mark, a mark fixed at some distance due north or south of an astronomical instrument, by pointing at which the instrument is set in the meridian.

\P 1849 HERSCHEL  \textit{Outl. Astron.} §190. 114 Thus also a meridian line may be drawn and a meridian mark erected.

\itembf{5.} transf. and fig. A locality or situation, considered as separate and distinct from others, and as having its own particular character; the special character or circumstances by which one place, person, set of persons, etc. is distinguished from others. Chiefly in figurative uses of astronomical phrases such as calculated to or for the meridian of = ‘suited to the tastes, habits, capacities, etc., of’.

\P 1589 R. HARVEY  \textit{Pl. Perc.} Ded. 4. I will present you at the law day for a ryot, though I be neither side man for this Meridian, nor Warden.    
\P 1621 BURTON  \textit{Anat. Mel.} ii. ii. i. i. (1651) 231 Which howsoever I treat of, as proper to the Meridian of Melancholy.    
\P 1625 B. JONSON  \textit{Staple of N.}, Prol. Court, A Worke‥fitted for your Maiesties disport, And writ to the Meridian of your Court.    
\P 1647 CLARENDON  \textit{Hist. Reb.} vii. §73 He was, at his suit, brought to the House of Commons' bar; where‥with such flattery as was most exactly calculated to that meridian [etc.].    a 
\P 1677 HALE  \textit{Prim. Orig. Man.} i. i. 7 All other knowledge meerly or principally serves the concerns of this Life, and is fitted to the meridian thereof.    
\P 1712 ARBUTHNOT  \textit{John Bull} iii. Publisher's Pref., Though they had been calculated by him only for the meridian of Grub-street, yet they were taken notice of by the better sort.    a 
\P 1718 PENN \textit{Tracts} Wks. 1726 I. 471  His words of the Trinity are modest, neither highly Athanasian, nor yet Socinian,‥but calculated to both Meridians.    
\P 1748 SMOLLETT  \textit{Rod Rand.} xxviii. (1804) 186 This suggestion‥had the desired effect upon the captain, being exactly calculated for the meridian of his intellects.    
\P 1751 EARL OF ORRERY  \textit{Remarks Swift} (1752) 141 As this pamphlet was written for the meridian of Ireland.    
\P 1816 \textit{Sporting  Mag.} XLVIII. 34 This‥could not fail in exciting ludicrous ideas, in the minds of the illiterate vulgar, for whose meridian it was calculated.    
\P 1835 W. IRVING  \textit{Newstead Abbey} Crayon Misc. (1863) 306 A course of anecdotes‥such as suited the meridian of the‥servants' hall.





\end{myenumerate}

%%%%%%%%%%%%%%%%%%%%%%%%%%%%%%%%
\myitem{defer} v.1

\noindent \phonetic{(dɪˈfɜː(r))}

\noindent [ME. differre-n, a. OF. différer (il diffère), 14th c. in Littré, ad. L. differ-re to carry apart, put off, postpone, delay, protract; also, intr., to bear in different directions, have diverse bearings, differ. Orig. the same word as differ v. (q.v. for the history of their differentiation), and often spelt differ in 16-17th c.; but forms in de-, def-, are found from the 15th, and have prevailed, against the etymology, mainly from the stress being on the final syllable; but partly, perhaps, by association with delay.]
\vspace{-0.3cm}

\begin{myenumerate}

\itembf{1.} trans. To put on one side; to set aside. Obs.

\P 1393 GOWER  \textit{Conf.} I. 262 At mannes sighte Envie for to be preferred Hath conscience so differred, That no man loketh to the vice Whiche is the moder of malice.    c 
\P 1430 LYDG.  \textit{Hors, Shepe \& G.} 96 The Syrcumstaunce me lyst nat to defer.      Min. Poems (Percy Soc.) 14 Grace withe her lycour cristallyne and pure Defferrithe vengeaunce off ffuriose woodnes.

\itembf{b.} To set or put ‘beside oneself’; to bereave of one's wits. Obs. rare.

\P 1375  \textit{Sc. Leg. Saints, Matthæus} 84 Quhame þat þai [two sorcerers] had euir marryte Ine þare wittis or differryte.

\itembf{c.} refl. To withdraw or remove oneself. Obs.

\P 1375  \textit{Sc. Leg. Saints, Martha} 171 Hely, defere þe nocht fra me, Bot in myn helpe nov haste þu þe!

\itembf{2.} trans. To put off (action, procedure) to some later time; to delay, postpone.

\P 1382 WYCLIF  \textit{Num.} xxx. 15 If the man‥into another day deferre the sentence.    14‥ Prose Legends in Anglia VIII. 132 [She] differred þe questyone.    
\P 1483  \textit{Cath. Angl.} 99 To Differ, differre, prolongare.    
\P 1489 CAXTON  \textit{Faytes of A.} ii. vii. 104 The Lacedemonyens with drewe them self and differde the bataylle.    
\P 1526 TINDALE  \textit{Matt.} xxiv. 48 My master wyll differ his commynge.    
\P 1593 SHAKES.  \textit{2 Hen. VI,} iv. vii. 141 Soldiers, Deferre the spoile of the Citie vntill night.    
\P 1651 HOBBES  \textit{Leviath.} ii. xxx. 183 Sometimes a Civill warre, may be differred, by such wayes.    
\P 1711 ADDISON  \textit{Spect.} No. 92 \phonetic{⁋}2, I have deferred furnishing my Closet with Authors, 'till I receive your Advice.    
\P 1795 SOUTHEY  \textit{Joan of Arc} iv. 499 O chosen by Heaven! defer one day thy march.    
\P 1863 GEO. ELIOT  \textit{Romola} ii. iv, She deferred writing the irrevocable words of parting from all her little world.

\itembf{b.} Const. with inf. ? Obs.

\P 1426 CARD. BEAUFORT in Ellis \textit{Orig. Lett.} Ser. ii. I. 102 He hath long differred to parfourme them.    c 
\P 1450  \textit{St. Cuthbert} (Surtees) 7118 To wende hame þai noȝt deferde.    
\P 1535 COVERDALE  \textit{Josh.} x. 13 The Sonne‥dyfferred to go downe for the space of a whole daye after.    
\P 1609 BIBLE (Douay)  \textit{Ps.} lxxix. Comm., How long wilt thou differre to heare our prayer?    a 
\P 1656 USSHER  \textit{Ann.} (1658) 880 Neither did he long defer to put those Jews to death.    a 
\P 1732 ATTERBURY  (J.), The longer thou deferrest to be acquainted with them, the less every day thou wilt find thyself disposed to them.

\itembf{c.} absol. or intr. To delay, procrastinate: rarely with off.

\P 1382 WYCLIF  \textit{Deut.} vii. 10 So that he scater hem, and ferther differre not [1388 Differr  [v.r. tarie] no lengere].    c 
\P 1450  \textit{St. Cuthbert} (Surtees) 7523 He defard, and walde noȝt trus.    
\P 1577 J. NORTHBROOKE  \textit{Dicing} (1843) 180 Whyles he desired, they deferred.    a 
\P 1592 GREENE \& LODGE  \textit{Looking Glass} Wks. (Rtldg.) 129/1 Defer not off, to-morrow is too late.    
\P 1614 BP. HALL  \textit{Recoll. Treat.} 935 God differ's on purpose that our trials may be perfect.    
\P 1635 R. BOLTON  \textit{Comf. Affl. Consc.} ix. 252 The longer thou putst off and defferest the more unfit shalt thou be to repent.    
\P 1742 YOUNG  \textit{Nt. Th.} i. 390 Be wise to-day; 'tis madness to defer.    
\P 1771 P. PARSONS  \textit{Newmarket} I. 21, I have waited (demurred, my gentle reader, if you be a lawyer, deferred, if you be a divine)‥a full year.

\itembf{3.} trans. To put off (a person or matter) to a future occasion: \itembf{a.} a person. Obs.

\P 1382 WYCLIF  \textit{Acts} xxiv. 22 Sothli Felix deferride hem [1388 Delayede,  MS. K. ether differride; Tindale differde, 1539 Great B. deferede, 1557 Genev. differed, 1582 Rhem. differred, 1611 and 1881 Deferred].    
\P 1545 BRINKLOW  \textit{Compl.} 20 b, Men be differyd from tyme to tyme, yea from yere to yere.    
\P 1642 ROGERS  \textit{Naaman} 137 If it seem good to thy wisdome to deferre me.    
\P 1709 STRYPE  \textit{Ann. Ref.} I. xxxviii. 440 He was deferred until Monday.

\itembf{b.} a time, matter, question.

\P 1509 BARCLAY  \textit{Shyp of Folys} (1570) 49 Where they two borowed, they promise to pay three, Their day of payment longer to defarre.    
\P 1536 \textit{Exhort.  fr. North} 135 in Furniv. Ballads I. 309 Differ not your matteres tyll a new ȝere.    
\P 1559 MORWYNG  \textit{Evonym.} 95 Which conserveth the good health of man's body, prolongeth a man's youth, differeth age.    
\P 1559 WILLOCK  \textit{Lett. to Crosraguell in Keith Hist. Church Sc.} App. 198 (Jam.), I wold aske quhilk of us differreth the Caus.    
\P 1611 BIBLE  \textit{Prov.} xiii. 12 Hope deferred maketh the heart sicke.

\itembf{c.} To relegate to a later part of a treatise.

\P 1538 STARKEY  \textit{England} i. iv. 123 Let us not entur into thys dysputatyon now, but‥dyffer hyt to hys place.    
\P 1558 KNOX  \textit{First Blast} (Arb.) 37 The admonition I differe to the end.    
\P 1611 CORYAT  \textit{Crudities} 480, I had differred it till the end of the sermon.    
\P 1695 WOODWARD  \textit{Nat. Hist. Earth} i. (1723) 41 Which I choose, rather than trouble the Reader with a Detail‥here, to deferr to their proper Place.    
\P 1877 J. D. CHAMBERS  \textit{Divine Worship} 284 It has been found necessary to defer them to the Appendix.

\itembf{d.} To postpone the military call-up of (a person, esp. one in a protected occupation). Usu. in pass. U.S.

\P 1941 \textit{Nation}  (N.Y.) 17 May 596/1 The national draft board should promulgate a ruling to the effect that no worker deferred because of his employment in defense shall lose that deferment merely because he joins his fellow-workers in a strike.    
\P 1951 \textit{Senior  Scholastic} 25 Apr. 12/2 (heading) Should superior college students be deferred?    
\P 1969 M. PUZO  \textit{Godfather} i. i. 62 Paulie Gatto had been deferred from the draft himself because [he]‥had received electrical shock treatments for a mental condition.

\itembf{4.} To put off (time), waste in delay. Obs.

\P 1382 WYCLIF  \textit{Ezek.} xii. 22 Dais shulen be differrid, or drawen, in to loong [1388 Differrid in  to long tyme].    
\P 1548 HALL  \textit{Chron.} 184 Not mynding to differre the time any farther.    
\P 1579 LYLY  \textit{Euphues} (Arb.) 123 Idle to deferre ye time lyke Saint George, who is euer on horsebacke yet neuer rydeth.    
\P 1591 SHAKES.  \textit{1 Hen. VI}, iii. ii. 33 Deferre no tyme, delayes haue dangerous ends.    
\P 1633 G. HERBERT  \textit{Temple, Deniall} vi, O cheer and tune my heartlesse breast, Deferre no time.

\itembf{b.} To protract; also intr. to linger. Obs.

\P 1546 LANGLEY  \textit{Pol. Verg. De Invent.} i. xii. 24 a, The Warres were longe differred.    
\P 1561 NORTON \& SACKV.  \textit{Gorboduc} iv. ii, Why to this houre Have kind and fortune thus deferred my breath?    
\P 1561 HOLLYBUSH  \textit{Hom. Apoth.} 42 b, If the disease woulde differre, and the jaundis woulde not voyde.
\end{myenumerate}

%%%%%%%%%%%%%%%%%%%%%%%%%%%%%%%%
\myitem{procure} v.

\noindent \phonetic{(prəʊˈkjʊə(r))}

\noindent [a. F. procurer (13th c. in Littré), ad. L. prōcūrāre to take for, take care of, attend to, manage, to act as procurator: see pro-1 and cure v. In ME. usually stressed on the first syllable, 'procure (from F. inf. procu'rer); hence the weakened β-forms 'procur, etc. 
\vspace{-0.3cm}

\begin{myenumerate}
\itembf{I. 1.} trans. To care for, take care of, attend to, look after. [So in L., and OF.] Obs. rare.

\P 1425 WYNTOUN  \textit{Cron.} vi. iv. 357 (Cott. MS.) Bot þe possessoure to procure [Wemyss MS. trete]‥wiþe honoure, And habundance of reches.    
\P Ibid. viii. xxiv. 3648 Our Kynge Dauid was sende in Frawns, Qwhar he‥was‥procuryt [v.r. tretit] in al esse ilk deil.

\itembf{2.} intr. To put forth or employ care or effort; to do one's best; to endeavour, labour; to use means, take measures. Const. inf. with to (for to); for, to, unto a thing. Obs.

\P 1330 R. BRUNNE  \textit{Chron. Wace} (Rolls) 7462 Þus þey þrete wyþ manace, \& ful yuel þey procure \& purchace.    c 
\P 1380 \textit{Antecrist} in Todd  \textit{Three Treat. Wyclif} (1851) 127 Crist fleed from seculer lordschip \& office; þei procuren fast to have it.    c 
\P 1380 \textit{Sir Ferumb.}  5825 Thar-for ert þow mys-byþoȝte, To procury hym to slee.    c 
\P 1400 \textit{Brut}  249 Þai were his enemys‥and procurede forto make debate and contak bituene him and his sone.    c 
\P 1430 \textit{Syr  Gener.} (Roxb.) 9220 Vnto his deliueraunce he procured.    
\P 1509 \textit{Parl.  Devylles} ad fin., Who that wyll for heuen procure, Kepe hym fro the deuylles combrement.    
\P 1548 UDALL  \textit{Erasm. Par.} Pref. 3 To procure for the commodities and welth of Englande.    
\P 1561 T. HOBY tr. \textit{Castiglione's Courtyer} i. (1577) D iv b, Such a countenaunce as this is,‥and not so softe and womanish as many procure to haue.    
\P 1582 N. LICHEFIELD tr. \textit{Castanheda's Conq. E. Ind.} i. i. 3 Hee gaue them charge‥that they shoulde procure to atteine to the sight of Presbiter Ioan.    
\P 1608 R. JOHNSON  \textit{Seven Champions} ii. I iv b, Rosana‥did procure to defend her selfe and offend hir enemie.

\itembf{3.} trans. To contrive or devise with care (an action or proceeding); to endeavour to cause or bring about (mostly something evil) to or for a person. Obs.

\P 1290 BEKET 1258 in  \textit{S. Eng. Leg.} I. 142 A-morewe comen þis bischopes and þe eorles also, To procuri seint thomas al þat vuel þat heo miȝten do.    
\P 13‥ \textit{Seuyn Sag.} (W.) 1201 He the  procureth, night and dai, Al the sschame that he mai.    
\P 13‥ \textit{Coer de L.} 1730, I pray thee, Sir Tanker king, Procure me none evil thing.    
\P 1484 CAXTON  \textit{Fables of Alfonce} v, Ofte‥the euyll whiche is procured to other cometh to hym whiche procureth it.    
\P 1530 PALSGR. 667/1, I procure, I cause a thyng to be done, or I devyse meanes to bringe a thynge to passe, je procure.    
\P 1573-80 BARET  \textit{Alv.} P 740 To procure hatred, or euill will to men, struere odium in aliquos.    
\P 1620 J. WILKINSON  \textit{Courts Leet} 136 Yee shall reasonably and honestly procure the profit of the corporation of this Towne.

\itembf{b.} ? To care for; ? to endeavour to get or do.

\P 1574 HELLOWES  \textit{Gueuara's Fam. Ep.} (1577) 308 For women be of such quality, that they procure nothing [que ninguna cosa tanto procuran] so much as that which is most forbidden them.

\itembf{II. 4.} To bring about by care or pains; also (more vaguely) to bring about, cause, effect, produce. 

\itembf{a.} with simple object. Now rare.

\P 1340 HAMPOLE  \textit{Prose Tr.} 11 All maner of wilfull pollusyone procurede one any maner agaynes kyndly oys.    
\P 1387 TREVISA  \textit{Higden} (Rolls) V. 215 Þe emperesse Eudoxia had i-procured þe out puttynge [procuravit ejectionem] of Iohn.    Ibid. VI. 243 He sente Alcuinus‥for to procure pees.    
\P 1554 BRADFORD in  \textit{Strype Eccl. Mem.} (1721) III. App. xxx. 84 It is we‥that have sinned and procured thy grievous wrath upon us.    
\P 1615 G. SANDYS  \textit{Trav.} i. 66 A drinke called Coffa‥which helpeth‥digestion, and procureth alacrity.    
\P 1677 W. HARRIS tr. \textit{Lemery's Chym.} (1686) 536 It is good to procure sweat.    
\P 1748 SMOLLETT  \textit{Rod. Rand.} xii, This second sneer procured another laugh against him.    
\P 1861 E. O'CURRY  \textit{Lect. MS. Materials} 252 His uncle Cobhthach soon procured his death by means of a poisoned drink.

\itembf{b.} with subordinate clause. arch.

\P 1340 HAMPOLE  \textit{Psalter} lxviii. 12 Sum procurd þat .i. sould dye.    
\P 1551 ROBINSON tr.  \textit{More's Utop. Ep. P. Giles} (1895) 8 He is mynded to procure that he maye be sent thether.    
\P 1654 tr.  \textit{Martini's Conq. China} 226, I will procure all Europe shall understand the Issue of these prodigious revolutions.    
\P 1711 \textit{Medley}  No. 40 They procur'd that Mony shou'd be lent at 5 per Cent.    
\P 1894 R. BRIDGES  \textit{Feast of B.} i. 301 Could you procure that I should speak with her?

\itembf{c.} with inf. To manage (to do something). Obs.

\P 1559  \textit{Mirr. Mag.} (1563) H v b, Eyther I must procure to see them dead, Or for contempt as a traytour lose my head.    
\P 1587 FLEMING  \textit{Contn. Holinshed} III. 1378/2 Sir Roger Manwood‥procured to pas another act of parlement, ‥ wherein is further prouision made for the said bridge.    
\P 1678 R. BARCLAY  \textit{Apol. Quakers} ii. iii. 25 Men‥have procured to be esteemed as Masters of Christianity, by certain Artificial Tricks.

\itembf{d.} with obj. and inf. pass. To cause or get (a person or thing) to be treated in some way; to get something done to (a person). Now rare.

\P 1450 MYRC  696 All that vnrightfully defameth eny person or prokereth to be famed.    
\P 1577 B. GOOGE  \textit{Heresbach's Husb.} i. (1586) 7 b, Procuring him to be sent in embassage.    a 
\P 1626 BACON  \textit{Civ. Char. Jul. Cæsar Ess.} (1696) 161 He procured to be enacted no wholsome Laws.    
\P 1724 A. COLLINS  \textit{Gr. Chr. Relig.} 34 They procur'd him to be crucify'd.    
\P 1794 PALEY  \textit{Evid.} ii. ix. (1817) 216 [Nero] procured the Christians to be accused.    
\P 1866 HOWELLS  \textit{Venet. Life} v. 68 An ingenious lover procured his‥rival to be arrested for lunacy.

\itembf{5.} To obtain by care or effort; to gain, win, get possession of, acquire. (Now the leading sense.) In early use, to gain the help of, to win over (a person) to one's side.

\P 1297 R. GLOUC.  (Rolls) 11483 Sir Ion‥turnde  aȝe sir simond \& procurede oþer mo.    c 
\P 1330 R. BRUNNE  \textit{Chron.} (1810) 119 Mald in Bristow lettres fast sendes, Bi messengers trow, forto procore frendes.

\P 1387 TREVISA  \textit{Higden} (Rolls) VI. 355 He was þe firste þat ordeyned comyn scole at Oxenforde‥, and procrede fredom and priveleges in many articles to þat citee.    
\P 1451 J. CAPGRAVE  \textit{Life St. Aug.} 50 The first þat he schuld neuyr procur no wyf to no man.    
\P 1538 STARKEY  \textit{England} i. i. 7 Hyt ys bettur‥for a man being in gret pouerty, rather to procure some ryches then hye phylosophy.    
\P 1596 DALRYMPLE tr.  \textit{Leslie's Hist. Scot.} iv. 256 To him selfe he procuiret the fame of all æquitie.    
\P 1611 BIBLE  \textit{Transl. Pref.} 2 This‥procured to him great obloquie.    
\P 1718 LADY  M. W. MONTAGU \textit{Let. to Abbé Conti} 19 May, Things that 'tis very easy to procure lists of.    
\P 1776 \textit{Carlisle  Mag.} 7 Sept. 143 She endeavoured to procure employment as a needle$\sim$woman.    
\P 1874 GREEN  \textit{Short Hist.} iii. §4. 134 Books were difficult and sometimes even impossible to procure.    Mod. Could you procure me specimens?

\itembf{b.} To obtain (women) for the gratification of lust. Usually absol. or intr. To act as a procurer (sense 4) or procuress.

\P 1603 SHAKES.  \textit{Meas. for M.} iii. ii. 68 How doth my deere Morsell, thy Mistris? Procures she still?    
\P 1706 PHILLIPS,  \textit{Procure},‥is also taken in an ill Sense, for to act as a Pimp or Bawd.    
\P 1745 CHESTERFIELD  \textit{Lett.} (1792) I. 282 Juno‥offers to procure for Aeolus, by way of bribe.    
\P 1891  \textit{Daily News} 26 Jan. 7/2 Charged‥at the Lambeth Police-court, on Saturday, with that he did by false pretences procure E. A. H.

\itembf{6.} To prevail upon, induce, persuade, get (a person) to do something. Obs. or arch.

\P 1340-70 \textit{Alex. \& Dind. } 347 Ne we agayn hem to do [ed. go] nol no gome procre.    c 
\P 1380 WYCLIF  \textit{Sel. Wks.} III. 342 Hou þat Clement left his office and procuride oþir to helpe him.    
\P 1401  \textit{Pol. Poems} (Rolls) II. 25 Why procurest thou men to yeve the their almes?    
\P 1568 GRAFTON  \textit{Chron.} II. 184 Pope Boniface being informed and procured by the Scottes, sent his letters vnto the king of England.    
\P 1579 FENTON  \textit{Guicciard.} ii. (1599) 75 The newes of the reuolt of Nouaro, procured the King‥to make way.    
\P 1667 EVELYN  \textit{Diary} 19 Sept., I procur'd him to bestow them [the Arundelian Marbles] on the University of Oxford.    
\P 1736 \textit{Hale's  Placit. Coron.} I. 615 An accessory before is he, that being absent at the time of the felony committed doth yet procure, counsel, command, or abet another to commit a felony.    
\P 1756 C. LUCAS  \textit{Ess. Waters} II. 144 The writer is influenced or procured to write for the one, against the other.    
\P 1828 S. TURNER  \textit{Anglo-Sax.} (ed. 5) I. iii. x. 245 Charlemagne communicates to him [Offa]‥his success in procuring the continental Saxons to adopt Christianity.

\itembf{b.} spec. Law. To induce privately, to suborn, to bribe (a witness, juryman, etc.). Obs.

\P [1292 BRITTON  i. ii. §11 Et si defendoms a touz Corouners‥qe nul face ses enquestes‥par amis procurez.]    
\P 1433  \textit{Rolls of Parlt.} IV. 476/1 Whether they‥be procured to chese eny persone‥to eny maner Office‥and yf eny persone‥be founde procured, that then he or thei be remeved.    
\P 1573-80 BARET  \textit{Alv. P} 741 A witnes procured with monie, or bribes, conflatus pecuniâ testis.    
\P 1620 J. WILKINSON  \textit{Coroners \& Sherifes} 44 Ye shall‥make your pannels your selfe of such persons, as bee‥not suspect, nor procured.

\itembf{c.} With adv. of place: To induce or prevail upon (a person) to come; to bring, lead. Obs.

\P 1586 J. HOOKER  \textit{Hist. Irel.} in \textit{Holinshed} II. 130/2 [They] agreed to cause Tirlough Lennough to procure in the Scots.    
\P 1592 SHAKES.  \textit{Rom. \& Jul.} iii. v. 68 What vnaccustom'd cause procures her hither?    a 
\P 1604 HANMER  \textit{Chron. Irel.} (1633) 128 Neither were we procured hither to be idle, or live deliciously.    
\P 1625 SHIRLEY  \textit{Love Tricks} iv. ii, Yonder is a pleasant arbour, procure him thither.

\itembf{7.} To try to induce; to urge, press. Obs.

\P 1551 EDW. VI \textit{Let. Sir B. Fitz-Patrick} 20 Dec. in \textit{Lit. Rem.} (Roxb.) I. 69 If yow be vehemently procured yow may goe as waiting on the king.    
\P 1581 J. BELL  \textit{Haddon's Answ. Osor.} 219 b, Where did he euer shake of the obedience of due allegeaunce? or procured any Subjectes to rebellion agaynst their Gouernours?    
\P 1590 SPENSER  \textit{F.Q.} iii. i. 1 The famous Briton Prince and Faery Knight,‥Of the faire Alma greatly were procur'd To make there lenger soiourne and abode.

\itembf{III. 8.} intr. To act as a procurator or legal agent; to solicit. (In quot. 1401, To act by a proctor or attorney.) Obs.

\P 1380 WYCLIF  \textit{Serm.} Sel. Wks. I. 383 Many trewe men, boþe aprentis and avocatis, wolen not procure in a cause bifore þat þei heeren it.    
\P 1401  \textit{Pol. Poems} (Rolls) II. 34 You wend or send or procure to the court of Rome, to be made cardinals or bishops of the popes chaplens.    
\P 1528 WOLSEY in  \textit{St. Papers Hen. VIII}, I. 291 What promysse I demaunded of the said Emperours Ambassadour, who said he wolde procure for restitution.    
\P 1536 in  \textit{Strype Cranmer} ii. (1694) 36 There should be as many‥admitted to procure there as shuld be seen convenient to my said Lord of Canterbury.    
\P 1539  \textit{Sc. Acts Jas. V} (1814) II. 353/2 Ane writing subscriuit be þe kingis grace‥chargeing him \& certane vþeris his collegis to procure for þe said James.

\itembf{b.} fig. To plead, make supplication. Obs.

\P 1563 WINȝET  \textit{Four Scoir Thre Quest.} To Rdr., Wks. I. 57 For in defence of that thing only procuir I, quhilk‥the haill Kirk of God‥maist clerlie appreuis.    a 
\P 1568 R. NORVALL  \textit{O most eternall King} 91 in Bannatyne MS. 51 Thairfoir to God for grace procure: He that wold leif most lerne to dy.    a 
\P 1578 LINDESAY  (Pitscottie) \textit{Chron. Scot.} ii. xxiii. (S.T.S.) I. 351 The king‥procurit for his lyfe at the bischopis handis.    a 
\P 1615 BRIEUE  \textit{Cron. Erlis of Ross} (1850) 13 He procurit to him, by nature inclynit to follow such counsel, to mak war in his favour.

\itembf{IV. 9.} intr. ? To proceed, advance. Obs. rare.
   (Sense and sematology obscure.)

\P 1490 CAXTON  \textit{Eneydos} xiii. 47 In her thoughte the wounde of ambycyouse desyre‥is so procured that she can not hyde it noo lenger.    
\P 1573 TUSSER  \textit{Husb.} (1878) 146 His hatred procureth from naughtie to wurse, His friendship like Iudas that carried the purse.
\end{myenumerate}


%%%%%%%%%%%%%%%%%%%%%%%%%%%%%%%%
\myitem{exonerate} v.

\noindent \phonetic{(ɛgˈzɒnəreɪt)}

\noindent [f. L. exonerāt- ppl. stem of exonerā-re, f. ex- (see ex- prefix1) + oner-, onus burden. Cf. Fr. exonérer.]
\vspace{-0.3cm}

\begin{myenumerate}

\itembf{1.} trans. To take off a burden from; to relieve of (a burden, material or immaterial); to unload, lighten (a ship); also humorously, to ‘relieve’ (a person) of his money. Now rare.

\P 1524 HEN. VIII. in  Strype \textit{Eccl. Mem.} I. App. xiii. 30 Discharging or exonerating their galeis.    
\P 1566 PAINTER  \textit{Pal. Pleas.} I. 46 [They] haue prayed to God to be exonerated of loue, aboue all other diseases.    
\P 1615 T. ADAMS  \textit{Spir. Navigator} 34 He strives to exonerate his shoulders.    a 
\P 1634 CHAPMAN  \textit{Bacchus} 110 Exonerate Our sinking vessel of his deified lode.    
\P 1637 BASTWICK  \textit{Litany} iii. 13 They would quickly exonerate their families of them.    
\P 1640 BP. REYNOLDS  \textit{Passions} xxi. 218 It exonerateth the mind of all those dulling Indispositions.    
\P 1785 BURKE  \textit{Sp. Nabob Arcot's Debts} Wks. IV. 308 The debt thus exonerated of so great a weight of its odium.    
\P 1798 WELLINGTON in  \textit{Owen Disp.} 29 Success would certainly exonerate our finances.    1807-8 Syd. Smith Plymley's Lett. x, Be exonerated of his ready money and his constitution.

\itembf{2.} To discharge the contents of (the body, an organ), esp. by evacuation. to exonerate nature, exonerate oneself: to relieve the bowels. Obs.

\P 1542 BOORDE  \textit{Dyetary} viii. (1870) 248 And exonerate your selfe at all tymes that nature wold expell.    Ibid. xxx. 293 To exonerat the blader and the bely whan nede shall requyre.    
\P 1615 G. SANDYS  \textit{Trav.} 65 They sit all the day long, vnlesse they rise to exonerate nature.    
\P 1634 SIR T. HERBERT  \textit{Trav.} 149 [They] over-load their mouthes‥and by a sudden laughter exonerate their chaps.    
\P 1710 T. FULLER  \textit{Pharm. Extemp.} 322 Cachectic Pills‥exonerate the Habit of the Body.    
\P 1829 \textit{Health \& Longevity } 269 The bowels‥ought to be exonerated at least once in two days.

\itembf{b.} intr. for refl. Obs.

\P 1631 R. H. ARRAIGNM.  \textit{Whole Creature} xiii. §1. 178 Over$\sim$charged‥till they‥exonerate as a Wolfe or Dog, too full gorged, with Carion.    
\P 1704 J. PITTS  \textit{Relig. \& Mann. Mahometans} iv. 25 These Moors‥accounting it a great piece of Rudeness to exonerate in the sight of another.    
\P 1762 B. STILLINGFL.  \textit{Econ. Nat. Misc. Tracts} 123 Care is taken that these animals should exonerate upon stones, etc.

\itembf{3.} refl. Of a lake, river, sea, etc., also of a blood-vessel: To empty itself, its waters, or contents; to disembogue, discharge. Obs.

\P 1598 HAKLUYT  \textit{Voy.} I. 113 Neither did this riuer exonerate itself into any sea.    
\P 1635 JACKSON  \textit{Creed} viii. xx. Wks. VIII. 43 We all meet in the main or ocean whereinto this psalm and others do exonerate themselves.    
\P 1659 MACALLO  \textit{Can. Physick} 25 The great Veines‥do exonerate themselves into the little.    
\P 1715 HALLEY in  \textit{Phil. Trans.} XXIX. 298 That [gulf] of Paria, into which the Lake of Titicaca does in part exonerate it self.

\itembf{4.} trans. \textbf{a.} To discharge, pour off (a fluid product, a body of water). \textbf{b.} To cast off, get rid of (persons, population). Obs. rare.

a. \P 1615 CROOKE  \textit{Body of Man} 429 It [the bile] is‥exonerated into that which is called the Caua or hollow veine.    
\P 1635 N. CARPENTER  \textit{Geog. Del.} ii. vi. 96 The streitnesse of the channell, wherein a great‥sea is to bee exonerated.    
\P 1672  \textit{Phil. Trans.} VII. 5009 The Lympha does wholly exonerate itself into the sub-clavial and jugular veins.

b. \P 1614 RALEIGH  \textit{Hist. World} i. viii. §4 These borderers‥might exonerate their swelling multitudes.    
\P 1657 M. HAWKE  \textit{Killing is M.} 23 Whereby such nefarious and facinerous persons may be exonerated.

\itembf{5.} To relieve from, of (anything burdensome, a duty, obligation, payment, task, etc.).

\P 1548 HALL  \textit{Chron.} 227 That he might‥exonerate them of the great charges, travayles \& labors, that they now were in.    c 
\P 1555 HARPSFIELD  \textit{Divorce Hen. VIII} (1878) 25 Would God Sir Thomas Moore‥had exonerated and discharged me of this my pains \& labour.    
\P 1692  \textit{Lond. Gaz.} No. 2786/3 To exonerate and discharge them from all Arrears of Heath-money.    
\P 1783 BURKE  \textit{Rep. Affairs India} Wks. 1842 II. 62 Mr.  Hastings‥offered to exonerate the company from that ‘charge’.    
\P 1835 I. TAYLOR  \textit{Spir. Despot.} ii. 75 A body of clergy exonerated of all solicitude.    
\P 1851  \textit{Ord. \& Regul. R. Engineers} ii. 2 Commanding Royal Engineers will not exonerate any Officers‥from the performance of such Duties.

\itembf{6.} To free from blame; to exculpate; also, to relieve from the blame or burden of; to relieve or set free from (blame, reproach).

\P 1575 CHURCHYARD  \textit{Chippes} (1817) 40 That lord Oxford might be induced‥to exonerate Churchyard.    
\P 1654 H. L'ESTRANGE  \textit{Chas. I} (1655) 21 Nothing would prevail, nor would the Duke be exonerated.    
\P 1678 R. BARCLAY  \textit{Apol. Quakers} v. §12. 136 Such a season‥sufficiently exonerateth God of every Man's Condemnation.    
\P 1824 W. IRVING  \textit{T. Trav.} I. 334 To exonerate myself of a greater crime.    
\P 1825 F. BURNEY  \textit{Diary} I. 561 To exonerate her from the banal reproach of yielding unresisting to her passions.    a 
\P 1848 R. W. HAMILTON  \textit{Rew. \& Punishm.} viii. 489 Do we seek to exonerate His justice‥by the denial of His faithfulness?    
\P 1884 PAE  \textit{Eustace} 187, I won't exonerate the Government.
\end{myenumerate}


%%%%%%%%%%%%%%%%%%%%%%%%%%%%%%%%
\myitem{exculpate} v.

\noindent \phonetic{(ˈɛkskəlpeɪt, ɛksˈkʌlpeɪt)}

\noindent [f. ex- prefix1 + L. culp-a blame + -ate3. Cf. It. scolpare, med.L. *exculpāre implied in exculpātio (Du Cange).]
\vspace{-0.3cm}

\begin{myenumerate}

\itembf{1.} trans. To free from blame; to declare free from guilt; to clear from an accusation or blame.

\P 1656-81 [see 1 b].    
\P 1721 in  BAILEY.    
\P 1758-9 LOWTH \textit{Life Wykeham} v. 156 Men who had been‥ punished in the parliament of 1376, and who had gotten themselves exculpated in the succeeding parliament.    
\P 1841 JAMES  \textit{Brigand} xx, She exculpates me from blame in this matter.    
\P 1850 GROTE  \textit{Greece} ii. lxii, The latter stood exculpated on both charges.

refl. \P 1748 RICHARDSON  \textit{Clarissa} (J.), A good child will not seek to exculpate herself at the expence of the most revered characters.    
\P 1809-10 COLERIDGE  \textit{Friend} (1865) 110 From this charge of inconsistency I shall best exculpate myself by the full statement of the third system.    
\P 1863 MRS. OLIPHANT  \textit{Salem} Ch. iv. 63 Poor Vincent made a hasty effort to exculpate himself from the soft impeachment.

\itembf{b.} intr. for refl. Obs. rare.

\P 1656-81 BLOUNT  \textit{Glossogr., Exculpate}, to cleer ones self of a fault.    
\P 1780 BURKE  \textit{Corr.} (1844) II. 315 To be over earnest in endeavours to exculpate, previous to accusation, would imply [etc.].    
\P 1783 \textit{Rep. Affairs India} Wks. XI. 326 Doubts whether the refusal to exculpate by oath can be used‥to infer any presumption of guilt.

\itembf{2.} Of things: \textbf{a.} To serve as an excuse for; to justify. Obs. rare. \textbf{b.} To furnish ground for exculpating. Const. from.

\P 1706 PHILLIPS  (ed. Kersey) s.v., Good meaning will never exculpate blind and Superstitious Devotion.    
\P 1783 BURKE  \textit{Rep. Affairs India} Wks. XI. 132 Evidence, which may tend to criminate, or exculpate, every person.    
\P 1875 FARRAR  \textit{Seekers} i. vi. 83 The tenor of his life has sufficient weight to exculpate him from an unsupported accusation.
\end{myenumerate}


%%%%%%%%%%%%%%%%%%%%%%%%%%%%%%%%
\myitem{remit} v.

\noindent \phonetic{(rɪˈmɪt)}

\noindent [ad. L. remitt-ĕre, f. re- re- + mittĕre to send; cf. admit, commit, etc. In Eng. use the secondary senses appear earlier and are more prominent than the primary: cf. remission.]
\vspace{-0.3cm}

\begin{myenumerate}

\itembf{I.} trans.

\itembf{1.} To forgive or pardon (a sin, offence, etc.).

\P 1375  \textit{Sc. Leg. Saints} vii. (James less) 209 Lord, remyt þis gilt þam to.    Ibid. xxx. (Theodora) 698 He hyr reconsalyt‥\& remyted hyre al hyr syne.    c 
\P 1440 GESTA  \textit{Rom.} lxxviii. 399 (Add. MS.), Afterwarde the kyng made men to seke the queen,‥and all that was done was remytte.    1503-4 Act 19 Hen. VII, c. 37 Preamble, It pleased your Highnesse‥to pardone remitte \& forgyve unto your seid Subgiect all the seid Mesprisions.    
\P 1535 COVERDALE  \textit{John} xx. 23 Whose synnes soeuer ye remytte they are remytted vnto them.    
\P 1608 HIERON  \textit{Wks.} I. 695 Bee pleased‥for His sake to remit my former vngratefulnesse.    
\P 1708 J. CHAMBERLAYNE  \textit{St. Gt. Brit.} i. iii. viii. 254 The English being easily to be reconciled, to pardon and remit Offences.    
\P 1823 SCOTT  \textit{Peveril} xl, Your Majesty was pleased to remit his more outrageous and insolent attempt upon your royal crown.    
\P 1884 A. R. PENNINGTON  \textit{Wiclif} ix. 297 It is impossible for the priest to remit the sins of any unless they are first remitted by Christ.

\itembf{b.} To spare, pardon, or forgive (a person).

\P 1526  \textit{Pilgr. Perf.} (W. de W. 1531) 78 He wolde not his prelate to shewe ony mercy on hym, nor to remyt or spare hym in ony thynge.    
\P 1549 COVERDALE, etc. \textit{Erasm. Par. John} 44 For God remitteth not hym that forgeueth not his brother.    
\P 1583 STUBBES  \textit{Anat. Abus.} ii. (1882) 13 Can man pardon or remit him whom God doth condemne?    
\P 1633 BP. HALL  \textit{Hard Texts, N.T.} 79 Bee comforted in God who hath remitted thee.

\itembf{2.} To give up, resign, surrender (a right or possession). Obs.

\P 1450 GODSTOW  \textit{Reg.} (E.E.T.S.) 42 Milo Basset remitted and furthermore quyte-claymed‥to the abbesse of Godestowe‥, all the right and clayme that he had.    1472-3 Rolls of Parlt. VI. 6/1 That it may please youre seid Highnes‥to remitte and release‥to us‥all youre right.    
\P 1588 SHAKES.  \textit{L.L.L.} v. ii. 459 Qu. Will you haue me, or your Pearle againe? Ber. Neither of either, I remit both twaine.    1647-8 Sir C. Cotterell Davila's Hist. Fr. (1678) 12 He was led‥to remit his whole authority into the hands of allies.    
\P 1654 tr.  \textit{Scudery's Curia Pol.} 96 If Queen Elizabeth had not believed‥she would not have‥remitted her Scepter to my hands.    
\P 1670 DRYDEN  \textit{Tyran. Love} iii. i, Th' Ægyptian Crown I to your hands remit.

\itembf{3.} To abstain from exacting (a payment or service of any kind); to allow to remain unpaid (or unperformed).

\P 1463  \textit{Rolls of Parlt.} V. 498/2 To pardon and remitte unto the seid Commons the seid vi M li.    
\P 1560 J. DAUS tr. \textit{Sleidane's Comm.} 60 It is reason that the lordes remit some part therof [sc. rent].    c 
\P 1645 HOWELL  \textit{Lett.} (1713) 16 All this his Majesty remitted, and only took the Principal.    a 
\P 1661 FULLER  \textit{Worthies} (1840) II. 508 The Queen‥rigorously demanded the present payment of some arrears which Sir Christopher did not hope to have remitted.    
\P 1701 W. WOTTON  \textit{Hist. Rome} vi. 109 She remitted the Arrears that were owing.    
\P 1783 BURKE  \textit{Rep. Aff. India} Wks. 1842 II. 18/1  They remit, by the like authority, the duties, to which all private trade is subject.    
\P 1817 JAS. MILL  \textit{Brit. India} I. iii. iv. 575 The rents of the husbandman, and other taxes, were remitted.    
\P 1863 FAWCETT  \textit{Pol. Econ.} iii. iii. 323 Let it be assumed that every farmer has the rent of his farm remitted for the next thirty years.

\itembf{b.} To refrain from inflicting (a punishment) or carrying out (a sentence); to withdraw, cancel; to grant remission of (suffering).

\P 1483  \textit{Rolls of Parlt.} VI. 250/2 Oure said soveraigne Lorde‥remitteth and woll forbere the greate punysshement of atteynder.    
\P 1553 T. WILSON  \textit{Rhet.} 15 b, The whole citie thought to remitte the necessitie of his punishment for the honour of his father.    
\P 1616 R. C. \textit{Times'  Whistle} iv. 1344 The  officer deputed for th' offence Will winck at smale faultes \& remit correction.    
\P 1693 LUTTRELL  \textit{Brief Rel.} (1857) III. 118 The queen remitted the quartering of his body.    
\P 1754 SHERLOCK  \textit{Disc.} I. i. 46 God may freely forgive the Sins of the World, and remit the Punishment.    
\P 1807 CRABBE  \textit{Hall of Just.} 3 Remit awhile the harsh command.    
\P 1841 JAMES  \textit{Brigand} xxxiii, We come to beseech you to remit the sentence of this unhappy young gentleman.    
\P 1857 BUCKLE  \textit{Civiliz.} I. xii. 673 The exile which followed the imprisonment seems to have been soon remitted.    
\P 1868 BROWNING  \textit{Ring \& Bk.} vi. 127 How does lenity to me Remit one death-bed pang to her?

\itembf{c.} To exempt from confiscation. rare—1.

\P 1741 MIDDLETON  \textit{Cicero} I. ii. 104 Verres for a valuable consideration sometimes remitted the ship.

\itembf{d.} To allow as a respite. rare—1.

\P 1813 BYRON  \textit{Corsair} ii. xiv, I will, at least, delay The sentence that remits thee scarce a day.

\itembf{4.} To discharge, set free, release, liberate (a person). Also const. of, to. Obs.

\P 1548 HALL  \textit{Chron., Hen. VIII} 169 b, Wee clerely remitted, and deliuered hym into his countrey.    
\P 1575 R. B. \textit{Appius  \& Virg.} D j b, If treason none by me be done, or any fault committed, Let my accusers beare the blame, and let me be remitted.    
\P 1634 GARRARD in  \textit{Strafford's Lett.} (1739) I. 373 Mr. Seldon is remitted of those Fetters that lay upon him.    
\P 1647 CLARENDON  \textit{Hist. Reb.} vi. §35 His Lordship was committed to the Tower‥; and though he was afterwards remitted to more Air, he continued a Prisoner to his death.

\itembf{II. 5.} To give up, lay aside (anger, displeasure, etc.) entirely or in part.

\P 1375  \textit{Sc. Leg. Saints} vii. (James less) 635 Þare-for his malancoly to þat man he remyttyte þare.    
\P 1393-4 \textit{Rolls of Parlt.} III. 314/1 Hit forthynketh me, and byseche yowe of your gode Lordship to remyt me your mautalent.    
\P 1413 PILGR.  \textit{Sowle} (Caxton 1483) i. xxvii. 31 This blessid lord Ihesu Crist remitted his rigour, descending downe to the erthe.    
\P 1560 J. DAUS tr. \textit{Sleidane's Comm.} 317 b, I beseche him to remit all displeasure.    
\P 1577 HANMER  \textit{Anc. Eccl. Hist.} (1619) 180 [He] would not thus much have remitted his tyranny, had he not been compelled.    
\P 1667 MILTON  \textit{P.L.} ii. 210 Our Supream Foe in time may much remit His anger.    
\P 1761 HUME  \textit{Hist. Eng.} I. App. ii. 258 That he would remit his displeasure.    
\P 1820 SHELLEY  \textit{Œd. Tyr.} ii. ii. 99 Remit, O Queen! thy accustomed rage!

\itembf{b.} To give up or give over, abandon, desist from (a pursuit, occupation, etc.).

\P 1587 R. HOVENDEN in  \textit{Collect.} (O.H.S.) I. 220 The Ladi Stafford was resolved to remyt hir suite.    
\P 1608 WILLET  \textit{Hexapla Exod.} 60 They‥caused them to remit their workes.    
\P 1687 LADY R. RUSSELL \textit{Lett.} I. li. 123 It seems I must remit seeing you, as you once kindly intended.    
\P 1726 POPE  \textit{Odyss.} xxiv. 286 Who digging round the plant still hangs his head, Nor ought remits the work.    
\P 1880 KINGLAKE  \textit{Crimea} VI. vi. 159 Engaged‥in a siege which they could not remit.

\itembf{6.} To allow (one's diligence, attention, etc.) to slacken or abate.

\P 1510 MORE  \textit{Picus} Wks. 15/1 Ye shall not think, that my trauaile and diligence in study is any thing remitted or slacked.    
\P 1590 MARLOWE  \textit{Edw. II}, ii. v, He that the care of his realm remits [etc.].    1742-3 Ld. Hervey in Johnson's Debates (1787) II. 409 To make the attainment of it more and more difficult, that they may insensibly remit their ardour.    
\P 1780 JOHNSON  \textit{Let. to Mr. Thrale} 30 May, Do not remit your care.    
\P 1803 M. EDGEWORTH  \textit{Manuf.} ii. (1832) 101, I have never remitted my attention to business.    
\P 1827 HALLAM  \textit{Const. Hist.} (1876) I. iii. 143 Nor did the voluntary exiles established in Flanders remit their diligence in filling the kingdom with emissaries.

\itembf{b.} To admit or manifest an abatement of some quality. ? Obs.

\P 1621 BURTON  \textit{Anat. Mel.} i. i. i. i, When he‥remembred that he was but a man, and remitted of his pride.    
\P 1628 HOBBES  \textit{Thucyd.} (1822) 8 To try if the Athenians‥would yet in some degree remit of their obstinacy.    
\P 1702  \textit{Eng. Theophrast.} 342 The strongest passions sometimes remit of their violence.    
\P 1775 S. J. PRATT  \textit{Liberal Opin.} v. (1783) I. 84 At the end of about two months, the severity of my fate began to remit of its rigour.

\itembf{c.} To mitigate, diminish, or abate. ? Obs.

\P 1615 G. SANDYS  \textit{Trav.} 39 Stiffe winter which no spring remits.    
\P 1656 RIDGLEY  \textit{Pract. Physick} 316 When the heat, pain, Feaver are remitted.    
\P 1658 ROWLAND tr.  \textit{Moufet's Theat. Ins.} 979 The light by little and little is remitted and slackned.    
\P 1750 JOHNSON  \textit{Rambler} No. 17 \phonetic{⁋}5 Every man has experienced how much of this ardour has been remitted, when a sharp‥sickness has set death before his eyes.

\itembf{7.} To relax, relieve from tension. Obs.

\P 1510 BARCLAY  \textit{Mirr. Gd. Manners} (1570) D j, Ceasse not, perseuer, knock \& stande, Remitte not thine armes by knocking fatigate.    
\P 1668 CULPEPPER \& COLE  \textit{Barthol. Anat.} ii. iii. 92 When the Breath is drawn in the Midriff is stretched, when it is blowne out, it is remitted or slackned.    a 
\P 1676 HALE  \textit{Prim. Orig. Man.} i. i. (1677) 29 'Tis by this‥the Lungs are intended or remitted.    
\P 1711 tr.  \textit{Werenfelsius' Logomachys, Disc. Meteors Stile} 192 Let the Judgement‥sometimes remit, and sometimes contract the Reins.

\itembf{III. 8.} To refer (a matter) for consideration, decision, performance, etc., to a person or body of persons, now usu. to one specially empowered or appointed to deal with it; also spec. in Law, to send back (a case) to an inferior court.

\P 1400 MANDEVILLE  (1839) xxxi. 315 Oure holy Fadir‥remytted my Boke to ben examyned and preved be the Avys of the seyd Conseille.    
\P 1455 PASTON  \textit{Lett.} I. 321 Wheche mater I remytte‥to youre ryght wyse discrecion.    
\P 1484 CAXTON  \textit{Fables of Alfonce} ix, They remytted the cause to be discuted or pleted before the Juge.    
\P 1523 FITZHERB.  \textit{Husb.} §7 The spirytuall constructyon of this texte, I remytte to the doctours of dyuynitie.    
\P 1586 T. B. \textit{La Primaud. Fr. Acad.} i. (1594) 514 Let them remit the judgement and deciding of their controversies to the arbitrement of some good men.    
\P 1654 tr.  \textit{Martini's Conq. China} 14 He remitted the business to the chief Governors and Commanders.    
\P 1762 FOOTE  \textit{Orators} i. Wks. 1799 I. 203  We shall‥remit the examination of the ignoble ones to the care of subaltern artists.    
\P 1863 P. BARRY  \textit{Dockyard Econ.} 59 The task and job question was remitted to the Commissioners on the Civil Affairs of the Navy.    
\P 1884  \textit{Law Times Rep.} L. 174/1 The defendants gave notice of their motion to set aside and remit the report [of the special referee].

absol. \P 1838 W. BELL  \textit{Dict. Law Scot.} 52 The circuit judge‥may recall the judgment appealed from, and remit to the inferior court with instructions.

\itembf{b.} To send (a person) from one tribunal to another for trial or hearing. rare.

\P 1538 STARKEY  \textit{England} ii. ii. 190 At London the jugys schold admyt non in sute, but such only as, for some resonabul cause, were remyttyd to them by the gentylmen of the scyre.    
\P 1740 HOWE in  \textit{Johnson's Debates} (1787) I. 31 If we remit this offender‥to any inferior court [etc.].

\itembf{c.} To commit (a person) to the charge or control of another. Also refl. Obs.

\P 1741 RICHARDSON  \textit{Pamela} (1883) I. 407 As he knew best what befitted his own rank and condition, I would wholly remit myself to his good pleasure.

\itembf{d.} refl. = REFER v. 5. Obs. rare—1.

\P 1674  \textit{Govt. Tongue} 18, I dare in this remit me to themselves, and challenge‥their natural ingenuity to say [etc.].

\itembf{9.} To refer (one) to a book, person, etc., for information on some point.

\P 1417 HEN. V in  \textit{Ellis Orig. Lett.} Ser. iii. I. 62 We remitte hem to have ful declaracion and verrai knaweleche of you in that matere.    c 
\P 1425 WYNTOUN  \textit{Cron.} ii. 1346 (WEMYSS  MS.), Gif ȝe of þat thing mare will wit, To Ovidis buke I ȝow remytt.    
\P 1533 MORE  \textit{Debell. Salem} Pref., Wks. 931/1 And some suche places yet as I had happed to finde, I haue remitted the reader vnto in myne apologye.    
\P 1590 SIR J. SMYTH  \textit{Disc. Weapons} 49 To the particularities whereof‥I remit those that are disposed to see and consider.    
\P 1650 FULLER  \textit{Pisgah} ii. iv. 113 Well might profane persons be remitted to this river, thereby to be instructed in the Sabbaths due observation.    
\P 1714 \textit{Ellwoods'  Autobiog.} Pref., Much of this being already done in the ensuing Pages, I chuse to remit the Reader thither.    
\P 1769 ROBERTSON  \textit{Chas. V}, vii. III. 16 The Emperor‥without deigning to answer a single word, remitted him to his ministers.    
\P 1835-8 S. R. MAITLAND \textit{Dark Ages} (1844) 156 Let us hear Du Cange, to whom Robertson remits us.

\P 1410 \textit{Master  of Game} (MS. Digby 182) x, Of þe remenaunt of his nature I remytte to Milbournn þe kynges Otyr hunter.    
\P 1523 FITZHERB.  \textit{Husb. Prol.}, I remytte [? to] that boke as myn auctour therof.

\itembf{b.} To direct (one) to a task. Obs. rare—1.

\P 1544 \textit{Supplic.  Hen. VIII} (1871) 51 Remyttynge byshops to attende their offyce and vocacyon by God‥appoynted.

\itembf{10. a.} To send (a person) back to prison, or to other custody; to recommit. Now rare.

\P 1414  \textit{Rolls of Parlt.} IV. 57/2 Whan I was remitted to the Prison of Flete.    
\P 1474  Ibid. VI. 103/1 The seid Chaunceller there remitted the seid Thomas Buysshop ageyn.    
\P 1653 LD. VAUX tr. \textit{Godeau's St. Paul} 300 The Captain‥remitted him, with the rest of his prisoners, into the hands of the Prefect of the Pretorium.    
\P 1700 DRYDEN  \textit{Sigism. \& Guisc.} 287 The prisoner was remitted to the guard.    
\P 1827 HALLAM  \textit{Const. Hist.} (1876) I. vii. 383 Whether such a return was sufficient in law to justify the court in remitting the parties to custody.

\itembf{b.} To send in return; to send back. Obs. rare.

\P 1461 PASTON  \textit{Lett.} II. 67 Remitte me summe letter, by the bringer her of, of all thes maters.    
\P 1660 F. BROOKE tr. \textit{Le Blanc's Trav.} 113 He gave them freedom, and remitted them ransomlesse, sent them all back again.

\itembf{c.} To emit or send out again. Obs. rare—1.

\P 1700 DRYDEN  \textit{Ovid's Met.} xv. 522 Whether Earth's an Animal, and Air Imbibes; her Lungs with coolness to repair, and what she sucks remits.

\itembf{11. a.} Law. To restore to a former and more valid title: see REMITTER 1. Obs.

\P 1544 tr.  \textit{Littleton's Tenures} 141 In so much the wyfe is in her remytter, he is remitted to his reuercion.    
\P 1632 \textit{Womens  Rights} xix. 156 The eldest daughter is remitted, that is remaunded and setled in the ancient estate.    
\P 1768 BLACKSTONE  \textit{Comm.} III. ii. 21 If the issue in tail be barred by the fine‥of his ancestor, and the freehold is afterwards cast upon him; he shall not be remitted to his estate tail.

\itembf{b.} To put back into, to admit or consign again to a previous position, state, or condition.

\P 1591 SPENSER \textit{M. Hubberd} 1254 He bad  the Lyon be remitted Into his seate.    
\P 1642 FULLER  \textit{Holy \& Prof. St.} ii. xxii. 142 Thus his indiscretion remitted him to the nature of an ordinary person.    
\P 1654 EARL OF MONMOUTH tr. \textit{Bentivoglio's Warrs Flanders} 186 It was a long while ere it [the city] could be remitted into its former condition.    
\P 1671 MILTON  \textit{Samson} 687 Nor only dost [thou] degrade them, or remit To life obscur'd which were a fair dismission.    
\P 1761  \textit{New Comp. Fest. \& Fasts} xxxvi. §2. 353 When death‥is making his near approach to‥ remit us to darkness and oblivion.    
\P 1863 BRIGHT  \textit{Sp., Amer.} 30 June (1876) 142 You propose to remit to slavery three millions of negroes.
 
\itembf{12.} To postpone, to put off or defer.

\P 1635 J. HAYWARD tr. \textit{Biondi's Banish'd Virg.} 166 Willingly would hee have knowne then presently the story‥but‥he remitted it till after supper.    
\P 1663 GERBIER  \textit{Counsel} 62 Remitting setting of walls untill the next Spring after.    
\P 1769 GOLDSM.  \textit{Hist. Rome} (1786) II. 25 The conspirators‥remitted the execution of their design to the ides of March.    
\P 1786 JEFFERSON  \textit{Writ.} (1859) I. 511 We remitted all further discussion till he should send me a copy of his letter.    
\P 1836 J. GILBERT  \textit{Chr. Atonem.} iii. (1852) 73 We must for the present remit our reply to that part of our subject.

\itembf{b.} To defer the reception of (a person). Obs.—1

\P 1663 H. COGAN tr. \textit{Pinto's Trav.} xliv. 175, I hold it fit to remit him unto some other time, when as he may be better acquainted.

\itembf{13.} To refer, assign, or make over to a thing or person.

\P 1641 \textit{Vind.  Smectymnuus} vi. 78 That which Hierome speakes in the present tense‥he would remit to time past.    
\P 1720 WATERLAND  \textit{Answ. Whitby's Reply} 58 You‥object farther‥that Christ would not suffer Himself to be called Good, but remitted that Title to the Father only.    
\P 1788 REID  \textit{Aristotle's Log.} iv. §6. 89 He thinks that the doctrine of modals ought to be banished out of logic and remitted to grammar.    
\P 1837 G. PHILLIPS  \textit{Syriac Gram.} 9 The vowel in such places is remitted to the preceding letter, if it has been previously without one.

\itembf{b.} To enter or insert in (or into) a book. Obs.

\P 1670 WOOD  \textit{Life} (O.H.S.) II. 204 This book he gave A. W. because he had, in his great reading, collected some old words for his use, which were remitted therein.    
\P 1716 M. DAVIES  \textit{Athen. Brit.} II. 219 Which Examinations‥were‥remitted by John Fox into his Book of Martyrs.

\itembf{14.} To send or transmit (money or articles of value) to a person or place.

\P 1640 HOWELL  \textit{Dodona's Gr.} 98 [He] makes one of her proudest Cities his Scale, for remitting his Moneyes to Leoncia.    
\P 1690 in  J. Mackenzie \textit{Siege London-Derry} 54/1 You are to receive and dispose of the Thousand pounds which shall be remitted to you, to the best advantage.    
\P 1758 JOHNSON  \textit{Idler} No. 62 \phonetic{⁋}4 We parted; and he remitted me a small annuity.    
\P 1787 JEFFERSON  \textit{Writ.} (1859) II. 149 This has prevented the treasury board from remitting any money to this place.    
\P 1840 MACAULAY  \textit{Ess., Clive} (1852) III. 61 He had recently remitted a great part of his fortune to Europe, through the Dutch East India Company.    
\P 1861 GOSCHEN  \textit{For. Exch.} 91 Was it probable‥that in a time of great national emergency the New York bankers would remit their capital for employment to Europe‥?

absol. \P 1682 [See  \textit{remitted}, below].    
\P 1705 ADDISON  \textit{Italy} 471 They oblig'd themselves to remit, after the rate of Twelve Hundred Thousand Pounds Sterling per Annum.    
\P 1809 BYRON  \textit{Let. to Mrs. Byron} 12 Nov., I expect Hanson to remit regularly.

\itembf{IV.} intr.

\itembf{15.} To abate, diminish, slacken.

\P 1629 \textit{Drayner  Conf.} (1647) C, The whole masse of waters continue upon the face of the Fenne till those windes remit.    
\P 1643 MILTON  \textit{Divorce} (1645) 39 The vigor of his Law could no more remit, then the hallowed fire on his altar could be let go out.    
\P 1695 WOODWARD  \textit{Nat. Hist. Earth} iv. 198 Till such time as its motion begins to remit and be less rapid.    
\P 1770 GOLDSM.  \textit{Des. Vill.} 16 How often have I blest the coming day, When toil remitting lent its turn to play.    
\P 1850 L. HUNT  \textit{Autobiog.} I. viii. 309 The fishermen's wives‥seemed equally determined not to let the intention remit.    
\P 1870 BRYANT  \textit{Iliad} II. xiii. 23 Meantime the valor of Idomeneus Remitted not.

\itembf{b.} of pain, fever, etc. Also in fig. context.

\P 1685 tr.  \textit{Willis' Lond. Pract. Physick} 533 If upon sore Lips the Fever does not remit, it will prove of long continuance and severe.    
\P 1737 WHISTON  \textit{Josephus, Antiq.} ii. iii. §4 Neither did his pains remit by length of time.    
\P 1747 tr.  \textit{Astruc's Fevers} 195 The fever thus treated, remits generally towards the sixth or seventh day.    
\P 1783 JOHNSON \textit{Let.} in  \textit{Boswell} 30 Sept., I have been‥much harassed with the gout; but that has now remitted.    
\P 1887  \textit{Pall Mall G.} 17 Feb. 13/2 The ‘Otello’ fever at Milan seems at last a little inclined to remit.

\itembf{16.} To relax from labour; to give over.

\P 1760-72 H. BROOKE  \textit{Fool of Qual.} (1809) I. 84 They remitted from their toil.    
\P 1841 EMERSON  \textit{Ess., Man the Reformer} Wks. (Bohn) II. 240 Their enemies will not remit; rust, mould, vermin‥all seize their own.

\noindent Hence \textbf{remitted} ppl. a.

\P 1682 J. SCARLETT  \textit{Exchanges} 65 Every Remitter that remits not directly, but designs to draw in the remitted Sum again [etc.].    a 
\P 1700 KEN  \textit{Hymnotheo Poet.} Wks. 1721 III.  130 The happy symptons of remitted sin.    
\P 1896 H. DE WINDT  \textit{New Siberia} iv. 59 There is also a graduated scale of what are called remitted sentences.    
\P 1897  \textit{Westm. Gaz.} 13 Apr. 2/1 But it is not merely in respect of these remitted actions that the County Courts have weighty and important functions.
\end{myenumerate}

%%%%%%%%%%%%%%%%%%%%%%%%%%%%%%%%
\myitem{eschew} v.1

\noindent \phonetic{(ɛsˈtʃuː)}

\noindent [a. OF. eschiver, eschever (also in other conjugations, as eschevoir, eschivir, eschivre), corresp. to Pr., Sp., Pg. esquivar, It. schivare (whence prob. mod.F. esquiver to dodge, the retention of the s being otherwise anomalous):—Common Romanic *skivāre, f. *skivo: see prec.; cf. OHG. sciuhen, MHG. schiuhen, schiuwen, mod.Ger. scheuen to dread, avoid, shun; also Eng. shy v.]
\vspace{-0.3cm}

\begin{myenumerate}

\itembf{1.} trans. To avoid, shun. \textbf{a.} To avoid, keep clear of, escape (a danger or inconvenience). Rarely with clause as obj.

\P 1375  \textit{Sc. Leg. Saints, Mathias} 205 [A sone] þat scho, til eschewe destiny, Ine a cophyne kest ine þe se.    c 
\P 1460 FORTESCUE  \textit{Abs. \& Lim. Mon.} (1714) 105 To eschewe thees two Harmes, hyt may than be advised, etc.    
\P 1514 BARCLAY  \textit{Cyt. \& Uplondyshm.} (Percy Soc.) 1 Pastoures‥drawe to cotes for to eschewe the colde.    
\P 1526 TINDALE  \textit{2 Cor.} viii. 20 Thus we eschue thatt eny man shulde rebuke us in this aboundance.    c 
\P 1530 LD. BERNERS  \textit{Arth. Lyt. Bryt.} (1814) 17 To exchewe therby the displeasure of my lorde.    
\P 1598 SHAKES.  \textit{Merry W.} v. v. 251 What cannot be eschew'd, must be embrac'd.    
\P 1671 J. WEBSTER  \textit{Metallogr.} iv. 61 To eschew tediosness, [I] shall transcribe what Dr. Jorden hath written.    
\P 1721 \textit{St. German's  Doctor \& Stud.} 60 To eschew that in$\sim$convenience that Statute was made.

\itembf{b.} To ‘fight shy of’, avoid (a place); to stand aloof from (a person). Obs.

\P 1377 LANGL.  \textit{P. Pl.} B. vi. 55 Suche men eschue.    
\P 1413 LYDG.  \textit{Pilgr. Sowle} iv. iii. (1483) 59 The quene of Saba‥eshewed it [that brydge] and took another wey.    c 
\P 1450 CASTLE  \textit{Hd. Life St. Cuthbert} (Surtees) 160 Fra þen forthe sho forhewed Þe kynges presence, and it eschewed.    
\P 1553 T. WILSON  \textit{Rhet.} 2 Beware‥of straunge woordes, as thou wouldest take hede and eschewe greate rockes in the sea.    
\P 1621 BURTON  \textit{Anat. Mel.} iii. ii. vi. iii. (1651) 564 A woman a man may eschue, but not a wife.

\itembf{c.} To abstain carefully from, avoid, shun (an action, a course of conduct, an indulgence, an article of food or drink, etc.). The current sense: Formerly with obj.-inf. preceded by to.

JOHNSON 1755 Notes  the word as ‘almost obsolete’; it is now not uncommon in literary use.

\P 1340-70  \textit{Alex. \& Dind.} 1001 But  al þat badde is for a burn here abouen erþe, Huo so haþ chaunce to echue \& chese the betture.    c 
\P 1375 \textit{Lay Folks Mass-bk.} (MS. B.) 358 Gyue me grace for to etchewe to do þat þing þat me shuld rewe.    
\P 1388 WYCLIF  \textit{2 Tim.} ii. 16 Eschewe thou vnhooli and veyn spechis.    c 
\P 1450 MYRC 28 Grete othes thow moste enchewe.    
\P 1509 HAWES  \textit{Joyful Medit.} 20 They may extue For to do wronge.    
\P 1535 COVERDALE  \textit{Ps.} xvii. 23, I‥will eschue myne owne wickednes.    
\P 1637 EARL STIRLING  \textit{Doomesday 9th Hour} (R.), These curious doubts which good men doe eschew Make many atheists.    
\P 1656 RIDGLEY  \textit{Pract. Physick} 22 Fat things must be eschewed.    a 
\P 1707 BEVERIDGE  \textit{Serm.} II. lxxxiii. (R.), They must not only eschew evil but do good in the world.    
\P 1801 WORDSW.  \textit{Cuckoo \& Night.} xxiii, For every wight eschews thy song to hear.    
\P 1848 THACKERAY  \textit{Van. Fair} xlv, He has already eschewed green coats, red neckcloths, and other worldly ornaments.    
\P 1855 MACAULAY  \textit{Hist. Eng.} IV. 693 Observers‥thought that capitalists would eschew all connection with what must necessarily be a losing concern.    
\P 1876 BLACKIE  \textit{Songs Relig. \& Life} 228 Eschew the cavilling critic's art, The lust of loud reproving.

absol. \P 1621 BURTON  \textit{Anat. Mel.} i. i. ii. viii. (1651) 25 The power to prosecute or eschue.

\itembf{2.} intr. To get off, escape. Obs.

\P 1375 BARBOUR  \textit{Bruce} xi. 391 Thai sall nocht weill eschew foroutyn fall.    c 
\P 1450 CASTLE  \textit{Hd. MS. Life St. Cuthb.} (Surtees) 2525 And þat he couet to eschew.    
\P 1560 ROLLAND  \textit{Crt. Venus} iv. 441 Grant him his life‥And I promit‥That he sall not eschew away, nor fle.

\itembf{3.} trans. To rescue. Obs. rare. [So Fr. eschiver.]

\P c1500 \textit{Melusine}  170 Þey recouered there six of theire galeyes, \& eschiewed þem fro the fyre.

\noindent Hence \textbf{eschewal}, an eschewing, a keeping clear of (evil). 
\textbf{eschewance}, the action of eschewing; avoidance. 
\textbf{eschewer}, one who eschews, avoids, shuns. 
\textbf{eschewing} vbl. n., the action of the vb. eschew in various senses. 
\textbf{eschewment}, the action of eschewing.

\P 1583 BABINGTON  \textit{Commandm.} vii. (1590) 278 Things which keepe chastitie vncorrupted‥sobrietie, labour‥\& *eschewall [ed. 1637 eschewing]  of oportunitie.    
\P 1656 JEANES  \textit{Mixt. Scho. Div.} 22 The bare eschewall of an evill is sufficient for the denomination of feare.    
\P 1841 G. S. FABER  \textit{Prov. Lett.} (1844) I. 182 The convenient negative process of an eschewal of all cross-questioning.

\P 1842 JAMES  \textit{Morley Ernstein} xv, With that careful *eschewance of all listening ears‥that gentleman remained bowing in silence till the waiter was out of the room.

\P 1578 CH. PRAYERS in  \textit{Priv. Prayers} (1851) 460 Give them such judges, as are‥*eschewers of all partiality.    
\P 1621 DK. BUCKHM. in  \textit{Life Bacon} xxii. (1861) 501 A messenger of good news to you and an eschewer of evil.    
\P 1825 COLERIDGE  \textit{Aids Refl.} (1848) I. 188 These eschewers of mystery.

\P 1374 CHAUCER  \textit{Boeth.} iii. xi. 99 The ferme stablenesse of pedurable dwellynge and ek the *eschuynge of destruccyoun.    
\P 1563 in  \textit{Vicary's Anat.} (1888) App. iii. 164 Theschuynge of the greate Daunger \& perill of the‥plage.

\P 1864 WEBSTER, *Eschewment (rare).
\end{myenumerate}

%%%%%%%%%%%%%%%%%%%%%%%%%%%%%%%%
\myitem{comport} v.

\noindent \phonetic{(kəmˈpɔət)}

\noindent [ad. L. comportā-re to carry together, and F. comport-er to endure, bear, suffer, conduct (oneself), behave: the L. f. com- + portāre to carry.]
\vspace{-0.3cm}

\begin{myenumerate}

\itembf{1.} trans. To bear, endure; to tolerate. Obs.

\P 1588 A. KING tr. \textit{Canisius' Catech.} 175 We that ar stark (sayes the apostle) man comport the imbecillitie of the waiker.    
\P 1597 DANIEL  \textit{Civ. Wares} i. lxx, The malecontented sort, That‥never can the present state comport.    a 
\P 1619 \textit{Coll. Hist. Eng.} (1626) 129 A Queene Dowager of England‥could not comport a superior so neare her doore.    
\P 1667 G. DIGBY  \textit{Elvira} ii. in Hazl. Dodsley XV. 25 How does that noble beauty‥Comport her servile metamorphosis?    
\P 1716 M. DAVIES  \textit{Athen. Brit.} iii. Pallas Angl. 31 Whose Necessities they are oftentimes as far from‥Bearing or Comporting.    
\P 1818 COLEBROOKE  \textit{Oblig. \& Contracts} I. 70 Words taken in a sense which they comport.

\itembf{b.} To bear, suffer, allow, permit that. Obs.

\P 1616 BRENT tr.  \textit{Sarpi's Hist. Council Trent} (1676) 662 The time did not comport that the course of divine matters‥should be hindred by humane contentions.    
\P 1646 F. HAWKINS  \textit{Youth's Behav.} iii. §2 (1663) 14 Amongst them the custome doth comport in certain places that they Thou one another more freely.

\itembf{2.} intr. to comport with: to bear with, put up with, tolerate, endure, suffer. Obs.

\P 1565 SIR W. CECIL in \textit{Ellis Orig. Lett.} ii. 172 II. 296 She‥prayeth hir Maty here to comport with hir untill she will send on of hirs hyther.    a 
\P 1661 FULLER  \textit{Worthies} ii. 9 Being unable to comport with his Oppression.    
\P 1679 in  \textit{Gutch Coll. Cur.} I. 274 If the University of Oxford‥were to comport with the privileges granted before to the King's Printers.    
\P 1697 R. PIERCE  \textit{Bath Mem.} i. xi. 242 She needed both drinking, bathing, and pumping, but had not Strength to comport with either.    
\P 1851 CARLYLE  \textit{Sterling} iii. v. (1872) 214 The family‥could at any rate comport with no long absence.

\itembf{b.} refl. in same sense. Obs. rare.

\P 1655 FULLER  \textit{Ch. Hist.} iii. i. §2 Many‥Bishops‥unable to Comport themselves with his harshness‥quitted their preferments.

\itembf{3.} refl. To conduct or behave oneself; to act in a particular manner, to behave. Also transf.

\P 1616 LANE  \textit{Sqr.'s Tale} xi. 53 How thwhole court of knightes gann them comport in glorious wellcoms.    
\P 1669 WOODHEAD  \textit{St. Teresa} ii. iii. 20 He comported himself with extraordinary courage.    
\P 1830 HERSCHEL  \textit{Stud. Nat. Phil.} 314 The heat which accompanies the sun's rays comports itself, in all respects, like light.    
\P 1858 J. MARTINEAU  \textit{Stud. Christianity} 221 It would be curious to know how the Christians comported themselves when the priest of the Sun became monarch of the world.

\itembf{4.} intr. (for refl.) To behave. Obs.

\P 1616 LANE  \textit{Sqr.'s Tale} xi. 233 Wheare they with goodliest complementes comported.    
\P 1663 R. HAWKINS  \textit{Youths Behav.} 100 Comport, to compose the gesture.    
\P 1673 \textit{Rules  of Civility} ix. 86 How we are to Comport in our Congratulations and Condolements with great Persons.    a 
\P 1734 NORTH  \textit{Lives} (1826) III. 371, I cannot say how he would have comported under it.

\itembf{b.} to comport with: to deal with, treat. Obs.

\P 1675 tr.  \textit{Machiavelli's Prince} xv. Wks. 219 In what manner a prince ought to comport with his subjects.    
\P 1689 \textit{Dial.  betw. Timothy \& Titus} 11 Now how do you Comport with it in your Practice?

\itembf{5.} intr. to comport with: to agree with, accord with; to suit, befit.

\P 1589 R. BRUCE  \textit{Exhort. 2 Tim.} ii. (Wodrow) 375 Sik a meaning as the words may bear, and as their signification may comport with.    
\P 1603 DANIEL  \textit{Def. Rhime} (1717) 31 A Tragedy would indeed best comport with a Blank Verse.    
\P 1685 EVELYN  \textit{Mrs. Godolphin}, How her detachment from Royall servitude would comport with her.    
\P 1734 WATTS  \textit{Reliq. Juv.} (1789) 214 They do all that nature and art can do to comport with his will.    
\P 1884 T. SPEEDY  \textit{Sport} xvi. 288 Such wholesale slaughter does not comport with our opinion as to what really constitutes sport.

\itembf{6.} trans. ? To befit, or ? to bear upon. Obs. rare.

\P 1604 DRAYTON  \textit{Moses} 1, What respects he the negociating Matters comporting emperie and state?

\itembf{7.} lit. To carry or bring together, collect. Obs. rare.

\P 1641 BP. R. MONTAGU  \textit{Acts \& Mon.} 40 The materialls were comported from the Gentiles.    a 
\P 1660 [see comportation].

\itembf{8.} to comport the pike: to carry it grasped near the middle and pressed to the right side of the body, with the point raised. Obs.

   See description and figure in Pistofilo, Oplomachia (1621), where this ‘modo’ is said to be new, and practised by some French captains, particularly those of the King's Guard; also in Alfieri La Picca (1641) 16 ‘Come porti la picca il capitano.’ (In neither of these is any particular name applied to this ‘modo’.) The mode of coming to the ‘comport’ is fully described in The Perfection of Military Discipline after Newest Methods (1690) p. 24.

\P 1635-43 W. BARRIFFE  \textit{Mil. Discip.} cxiii. (1661) 150 Comporting your Half-pikes martching, is to be understood, when you martch under Trees, or some such place where they cannot be ordered or advanced.    
\P 1634 PEACHAM  \textit{Compl. Gent.} (1661) 299 Postures for the Pike. (15) Shoulder. (16) Port your Pikes. (17) Comport your Pikes. (18) Order your Pikes.    
\P 1650 R. ELTON  \textit{Art Milit.} viii. (1668) 6 The comporting of the Pike is only useful to the souldier marching up a hill; for if then he should be shouldered, the butt-end of the Pike would always be touching of the ground.    
\P 1688 J. S. \textit{Art  of War} 7 Captains and Lieutenants are to carry their pikes comported.
\end{myenumerate}


%%%%%%%%%%%%%%%%%%%%%%%%%%%%%%%%
\myitem{proscribe} v.

\noindent \phonetic{(prəʊˈskraɪb)}

\noindent [ad. L. prōscrīb-ĕre to write in front of; to write before the world, publish by writing, offer in writing for sale, etc.; to ‘post’ a person as condemned to confiscation or outlawry, f. prō, pro-1 1 f + scrīb-ĕre to write.]
\vspace{-0.3cm}

\begin{myenumerate}

\itembf{I. 1.} trans. To write in front; to prefix in writing. Obs. rare.
   Perhaps a scribal error for prescribe: see pro-1 3.

\P 1432-50 tr.  \textit{Higden} (Rolls) I. 21 When the compilator [Ranulphus] spekethe, the letter shall be proscribede [L. præscribitur] in this forme folowenge [R].

\itembf{II. 2.} To write up or publish the name of (a person) as condemned to death and confiscation of property; to put out of the protection of the law, to outlaw; to banish, exile. Also fig.

\P 1560 J. DAUS tr.  \textit{Sleidane's Comm.} 33 b, He‥doth condemne, \& proscribe him as aucthor of Scismes.    
\P 1596 SPENSER  \textit{State Irel.} Wks. (Globe) 637/1 Ro. Vere, Earle of Oxford, was‥banished the realme and proscribed.    
\P 1678 R. L'ESTRANGE  \textit{Seneca's Mor.} (1776) 200 He that proscribes me today, shall himself be cast out tomorrow.    
\P 1840 THIRLWALL  \textit{Greece} VII. lvii. 226 He was himself outlawed and proscribed in the name of his sovereign.    
\P 1842 ALISON  \textit{Hist. Europe} X. lxxvii. 840 A declaration was‥signed by all the Powers, which‥proscribed Napoleon as a public enemy, with whom neither peace nor truce could be concluded.

\itembf{b.} To ostracize, to ‘send to Coventry’.

\P 1680 EARL ROSCOM. tr. \textit{Horace's Art Poet.} 31 Then Poetasters in their raging fits‥dreaded and proscrib'd by Men of sense.

\itembf{3.} To reject, condemn, denounce (a thing) as useless or dangerous; to prohibit, interdict; to proclaim (a district or practice); = PROCLAIM v. 2 e, f.

\P 1622 MABBE tr.  \textit{Aleman's Guzman d'Alf.} ii. 319 This Custome is that vncontrouled Lord, that prescribes, and proscribes Lawes at his pleasure.    
\P 1768 HUME  \textit{Ess. \& Treat.} (1777) II. Notes 507 They [plays] have been zealously proscribed by the godly in later ages.    
\P 1772 PRIESTLEY  \textit{Inst. Relig.} (1782) I. 219 The Stoics‥proscribed‥Compassion.    
\P 1774 GOLDSM.  \textit{Nat. Hist.} (1862) I. iv. iii. 424 Persons of taste or elegance seem to proscribe it [civet] even from the toilet.    
\P 1841 D'ISRAELI  \textit{Amen. Lit.} (1867) 342 The ecclesiastics in vain proscribed these licentious revelries.    
\P 1850 A. JAMESON  \textit{Leg. Monast. Ord.} (1863) 194 Before their religion was proscribed and their country confiscated.

\noindent ¶As a literalism of rendering in Rhemish N.T.

\P 1582 N.T.  (Rhem.) Gal. iii. 1 O sensles Galatians, who hath bewitched you, not to obey the truth, before whose eies Iesus Christ was proscribed [Gr. προεγραϕη; Vulg. præscriptus est; 1388 WYCLIF  exilid; Tindale, Coverd. described; 1611 Euidently set forth; 1881 R.V.  openly set forth], being crucified among you?

\noindent Hence \textbf{proscribed} ppl. a.

\P 1611 B. JONSON  \textit{Catiline} i. i, I hid for thee Thy murder of thy brother,‥And writ him in the list of my proscrib'd After thy fact, to save thy little shame.    
\P 1689 SHADWELL  \textit{Bury F.} 11, As the proscribed emperor was by his perfumes betrayd.    
\P 1868 J. H. BLUNT  \textit{Ref. Ch. Eng.} I. 66 A well$\sim$known favourer of the proscribed opinions.    
\P 1869 RAWLINSON  \textit{Anc. Hist.} 447 The property of the proscribed was confiscated.



\end{myenumerate}


%%%%%%%%%%%%%%%%%%%%%%%%%%%%%%%%
\myitem{sapid} a.

\noindent \phonetic{(ˈsæpɪd)}

\noindent [ad. L. sapid-us savoury, f. sapĕre (see sapient a.). Cf. F. sapide; the direct descendant is sade (obs.).]
\vspace{-0.3cm}

\begin{myenumerate}

\itembf{1.} Of food, etc.: Readily perceptible by the organs of taste, having a decided taste or flavour; esp. having a pleasant taste, savoury, palatable.

\P 1646 SIR T. BROWNE  \textit{Pseud. Ep.} iii. xxii. 165 Thus Camels to make the water sapide do raise the mud with their feet.    
\P 1656 BLOUNT  \textit{Glossogr.}, Sapid, well seasoned, savory, that hath a smack.    
\P 1761 ARMSTRONG  \textit{Day} 140 In salt itself the sapid savour fails.    
\P 1837 M. DONOVAN  \textit{Dom. Econ.} II. 103 It [venison] is certainly more sapid than any butchers' meat, and is even strong.    
\P 1898 P. MANSON  \textit{Trop. Diseases} xxi. 325 If the patient attempts to take any sapid food‥the pain and burning in the mouth are intolerable.

\itembf{2.} In neutral sense: Having the power of affecting the organs of taste; having taste or flavour.

\P 1634 T. JOHNSON  \textit{Parey's Chirurg.} xxvi. vii. 1034 Therefore  nature observes this order in the concoction of sapide bodies, that at the first the acerbe taste should take place, then the austere, and lastly, the acide.    
\P 1686 GOAD  \textit{Celest. Bodies} i. ix. 32 They are genericall Natures, common to all Sapid and Odorate Bodies.    
\P 1756 C. LUCAS  \textit{Ess. Waters} II. 95 Epsom water‥scentless, and hardly sapid.    
\P 1831 J. DAVIES  \textit{Manual Mat. Med.} 10 Those [salts] which are insoluble in water are insipid; such‥as are soluble in it, are more or less sapid.    
\P 1862 G. WILSON  \textit{Relig. Chem.} 5 Neither plants nor animals can exist‥in any of the odorous or sapid gases.

\itembf{3.} fig. Grateful to the mind or mental taste.

\P 1640 HOWELL  \textit{Dodona's Gr.} 217, I must confesse there may some few criticismes or graines of browne salt, and small dashes of vineger be found here and there, to make the discourse more sapid, but this tartnesse is farre from any gall or venome.    
\P 1649 JER. TAYLOR  \textit{Great Exemp.} i. Dis. iv. 125 The life of the spirit, is lessened and impaired according as the gusts of the flesh grow high and sapid.    a 
\P 1677 HALE  \textit{Prim. Orig. Man.} iv. viii. 373 These are things‥more grateful, sapid, and delightful to the Mind, than the best Apparatus or Provisions of a sensible Good.    
\P 1690 NORRIS  \textit{Refl. Cond. Hum. Life} (1691) 179 Such Books‥as are Sapid, Pathetic, and Divinely-relishing.    
\P 1864 CARLYLE  \textit{Fredk. Gt.} IV. 356 Pamphlets‥sapid, exhilarative.    
\P 1868  \textit{Sat. Rev.} 19 Dec. 794/2 Quite as important as the possession‥of all these faculties, is the temper, spirit, tone, or manner of their use, the something which makes them sapid.

\itembf{4.} absol. \textbf{a.} the sapid, that which is sapid, sapidity. \textbf{b.} quasi-n. A sapid substance.

\P 1715 \textit{Pancirollus'  Rerum Mem.} II. v. 299 Sugar‥seems to tame and to triumph over all Sapids.    
\P 1831 T. L. PEACOCK  \textit{Crotchet Castle} iv, I speak of the cruet sauces, where the quintessence of the sapid is condensed in a phial.



\end{myenumerate}


%%%%%%%%%%%%%%%%%%%%%%%%%%%%%%%%
\myitem{luscious} a.

\noindent \phonetic{(ˈlʌʃəs)}

\noindent [Of obscure origin.
   The form lucius, occurring in a MS. which elsewhere has licius in the same sense (see licious) suggests (as Prof. Skeat has remarked) that the word may be an aphetic form of delicious, with altered vowel. But phonetically this is unsatisfactory, and no better suggestion has been made.]
\vspace{-0.3cm}

\begin{myenumerate}
\itembf{1.} Of food, perfumes, etc.: Sweet and highly pleasant to the taste or smell.

\P 1420 \textit{Anturs  of Arth.} 458 (Irel. MS.) With lucius drinkes, and metis of the best.    
\P 1566 DRANT  \textit{Horace's Sat.} ii. iv. H, The stronge may eate good looshiouse meate.    
\P 1590 SHAKES.  \textit{Mids. N.} ii. i. 251, I know a banke‥Quite ouer-cannoped with luscious woodbine.    
\P 1604 \textit{Oth.} i. iii. 344 The Food that to him now is as lushious as Locusts, shalbe to him shortly, as bitter as Coloquintida.    
\P 1630 DRAYTON  \textit{Muses Elizium} (1892) 29 The lushyous smell of euery flower.    
\P 1655 FULLER  \textit{Waltham Abb.} 5 The grass‥is so sweet and lushious to Cattle, that they diet them.    a 
\P 1700 DRYDEN  \textit{Daphnis \& Chloris} Poems 1743 II. 40 Blown  roses hold their Sweetness to the last, And Raisins keep their luscious native taste.    
\P 1733 CHEYNE  \textit{Eng. Malady} ii. v. §5 (1734) 159 The Means us'd commonly in making it [food] more luscious and palatable.    
\P 1758 JOHNSON  \textit{Idler} No. 96 \phonetic{⁋}4 The most luscious fruits had been allowed to ripen and decay.    
\P 1840 BROWNING  \textit{Sordello} 634 Like the great palmer$\sim$worm that‥Eats the life out of every luscious plant.    
\P 1869 BROWNING  \textit{Ring \& Bk.} ix. 401 The luscious Lenten creature [sc. the eel].    
\P 1870 H. MACMILLAN  \textit{Bible Teach.} ix. 187 Its luscious clusters of golden or purple fruit.

quasi-adv. \P 1588 T. HARRIOT  \textit{Rep. Virginia} B 2 b, There are two kinds of grapes‥: the one is small and sowre‥: the other farre greater \& of himselfe lushious sweet.

fig. \P 1665 BOYLE  \textit{Occas. Refl.} v. iii. (1848) 305 The luscious sweets of sin.    a 
\P 1716 SOUTH  \textit{Serm.} (1823) IV. 309 May there not be‥something more glistering than a crown? and more luscious than revenge?    
\P 1848 KINGSLEY  \textit{Saint's Trag.} iii. ii. 250 Sinking down In luscious rest again.

\itembf{b.} transf. of a young person. Obs.

\P 1742 FIELDING  \textit{J. Andrews} i. vii, He‥really is‥a strong, healthy, luscious boy enough.

\itembf{2.} In bad sense: Sweet to excess, cloying, sickly.

\P 1530 PALSGR. 313/1 Fresshe or lussyouse as meate that is nat well seasoned, or that hath an unplesante swetnesse in it, fade.    
\P 1616 SURFL. \& MARKH.  \textit{Country Farm} 239 The smell of them [sc. other Lillies] is lussious, grosse, and vnwholesome.    
\P 1706 PHILLIPS  (ed. Kersey), Lushious, over$\sim$sweet, cloying.    
\P 1816 SCOTT  \textit{Old Mort. Conclus.}, The last cup‥is by no means improved by the luscious lump of half-dissolved sugar usually found at the bottom of it.    
\P 1830 M. DONOVAN  \textit{Dom. Econ.} I. 275 Without the addition of water‥the resulting wine will be luscious and heavy.    
\P 1877 ‘RITA’  \textit{Vivienne} iii. vi, And the luscious dreary odours of‥fading flowers and trodden fruits, were heavy in the air.

\itembf{3.} Of immaterial things, esp. of language or literary style: Sweet and highly pleasing to the eye, ear, or mind. Chiefly in unfavourable use, implying a kind of ‘sweetness’ not strictly in accordance with good taste.

\P 1651 FULLER  \textit{Abel Rediv.}, Berengarius (1867) I. 4 He often‥addulced his discourse with all luscious expressions unto him.    
\P 1653 A. WILSON  \textit{Jas. I}, Pref. 8 Lushious words, that give no good rellish to the sense.    
\P 1708 BURNET  \textit{Lett.} (ed. 3) 304 All those luscious Panegyricks of Mercenary Pens.    
\P 1738 BIRCH  \textit{App. Life Milton} I. 78 A luscious Style stuffed with gawdy Metaphors and Fancy.    
\P 1822 HAZLITT  \textit{Table-t.} Ser. ii. iii. (1869) 66 A stream of luscious panegyrics.    
\P 1840 KINGSLEY  \textit{Lett.} (1878) I. 50, I have shed strange tears at the sight of the most luscious and sunny prospects.    
\P 1902 \textit{Longm.  Mag.} Mar. 479 The Lotus Eaters‥is what may be called a luscious expansion of four or five lines of the Odyssey.

\itembf{b.} Of colouring, design, etc.

\P 1849 RUSKIN  \textit{Sev. Lamps} ii. §15. 42 The groups of children,‥luscious in colour and faint in light.    
\P Ibid. iv. §13. 105 This extraordinary piece of luscious ugliness [a festoon].

\itembf{4.} Of tales, conversation, writing, etc.: Gratifying to lascivious tastes, voluptuous, wanton. Rarely of a person: Lascivious. Obs.

\P 1613 OVERBURY  \textit{A Wife} (1638) 63 She leaves the neat youth, telling his lushious tales.    a 
\P 1694 TILLOTSON  \textit{Serm.} (1744) XI. ccviii. 4717 Those luscious doctrines of the Antinomians.    
\P 1702 POPE  \textit{Jan. \& May} 379 Cantharides,‥Whose use old Bards describe in luscious rhymes.    
\P 1748 RICHARDSON  \textit{Clarissa} (1768) VII. xliv. 123 Calista [in ‘The Fair Penitent’] is a desiring luscious wench.    
\P 1766 FORDYCE  \textit{Serm. Yng. Wom.} (1767) I. iv. 149 Their descriptions are often loose and luscious in a high degree.    
\P 1815 W. H. IRELAND  \textit{Scribbleomania} 143 Descriptions so luscious—such pictures of passion That prudes, ta'en with furor, to ruin might dash on.

\itembf{5.} absol. (with the).

\P 1708  \textit{Brit. Apollo} No. 78. 3/1 There's a Great deal of Wit, But the Devil a Bit Of the lushious, can I find In't.    
\P 1790 A. WILSON  \textit{Ep. to Mr. T  B  Poet.} Wks. (1846) 87 A poet, Whose mem'ry will live while the luscious can charm.
\end{myenumerate}


%%%%%%%%%%%%%%%%%%%%%%%%%%%%%%%%
\myitem{caustic} a. and n.

\noindent \phonetic{(ˈkɔːstɪk, ˈkɒstɪk)}

\noindent [ad. L. caustic-us a. Gr. καυστικός capable of burning, caustic, f. καυστ-ός burnt, burnable, f. και- (future καυσ-) to burn. Cf. F. caustique.]
\vspace{-0.3cm}

\begin{myenumerate}

\itembf{A.} adj.

\itembf{1. a.} Burning, corrosive, destructive of organic tissue.

\P 1555 EDEN  \textit{Decades} W. Ind. (Arb.) 229 Albeit the water of the sea haue a certeyne caustike qualitie ageynst poyson.    
\P 1563 T. GALE  \textit{Antidot.} i. vii. 5 Causticke medicynes which doe remoue, and take away fylthines in vlcers.    
\P 1605 TIMME  \textit{Quersit.} i. vi. 25 Causticke and burning simples.    
\P 1727 BRADLEY  \textit{Fam. Dict.} I. s.v. Gourdy legs, This Stone‥from its‥caustick or burning Quality, alone destroys Warts.    
\P 1863-72 WATTS  \textit{Chem. Dict.} I. 818 In the old language of surgery, caustics were divided into the actual, such as red-hot iron and moxa, and the potential, such as strong alkalis, acids, nitrate of silver.

\itembf{b.} caustic bougie: a bougie armed with a piece of caustic.

\P 1800  \textit{Med. Jrnl.} III. 480 Caustic bougies, applied to the urethra under pretence of removing strictures.    
\P 1805  Ibid. XIV. 474 The superiority of the caustic over the common bougie.

\itembf{c.} Chem. caustic alkali: a name given to the hydrates of potassium and sodium, called caustic potash (KHO) and caustic soda (NaHO) respectively; caustic volatile alkali or caustic ammonia, ammonia as a gas or in solution; caustic lime, quick lime (CaO).

\P 1774 GOLDSM.  \textit{Nat. Hist.} (1776) VIII. 143 These flies, thus dried‥yield a great deal of volatile caustic-salt.    
\P 1791 HAMILTON  \textit{Berthollet's Dyeing} I. i. i. v. 80 Caustic alkali tinges the infusion of galls of a dark red.    
\P 1811 A. T. THOMSON  \textit{Lond. Disp.} (1818) 564 Take‥water of caustic kali, nine fluid ounces.    
\P 1813 SIR H. DAVY  \textit{Agric. Chem.} (1814) 21 Lime applied in its Caustic state acquires its hardness and durability, by absorbing the aerial acid.    
\P 1845 TODD \& BOWMAN  \textit{Phys. Anat.} I. 102 Add solution of caustic ammonia.    
\P 1869 ROSCOE  \textit{Elem. Chem.} 200 Potassium hydroxide or Caustic potash‥is a white substance soluble in half its weight of water, and acts as a powerful cautery, destroying the skin.    
\P 1876 HARLEY  \textit{Mat. Med.} 147 Caustic Soda.

\itembf{d.} gen. Burning. (rare.)

\P 1863 \textit{Possibil.  Creation} 148 At the tops of mountains‥the sun's rays are capable of producing very caustic results.

\itembf{e.} caustic bush, plant, vine, Australian names for Sarcostemma australe, a plant poisonous to cattle and sheep; caustic creeper, weed, Australian names for Euphorbia drummondii, the milky juice of which is used by the natives as a remedy for various diseases, but which is poisonous to sheep.

\P 1887 BAILEY \& GORDON  \textit{Plants reputed Poisonous} 43 Sarcostemma Australe. Known as ‘Caustic plant’ or ‘Caustic vine’ in Queensland.    
\P Ibid. 79 Euphorbia Drummondii, Caustic Creeper.‥ This weed is unquestionably poisonous to sheep.    
\P 1889 J. H. MAIDEN  \textit{Useful Native Plants} 127 Euphorbia Drummondii.‥ Called ‘Caustic Creeper’ in Queensland. Called ‘Milk Plant’ and ‘Pox Plant’ about Bourke. This weed is unquestionably poisonous to sheep.    
\P 1922 \textit{Jrnl.  Proc. R. Soc. N.S.W.} LVI. 183 This plant [sc. Sarcostemma australe], which occurs in all the Australian States except Victoria and Tasmania, is known as ‘Caustic Vine’, or ‘Caustic Plant’.    
\P 1926 J. M. BLACK  \textit{Flora S. Austral.} iii. 463 S[arcostemma] australe, R. Br. Milk Bush; Tableland Caustic Bush.    
\P 1954 W. E. BLACKALL  \textit{W. Austral. Wildflowers} 263 E[uphorbia] Drummondii. Caustic-weed.

\itembf{2.} fig. That makes the mind to smart: said of language, wit, humour, and, by extension, of persons; sharp, bitter, cutting, biting, sarcastic.

\P 1771 SMOLLETT  \textit{Humph. Cl.} (L.) And mirth he has a particular knack in extracting from his guests, let their humour be never so caustic or refractory.    
\P 1818 SCOTT  \textit{Rob Roy} iv, His shrewd, caustic, and somewhat satirical remarks.    
\P 1842 MACAULAY  \textit{Fredk. Gt., Ess.} (1877) 677 Those who smarted under his caustic jokes.    
\P 1876 GEO. ELIOT  \textit{Dan. Der.} ii. xviii. 147 Well, ma, I think you are more caustic than Amy.

\itembf{3.} Math. Epithet of a curved surface formed by the ultimate intersection of luminous rays proceeding from a single point and reflected or refracted from a curved surface; also of the curve formed by a plane section of a caustic surface. A caustic by reflexion is called a catacaustic, that by refraction a diacaustic. So caustic line, surface.

   [So called because the intensity of the light, and consequently of the heat, is in general greater at a point on this surface than at neighbouring points not on it, and at special points may become sufficiently intense to initiate combustion in a body there placed. The focus of a concave mirror is the cusp of its caustic for incident parallel rays.]

\P 1727-51 CHAMBERS  \textit{Cycl.,} Caustic curve, in the higher geometry, a curve formed by the concourse or coincidence of the rays of light reflected or refracted from some other curve.    
\P 1869 TYNDALL  \textit{Notes on Light} §101 The interior surface of a common drinking-glass is a curved reflector. Let the glass be nearly filled with milk, and a lighted candle placed beside it, a caustic curve will be drawn on the surface of the milk.    Ibid. §166 Spherical lenses have their caustic curves and surfaces formed by the intersection of the refracted rays.

\itembf{B.} n.

\itembf{1. a.} Med. A substance which burns and destroys living tissue when brought in contact with it. common caustic or lunar caustic: nitrate of silver prepared in sticks for surgical use.

\P 1582 J. HESTER  \textit{Secr. Phiorav.} i. vii. 8 Costicke‥beeyng laid on the sore doeth mortefie it.    c 
\P 1600 B. JONSON  \textit{Elegy Lady Pawlet} (R.) Put Your hottest causticks to, burne, lance, or cut.    
\P 1722 DE FOE  \textit{Plague} (1884) 111 They burnt them with Causticks.    
\P 1771 SMOLLETT  \textit{Humph. Cl.} (L.) He applied caustic to the wart.    
\P 1800  \textit{Med. Jrnl.} III. 290 The application of lunar caustic to strictures.    
\P 1879 G. C. HARLAN  \textit{Eyesight} v. 52 Quick-lime acts as a powerful caustic.

\itembf{b.} fig.

\P 1635 AUSTIN  \textit{Medit.} 197 With his Causticks of Repentance, he charitably burnt out, and purged the corruptions of Mens consciences.    
\P 1817 SCOTT  \textit{Wav.} xx, Pride‥applies its caustic as an useful though severe remedy.    
\P 1832 L. HUNT  \textit{Bacchus in Tusc.} 221, I should like to see a snake‥fasten with all his teeth and caustic upon that sordid villain.

\itembf{2.} Math. = caustic curveor surface: cf. A. 3.

\P 1727-51 CHAMBERS  \textit{Cycl.} s.v., Every curve has its twofold caustic.    
\P 1743  \textit{Phil. Trans.} XLII. 343 In the next place, the Caustics, by Reflexion and Refraction, are determined.    
\P 1869 TYNDALL  \textit{Notes on Light} §100 When a large fraction of the spherical surface is employed as a mirror, the rays are not all collected to a point; their intersections‥form a luminous surface‥called a caustic (German, Brennfläche).
\end{myenumerate}


%%%%%%%%%%%%%%%%%%%%%%%%%%%%%%%%
\myitem{mordant} a.

\noindent \phonetic{(ˈmɔːdənt)}

\noindent [a. F. mordant, pres. pple. of mordre to bite:—popular L. *mordĕre (= classical L. mordēre); the form mordent is assimilated to the L. pple. mordentem.]
\vspace{-0.3cm}

Biting (in various senses).

\begin{myenumerate}
\itembf{1.} Of satiric utterances (hence also of speakers or writers): Caustic, incisive.

\P 1474 CAXTON  \textit{Chesse} ii. v. (1481) d viij b, They ben‥right mordent and bytyng detractours.    
\P 1858 ELLICOTT  \textit{Destiny Creature} (ed. 3) 22 A petty spirit of detraction, with unkindly words or mordant satire.    
\P 1881  \textit{Spectator} 19 Nov. 1454/1 Lord Salisbury was, as usual, very mordant in his tone towards Mr. Gladstone.    
\P 1903  \textit{Blackw. Mag.} July 12/2 He was endowed with a peculiarly mordant wit.

\itembf{2.} Corrosive. Now rare.

\P 1601 HOLLAND  \textit{Pliny} I. 506 Of those marles which are found to be fat, the white is chiefe; and thereof be many sorts. The most mordant and sharpest of them all, is [etc.].    
\P 1666 G. HARVEY  \textit{Morb. Angl.} v. 61 The consumption of the kidneys is to be imputed to‥mordant armoniack salt.

fig. \P 1870 BALDW.  \textit{Brown Eccl. Truth} 225 The mordant acid of what they were pleased to conceive of as pure reason.

\itembf{3.} That causes pain or smart; pungent; biting. Of pain: acute, burning.

\P 1845 SYD. SMITH  \textit{Recipe for Salad} 7 in Lady Holland Mem. (1855) I. 373 Of mordant mustard add a single spoon.    
\P 1876 G. MEREDITH  \textit{Beauch. Career} III. xii. 218 With a shadow of an elevation of her shoulders as if in apprehension of mordant pain.

\itembf{4. a.} Having the property of fixing colouring matter or gold-leaf (see MORDANT n. 3, 3b).

\P 1825 J. NICHOLSON  \textit{Operat. Mechanic} 748 Mordant Varnish for Gilding.    
\P 1836 PENNY  \textit{Cycl.} VI. 156/1 [Calico-printing.] Mordant reserves, which form the lapis lazuli style.    
\P 1847-64 in  \textit{Webster.}

\itembf{b.} Of a dye: becoming fixed on the fibre as a result of forming an insoluble compound with a mordant.

\P 1902  \textit{Encycl. Brit.} XXVII. 559/2 Employed by themselves, Mordant Colours are usually of little or no value as dyestuffs, because‥either they are not attracted by the fibre‥or they only yield a more or less fugitive stain. Their importance and value as dyestuffs are due to the fact that they act like weak acids and have the property of combining with metallic oxides to form insoluble compounds termed ‘lakes’, which vary in colour according to the metallic oxide or salt employed.    
\P 1917 FORT \& LLOYD  \textit{Chem. Dyestuffs} xiii. 112 Acid mordant dyes may be first dyed on wool like acid dyes and then after-chromed.    
\P 1940 \textit{Thorpe's  Dict. Appl. Chem.} (ed. 4) IV. 127/2 Mordant dyes rank amongst the oldest dyes used by mankind for colouring purposes.    
\P 1963 [See  AFTER-CHROME a.].    
\P 1965 E. GURR  \textit{Rational Use of Dyes in Biol.} i. 115 Since sun yellow does not contain a hydroxyl group it cannot be classified as a mordant dye.

\itembf{5.} In literal sense: Given to biting. rare.

\P 1891 BAX  \textit{Outlooks New Standp.} iii. 174 Those who would take steps to restrain the mordant liberty of the cur, since they do not hold the doctrine of the divine right of dogs to bite.    
\P 1895 \textit{Pop.  Sci. Monthly} Sept. 652 The boy C  was for some time vigorously mordant in his angry fits.



\end{myenumerate}


%%%%%%%%%%%%%%%%%%%%%%%%%%%%%%%%
\myitem{morose} a.1

\noindent \phonetic{(mɒˈrəʊs)}

\noindent [ad. L. mōrōs-us peevish, fretful, wayward, fastidious, scrupulous (transf. of things, hard to manage), f. mōr-, mōs manner: see moral a. and -ose.]
\vspace{-0.3cm}

\begin{myenumerate}

\itembf{1.} Of persons, their attributes and actions: Sour-tempered, sullen, gloomy, and unsocial.

\P 1565 COOPER  \textit{Thesaurus, Morosus}, waywarde: frowarde: overthwarte: morose: diuers in condition: harde to please.
\P [1609 B. JONSON  \textit{Sil. Wom. Dram. Pers.} (1620), Morose, a Gentleman that loues no noyse.]    
\P 1620 VENNER  \textit{Via Recta} viii. 166 Neither‥am I against sauces so morose as that I doe altogether deny them.    
\P 1647 CLARENDON  \textit{Hist. Reb.} i. §185 He was a man of very morose manners, and a very sowr aspect.    
\P 1694 F. BRAGGE  \textit{Disc. Parables} xiv. 458 They were‥of very morose countenances, as greatly mortified, and strangers to the world.    a 
\P 1770 JORTIN  \textit{Serm.} (1771) VI. i. 18 A man should not give way to a morose, captious and cavilling humour and be eager to find fault.    
\P 1775 MASON  \textit{Mem. Gray Poems} 119 He was also morose, unsocial, and obstinate.    
\P 1815 J. SMITH  \textit{Panorama Sci. \& Art} I. 242 There are very few so obstinately morose, as to be uninfluenced by the opinions of others.    
\P 1853 C. BRONTË  \textit{Villette} xi, She looked stony and stern, almost mortified and morose.    
\P 1849 MACAULAY  \textit{Hist. Eng.} i. I. 3 No man who is correctly informed as to the past will be disposed to take a morose or desponding view of the present.    
\P 1907  \textit{Spectator} 5 Jan. 9/2 That great morose genius [sc. Swift].

absol. \P 1620 T. GRANGER  \textit{Div. Logike} 275 This to delight, to moue, and to allure with wiles, euen the refractory, and morose.    
\P 1762 GOLDSM.  \textit{Nash} 40 Let the morose and grave censure an attention to forms and ceremonies.

\itembf{b.} of opinions, principles, etc.

\P 1791 MAXWELL in  \textit{Boswell Johnson} an. 1770, His philosophy‥was by no means morose and cynical.    
\P 1838 LYTTON  \textit{Alice} ii. iv, Morbid and morose philosophy, begot by a proud spirit on a lonely heart.    
\P 1861 J. A. ALEXANDER  \textit{Gospel of Christ} xiv. 194 Pleasures which a more morose religion would proscribe as dangerous.

\itembf{c.} transf.

\P 1658 FRANCK  \textit{North. Mem.} (1821) 311 The carp is a fish complicated of a moross mixture, and a torpid motion.    
\P 1902 A. LANG  \textit{Hist. Scot.} II. v. 104 Mary's arrival was darkened by the morose climate.

\itembf{2.} Scrupulous, painstaking. Obs.

\P 1696 BENTLEY  \textit{Serm.} ix. (1724) 354 Unworthy of the most cautious and morose searcher of truth.    
\P 1695 J. EDWARDS  \textit{Perfect. Script.} 482 He was a very morose interpreter.

\itembf{3.} Of a thing: Hard to manage. Obs.

\P 1652 L. S. \textit{People's  Liberty} xxii. 53 This knot is somewhat morose, and will not easily be untied.

\itembf{4.} Comb., as morose-looking, morose-natured.

\P 1845 JAMES  \textit{Arrah Neil} ii, The elder of the two was a hard-featured somewhat morose-looking personage.    
\P 1884 J. PAYN  \textit{Lit. Recollect.} 62 A morose-natured man.
\end{myenumerate}


%%%%%%%%%%%%%%%%%%%%%%%%%%%%%%%%
\myitem{fastidious} a.

\noindent \phonetic{(fæˈstɪdɪəs)}

\noindent [ad. L. fastīdiōs-us, f. fastīdium loathing: see -ous. Cf. Fr. fastidieux.]
\vspace{-0.3cm}

\begin{myenumerate}

\itembf{1.} That creates disgust; disagreeable, distasteful, unpleasant, wearisome. Obs.

\P 1531 ELYOT  \textit{Gov.} i. ix, That thinge for the whiche children be often tymes beaten is to them‥fastidious.    
\P 1582 J. HESTER  \textit{Secr. Phiorav.} ii. xxiii. 102 A fastidious Ulcer.    
\P 1630 \textit{R. Johnson's  Kingd. \& Commw.} 193 A fastidious and irksome companion.    a 
\P 1677 BARROW  \textit{Serm. Wisdom in Beauties of B.} (1846) 9 Folly is‥fastidious to society.    a 
\P 1734 NORTH  \textit{Lives} II. 399 His partner, whose usage was‥fastidious to him.

\itembf{2. a.}  That feels or is full of disgust; disgusted.

\P 1534 MORE  \textit{On the Passion} Wks. 1312/1 Hee hadde of theym so muche, that he was full thereof, fastidious and wery.    
\P 1678 CUDWORTH  \textit{Intell. Syst.} 81 All desire of Change and Novelty, argues a Fastidious Satiety.

\itembf{b.} Full of pride; disdainful; scornful. Obs.

\P 1440 \textit{Foundation  Barts Hosp.} (E.E.T.S.) 15 A lamentable querell, expressynge‥whate fastidious owtbrekyngys hadde temptid hym.    1623-6 Cockeram, Fastidious, disdainfull, proud.    
\P 1634 SIR T. HERBERT  \textit{Trav.} (1638) 189 Regardlesse of the rodomantadoes of the fastidious Pagan.    
\P 1631 B. JONSON  \textit{New Inn, Ode} 7 Their fastidious vaine Commission of the braine.    
\P 1744 YOUNG  \textit{Night Thoughts} vi. 551 Proud youth! fastidious of the lower world.    
\P 1791 BOSWELL  \textit{Johnson} (1816) II. 277 (an. 1773) We see the Rambler with fastidious smile Mark the lone tree.    
\P 1796 C. MARSHALL  \textit{Garden.} xxii. (1813) 447 Those who have much practical skill‥slight what is written upon subjects of their profession, which is a fastidious temper.

\itembf{c.} transf. Of things: ‘Proud’, magnificent.

\P 1638 SIR T. HERBERT  \textit{Trav.} 62 One of them [Courts] fastidious in foure hundred porphirian pillars.    
\P Ibid. 102 Temples of Idolatry‥once lofty in fastidious Turrets.

\itembf{3.} Easily disgusted, squeamish, over-nice; difficult to please with regard to matters of taste or propriety.

\P 1647 WARD  \textit{Simp. Cobler} 77, I hold him prudent, that in these fastidious times, will helpe disedged appetites with convenient condiments.    
\P 1691 RAY  \textit{Creation} Pref. (1704) 7 Fastidious Readers.    
\P 1784 COWPER  \textit{Task} i. 513 The weary sight, Too well acquainted with their smiles, slides off Fastidious.    
\P 1848 MACAULAY  \textit{Hist. Eng.} II. 266 People whom the habit of seeing magnificent buildings‥had made fastidious.    
\P 1853 TRENCH  \textit{Proverbs} 3 A fastidious age‥and one of false refinement.    
\P 1865 LIVINGSTONE  \textit{Zambesi} xvii. 342 Though being far from fastidious, refused to eat it.    
\P 1877 BLACK  \textit{Green Past.} xlii. (1878) 338 The society‥was not at all fastidious in its language.
\end{myenumerate}


%%%%%%%%%%%%%%%%%%%%%%%%%%%%%%%%
\myitem{peevish} a.

\noindent \phonetic{(ˈpiːvɪʃ)}

\noindent [First evidenced in end of 14th c., but rare before 1500. Derivation unknown. The exact sense of the adj. in many of the early quots. is difficult to fix, and the following treatment is in many respects only provisional.

   None of the etymological conjectures hitherto offered are compatible with the sense-history.]
\vspace{-0.3cm}

\begin{myenumerate}
\itembf{1.} Silly, senseless, foolish. Obs.

\P 1393 LANGL.  \textit{P. Pl.} C. ix. 151 And bad hym ‘go pisse with hus plouh, peyuesshe shrewe!’ [A. vii. 143 pillede screwe; B. vi. 157 for-pyned schrewe].    
\P 1519 W. HORMAN  \textit{Vulg.} 21 b, Some make serche and dyuynacion by water, some by basyns,‥some by coniuryng of a soule, and suche other: and al be acurst or pyuysshe [partim execrabilia, partim mera ludibria].    
\P 1529 MORE  \textit{Dyaloge} iv. Wks. 271/1 The piuishe pleasure of the vayne prayse puffed oute of poore mortall mens mouthes.    
\P 1542 UDALL  \textit{Erasm. Apoph.} 94 b, To laugh such a peuishe trifleyng argument to skorne.    
\P 1565 JEWEL  \textit{Def. Apol.} (1567) 669 That whole tale‥is nothing els, but a peeuishe fable.    c 
\P 1586 C'TESS  PEMBROKE \textit{Ps.} xlix. v, These, whose race approves their peeuish waie [1611 This  their way is their folly].    
\P 1633 FORD  \textit{'Tis Pity} v. iii, This is your peevish chattering, weak old man!    
\P 1676 \textit{Doctrine  of Devils} 56 Christ did his Miracles among a peevish, foolish, sottish people, (as the World accounted them).

\itembf{b.} Beside oneself; out of one's senses; mad.

\P 1523 SKELTON  \textit{Garl. Laurel} 266 Some tremblid, some girnid, some gaspid, some gasid, As people halfe peuysshe, or men that were masyd.    
\P 1548 UDALL, etc. \textit{Erasm. Par. Acts} xii. 15 [They] aunswered to the mayden, Surely thou arte peuyshe.    
\P 1578 LYTE  \textit{Dodoens} iii. lxxvii. 426 Suche as by taking of poyson, are become peeuishe or without vnderstanding.    
\P 1591 LYLY  \textit{Endym.} i. i, There was neuer any so peeuish to imagin the Moone eyther capable of affection, or shape of a Mistris.

\itembf{2.} Spiteful, malignant, mischievous, harmful.

\P 1468 [IMPLIED in  PEEVISHNESS 2].     
\P ?a1500 CHESTER  \textit{Pl.} viii. 317 Alas! what presumption shold move that peeuish page, or any eluish gedling to take from me my crowne?    
\P 1513 DOUGLAS  \textit{Æneis} xi. xiv. 111 This ilk Aruns‥thys pewech man of weir‥schuke in hand hys oneschewabill speir.    
\P 1567 HARMAN  \textit{Caveat Ep.} Ded. 2 b, Their peuish peltinge and pickinge practyses.    
\P 1568 GRAFTON  \textit{Chron.} II. 176 In derision of the king, they made certaine peeuishe and mocking rymes which I passe ouer.    
\P 1570 LEVINS  \textit{Manip.} 145/42 Peuish, prauus.    
\P 1601 ?  MARSTON \textit{Pasquil \& Kath.} ii. 245 This crosse, this peeuish hap, Strikes dead my spirits like a thunder-clap.

\itembf{b.} In mod. dial. Of the wind: Piercing, ‘shrewd’.

\P 1828 CRAVEN  \textit{Gloss.} (ed. 2), Peevish, piercing, very cold; a peevish wind.    
\P 1863 MRS. TOOGOOD  \textit{Yorksh. Dial.}, The wind is very peevish to night.

\itembf{3.} An epithet of dislike, hostility, disparagement, contempt, execration, etc., expressing the speaker's feeling rather than any quality of the object referred to. Obs. Cf. mod. plaguy, wretched, etc.

\P 1513 DOUGLAS  \textit{Æneis} xi. viii. 78 For thou sal neuer los‥Be my wappin nor this richt hand of myne, Sik ane pevyche and cative saule as thyne [Nunquam animam talem dextra hac‥amittes].    
\P 1523 LD. BERNERS  \textit{Froiss.} I. ccclxi. 587 Sirs, howe is it thus‥that this peuysshe douehouse holdeth agaynst vs so longe?    
\P 1534 MORE  \textit{Comf. agst. Trib.} Wks. 1185 The  wolf‥spyed a fayre cowe in a close.‥ as for yonder peeuish cowe semeth vnto me in my conscience worth not half a grot.    a 
\P 1548 HALL  \textit{Chron., Hen. VI} 115 Such‥craftie imageners, as this peuishe painted Puzel was.

\itembf{4.} Perverse, refractory, froward; headstrong, obstinate; self-willed, skittish, capricious, coy. Obs.

\P 1539 CRANMER  \textit{Great Bible} Pref., Not onely foolyshe frowarde and obstinate but also peuysshe, peruerse and indurate.    a 
\P 1553 UDALL  \textit{Roysteri D.} ad fin., These women be all suche madde pieuish elues, They wyll not be woonne except it please them selues.    
\P 1589 NASHE  \textit{Anat. Absurd.} 39 Nothing is so great an enemie to a sounde iudgment, as the pride of a peeuish conceit.    
\P 1591 SHAKES.  \textit{Two Gent.} v. ii. 49 This it is to be a peeuish girle, That flies her fortune when it followes her.    
\P 1621 BP. R. MONTAGU  \textit{Diatribæ} 515 Diana, evermore a peevish angry goddesse.    
\P 1623 WEBSTER  \textit{Duchess of Malfi} iii. ii, We read how Daphne, for her peevish flight, Became a fruitless bay-tree.    a 
\P 1655 VINES  \textit{Lords Supp.} (1677) 269 It would be unnatural and pievish in a child to forsake his mother.    
\P 1671 H. FOULIS  \textit{Hist. Rom. Treas.} (1681) 23 Birds were not so shie and peevish formerly.

\itembf{5.} Morose, querulous, irritable, ill-tempered, childishly fretful. \textbf{a.} Of persons.

   In early quots. often referred to as the result of religious austerities, fasting, and the like.

\P 1530 \textit{Hickscorner}  D iij, And I sholde do after youre schole, To lerne to patter to make me peuysshe.    
\P 1596 SHAKES.  \textit{Merch. V.} i. i. 86 Why should a man whose bloud is warme within, Sit like his Grandsire, cut in Alabaster?‥and creep into the laundies By being peeuish?    
\P 1653 JER. TAYLOR  \textit{Serm. for Year} xxxix, Some men fast to mortifie their lust: and their fasting makes them peevish.    
\P 1708 SWIFT  \textit{Abolit. Chr.}, Excellent materials to keep children quiet when they grow peevish.    
\P 1742 YOUNG  \textit{Nt. Th.} ii. 175 Body and soul, like peevish man and wife, United jar, and yet are loth to part.    
\P 1862 SIR B. BRODIE  \textit{Psychol. Inq.} II. iii. 77 One whose state of health renders him fretful and peevish in his own family.

\itembf{b.} Of personal qualities, actions, etc.: Characterized by or exhibiting petty vexation.

\P 1577 FULKE  \textit{Answ. True Christian} 89 Without any contention of peuishe enuie.    
\P 1650 FULLER  \textit{Pisgah} iv. iii. 57 Gods providence on purpose permitted Moses to fall into this peevish passion [at Kadesh].    
\P 1711 STEELE  \textit{Spect.} No. 107 \phonetic{⁋}1 Unapt to vent peevish Expressions.    
\P 1822 HAZLITT  \textit{Table-t.} II. iv. 73 With a peevish whine in his voice like a beaten school-boy.

\itembf{c.} Const. to, with. Obs. rare.

\P 1655 in  \textit{Nicholas Papers} (Camden) III. 128 He is uery peuish to Mr. Ouerton and will tell him uery litle.    
\P 1697 FLOYER  \textit{Cold Baths} i. iii. (1700) 61 The People grew peevish with all Ancient Ceremonies.

\itembf{6.} See quot. (Perhaps some error.)

\P 1674 RAY  \textit{N.C. Words}, Peevish, witty, subtill.

\itembf{7.} in advb. constr. = peevishly.

\P 1529 SKELTON  \textit{El. Rummyng} 589 She was not halfe so wyse As she was peuysshe nyse [= foolishly particular].    [
\P 1594 SHAKES.  \textit{Rich. III,} iv. iv. 417 (Qo. 1, 1597) Be not pieuish, fond in great designes. Qo. 2 peeuish, fond; Qos. 3-8 peeuish fond; Folios peeuish found; Malone conjectured peevish-fond, the reading adopted in mod. edd.]
\end{myenumerate}


%%%%%%%%%%%%%%%%%%%%%%%%%%%%%%%%
\myitem{scruple} n.2

\noindent \phonetic{(ˈskruːp(ə)l)}

\noindent [ad. F. scrupule (14th c.), ad. L. scrūpulus, lit. a pebble (recorded only in late L.), fig. a cause of uneasiness, scruple, dim. of scrūpus rough or hard pebble, used fig. by Cicero for a cause of uneasiness or anxiety.

   Cf. F. scrupule (14th c.), Sp. escrúpulo, Pg. escrupulo, It. scrupolo, G. skrupel.]
\vspace{-0.3cm}

\begin{myenumerate}
\itembf{1.} A thought or circumstance that troubles the mind or conscience; a doubt, uncertainty or hesitation in regard to right and wrong, duty, propriety, etc.; esp. one which is regarded as over-refined or over-nice, or which causes a person to hesitate where others would be bolder to act. Often, scruple of conscience.

\P 1526  \textit{Pilgr. Perf.} (W. de W. 1531) 63 b, He wyll‥lette the‥symple persone from the performynge of his dutyes‥, by the reason of‥feares and scruples.    c 
\P 1534 MORE  \textit{Wks.} 1435/1 Though men‥say it is no consience but a foolish scruple.    a 
\P 1548 HALL  \textit{Chron., Hen. VIII} 179 The kyng of England‥was in a great scruple of his conscience and not quiet in his mynde.    
\P 1602 SHAKES.  \textit{Ham.} iv. iv. 40 (2nd Qo.) Some crauen scruple Of thinking too precisely on th'euent.    
\P 1660 JER. TAYLOR  \textit{Duct. Dubit.} i. vi. Rule 1, A Scruple is a great trouble of mind proceeding from a little motive.    
\P 1692 R. L'ESTRANGE  \textit{Fables} xli. 43 Upon the nicest Scruples of Honour.    
\P 1759 FRANKLIN \textit{Ess.} Wks. 1840 III.  389 The assembly did not, however, start any scruple on this head.    
\P 1788 GIBBON  \textit{Decl. \& F.} xlix. V. 90 The scruples of reason, or piety, were silenced by the strong evidence of visions and miracles.    
\P 1854 FABER \textit{Growth in  Holiness} xvii. (1872) 317 A scruple is‥a vain fear of sin where there is no reasonable ground for suspecting sin.    
\P 1868 E. EDWARDS  \textit{Ralegh} I. ii. 34 They had to deal with enemies who were troubled with few scruples.

\itembf{b.} in generalized sense. (Sometimes = scrupulosity.)

\P 1660 JER. TAYLOR  \textit{Duct. Dubit.} i. vi. Rule 2 §1 This is a right course in the matter of scruple; proceed to action.    
\P 1689 EVELYN  \textit{Diary} 21 Feb., The Abp. of Canterbury and some of the rest, on scruple of conscience‥enter'd their Protests and hung off.    
\P 1788 GIBBON  \textit{Decl. \& F.} xlix. V. 90 At first, the experiment was made with caution and scruple.    
\P 1848 BARONESS BUNSEN in  \textit{Hare Life} (1879) II. iii. 114 He expresses much concern and scruple about the trouble he occasions.    
\P 1872 BLACKMORE  \textit{Maid of Sker} vi, Just as I had made up my mind to lift up the latch, and to walk in freely, as I would have done in most other houses, but stood on scruple with Evan Thomas.

\itembf{c.} Phr. \textbf{without scruple}.

\P 1526 TINDALE  \textit{Acts} x. 29 Therfore cam I unto you with outen scruple [orig. ἀναντιρρήτως].    
\P 1598 SHAKES.  \textit{Merry W.} v. v. 157.    
\P 1788 GIBBON  \textit{Decl. \& F.} xlix. V. 98 The Jewish king, who had broken without scruple the brazen serpent.    
\P 1849 MACAULAY  \textit{Hist. Eng.} ii. I. 186 Attacked by the civil power, they without scruple repelled force by force.

\itembf{d.} Phr. to have scruples; to have little scruple, no scruple, etc. Const. about (a matter), in (doing something).

\P 1719 DE FOE  \textit{Crusoe} i. 340, I had some little Scruple in my Mind about Religion, which insensibly drew me back.    
\P 1736  \textit{Gentl. Mag.} VI. 709/2 That the Quakers can have no Scruple of Conscience in paying Tythes.    
\P 1828 MACAULAY  \textit{Ess., Hallam's Const. Hist.} (1897) 80 A man without truth or humanity may have some strange scruples about a trifle.    
\P 1850 J. W. CROKER in \textit{C. Papers} 14 June (1884) I. i. 18 If you have the slightest [objection], pray have no scruple in leaving my curiosity ungratified.    
\P 1865 KINGSLEY  \textit{Herew.} viii, [They] had little scruple in applying to a witch.

\itembf{e.} \textbf{to make scruple} (also a, no, etc. scruple): to entertain or raise a scruple or doubt; to hesitate, be reluctant, esp. on conscientious grounds. Const. infin.; also with of (at, in) = to stick at, hesitate to do or allow, etc. ? Obs. (Cf. F. faire scrupule, with similar constructions.)

\P 1589 NASHE  \textit{Pasquill \& Marf.} B j, They presume to make a shrewde scruple of their obedience.    
\P 1591 SAVILE  \textit{Tacitus, Hist.} i. lxxxix. 51 Making a scruple that the holy shields called Ancilia were as yet not layed up againe.    
\P 1603 B. JONSON  \textit{Sejanus} iv. v. (1605) I 4 b, Lac. But is that true, it 'tis prohibited To sacrifice vnto him? Ter. Some such thing Cæsar makes scruple of, but forbids it not.    
\P 1605 BACON  \textit{Adv. Learn.} ii. xxiii. §36 Cæsar‥made no scruple to professe that hee had rather bee first in a village, then second at Rome.    
\P 1639 N. N. TR.  \textit{Du Bosq's Compl. Woman} i. 57 The superstitious make more scruple of a little sinne then of a great.    
\P 1669-70 MARVELL  \textit{Corr.} cxxxii. Wks. (Grosart) II. 298 One of those who thinke it the greatest point of wisdome to make the most scruples.    
\P 1722 DE FOE  \textit{Moll Flanders} (1840) 210, I made no scruple at taking these goods.    
\P 1845 FORD  \textit{Handbk. Spain} i. 14 Small scruple is made by the authorities in opening private letters.

\itembf{2.} A doubt or uncertainty as to a matter of fact or allegation; an intellectual difficulty, perplexity, or objection. beyond a scruple, beyond doubt or cavil. Obs.

   The phrase ‘scruple of suspition’ (quot. 1534) perh. contains an etymologizing reference to scruple n.1 6. Cf. ‘un seul scrupule de doubte’, 16th c. in Littré.

\P 1534 MORE in  \textit{Ellis Orig. Lett. Ser.} i. II. 49 In eny parte of all which my dealing, whither eny other man may peradventure put eny dowt, or move eny scruple of suspition.    
\P 1568 GRAFTON  \textit{Chron.} II. 644 For auoyding of which scruple and ambiguity: Edmund Erle of Marche‥made his tytle and righteous clayme.    
\P 1597 MORLEY  \textit{Introd. Mus.} 16 In the Table there is no difficultie‥yet, to take away all scruple, I will shew you the vse of it.    
\P 1662 STILLINGFL.  \textit{Orig. Sacræ} i. v. §2 The only scruple is whether it was used in their sacred accounts or no.    a 
\P 1718 PENN  \textit{Innocency with open Face} Wks. 1726 I. 267,  I hope my Innocency will appear beyond a Scruple.    
\P 1725 DE FOE  \textit{Voy. round World} (1840) 22 Our captain‥raised several scruples about the latitude which we should keep in such a voyage.    
\P 1741 HARRIS  \textit{Three Treat.} iii. i. (1765) 140 A Subject, where one's own Interest appeared concerned so nearly would well justify every Scruple, and even the severest Inquiry.

\itembf{b.} Disbelief or doubt of. to have or make scruple of: to hesitate to believe or admit. Also rarely with how and clause. Obs.

\P 1597 SHAKES.  \textit{2 Hen. IV}, i. ii. 149 But how I should bee your Patient, to follow your prescriptions, the wise may make some dram of a scruple, or indeede, a scruple it selfe.    
\P 1611 \textit{Cymb.} v. v. 182 Whereat, I wretch Made scruple of his praise.    a 
\P 1628 PRESTON  \textit{New Covt.} (1634) 116 When there is no scruple in our hearts of Gods love towards us.    
\P 1662 EVELYN  \textit{Chalcogr.} 12 That Letters, and consequently Sculpture, was long before the Flood, we make no scruple of.    1666-7 Marvell Corr. lxix. Wks. (Grosart) II. 210 If you find any thing perplext in it, I shall‥resolve any scruple that you may have of its exposition.    
\P 1672 VILLIERS  (Dk. Buckhm.) \textit{Rehearsal} i. (Arb.) 33 If you make the least scruple of the efficacie of these my Rules, do but come to the Play-house, and you shall judge of 'em by the effects.

\itembf{c.} without scruple: without doubt or question, doubtless. (Used to qualify an assertion.) Obs.

\P 1612 SELDEN  \textit{Illustr. Drayton's Poly-olb.} xi. 189 As is, without scruple, apparant in the date of the synod.    
\P 1690 CHILD  \textit{Disc. Trade} (1698) 49 The same house to be sold‥would have yielded without scruple
\P 1000 OR
\P 1200 L.

\itembf{d.} A suspicion of (something). rare—1.

\P 1597 SIR R. CECIL  in Ellis \textit{Orig. Lett.} Ser. i. III. 42 Wherein that you may see the poore unfortunate Secretarie will leave no scrupule in you of lack of industry, to yeald you all satisfaction‥I have thought good to [etc.].

\itembf{e.} A quibble, fine distinction. Obs.

\P 1709 FELTON  \textit{Diss. Classics} (1718) 43 If there is any Thing else Commentators concern themselves about, it is Property of Expression, or rather some Verbal Niceties, and Grammatical Scruples.

\itembf{3.} Comb., as scruple-drawer (applied to a confessor), scruple-monger; scruple-selling ppl. a.

\P 1704 T. BROWN  \textit{Laconics} Wks. 1711 IV. 19 The  late Ordinary of Newgate, Mr. Smith, who was one of the most famous *Scruple-drawers of his Time.

\P 1675 WALTON HOOKER in  \textit{Wordsw. Eccl. Biog.} (1818) IV. 223 There were also many of these *Scruplemongers that pretended a tenderness of conscience, refusing to take an oath before a lawful magistrate.

\P 1704 T. BROWN  \textit{Reas. Oaths} Wks. 1711 IV. 91 B,  Printed by one of those Godly Wholesale Dealers in Scandal, those *Scruple-selling Vermin of the Poultry.
\end{myenumerate}


%%%%%%%%%%%%%%%%%%%%%%%%%%%%%%%%%
\myitem{scrupulous} a.

\noindent \phonetic{(ˈskruːpjʊləs)}

\noindent [ad. F. scrupuleux (16th c., scrupuleusement 14th c.), or ad. L. scrūpulōs-us, f. scrūpul-us: see scruple n.2 and -ous.]
\vspace{-0.3cm}

\begin{myenumerate}

\itembf{1.} Troubled with doubts or scruples of conscience; over-nice or meticulous in matters of right and wrong. Also (of things, actions, etc.), characterized by such scruples.

\P 1530  \textit{Myrr. our Ladye} 52 Yt is good in suche case to be gouernyd by the consayle of a dyscrete gostly father leste the dome of hys owne conscyence be other to scrupulous or to recheles.    
\P 1513 MORE  \textit{Rich. III}, Wks. 58/1 Of spiritual men thei toke such as had wit,‥\& had no scrupilouse consience.    
\P 1528 HENRY VIII in  \textit{R. Hall Life Fisher F.'s Wks.} (E.E.T.S.) ii. 61 Whiche thinge‥ingendred such a scrupilous doubt in me, that my mind was incontinently accombred, vexed, and disquyeted.    
\P 1593 SHAKES.  \textit{3 Hen. VI}, iv. vii. 61 Rich. Why Brother, wherefore stand you on nice points?‥ Hast. Away with scrupulous Wit, now Armes must rule.    
\P 1594 HOOKER  \textit{Eccl. Pol.} iv. xi. §5 Abusing their libertie and freedom to the offence of their weake brethren which were scrupulous.    
\P 1667 in  \textit{Cath. Rec. Soc. Miscell.} III. 64 And yet, though he spent so much time in examining his consciens, he was not the least scrupulous nor long at Confession.    
\P 1765 BLACKSTONE  \textit{Comm.} i. vi. 226 Whatever doubts might be formerly raised by weak and scrupulous minds about the existence of such an original contract.    
\P 1835 I. TAYLOR  \textit{Spir. Despot.} iii. 108 The common people superstitious, fanatical, scrupulous, licentious.    
\P 1907 A. C. BENSON  \textit{Altar Fire} 134 The religion recommended was a religion of scrupulous saints and self-torturing ascetics.

\itembf{b.} Prone to hesitate or doubt; distrustful; cautious or meticulous in acting, deciding, etc. Also (of actions, etc.), characterized by doubt or distrust; (of objections) cavilling. Obs.

\P 1559 W. CUNINGHAM  \textit{Cosmogr. Glasse} 46 It is truely said, that knowledge hath no enemie but ignoraunce. There are‥no small number of Lactantius sort, not scrupulous enemies onely, but also Physicians, of whom [etc.].    
\P 1560 J. DAUS tr. \textit{Sleidane's Comm. Pref.} 2 b, Thucydides was so desyrous of the verity, and so doubt full and scrupulous in wryting of his story.    
\P 1611 CORYAT  \textit{Crudities} 67 The Italians are so curious and scrupulous in many of their cities‥that they will admit no stranger within the wals‥except he bringeth a bill of health from the last citie he came from.    
\P 1614 RALEIGH  \textit{Hist. World} ii. xxiii. §4. 574 But in filling vp the blankes of old Histories, we neede not be so scrupulous.    a 
\P 1681 WHARTON  \textit{Apotelesma} Wks. (1683) 44 Nor any one [sc. art or science] that can truly say, it is free from every scrupulous exception.    
\P 1695 WOODWARD  \textit{Nat. Hist. Earth Acc. Observ.} 8, I have been the more scrupulous and wary, in regard the Inferences drawn from these Observations are of some importance.

\itembf{c.} with const.: Loth or reluctant, through scruples, to (do something); doubtful or suspicious of (a person or thing); chary of or in (doing something); anxious or fearful about. Obs.

\P 1608 D. T[UVILL]  \textit{Ess. Pol. \& Mor.} 125 Hee was no way scrupulous to circumvent, and kill, insontes sicuti sontes.    
\P 1643 SIR T. BROWNE  \textit{Relig. Med.} i. §3. 4 And therefore I am not scrupulous to converse and live with them.
\P c1645 HOWELL  \textit{Lett.} (1650) II. 32 The Father is scrupulous of the Son, the Son of the Sisters, and all three of me, to whose award they referr'd the business three severall times.    
\P 1658 SIR T. BROWNE  \textit{Hydriot.} i. 5 The Jews‥as they raised noble Monuments and Mausolæums for their own Nation, so they were not scrupulous in erecting some for others.    
\P 1662 STILLINGFL.  \textit{Orig. Sacræ} ii. ix. §21. 320 The primitive Christians were very scrupulous of calling the Emperours Dominus.    
\P 1754 RICHARDSON  \textit{Grandison} IV. xxi. 161 She often directed herself to me in Italian. I do not talk it well: But‥I was not scrupulous to answer in it.    
\P 1785 PHILLIPS  \textit{Treat. Inland Nav.} 33 Those‥whom I have consulted on the subject, where I was scrupulous of my knowledge.    
\P 1845 S. JUDD  \textit{Margaret} ii. viii. (1871) 284 Don't you stir out of the house; I am scrupulous about what might happen.

\itembf{d.} absol. (\textbf{the scrupulous} = scrupulous persons.)

\P 1625 B. JONSON  \textit{Staple of N.} iii. ii. 118 'Tis the house of fame, Sir, Where both the curious, and the negligent, The scrupulous, and carelesse;‥all doe meet.    
\P 1690 LOCKE  \textit{Hum. Und.} iii. vi. §12 There are some Birds‥whose Bloud is cold as Fishes, and their Flesh in taste so near akin, that the Scrupulous are allow'd them on Fish-days.

\itembf{2.} Of a thing: Causing or raising scruples; liable to give offence; meriting scruple or cavil, dubious, doubtful. to make it scrupulous: to scruple, hesitate (to do something). Obs.

\P 1548 HALL  \textit{Chron., Hen. VII} 57 The scrupulous stynges of domesticall sedicion.    
\P 1574 HELLOWES  \textit{Guevara's Fam. Epist.} (1577) 66 If your warre had ben vpon Ierusalem, it were to be holden for iust, but for that it is vpon Marsillius, alway we hold it for scrupulous.    
\P 1593 \textit{Tell-trothe's  New Yeare's Gift} 3 And it being my hap to enquire first from whence hee came, hee made it not scrupulous to certifie his comming from hell.    
\P 1622 BACON  \textit{Holy War} Misc. Wks. (1629) 117 As the Cause of a Warre ought to be Iust; So the Iustice of that Cause ought to be Euident; Not Obscure, not Scrupulous.    
\P 1685 BUNYAN  \textit{Quest. Seventh-day Sabbath} ii. 16 This yet seems to me more scrupulous, because that the punishment due to the breach of the Seventh-day Sabbath was hid from men to the time of Moses.

\itembf{b.} Of the nature of a mere scruple. Obs.

\P 1605 in  \textit{10th Rep. Hist. MSS. Comm.} App. v. 372 Let not any man mervaylle of the manyfould downefalles into synne, or think it a thing scrupulous.

\itembf{3.} Careful to follow the dictates of conscience; giving heed to the scruples of conscience so as to avoid doing what is wrong; strict in matters of right and wrong.

   A use of sense 1 developed chiefly in contexts with a negative expressed or implied.

\P 1545 ELYOT  \textit{Dict.} s.v. Religiosus, In testimonio religiosi, scrupulouse in bearynge wytnesse.    
\P 1849 MACAULAY  \textit{Hist. Eng.} ii. I. 210 His more scrupulous brother ceased to appear in the royal chapel.    
\P 1863 MRS. GASKELL  \textit{Sylvia's L.} iii, Yet, though scrupulous in most things, it did not go against the consciences of these good brothers to purchase smuggled articles.

\itembf{b.} With inf.: Careful (to do something) in obedience to one's conscience.

\P 1729 BUTLER \textit{Serm.} Wks. 1874 II. 50 We should be religiously scrupulous and exact to say nothing‥but what is true.

\itembf{4.} Of actions, etc.: Rigidly directed by the dictates of conscience; characterized by a strict and minute regard for what is right.

\P 1756 BURKE  \textit{Tracts Popery Laws} Wks. IX. 338 This point is carried to so scrupulous a severity, that chamber practice, and even private conveyancing‥are prohibited to them under the severest penalties.    
\P 1779 \textit{Mirror}  No. 37 While he gave to business the most scrupulous attention.    
\P 1855 MACAULAY  \textit{Hist. Eng.} xiii. III. 248 William saw that he must not think of paying to the laws of Scotland that scrupulous respect which he had wisely and righteously paid to the laws of England.    
\P 1876 M. E. BRADDON  \textit{J. Haggard's Dau.} I. 9 A scrupulous honesty recommended him even to careful housekeepers.

\itembf{5.} Minutely exact or careful (in non-moral matters); strictly attentive even to the smallest details; characterized by punctilious exactness.

\P 1638 JUNIUS  \textit{Paint. Ancients} 77 Examining‥every little moment of Art with such infatigable though scrupulous care.    
\P 1711 ADDISON  \textit{Spect.} No. 160 \phonetic{⁋}4 Where we would make some Amends for our want of Force and Spirit, by a scrupulous Nicety and Exactness in our Compositions.    
\P 1779 JOHNSON  \textit{L.P., Cowley} (1805) I. 44 Thus all the power of description is destroyed by a scrupulous enumeration.    
\P 1837 DICKENS  \textit{Pickw.} ii, Great men are seldom over scrupulous in the arrangement of their attire.    
\P 1862 MILLER  \textit{Elem. Chem., Org.} (ed. 2) 11 Scrupulous attention to the purity of the matter submitted to analysis is of course of primary importance.    
\P 1881 WESTCOTT \& HORT  \textit{Grk. N.T.} Introd. §11 A scrupulous jealousy as to their text.    
\P 1863 GEO. ELIOT  \textit{Romola} v, Shelves, on which books‥were arranged in scrupulous order.    
\P 1886  \textit{Manch. Exam.} 14 Jan. 5/4 The various performances were gone through with scrupulous exactitude.

\itembf{6.} Wrought or produced with minute care and exactness. Obs.

\P 1634 RAINBOW  \textit{Labour} (1635) 34 If seelings be an ornament, what are scrupulous carvings?
\end{myenumerate}


%%%%%%%%%%%%%%%%%%%%%%%%%%%%%%%%%
\myitem{wayward} a. Not now in colloquial use.

\noindent \phonetic{(ˈweɪwəd)}

\noindent [Aphetic f. AWAYWARD. Cf. \textit{froward}.

   The word has prob. often been apprehended as a derivative of WAY n.1, with the literal sense ‘bent on going one's own way’; this notion seems to have influenced the development of meaning.]
\vspace{-0.3cm}

\begin{myenumerate}

\itembf{1.} Disposed to go counter to the wishes or advice of others, or to what is reasonable; wrongheaded, intractable, self-willed; froward, perverse. Of children: Disobedient, refractory.

   In recent use the sense is somewhat milder, and perhaps always with some mixture of 2. If applied to conduct deserving severe moral reprobation it would now be apprehended as euphemistic.

\P 1380 WYCLIF  \textit{Wks.} (1880) 376 As waiwerd clerkis wolden in seynt Austyns time haue done owte‥þis worde of þe gospelle.    
\P 1382 \textit{Matt.} xvii. 16 A! thou generacioun vnbyleeful and weiward [Vulg. perversa].    c 
\P 1425 \textit{Eng. Conq. Irel.} 142 Folk so weyward \& so vnredy.    c 
\P 1475 \textit{Lament.  Mary Magd.} 237 Wherfore ye lyke tyrantes wode \& waywarde Now haue him thus slayne for his rewarde.    
\P 1526  \textit{Pilgr. Perf.} (W. de W. 1531) 20 Than he waxeth testy and weywarde, and for every tryfell is impacyent and angry.    
\P 1557 NORTH  \textit{Gueuara's Diall} Pr. Gen. Prol. A ij, Many sorowes endureth the woman in nouryshyng a waywerde chylde.    
\P 1583 STUBBES  \textit{Anat. Abus.} ii. 102 [They] shewe them selues either wilfull, waiwarde, or maliciouslye blinde.    
\P 1583 WHITGIFT  \textit{Serm.} (1589) C 6 b, The third kinde is of those that are conceited and wayward, who onely obey when they list, wherein they list, and so long as they list.    
\P 1590 SHAKES.  \textit{Com. Err.} iv. iv. 4 My wife is in a wayward moode to day.    
\P 1651 FEATLY  \textit{Abel Rediv.}, Reinolds 486 A waward Patient maketh a froward Physitian.    
\P 1830 D'ISRAELI  \textit{Chas. I.} III. 97 Charles‥used the wayward genius with all a brother's tenderness.    
\P 1833 TENNYSON  \textit{New-Year's Eve} 25, I have been wild and wayward, but you'll forgive me now.    
\P 1840 DICKENS  \textit{Old C. Shop} lxix, The wayward boy soon spurned the shelter of his roof, and sought associates more congenial to his tastes.    
\P 1894 LADY  M. VERNEY \textit{Verney Mem.} III. 326 Sir Ralph treated the wayward girl with a courtesy to which her mother never condescended.

absol. \P 1581 J. BELL  \textit{Haddon's Answ. Osor.} 63 b, Here our old peevish wayward, piketh a new quarell agaynst me.    
\P 1582 N. T. (RHEM.)  \textit{1 Pet.} ii. 18 Not only the good and modest, but also the waiward [Vulg. dyscolis].    
\P 1912  \textit{Spectator} 27 July 135/2 The two together supply the unwise and the wayward with the necessary instructions.

\itembf{b.} Of things personified. Also of conditions, natural agencies, etc.: Untoward. Obs.

\P 1567 TURBERV.  \textit{Epit., etc.} 80 b, When waywarde Winter spits his gall.    a 
\P 1586 SIDNEY  \textit{Arcadia} iii. xxix. §1 What spiteful God‥hath brought me to such a waywarde case, that neither thy death can be a reuenge, nor thy ouerthrow a victorie.    
\P 1608 SHAKES.  \textit{Per.} iv. iv. 10 Pericles Is now againe thwarting thy wayward seas.    
\P 1718 PRIOR  \textit{Solomon} ii. 803 My Coward Soul shall bear it's wayward Fate.    
\P 1792 F. BURNEY  \textit{Diary Apr.}, This wayward month opened upon me with none of its smiles.    
\P 1821 J. BAILLIE  \textit{Metr. Leg.}, Ghost of Fadon vii, We war with wayward fate.

\itembf{c.} Of judgement: Perverse, wrong, unjust. Also of the eye: Perverted. Obs.

\P 1382 WYCLIF  \textit{Matt.} vi. 23 Ȝif thyn eiȝe be weyward [Vulg. nequam].      Hab. i. 4 Weywerd dom [Vulg. judicium perversum].    
\P 1551 ROBINSON tr.  \textit{More's Utopia} (1895) 40 Suche prowde, lewde, ouerthwarte, and waywarde iudgementes [L. superba, absurda ac morosa iudicia].    
\P 1668 DRYDEN  \textit{Dram. Poesy} 51 The wayward authority of an old man in his own house.

\itembf{d.} Of words, actions, countenance: Indicating or manifesting obstinate self-will. Obs.

\P 1530  \textit{Myrr. Our Ladye} 44 An other he [the Evil One] sturreth to make som weywarde token.    
\P 1599 SANDYS  \textit{Europæ Spec.} (1632) 94 If a man should heap together all the cholerike speeches, all the way-ward actions, that ever scaped from him in his life.    
\P 1630 \textit{Pathomachia}  i. iv. 8 From wayward words they passed on to bloody blowes.    
\P 1818 SCOTT  \textit{Rob Roy} xii, I shall never forget the diabolical sneer which writhed Rashleigh's wayward features.

\itembf{e.} Of a disease, etc.: Not yielding readily to treatment, obstinate. Obs.

\P 1541 R. COPLAND  \textit{Galyen's Terap.} 2 F iv, By the occasyon of them the vlcere is waywarde and rebel to be healed.

\itembf{2.} Capriciously wilful; conforming to no fixed rule or principle of conduct; erratic.

\P 1533 LD. BERNERS  \textit{Golden Bk. M. Aurel. Let.} iv. (1537) 118 b, Our lyfe is so doubtefull, and fortune so waywarde, that she dothe not alway threate in strykynge, nor striketh in thretnynge.    
\P 1604 DEKKER  \textit{Honest Wh.} i. B 1, My longings are not wanton, but wayward.    
\P 1750 GRAY  \textit{Elegy} 106 Hard by yon wood‥Mutt'ring his wayward fancies he would rove.    
\P 1832 WORDSW.  \textit{Loving \& Liking} 44 Instinct is neither wayward nor blind.    
\P 1881 JOWETT  \textit{Thucyd.} I. 88 The movement of events is often as wayward and incomprehensible as the course of human thought.

\itembf{b.} transf. and fig. (of things).

\P 1786 BURNS  \textit{Brigs of Ayr} 51 He left his bed and took his wayward rout, And down by Simpsons wheel'd the left about.    
\P 1799 WORDSW.  \textit{Poems Imag.} x. 28 In many a secret place Where rivulets dance their wayward round.    
\P 1817 SCOTT  \textit{Harold} ii. xv, Thus muttering, to the door she bent Her wayward steps.    18‥ Smithson Usef. Bk. Farmers 32 (Cassell) Send its rough wayward roots in all directions.    
\P 1905 \textit{C.T.C.  Gaz.} June 254/1 The wayward hoop is a fruitful cause of those accidents for which no one except the victim gets punished.
\end{myenumerate}


%%%%%%%%%%%%%%%%%%%%%%%%%%%%%%%%%
\myitem{fretful} a.

\noindent \phonetic{(ˈfrɛtfʊl)}

\noindent [f. fret v.1 + -ful.]
\vspace{-0.3cm}

\begin{myenumerate}

\itembf{1.} \textbf{a.} Corrosive, irritating, lit. and fig. \textbf{b.} Irritated, inflamed. Obs.

\P 1593 SHAKES.  \textit{2 Hen. VI,} iii. ii. 403 Though parting be a fretfull corosiue, It is applyed to a deathfull wound.    
\P 1594 PLAT  \textit{Jewell-ho.} i. 56 More sharpe, and fretfull to their fingers than their vsuall morter.    
\P 1804 ABERNETHY  \textit{Surg. Observ.} 126 The ulcer‥was of the size of a shilling, with fretful edges.

\itembf{2.} Disposed to fret, irritable, peevish, ill-tempered; impatient, restless.

\P 1602 SHAKES.  \textit{Ham.} i. v. 20 A Tale‥whose lightest word would‥make‥each particular haire to stand on end, Like Quilles vpon the fretfull Porpentine.    
\P 1632 J. HAYWARD tr. \textit{Biondi's Eromena} 96 In so much as he became fretfull, and pettish.    
\P 1739 CIBBER  \textit{Apol.} (1756) II. 34 The fretful temper of a friend.    
\P 1774 GOLDSM.  \textit{Nat. Hist.} (1776) IV. 209 Impelled by a fretful impetuosity.    
\P 1802  \textit{Med. Jrnl.} VIII. 528 The child had become more silly and fretful.    
\P 1833 \textit{Regul.  Instr. Cavalry} i. 83 A horse continues uneasy and fretful with the bit.    
\P 1837 LYTTON  \textit{E. Maltrav.} iii. ii, Men of second-rate faculties‥are fretful and nervous.    a 
\P 1848 ROSSETTI  \textit{Blessed Damozel} vi, Where this earth Spins like a fretful midge.

\itembf{3. a.} Of water, etc.: Agitated, troubled, broken into waves. \itembf{b.} Of the wind: Blowing in frets or gusts; gusty.

\P 1613-16 W. BROWNE  \textit{Brit. Past.} ii. iv. 691 Two goodly streames‥Whose fretfull waues beating against the hill, Did all the bottome with soft muttrings fill.    
\P 1793 SMEATON  \textit{Edystone L.} §322 The horizon‥was so extremely black, fretful, and hazy, that nothing could be seen.    a 
\P 1849 J. C. MANGAN  \textit{Poems} (1859) 122 Bitter blows the fretful morning wind.    
\P 1887  \textit{Pall Mall G.} 25 July 2/2 A pretty picture framed by the fretful sea and the cloudless sky.

\itembf{4.} Characterized by or apt to produce fretting.

\P 1737 THOMSON  \textit{Mem. Ld. Talbot} 340 The kindred Souls of every Land, (Howe'er divided in the fretful Days Of Prejudice and Error) mingled now.    
\P 1798 WORDSW.  \textit{Tintern Abbey}, The fretful stir Unprofitable and the fever of the world.    
\P 1852 BLACKIE  \textit{Study Lang.} 33 To pick words out of a dictionary is fretful.    
\P 1890 \textit{Murray's  Mag.} June 737 The fearsome, fretful, forest, dank and deep.

Hence \textbf{fretfully} adv., in a fretful manner; \textbf{fretfulness}, the quality or condition of being fretful.

\P 1615 CROOKE  \textit{Body of Man} 274 And this we tearme fretfulnesse or pettishnes.    
\P 1789 F. BURNEY  \textit{Diary Apr.}, Really frightened at she knew not what, she fretfully exclaimed, [etc.].    
\P 1843 J. MARTINEAU  \textit{Chr. Life} (1867) 239 Drives away every trace of fretfulness.    
\P 1860 FROUDE  \textit{Hist. Eng.} V. 174 The Carews rode fretfully up and down the river banks, probing the mud with their lances to find footing for their horses.    
\P 1880 OUIDA  \textit{Moths} I. ix. 228 ‘What is the use of putting off?’ said her mother fretfully, ‘you will be ill’.
\end{myenumerate}


%%%%%%%%%%%%%%%%%%%%%%%%%%%%%%%%%
\myitem{sullen} a. adv. and n.

\noindent \phonetic{(ˈsʌlən)}

\noindent [Later form of solein.]
\vspace{-0.3cm}

\begin{myenumerate}

\itembf{A.} adj.
\itembf{1. a.} Of persons, their attributes, aspect, actions: Characterized by, or indicative of, gloomy ill-humour or moody silence.

   In early use there is often implication of obstinacy or stubbornness.

\P 1573-80 TUSSER  \textit{Husb.} (1878) 180 Be lowly not sollen, if ought go amisse.    
\P 1592 \textit{Arden  of Feversham} i. i. 510 Who would haue thought the ciuill sir so sollen?    
\P 1641 ‘SMECTYMNUUS’  \textit{Vind. Answ. To Rdr.}, Wee are called‥sullen and crabbed peices.    
\P 1668  \textit{Extr. St. Papers rel. Friends} Ser. iii. (1912) 279 Their Saint Penn‥is divelishly cryed vp amongest that pervers sullen Faction.    
\P 1680 C. NESSE  \textit{Church Hist.} 55 Because they might not have what they would, grew sullain, and would have nothing.    
\P 1713 STEELE  \textit{Guard.} No. 18 \phonetic{⁋}2 These contemplations have made me serious but not sullen.    
\P 1718  \textit{Free-thinker} No. 149. 323 In the Middle sits Cato, with a sullen Brow.    
\P 1795 BURKE  \textit{Corr.} (1844) IV. 315 If the better part lies by, in a sullen silence, they still cannot hinder the more factious part both from speaking and from writing.    
\P 1814 WORDSW.  \textit{Excurs.} vi. 459 Here‥they met,‥flaming Jacobite And sullen Hanoverian!    
\P 1849 MACAULAY  \textit{Hist. Eng.} vi. II. 28 The answer of James was a cold and sullen reprimand.    
\P 1879 FROUDE  \textit{Cæsar} xxvi. 438 Some were still sullen, and refused to sue for a forgiveness.

\itembf{b.} transf. Of animals and inanimate things: Obstinate, refractory; stubborn, unyielding.

\P 1577 B. GOOGE  \textit{Heresbach's Husb.} iii. 128 b, Which being well punished with hunger, and thyrst, wyll teache him [sc. a plough-ox] to leaue that sullen tricke.    
\P 1648 GAGE  \textit{West Ind.} 89, I got up again and spurred my sullen jade.    
\P 1678 CUDWORTH  \textit{Intell. Syst.} i. v. 888 Things are Sullen, and will be as they are, what ever we Think them, or Wish them to be.    
\P 1691 RAY  \textit{Creation} i. (1692) 38 The stupid Matter‥would be as sullen as the Mountain was that Mahomet commanded to come down to him.    
\P 1725 DE FOE  \textit{Voy. round World} (1840) 339 The other [bull] proved untractable, sullen, and outrageous.    
\P 1859 TENNYSON  \textit{Geraint \& Enid} 862 As sullen as a beast new-caged.

\itembf{c.} Holding aloof. Obs.

\P 1628 EARLE  \textit{Microcosm., Acquaintance} (Arb.) 86 Friendship is a sullener thing, as a contracter and taker vp of our affections to some few.

\itembf{d.} fig. Baleful, malignant. Obs.

\P 1676 DRYDEN  \textit{Aurengz.} i. i. 360 Such sullen Planets at my Birth did shine, They threaten every Fortune mixt with mine.    
\P 1679 DRYDEN \& LEE  \textit{Œdipus} iii, Ye sullen Pow'rs below.    
\P 1703 ROWE  \textit{Fair Penit.} ii. i, Some sullen Influence, a Foe to both.

\itembf{2.} Solemn, serious. Obs.

\P 1583 B. MELBANCKE  \textit{Philotimus} M iij b, So was he free from sulleyne sterne seuerity.    a 
\P 1586 SIDNEY  \textit{Apol. Poetrie} (Arb.) 30 Morrall Philosophers, whom me thinketh, I see comming towards me with a sullen grauity.    
\P 1640 BP. REYNOLDS  \textit{Passions} iv, Some plausible Fancy doth more prevail with tender Wills than a severe and sullen argument.    
\P 1719 YOUNG  \textit{Busiris} i. i, In sullen Majesty they stalk along, With Eyes of Indignation, and Despair.

\itembf{3. a.} Of immaterial things, actions, conditions: Gloomy, dismal, melancholy; sometimes with the notion of ‘passing heavily, moving sluggishly’.

\P 1593 SHAKES.  \textit{Rich. II,} i. iii. 265 The sullen passage of thy weary steppes.    
\P 1604 \textit{Oth.} iii. iv. 51 (Q1), A salt and sullen rhume.    
\P 1605 DANIEL  \textit{Philotas Ep.} 59 To sound The deepe reports of sullen Tragedies.    
\P 1648 MILTON  \textit{Sonn.} xvii, Where shall we sometimes meet, and by the fire Help wast a sullen day.    
\P 1712-14 POPE  \textit{Rape Lock} iv. 19 No cheerful breeze this sullen region knows.    
\P 1775 JOHNSON  \textit{Let. to Mrs. Thrale} 1 Aug., The place [sc. Oxford] is now a sullen solitude.    
\P 1816 BYRON  \textit{Prisoner of Chillon} xiv, With spiders I had friendship made, And watch'd them in their sullen trade.    
\P 1858 KINGSLEY  \textit{Lett.} (1878) I. 21 It was an afternoon of sullen Autumn rain.    a 
\P 1864 HAWTHORNE  \textit{Amer. Note-bks.} (1879) II. 52 A bleak, sullen day.

\itembf{b.} Of a sound or an object producing a sound: Of a deep, dull, or mournful tone. Chiefly poet.

\P 1592 SHAKES.  \textit{Rom. \& Jul.} iv. v. 88 Our solemne Hymnes, to sullen Dyrges change.    
\P 1632 MILTON  \textit{Penseroso} 76, I hear the far-off Curfeu sound,‥Swinging slow with sullen roar.    
\P 1742 COLLINS  \textit{Ode} ix. 12 Where the beetle winds His small but sullen horn.    
\P 1819 SCOTT  \textit{Ivanhoe} xliv, The heavy bell‥broke short their argument. One by one the sullen sounds fell successively on the ear.    
\P 1849 KINGSLEY  \textit{North Devon} in \textit{Misc.} (1859) II. 264 The sullen thunder of the unseen surge.

\itembf{4. a.} Of sombre hue; of a dull colour; hence, of gloomy or dismal aspect. (Also qualifying an adj. of colour = dull-.) Cf. sad a. 8.

\P a1586 [implied in  SULLENLY 2].    
\P 1592 \textit{Arden  of Feversham} iii. i. 45 Now will he shake his care oppressed head, Then fix his sad eis on the sollen earth.    
\P 1596 SHAKES.  \textit{1 Hen. IV,} i. ii. 236 Like bright Mettall on a sullen ground.    
\P 1647 HARVEY  \textit{Sch. of Heart} xxi. i, Take sullen lead for silver, sounding brass Instead of solid gold.    
\P 1665 J. REA  \textit{Flora} 130 A dark sullen violet purple colour.    
\P 1710 STEELE  \textit{Tatler} No. 266 \phonetic{⁋}3 Two apples that were roasting by a sullen sea coal fire.    
\P 1713  \textit{Phil. Trans.} XXVIII. 224 A sort of sullen greenish Wood-like rust.    
\P 1784 COWPER  \textit{Task} ii. 212, I would not yet exchange thy sullen skies‥for warmer France With all her vines.    
\P 1811 SCOTT  \textit{Don Roderick} ii. i, All sleeps in sullen shade, or silver glow.    
\P 1818 KEATS  \textit{Sonn. Ben Nevis} 6, I look o'erhead, And there is sullen mist.    
\P 1855 TENNYSON  \textit{Maud} i. x. i, The sullen-purple moor.    
\P 1894 HALL  \textit{Caine Manxman} v. iii. 286 The sky to the north-west was dark and sullen.

\itembf{b.} \textbf{sullen lady}, ? Fritillaria nigra. Obs.

\P 1688 HOLME  \textit{Armoury} ii. iv. 74/1 The sullen Lady, hangeth her head down‥and is of an umberish dark hair colour, without any checker or spots. Some call it the black Fritillary.

\itembf{5.} Of water, etc.: Flowing sluggishly. poet.

\P 1622 DRAYTON  \textit{Poly-olb.} xxviii. 91 Small Cock, a sullen Brook, comes to her succour then.    
\P 1628 MILTON  \textit{Vac. Exerc.} 95 Sullen Mole that runneth underneath.    
\P 1814 SCOTT  \textit{Wav.} xxii, The larger [stream] was placid, and even sullen in its course.    
\P 1818 SHELLEY  \textit{Rosal. \& Helen} 398 Each one lay Sucking the sullen milk away About my frozen heart.

\itembf{6.} Comb.: parasynthetic adjs., as sullen-browed, sullen-eyed, sullen-faced, sullen-hearted; complementary, as sullen-blooming, sullen-looking, sullen-seeming, sullen-smiling; with other adjs., as sullen-sour, sullen-wise.

\P 1879 O. WILDE in  \textit{Time} July 402 No *sullen-blooming poppies stain thy hair.

\P 1831 SCOTT  \textit{Cast. Dang.} ii, This *sullen-browed Thomas Dickson.

\P 1961 R. S. THOMAS  \textit{Tares} 47 And given to watching, *sullen-eyed, Love still-born, as it was then.

\P 1914 JOYCE  \textit{Dubliners} 117 A very *sullen-faced man.

\P 1909 R. BRIDGES  \textit{Par. Virg. Æn.} VI, 434 The *sullen-hearted, who‥Their own life did-away.

\P 1855 TENNYSON  \textit{Maud} i. xviii. vi, *Sullen-seeming Death.

\P 1849 J. A. CARLYLE  tr. \textit{Dante's Inf.} p. xliv, The *Sullen-sour or Gloomy-sluggish.

\P 1919 J. MASEFIELD  \textit{Reynard the Fox} i. 29 Surly, Tall, shifty, *sullen-smiling.

\P 1710 STEELE  \textit{Tatler} No. 149 \phonetic{⁋}5 A *sullen-wise Man is as bad as a good-natured Fool.

\itembf{B.} adv. = SULLENLY. rare.

\P 1718 PRIOR  \textit{Solomon} ii. 201 Sullen I forsook th' Imperfect Feast.    
\P 1810 SCOTT  \textit{Lady of L.} ii. xxxiv, Sullen and slowly they unclasp.

\itembf{C.} n. \textbf{a.} (in pl., usually \textbf{the sullens}; rarely sing.) A state of gloomy ill-humour; sullenness, sulks. Phr. \textbf{in the sullens, sick of the sullens}.

\P 1580 LYLY  \textit{Euphues} (Arb.) 285 She was solitaryly walking, with hir frowning cloth, as sick lately of the solens.    
\P 1631 R. H. ARRAIGNM.  \textit{Whole Creature} xvi. 280 So long he is sicke in the suds, and diseas'd in the sullens.    
\P 1633 MARMION  \textit{Fine Comp.} i. iii. B 2, They can doe no more good upon me, then a young pittifull Lover upon a Mistresse, that has the sullens.    
\P 1662 HIBBERT  \textit{Body Divinity} i. 142 Its a dangerous thing to sit sick of the sullens, or be discontented.    a 
\P 1670 HACKET  \textit{Abp. Williams} i. (1692) 84 If his Majesty were moody‥he would fetch him out of that Sullen with a pleasant Jest.    
\P 1671 WOOD  \textit{Life} (O.H.S.) II. 215 When William Lenthall was troubled with the sullins.    
\P 1679 DRYDEN  \textit{Troil. \& Cress.} iv. ii, I'll e'en go home, and shut up my doors, and die o' the sullens, like an old bird in a cage.    
\P 1747 RICHARDSON  \textit{Clarissa} (1811) I. xviii. 134 No sullens, my Mamma; no perverseness.    
\P 1819 SCOTT  \textit{Leg. Montrose} xxiii, Annot Lyle could always charm Allan out of the sullens.    
\P 1864 CARLYLE  \textit{Fredk. Gt.} xvi. viii. IV. 362 Russian Czarina evidently in the sullens against Friedrich.    
\P 1868 ‘HOLME  Lee’ \textit{B. Godfrey} xxxvi, Gerrard was in a fit of sullens.

\itembf{b.} Comb., \textbf{sullen-sick} a., ‘sick of the sullens’, ill from ill-humour.

\P 1614 T. ADAMS  \textit{Sinners Passing Bell} Wks. (1629) 247 If the state‥lie sullen-sicke of Naboths vineyard.    
\P 1650 FULLER  \textit{Pisgah} ii. vii. §7. 158 On the denyall Ahab falls sullen-sick.
\end{myenumerate}


%%%%%%%%%%%%%%%%%%%%%%%%%%%%%%%%%
\myitem{crabbed} a.

\noindent \phonetic{(ˈkræbɪd, kræbd)}

\noindent [orig. f. crab n.1 + -ed: cf. dogged. The primary reference was to the crooked or wayward gait of the crustacean, and the contradictory, perverse, and fractious disposition which this expressed. Cf. Ger. krabbe crab, whence, according to Grimm, ‘because these animals are malicious and do not easily let go what they have seized, LG. ene lütje krabbe (a little crab) a little quarrelsome ill-conditioned man (Bremen Wbch.)’; also in Saxony said of self-willed, refractory children. So E.Fris. krabbe crab, transf. a cantankerous, cross-grained man (who is refractory and froward like a crab, sticking fast or going backwards, when he ought to advance); whence krabbîg contentious, cantankerous, fractious, cross-grained (Doornkaat Koolman). Literal senses of ‘cross-grained, crooked’, and ‘knotted, gnarled, un-smooth’, applied to sticks, trees, and the like, also appear; these re-act upon the sense in which the word is applied to persons and their dispositions. In later use there is association with the fruit, giving the notion of ‘sour-tempered, morose, peevish, harsh’.]
\vspace{-0.3cm}

\begin{myenumerate}

\itembf{1.} Of persons (or their dispositions): orig. Of disagreeably froward or wayward disposition, cross-grained, ill-conditioned, perverse, contrarious, fractious. (Now blending with b.)

\P 1300  \textit{Cursor M.} 8943 (Gött.) Þe iuus þat war sua crabbid [Cott. \& Fairf. cant] and kene.    c 
\P 1440  \textit{Promp. Parv.} 99 Crabbyd, awke, or wrawe [W. wraywarde], ceronicus, bilosus, cancerinus.    c 
\P 1440  \textit{York Myst.} xxix. 130 For women are crabbed, þat comes þem of kynde.    
\P 1547 LATIMER  \textit{Serm. \& Rem.} (1845) 426 He that is so obstinate and untractable in wickedness and wrong doing, is commonly called a crabbed and froward piece.    
\P 1570 LEVINS  \textit{Manip.} 49/9 Crabbed, froward, prauus, iratus.    
\P 1643 MILTON  \textit{Divorce} Introd., The little that our Saviour could prevail‥against the crabbed textuists of his time.    
\P 1844 ALB.  SMITH \textit{Adv. Mr. Ledbury} vii. (1886) 22 Despite the persevering labours of those crabbed essayists.    a 
\P 1845 HOOD  \textit{Tale of Temper} i, Of all cross breeds of human sinners, The crabbedest are those who dress our dinners.

\itembf{b.} In later use: Cross-tempered, ill-conditioned, irritable, acrimonious, churlish; having asperity or acerbity of temper. Since 16th c. a frequent epithet of old age, in which perhaps there was at first the sense ‘crooked’; cf. sense 5. Also often influenced by, and passing insensibly into, sense 9.

\P 1535 STEWART  \textit{Cron. Scot.} II. 542 That I thairfoir crabit or cruell be.    
\P 1579 LYLY  \textit{Euphues} (Arb.) 43 To you they breed more sorrow and care‥because of your crabbed age.    
\P 1583 STUBBES  \textit{Anat. Abus.} ii. 65 He that is borne vnder Cancer, shall be crabbed and angrie, because the crab fish is so inclined.    
\P 1590 SPENSER  \textit{F.Q.} iii. ix. 3 Therein a cancred crabbed carle‥That has no skill of court nor courtesie.    
\P 1601 WEEVER  \textit{Mirr. Mart.} C j, Craft, anger, vsury, neuer seene in youth: In crabbed age these vices we behold.    
\P 1610 SHAKES.  \textit{Temp.} iii. i. 8 O She is Ten times more gentle, then her Father's crabbed; And he's compos'd of harshnesse.    
\P 1635 N. R. tr.  \textit{Camden's Hist. Eliz.} ii. xvi. 170 A man of a crabbed disposition and rash to raise commotions.    
\P 1779 F. BURNEY  \textit{Lett. Aug.}, Calling you a crabbed fellow.    
\P 1837 CARLYLE  \textit{Fr. Rev.} ii. iii. vii, His Father, the harshest of old crabbed men, he loved with warmth, with veneration.    
\P 1863 GEO. ELIOT  \textit{Romola} iii. xviii, A crabbed fellow with crutches is dangerous.    
\P 1875 JOWETT  \textit{Plato} (ed. 2) V. 302 [The] ignorant‥lays up in store for himself isolation in crabbed age.

\itembf{c.} transf. of things.

\P 1400-50 \textit{Alexander}  3794 Colwers‥\& crabbed snakis And oþire warlaȝes wild.    
\P 1634 MILTON  \textit{Comus} 477 How charming is divine Philosophy! Not harsh and crabbed, as dull fools suppose.    
\P 1682 DRYDEN  \textit{Dk. of Guise} iii. i, But if some crabbed virtue turn and pinch them, Mark me, they'll run‥and howl for mercy.

\itembf{2.} Of the temporary mood: Cross, vexed, irate, irritated; out of humour. (In early use only Sc.: now dial.; often pronounced crab'd.)

\P 1375  \textit{Sc. Leg. Saints, Laurentius} 786 Sume mene sait he crabyt is.    
\P 1513-75 \textit{Diurn.  Occurrents} (1833) 81 Quhaira he was crabbit and causit discharge the said Johne of his preitching.    
\P 1530 PALSGR. 773/2, I waxe crabbed, or angrye countenaunced. Je me rechigne.    
\P 1552 ABP. HAMILTON  \textit{Catech.} (1884) 9 It is nocht ane thing to be crabit at our brotheris persone and to be crabit at our brotheris falt.    
\P 1812 J. H. VAUX  \textit{Flash Dict., Crab'd}, affronted; out of humour; sometimes called being in Crab-street.    
\P 1861 HOLLAND  \textit{Less. Life} i. 19 A business man‥will enter his house for dinner as crabbed as a hungry bear.

\itembf{3.} Of words, actions, etc.: Proceeding from or showing an ill-tempered or irritable disposition; angry; ill-natured. Obs.

\P 1362 LANGL.  \textit{P. Pl.} A. xi. 65 For nou is vche Boye Bold‥to‥Craken aȝeyn þe Clergie Crabbede wordes.    c 
\P 1430 LYDG.  \textit{Bochas} vii. iv. (1554) 168 b, Her feminine crabbed eloquence.    
\P 1581 J. BELL  \textit{Haddon's Answ. Osor.} 277 Your crabbed and snappish accusation agaynst Luther.    a 
\P 1632 T. TAYLOR  \textit{God's Judgem.} i. ii. i. (1642) 155 He‥chased him away with bitter and crabbed reproaches.

\itembf{b.} Of the countenance: Expressing a harsh or disagreeable disposition: cf. crab-face, crab n.1 13.

\P [c1375  \textit{Sc. Leg. Saints, Vincentius} 202 Dacyane hyme-self nere wod Become‥And kest his handis to \& fra And trawit [editor reads crabbit] continence cane ma.]

\P 1603 H. CROSSE  \textit{Vertues Commw.} (1878) 51 When a crabbed visage and a misshapen body, shall stand by an amiable and louely personage.    
\P 1641  \textit{Hist. Edw. V} 6 Hard favoured of visage, such as‥is called‥among common persons, a crabbed face.

\itembf{4.} Of things: Harsh or unpleasant to the taste or feelings; unpalatable, bitter. Obs. or arch. (Cf. sense 9.)

\P 1340  \textit{Gaw. \& Gr. Knt.} 502 After crysten-masse com þe crabbed lentoun, Þat fraystez flesch wyth þe fysche \& fode more symple.    
\P 1593 \textit{Tell-Troth's  N.Y. Gift} 40 A kinde dinner and a crabbed supper.    
\P 1622 R. HAWKINS  \textit{Voy. S. Sea} (1847) 128 The crabbed entertainment it gave us.

\itembf{5.} Of trees, sticks: Crooked; having an uneven and rugged stem, gnarled, knotted; having cross-grained and knotted wood. Obs.

\P 1510 BARCLAY  \textit{Mirr. Gd. Manners} (1570) B vj, To make a streyght Jauelin of a crabbed tree.    
\P 1539 TAVERNER  \textit{Erasm. Prov.} (1552) 5 To a crabbed knotte muste be soughte a crabbed wedge.    
\P 1594 NASHE  \textit{Unfort. Trav.} 53 A crabbed briery hawthorne bush.    
\P 1675 TRAHERNE  \textit{Chr. Ethics} xxxiii. 540 A crabbed and knotty piece of matter.

\itembf{b.} of the human body and (fig.) nature.

\P 1601 DENT  \textit{Pathw. Heaven} (1831) 18 Troubled‥with a crabbed and crooked nature.    
\P 1623 COCKERAM  iii, Thersites, one that was as crabbed in person as he was Cinicall and doggish in condition.    
\P 1632 J. HAYWARD  tr. \textit{Biondi's Eromena} 16 This king‥being of a crabbed nature, pimple faced and a creple.    
\P 1799 SOUTHEY  \textit{Sonn.} xv, A wrinkled, crabbed man they picture thee, Old Winter.

\itembf{c.} Of land, weather, etc.: Rough, rugged.

\P 1579 FENTON  \textit{Guicciard.} v. (1599) 221 A crabbed mountaine, where they lost threescore men at armes and manie footmen.    
\P 1583 STANYHURST  \textit{Aeneis} iii. (Arb.) 71 God Mars the Regent of that soyle crabbed adoring [Virg. iii. 35 Geticis arvis].    
\P 1622 R. HAWKINS  \textit{Voy. S. Sea} (1847) 128 The crabbed mountains which overtopped it.    
\P 1876 ROBINSON  \textit{Whitby Gloss., Crabb'd} or Crabby. Weather terms. ‘Bits o' crabb'd showers’, the rain or sleet driven by cold winds.

\itembf{6.} Rough, rugged, and inelegant in language.

\P 1561 T. NORTON  \textit{Calvin's Inst.} i. 41 Though he be rough somtime \& crabbed in his maner of speach.    
\P 1656 COWLEY  \textit{Misc., Answ. Copy of Verses} 13 Such base, rough, crabbed, hedge Rhymes‥set the hearers Ears on Edge.

\itembf{7.} Of writings, authors, etc.: Ruggedly or perversely intricate; difficult to unravel, construe, deal with, or make sense of.

\P 1561 T. NORTON  \textit{Calvin's Inst.} iii. 310 To debarre crabbed questions.    
\P 1612 BRINSLEY  \textit{Lud. Lit.} viii. (1627) 122 The best and easiest Commentaries of the hardest and most crabbed Schoole-Authors.    
\P 1675 BAXTER  \textit{Cath. Theol.} ii. i. 2 Writing‥in crabbed Scholastick style.    1763-5 Churchill Poems, Author, O'er crabbed authors life's gay prime to waste.    
\P 1788 REID  \textit{Aristotle's Log.} iv. §6 Those crabbed geniuses made this doctrine very thorny.    
\P 1830 MACKINTOSH  \textit{Eth. Philos.} Wks. 1846 I. 179  Mr. Hume, who has translated so many of the dark and crabbed passages of Butler into his own transparent and beautiful language.    a 
\P 1839 PRAED  \textit{Poems} (1864) II. 76 Since my old crony and myself Laid crabbed Euclid on the shelf.    
\P 1890  \textit{Times} 20 Jan. 9/2 A hard, dry, and rather crabbed collection of notes and statistics.

\itembf{b.} Of handwriting: Difficult to decipher from the bad formation of the characters.

\P 1612 DEKKER  If it be not good Wks. 1873 III.  287 Lawes Wrap'd vp in caracters, crabbed and vnknowne.    
\P 1800 E. HERVEY  \textit{Mourtray Fam.} I. 91 It is such a crabbed hand, I can't read half of it.    
\P 1853 FARADAY in  B. Jones \textit{Life} (1870) II. 318 Do you see how crabbed my hand-writing has become?    
\P 1879 F. HARRISON  \textit{Choice Bks.} (1886) 18 A few worn rolls of crabbed manuscript.

\itembf{8.} Of or pertaining to the zodiacal sign Cancer. Obs. rare.

\P 1634 SIR T. HERBERT  \textit{Trav.} 43 Muskat is a citie‥upon the Persian Gulfe and almost Nadyr to the crabbed Tropique.

\itembf{9.} Of the nature of the crab-tree or its fruit; fig. sour-tempered, peevish, morose; harsh.

\P 1565-73 COOPER  \textit{Thesaurus} s.v. Acerbus, Vultus acerbus, sower or crabbed.    
\P 1599 MARSTON  \textit{Sco. Villanie} 170 Against the veriuice-face of the Crabbedst Satyrist that euer stuttered.    
\P 1611 SHAKES.  \textit{Wint. T.} i. ii. 102 Three crabbed Moneths had sowr'd themselues to death.    
\P 1656 DUCHESS  OF NEWCASTLE in \textit{Life of Dk.} (1886) 313 As for my disposition, it is‥not crabbed or peevishly melancholy.    
\P 1726 AMHERST  \textit{Terræ Fil.} xxxvi. 189 This philosophical apple-tree‥never grew kindly, nor produced any thing but sour crabbed stuff.    
\P 1865 HOLLAND  \textit{Plain T.} iii. 107 Only treated respectfully by wives and children because they are crabbed and sour.

\itembf{10.} Comb., as crabbed-looking, crabbed-handed adjs.

\P 1837 SIR F. PALGRAVE  \textit{Merch. \& Friar} i. (1844) 34 A lean-visaged, crabbed-looking personage.    
\P 1848 THACKERAY  \textit{Van. Fair} xliii, That crabbed-handed absent relative.
\end{myenumerate}


%%%%%%%%%%%%%%%%%%%%%%%%%%%%%%%%%
\myitem{venomous} a.

\noindent \phonetic{(ˈvɛnəməs)}

\noindent [a. AF. venimus, venimous, = OF. (also mod.F.) venimeux, f. venim venom n., after L. venēnōsus: see venenous a.]
\vspace{-0.3cm}

\begin{myenumerate}

\itembf{1.} fig. Morally or spiritually hurtful or injurious; pernicious. Obs.

\P 1290 \textit{S. Eng.  Leg.} I. 120/484 Þat word me þinchez venimous to þe pays of þe londe.    a 
\P 1340 HAMPOLE  \textit{Psalter} cxlix. 2 To forsake þe venymous delitis of þis warld.    c 
\P 1380 WYCLIF  \textit{Sel. Wks.} III. 20 Venemouse lustis and likingis of deedly synnes.    c 
\P 1480 HENRYSON  \textit{Fables, Cock \& Fox} 606 (Harl. MS.), Thir twa sinnis, flatterie and vane gloir, Ar vennomous.    c 
\P 1490 CAXTON  \textit{Rule St. Benet} (E.E.T.S.) 129 Yf ony be founde gylty in this venemouse offence of properte.    
\P 1526  \textit{Pilgr. Perf.} (W. de W. 1531) 55 The religyous seruaunt of god‥destroyeth by holy meditacyon ye flyes \& spyders of venymous thoughtes.    
\P 1580 LYLY  \textit{Euphues} (Arb.) 414, I will at large proue that there is nothing in loue more venemous then meeting.    
\P 1610 HOLLAND  \textit{Camden's Brit.} 707 Saint German, who happily confuted that venemous Pelagian Heresie.

\itembf{2.} Containing, consisting or full of, infected with, venom; possessing poisonous properties or qualities; destructive of, harmful or injurious to, life on this account.

Common from c1470 to c1650; now rare.

\P 1330 R. BRUNNE  \textit{Chron. Wace} (Rolls) 16594 By passagers wel herde he seye Þe venimouse eyr was al a-weye.    c 
\P 1340 HAMPOLE  \textit{Pr. Consc.} 6751 Another manere of drynk þat es ille, Þat sal be bitter and venemus.    c 
\P 1366 CHAUCER  \textit{A.B.C.} 149 With thornes venymous, O heuene queen,‥I am wounded.    
\P 1474 CAXTON  \textit{Chesse} iii. v. (1883) 126 That they put in theyr medicynes no thynge venemous.    
\P 1490 \textit{Eneydos} xxiv. 88 Herbes‥wherof the Iuse is passyng venymouse.    
\P 1555 EDEN  \textit{Decades} (Arb.) 45 Of the venemous apples wherwith the Canibales inueneme theyr arrowes.    
\P 1584 COGAN  \textit{Haven Health} ccxliii. (1636) 297 Not that the ayre is venomous of it selfe, but through corruption hath now gotten such a quality.    c 
\P 1614 SIR W. MURE  \textit{Dido \& Æneas} iii. 108 Collecting als‥The milkie poyson of each ven'mowse weed.    
\P 1651 HOBBES  \textit{Leviath.} ii. xxix. 173 The fleshy parts being‥by venomous matter obstructed.    
\P 1672 MARVELL  \textit{Reh. Transp.} i. 132 The cultivating of a Garden of venimous Plants.    
\P 1817 SHELLEY  \textit{Rev. Islam} x. xxxviii, On the heap Pour venomous gums.    a 
\P 1839 PRAED  \textit{Red Fisherman} Poems 1864 I. 197  The trees and herbs that round it grew Were venomous and foul.

\itembf{b.} Of a wound, etc.: Marked or characterized by the presence of poisonous matter; foul with venom; envenomed. Obs.

\P 1398 TREVISA  \textit{Barth. De P.R.} xix. lvii, Aȝens þe venemos posteme þat hatte antrax \& aȝens oþer venemous postemes.    
\P 1541 R. COPLAND  \textit{Guydon's Form.} U j, It shulde be an oyntment profitable to all sores that be venymous.    c 
\P 1550 H. LLOYD  \textit{Treasury Health} T v, Leuen of whete breketh the venemouse humors and apostumes.    
\P 1656 J. SMITH  \textit{Pract. Physic} 363 A wound made by bullets is not venemous, nor alwaies bruised.    
\P 1702 ECHARD  \textit{Eccl. Hist.} i. i. 36 His Distemper daily encreas'd,‥and he himself labour'd under‥venomous Swellings in his Feet,‥accompany'd with intolerable Smells.    
\P 1707 WATTS  \textit{Hymns} ii. cliii. Poet. Wks. IV. 148 Sin like a venomous disease Infects our vital blood.    
\P 1774 GOLDSM.  \textit{Nat. Hist.} VII. ix. 196 When the serpent is irritated to give a venomous wound.

fig. \P 1597 HOOKER  \textit{Eccl. Pol.} v. lii. (1611) 292 A soueraigne preseruatiue‥from the venemous infection of heresie.

\itembf{c.} Of a bite or sting.

\P 1567 \textit{Gude \& Godlie  Ball.} (S.T.S.) 81 He ouerthrew The Serpent, and his vennemous stang.    
\P 1653 WALTON  \textit{Angler} 146 The biting of a Pike is venemous and hard to be cured.    
\P 1753 J. BARTLET  \textit{Gentl. Farriery} 322 Of Venomous Bites from Vipers and Mad Dogs.    
\P 1787 BEST  \textit{Angling} (ed. 2) 48 Be careful how you take a pike out of the water, for his bite is venomous.

\itembf{d.} Harmful or injurious to something. Obs.

\P 1607 SHAKES.  \textit{Cor.} iv. i. 23 Thy teares are salter then a yonger mans, And venomous to thine eyes.    
\P 1691 T. H[ALE]  \textit{Acc. New Invent.} 17 A Cancarous and Corroding substance, and venomous to Iron.

\itembf{3.} Of animals, esp. snakes, or their parts: Secreting venom; having the power or property of communicating venom by means of bites or stings; inflicting or capable of inflicting poisonous wounds in this way.

   Formerly in general literary use, now chiefly restricted to certain species of poisonous snakes.

α \P 1375  \textit{Sc. Leg. Saints} xxxi. (Eugenia) 396 Na serpent has a hed sa fel, sa venamuse, na sa cruel, as þe hed of þe colubre is.    
\P 1387 TREVISA  \textit{Higden} (Rolls) I. 51 Yuel doers, corrupte ayre, wylde bestes and venemous woneþ þerynne.    c 
\P 1400 MANDEVILLE  (1839) 199 Thanne have thei no drede of no Cocodrilles, ne of non other venymous Vermyn.    c 
\P 1450 J. METHAM  \textit{Wks.} (E.E.T.S.) 46 For off summe off thise serpentys, the eyn so venymmus be That with her loke thei slee yche erthly creature.    
\P 1480 CAXTON  \textit{Myrr.} ii. xiv. 97 Irland is a grett Ilonde in whiche is no serpent ne venemous beeste.    
\P 1522 MORE  \textit{De quat. Noviss.} Wks. 85/1 Like as the venemous spider bringeth forth her cobweb.    
\P 1596 SPENSER  \textit{F.Q.} vi. vi. 9 That beastes teeth, which‥Are so exceeding venemous and keene.    
\P 1600 SHAKES.  \textit{A.Y.L.} ii. i. 13 Aduersitie Which like the toad, ougly and venemous, Weares yet a precious Iewell in his head.    
\P 1653 W. RAMESEY  \textit{Astrol. Restored} 229 Those places subject thereunto shall be afflicted with water, and venemous Creatures.    
\P 1748  \textit{Anson's Voy.} iii. ii. 314 We found‥scorpions, which we supposed were venemous.    1791-3 in Spirit Public Jrnls. (1799) I. 225 To sleep in a dungeon with venemous reptiles.

β \P 1515 HENRYSON'S  \textit{Orpheus \& Eurydice} (Asloan MS.) 105 As scho ran, all bairfut, in ane bus Scho trampit on a serpent wennomus.    
\P 1595 \textit{Locrine}  i. i. 76 Triple Cerberus with his venomous throte.    
\P 1651 WITTIE tr.  \textit{Primrose's Pop. Err.} iv. xxxviii. 271 If poyson, or some venomous creature be neare unto it, it sweats.    
\P 1671 SALMON  \textit{Syn. Med.} iii. xxii. 442 It‥cures the bitings of venomous beasts.    
\P 1713 DERHAM  \textit{Phys.-Theol.} ii. vi. 56 Many‥of our European venomous animals carry their Cure‥in their own Bodies.    
\P 1774 GOLDSM.  \textit{Nat. Hist.} VII. ix. 194 If it [sc. the serpent] has the fang teeth, it is to be placed among the venomous class.    
\P 1834 MCMURTRIE  \textit{Cuvier's Anim. King.} 182 Serpents are divided into venomous and non-venomous; and the former are sub-divided into such as are venomous with several maxillary teeth, and those which are venomous with insulated fangs.    
\P 1876 M. E. BRADDON  \textit{J. Haggard's Dau.} III. 23 The serpent had lifted his venomous crest from among the flowers.    c 
\P 1880  \textit{Cassell's Nat. Hist.} IV. 301 The poisonous Snakes are divided into two groups—the Viperiform Snakes and the Venomous Colubrines.

\itembf{b.} fig., chiefly with allusion to the Devil.

\P 1340  \textit{Ayenb.} 171 Þe uenimouse eddre of helle.    c 
\P 1450 \textit{Mankind} 40 in  \textit{Macro Plays} 2 Yt hath dyssoluyde mankynde from þe bittur bonde Of þe mortall enmye, þat vemynousse serpente.    a 
\P 1548 HALL  \textit{Chron., Hen. IV}, 25 The Earle of Northumberland‥bare still a venemous scorpion in his cankered heart.    
\P Ibid., \textit{Hen. VI}, 169 That venemous worme, that dreadfull dragon, called disdain of superioritie.    a 
\P 1578 LINDESAY  (Pitscottie) \textit{Chron. Scot.} (S.T.S.) II. 239 The Devill,‥that wicked and venimus serpent quho gois about to sie quhome he may catch.

\itembf{4.} fig. Having the virulence of venom; rancorous, spiteful, malignant, virulent; embittered, envenomed.

\P 1340 HAMPOLE  \textit{Psalter} x. 2 Þai haf redy in þaire hertis venymouse wordis and sharpe.    Ibid. xxviii. 8 Þaim‥þat‥puttis away venomus tongis.    
\P 1340  \textit{Ayenb.} 27 Þe venimouse herte of þe enuiouse zeneȝeþ generalliche.    c 
\P 1400  \textit{Rom. Rose} 5528 With tonge woundyng, as feloun, Thurgh venemous detraccioun.    a 
\P 1450  \textit{Knt. de la Tour} (1868) 56 It is not good to‥take sodeyne acqueintaunce that hathe the herte of faire speche, for sum tyme her speche is deseyuable and venomous.    c 
\P 1489 CAXTON  \textit{Blanchardyn} li. 196 The venymouse malyce of the false traytoure Subyon.    
\P 1555 EDEN  \textit{Decades} (Arb.) 52 To speake venemous woordes‥ageynst the annoynted of god.    
\P 1588 SHAKES.  \textit{Tit. A.} v. iii. 13 The Venemous Mallice of my swelling heart.    
\P 1648 HEXHAM II, \textit{Feenijnighlick}, venommously, spightfully, or [with] a venomous envy.    a 
\P 1721 PRIOR  \textit{Session of Poets} 36 That with very much Wit he no anger exprest Nor sharpen'd his Verse with a Venemous Jest.    
\P 1737  \textit{Gentl. Mag.} VII. 623/2 One R. C.‥sent me venemous Libels against the Great Man.    
\P 1857 PALGRAVE  \textit{Hist. Normandy \& Eng.} II. 18 A venomous opposition was festering against him.    
\P 1879 FROUDE  \textit{Cæsar} xii. 153 The most innocent intimacies would not have escaped misrepresentation from the venomous tongues of Roman society.    
\P 1885  \textit{Manch. Exam.} 20 May 4/7 A venomous and scurrilous attack.

\itembf{b.} Of persons, their character, etc.

\P 1400  \textit{Morte Arth.} 299 Of this grett velany I salle be vengede ones On ȝone venemus mene, wyth valiant knyghtes!    
\P 1567 \textit{Satir.  Poems Reform.} iv. 109 O wickit wemen, vennomus of nature!    
\P 1579 TOMSON  \textit{Calvin's Serm. Tim.} 901/2 What shall men say, when a mortall man dareth thus to become venemous against God.    
\P 1585 T. WASHINGTON tr. \textit{Nicholay's Voy.} iii. ii. 71 [Of these] christian children Mahometised, the venemous nature is so great, mischieuous and pernitious.    
\P 1607 HIERON  \textit{Wks.} I. 225 [Satan is] a venimous aduersary to empoyson our soule.    
\P 1643 SIR T. BROWNE  \textit{Relig. Med.} ii. §10 There are in the most depraved and venemous dispositions, certaine pieces that remaine untoucht.    
\P 1882 J. H. BLUNT  \textit{Ref. Ch. Eng.} II. 244 His most bitter enemy, the venomous and unscrupulous Foxe.    
\P 1911  \textit{Blackw. Mag.} Aug. 221 The doctor seemed to me a venomous little creature.

\itembf{5.} Treated with venom or poison; envenomed, poisoned. Obs.

\P 1400  \textit{Morte Arth.} 2570 With the venymous swerde a vayne has he towchede.
\P c1400 \textit{Pilgr.  Sowle} i. i. (1859) 1 Thenne comme cruel dethe and smote me with his venemous darte.
\P a1470 HARDING  \textit{Chron.} ii. cxxxix, Kyng Rychard‥Was hurt right ther, with dartes venemous.    
\P 1555 EDEN  \textit{Decades} (Arb.) 107 These people also, vse bowes and venemous arrowes.    
\P 1578 LYTE  \textit{Dodoens} 305 It is good against‥venimous shot of dartes and arrowes.    
\P 1631 GOUGE  \textit{God's Arrows} Ded. p. ix, How farre the venime thereof (for it is a venimous arrow) may infect, who knowes?

\itembf{6.} Of or pertaining to, of the nature of, venom.

\P 1425 WYNTOUN  \textit{Cron.} viii. clviii. 3135 Þai thoucht to gere Him with sum venamus poisoun Be distroyit.    
\P 1604 JAS. I  \textit{Counterbl. to Tobacco} (Arb.) 103 Tobacco‥hath a certaine venemous facultie ioyned with the heate thereof.    
\P 1650 BULWER  \textit{Anthropomet.} 159 There being a venemous quality in the paint.    
\P 1675 J. OWEN  \textit{Indwelling Sin} vi. (1732) 50 It is in the Heart like Poison, that hath nothing to allay its venemous Qualities, and so infects whatever it touches.    
\P 1774 GOLDSM.  \textit{Nat. Hist.} VII. ix. 195 The glands that serve to fabricate this venomous fluid.    
\P 1826 MISS MITFORD  \textit{Village Ser.} ii. (1863) 417 It has a fine venomous smell,‥and will certainly when stilled be good for something or other.    
\P 1887 A. M. BROWN  \textit{Anim. Alkaloids} 2 Gaspard and Stick‥had detected a venomous principle in cadaverous extracts.

fig. \P 1572 PERRY in  \textit{Strype Eccl. Mem.} (1721) III. 363 The God of Truth defend you‥from the venomous Poyson of Lyars.    
\P 1596 DALRYMPLE tr.  \textit{Leslie's Hist. Scot.} II. 41 Lyk a traytour he steilis in, that‥he may saw his venumous poyson.    
\P 1866 C. J. VAUGHAN  \textit{Plain Words} i. 10 The personal sins of each one of us‥eating like a venomous poison into his soul.

\itembf{7.} Comb. in venomous-hearted, venomous-looking adjs.

\P 1740 RICHARDSON  \textit{Pamela} (1824) I. xv. 256 Several innocent creatures, might have been entangled‥in the ensnaring web of this venomous-hearted spider.    
\P 1899 F. T. BULLEN  \textit{Way Navy} 65 We sighted the enemy in the shape of one of those venomous-looking four-funnelled destroyers.
\end{myenumerate}


%%%%%%%%%%%%%%%%%%%%%%%%%%%%%%%%%
\myitem{wroth} a.

\noindent \phonetic{(rəʊθ, rɒθ)}

\noindent [OE. wráþ, = OFris. wrêth evil, OS. wrêđ (MLG. wrede, wrêt, LG. wrêd), MDu. wrêt, wreet (Du. and Flem. wreed cruel), OHG. reid, reidi (MHG. reit, reide curled, twisted), ON. *wreiðr, reiðr (Norw. vreid, reid, Da. and Sw. vred) angry, offended, f. the pa. tense of wríðan to writhe. Cf. wrath a.

   In very freq. use c 1250-c 1450. Rare (exc. in or after Biblical usage), c 1530-c 1850, being regarded as ‘out of use’ by Johnson, ‘nearly obsolete’ by Ash, but as ‘an excellent word and not obsolete’ by Webster (1828-32). Revived in sense 1, esp. in formal or dignified style, c 1800.]
\vspace{-0.3cm}

\begin{myenumerate}
\itembf{1.} Stirred to wrath; moved or exasperated to ire or indignation; very angry or indignant; wrathful, incensed, irate.

   Rarely attrib., as in quots. a 1225, 1375, c 1400.

α \P c950 \textit{Lindisf. Gosp. Matt.} xxii. 7 \phonetic{Ðe cyniᴁ uutedlice mið ðy ᴁeherde wurað wæs}.
\P 1000 \textit{Genesis}  2260 Ða wearð unbliðe Abrahames cwen, hire worcþeowe wrað on mode.
\P a1122  \textit{O.E. Chron.} (Laud MS.) an. 1066, \phonetic{Þa þe cyng Willelm ᴁeherde þæt secgen þa wearð he swiðe wrað}.
\P c1175 \textit{Lamb.  Hom.} 15 Ne beo þu nefre ene wrað þer fore.
\P c1200 ORMIN  19603 And ta warrþ wraþ Herode.
\P c1205 LAY.  8268 Þa wes he wræð ful iwis.    
\P Ibid. 28723 Þus þe king wordede, wræð on his þonke.
\P a1300  \textit{Cursor M.} 1599 Þof he was wrath it was na wrang.    
\P 1375 BARBOUR  \textit{Bruce} xvi. 245 Micht no man se ane vrathar man.
\P c1400 \textit{Rule  St. Benet} (Prose) 1 He, as a wrath fader,‥deseret vs os not hys sons.
\P c1450  \textit{Merlin} i. 18 Tho gan the Iuge to be right wrath.
\P c1475 \textit{Rauf  Coilȝear} 100 The Carll‥wox wonder wraith.
\P c1520 M. NISBET  \textit{Ephes.} iv. 26 Be ye wrathe, and will ye nocht do synn.
\P c1560 A. SCOTT  \textit{Poems} (S.T.S.) vi. 38 For be scho wreth I will not wow it.    
\P 1590 J. BUREL in  \textit{Watson Coll.} ii. (1709) 2 Anna, wondrous wraith, Deplors hir sister Didos daith.
\P a1776 \textit{Lord Ingram} in CHILD  \textit{Ballads} II. 131/2 A' was blyth at Auld Ingram's cuming, But Lady Maisdrey was wraith.

β \P c1200  \textit{Trin. Coll. Hom.} 183 Al þat me was leof, hit was þe loð; þu ware a sele ȝief ich was wroð.
\P a1225  \textit{Ancr. R.} 120 Wroð mon is he wod? 
\P c1290 \textit{Becket}  413 Þo was þe king wel of i-nouȝ, wroþere þane he was er.    
\P 13‥ \textit{Cursor M.} 4889 (Gött.), If he it wit he wil be wroght [Trin. wrooþ].    
\P 1398 TREVISA  \textit{Barth. De P.R.} v. xli. (BM. Addit. MS.), By þe galle we ben wrooþ, by þe herte we ben wys.
\P c1450  \textit{Knt. de la Tour} (1906) 22 Thanne she was wrother thanne afore.
\P c1489 CAXTON  \textit{Sonnes of Aymon} iii. 113 Sire,‥ye be wroth of som other thyng.
\P 1526 TINDAL  \textit{Matt.} xxii. 7 When the kyng hearde that, he was wroth. 
\P 1548 UDALL, etc. \textit{Erasm. Par. Mark} x. 65 For he was nether wroth, nor murmured against Christ. 
\P a1599 SPENSER  \textit{F.Q.} vii. vi. 35 There-at Ioue wexed wroth.    
\P 1611 BIBLE  \textit{1 Sam.} xx. 7 If he be very wroth,‥euill is determined by him.    
\P 1656 BLOUNT \textit{Glossogr}.    
\P 1716 M. DAVIES  \textit{Athen. Brit.} III. 25 Our modern Dissenters seem wroth, when they are deem'd a vulgar‥kind of People.    
\P 1749 FIELDING  \textit{Tom Jones} vi. ix, The parson‥saying, ‘You behold, Sir, how he waxeth wroth at your abode here’.    
\P 1820 WORDSW.  \textit{‘A Book came forth’} 7 But some‥Waxed wroth, and with foul claws‥On Bard and Hero clamorously fell.    
\P 1842 TENNYSON  \textit{Dora} 23 Then the old man Was wroth, and doubled up his hands.    
\P 1852 DICKENS  \textit{Bleak Ho.} xl, Sir Leicester is majestically wroth.    
\P 1880 BLACKMORE  \textit{Mary Anerley} xxxiii, ‘I know it,’ said Carroway, too wroth to swear.

\noindent absol. \P a1250 \textit{Owl \& Night.}  944 Selde endeþ wel þe loþe \& selde playdeþ wel þe wroþe.

\noindent transf. \P c1386 CHAUCER  \textit{Cook's T.} 34 Reuel and trouthe‥been ful wrothe al day as men may see.

\itembf{b.} Said of the Deity.

\P a1100 in  \textit{Earle Land-Charters} (1888) 253 Crist‥him wurðe wrað þe hi hæfre \phonetic{ᴁeþywie}.
\P a1300  \textit{Cursor M.} 959 Wa es me! lauerd,‥þat euer i mad þe wrath.
\P c1340 HAMPOLE  \textit{Pr. Consc.} 5479 When he es wrathe þat es maker of alle.
\P c1386 CHAUCER  \textit{Pars. T.} \phonetic{⁋}96 Ther shal the‥wrothe Iuge sitte aboue.    
\P 1393 LANGL.  \textit{P. Pl.} C. i. 117 God was wel þe wroþer.
\P a1450 \textit{Mirk's  Festial} i. 4 Aboue hym schall be Crist his domes-man so wroþe, þat [etc.].    
\P 1533 BELLENDEN  \textit{Livy} (S.T.S.) I. 106 The goddis war sa commovit and wraith, þat [etc.].    
\P 1611 BIBLE  \textit{Isaiah} lxiv. 9 Be not wroth very sore, O Lord.    
\P 1697 DRYDEN \textit{Æneis} v. 1110 The  God was wroth.    
\P 1820 KEATS  \textit{Hyperion} ii. 351 He saw full many a God Wroth as himself.    
\P 1877 TENNYSON  \textit{Harold} i. i. 28 Why should not Heaven be wroth?

\itembf{c.} With dative, or const. with preps., as against, at, on, to, toward, upon, or esp. with.

(a) \P a1000 \textit{Genesis}  405 Þonne weorð he him wrað on mode.
\P c1000  \textit{Ags. Ps.} (Thorpe) lxxxiv. 4 Þæt ðu us ne weorðe wrað on mode.
\P c1200 ORMIN  4814 Forr whatt iss Drihhtin me þuss wraþ?
\P c1230 \textit{Hali  Meid.} 31 Beo hit nu, þat‥ti were beo þe wrað.

(b) \P 1175 \textit{Lamb.  Hom.} 117 Þi les ðe god iwurðe wrað wið eou. 
\P c1205 LAY.  6369 A-nan se he wes wrað wid eni.    
\P 1297 R. GLOUC.  (Rolls) 570 Corineus‥wroþ inou was Toward þe king lotrin.    
\P 1303 R. BRUNNE  \textit{Handl. Synne} 12293 Al tymes ys God more wroþer with þys Þan [etc.].
\P a1352 MINOT  \textit{Poems} iii. 5 For mani men to him er wroth.    
\P 1375 BARBOUR  \textit{Bruce} i. 201 Gyff ony thar-at war wrath.    
\P 1388 WYCLIF  \textit{Num.} xxiv. 10 Balaach was wrooth aȝens Balaam.    
\P 1412 \textit{26 Pol.  Poems} 47 First whan god wiþ man was wroþ.    
\P 1471 CAXTON  \textit{Recuyell} (Sommer) 535 Dyane‥was wrothe and angry vpon them.
\P c1489 \textit{Sonnes of Aymon} i. 50 Charlemayne‥was wrothe to theym.    
\P 1535 COVERDALE  \textit{2 Chron.} xxviii. 9 The Lorde God‥is wroth at Iuda.    
\P 1590 SPENSER  \textit{F.Q.} iii. vi. 19 She‥woxe halfe wroth against her damzels slacke.    
\P Ibid. vii. 8 Be not wroth With silly Virgin.    
\P 1611 BIBLE  \textit{Ps.} lxxxix. 38 Thou hast bene wroth with thine anointed.    
\P 1794 MRS. RADCLIFFE  \textit{Myst. Udolpho} xxv, The signor, it seems, had lately been very wroth against her.    
\P 1859 TENNYSON  \textit{Elaine} 160 Then got Sir Lancelot suddenly to horse, Wroth at himself.    
\P 1873 ‘OUIDA’  \textit{Pascarel} I. 39 She, dear soul, was very wroth against him always.    
\P 1883 WHITELAW \textit{Sophocles, Antigone} 1177 Wroth  with his pitiless sire, he slew himself.

\noindent fig. \P 1300  \textit{Cursor M.} 30 Þe wrang to here o right is lath, And pride wyt buxsumnes is wrath.

\itembf{2.} Marked or characterized by anger or wrath; indicative of ire or indignation. Obs.

\P c1000  \textit{Ags. Ps.} (Thorpe) lxiii. 4 Hi‥hi mid wraðum wordum trymmað.
\P a1300 E.E.  \textit{Psalter} lxxiii. 1 Wrathe es þi breth, ouer schepe of þi fode.
\P a1325 \textit{Prose  Psalter} cxxiii. 3 Her wodeship was wroþe oȝains us.    
\P 13‥ \textit{Gaw. \& Gr. Knt.} 1706 Þay  sued hym [sc. a fox] fast, Wreȝande hym ful weterly with a wroth noyse.
\P c1375  \textit{Cursor M.} 828 (Fairf.), Sone bigan veniaunce to kithe, al was wraþ þat er was blithe.    
\P 1582 STANYHURST  \textit{Æneis} i. (Arb.) 22 Wroth woords statelye thus [he] vsed.    
\P 1648 J. BEAUMONT  \textit{Psyche} xii. xxxiii, Wroth fiery Knots are marshalled upon Her Forehead.

\itembf{3.} Of a fierce, savage, or violent disposition or character; stern, truculent. Obs.

\P 1000  \textit{Ags. Ps.} (Thorpe) lxvii. 5 Þa þe wydewum syn wraðe æt dome.
\P c1205 LAY.  18583 Þis iherde Gorlois‥\& he andsware ȝaf, eorlene wraðest.    
\P Ibid. 28503 Arður þat iherde, wraðest kinge.
\P c1275  \textit{Ibid.} 6402 Þar was mani bold Brut, and mani cnihtes wroþe [c1205 bisi kempen].

\itembf{b.} In the phrase as wroth as (the) wind. Obs.

\P 1377 LANGL.  \textit{P. Pl.} B. iii. 328 Also wroth as þe wynde Wex Mede in a while.
\P c1400  \textit{Destr. Troy} 13091 And he [was] wrothe as the wynde to his wale eme.    
\P 14‥ Erthe upon Erthe 33/48 Erthe is as sone wroth as is the wynde.
\P c1470 \textit{Gol. \& Gaw.}  770 Golograse‥, Wod wraith as the wynd, his handis can wryng.

\itembf{4.} Of animals: Of a violent or fierce nature; irritated, enraged. Obs.

\P a900 CYNEWULF \textit{Crist} 1548 Se deopa  seað‥æleð hy mid þy ealdan liᴁe‥, wraþum wyrmum.
\P a1250 \textit{Owl \& Night.} 1043 Þe  vle wes wroþ, to cheste rad, Mid þisse worde hire eyen abraid.    
\P 13‥ \textit{E.E. Allit. P.} B. 1676 þou‥on  mor most abide‥With wroþe wolfes to won.
\P c1375  \textit{Sc. Leg. Saints} i. (Peter) 523 Þan wes þe hound na thing wrath, Na schup to do na man schath.
\P a1400-50  \textit{Wars Alex.} 738 As wrath as a waspe.    
\P 1526 TINDALE  \textit{Rev.} xii. 17 The dragon was wroth with the woman.

\itembf{b.} transf. Of the wind, sea, etc.: Moved to a state of turmoil or commotion; violent, stormy.

\P 13‥ \textit{E.E. Allit. P.} C. 162 Euer was ilyche loud þe lot of þe wyndes, \& euer wroþer þe water, \& wodder þe stremes.    
\P 13‥ \textit{Gaw. \& Gr. Knt.} 525 Wroþe wynde of þe welkyn wrastelez with þe sunne.  
\P 13‥, etc. [see 3 b].   
\P 1426 AUDELAY  \textit{Poems} 47 Wry not fro Godis word as the wroth wynd.    
\P 1590 SPENSER  \textit{F.Q.} ii. xi. 19 When the wroth Western wind does reaue their locks.    
\P 1835 BROWNING  \textit{Paracelsus} v. 661 The wroth sea's waves are edged With foam.    
\P 1852 C. B. MANSFIELD  \textit{Paraguay, etc.} (1856) 123 It rained heavily.‥ So I was wroth, and the weather too.    
\P 1876 SWINBURNE \textit{Erechtheus} 1649 The  most holy heart of the deep sea, Late wroth, now full of quiet.

\itembf{5.} Bad, evil; grievous, perverse. Obs.

   In later use in to wrothe hele, wroth-haile (see wrother-heal).

\P 1000  \textit{Ags. Ps.} (Thorpe) cxviii. 101 \phonetic{Ic minum fotum fæcne siðas, þa wraþan weᴁas, werede ᴁeorne}.
\P a1023 WULFSTAN  \textit{Hom.} l. (1883) 273 Hu læne and hu lyðre þis lif is,‥hu tealt and hu wrað.
\P a1225 \textit{Juliana}  57 Weila as þu were iboren wrecche o wraðe [v.r. wraðer] time.
\P a1225  \textit{Leg. Kath.} 171 Þe wrecches þet ha seh‥wraðe werkes wurchen.
\P a1250  \textit{Prov. Alfred} 115 Þenne beoþ his wene ful wroþe isene.    
\P 1297 R. GLOUC.  (Rolls) 3019 To wroþe hele al þis lond was he so milde þo.
\P c1330 \textit{King  of Tars} 131 To wrothe hele that he was bore.
\P c1400 \textit{Laud  Troy Bk.} 7872 That was him to wrothe-haile: For thei of Grece opon him throng.

\itembf{6.} Displeased, grieved; sorrowful, sad. Obs.

\P 13‥ \textit{K. Alis.} 4528 (Laud MS.), Alisaunder haþ vnderstonde Þe lettre þat com from darries sonde. Wroþ he was, \& hadde pyte.    
\P 13‥ \textit{Gaw. \& Gr. Knt.} 70 Ladies laȝed ful loude, þoȝ þay lost haden, And he þat wan was not wrothe.    
\P 1450 \textit{Ludus  Coventriæ} 329 Lombe of love with-owt loth, I ffynde þe not, myn hert is wroth.

\itembf{b.} Fearful, apprehensive, afraid. Obs. rare—1.

\P 13‥ \textit{K. Alis.} 544 (Laud MS.), Vche of hem so bycom wrooþ: For a dragon þer com in fleen.
\end{myenumerate}


%%%%%%%%%%%%%%%%%%%%%%%%%%%%%%%%%
\myitem{urbane} a.

\noindent \phonetic{(ɜːˈbeɪn)}

\noindent [ad. F. urbain (14th c.), or L. urbān-us urban a. For the difference, in form and stress, between urban and urbane, cf. human and humane.]
\vspace{-0.3cm}

\begin{myenumerate}

\itembf{1.} Of or pertaining to, characteristic of or peculiar to, a town or city. Now arch. or Obs.

\P 1533 BELLENDEN  \textit{Livy} i. xx. (S.T.S.) I. 114 Siclike vrbane \& civil laubouris.    
\P Ibid. v. v. II. 161 Thus had al þe romane tentis almaist bene replete of seditioun vrbane.    
\P 1570 LEVINS  \textit{Manip.} 19 Vrbane, vrbanus.    
\P 1607 R. C[AREW] tr. \textit{Estienne's World Wond.} 233 They see greater cunning and dexteritie, and a more ciuill and vrbane kind of life.    
\P 1681 STAIR  \textit{Inst. Law Scot.} xvii. 343 Negative Urbane Servitudes, do chiefly concern the light view or prospect of Tenements.    
\P 1788 \textit{Trifler}  No. 26. 344 In the simple beauty of the country the once wealthy merchant of Bassora lost the recollection of urbane magnificence.    
\P 1809-14 WORDSW.  \textit{Excurs.} viii. 71 A poor brotherhood who walk the earth,‥Raising‥savage life To rustic, and the rustic to urbane.

\itembf{b.} Exercising jurisdiction over, dwelling or residing in, a town or city. Obs.

\P 1651 HOWELL  \textit{Venice} 16 Among the Urbane or Cittie Magistrats the Judges are rankd.    
\P 1652 GAULE  \textit{Magastrom.} 373 M. Æmilius, the urbane prætor.    
\P 1658 J. HARRINGTON  \textit{Oceana} Introd. B j b, The Urbane Tribes of Rome consisting of the Turbaforensis [etc.].    
\P 1681 H. NEVILE  \textit{Plato Rediv.} 61 The Rustik Tribes being twenty seven, and the Vrbane nine.

\itembf{c.} Following the pursuits, having the ideas or sentiments, characteristic of town or city life.

\P 1698 FRYER  \textit{Acc. E. Ind. \& P.} 54 The Citizens are urbane, being trained up to Commerce.    
\P 1870 LOWELL  \textit{Study Wind.} (1871) 177 The same combination of circumstances produced Béranger, an urbane or city poet.

\itembf{2.} Having the manners, refinement, or polish regarded as characteristic of a town; courteous, civil; also, blandly polite, suave.

\P 1623 COCKERAM I, \textit{Vrbane}, ciuill, courteous.    
\P 1656 BLOUNT  \textit{Glossogr.}, Urbane,‥civil in curtesie,‥pleasant in behaviour and talk.    
\P 1796 T. HOLCROFT tr. \textit{Stolberg's Trav.} lxii. I. 483 The urbane youth‥gave due praise to the country of Menelaus.    
\P 1827 LYTTON  \textit{Pelham} xv, We took advantage of our acquaintance with the urbane Frenchman to join his party.    
\P 1873 DIXON  \textit{Two Queens} IV. 139 In Eustace Chapuys, master of requests, he had a man of law,‥urbane, alert, unscrupulous.    
\P 1882 STEVENSON  \textit{Mem. \& Portr.} xi. (1887) 170, I feel never quite sure of your urbane and smiling coteries.

\itembf{b.} Characterized by urbanity, courtesy, or politeness.

\P 1679 MARG. MASON  \textit{Tickler Tickled} 2 To treat a Lady of Mrs. Ellen Rigby's Quality, with the name of Bitch-Fox,‥is not at all Urbane.    
\P 1800 W. TOOKE  \textit{Cath. II}, III. 105 n., A man remarkable for his talents and urbane manners.    
\P 1832 W. IRVING  \textit{Alhambra} II. 289 His manners were gentle, affable, and urbane.    
\P 1860 W. COLLINS  \textit{Wom. in White} II. 279 Stepping forward in the most urbane manner.    
\P 1871 BROWNING \textit{Balaust.} 1839 To guests,  a servant should not sour-faced be, But do the honours with a mind urbane.

\itembf{3.} Refined in expression; politely expressed.

\P 1806 W. L. BOWLES  \textit{Pope's Wks.} I. 298 The latter part of it [sc. an epistle] is certainly urbane, elegant, and unaffected.    
\P 1876 LOWELL  \textit{Among my Bks.} Ser. ii. 139 We miss the point, the compactness, and above all the urbane tone of the original.

\noindent Hence \textbf{urbanely} adv.; \textbf{urbaneness} (Bailey, 1727).

\P 1822  \textit{Monthly Rev.} XCVII. 540 This taste is so finely polished and so urbanely expressive.    
\P 1881 ‘RITA’  \textit{My Lady Coquette} xiii, ‘I am going to the wood,’ he answers urbanely.
\end{myenumerate}


%%%%%%%%%%%%%%%%%%%%%%%%%%%%%%%%%
\myitem{irascible} a.

\noindent \phonetic{(ɪˈræsɪb(ə)l, aɪˈræs-)}

\noindent [a. F. irascible (12th c. in Littré), ad. L. īrāscibil-is, f. īrāscī to grow angry.]
\vspace{-0.3cm}

\begin{myenumerate}

\itembf{a.} Easily provoked to anger or resentment; prone to anger; irritable, choleric, hot-tempered, passionate.

\P 1530 PALSGR. 316/2 Irascible, inclyned or disposed to anger, irascible.    
\P 1656 BLOUNT  \textit{Glossogr.}, Irascible, cholerick, soon angred, subject to anger.    
\P 1759 ROBERTSON  \textit{Hist. Scot.} (1817) I. ii. 345 The Scots, naturally an irascible and high spirited people.    
\P 1831 SCOTT  \textit{Cast. Dang.} vii, The boar‥was a much more irascible and courageous animal.    
\P 1873 BLACK  \textit{Pr. Thule} viii. (1874) 114 The only daughter of a solitary and irascible old gentleman.

\itembf{b.} Of emotions, actions, etc.: Characterized by, arising from, or exhibiting anger.

\P 1659 D. PELL  \textit{Impr. Sea} 426 Irascible, and objurgatory speech.    
\P 1734 WATTS  \textit{Reliq. Juv.} lx. (1789) 200 Our irascible passions‥indulged‥are ready to defile the whole man.    
\P 1774 GOLDSM.  \textit{Nat. Hist.} (1776) VII. 296 No animal in the creation seems endued with such an irascible nature.    
\P 1824 W. IRVING  \textit{T. Trav.} I. 302 Dignity is always more irascible the more petty the potentate.    
\P 1882 A. W. WARD  \textit{Dickens} v. 119 His irascible nature failed to resent a rather doubtful compliment.

\itembf{c.} irascible appetite, irascible affection, irascible part of the soul, in Plato's tripartite division of the soul, τὸ θυµοειδές, one of the two parts of the irrational nature, being that in which courage, spirit, passion, were held to reside; and which was superior to τὸ ἐπιθυµητικόν, the concupiscible part in which resided the appetites.

\P 1398 TREVISA  \textit{Barth. De P.R.} iii. vi. (Add. MS. 27944) lf. 20 b/2 Drede \& sorwe comeþ of þe irascibel, for of þing þat we hatiþ, we haueþ sorowe.    
\P 1526  \textit{Pilgr. Perf.} (W. de W. 1531) 112 b, It is called the appetyte irascyble, or the angry appetyte.    
\P 1606 L. BRYSKETT  \textit{Civ. Life} 48 The seates of the two principall appetites, the irascible and the concupiscible; of that the heart, of this the liuer.    
\P 1691 HARTCLIFFE  \textit{Virtues} 23 Pride, Contempt, Impatience, Anger, Fear, Boldness and the like generous and brave Passions, belong to what we say is the irascible part of the mind.    
\P 1863 DRAPER  \textit{Intell. Devel. Europe} v. (1865) 116 Now, the reason being seated in the head, the spirit or irascible soul has its seat in the breast.

\itembf{d.} quasi-n. = Irascible appetite, etc. Obs.

\P 1594 [see  CONCUPISCIBLE 2 b].    
\P 1656 H. MORE  \textit{Enthus. Tri. To Rdr.} A iij a, These I spread before him‥to provoke his Irascible.

\noindent Hence \textbf{irascibleness}, irascibility; \textbf{irascibly} adv., in an irascible manner, angrily.

\P 1727 BAILEY  vol. II, \textit{Irascibleness}.    
\P 1828 \textit{Mirror}  V. 264/1 Nothing irascibly said will‥make way with an obstinate or wilful man.
\end{myenumerate}


%%%%%%%%%%%%%%%%%%%%%%%%%%%%%%%%
\myitem{cranky} a.1

\noindent \phonetic{(ˈkræŋkɪ)}

\noindent [A comparatively modern formation, covering a group of senses that hang but loosely together, and have various associations with crank n.2 and n.3, crank a.2 and a.3.]

   (The order here followed is merely provisional.)
\vspace{-0.3cm}

\begin{myenumerate}
\itembf{1.} Sickly, in weak health, infirm in body; = crank a.3 3. dial.

\P 1787 GROSE  \textit{Prov. Gloss.}, Cranky, ailing, sickly; from the dutch crank, sick. N[orth].    
\P 1869 LONSDALE  \textit{Gloss.}, Cranky, ailing, sickly. [So in dial. Glossaries of Cumberland, Whitby, Holderness, Leicestersh., Berkshire; W. Somerset has crankety; in others prob. omitted as being a general word.]    
\P 1891  \textit{Science} (N.Y.) 21 Aug. 102/2 The vigorous sheep being constantly drafted away for sale‥these ‘cranky’ sheep (as they came to be called) were left behind.

\itembf{2.} Naut. = crank a.2

\P 1861 WYNTER  \textit{Soc. Bees} 358 ‘Beg pardon, sir, but the boat is very cranky‥if you goes on so, she will be over.’    
\P 1870 LOWELL  \textit{Study Wind.} (1886) 126 The craft is cranky.

\itembf{3.} Out of order, out of gear, working badly; shaky, crazy; = crank a.3 4.

\P 1862 SMILES  \textit{Engineers} III. 90 It was constantly getting out of order‥at length it became so cranky that the horses were usually sent out after it to bring it along.    
\P 1863 MRS. TOOGOOD  \textit{Yorksh. Dial.}, ‘Don't sit on that chair, it is cranky.’    
\P 1888 BERKSHIRE  \textit{Gloss.}, Cranky‥for machinery, out of gear; for a structure, in bad repair, likely to give way.

\itembf{4.} Of capricious or wayward temper, difficult to please; cross-tempered, awkward; ‘cross’.

\P 1821  \textit{Blackw. Mag.} IX. 82 Cranky Newport, not annoyed with νοῦς.    
\P 1840 DICKENS  \textit{Old C. Shop} vii, That his friend appeared to be rather ‘cranky’ in point of temper.    
\P 1851 D. JERROLD  \textit{St. Giles} xv. 151 He got plaguy cranky of late; wouldn't come down with the money.    
\P 1876 C. M. YONGE  \textit{Womankind} xxiii. 199 We view our maids as cranky self-willed machines for getting our work done.    [In dial. Glossaries of Cumberland, Whitby, Holderness, Leicester.]

\itembf{5.} Mentally out of gear; crotchety, ‘queer’; subject to whims or ‘cranks’; eccentric or peculiar in notions or behaviour. Cf. crank n.2 4, 5.

\P 1850 DICKENS  \textit{Poor Man's Tale of Patent} (Househ. Wds. 19 Oct. 70), I said, ‘William Butcher‥You are sometimes cranky’.    
\P 1863 C. READE  \textit{Hard Cash} II. 113 He [a mad-doctor] had‥almost invariably found the patient had been cranky for years.    
\P 1876  \textit{Whitby Gloss.} s.v., Cranky ways, crotchets.    
\P 1879 G. MACDONALD  \textit{P. Faber} II. iv. 66 A cranky, visionary, talkative man.    
\P 1884 \textit{Boston  (Mass.) Jrnl.} July 11, Butler makes a long fight over his cranky notions.

\itembf{6.} Full of twists or windings, crooked; full of corners or crannies. Cf. crank n.2 1, 2.

\P 1836 W. S. LANDOR  \textit{Wks.} 1876 VIII.  94 No curling dell, no cranky nook.    
\P 1876  \textit{Whitby Gloss.} s.v., Cranky roads, crooked roads.    
\P 1887 JESSOPP  \textit{Arcady} iii. 71 Old closets, dim passages, and cranky holes and corners.

\itembf{7.} (See quot.) dial. Cf. crank v.1 2.

\P 1788 MARSHALL  \textit{Yorksh. Gloss.}, Cranky, checked [i.e. striped] linen; cranky apron, a checked-linen apron.    
\P 1876 WHITBY  \textit{Gloss.}, Cranky adj., of stout old-fashioned linen for housewives' aprons, with a blue stripe on a white ground.
\end{myenumerate}


%%%%%%%%%%%%%%%%%%%%%%%%%%%%%%%%
\myitem{rancour} n.

\noindent \phonetic{(ˈræŋkə(r))}

\noindent [a. OF. rancor, -cour, -cuer, raunkour, etc.:—L. rancōr-em rancidity, rankness, hence (in the Vulgate) bitter grudge.]
\vspace{-0.3cm}

\begin{myenumerate}

\itembf{1.} Inveterate and bitter ill-feeling, grudge, or animosity; malignant hatred or spitefulness.

\P [a1225  \textit{Ancr. R.} 200 Þe oðer kundel is Rancor siue odium: þet is, hatunge oðer great heorte.]    
\P 13‥ \textit{E.E. Allit. P.} B. 756, I schal‥my rankor refrayne for þy reken wordez.
\P c1380  \textit{Sir Ferumb.} 5759 Fyrumbras‥prayede him cesse of his rauncour.    
\P 1413  \textit{Pilgr. Sowle} ii. xlv. (1859) 51 Wretched folke and irous, ful of venym, of rancour, and of hate.    c 
\P 1440  \textit{Jacob's Well} 249 Whanne þou mercyfully forȝeuyst þi wrongys, wyth-oute wreche \& rankure in herte, þat is mercy.    a 
\P 1533 LD. BERNERS  \textit{Huon} lxxxiv. 266, I‥pardon you of all myn yll wyll, and put al rancoure fro me.    
\P 1547 J. HARRISON  \textit{Exhort. Scottes} A iv b, Peace in their mouthes, and all rancor and vengeaunce in their hartes.    
\P 1605 WILLET  \textit{Hexapla Gen.} 234 Yet doe retaine ranker and seedes of malice in their heart.    
\P 1667 MILTON  \textit{P.L.} x. 1044 Rancor  and pride, impatience and despite.    
\P 1725 POPE  \textit{Odyss.} iii. 182 Each burns with rancour to the adverse side.    
\P 1828 D'ISRAELI  \textit{Chas. I}, II. vii. 174 To envy‥Charles traced their personal rancour to the friend of his heart.    
\P 1865 MAFFEI  \textit{Brig. Life} II. 37 The gratification of private rancour, and personal revenge.

\itembf{b.} transf. and fig. of things.

\P 1582 STANYHURST  \textit{Æneis} i. (Arb.) 22 Billows theire swelling ranckor abated.    
\P 1605 CAMDEN  \textit{Rem.} 207 Through the rancor of the poyson, the wound was iudged incurable.    
\P 1663 BUTLER \textit{Hud.} i. i. 364 The peaceful Scabbard‥The Rancor of its edge had felt.    
\P 1719 D'URFEY  \textit{Pills} (1872) I. 48 Let the frozen North its rancour show.    
\P 1860 EMERSON  \textit{Cond. Life, Power} Wks. (Bohn) II. 333 The rancour of the disease attests the strength of the constitution.

\itembf{2.} Rancid smell; rancidity; rankness. Obs. rare.

\P 1400  \textit{Laud Troy Bk.} 6028 Ther come of hem a foul savour And smot to hem a gret rancour.    c 
\P 1420 PALLAD.  \textit{on Husb.} xi. 111 Lest rancour oil enfecte, do fier away.    
\P 1567 J. MAPLET  \textit{Naturall Hist.} 33 b, It is also said somtime through the rancour of grounds to come vp vnsowne.

\noindent Hence \textbf{rancourless} a., free from rancour.

\P 1886 H. JAMES  \textit{Bostonians} II. ii. xx. 26 She was too rancourless,‥too free from private self-reference.
\end{myenumerate}


%%%%%%%%%%%%%%%%%%%%%%%%%%%%%%%%%
\myitem{irenic} a. and n.

\noindent \phonetic{(aɪˈrɛnɪk, aɪˈriːnɪk)}

\noindent [ad. Gr. εἰρηνικ-ός, f. εἰρήνη peace. Cf. eirenic and F. irénique (Littré).

   In this and the following word, the first pronunciation is that given by Smart, Ogilvie, and Cassell, and by Webster and the other American Dictionaries, and is in accordance with the general analogies of the language, as in academic, clinical, energetic, euphonic, Platonic, in which the long vowel of the Greek is uniformly shortened; but the modern use of the Greek Εἰρηνικόν, Eirēnicon, to which scholars naturally give the English academic pronunciation of Greek, affects the derivatives also, and makes the second pronunciation frequent among university men.]
\vspace{-0.3cm}

\begin{myenumerate}

\itembf{A.} adj. Pacific, non-polemic; = irenical.

\P 1864 in  WEBSTER.    
\P 1878 \textit{N. Amer.  Rev.} 335 President Porter, in his admirable and irenic opening of this discussion, makes it very difficult, for one who follows him.    
\P 1882-3 SCHAFF \textit{Encycl. Relig. Knowl.} I. 710 He was a man of irenic temperament.    
\P 1885 \textit{Ch. Times}  343/1 No irenic propositions will do the least good till we have had those standards restored.

\itembf{B.} n. pl. \textbf{irenics}: irenical theology.

\P 1882-3 SCHAFF \textit{Encycl. Relig. Knowl.} II. 1118 Irenical  Theology, or Irenics ‥ presents the points of agreement among Christians with a view to the ultimate unity‥of Christendom.    
\P 1890 \textit{Congreg.  Rev. Apr.} 158 Our mission is not one of polemics but irenics.
\end{myenumerate}


%%%%%%%%%%%%%%%%%%%%%%%%%%%%%%%%%
\myitem{fray} v.1

\noindent \phonetic{(freɪ)}

\noindent [aphetic f. affray, effray v.]
\vspace{-0.3cm}

\begin{myenumerate}

\itembf{1.} trans. To affect with fear, make afraid, frighten. Cf. affray v. 2. Obs. exc. poet.

\P 1330 [see frayed ppl. a.].    
\P 13‥ \textit{E.E. Allit. P.} B. 1553 For  al hit frayes my flesche þe fyngres so grymme.    
\P 14‥ \textit{Sir Beues} 2396 (MS. M.) The dragon kest vp a yelle, That it wolde haue frayed the deuyl of hel.    
\P 1531 TINDALE  \textit{Exp.} 1 John (1537) 14 That‥we shulde exalte our selues ouer you‥frayenge you with the bugge of excommunicacyon.    
\P 1604 BP. W. BARLOW  \textit{Confer. Hampton Crt. in Phenix} (1721) I. 154 A Puritan is a Protestant fray'd out of his Wits.    
\P 1742 SHENSTONE  \textit{Schoolmistress} 149 And other some with baleful sprig she 'frays.    
\P 1832 J. BREE  \textit{St. Herbert's Isle} 98 He frayed the monsters with his bugle's sound.    
\P 1850 BROWNING  \textit{Christmas Eve \& Easter Day}, My warnings fray No one, and no one they convert.

\noindent absol. \P 1496  \textit{Bk. St. Albans, Fishing} C j, And when she hath plumyd ynough: go to her softly for frayenge.    
\P 1590 SPENSER  \textit{F.Q.} ii. xii. 40 Instead of fraying they themselves did feare.

\itembf{2.} To frighten or scare away. Also to fray away, fray off, or fray out. Cf. affray v. 4. Obs. exc. arch.

\P 1526  \textit{Pilgr. Perf.} (W. de W. 1531) 55 God hath ordeyned‥a specyall remedy, wherwith we may fray them away.    
\P 1533 TINDALE  \textit{Supper of Lord} cv b, Why fraye ye the commen people from the lytteral sense with thys bugge?    
\P 1586 MARLOWE  \textit{1st Pt. Tamburl.} v. ii, Are the turtles frayed out of their nests?    
\P 1613 PURCHAS  \textit{Pilgrimage} vi. i. 560 It [the Basilisk]‥frayeth away other serpents with the hissing.
\P a1716 SOUTH  \textit{Serm.} (1744) X. 232 Can he fray off the vultur from his breast?    
\P 1825 SCOTT  \textit{Betrothed} xxiii, It is enough to fray every hawk from the perch.    
\P 1867 MANNING  \textit{Eng. \& Christendom} 154 We should have to answer to the Good Shepherd, if so much as one of His sheep were frayed away from the fold by harsh voices.

\P 1542 BECON  \textit{David's Harp} Wks. 1564 I. 147  Exhort unto virtue. Fray away from vice.

\itembf{b.} simply. To drive away, disperse.

\P 1635 QUARLES  \textit{Embl.} i. xiv. (1718) 57 Thy light will fray These horrid mists.    
\P 1655 H. VAUGHAN  \textit{Silex Scint.} ii. Death (1858) 205 Thy shades‥Which his first looks will quickly fray.

\itembf{3.} intr. To be afraid or frightened; to fear. Obs.

\P a1529 SKELTON  \textit{Image Hypocr.} 509 Yow fray not of his rod.    
\P 1535 STEWART  \textit{Cron. Scot.} I. 606 Thai had no caus to dreid Nor ȝit to fray.    
\P 1638 R. BAILLIE  \textit{Lett.} (1775) I. 80 This and the convoy of it make us tremble for fear of division‥Thir thingis make us fray.

\itembf{4.} trans. To assault, attack, or make an attack upon; to attack and drive off; rarely to make a raid on (a place). Obs.

\P 1400  \textit{Destr. Troy} 5237 The grekys‥segh the kyng‥With fele folke vppon fote þat hom fray wold.
\P a1440 \textit{Sir Degrev.}  237 Thus the forest they fray, Hertus bade at abey.
\P c1575 DURHAM  \textit{Depos.} (Surtees) 286 Neither this examinate nor his brother‥ever did lay in wayt nor frayd off the said Sir Richard Mylner.

\itembf{5.} intr. To make a disturbance; to quarrel or fight. Also, to make an attack upon. to fray it out: to settle by fighting. Obs. exc. arch.

\P 1460 TOWNELEY  \textit{Myst.} (Surtees) 147 Why shuld we fray?    
\P 1465 \textit{Paston  Lett.} No. 512 II. 205 My Lord of Suffolks men‥fray uppon us, this dayly.    
\P 1494 FABYAN \textit{Chron.} iv. lxxi. (1811) 50 Conan Meridok with a certayne of knyghtes of his affynyte, was purposed to haue frayed with the sayd Maximus, and to haue distressed hym.    
\P 1566 DRANT  \textit{Horace's Sat.} iii. B v b, For foode and harboure gan they fray‥with clubbes.    
\P 1570 \textit{Song} in  \textit{Wit \& Sci.} etc. (Shaks. Soc.) 90 The sonne is up with hys bryght beames, As thoughe he woolde with the now fraye, And bete the up out of thy dreames.    
\P 1657 HOWELL  \textit{Londinop.} 337 A gaol‥for such as should brabble, fray, or break the peace.    
\P 1889  \textit{Univ. Rev.} Sept. 38 Sooner than fray it out thou wouldst retire.

\noindent Hence \textbf{fraying} vbl. n. and ppl. a.

\P 1450  \textit{Merlin} 339 Arthur was also fallen to grounde with the frayinge that thei hurteled to-geder.    
\P 1548 UDALL, etc. \textit{Erasm. Par. John} x. 1 They doe their endeuour to maynteyn their tyrannie with disceytes, frayinges, wiles [etc.].    
\P 1562 J. HEYWOOD  \textit{Prov. \& Epigr.} (1867) 194 Of fraying of babes.    
\P 1577 HANMER  \textit{Anc. Eccl. Hist.} (1619) 394 But only avoideth this clause‥as a fraying ghost.
\end{myenumerate}


%%%%%%%%%%%%%%%%%%%%%%%%%%%%%%%%%
\myitem{galvanize} v.

\noindent \phonetic{(ˈgælvənaɪz)}

\noindent [ad. F. galvaniser: see galvanism and -ize.]
\vspace{-0.3cm}

\begin{myenumerate}

\itembf{1.} trans. To apply galvanism to; to stimulate by means of a galvanic current. Also absol.

\P 1802  \textit{Med. Jrnl.} VIII. 259 The heat is likewise increased in the part which is galvanised.    
\P 1825 SYD. SMITH  \textit{Wks.} (1867) II. 203 Galvanise a frog, don't galvanise a tiger.    
\P 1831 CARLYLE  \textit{Sart. Res.} (1858) 142 Those spasmodic, galvanic sprawlings are not life; neither indeed will they endure, galvanise as you may, beyond two days.    
\P 1839-47 TODD  \textit{Cycl. Anat.} III. 41/2, I galvanized a little boy with paralysis of the left leg.    
\P 1850 ROBERTSON  \textit{Serm. Ser.} iii. ix. 117 You may galvanize the nerve of a corpse till the action of a limb startles the spectator with the appearance of life.

\itembf{b.} fig. esp. in phrase to galvanize to or into life (also to galvanize life into).

\P 1853 C. BRONTË  \textit{Villette} iii, Her approach always galvanized him to new and spasmodic life.    
\P 1869 GOULBURN  \textit{Purs. Holiness} xxi. 203 She would fain galvanize the soul into life by a sudden shock.    
\P 1880  \textit{Daily News} 9 Jan. 3/1 To galvanise a little more life into the market.    
\P 1883  \textit{Harper's Mag.} Mar. 537/1 A very old inn, that seemed suffering the first pangs of being galvanized back to life and modernity.

\itembf{2.} To cover with a coating of metal by means of galvanic electricity. Commonly but incorrectly applied to the coating of iron with zinc to protect it from rusting, though no galvanic process is ordinarily employed.

\P 1839 [see GALVANIZED ppl. a. 2].    
\P 1864 WEBSTER,  \textit{Galvanize}, to plate, as with gold, silver, \&c., by means of galvanism.    
\P 1869 ROSCOE  \textit{Elem. Chem.} 230 Zinc‥is employed as a protecting covering for iron, which when thus coated is said to be galvanized.    
\P 1879  \textit{Cassell's Techn. Educ.} i. 61/2 The wire is ‘galvanised’ or coated with metallic zinc.

\noindent absol. \P 1892 \textit{Workshop  Receipts} 287 It is an advantage, with all sheets thicker than 20 gauge, to galvanize after corrugation.
\end{myenumerate}


%%%%%%%%%%%%%%%%%%%%%%%%%%%%%%%%
\myitem{goad} n.1

\noindent \phonetic{(gəʊd)}

\noindent [OE. gád str. fem. corresponds to Lombard gaida arrow-head:—OTeut. type *gaiđâ; for possible cognates see gare n.1 The northern form is gaid (q.v.), but in ME. both northern and southern forms are less common than the synonymous, though unrelated, gad n.1]
\vspace{-0.3cm}

\begin{myenumerate}

\itembf{1.} A rod or stick, pointed at one end or fitted with a sharp spike and employed for driving cattle, esp. oxen used in ploughing (cf. gad n.1 4).

\P c725 \textit{Corpus Gloss.} 1937 Stiga  [sic], gaad.
\P a1000 \textit{Sal. \& Sat.}  91 (Gr.) \phonetic{Hafað gudmæcga ᴁierde lanᴁe, gyldene gade}.
\P 1388 WYCLIF  \textit{Ecclus.} xxxviii. 26 He that holdith the plow, and he that hath glorie in a gohode [L. in jaculo], dryueth oxis with a pricke.
\P c1394 \textit{P. Pl.  Creed} 433 His wijf walked him wiþ [at the plough] with a longe gode.    
\P 14‥ \textit{Voc} in Wr.-Wülcker 586/23 Gerusa, a goode.
\P c1440  \textit{Promp. Parv.} 184/1 Gad or gode, gerusa.    
\P 1539 TAVERNER  \textit{Erasm. Prov.} (1552) 15 It is harde kyckynge agaynst the gode.    
\P 1627 DRAYTON  \textit{Sheph. Sirena} 361 They their Holly whips haue brac'd, And tough Hazell goades haue gott.    
\P 1635-56 COWLEY  \textit{Davideis} iv. 166 With the same Goad Samgar his Oxen drives Which took‥six hundred lives.    
\P 1703 MAUNDRELL  \textit{Journ. Jerus.} (1732) 110 In ploughing they us'd Goads‥about eight foot long.    
\P 1783 HOOLE  \textit{Orl. Fur.} xxxvii. 804 A hind‥A rustic weapon for her rage supply'd, A pointed goad he brought.    
\P 1816 SCOTT  \textit{Old Mort.} xv, Countrymen armed with scythes‥hay-forks‥goads.    
\P 1875 HELPS  \textit{Ess., Organiz. in Daily Life} 109, I had a thought that drove me like a goad.

\itembf{2.} fig. Something that pricks or wounds like a goad. \itembf{a.} A torment, ‘thorn’, ‘sting’.

\P 1561 tr.  \textit{Calvin's 4 Serm. agst. Idolatries} i. C ij b, Those same goads and prickes wherwith their consciences are prikt and wounded.    
\P 1641 J. JACKSON  \textit{True Evang. T.} ii. 138 These pointed and diamonded speeches, which doe indeed leave a sting, and goad in the mind of the pious Auditor.    
\P 1689 SHADWELL  \textit{Bury F.} iii. 181 Where is my Goad' my damned for better or worse.    
\P 1759 FRANKLIN \textit{Ess.} Wks. 1840 III.  255 French forts and French armies so near us will be everlasting goads in our sides.    
\P 1861 TRENCH  \textit{Comm. Ep. to Ch. Asia} 80 There are ever goads in the memory of a better and a nobler past.    
\P 1879 FARRAR  \textit{St. Paul} (1883) 140 The wounding goad of a reproachful conscience.

\itembf{b.} A strong incitement or instigation, ‘spur’, stimulus.

\P 1600 HOLLAND  \textit{Livy} xxxix. xv. (1609) 1032 These‥who  pricke and provoke (as it were) with goads [L. stimulis] of furies your spirits and minds.    
\P 1608 R. ARMIN  \textit{Nest Ninn.} (1842) 4 That's the way to spoyle all, but with your goad pricke me on the true tract.    
\P 1615 CROOKE  \textit{Body of Man} 284 Those Females which are castrated or gelt‥the goads of lust are in them vtterly extinguished.    
\P 1798 MALTHUS  \textit{Popul.} iii. i. (1806) II. 82 The labour‥will not be performed without the goad of necessity.
\P a1859 MACAULAY  \textit{Biog.} (1867) 110 He no longer felt the daily goad urging him to the daily toil.    
\P 1876 MOZLEY  \textit{Univ. Serm.} iv. (1877) 94 Knowledge is a goad to those who have it.

\itembf{3.} A measure of length. \itembf{a.} A cloth-measure = 4½ feet. Obs.

\P 1481 HOWARD  \textit{Househ. Bks.} (Roxb.) 17 My Lord schal haue of hym iiij.c goodes off white‥and my Lord schal pay him for euery goode, ix.d.    
\P 1552  \textit{Act 5 \& 6 Edw VI}, c. 6 §1 Cottonnes called Manchester‥and Chesshire Cottonnes‥shalbe in lenghe twentie two goades and conteyne in bredith thre quarters of a yarde in the water.    
\P 1674 S. JEAKE  \textit{Arith.} (1696) 65 In 1 Goad‥4½ Feet, a Measure in some places for Land and Cloth received by Custom.    
\P 1721 C. KING  \textit{Brit. Merch.} I. 181, 1200 C. Goads  of Cotton.    
\P 1727 W. MATHER  \textit{Yng. Man's Comp.} 399 In London, the Yard is used for Silks, Woollen Cloth, \&c. The Ell for Linnen Cloth, \&c., and the Goad for Frizes, Cotton, and the like.

\itembf{b.} A land-measure (see quots. and cf. gad 6).

\P 1587 FLEMING  \textit{Contn. Holinshed} III. 1353/1 The space of fortie goad (euerie goad conteining fifteene foot).    
\P 1880 E. CORNW.  \textit{Gloss.} s.v., It represents nine feet, and two goads square is called a yard of ground.

\itembf{4.} A spike = gad n.1 1.

\P 1855 J. HEWITT  \textit{Anc. Armour} I. 81 The spur of this period consisted of a single goad, sometimes of a lozenge form, sometimes a plain spike.

\itembf{5.} Comb., as goad-groom, goad-prick; also goad(s)-man = gadman; goad-spur, a spur without a rowel and with one point (cf. prickspur).

\P 1614 SYLVESTER  \textit{Little Bartas} 877 Thou‥by one man, one *Goad-groom (silly Sangar), Destroy'dst six hundred in religious anger.

\P 1605 \textit{Du Bartas} ii. iii. iv. Captaines 710 And *Goad-man Sangar.    
\P 1765 A. DICKSON  \textit{Treat. Agric.} (ed. 2) 248 The goadman or driver.    
\P 1816 SCOTT  \textit{Old Mort.} vi, Ye may be goadsman‥and tak tent ye dinna o'erdrive the owsen.
\P c1826 HOGG in  \textit{Wilson's Wks.} (1855) I. 176 The goadman whistles sparely.

\P 1609 BIBLE (Douay)  \textit{1 Sam.} xiii. 21 Even to the *godeprick, which was to be mended.

\P 1889 \textit{Century Dict.}, *Goad-spur.
\end{myenumerate}

%%%%%%%%%%%%%%%%%%%%%%%%%%%%%%%%
\myitem{instigate} v.

\noindent \phonetic{(ˈɪnstɪgeɪt)}

\noindent [f. L. instigāt-, ppl. stem of instigāre to urge, set on, incite, f. in- (in-2) + *stigāre: cf. Gr. στίζειν (root στιγ-) to prick.]
\vspace{-0.3cm}

\begin{myenumerate}

\itembf{1.} trans. To spur, urge on; to stir up, stimulate, incite, goad (now mostly to something evil).

\P 1542 BOORDE  \textit{Dyetary} viii. (1870) 245 It doth instygate and lede a man to synne.    
\P 1639 WOODALL  \textit{Wks.} Pref. (1653) 2 Some Noble man, who was instigated thereunto through an excellent and divine power.    
\P 1651 HOBBES  \textit{Leviath.} iii. xlii. 278 To instigate Princes to warre upon one another.    
\P 1671  \textit{True Nonconf.} 469 The only motive..whereby Henry was instigat to reject the Pope.    
\P 1747 JOHNSON  \textit{Plan Eng. Dict.} Wks. 1787 IX. 185  Commonly, though not always, we exhort to good actions, we instigate to ill.    
\P 1841 BREWSTER  \textit{Mart. Sc.} iii. iii. (1856) 204 The proud Duke of Tuscany, instigated no doubt by Galileo, sent Kepler a gold chain.    
\P 1855 BROWNING  \textit{Fra Lippo} 316 ‘Ay, but you don't so instigate to prayer!’ Strikes in the Prior.    
\P 1875 JOWETT  \textit{Plato} (ed. 2) IV. 335 You..must not instigate your elders to a breach of faith.

\itembf{2.} To bring about by incitement or persuasion; to stir up, foment, provoke.

\P 1852 THACKERAY  \textit{Esmond} ii. iv, What he and they called levying war was, in truth, no better than instigating murder.    
\P 1868 MILMAN  \textit{St. Paul's} iii. 47 The mission of Otho had been instigated by the King.

Hence \phonetic{ˈinstigated, ˈinstigating} ppl. adjs.; \phonetic{ˈinstigatingly} adv., in an instigating manner, so as to instigate.

\P 1611 COTGR.,  \textit{Instigué}, instigated, incited, vrged.    
\P 1702 DE FOE  \textit{Reform. Manners Misc.} (1703) 81 How Clito comes from instigating Whore, Pleads for the Man he cuckold just before.    
\P 1856 WEBSTER, \textit{Instigatingly}.
\end{myenumerate}


%%%%%%%%%%%%%%%%%%%%%%%%%%%%%%%%
\myitem{foment} v.

\noindent \phonetic{(fəʊˈmɛnt)}

\noindent [ad. Fr. foment-er, ad. late L. fōmentāre, f. fōmentum foment n.]
\vspace{-0.3cm}

\begin{myenumerate}

\itembf{1.} trans. To bathe with warm or medicated lotions; to apply fomentations to. Also, to lubricate.

\P 1611 COTGR,  \textit{Bassiner}, to warme, foment.    
\P 1643 J. STEER  tr. \textit{Exp. Chyrurg.} xii. 47 Foment the place affected with the following foment.    
\P 1656 RIDGLEY  \textit{Pract. Physick} 131 Foment it with white wax.    
\P 1748 tr.  \textit{Vegetius' Distemp. Horses} 144 You shall foment it for the Space of four Days.    
\P 1802  \textit{Med. Jrnl.} VIII. 516 The breasts were frequently fomented.    
\P 1894 SIR F. FITZWYGRAM  \textit{Horses \& Stables} §255 The leg.. may be conveniently fomented by putting it in a deep bucket of warm water.

\indent absol. \P 1612 WOODALL  \textit{Surg. Mate} Wks. (1653) 303 Foment not too long at any one time.

\itembf{2.} ‘To cherish with heat, to warm’ (J.). Always in conjunction with another verb, as chafe, heat, warm. Obs.

\P 1648 J. BEAUMONT  \textit{Psyche} i. clv, Creeps chillness on him? She foments and heats His flesh.    
\P 1667 MILTON  \textit{P.L.} iv. 669 All things..these soft fires..foment and warme.

\itembf{3.} To rouse or stir up (a person or his energies); to excite, irritate. Obs.

\P 1642 R. CARPENTER  \textit{Experience} v. xix. 326, I was active..fomented with your envenomed suggestions.    
\P 1680 OTWAY  \textit{Orphan} iv. v. 1506 Still  chaft and fomented let my heart swell on.    
\P 1704 SWIFT  \textit{Batt. Bks.} (1711) 226 By its Bitterness and Venom..to foment the Genius of the Combatants.    
\P 1724 DE FOE  \textit{Mem. Cavalier} (1840) 127 The old general, not to foment him, with a great deal of mildness stood up.

\itembf{b.} intr. for refl.: To become excited or heated.

\P 1665 J. WEBB  \textit{Stone-Heng} 16 In like manner, this Doctor fomenteth, saying; The one stumbles upon an Alter-stone..over which the other leaped clearly.    
\P 1680 OTWAY  \textit{Orphan} v. ii. To think  of Women were enough to taint my Brains, Till they foment to madness.

\itembf{4. a.} To promote the growth, development, effect, or spread of (something material or physical).

\P 1644 QUARLES  \textit{Barnabas \& B.} 150 That humour which foments thy malady.    
\P 1661 \textit{Burning  of Lond.} in \textit{Select. Harl. Misc.} (1793) 463 A violent easterly wind fomented it, and kept it burning all that day.    
\P 1667 MILTON  \textit{P.L.} x. 1071 How  we his gather'd beams Reflected, may with matter sere foment.    
\P 1707 \textit{Curios.  Husb. \& Gard.} 180 Plants receive from their Roots this Nitre, which feeds, foments and preserves them.    
\P 1725 POPE  \textit{Odyss.} xix. 77 While those with unctuous fir foment the flame.

\itembf{b.} To cherish, cultivate, foster; to stimulate, encourage, instigate (a sentiment, belief, pursuit, course of conduct, etc.). Esp. in a bad sense.

\P 1622 BACON  \textit{Hen. VII}, 12 Which bruite was cunningly fomented by such as desired innouation.    
\P 1664 MARVELL  \textit{Corr.} Wks. 1872-5 II. 164 His Majesty..offers himself as a third to foment so amiable a controversy.    
\P 1725 POPE  \textit{Odyss.} xi. 226 Thy sire in solitude foments his care.    1726-7 Swift Gulliver i. iv, These civil commotions were constantly fomented by the monarchs of Blefuscu.    
\P 1774 FLETCHER  \textit{Equal Check} Wks. 1795 IV. P. V,  Is not the Antinomianism of hearers fomented by that of preachers?    
\P 1868 M. PATTISON  \textit{Academ. Org.} iv. 75 To encourage indolence or foment extravagance.    
\P 1873 H. ROGERS  \textit{Orig. Bible} ii. (1875) 59 Persecutions which the Jews always fomented.

Hence \phonetic{foˈmenting} vbl. n. Also attrib.

\P 1611 COTGR.,  \textit{Bassinement}, warming, a fomentation or fomenting.    
\P 1894 SIR F. FITZWYGRAM  \textit{Horses \& Stables} §255 During the fomentation a thick rug should be thrown over the fomenting cloth.
\end{myenumerate}


%%%%%%%%%%%%%%%%%%%%%%%%%%%%%%%%
\myitem{abet} v.

\noindent \phonetic{(əˈbɛt)}

\noindent [a. OFr. abeter, f. à to + beter to bait, hound on; prob. ad. Norse beita to cause to bite, hence to ‘bait,’ to hound on dogs, etc.; causal of bíta to bite.]
\vspace{-0.3cm}

\begin{myenumerate}

\itembf{1.} To urge on, stimulate (a person to do something). Obs.

\P c1380 \textit{Sir Ferumb.} 5816 Bot if he þanne wold take fulloȝt, As he hym wolde abette.    
\P 1587 FLEMING  \textit{Cont. of Holinsh.} III. 1579/2 The Scottish queene did not onelie advise them, but also direct, comfort, and abbet them, with persuasion, counsell, promise of reward, and earnest obtestation.

\itembf{2.} esp. in a bad sense: To incite, instigate, or encourage (a person, to commit an offence (obs.), or in a crime or offence). In legal and general use.

\P 1590 SHAKES.  \textit{Com. Err.} ii. ii. 172 Abetting him to thwart me in my moode.    
\P a1593 H. Smith \textit{Wks.} (1867) II. 429 He will not only pardon without exception, but he will abet them in their damnable courses.    
\P 1658-9 MR. SCOTT in \textit{Burton's Diary} (1828) IV. 36 Are those fit to have a parliamentary authority, that will undertake to abet the single person to levy taxes without you?    
\P 1770 BURKE  \textit{Pres. Discon.} Wks. II. 259 He abets a faction that is driving hard to the ruin of his country.    
\P 1809 TOMLINS  \textit{Law Dict.} s.v. To abet..in our law signifies to encourage or set on.    
\P 1866 KINGSLEY  \textit{Hereward} xviii. 219 The two regents abetted the ill-doers.    
\P 1876 FREEMAN  \textit{Norm. Conq.} III. xii. 113 To abet them against their sovereign.

\itembf{3.} To support, countenance, maintain, uphold, any cause, opinion, or action. Obs. in a good sense.

\P 1596 SPENSER  \textit{F.Q.} i. x. 64 Then shall I soone..abett that virgins cause disconsolate.    
\P 1603 DRAYTON  \textit{Heroical Epist.} (1619) xvi. 29 Who moves the Norman to abet our Warre?    
\P 1646 SIR T. BROWNE  \textit{Pseud. Ep.} 26 No farther to abet their opinions then as they are supported by solid reason.    
\P 1649 MILTON  \textit{Eikon.} Wks. 1738, I. 387 The Parlament..had more confidence to abet and own what Sir John Hotham had done.    
\P 1725 WOLLASTON  \textit{Relig. Nat.} §2. 31 That which demands next to be considered..as abetting the cause of truth.

\itembf{4.} esp. in a bad sense: To encourage, instigate, countenance a crime or offence, or anything disapproved of.

\P 1779 JOHNSON  \textit{L.P. Dryden} II. 367 He abetted vice and vanity only with his pen.    
\P 1786 BURKE  \textit{Warren Hastings}, Wks. 1842, II. 214 To abet, encourage, and support the dangerous projects of the presidency of Bombay.    
\P 1849 MACAULAY  \textit{Hist. Eng.} II. 36 Having abetted the western insurrection.    
\P 1876 FREEMAN  \textit{Norm. Conq.} I. v. 286 The invasion was aided and abetted by Richard's subjects.

\itembf{5.} To back up one's forecast of a doubtful issue, by staking money, etc., to bet. Obs.

\P 1630 TAYLOR  (Water P.) \textit{Travels, Ded.} Wks. iii. 76 I doe (out of mine own cognition) auerre and abett that he is senselesse.



\end{myenumerate}


%%%%%%%%%%%%%%%%%%%%%%%%%%%%%%%%
\myitem{thwart} v.

\noindent \phonetic{(θwɔːt)}

\noindent [f. prec. adv.]
\vspace{-0.3cm}

\begin{myenumerate}

\itembf{I. 1.} trans. To pass or extend across from side to side of; to traverse, cross; also, to cross the direction of, to run at an angle to. Obs. or arch.

\P 1413  \textit{Pilgr. Sowle} (Caxton) v. i. (1859) 70 A Cercle embelyfyng somwhat, and thwartyng the thycknes of the spyere.    
\P 1530 PALSGR. 757/2, I thwarte the waye, I go over the waye to stoppe one, je trenche le chemyn.    
\P 1608 SHAKES.  \textit{Per.} iv. iv. 10 Pericles Is now againe thwarting thy wayward seas.    
\P 1627 CAPT. SMITH  \textit{Seaman's Gram.} ix. 39 You set your sailes so sharp as you can to lie close by a wind, thwarting it a league or two,..first on the one boord then on the other.    
\P 1653 R. SANDERS  \textit{Physiogn.} 50 If the Hepatique line be thwarted by other small lines.    
\P 1769 FALCONER  \textit{Dict. Marine} N iij, The current thwarts the course of a ship.    
\P 1805-6 CARY \textit{Dante's Inf.} xxv. 72 The lizard seems A flash of lightning, if he thwart the road.    
\P 1863 P. S. WORSLEY  \textit{Poems \& Transl.} 10 That white reach Thwarting the blue serene, a belt of fire.

\itembf{b.} intr. To pass or extend across, to cross. Obs. or arch.

\P a1552 LELAND \textit{Itin.} (1744) VII. 53 The Towne of Cokermuth stondeth on the Ryver of Coker, the which thwartheth over the Town.    
\P 1598 STOW  \textit{Surv.} xli. (1603) 436 A close cart, bayled ouer and couered with blacke, hauing a plaine white Crosse thwarting.    
\P 1609 HEYWOOD  \textit{Brit. Troy} xiv. xciii, Through the mid-throng the nearest way he thwarted.    
\P 1627 HAKEWILL  \textit{Apol.} Pref. 10 It led them some other way, thwarting, and upon the by, not directly.    
\P 1856 T. AIRD  \textit{Poet.} Wks. 189 They scream, they mix, they thwart, they eddy round.

\itembf{c.} trans. To cross the path of; to meet; to fall in with, come across. Obs.

\P 1601 CHESTER  \textit{Love's Mart., K. Arth.} xx, Merlin..Who by great fortunes chance sir Vlfius thwarted, As he went by in beggers base aray.    
\P 1674 N. FAIRFAX  \textit{Bulk \& Selv.} 146 Motions to be checkt..without the least hit or stop from other bodies that thwart them.    
\P 1812 CARY  \textit{Dante's Par.} iv. 89 Another question thwarts thee.

\itembf{d.} Naut. Of a ship, etc.: To get athwart so as to be foul of. Also intr. Obs.

\P 1809  \textit{Naval Chron.} XXIV. 23 The boat having thwarted against the moorings.    
\P 1810  \textit{Ibid.} XXIII. 97 The frigate now..thwarted the Lord Keith's hawse.    
\P 1813 \textit{Gen. Hist.} in  \textit{Ann. Reg.} 107/1 The Amelia twice fell on board the enemy in attempting to thwart his hawse.

\itembf{2.} To lay (a thing) athwart or across; to place crosswise; to set or put (things) across each other.

   \textbf{thwart over thumb} (quot. 1522) app. = to cross (one) over the thumbs: see thumb n. 5 d.

\P 1522 SKELTON  \textit{Why not to Court} 197 Thus thwartyng ouer thom, He ruleth all the roste.    
\P 1588 SPENSER  \textit{Virgil's Gnat} 514 The noble sonne of Telamon..thwarting his huge shield, Them battell bad.    
\P 1602 CAREW  \textit{Cornwall} i. 25 b, Their bils were thwarted crossewise at the end, and with these they would cut an Apple in two at one snap.    Ibid. 26 b, The inhabitants make use of divers his Creekes, for griste-milles, by thwarting a bancke from side to side.    
\P 1623 MARKHAM  \textit{Cheap Husb.} i. ii. (1631) 14 Carry your rod..in your right hand, the point either directly upright, or thwarted towards your left shoulder.    
\P 1632 LITHGOW  \textit{Trav.} vii. 309 They make..the signe of the Crosse.., thwarting their two foremost fingers.

\itembf{3.} To cross with a line, streak, band, etc. (Only in pa. pple.) Obs. or arch.

\P 1610 J. GUILLIM  \textit{Heraldry} iii. xiv. (1660) 162 The blacke line on the ridge of all Asses backes, thwarted with the like over both the Shoulders.    
\P 1615 G. SANDYS  \textit{Trav.} i. 63 Turbants are made like great globes of callico too, and thwarted with roules of the same.    
\P 1658 J. ROWLAND  \textit{Moufet's Theat. Ins.} 942 The body all over of a yellow colour, except where it is thwarted with cross streaks or lines.    
\P 1861 \textit{Temple  Bar Mag.} II. 256, I saw Vesuvius..thwarted by a golden cloud.

\itembf{b.} To cross-plough; also, to cut crosswise.

\P 1847 \textit{Jrnl.  R. Agric. Soc.} VIII. ii. 318 The burnt earth is then spread on the land and thwarted in (that is, ploughed across the direction in which the land is ploughed when laid up in stetches for sowing).    
\P 1871 COUCH  \textit{Hist. Polperro} vi. 117 Land broken for wheat is thwarted in the Spring.    
\P 1888 ELWORTHY  \textit{W. Somerset Word-bk.} s.v. Thurt, Why, 'tis a wo'th vive shillings to thurt thick there butt.    
\P 1898 RIDER HAGGARD in  \textit{Longm. Mag.} Nov. 38 All my three ploughs were at work ‘thwarting’—that is crossploughing—rootland on the Nunnery Farm.

\itembf{4.} To obstruct (a road, course, or passage) with something placed across; to block. Obs. exc. fig.

\P c1630 RISDON \textit{Surv. Devon} §65 (1810) 63 The rebellious commons..thwarted the ways with great trees.    Ibid. §269. 278 [A stream] whose course is thwarted with a damm, which we call a wear.    
\P 1725 POPE  \textit{Odyss.} x. 72 What Dæmon cou'dst thou meet To thwart thy passage and repel thy fleet?    
\P 1760-72 H. BROOKE  \textit{Fool of Qual.} (1809) IV. 58 They met with a six-barred gate that directly thwarted their passage.    
\P 1807 CRABBE  \textit{Par. Reg.} ii. 72 They sometimes speed, but often thwart our course.    
\P 1856 KANE  \textit{Arct. Expl.} II. v. 60 If no misadventure thwarted his progress.

\itembf{II. 5.} To act or operate in opposition to; to run counter to, to go against; to oppose, hinder. Also absol. Now rare.

\P c1250 \textit{Gen. \& Ex.} 1324 Quat-so  god bad, ðwerted he it neuer a del.    
\P c1430,1530 [implied in  THWARTING vbl. n. 2 and ppl. a. 2].    
\P 1600 HOLLAND  \textit{Livy} xxxv. xxxii. 907 Such as might..not sticke to speake their minds franckly, yea, \& thwart the king his embassadour.    
\P 1671 BP. PARKER  \textit{Def. Eccl. Pol.} iii. §15. 298 To what purpose does he so briskly taunt me for thwarting my own Principles.    
\P 1676 W. ALLEN  \textit{Address Nonconf.} 130 The danger of Schism, and the evil of thwarting publick Laws.    
\P 1783 JUSTAMOND tr.  \textit{Raynal's Hist. Indies} VII. 379 They had unfortunately been so much thwarted by the winds as to prevent their landing before summer.    
\P 1802 PALEY  \textit{Nat. Theol.} xxvi. (1819) 436 General laws, however well set and constituted, often thwart and cross one another.    
\P 1811 L. M. HAWKINS  \textit{C'tess \& Gertr.} II. 370 The countess was not always disposed to thwart and vex: a little flattery would soothe her.

\itembf{b.} intr. To speak or act in contradiction or opposition; to be adverse or at variance, to conflict. Const. with. Now rare or Obs.

\P 1519 W. HORMAN  \textit{Vulg.} 59 b, I wyll nat multyplie wordes or thwarte with the.    
\P 1601 ? MARSTON  \textit{Pasquil \& Kath.} ii. 185 Is't possible that sisters should so thwart In natiue humours?    
\P 1656 \textit{Burton's  Diary} (1828) I. 15 This clause thwarts with his Highness's ordinances.    
\P 1737 BRACKEN  \textit{Farriery Impr.} (1757) II. 272 It would thwart with my intended Brevity.    
\P 1862 F. HALL  \textit{Hindu Philos. Syst.} 42 They also accept..the Smritis, the Puránas, \&c., the work of Rishis, when those books do not thwart with the Veda.

\itembf{6.} trans. To oppose successfully; to prevent (a person, etc.) from accomplishing a purpose; to prevent the accomplishment of (a purpose); to foil, frustrate, balk, defeat. (The chief current sense.)

\P 1581 MULCASTER  \textit{Positions} iv. (1887) 17 He may either proceede at his owne libertie, if nothing withstand him, or may not proceede, if he be thwarted by circunstance.    
\P 1641 EARL OF MONMOUTH  tr. \textit{Biondi's Civil Warres} v. 166 The Earle seeing himselfe twharted, resolved to fight.    
\P 1697 J. LEWIS  \textit{Mem. Dk. Glocester} (1789) 34 From being sometimes a little thwarted, and thro' dissatisfaction, she grew sick.    
\P 1718  \textit{Free-thinker} No. 65 \phonetic{⁋}6 Perpetual Obstacles..thwarted his Designs.    
\P 1803 DK. WELLINGTON in  \textit{Gurw. Desp.} (1837) II. 352 Thus are all our best plans thwarted.    
\P 1849 MACAULAY  \textit{Hist. Eng.} iv. I. 429 The party which had long thwarted him had been beaten down.    
\P 1871 FREEMAN  \textit{Norm. Conq.} IV. xvii. 15 But all these good intentions were thwarted by the inherent vice of his position.
\end{myenumerate}


%%%%%%%%%%%%%%%%%%%%%%%%%%%%%%%%
\myitem{balk} v.1

\noindent \phonetic{(bɔːk)}

\noindent [f. balk, baulk n.1]
\vspace{-0.3cm}

\begin{myenumerate}

\itembf{I. 1.} trans. (and absol.) To make balks in ploughing; to plough up in ridges. Obs.

\P 1393 GOWER  \textit{Conf.} III. 296 But so well halt no man the plough, That he ne balketh other while.    
\P c1420 \textit{Pallad. on Husb.} i. 184 To tille a felde man must have diligence, And balk it not.    
\P 1583 STANYHURST  \textit{Æneis} i. (Arb.) 22 With forck King Neptun is ayding. He balcks thee quicksands, and fluds dooth mollefye.    
\P 1611 COTGR.,  \textit{Assilloner}, to baulke, or plow up in baulkes.    [
\P [a1640 JACKSON \textit{Creed} xi. cxxxix. Wks. XI. 203 Whilst we labour to plough up your hearts..we must not balk that saying of St. John.]

\itembf{II. 2.} trans. To miss or omit intentionally. a.II.2.a lit. To pass by (a place), to avoid in passing; to shun.

\P 1484 PASTON  \textit{Lett.} 859 III. 279 Mastyer Baley..woold not have balkyd this pore loggeyng to Norwyche wardes.    
\P 1612-5 BP. HALL \textit{Contempl. N.T.} iv. iii. 173 Jericho was in his way from Galilee to Jerusalem: he baulks it not, though it were outwardly cursed.    
\P 1684 LADY  R. RUSSELL \textit{Lett.} I. xv. 43, I hope you will not balk Totteridge, if I am here.    
\P a1733 North Exam. ii. iv. \phonetic{⁋}94 Going to Lord Clarendon..baulking the Secretary.    
\P 1783 AINSWORTH  \textit{Lat. Dict.} (Morell) s.v. Balk, I will not balk your house.

\itembf{b.} fig. To pass over, overlook, refrain from noticing (what comes in one's way); to shirk, ignore.

\P c1440 \textit{Promp. Parv.} 22 Balkyn, or ouerskyppyn, omitto.    
\P 1582 FLEETWOOD in  \textit{Ellis Orig. Lett.} ii. 216 III. 90 As for my Lo. Maior..I am dryven every daie to bawk hym and his doynges.    
\P 1640 BP. HALL  \textit{Episc.} i. §11. 39, I may not baulke two pregnant testimonies of the Fathers.    
\P 1656 SANDERSON  \textit{Serm.} II. 160 The spying of motes in our brother's eye, and baulking of beams in our own.    
\P 1684 \textit{Cont.  Foxe's A. \& M.} III. 900 The Bayliff would fain have baulked him, As if he had not seen him.    
\P 1742 RICHARDSON  \textit{Pamela} III. 42 Let me tell you, (nor will I balk it) my Brother..will want one Apology for his Conduct.    
\P 1848 L. HUNT  \textit{Jar of Honey} Pref. 4 No topic is baulked if it come uppermost.

\itembf{c.} To refuse (anything offered or that comes in course, e.g. food or drink).

\P 1587 TURBERV.  \textit{Trag. T.} (1837) 230 And balke your bed for shame.    
\P 1619 FLETCHER  \textit{M. Thomas} i. i. 386 A bait you cannot balk Sir.    
\P 1649 W. BLITHE  \textit{Eng. Improv. Impr.} (1653) 183 If the stalk grow big, cattell will balk it.    
\P a1784 JOHNSON in \textit{Boswell} (1831) I. 236, I never..balked an invitation out to dinner.    
\P 1810 CRABBE  \textit{Borough} xvi, He took them all and never balk'd his glass.

\itembf{d.} To avoid (a duty or responsibility).

\P 1631 PRESTON  \textit{Effect. Faith} 146 Thou must not balke the way of Religion, because of the troubles thou meetest.    
\P a1707 BEVERIDGE \textit{Priv. Th.} ii. 103 Not that we should run ourselves into danger, but that we should baulk no Duty to avoid it.    
\P 1785 COWPER  \textit{Tirocin.} 257 Such an age as ours baulks no expence.

\itembf{e.} To let slip, fail to use, seize, keep, reach, etc.

\P 1601 SHAKES.  \textit{Twel. N.} iii. ii. 26 This was look't for at your hand, and this was baulkt.    
\P 1697 DRYDEN  \textit{Virg. Georg.} Ded. If I balk'd this opportunity.    
\P 1724 A. RAMSAY  \textit{Tea-t. Misc.} (1733) I. 2 This point of a' his wishes He wadna with set speeches bauk.    
\P 1826 HOR. SMITH  \textit{Gai. \& Grav.} in \textit{Casquet of Lit.} I. 326/2 My adviser insisted upon my not baulking my luck.

\itembf{3. a.} intr. To stop short as at an obstacle, to pull up, swerve. Esp. of a horse: To jib, refuse to go on, or to leap, to shy; also of the rider, and of any one on foot, refusing a leap. Also fig. (colloq.) to shy or jib at.

\P 1481 CAXTON  \textit{Reynard} (Arb.) 32 Isegrym balked and sayde, ye make moche a doo, sir Tybert.    
\P 1596 SPENSER  \textit{F.Q.} iv. x. 25 Ne ever ought but of their true loves talkt, Ne ever for rebuke or blame of any balkt.    
\P 1722 DE FOE  \textit{Moll. Fl.} (1840) 78 If he balked, I knew I was undone.    
\P 1756 C. LUCAS  \textit{Ess. Waters} III. 340 No man, that drinks water, baulks at a pint..in the day.    
\P 1843 LEVER  \textit{J. Hinton} xxv, Burke..suddenly swerved his horse round, and affecting to baulk, cantered back.    
\P 1862 \textit{Melbourne  Leader} 5 July, His horse balked at a leap, and threw him.    
\P 1908 J. M. DILLON  \textit{Motor Days Eng.} xx. 241 It was the only time I ever saw Maud balk at gooseberries.

\itembf{b.} To lie out of the way. Obs.

\P 1591 SPENSER  \textit{M. Hubberd} 268 Labour that did from his liking balke.

\itembf{4.} trans. To miss by error or inadvertence. Obs.

\P 1579 SPENSER  \textit{Sheph. Cal.} Sept. 93 They..balk the right way, and strayen abroad.    
\P 1659 FELTHAM  \textit{Low Countr.} (1677) 46 You cannot baulk your Road without the hazard of drowning.    
\P 1710 PALMER  \textit{Proverbs} 6 Young dogs..balk the true game to ply every scent.

\itembf{III. 5.} trans. To place a balk in the way of. a.III.5.a To check, hinder, thwart (a person or his action).

\P 1589 WARNER  \textit{Alb. Eng.} vi. xxxi. (1612) 153, I sometimes proffered kindnesse..but..was balked with a blush.    
\P 1635 SWAN  \textit{Spec. M.} v. §2 (1643) 105 The King..must not be baulked in his late proceedings.    
\P 1726 DE FOE  \textit{Hist. Devil} i. xi. (1840) 155 An enemy who is baulked and defeated, but not overcome.    
\P 1821 BYRON  \textit{Two Foscari} i. i, They shall not balk my entrance.    
\P 1855 PRESCOTT  \textit{Philip II}, I. ii. xiii. 292 The sturdy cavalier was not to be balked in his purpose.

\itembf{b.} To check (feelings, or a person in his feelings).

\P 1682 DRYDEN  \textit{Rel. Laici} 212 Nor doth it balk my charity to find The Egyptian Bishop of another mind.    
\P 1746 LD. MALMESBURY  \textit{Lett.} I. 37 Lord Talbot was not much baulked with this rebuke.    
\P 1855 H. MARTINEAU  \textit{Autobiog.} I. 92 My home affections..all the stronger for having been repressed and baulked.

\itembf{c.} To disappoint (expectations, or any one in his expectations).

\P 1590 MARLOWE  \textit{Edw. II}, ii. v, We..must not come so near to balk their lips.    
\P 1652 BROME  \textit{Jov. Crew} ii. 389 May your Store Never decay, nor baulk the Poor.    
\P 1725 POPE  \textit{Odyss.} x. 135 Balk'd of his prey, the yelling monster flies.    
\P 1854 THACKERAY  \textit{Newcomes} I. 286 Balk yourself of the pleasure of bullying.    
\P 1873 SPENSER  \textit{Stud. Sociol.} vii. 161 Time after time our hopes are balked.

\itembf{d.} To frustrate, foil, render unsuccessful.

\P 1635 QUARLES  \textit{Emblems} iii. xiv. (1718) 182 To baulk those ills which present joys bewray.    
\P 1727 SWIFT  \textit{Censure Misc.} (1735) V. 104 The most effectual Way to baulk Their Malice, is to let them talk.    
\P 1848 KINGSLEY  \textit{Saint's Trag.} ii. v. 90 With which we try to balk the curse of Eve.

\itembf{6.} trans. and absol. To meet arguments with objections; to quibble, chop logic, bandy words.

\P 1596 SPENSER  \textit{F.Q.} iii. ii. 12 Her list in stryfull termes with him to balke.    
\P 1596 SHAKES.  \textit{Tam. Shr.} i. i. 34 Balke Lodgicke with acquaintaince that you haue.    
\P 1653 MANTON  \textit{Exp. James} iii. 2 Wks. IV. 227 They do not divide and baulk with God.
\end{myenumerate}

%%%%%%%%%%%%%%%%%%%%%%%%%%%%%%%%%
\myitem{brisk} a. and n.

\noindent \phonetic{(brɪsk)}

\noindent [First found in end of 16th c.; evidently familiar to Shakespeare and his contemporaries. Derivation uncertain: Welsh brysg (used of briskness of foot) occurs in a poem of the 14th c. This appears to answer in form to OIr. brisc, Ir. briosg, Gael. brisg, Breton bresk, ‘brittle’, ‘crumbly’; but it is not easy to connect the senses.

   It is however possible that brisk is identical with F. brusque (which appears as bruisk in Sc. c 1560, and as bruske as early as 1600); at least Cotgr. gives brisk as a translation of brusque, and the words appear to have influenced each other in early use. See brusque.]
\vspace{-0.3cm}

\begin{myenumerate}

\itembf{A.} adj.
\itembf{1.} Sharp or smart in regard to movement (in a praiseworthy sense) quick and active, lively. a.A.1.a of persons. (Sometimes used of disposition = ‘cheery, sprightly, lively’, but this is now chiefly dial.)

\P [1560 T. ARCHBALD  \textit{Let.} in Keith \textit{Hist. Scotl.} (1734) 489 (Jam.) Thir ar the imbassadoris..thai depart wondrous bruisk.]    
\P 1592 SHAKES.  \textit{Rom. \& Jul.} i. v. 16 Chearly Boyes, Be brisk awhile.    
\P 1611 COTGR.,  \textit{Brusque}, briske, liuely, quicke, etc.    Ibid. Frisque, friske, liuely, iolly, blithe, briske, fine, spruce, gay.    
\P 1613 R. C. TABLE  \textit{Alph.}, Brisque, quick, liuely, fierce.    
\P 1725 DE FOE  \textit{Voy. round World} (1840) 298 A company of bold, young brisk fellows.    
\P 1828 SCOTT  \textit{F.M. Perth} I. 5 The brisk, alert agent of a great house in the city.    
\P 1882 C. PEBODY  \textit{Eng. Journalism} xvi. 120 A bright, brisk lad, fresh from Oxford.

\itembf{b.} of actions and motions. (The prevalent modern use.)

\P 1684 BUNYAN  \textit{Pilgr.} ii. 101 To enter with him a brisk encounter.    
\P 1690 LOCKE  \textit{Hum. Und.} iv. xi. §5 It must needs be some exteriour Cause, and the brisk acting of some Objects without me.    
\P 1756 BURKE  \textit{Subl. \& B.} Wks. I. 245 A slow and languid motion [of the eye] is more beautiful than a brisk one.    
\P 1777 WATSON  \textit{Philip II} (1839) II. 213 He made a brisk attack upon one of the gates.    
\P 1855 PRESCOTT  \textit{Philip II}, I. i. vii. 91 He..opened a brisk cannonade on the enemy.    
\P 1863 GEO. ELIOT  \textit{Romola} ii. xxii, The brisk pace of men who had errands before them.

\itembf{c.} of trade: Active, lively.

\P 1719 W. WOOD  \textit{Surv. Trade} 339 When Trade is brisk, Money..is more in view.    
\P 1832 H. MARTINEAU  \textit{Hill \& Vall.} iv. 49 The demand for iron was so brisk.    
\P 1833   \textit{Br. Creek} iii. 64 A brisk traffic took place in the remaining articles.

\itembf{d.} of wind, fire, etc.

\P 1725 POPE  \textit{Odyss.} xii. 184 Up sprung a brisker breeze.    
\P 1759 ROBERTSON  \textit{Hist. Scot.} I. iii. 203 At last a brisk gale arose.    
\P 1796 MORSE  \textit{Amer. Geog.} I. 133 New and brisk fountains of water rise at spring tides.    
\P 1837 M. DONOVAN  \textit{Dom. Econ.} II. 269 The brisk fire should..be only employed when the meat is half roasted.

\itembf{e.} of purgatives.

\P 1799  \textit{Med. Jrnl.} II. 236 He had a brisk cathartic given him.    
\P 1815  \textit{Scribbleomania} 207 note, They've drench'd her with cathartics brisk.

\itembf{2.} In allied senses, chiefly unfavourable. \textbf{a.} Sharp-witted, pert; curt. \textbf{b.} ‘Fast’ of life. \textbf{c.} Over hasty. \textbf{d.} Unpleasantly sharp of tone. (With c, d, cf. Fr. brusque.) \textbf{e.} Quickly passing, brief.

\P 1601 SHAKES.  \textit{Twel. N.} ii. iv. 6 These most briske and giddy-paced times.    
\P 1665 GLANVILL  \textit{Sceps. Sci. Addr.} 13 Divers of the brisker Geniusses, who desire rather to be accounted Witts, then endeavour to be so.    
\P 1667 EVELYN in  \textit{Four C. Eng. Lett.} 108 The smoothest or briskest strokes of his Pindaric lyre.    
\P 1667 PEPYS  \textit{Diary} (1877) V. 422 The Surveyor began to be a little brisk at the beginning.    
\P a1674 CLARENDON  \textit{Hist. Reb.} I. i. 8 When that brisk and improvident Resolution was taken.    
\P 1676 G. ETHEREGE  \textit{Man of Mode} i. i. (1684) 11 He has been, as the sparkish word is, Brisk Upon the Ladies already.    
\P 1700 \textit{Penn.  Archives} I. 138, I send yee ye Coots [= Court's] Lettr wch is very brisk.    
\P 1739 CIBBER  \textit{Apol.} vii. 214 The briskest loose Liver or intemperate Man.    [
\P 1879 BROWNING  \textit{Ned Bratts} 23 Some trial for life and death, in a brisk five minutes' space.]

\itembf{3.} Smartly or finely dressed; spruce. Obs.

\P 1590 MARLOWE  \textit{Edw. II}, i. iv. ad fin., I have not seen a dapper jack so brisk.    
\P 1596 SHAKES.  \textit{1 Hen. IV,} i. iii. 54 To see him shine so briske, and smell so sweet.    
\P 1603 PATIENT  \textit{Grissil} 17 My brisk spangled baby will come into a stationer's shop.

\itembf{4.} Of liquors: Agreeably sharp or smarting to the taste; effervescent, as opposed to ‘flat’ or ‘stale’. (So It. brusco, Fr. vin brusque in Cotgr.) Similarly of the air: Fresh, keen, stimulating.

\P 1597 SHAKES.  \textit{2 Hen. IV}, v. iii. 48 A Cup of Wine, that's briske and fine.    
\P 1697 POTTER  \textit{Antiq. Greece} iii. ix. (1715) 75 Brisk Wines and Viands animate Their Souls.    
\P 1741 BROWNRIGG in  \textit{Phil. Trans.} LV. 242 The brisk and pungent taste of the acidulæ.    
\P 1776 SIR W. FORBES in \textit{Boswell Johnson} II. 404 A bottle of beer..is made brisker by being set before the fire.    
\P 1837 DISRAELI  \textit{Venetia} i. ii, The air was brisk.    
\P 1846 J. JOYCE  \textit{Sci. Dialogues} vii. 213 You see of what importance air is to give to all our liquors their pleasant and brisk flavour.    
\P 1877 L. MORRIS  \textit{Epic Hades} ii. 198.

\itembf{5.} Sharp to other senses; distinct, vivid. \textbf{a.} to the hearing. Obs.

\P 1660 BOYLE  \textit{New Exp. Phys.-Mech.} i. 21 There is..produced a considerably brisk noise.    
\P 1667 PRIMATT  \textit{City \& C. Build.} 51 Bricks well burnt..if you strike them with any thing, will make a brisk sound.

\itembf{b.} to the sight. Obs.

\P a1727 NEWTON  (J.) Had it [my instrument] magnified thirty or twenty-five times, it had made the object appear more brisk and pleasant.

\itembf{6.} Comb. \textbf{a.} adverbial, as brisk-going, brisk sparkling; \textbf{b.} parasynthetic, as brisk-spirited.

\P 1711  \textit{Lond. Gaz.} No. 4868/4 A..Cart Horse..brisk Spirited.    
\P 1831 CARLYLE  \textit{Sart. Res.} ii. iii. 132 Like a strong brisk-going undershot-wheel.    
\P 1837   \textit{Fr. Rev.} II. iii. i. 128 Our brisk-sparkling assiduous official person.

\itembf{B.} n. a.B.a A ‘brisk’ or smart person; a gallant, a fop. (Cf. A3 above.) b.B.b A lively, forward woman, a wanton.

\P 1621 BURTON  \textit{Anat. Mel.} iii. iii. i. ii. (1651) 604 A yong gallant..a Fastidious Brisk, that can wear his cloaths well in fashion.    
\P 1689 N. LEE  \textit{Princ. of Cleve} (N.) The forward brisk, she that promis'd me the ball assignation.
\end{myenumerate}


%%%%%%%%%%%%%%%%%%%%%%%%%%%%%%%%%
\myitem{supine} a.

\noindent \phonetic{(ˈs(j)uːpaɪn, formerly s(j)uːˈpaɪn)}

\noindent [ad. L. supīnus (whence OF. souvin, Pr. sobi(n), supi(n), It., Sp., Pg. supino), f. Italic *sup-, root of super above, superus higher: see -ine1.]
\vspace{-0.3cm}

\begin{myenumerate}

\itembf{1.} Lying on one's back, lying with the face or front upward. Also said of the position. Often predicatively or quasi-advb.

   Sometimes used loosely for ‘lying, recumbent’.

\P c1500 KENNEDY  \textit{Passion of Christ, At Cumplin Tyme} 1290 Apoun his bak he did ly on suppyne.    
\P 1615 CROOKE  \textit{Body of Man} 268 The position or manner of lying of the sickeman, eyther prone that is downeward, or supine that is vpward.    
\P 1646 SIR T. BROWNE  \textit{Pseud. Ep.} iv. vi. 193 That women drowned swim prone but men supine, or upon their backs, are popular affirmations, whereto we cannot assent.    
\P 1658   \textit{Hydriot.} iv. 21 They buried their dead on their backs, or in a supine position.    
\P 1700 DRYDEN  \textit{Ceyx and Alcyone} 295 Where lay the God And slept supine, his Limbs display'd abroad.    
\P 1715 POPE  \textit{Iliad} iv. 603 Supine he tumbles on the crimson sands.    
\P a1788 POTT  \textit{Chirurg.} Wks. II. 57 When the patient is in a supine posture.    
\P a1806 H. K. WHITE  \textit{‘Ye unseen Spirits’} 4 As by the wood-spring stretch'd supine he lies.    
\P 1876 \textit{Trans.  Clinical Soc.} IX. 72 Having placed the patient in the supine position.    
\P 1881 J. PAYN  \textit{Grape from Thorn} xi, The ancient Romans, taking their meals, as they did, supine, and resting on one elbow.

\itembf{b.} Of the hand or arm: With the palm upward; supinated.

\P 1668 CULPEPPER \& COLE  \textit{Barthol. Anat.} iv. viii. 165 The Radius makes the whole Arm prone or supine.    
\P 1865 TYLOR  \textit{Early Hist. Man.} iii. 48 The rustic Phidyle should hold out her supine hands.    
\P 1868 LIVINGSTONE  \textit{Last Jrnls.} 15 Nov. (1873) I. 346 The Africans all beckon with the hand, to call a person, in a different way from what Europeans do. The hand is held, as surgeons say, prone, or palm down, while we beckon with the hand held supine, or palm up.

\itembf{c.} (a) Of a part of the body: Situated so as to be upward; upper, superior.

\P 1661 LOVELL  \textit{Hist. Anim. \& Min.} b5, Their finns are foure, two in the prone part, two in the supine, \& circumvallate round.    Ibid., The eyes [of fishes] are in the supine part of their heads.    
\P 1826 KIRBY \& SP.  \textit{Entomol.} xxxiv. III. 415, I have seen a fly turn its head completely round, so that the mouth became supine and the vertex prone.    Ibid. xlvi. IV. 268 Supine Surface... The upper surface.

(b) Bot. See quot., and cf. procumbent a. 2.

\P 1853 MACDONALD \& ALLAN  \textit{Bot. Wordbk.} 32 Supine... The face of a leaf is called the supine disc.

\itembf{d.} transf. Sloping or inclining backwards. poet.

\P 1697 DRYDEN  \textit{Virg. Georg.} ii. 373 If the Vine On rising Ground be plac'd, or Hills supine, extend thy loose Battalions.    
\P 1817 SHELLEY  \textit{Rev. Islam} xii. xxi. 4 The prow and stern did curl, Horned on high, like the young moon supine.

\itembf{2.} fig. Morally or mentally inactive, inert, or indolent.

\P 1603 [implied in  SUPINELY 2].    
\P 1621 BURTON  \textit{Anat. Mel.} ii. i. iv. ii. 301 Through their..contempte, supine negligence, extenuation, wretchednes \& peeuishnesse, they vndoe themselues.    
\P 1630 DONNE  \textit{Serm. Easter-day} (1640) 246 So also did they fall under the rebuke and increpation of the Angell for another supine inconsideration.    
\P 1650 SIR E. NICHOLAS in \textit{N. Papers} (Camden) I. 198 The Pr. of Orange..died..of the Small Pox thro' the supine negligence or worse of some of his Physicians.    
\P 1732 BERKELEY  \textit{Alciphr.} iv. §13 The lazy supine airs of a fine gentleman.    
\P 1761 HUME  \textit{Hist. Eng.} lv. (1806) IV. 225 They lived in the most supine security.    
\P 1779 BOSWELL  \textit{Let. to Johnson} 17 July, A supine indolence of mind.    
\P 1807 JEFFERSON  \textit{Writ.} (1830) IV. 72 The first ground of complaint was the supine inattention of the administration.    
\P 1819 SHELLEY  \textit{Cenci} iv. iv. 181 The supine slaves Of blind authority.    
\P 1852 THACKERAY  \textit{Esmond} i. v, He wakened up from the listless and supine life which he had been leading.

advb. \P 1615 G. SANDYS  \textit{Trav.} i. 36 So supine negligent are they.

\itembf{b.} supine of: indifferent to, negligent of. (Cf. listless a.) Obs. rare.

\P 1724 WELTON  \textit{Chr. Faith \& Pract.} 195 A profane..mind that is altogether supine of religion.

\itembf{c.} Not active; passive.

\P 1843 RUSKIN  \textit{Mod. Paint.} ii. v. iii. §21 The stream in their hands looks active, not supine, as if it leaped, not as if it fell.    
\P 1878 H. S. WILSON  \textit{Alpine Ascents} i. 11 In which the body is supine while the fancy remains active.
\end{myenumerate}


%%%%%%%%%%%%%%%%%%%%%%%%%%%%%%%%%
\myitem{indolent} a. (n.)

\noindent \phonetic{(ˈɪndəʊlənt)}

\noindent [ad. late L. indolēnt-em (Jerome: ‘dicamus ἀπηλγήκοτες indolentes sive indolorios’), f. in- (in-3) + dolēns grieving, dolent. Cf. F. indolent (16-17th c.).]
\vspace{-0.3cm}

\begin{myenumerate}

\itembf{1.} Path. Causing no pain, painless; esp. in indolent tumour, indolent ulcer.

\P 1663 BOYLE  \textit{Usef. Exp. Nat. Philos.} ii. i. 25 Curing of cancers..by the outward application of an indolent powder.    
\P 1713 R. RUSSELL in  \textit{Phil. Trans.} XXVIII. 277 An Indolent Tumour in her Breast.    
\P 1783 POTT  \textit{Chirurg.} Wks. II. 286 As he lay on his back, it was perfectly indolent; but in an erect posture..he complained of pain.    
\P 1804 ABERNETHY  \textit{Surg. Obs.} 58, I was led to inquire further, whether the surface might not be sometimes irritable and sometimes indolent.    
\P 1861 HULME tr.  \textit{Moquin-Tandon} ii. iii. iii. 133 Ceratum Cantharidis..is used to..stimulate issues and indolent ulcers.

\itembf{b.} loosely. Of a pain: Very slight. Obs.

\P 1758 J. S. LE  \textit{Dran's Observ. Surg.} (1771) 155 He felt an indolent Pain on the Shoulder.

\itembf{2.} Of persons, their disposition, action, etc.: Averse to toil or exertion; slothful, lazy, idle.

\P 1710 STEELE  \textit{Tatler} No. 132 \phonetic{⁋}4 A good-natured indolent Man.    
\P 1711 ADDISON  \textit{Spect.} No. 5 \phonetic{⁋}1 To gratifie the Senses, and keep up an indolent Attention in the Audience.    
\P 1744 H. WALPOLE  \textit{Lett. H. Mann} (1834) I. xciv. 324, I am naturally indolent and without application to any kind of business.    
\P 1839 LONGFELLOW  \textit{Hyperion} i. vi, An easy and indolent disposition.    
\P 1885 S. COX  \textit{Exposit. Ser.} i. ix. 112 [To] rouse the indolent and indifferent.

\noindent transf. \P 1839 LONGFELLOW  \textit{Hyperion} iii. i, Through the meadow winds the river—careless, indolent.

\itembf{B.} n. An indolent person. Obs.

\P 1720 \textit{Humourist}  49 The Indolent remains in Suspense and Anguish.    
\P 1810 \textit{Splendid  Follies} I. 144 ‘Yes, yes, I see her’, replied the fair indolent.

\noindent Hence \phonetic{ˈindolentness} (Bailey vol. II, 1727).
\end{myenumerate}

%%%%%%%%%%%%%%%%%%%%%%%%%%%%%%%%
\myitem{reify} v.

\noindent \phonetic{(ˈriːɪfaɪ, ˈreɪɪf-)}

\noindent [f. as reification + -ify.]
\vspace{-0.3cm}

trans. To convert mentally into a thing; to materialize.

\P 1854  \textit{Fraser's Mag.} LXIX. 75 The gods of their final and accepted polytheism were, in point of fact, only those sublimer portions of nature which..they had not yet dared to reify.    
\P 1882 \textit{Pop.  Sci. Monthly} XXI. 151 When people make or find a new ‘abstract noun’, they instantly try to put it on a shelf or into a box, as though it were a thing; thus they reify it.    
\P 1931 M. R. COHEN  \textit{Reason \& Nature} iii. iii. 390 There is..a fundamental philosophic issue: the extent to which the principle of unity should be hypostatized or reified (I wish the use of the word thingified were more common).    
\P 1953 C. E. OSGOOD  \textit{Method \& Theory} in \textit{Experim. Psychol.} xvi. 680 The second hindrance to objectivity is the ubiquitous tendency to reify the word, to assume the word itself some$\sim$how carries its own meaning.    
\P 1971  \textit{Times Lit. Suppl.} 31 Dec. 1619/3 To look upon them [sc. economic laws] as objective necessities, as bourgeois economists do, is to reify them.    
\P 1979 E. H. GOMBRICH  \textit{Sense of Order} x. 282 The temptation to ‘reify’ the shield into the open mouth of a gaping mask..proved as irresistible as did the opportunity of turning spiralling volutes into suggestions of eyes.

\noindent Hence \phonetic{ˈreified} ppl. a., \phonetic{ˈreifying} vbl. n. and ppl. a.

\P 1941 H. MARCUSE  \textit{Reason \& Revol.} iv. 115 Lordship and bondage result of necessity from certain relationships of labor, which are, in turn, relationships in a ‘reified’ world.    
\P 1962 MACQUARRIE \& ROBINSON tr.  \textit{Heidegger's Being \& Time} ii. vi. 487 Why does this reifying always keep coming back to exercise its dominion?    
\P 1965 B. PEARCE tr.  \textit{Preobrazhensky's New Economics} 47 One can..understand its laws in the spirit of vulgar economics, that is, by offering in the guise of science mere superficial description, complete with the reified relations of commodity production.    
\P 1969 R. BLACKBURN in  \textit{Cockburn \& Blackburn Student Power} 207 An alienated society naturally encourages a re-ifying vocabulary.    
\P 1979 E. H. GOMBRICH  \textit{Sense of Order} ix. 242 It is surely not far-fetched to interpret its coiling frame as a reified flourish on a reified support.



%%%%%%%%%%%%%%%%%%%%%%%%%%%%%%%%
\myitem{sway} v.

\noindent \phonetic{(sweɪ)}

\noindent [\phonetic{Properly two distinct words. (1) ME. sweȝe (14th c.), conjugated strong and weak, also swye, to go, move (cf. ME. forsueie to go astray), may have been a native word orig. of the OE. type *sweᴁan, (3 pres. ind. *swiᴁeþ), pa. tense *swæᴁ, parallel to OE. weᴁan to move, carry, weigh, (wiᴁeþ), wæᴁ, ME. weȝe, occas. wye, pa. tense weȝe, wei(ȝ), wei(e)de. (Cf. also the parallelism of swag and wag, sweight and weight.) Formally, sweᴁe might also be ad. ON. sveigja to bend (a bow), swing (a distaff), etc., give way, yield (cf. sveigr switch, twig), causative vb. f. svig-, in svig bend, curve, svigi switch, svigna to give way; but the ME. and ON. verbs do not agree in sense. (2) The modern sway dates only from c 1500, and agrees in form and sense with, and appears to be ad., LG. swâjen to be moved hither and thither by the wind (whence Sw. svaja to swing, Da. svaie to move to and fro, G. schwaien, schweien), Du. zwaaien to swing, wave, walk totteringly, slant, bevel.}]
\vspace{-0.3cm}

\begin{myenumerate}

\itembf{I. 1.} intr. To go, move. Obs.

\P 13..  \textit{E.E. Allit. P.} B. 87 Swyerez þat swyftly swyed on blonkez.    Ibid. C. 72 Now sweȝe me þider swyftly \& say me þis arende.    Ibid. 151 Þe sayl sweyed on þe see.    
\P 13..  \textit{Gaw. \& Gr. Knt.} 1429 Al in  a semblé sweyed to-geder.
\P ?a1400  \textit{Morte Arth.} 57 [He] Sweys in-to Swaldye wiþ his snelle houndes.

\itembf{b.} Often with down: To go down, fall (lit. and fig.); spec. to fall or sink into a swoon. Obs.

\P 13..  \textit{Gaw. \& Gr. Knt.} 1796 Sykande  ho sweȝe doun, \& semly hym kyssed.    
\P 13..  \textit{E.E. Allit. P.} B. 956 Þe rayn rueled adoun..Of felle flaunkes of fyr..Swe aboute sodamas.    Ibid. C. 429 Þe soun of oure souerayn þen swey in his ere.
\P ?a1400  \textit{Morte Arth.} 1467 So many sweys in  swoghe swounande att ones!    Ibid. 3676 With þe swynge of þe swerde sweys þe mastys.    
\P c1400  \textit{Destr. Troy} 9454 Parys..Sweyt into swym, as he swelt wold.    
\P a1400-50  \textit{Wars Alex.} 2057 (Dublin), Þe power owt of perse..Sweyd sleghtly downe slayn of þair blonkes.    
\P c1415 \textit{Crowned  King} 29 Swythe y swyed in a sweem þat y swet after.    
\P 1513 DOUGLAS  \textit{Æneis} ii. x. 86 Quhar thir towris thou seis doun fall and sweye, And stane fra stane doun bet.    
\P 1533 BELLENDEN  \textit{Livy} iv. xv. (S.T.S.) II. 103 Þe hewmondis of romanis semyt as þai war sweyand doun.

\itembf{c.} causative. To cause to go or move; to drive. Obs. rare.

\P 13..  \textit{E.E. Allit. P.} C. 236 Styffe stremes..Þat drof hem dryȝlych adoun þe depe to serue, Tyl a swetter ful swyþe hem sweȝed to bonk.

\itembf{II. 2.} intr. To move or swing first to one side and then to the other, as a flexible or pivoted object: often amplified by phr., e.g. backwards and forwards, to and fro, from side to side.

   Not common before the 19th century.

\P c1500  \textit{Bk. Mayd Emlyn} 334 in Hazl. \textit{E.P.P.} IV. 94 An halfepeny halter made hym fast, And therin he swayes.    
\P 1555 EDEN  \textit{Decades} (Arb.) 120 Yet are they [sc. the branches of the trees] tossed therewith, and swaye sumwhat from syde to syde.    
\P 1797 S. \& HT.  LEE \textit{Canterb. T.} (1799) I. 375 The lamp swayed with the blast.    
\P 1859 TENNYSON  \textit{Marr. Geraint} 171 A purple scarf, at either end whereof There swung an apple of the purest gold, Sway'd round about him as he gallop'd up.    
\P 1863 MRS. OLIPHANT  \textit{Salem Chapel} x, That stick over which his tall person swayed with fashionable languor.    
\P 1874 L. STEPHEN \textit{Hours in  Libr.} (1892) II. ii. 51 The dreary estuary, where the slow tide sways backwards and forwards.

\itembf{b.} fig. To vacillate. rare.

\P 1563 WINȝET tr.  \textit{Vincent. Lirin.} xv. Wks. (S.T.S.) II. 35 Thai, sweand and swounand betuix thame twa, determinatis nocht quhat wes specialie erast to be chosin be thame.    
\P 1825 JAMIESON,  \textit{Swee},..to be irresolute.    
\P 1871 B. TAYLOR  \textit{Faust} (1875) II. i. i. 5 When the crowd sways, unbelieving.

\itembf{3.} trans. To cause to move backward and forward or from side to side (cf. 2). (See also 13.)

   Not common before the 19th century.

\P 1555 EDEN  \textit{Decades} (Arb.) 152 Swayinge her bodye twyse or thryse too and fro.    
\P 1667 MILTON  \textit{P.L.} iv. 983 As when a field Of Ceres ripe for harvest waving bends Her bearded Grove of ears, which way the wind Swayes them.    
\P 1717 PRIOR  \textit{Alma} ii. 215 Have you not seen a Baker's Maid Between two equal Panniers sway'd?    
\P 1784 COWPER  \textit{Task} vi. 73 The roof,..moveable through all its length As the wind sways it.    
\P 1819 SHELLEY  \textit{Julian} 276 The ooze and wind Rushed through an open casement, and did sway His hair.    
\P 1865 TROLLOPE  \textit{Belton Est.} xii. 137 He swayed himself backwards and forwards in his chair, bewailing his own condition.    
\P 1902 R. BAGOT  \textit{Donna Diana} xv. 178 When the cool breeze sweeps up from the sea, gently swaying the tops of the cypress-trees.

\itembf{b.} fig.

\P a1586 SIDNEY  \textit{Arcadia} ii. xxix. (1912) 330 He was swayed withall..as everie winde of passions puffed him.    
\P 1592 W. WYRLEY  \textit{Armorie, Ld. Chandos} 29 Some turning fate, Which like wild whirlwind all our dooings sweath.    
\P 1596 SHAKES.  \textit{Merch. V.} iv. i. 51 Affection, Maisters [? = Mistress] of passion, swayes it to the moode Of what it likes or loaths.    
\P a1650 MAY  \textit{Old Couple} ii. i. (1658) C2, He has got A great hand over her, and swayes her conscience Which way he list.    
\P 1866 G. MACDONALD  \textit{Ann. Q. Neighb.} xv. (1878) 307, I was swayed to and fro by the motions of a spiritual power.    
\P 1870  \textit{Edin. Rev.} Oct. 388 Dr. Newman..tells us..with the utmost frankness, the persons who..swayed his beliefs hither and thither.

\itembf{4.} intr. To bend or move to one side, or downwards, as by excess of weight or pressure; to incline, lean, swerve.

   In mod. quots. only a contextual use of 2.

\P 1577 HOLINSHED  \textit{Chron.} II. 1624/1 The left side of the enimies..was..compelled to sway a good way backe, and giue grounde largely.    
\P 1593 SHAKES.  \textit{3 Hen. VI}, ii. v. 5.    
\P 1610 BOYS  \textit{Wks.} (1622) 223 The tree falleth as it groweth..Learne then in growing to sway right.    
\P 1624 BACON  \textit{Consid. War w. Spain} Wks. 1879 I. 542/1  In these personal respects, the balance sways on our part.    
\P 1631 GOUGE  \textit{God's Arrows} iii. §48. 273 Aaron and Hur..kept his hands that they could not sway aside one way or other.    
\P 1670-1 NARBOROUGH JRNL. in  \textit{Acc. Sev. Late Voy.} i. (1694) 166 Could not get the Ship off, for the Water did Ebb, and the Ship Sued above 3 Foot.

\P 1860 TYNDALL  \textit{Glac.} i. xxvii. 196 The carriage swayed towards the precipitous road side.    
\P 1881 ‘RITA’  \textit{My Lady Coquette} xv, She sways towards him like a reed.

\itembf{b.} transf. To have a certain direction in movement; to move. Obs.

\P 1597 SHAKES.  \textit{2 Hen. IV,} iv. i. 24 Let vs sway-on, and face them in the field.    
\P 1601  \textit{ Twel. N.} ii. iv. 32 So swayes she leuell in her husbands heart.    
\P 1605  \textit{ Macb.} v. iii. 9 The minde I sway by, and the heart I beare, Shall neuer sagge with doubt, nor shake with feare.    
\P 1650 W. D. tr.  \textit{Comenius' Gate Lat. Unl.} §233 Man's estate swaieth (is going downwards) [L. vergit] towards a declining age.

\itembf{c.} To move against in a hostile manner. rare.

\P 1590 SPENSER  \textit{F.Q.} ii. viii. 46 How euer may Thy cursed hand so cruelly haue swayd Against that knight.    Ibid. x. 49 Yet oft the Briton kings against them [sc. the Romans] strongly swayd.    
\P 1603 KNOLLES  \textit{Hist. Turks} (1621) 195 A man would have thought two rough seas had met together swaying one against the other.    
\P 1871 DIXON  \textit{Tower} III. xxvi. 284 The Duke had grown too great to live. All passions swayed against him.

\itembf{5.} trans. To cause to incline or hang down on one side, as from excess of weight; dial. to weigh or press down; also, to cause to swerve.

\P 1570 BUCHANAN  \textit{Chamæleon} Wks. (S.T.S.) 45 The said Chamæleon..changeing hew as the quene sweyit ye ballance of hir mynd.    
\P 1625 BACON  \textit{Ess., Simulation} (Arb.) 509 To keepe an indifferent carriage, betweene both, and to be Secret, without Swaying the Ballance, on either side.    
\P 1663 CHARLETON  \textit{Chor. Gigant.} 27 As that no force of wind or tempest..by diminishing the gravity on one side, might incline or sway them to sink down on the other.    
\P 1664 POWER  \textit{Exp. Philos.} ii. 145 The greater weight of water in the pendent Leg [of the Syphon]..sways down that in the shorter, as in a pair of Skales.    
\P 1678 BUTLER  \textit{Hud.} iii. ii. 1368 As bowls  run true, by being made Of purpose false, and to be sway'd.    
\P 1797 HOLCROFT tr.  \textit{Stolberg's Trav.} (ed. 2) II. xliii. 81 The..tower of Pisa..is swayed fifteen feet from the centre.    
\P 1846 HOLTZAPFFEL  \textit{Turning} II. 848 They have learned to avoid swaying down the file at either extreme.    
\P 1856 KANE  \textit{Arctic Expl.} II. xiv. 143 These swayed the dogs from their course.    
\P 1857 WHITTIER  \textit{Poems, Funeral Tree Sokokis} Argt., The surviving Indians ‘swayed’ or bent down a young tree until its roots were upturned.

\noindent absol. \P 1624 BEDELL  \textit{Lett.} v. 84 A little weight is able to sway much, where the beame it self is false.

\itembf{b.} To strain (the back of a horse): see sway-backed, swayed 1. Obs. rare.

\P 1611 COTGR.,  \textit{Esflanquer}, to sway in the backe.    
\P 1639 T. DE GREY  \textit{Compl. Horsem.} 42 He might wrinch any member, or sway his back.

\itembf{6. a.} To turn aside, divert (thoughts, feelings, etc.); to cause to swerve from a course of action.

\P 1596 SHAKES.  \textit{1 Hen. IV,} iii. ii. 130 Heauen forgiue them, that so much haue sway'd Your Majesties good thoughts away from me.    
\P 1616 \textit{Marlowe's  Faustus} iv. ii. (1631) Fj, Let vs sway [ed. 1624 stay]  thy thoughts, From this attempt.    
\P 1673 CAVE  \textit{Prim. Chr.} ii. vi. 135 No dangers could then sway good men from doing of their duty.    
\P 1679 J. GOODMAN  \textit{Penit. Pard.} i. iii. (1713) 69 An huge advantage may sway him a little aside.    
\P 1822 B. W. PROCTOR  \textit{Ludovico Sforza} ii, No ill has happened..to sway Your promise from me?    
\P 1874 GREEN  \textit{Short Hist.} vi. §6. 335 No touch either of love or hate swayed him from his course.

\itembf{b.} To influence in a specified direction; to induce to do something. Obs.

\P 1625 \textit{Impeachm.  Dk. Buckhm.} (Camden) 292 To sweigh the people to accept the King's offers.    
\P 1634 SIR T. HERBERT  \textit{Trav.} 63 He answered, his businesse swayed him to another end.    
\P 1667 MILTON  \textit{P.L.} viii. 635 Least Passion sway Thy Judgement to do aught, which else free Will Would not admit.    
\P 1712 ADDISON  \textit{Spect.} No. 357 \phonetic{⁋}14 The Part of Eve..is no less..apt to sway the Reader in her Favour.    
\P a1720 SEWEL  \textit{Hist. Quakers} (1795) II. vii. 83 He so swayed the master that at last he agreed.    
\P 1807 WORDSW.  \textit{White Doe} vi. 48 Even that thought, Exciting self-suspicion strong, Swayed the brave man to his wrong.

\itembf{c.} To give a bias to. Obs.

\P 1593 BACON  \textit{Let. to Burghley} Apr., I spake simply and only to satisfy my conscience, and not with any advantage, or policy to sway the cause.

\itembf{7.} intr. To incline or be diverted in judgement or opinion; to swerve from a path or line of conduct; to lean (towards a side or party). Obs.

\P 1556 J. HEYWOOD  \textit{Spider \& F.} xxv. 94 We sweie From the streight lyne of iustice.    
\P 1581 LAMBARDE  \textit{Eiren.} ii. iv. (1588) 166 The common opinion swayeth to the other side.    
\P 1594 R. CAREW  \textit{Huarte's Exam. Wits} iii. (1596) 24 With which of these opinions the truth swaieth, time serueth not now to discusse.    
\P 1599 SHAKES.  \textit{Hen. V,} i. i. 73 He seemes indifferent: Or rather swaying more vpon our part, Then cherishing th' exhibiters against vs.    
\P 1659 W. GUTHRIE  \textit{Chr. Gt. Interest} (1724) 80 This imports a Sort of Impropriation: For the Heart, pleasing that Device, in so far swayeth towards it.    Ibid., Explic. Sc. Words, To sway or swey towards a Thing, is to bend towards it.

\itembf{8.} trans. To wield as an emblem of sovereignty or authority; esp. in phr. to sway the sceptre, sway the sword (also, by extension, sway the diadem, sway the rule), to bear rule.

   Cf. Du. den schepter zwaaien.

\P 1575 GASCOIGNE  \textit{Weedes, In Praise of Gentlewoman} 5 Golden Marcus he, that swaide the Romaine sword.    
\P 1576  \textit{ Steele Gl.} (Arb.) 61 You should not trust, lieftenaunts in your rome, And let them sway, the scepter of your charge.    
\P 1590 SPENSER  \textit{F.Q.} ii. x. 20 Madan was young, vnmeet the rule to sway.    
\P 1590 GREENE  \textit{Orl. Fur.} Wks. (Rtldg.) 99/1 It fits me not to sway the diadem.    
\P 1593 SHAKES.  \textit{3 Hen. VI}, iii. iii. 76 Though Vsurpers sway the rule a while.    
\P 1671 MILTON  \textit{P.R.} iii. 405 If I mean to raign David's true heir, and his full Scepter sway.    
\P 1750 GRAY  \textit{Elegy} 47 Hands, that the rod of empire might have sway'd.    
\P a1828 H. NEELE  \textit{Lit. Rem.} (1829) 26 Had Charles I. continued to sway the English sceptre.

\itembf{b.} transf. To wield (an implement or instrument). poet.

\P c1600 SHAKES.  \textit{Sonn.} cxxviii, When thou gently sway'st, The wiry concord that mine eare confounds.    
\P 1810 SCOTT  \textit{Lady of L.} ii. vii, This harp, which erst Saint Modan swayed.    
\P 1867 MORRIS  \textit{Jason} vi. 239 Erginous now, Great Neptune's so the brass-bound tiller swayed.

\itembf{9.} To rule, govern, as a sovereign. Chiefly poet.

\P 1595 SHAKES.  \textit{John} i. i. 13 To lay aside the sword Which swaies vsurpingly these seuerall titles.    Ibid. ii. i. 344 By this hand I sweare That swayes the earth this Climate ouerlookes.    
\P 1613 PURCHAS  \textit{Pilgrimage} vi. viii. 502 The Great Turke swayeth with his Ottoman Scepter..this Kingdome of Tunis, and all Africa, from Bellis de Gomera to the Redde Sea.    
\P 1634 MILTON  \textit{Comus} 825 A gentle Nymph..That with moist curb sways the smooth Severn stream.    
\P 1709 WATTS  \textit{Hymn, ‘The Lord! how fearful is his Name’} vi, Now let the Lord for ever reign, And sway us as he will.    
\P 1812 BYRON  \textit{Ch. Har.} ii. xlvii, With a bloody hand He sways a nation, turbulent and bold.    
\P 1896 A. AUSTIN  \textit{Eng. Darling} i. i, Buhred hath fled the land By him for two-and-twenty winters swayed.

\itembf{b.} transf. To have the command or control of; to control, direct.

\P 1587 GOLDING  \textit{De Mornay} xxiv. (1592) 366 There must be some pretie speech of Fortune, which swayth the battels. As for God..not one word.    
\P 1590 SHAKES.  \textit{Mids. N.} i. i. 193 Teach me..with what art You sway the motion of Demetrius hart.    Ibid. ii. ii. 115 The will of man is by his reason sway'd.    
\P 1665 BOYLE  \textit{Occas. Refl.} vi. iii. (1848) 352 Custom has much a larger Empire than men seem to be aware of, since whole Nations are wholly swai'd by it.    
\P 1791 BURKE  \textit{Corr.} (1844) III. 268, I have been long persuaded, that those in power here, instead of governing their ministers at foreign courts, are entirely swayed by them.    
\P 1874 GEO. ELIOT  \textit{Coll. Breakf. P.} 412 A sword..With edge so constant-threatening as to sway All greed and lust by terror.

\itembf{10.} intr. (occas. to sway it.) To rule; to hold sway. Also fig.

\P 1565 J. PHILLIP  \textit{Patient Grissell} Pref. (Malone Soc.) 17 Let Grissills Pacience swaye in you.    
\P 1586 A. DAY  \textit{Engl. Secretary} i. (1625) 16 Yours while life swaieth within me.    
\P 1591 SHAKES.  \textit{1 Hen. VI}, iii. ii. 135 A gentler Heart did neuer sway in Court.    
\P 1615 ROWLANDS  \textit{Melanch. Knight} 23 For shee's a Gentlewoman (though I say it) That doth deserue to domineere and sway it.    
\P 1633 BP. HALL  \textit{Hard Texts 1 Cor.} vi. 3 Those evill and apostate spirits, which doe now sway so much in the world.    
\P 1667 MILTON  \textit{P.L.} x. 376 There let him still Victor sway, As Battel hath adjudg'd.    
\P 1711 in  \textit{10th Rep. Hist. MSS. Comm. App.} v. 114 A tyrant is he..who swayes for his own onely pleasure.    
\P 1725 POPE  \textit{Odyss.} iii. 401 Lawless feasters in thy palace sway.    
\P 1853 J. HUNT  \textit{Spir. Songs, ‘Let all the world rejoice’} ii, He rules by sea and land, O'er boundless realms he sways.    
\P 1886 A. T. PIERSON  \textit{Crisis of Missions} 117 Turkey..still sways over one million square miles.

\itembf{11.} To have a preponderating weight or influence, prevail. Obs.

   This use combines senses 4 and 10.

\P 1586 A. DAY  \textit{Engl. Secretary} i. (1625) 126 His counsell..swaieth not..in our mindes, so much as it might haue done with many others.    
\P 1610 HOLLAND  \textit{Camden's Brit.} (1637) 586 Wee may understand..that gold swaied much yea in Church matters, and among Church-men.    
\P 1647 N. BACON  \textit{Disc. Govt. Eng.} i. lxx. (1739) 187 Nor did the King's Proclamation sway much this or that way.    
\P 1710 LADY  M. W. MONTAGU \textit{Let. to Mr. W. Montagu} 14 Nov., If my opinion could sway, nothing should displease you.    
\P 1768 TUCKER  \textit{Lt. Nat.} I. i. v. §7. 96 To distinguish what motive actually swayed with him upon every particular occasion.

\itembf{12.} trans. To cause (a person, his actions, conduct, or thoughts) to be directed one way or another; to have weight or influence with (a person) in his decisions, etc.

\P 1593 G. HARVEY  \textit{Pierce's Super.} Wks. (Grosart) II. 46 Had not affection otherwhiles swinged their reason, where reason should haue swayed their affection.    
\P 1605 B. JONSON  \textit{Volpone} iv. vi, Lady P. You shall sway me.    
\P a1674 CLARENDON  \textit{Surv. Leviath.} (1676) 108 Inclinations which sway them as much as other men.    
\P 1681 DRYDEN  \textit{Abs. \& Achit.} i. 939 Thus long have I by Native Mercy sway'd, My Wrongs dissembl'd.    
\P 1743 BULKELEY \& CUMMINS  \textit{Voy. S. Seas} 31 Believing we can sway most of the Seamen on Shore.    
\P 1760-2 GOLDSM.  \textit{Cit. W.} lvii, Swayed in their opinions by men who..are incompetent judges.    
\P 1818 SCOTT  \textit{Br. Lamm.} xxxiii, The honour of an ancient family, the urgent advice of my best friends, have been in vain used to sway my resolution.    
\P 1852 C. M. YONGE  \textit{Cameos} I. xii. 76 Bribery and every atrocious influence swayed the elections.    
\P 1870 MAX MÜLLER  \textit{Sci. Relig.} (1873) 292 The authority of their names continues to sway the public at large.    
\P 1892 \textit{Speaker}  3 Sept. 279/1 The jury..was swayed by the customary ethical code in these matters.

\itembf{13.} To swing (a weapon or implement) about; dial. to swing (something) to and fro, or from one place to another. Also intr. to swing.

\P 1590 SPENSER  \textit{F.Q.} i. xi. 42 When heauie hammers on the wedge are swaid.    Ibid. iii. i. 66 She..Here, there, and every where, about her swayd Her wrathfull steele.    
\P 1815 SCOTT  \textit{Guy M.} xlvi, Meg..lifted him into the vault ‘as easily,’ said he, ‘as I could sway a Kitchen's Atlas’.    
\P 1818 S. E. FERRIER  \textit{Marriage} xxxii. (1881) I. 320 Do I look like as if I was capable of hindering boys from sweein' gates?    
\P 1822 HOGG  \textit{Perils of Man} iv. I. 60 Bairns, swee that bouking o' claes aff the fire.    
\P 1823 SCOTT  \textit{Quentin D.} xxi, He..caught hold of one of the chains..and..swayed himself out of the water.    
\P 1894 P. H. HUNTER  \textit{James Inwick} xiv. 170 Ye've been sweein on the yett for a gey while.

\itembf{14.} Naut. (usually with up). To hoist, raise (esp. a yard or topmast).

\P 1743 BULKELEY \& CUMMINS  \textit{Voy. S. Seas} 15 He immediately gave Orders to sway the Fore-yard up.    
\P 1768 J. BYRON  \textit{Narr. Patagonia} (ed. 2) 15 He was going forward to get the fore$\sim$yard swayed up.    
\P 1835 MARRYAT  \textit{Jacob Faithful} xi, Forward there, Jacob, and sway up the mast.    
\P 1883 \textit{Man.  Seamanship for Boys} 61 A spanker is fitted with an outhaul and brails, the gaff being kept always swayed up in place.

\itembf{b.} absol.

\P 1836 MARRYAT  \textit{Midsh. Easy} xii, How long will it be, sir, before you are ready to sway away?    
\P 1840 R. H. DANA  \textit{Bef. Mast} xvii, We got a whip on the main-yard, and, hooking it to a strap round her body, swayed away.    
\P 1867 SMYTH  \textit{Sailor's Word-bk.}, Sway, or Sway away, to hoist simultaneously; particularly applied to the lower yards and top$\sim$masts, and topgallant-masts and yards. To sway away on all top-ropes, to go great lengths (colloquially).

\itembf{c.} To weigh (anchor). Obs.

\P 1772-84 \textit{Cook's  Voy.} (1790) IV. 1405 The  gale having subsided they swayed the anchor.
\end{myenumerate}

%%%%%%%%%%%%%%%%%%%%%%%%%%%%%%%%
\myitem{shrewd} a.

\noindent \phonetic{(ʃruːd)}

\noindent [ME. schrewed-e, etc., prob. orig. f. shrew n.2 (? or n.1) + -ed2. Cf. crabbed, dogged, wicked (all early ME.); the two former suggest the possibility that the animal (n.1) is alluded to. This formation coincided with the pa. pple. of shrew v., which may be the source of some of the senses; cf. the similar use of cursed.]
\vspace{-0.3cm}

\begin{myenumerate}

\itembf{1. a.} Of persons, their qualities, actions, etc.: Depraved, wicked; evil-disposed, malignant. Passing into a weaker sense: Malicious, mischievous. dial.

α \P 1303 R. BRUNNE  \textit{Handl. Synne} 4904 Ryche men haue shrewed sonys,—Shrewys yn dede and yn sawe.    
\P 13.. LAY  \textit{Folks Catech.} (MS. L) 139 Envye to oure neyȝbore with oþer schrewde castys.    
\P c1380 WYCLIF  \textit{Sel. Wks.} II. 349 Sclaundris and oþir shrewid wordis.    
\P c1400  \textit{Beryn} 1079 Fawnus..was  set oppon a purpose to make his sone leue All his shrewde tacchis.    
\P c1450  \textit{St. Cuthbert} (Surtees) 7330 Þe schrewed sonn of þe fende.    Ibid. 7742 A schrewyd counsaile toke þai þan.    
\P 1470-85 MALORY  \textit{Arthur} ix. xviii. 366 Whan he dyd ony shrewd dede they wold bete hym with roddes.    
\P 1483 CAXTON  \textit{Gold. Leg.} 35/1 Thenemye the fende with his angellis cursed and shrewd.    
\P c1490  \textit{ Rule St. Benet} 122 Kepe euer your tongue from euyll and shrewde langage, \& speke lytyll \& well.    
\P 1548 CRANMER  \textit{Catech.} 165 Our owne euyl workes and shrewed wylles.    
\P 1570  \textit{Satir. Poems Reform.} xviii. 62 Schrewit is that seruice ȝe haif schawin to ȝour King.    
\P 1590 SHAKES.  \textit{Mids. N.} ii. i. 33 That shrew'd and knauish spirit Cal'd Robin Good-fellow.    
\P 1612 DAY  \textit{Festivals} ii. (1615) 29 How do they pule \& cry? nay, how wil they shew a shrewd stomach or ever they can go or speake?    
\P 1634 MILTON  \textit{Comus} 846 All urchin blasts, and ill luck signes That the shrewd medling Elfe delights to make.    
\P 1879 G. F. JACKSON  \textit{Shropsh. Word-bk., Shrewd} \phonetic{(s'roa·d),..(shr'oa·d)},..badly-disposed; wicked; vicious. ‘'E's gwun a despert srōde lad.’

β \P 1547 BOORDE  \textit{Brev. Health} cccxxix, Beware of anger, for it is a shrode hert that maketh al the body fare the worse.    
\P 1606 DEKKER  \textit{Seuen Deadly Sinnes} iii. Wks. (Grosart) II. 48 Drunkards, Vnthriftes and shrode Husbonds.

γ \P 13.. \textit{Beues}  (A.) 4498 Þar was a Lombard in þe toun, Þat was scherewed \& feloun.    
\P 14.. CHAUCER'S  \textit{H. Fame} 275 (Caxton), Ther may be vnder goodlyhede Couerd many a sherewd vyce.

\itembf{b.} Of children: Naughty. Obs.

\P [1526  \textit{Pilgr. Perf.} (W. de W. 1531) 91b, These ben called..capytall vyces, bycause other shrewde children ryseth of them.]    
\P a1548 HALL  \textit{Chron., Hen. IV}, 9 Experience teacheth, that..of a shreude boye, proveth a good man.    
\P 1584 COGAN  \textit{Haven Health} cii. 89, I haue knowen..many a shreude boye for the desire of Apples, to haue broken into other folkes orchardes.    
\P 1588 SHAKES.  \textit{L.L.L.} v. ii. 12 He [Cupid] hath beene fiue thousand yeeres a Boy. Kath. I and a shrewd vnhappy gallowes too.    
\P 1645 BP. HALL  \textit{Treat. Content.} 77 The best of us are but shrewd children.

\itembf{c.} Of animals: Of evil disposition, bad-tempered; vicious, fierce; = cursed 4b. Obs.

\P 1509 WATSON  \textit{Ship of Fools} vi. (1517) Bvij, Oftentymes a mylde bytche bryngeth forth shrewed whelpes.
\P ?a1533 FRITH  \textit{Another Bk. agst. Rastell} (1829) 242 And may be likened to a shrewd cow, which, when she hath given a large mess of milk, turneth it down with her heel.    
\P 1546 HEYWOOD  \textit{Prov.} i. x. (1867) 22 God sendth the shrewd coow short hornes.    
\P 1547-50 BAULDWIN  \textit{Mor. Philos.} iv. Qiv, As to a shrewde horse belongeth a sharpe brydle: so oughte a shrewde wyfe to be sharpely handeled.    
\P 1607 MARKHAM  \textit{Caval.} ii. 96 The practice of some Horse-men..to tie a shrewd Cat to a Poale, with her heade and feete at libertie, and so thrusting it vnder the horses bellye,..to make her..clawe him.    
\P 1630 DRAYTON  \textit{Noah's Flood} 319 [They] together sat By the shrewd Muncky, Babian, and the Ape.

\itembf{2.} Of material things (esp. animals): Mischievous, hurtful; dangerous, injurious. Obs.

\P c1380  \textit{Sir Ferumb.} 4431 An Axe had he þan an honde, A shrewedere wepene for to fonde Was neuere non yfounde.    
\P 1387 TREVISA  \textit{Higden} (Rolls) I. 335 Wel schrewed mys [mures nocentissimos].    
\P 1398  \textit{ Barth. De P.R.} v. xxviii. (Bodl. MS.), Blaynes..comeþ of schrewed and corrupt humours.    
\P 1399 LANGL.  \textit{Rich. Redeles} iii. 20 Þoru busschis and bromes þis beste.. Secheth and sercheth þo schrewed wormes.    
\P c1400 MANDEVILLE  (1839) v. 46 Egipt is a strong Contree: for it hathe manye schrewede Havenes, because of the grete Roches.    
\P c1450 \textit{Robyn \& Gandeleyn}  vi. (Child Ball.), There cam a schrewde arwe out of þe west.    
\P 1493 \textit{Festyvall}  31b, They wyll slee theym with a shrewed knyfe. That is with the euyll and cursed tonge.    
\P 1593 SHAKES.  \textit{Rich. II,} iii. ii. 59 To lift shrewd Steele against our Golden Crowne.    
\P 1607-12 BACON  \textit{Ess., Of Wisdome for a Mans selfe} (Arb.) 182 An Ant..is a shrewd thing, in an Orchard, or a garden.    
\P 1621 DONNE  \textit{Serm.} xv. (1640) 148 The Buls of Babylon, the shrewdest Buls of all, in temporall, in spirituall persecutions.

\itembf{3. a.} Of things (chiefly immaterial): Of evil nature, character, or influence; ill-conditioned, bad, vile. Obs.

\P 1382 WYCLIF  \textit{Luke} iii. 5 Schrewide thingis [prava] schulen be in to dressid thingis.    
\P 1387-8 T. USK  \textit{Test. Love} ii. vi. (Skeat) l. 72 Right so he is a shrewe, on whom shreude thinges and badde han most werchinge.    
\P c1400  \textit{Beryn} 2613 They have a custom, a shrewid for the nonys, Yf [etc.].    
\P c1470 HENRY  \textit{Wallace} ii. 94 At thi shrewed ws thow wenys me to leid.    
\P 1513 DOUGLAS  \textit{Æneis} ii. viii. 57 The eddir, with schrewit herbis fed.    
\P 1519 \textit{Interl.  Four Elem.} (ed. Pollard) 438 Though he loke never so well, I promyse you he hath a shrewde smell.    
\P c1535 \textit{Frere \& Boy}  283 The good wyffe sayd, wer hast thou be? In schrewyd plas as thynkys me.    
\P 1644 MILTON  \textit{Areop.} 16 There are shrewd books, with dangerous Frontispices set to sale.    
\P 1678 in  \textit{Lauderdale Papers} (1885) III. 140 His Majtie did highly signify his displeasure against Sir William Lowther... The shreud effects whereof he has since tasted.

\itembf{b.} Of reputation, opinion, meaning: Evil, bad, unfavourable. Obs.

\P c1384 CHAUCER  \textit{H. Fame} 1619, Y graunte yow That ye shal haue a shrewde fame And wikkyd loos.    
\P 1527 in  Froude \textit{Hist. Eng.} (1881) I. 523 note, Some of them, as Master Dean hath known a long time, hath had a shrewd name.    
\P 1565 COOPER  \textit{Thesaurus} s.v. Commode, To be ill reported of: to haue a shrewde name.    
\P 1598 SHAKES.  \textit{Merry W.} ii. ii. 232 Shee enlargeth her mirth so farre, that there is shrewd construction made of her.    
\P 1621 T. WILLIAMSON tr.  \textit{Goulart's Wise Vieillard} 82 Many men..giue good things a shrewd vnhappie, and wrong name.    
\P 1664 H. MORE  \textit{Apology} 491 That spirit is not of God, but in some shreud sense or other is the spirit of Antichrist.

\itembf{c.} Poor, unsatisfactory. Obs.

α \P 1426 LYDG.  \textit{De Guil. Pilgr.} 2 1126 Thow  hast..Mad a shrewde marchaundyse.    
\P 1470-85 MALORY  \textit{Arthur} ix. xxiv. 375 There is shrewde herberowe,..lodge where ye will, for I wille not lodge there.    
\P 1525 LD. BERNERS  \textit{Froiss.} II. viii. 17 They will make a shrewde marchaundyce for vs.    ?
\P 1537 \textit{Thersytes}  146 (Pollard) He that should medle with me shall have shrewde rest!    
\P 1565 COOPER  \textit{Thesaurus}, Coenare malum.., to suppe with sorow and shrewde rest.    
\P a1586 SIDNEY  \textit{Arcadia} i. (Sommer) 26b, The Helots..would haue giuen a shrewd welcome to the [invading] Arcadians.

β \P 1593 \textit{Tell-Troth's  N.Y. Gift} (1876) 8 You might haue tooke better heede, and It was your owne fault, are two shrode plasters for a greene wound.    
\P 1616 \textit{Marlowe's  Faustus} (ed. Brooke) 990 By Lady sir, you haue had a shroud iourney of it.

\itembf{d.} In bad physical condition (the precise meaning varying with the application); in bad order; ugly; tough. Obs.

\P c1430 \textit{Pilgr.  Lyf Manhode} ii. cxxvi. (1869) 123, j can with good vynture enoynte a shrewede wheel that cryeth.    
\P 1526 SKELTON \textit{Magnyf.} (E.E.T.S.) 1155 With  a shrewde face uilis imago.    
\P 1571 GOLDING  \textit{Calvin on Ps.} xviii. 26 A shrewd knot must haue a shrewd wedge [malo nodo quærendum esse malum cuneum].    
\P 1593 \textit{Tell-Troth's  N.Y. Gift} (1876) 34 The young tree will stoup, when the old shrewd cannot bend.

\itembf{4.} Of events, affairs, conditions: Fraught or attended with evil or misfortune; having injurious or dangerous consequences; vexatious, irksome, hard; (of a task) difficult, dangerous. Obs.

α \P 1508 STANBRIDGE  \textit{Vulgaria} (W. de W.) Bvj, It is shrewed to Iape with naked swerdes.    
\P 1513 DOUGLAS  \textit{Æneis} v. ix. 64 The feirfull spa men therof pronosticate Schrewit chancis to betyde.    
\P 1531 FRITH  \textit{Judgm. upon Tracy} Wks. (1572) 79 Those holy fathers were in shreud cause, which continuing in long penurie, scant lefte at theyr departing, a halfe pennie.    
\P 1563-83 FOXE  \textit{A. \& M.} 1936/2, I aduise thee beware of the fire, it is a shrewd matter to burne.    
\P 1595 SHAKES.  \textit{John} v. v. 14 Ah fowle, shrew'd newes.    
\P 1613 PURCHAS  \textit{Pilgrimage} (1614) 711 Strangers haue more shrewd entertainment, and scarsely in twentie daies..can shake off this Shaker [ague].    
\P 1623 MIDDLETON  \textit{More Dissemblers} iii. ii, By'r Lady a shrewd business, and a dangerous.    
\P 1627 DONNE  \textit{Serm.} xxii. (1640) 222 The King, that comes after a good Predecessour, hath a shrewd burthen upon him.    
\P 1632 ROWLEY  \textit{New Wonder} iii. i. E3, Sir, 'tis a shrewd taske.    
\P 1821 J. BAILLIE  \textit{Metr. Leg., Lady G. B.} liv, The times are shrewd, my treasures spent.

β \P 1482 \textit{Cely  Papers} (Camden) 108 Wee fere here that ther weil be schrode passage to thys Balling martt.    
\P 1536 \textit{St. Papers  Hen. VIII}, II. 355, I promes you I am in a schroyd case, oneles the Kinges highe Majestie..do see redresse in suche causes.    
\P 1538 STARKEY  \textit{England} i. iii. 79 Yf the yeomanry of Englond were not, in tyme of warre we schold be in schrode case.    
\P 1573 G. HARVEY  \textit{Letter-bk.} (Camden) 11 This singulariti in philosophi is like to grow to a shrode matter.

\itembf{5.} shrewd turn: \textbf{a.} a mischievous or malicious act (arch.); \textbf{b.} a piece of misfortune, an accident (obs.).

\P 1464 PASTON  \textit{Lett.} 29 Feb., He wold do Debenham a shrewd turne and he coud.    
\P 1530 PALSGR. 712/2, I provoke..him to do a shreude tourne.    
\P 1565 COOPER  \textit{Thesaurus} s.v. Fero, Infortunium ferre,..to haue a shrewde turne.    
\P 1593 \textit{Passionate  Morrice} (1876) 76 As a dogge doth that is crept into a hole, hauing done a shroude turne.    
\P 1612 BRINSLEY  \textit{Lud. Lit.} 9 They are..sent to the schoole to keepe them..from danger, and shrewd turnes.    
\P 1642 D. ROGERS  \textit{Naaman} 282 The nurses eie attends the feeble infant, for feare of shrewd turnes.    
\P 1660 JER. TAYLOR  \textit{Duct. Dubit.} ii. i. rule 5 §3 They can doe a good turne or a shrewd.    
\P 1702 \textit{Engl.  Theophrastus} 204 No enemy is so despicable but some time or other he may do a body a shrewd turn.    
\P 1724 DE FOE  \textit{Mem. Cavalier} (1840) 211 That town owed us a shrewd turn for having handled them coarsely.

\itembf{6.} As an intensive, qualifying a word denoting something in itself bad, irksome, or undesirable: Grievous, serious, ‘sore’. \itembf{a.} of injury, loss, disease, etc. Obs.

α \P 1387 TREVISA  \textit{Higden} (Rolls) VI. 357 Þe evel þat hatte ficus, þat is a schrewed evel.    
\P 1461 PASTON  \textit{Lett.} II. 4 Ther was shrewd rewle toward in this cuntre.    
\P 1542 UDALL  \textit{Erasm. Apoph.} i. 132b, He gaue a shrewd checke to ye vnmeasurable praiser.    
\P 1592 \textit{Soliman \& P.}  426 A shrewd losse, by my faith, sir.    
\P 1593 SHAKES.  \textit{2 Hen. VI,} ii. iii. 41 Humfrey, Duke of Gloster, scarce himselfe, That beares so shrewd a mayme.    
\P 1606 CHAPMAN  \textit{Gent. Usher} ii. i. 25, I have been hanted..with a shrewd fever.    
\P 1609 G. ARCHER in  \textit{Purchas Pilgrims} (1625) IV. 1734 Some  three or foure dayes after her, came in the Swallow,..and had a shrewd leake.    
\P 1626 B. JONSON  \textit{Staple of News} i. Interm. 73 O, but the poore man had got a shrewd mischance, one day.    
\P 1658 A. FOX  \textit{Wurtz' Surg.} iii. x. 248 A Wound closed up, where a piece of the vein is yet unhealed,..will cause shrewd Imposthumes.    
\P 1713 C'TESS  OF WINCHILSEA \textit{Misc. Poems} 180 Meeting with a shrew'd mischance.    
\P 1819 SCOTT  \textit{Ivanhoe} xxxi, That is a shrewd loss.

β \P 1482 CELY  \textit{Papers} (Camden) 112 Hytt woll be a shrode losse.    
\P 1610 HOLLAND  \textit{Camden's Brit.} 441 With shrowde fines eftsoones redoubled, if not answered.    
\P 1612 N. FIELD  \textit{Woman is a Weathercock} ii. i, Mrs. Wag...Haulke, hauke. [Coughs and spits.] Page. Shee has a shrowde reach, I see that.    
\P 1623 BRADFORD  \textit{Plymouth Plant.} (1856) 150 His father suffered a shrowd check.

\itembf{b.} of temptation. Obs.

\P 1601 \textit{Death  Rob. Earl Hunt.} iv. ii. in Hazl. Dodsley VIII. 297, I know thou shalt be offer'd wealth, Which is a shrewd enticement in sad want.    
\P 1650 FULLER  \textit{Pisgah} iii. ii. xii. 437 A shroud bait to tempt his hungry souldiers to sacriledge.    
\P 1696 WHISTON  \textit{Theory Earth} 61 They were under a shrewd Temptation of thinking very meanly of the Bible it self.

\itembf{c.} Qualifying an agent-noun. Obs.

\P 1576 FLEMING  \textit{Panopl. Epist.} 171 marg., Timorousnesse a shrewd hinderer of enterprises.    
\P 1591 SHAKES.  \textit{1 Hen. VI,} i. ii. 123 These women are shrewd tempters with their tongues.

\itembf{d.} ‘Hard to beat’, formidable. rare—1.

\P 1851 BORROW  \textit{Lavengro} xii, I was now a shrewd walker, thanks to constant practice.

\itembf{e.} As a vague intensive. Obs.

\P a1643 W. CARTWRIGHT  \textit{Ordinary} iv. i, Caster. He threw twice twelve. Credulous. By'r lady, a shrewd many!

\itembf{7.} Of persons and their actions: Severe, harsh, stern. Obs.

\P 1387 TREVISA  \textit{Higden} (Rolls) I. 379 Oure men beeþ schrewed and angry inow to hem self, but in Goddes seruauntes þey leye neuere no hond.    
\P c1470 HENRY  \textit{Wallace} ix. 1424 The  captane than a schrewed ansuer him gaiff.    
\P a1586 SIDNEY  \textit{Arcadia} ii. xvi, She being sharp-set vpon the fulfilling of a shrewde office in over-looking Philoclea.    
\P 1600 HOLLAND  \textit{Livy} xxvii. xxxiv. 654 The hard and shrewd dealings of a mans countrie.    
\P 1654 BRAMHALL  \textit{Just Vind.} vi. 133 The Bishop..gave him..such a shrew'd remembrance, partly with words, and partly with his crosier staffe.

\itembf{8.} Severe, sharp, hard. \itembf{a.} Of a blow, wound. arch.

\P 1481 CAXTON  \textit{Reynard} (Arb.) 27 They..gauen hym many a shrewde stroke.    
\P a1500  \textit{Brut} 593 This shal be þe shrewdest bofet þat euer thow yovyst.    
\P 1596 LODGE  \textit{Wit's Misery} (1879) 92 Hee [the devil] will giue a shroud wound with his tongue.    
\P 1597 SHAKES.  \textit{2 Hen. IV,} ii. iv. 228 Me thought hee made a shrewd thrust at your Belly.    
\P 1647 CLARENDON  \textit{Hist. Reb.} i. 39 Many..were drowned, or forced on shore with shrewd hurts, and bruises.    
\P a1713 T. ELLWOOD  \textit{Hist. Life} (1714) 237 He struck her with the Stick, a shrewd Blow over the Breast.    
\P 1872 MORLEY  \textit{Voltaire} (1886) 9/1 The shrewd thrusts, the flashing fire, with which the hated Voltaire pushed on his work of ‘crushing the Infamous’.    
\P 1885 V. L. CAMERON  \textit{Across Africa} xvi. (ed. 2) 224 One or two got some shrewd knocks.

\itembf{b.} Of conflict or effort. Obs.

\P 1576 FLEMING  \textit{Panopl. Epist.} 43 To abide other bitter bruntes and shrewde skirmishes of aduersitie.    
\P 1630 \textit{R. Johnson's  Kingd. \& Commw.} 111 Foure thousand men would have made a shrewd adventure to have taken his Indies from him.    
\P 1682 BUNYAN  \textit{Holy War} (1905) 412 Many a shrewd brush did some of the Townsmen meet with from them.    
\P 1698 FRYER  \textit{Acc. E. India \& P.} 21 They adventure with better force, and in shrewder Battels.

\itembf{9.} Sharp, piercing, keen. \textbf{a.} Of a weapon or the like; also of pain. arch. (After Shakes.: see quot. 1593 in 2.) 

\P 1842 TENNYSON  \textit{St. Sim. Styl.} 195 A sting of shrewdest pain Ran shrivelling thro' me.    
\P 1871 R. ELLIS  \textit{Catullus} lxxxiii. 5 A shrewder stimulus arms her, Anger.    
\P 1878 BROWNING  \textit{Poets Croisic} 107 Sharpest shrewdest steel that ever stabbed To death Imposture.

\itembf{b.} Of the air, wind, weather.

\P 1642 D. ROGERS  \textit{Naaman} 96 There comes a shrewd right winde, and gets into the hollow of the tree.    
\P 1784 COWPER  \textit{Task} iii. 581 All plants..that can endure The winter's frown, if screen'd from his shrewd bite.    
\P 1824 W. IRVING  \textit{T. Trav.} I. 23 The night was shrewd and windy.    
\P 1849 ROSSETTI  \textit{Ruggiero \& Angelica} 9 The sky is harsh, and the sea shrewd and salt.    
\P 1864 LOWELL  \textit{Fireside Trav.} 337 That shrewd Yorkshire atmosphere.    
\P 1894 CROCKETT  \textit{Raiders} xviii, The air was shrewd as it breathed from the north.

\noindent advb. \P 1603 SHAKES.  \textit{Ham.} (Qo.) 400 The ayre bites shrewd [Qo. 1604 shroudly];  it is an eager and An nipping winde.

\itembf{c.} Of sound: Harsh. rare.

\P 1876 SWINBURNE  \textit{Erechtheus} 10 The song-notes of our fear, Shrewd notes and shrill, not clear or joyful-sounding.

\itembf{10.} \textbf{a.} Of a sign, token, etc.: Of ill omen, ominous; hence, strongly indicative (of something unfavourable).

\P 1577 B. GOOGE  \textit{Heresbach's Husb.} iv. (1586) 177 Be sure to marke them well..whether they go all out or no: for if they doe, it is a shrewde signe they will away.    
\P 1619 T. TAYLOR  \textit{Titus} ii. 8 Bitternesse [is] a shrewd signe of a bad cause.    
\P 1630 DONNE  \textit{Serm.} xiii. (1640) 135 If our own heart..condemne us, this is shrewd evidence, saies S. Iohn.    
\P 1691 NORRIS  \textit{Pract. Disc.} 186 'Tis a shrewd Symptom of an ill habit of Body.    
\P 1692 BP. PATRICK  \textit{Answ. Touchstone} 262 We hear not a word of Fathers to countenance this Doctrine, which is a shrow'd sign it is so far from being Ancient, that they speak directly against it.    
\P 1732 BERKELEY  \textit{Alciphr.} vi. §17 When a man is against reason, it is a shrewd sign reason is against him.

\itembf{b.} Of probability, etc. Obs.

\P 1542 UDALL  \textit{Erasm. Apoph.} i. 149 A good plain maner of knowelage geuyng it was \& a shrewd likelyhood.    
\P 1619 SCLATER  \textit{Expos. 1 Thess.} v. 554 To array our selues..aboue our Calling [is] no lesse then Pride; at least a shrewd species and appearance of it.    
\P 1709 SHAFTESBURY  \textit{Moralists} ii. 52 If Pain be Ill..we have..a shrewd Chance on the ill side, but none at all on the better.

\itembf{11.} Of a piece of evidence: Hard to get over, ‘awkward’, damaging. arch.

\P 1606 HOLLAND  \textit{Sueton. Annot.} 4 If his Questour or Treasurer had beene condemned, it would haue beene a shrewde precedent for his conviction also in the same cause.    
\P 1633  LAUD in \textit{Strafford Lett.} (1739) I. 213, I am afraid that many of them will be found Guilty: You give me one shrewd Instance in the Bishop of Waterford.    
\P 1692 \textit{Vindiciæ  Carol.} ii. 31 The pinching Article against him [Strafford] was the Twenty third... A shrewd Article no doubt, and sufficiently evidences their Crime.    
\P 1849 H. MILLER  \textit{Footpr. Creator} xv. 310 A shrewd fact, which they who expect most from the future of this world would do well to consider.

\itembf{12. a.} Given to railing or scolding; shrewish. Obs.

α \P 1387 TREVISA  \textit{Higden} (Rolls) III. 285 Tweie schrewed [ligitiosissimas] wifes þat wolde alway chide and stryve.    
\P 1483 CAXTON  \textit{G. de la Tour} D vij b, The tale and matere of the euylle and shrewde wyues.    
\P 1550 COVERDALE  \textit{Spir. Perle} xv, His [Socrates'] curst and shrewd wife.    
\P 1599 SHAKES.  \textit{Much Ado} ii. i. 20 Thou wilt neuer get thee a husband, if thou be so shrewd of thy tongue.    
\P 1605 CAMDEN  \textit{Rem.} (1623) 250 Somewhat shrewd to her Seruants.    
\P a1661 FULLER  \textit{Worthies, Shropsh.} (1662) 2 The Poets faining Juno, chaste and thrifty, qualities which commonly attend a shrewd nature.

β \P a1500 \textit{Brome  Bk.} 11 The properte of a schrod qwen ys to have hyr wyll.    ?
\P c1530 in  \textit{Pol. Rel. \& Love Poems}, etc. (1903) 58 Thowe shalte bettyr chastise a shrode wyfe with myrthe, then with strokes or smytyng.    
\P 1596 SHAKES.  \textit{Tam. Shr.} i. ii. 70 As old as Sibell, and as curst and shrow'd As Socrates Zentippe.

\itembf{b.} Of words, language: Scolding, railing, abusive. Obs.

\P 1538 CROMWELL in  \textit{Merriman Life \& Lett.} (1902) II. 128 If ye had..sowght fully to instructe me in the matier, then thus to desire to conquer me by shrowde wordes.    
\P 1565 COOPER  \textit{Thesaurus} s.v. Confero, Maledicta in aliquem, to rayle at one; to geue shrewde woordes.    
\P 1606 HOLLAND  \textit{Sueton.} 191 She had reviled him \& given him shrewd words.    
\P 1632 LITHGOW  \textit{Trav.} x. 488 With shrew'd Acerbious speech, you Anathematize.    
\P a1661 FULLER  \textit{Worthies, London} (1662) 197 Shrewd words are sometimes improved into smart blows betwixt them.

\itembf{13. a.} In early use: Cunning, artful (obs.). Now only in favourable sense: Clever or keen-witted in practical affairs; astute or sagacious in action or speech. (The chief current sense.)

α \P 1520 CALISTO \& MELIB. in  Hazl. \textit{Dodsley} I. 60 Seeming to be sheep, and serpently shrewd.    
\P 1589 PUTTENHAM  \textit{Engl. Poesie} iii. xxi. (Arb.) 257 Least with their shrewd wits, when they were maried they might become a little too phantasticall wiues.    
\P 1638 JUNIUS  \textit{Paint. Ancients} 47 By acting sharpe old men, shrewd servants,..and all such parts as did require some noise and stirre.    
\P a1700 EVELYN  \textit{Diary} 15 June 1675, His lady had ben very handsome, and seem'd a shrewd understanding woman.    
\P 1706 STANHOPE  \textit{Paraphr.} III. 331 The Men of the World are abundantly more shrewd in the Business of it, than even Good Men are in the Management of their great and eternal Concern.    
\P 1807-8 W. IRVING  \textit{Salmag.} (1824) 228 A shrewd old gentleman, who stood listening by with a mischievously equivocal look.    
\P 1867 SMILES  \textit{Huguenots Eng.} ii. (1880) 25 Palissy was..by nature a shrewd observer and an independent thinker.    
\P 1880 L. STEPHEN  \textit{Pope} iv. 102 A woman of shrewd intellect and masculine character.    
\P 1884 TENNYSON  \textit{Falcon} i. i. 468 Lady, I find you a shrewd bargainer.

\noindent absol. \P 1867 LOWELL  \textit{Fitz Adam's Story} 360 Hard-headed and soft-hearted, you'd scarce meet A kinder mixture of the shrewd and sweet.

β \P 1594 NASHE  \textit{Unfort. Trav.} B4b, They told the King he was a foole, and that some shrowd head had knauishly wrought on him.    
\P 1605 CHAPMAN  \textit{All Fools} iv. i. H2, Rinal. Y'aue gotten a learned Notarie Signior Cornelio. Corn. Hees a shroad fellow indeed.    
\P 1606 SHAKES.  \textit{Tr. \& Cr.} i. ii. 206 He has a shrow'd wit.

\itembf{b.} Of action, speech: Cunning, artful (obs.); characterized by penetration or practical sagacity.

\P 1589 ?  \textit{Nashe Pasquill \& Marforius} B 1, Whereuppon they presume to make a shrewde scruple of their obedience.    
\P 1649 MILTON  \textit{Eikon.} xxvi. 502 The shrewdest and the cunningest obloquie that can be thrown upon thir actions.    
\P 1761 HUME  \textit{Hist. Eng.} II. xxvii. 120 Empson made a shrewd apology for himself.    
\P 1781 COWPER  \textit{Table-T.} 205 The cause..may yet elude Conjecture and remark, however shrewd.    
\P 1824 W. IRVING  \textit{T. Trav.} II. 259 An eminent man, who had waxed wealthy by driving shrewd bargains with the Indians.    
\P 1882 J. H. BLUNT  \textit{Ref. Ch. Eng.} II. 113 Taking shrewd advantage of the Lord Chancellor's unlucky mistake.    
\P 1884 R. W. CHURCH  \textit{Bacon} iii. 59 He liked to observe, to generalise in shrewd and sometimes cynical epigrams.

\itembf{c.} Of the face or look.

\P 1816 SCOTT  \textit{Antiq.} i, A shrewd and penetrating eye.    
\P 1877 MRS. FORRESTER  \textit{Mignon} i, Fred Conyngham..has a plain, shrewd face.    
\P 1877 BLACK  \textit{Green Past.} iii, The shaggy, dark brown eyebrows gave shadow and intensity to the shrewd and piercing grey eyes.

\itembf{14.} Of a suspicion or guess: Coming ‘dangerously’ near to the truth of the matter. (?Partly arising from sense 10.)

\P 1588 J. HARVEY  \textit{Disc. Probl.} 127, I denie not but the wisest..politiques may..giue a shrewd gesse, and go neare the marke.    
\P 1599 \textit{Warn.  Faire Women} ii. 1025 Should  you be guilty of this fact, As this your flight hath given shrewde suspition.    
\P 1604 SHAKES.  \textit{Oth.} iii. iii. 429 'Tis a shrew'd doubt, though it be but a Dreame.    
\P 1653 H. MORE  \textit{Antid. Ath.} iii. xii. §3 It is a shrewd presumption that he doth lie with them indeed.    
\P 1848 THACKERAY  \textit{Van. Fair} li, I have a shrewd idea that it is a humbug.

\itembf{15.} Comb., as shrewd-eyed, shrewd-headed, shrewd-hearted, shrewd-looking, shrewd-pated, shrewd-tongued, shrewd-wit, shrewd-working adjs.; shrewd-head Austral. and N.Z. slang, a cunning person.

\P c1440  \textit{Promp. Parv.} 449/1 Schrewyd hertyd, pravicors.    
\P 1582 STANYHURST  \textit{Æneis} ii. (Arb.) 47 The priest Calchas was broght by the shrewdwyt Vlisses.    
\P 1607 HIERON  \textit{Wks.} I. 197 A shrewd-tongued woman.    
\P 1628 FORD  \textit{Lover's Mel.} iv. ii, A shrewd-braine Whorson; there's pith In his vntoward plainenesse.    
\P 1629 MAXWELL tr.  \textit{Herodian} (1635) 199 A notable shrewd-pated Fellow.    
\P 1827 LYTTON  \textit{Pelham} xvi, She was a pretty, fair, shrewd-looking person.    
\P 1856 J. G. WHITTIER  \textit{Panorama} 9 The shrewd-eyed salesman, garrulous and loud.    
\P 1865 KINGSLEY  \textit{Herew.} ix, The .. shrewdest-headed .. Berserker in the North Seas.    
\P 1916 C. J. DENNIS  \textit{Songs Sentimental Bloke} 43 Now this 'ere gorspil bloke's a fair shrewd 'ead.    
\P 1946 J. MORRISON in  \textit{Coast to Coast} 163 Some shrewd-head overseas will get the blame for that pillaged case.    
\P 1959  \textit{Daily Tel.} 20 May 17/1 A smiling, shrewd-eyed woman.    
\P 1960 N. HILLIARD  \textit{Maori Girl} iii. i. 177 Only the shrewd-heads go for that hard stuff: the shysters the takes.
\end{myenumerate}


%%%%%%%%%%%%%%%%%%%%%%%%%%%%%%%%
\myitem{perspicacity} n.

\noindent \phonetic{(pɜːspɪˈkæsɪtɪ)}

\noindent [ad. L. perspicācitās, f. perspicāx: see prec. and -ity: cf. F. perspicacité (15-16th c. in Hatz.-Darm.).]
\vspace{-0.3cm}

\begin{myenumerate}

\itembf{1.} Keenness of sight. Obs. or arch.

\P 1607 TOPSELL  \textit{Four-f. Beasts} 493 From these fables of Lynceus came the opinion of the singular perspicacity of the beast Linx.    
\P 1646 SIR T. BROWNE  \textit{Pseud. Ep.} i. ii. 5 Nor can there any thing escape the perspicacity of those eyes which were before light, and unto whose opticks there is no opacity.    
\P 1774 GOLDSM.  \textit{Nat. Hist.} (1862) II. ii. vii. 55 The barn-owl..watches in the dark, with the utmost perspicacity and perseverance.

\itembf{2.} Clearness of understanding or insight; penetration, discernment.

\P 1548 BECON  \textit{Solace of Soule} Wks. (1560) ii. 115 Thou shalte neuer by the perspycacyte and quyckenes of thy owne reason perceyue how it maye be possible.    
\P 1663 BP. PATRICK  \textit{Parab. Pilgr.} xxviii. (1668) 323 The greatest wits want perspicacity in things that respect their own interest.    
\P 1779-81 JOHNSON  \textit{L.P., Blackmore} Wks. III. 173 [This] is the only reproach which all the perspicacity of malice..has ever fixed upon his private life.    
\P 1809-10 COLERIDGE  \textit{Friend} (1865) 153 A masterpiece of perspicacity as well as perspicuity.    
\P 1838 PRESCOTT  \textit{Ferd. \& Is.} (1846) III. xvi. 183 She showed the same perspicacity in the selection of her agents.    
\P 1876 GLADSTONE  \textit{Homeric Synchr.} 61 Lessing, in his Laocoon, has discussed with luminous perspicacity [etc.].
\end{myenumerate}

%%%%%%%%%%%%%%%%%%%%%%%%%%%%%%%%
\myitem{astute} a.

\noindent \phonetic{(əˈstjuːt)}

\noindent [(? a. F. astut) ad. L. astūtus, lengthened form of astus crafty, cunning.]
\vspace{-0.3cm}

Of keen penetration or discernment, esp. in regard to one's own interests; shrewd, subtle, sagacious; wily, cunning, crafty.

\P 1611 COTGR.,  \textit{Astut}, astute, crafty, subtill, wyly, guilefull.    
\P 1634 SIR M. SANDYS  \textit{Prudence} 168 Wee terme those most Astute, which are most Versute. [Not in Johnson 1755.]    
\P 1829 I. TAYLOR  \textit{Enthus.} x. 258 The astute atheism of Greece and Rome.    
\P 1878 R. B. SMITH  \textit{Carthage} 331 He had, with the astute fickleness of a barbarian, come to a secret understanding with Scipio.



%%%%%%%%%%%%%%%%%%%%%%%%%%%%%%%%
\myitem{obtuse} a.

\noindent \phonetic{(əbˈtjuːs)}

\noindent [ad. L. obtūs-us dulled, blunt, pa. pple. of obtundĕre to obtund. Cf. F. obtus, -use (1542 in Hatz.-Darm.).]

\vspace{0.1cm}
\noindent Blunt (in various senses): opp. to acute.

\vspace{-0.4cm}
\begin{myenumerate}
\itembf{1.} lit. Of a blunt form; not sharp or pointed: esp. in Nat. Hist. of parts or organs of animals or plants. The opposite of acute.

\P 1589 PUTTENHAM  \textit{Eng. Poesie} ii. xi[i]. (Arb.) 114 Such shape as might not be sharpe..to passe as an angle, nor so large or obtuse as might not essay some issue out with one part moe then other as the rounde.    
\P 1657 S. PURCHAS  \textit{Pol. Flying-Ins.} 6 Their tails are somewhat sharp (the Drones more obtuse).    
\P 1660 BOYLE  \textit{New Exp. Phys. Mech.} xxxix. 322 An Oval Glass..with a short Neck at the obtuser end.    
\P 1753 CHAMBERS  \textit{Cycl.} Supp. s.v. Leaf, Obtuse Leaf, one terminated by the segment of a circle.    
\P 1767 GOOCH  \textit{Treat. Wounds} I. 237 A blow with an obtuse weapon.    
\P 1845 LINDLEY  \textit{Sch. Bot.} i. (1858) 10 Leaves are obtuse, or acute, in the ordinary sense of those words.    
\P 1877-84 HULME  \textit{Wild Fl.} p. viii, Spur stout, and obtuse.

\itembf{2.} Geom. Of a plane angle: Greater than a right angle; exceeding 90°.

   \textbf{obtuse bisectrix}: the line bisecting an obtuse angle, e.g. between the optic axes of a crystal. 
\textbf{obtuse cone}: a cone of which the section by a plane through the axis has an obtuse angle at the vertex. \textbf{obtuse hyperbola}: a hyperbola lying within the obtuse angles between its asymptotes.

\P 1570 BILLINGSLEY  \textit{Euclid} i. def. x. 3 An obtuse angle is that which is greater then a right angle.    
\P 1633 P. FLETCHER  \textit{Purple Isl.} iii. xxi, Into two obtuser angles bended.    
\P 1701 GREW  \textit{Cosm. Sacra} ii. v. §18 All Salts are Angular; with Obtuse, Right, or Acute Angles.    
\P 1879 WRIGHT  \textit{Anim. Life} 6 This bone forms an obtuse angle with the pelvis.

\itembf{3.} fig. Not acutely affecting the senses; indistinctly felt or perceived; dull.

\P 1620 VENNER  \textit{Via Recta} ii. 31 The wine..carrieth the same, which otherwise is of an obtuse operation, vnto all the parts [of the body].    
\P 1726 SWIFT  \textit{To a Lady}, Bastings heavy, dry, obtuse.    
\P 1790 CRAWFORD in  \textit{Phil. Trans.} LXXX. 426, I..felt an obtuse pain..in my stomach.    
\P 1897  \textit{Allbutt's Syst. Med.} IV. 126 Pain, sharp or obtuse.

\itembf{4.} Not acutely sensitive or perceptive; dull in feeling or intellect, or exhibiting such dullness; stupid, insensible. (In quot. 1606, Rough, unpolished: = BLUNT a. 4.)

\P 1509 HAWES  \textit{Past. Pleas.} xiii. (Percy Soc.) 113, I am but yonge, it is to me obtuse Of these maters to presume to endyte.    
\P 1602 MARSTON  \textit{Antonio's Rev.} i. iii. Wks. 1856 I. 79,  I scorne to retort the obtuse jeast of a foole.    
\P 1606 WARNER  \textit{Alb. Eng.} xvi. civ. (1612) 408 Obtuse in phrase.    
\P 1667 MILTON  \textit{P.L.} xi. 541 Thy Senses then Obtuse, all taste of pleasure must foregoe.    
\P 1829 SCOTT  \textit{Anne of G.} ii, Obtuse in his understanding, but kind and faithful in his disposition.    
\P 1885 M. BLIND  \textit{Tarantella} I. xi. 121 We were too obtuse to understand their peculiar way of manifesting it.

\itembf{5.} Comb., as \textbf{obtuse-angled}, having an obtuse angle or angles (also \textbf{obtuse-angular}
rare—0); also in Nat. Hist., with another adj., expressing a combination of forms, as 
\textbf{obtuse-ellipsoid}.

\P 1660 BARROW  \textit{Euclid} i. Def. xxvii, An Amblygonium, or obtuse-angled Triangle, is that which has one angle obtuse.    
\P 1706 PHILLIPS,  \textit{Obtuse-angled Cone}.    
\P 1878 A. H. GREEN,  etc. \textit{Coal} iv. 146 The two types of fin-structure are sometimes distinguished as obtuse-lobate and acute-lobate.    
\P 1882 OGILVIE,  \textit{Obtuse-angular}, having obtuse angles.
\end{myenumerate}

%%%%%%%%%%%%%%%%%%%%%%%%%%%%%%%%
\myitem{trenchant} a. (n.)

\noindent \phonetic{(ˈtrɛn(t)ʃənt)}

\noindent [a. OF. trenchant (mod.F. tranchant), pr. pple. of trenchier, trancher to cut: see trench v. and -ant.]
\vspace{-0.3cm}

\begin{myenumerate}

\itembf{1.} Cutting, adapted for cutting; having a keen edge, sharp; sharp-pointed (obs.). arch. and poet.

\P c1330 R. BRUNNE  \textit{Chron. Wace} (Rolls) 4414 Nemny on þe heued he smot; Hit was trenchaunt, ouer fer hit bot.    
\P c1380  \textit{Sir Ferumb.} 537 Ich hem wolde wel conquere wiþ my swerd trenchaunt.    
\P c1400 MANDEVILLE (1839) v. 47 This monstre..hadde ij hornes trenchant on his forhede.    
\P c1470 HENRY  \textit{Wallace} iv. 662 The trensand blaid to-persyt euirydeill.    
\P c1477 CAXTON  \textit{Jason} 8 b, Jason smote another Centaure in the nekke with a trenchaunt arowe.    
\P 1590 SPENSER  \textit{F.Q.} i. i. 17 He..with his trenchand blade her boldly kept From turning backe.    
\P 1663 BUTLER  \textit{Hud.} i. i. 359 The trenchant Blade, Toledo trusty, For want of fighting was grown rusty.    
\P a1774 GOLDSM.  \textit{Surv. Exp. Philos.} (1776) I. 236 The thin or trenchant end [of the wedge] is applied to the timber to be cleft, and the thick end struck upon by an hammer.    
\P 1830 TENNYSON  \textit{‘Clear-headed friend’} ii, Nor martyr-flames, nor trenchant swords Can do away that ancient lie.

\itembf{b.} Zool. Of a tooth, bill, etc.: Having a cutting edge; sectorial.

\P 1831 MCMURTRIE  \textit{Cuvier's Anim. Kingd.} II. 136 In a fourth tribe [of fishes], the teeth are trenchant. It comprises two genera, Boops and Oblada.    
\P 1835-6 \textit{Todd's  Cycl. Anat.} I. 312/2 Trenchant bills which are..flattened horizontally.    
\P 1881 MIVART  \textit{Cat} 29 The lower molar..having a more completely trenchant form than any other tooth.

\itembf{c.} transf., or in fig. or allusive use.

\P 1603 HOLLAND  \textit{Plutarch's Mor.} 30 Whose blood..now Trenchant Mars hath shed.    
\P 1809 W. IRVING  \textit{Knickerb.} vi. viii. (1849) 369 Pursuing its trenchant course, it severed off a deep coat pocket.    
\P 1851 GLADSTONE  \textit{Glean.} VI. lix. 39 Must it not be dangerous to place weapons so keen and trenchant in the hands of raw recruits?    
\P 1865 \textit{Trav.  by ‘Umbra’} 10 Carve the impalpable and viewless air with thy trenchant paper knife.    
\P 1871 FREEMAN  \textit{Hist. Ess.} Ser. i. v. 117 The biographer of Edward [III], Mr. Longman, cannot wield the trenchant weapons of Lord Brougham.

\itembf{2.} fig. esp. of language: Incisive; vigorous and clear; effective, energetic.

\P a1325 [implied in  TRENCHANTLY].    
\P 1663 BUTLER  \textit{Hud.} i. iii. 882 Their Swords Were sharp and trencheant, not their Words.    
\P 1824 MISS MITFORD  \textit{Village} Ser. i. (1863) 208 Some trenchant repartee, that cuts off the poor answer's head like a razor.    
\P 1842  in L Estrange \textit{Life} (1870) III. ix. 159 The most trenchant and violent writer of the ‘Times’.    
\P 1877 OWEN  \textit{Wellesley's Desp.} p. xxxvi, For all these evils..Wellesley devised prompt and trenchant remedies, most unpalatable to his employers.

\itembf{3.} transf. and fig. Sharply defined or outlined; clear-cut; distinct.

\P 1849 RUSKIN  \textit{Sev. Lamps} iii. §14. 78 The use of the dark mass characterises, generally, a trenchant style of design.    
\P 1852 DANA  \textit{Crust.} ii. 745 This subtribe has trenchant limits.    
\P 1873 H. ROGERS  \textit{Orig. Bible} ii. 78 The line of demarcation is seemingly most sharp and trenchant.

\itembf{4.} erron. Capable of being cut.

\P 1824 LAMB  \textit{Elia} Ser. ii. Blakesmoor in Hshire, What herald shall go about to strip me of an idea? Is it trenchant to their swords?

\itembf{B.} n. One who or that which cuts or severs; a cutter, a divider. Obs. rare—1.

\P a1660 CONTEMP.  \textit{Hist. Irel.} (Ir. Archæol. Soc.) I. 133 A turne-coate of lawfull confederacie, a trinchante of holy union, a scandall and reproofe of all Christian pietie.

\itembf{b.} esquire trenchant, an esquire carver; cf. esquire n.1 1 c and 5, quot. 1797. Obs.

\P 1563 RANDOLPH in  \textit{Calr. Scott. Pap.} II. 3 A longe yonge man..one of her graces esquire trenchantes.    [Cf.
\P 1611 COTGR.,  \textit{Trenchant}, Escuyer,..valet trenchant, a Caruer.]
\end{myenumerate}


%%%%%%%%%%%%%%%%%%%%%%%%%%%%%%%%
\myitem{insolent} a. (n.)

\noindent \phonetic{(ˈɪnsələnt)}

\noindent [ad. L. insolēnt-em unaccustomed, unusual, excessive, immoderate, haughty, arrogant, insolent, f. in- (in-3) + solēnt-em, pr. pple. of solēre to be accustomed. Cf. F. insolent (R. Estienne, 1549).]
\vspace{-0.3cm}

\begin{myenumerate}

\itembf{I.} \itembf{1.} Proud, disdainful, haughty, arrogant, overbearing; offensively contemptuous of the rights or feelings of others. Said of the powerful, rich, or successful, their actions, etc. Obs. or blended with 2.

\P c1386 CHAUCER  \textit{Pars T.} \phonetic{⁋}325 Insolent is he that despiseth in his Iuggement alle othere folk, as to regard of his value and of his konnyng and of his spekyng and of his beryng.    
\P 1596 SPENSER  \textit{State Irel.} Wks. (Globe) 636/2 Thorough greatnes of their late conquests and seignories they grewe insolent.    
\P 1617 MORYSON  \textit{Itin.} ii. 87 These being neerer..were most insolent upon that City.    
\P 1676 tr.  \textit{Guillatiere's Voy. Athens} 16 A haughty insolent person who affected to make himself terrible.    
\P 1727-38 GAY  \textit{Fables} i. xxiv. 26 ‘What arrogance!’ the snail replied; ‘How insolent is upstart pride!’    
\P 1840 THIRLWALL  \textit{Greece} lvi. VII. 189 Antipater was neither insolent nor cruel.    
\P 1858 TRENCH  \textit{Synon. N.T.} §30 (1876) 101 The boastful in words, the proud in thoughts, the insolent and injurious in acts.

\noindent fig. \P 1822 SHELLEY  \textit{Hellas} 344 One star with insolent and victorious light Hovers above its fall.    
\P 1830 GALT  \textit{Lawrie T.} iii. iii. (1849) 93 The insolent and unknown waters which had so swelled the river, shrunk within their banks.

\itembf{b.} Comb., as insolent-looking adj.

\P 1886 W. J. TUCKER  \textit{E. Europe} 198 The numberless Jewish equipages with all those insolent-looking Hebrew women of the Leopoldstadt.

\itembf{2.} Contemptuous of rightful authority; presumptuously or offensively contemptuous; impertinently insulting. Said of those who treat superiors or equals with offensive familiarity or disrespect.

\P 1678 MARVELL  \textit{Growth Popery} 4 This last and Insolentest attempt upon the credulity of mankind.    
\P 1685 BAXTER  \textit{Paraphr. N.T., Matt.} xii. 39-40 God will not gratifie their insolent demand.    
\P 1706 PHILLIPS,  \textit{Insolent}, saucy, bold, malapert, proud, haughty, disdainful, presumptuous.    
\P 1793 BURKE  \textit{Policy Allies} Wks. 1842 I. 604  Their revolutionary tribunals, where every idea of natural justice..have been trodden under foot with the most insolent mockery.    
\P 1856 FROUDE  \textit{Hist. Eng.} (1858) II. vii. 128 Bonner's tongue was insolent, and under bad control.    
\P 1884 PAE  \textit{Eustace} 69 He is an idle, drunken, insolent fellow.

\itembf{3.} Extravagant, immoderate, going beyond the bounds of propriety. Obs.

\P c1480 HENRYSON  \textit{Mor. Fab.} i. ii, Damesellis wanton, and insolent, That fane wald play, and on the streit be sene.    
\P 1568 GRAFTON  \textit{Chron.} II. 15 Thurston wasted..the goodes of that place, in lechery, and by other insolent meanes.    
\P 1712 STEELE  \textit{Spect.} No. 312 \phonetic{⁋}2 The constant Pursuit of Pleasure has in it something insolent and improper for our Being.    Ibid. No. 426 \phonetic{⁋}4 All the Extremities of Houshold Expence, Furniture, and insolent Equipage.

\itembf{4.} (?) Swelling, exulting: in good sense. rare.

\P 1589 PUTTENHAM  \textit{Eng. Poesie} i. xxxi. (Arb.) 77 For dittie and amourous Ode I finde Sir Walter Rawleyghs vayne most loftie, insolent, and passionate.

\itembf{II.} \itembf{5.} Unfrequented. Obs. rare.

\P c1420  \textit{Pallad. on Husb.} xii. 57 Where is lond vnkept \& insolent [regio insolens et incustodita] Take from the tronke al clene, vntil so hie As beestis may..Atteyne.

\itembf{6.} Unaccustomed, unwonted, unusual, strange.

\P 1586 G. PETTIE  \textit{Guazzo's Civ. Conv. To Rdr.} A vij, If one chance to derive any word from the Latine, which is insolent to their eares..they forthwith make a jest at it.    
\P 1592 R. D. \textit{Hypnerotomachia}  26 Letting passe to speake of the insolent greatnes of the Piramides of Memphis.    
\P 1608 A. WILLET  \textit{Hexapla Exod.} 468 This is an vnwonted and insolent signification of the word.    
\P 1612 BRINSLEY  \textit{Lud. Lit.} x. (1627) 164 Words which are insolent, hard and out of use, are to be as warily avoided.    
\P 1651 \textit{Fuller's  Abel Rediv.}, Bradford 181 This favour, though extraordinary and insolent, was thought well bestowed upon him by the whole University.    
\P 1665 JER. TAYLOR  \textit{Unum Necess.} viii. §3 The phrase is insolent, and the exposition violent.

\itembf{7.} Unused or unaccustomed to a thing; inexperienced. Obs.

\P c1480 HENRYSON  \textit{Orph. \& Euryd.} 20 Tendouris to yung and insolent.    
\P 1598 MARSTON  \textit{Pygmal.} iv. 153 Would euer any erudite Pedant Seeme in his artles lines so insolent?

\itembf{B.} n. An insolent person (in senses 1 and 2).

\P 1595 SHAKES.  \textit{John} ii. i. 122 Out, insolent, thy bastard shall be King, That thou maist be a Queen, and checke the world!    
\P 1639 tr.  \textit{Du Bosq's Compl. Woman} ii. 61 The salvation of these insolents, seems desperate, their repentance..Miracles.    
\P 1672 J. PHILLIPS  \textit{Montelion's Predict.} 10 What Christian will be a Second to such Insolents?    
\P 1765 H. WALPOLE  \textit{Otranto} v. (1798) 82 Thou art an insolent.    
\P 1898 \textit{Academy}  8 Oct. 28/1, I am [acquainted] with insolents, and you are one.
\end{myenumerate}


%%%%%%%%%%%%%%%%%%%%%%%%%%%%%%%%
\myitem{impertinent} a. (n.)

\noindent \phonetic{(ɪmˈpɜːtɪnənt)}

\noindent [a. F. impertinent (14th c. in Hatz.-Darm.) or ad. L. impertinēns, -ēnt-em not belonging, in med.L. ‘ineptus, insulsus’ (Du Cange), f. im- (im-2) + pertinēns pertinent.]
\vspace{-0.3cm}

\begin{myenumerate}

\itembf{1.} Not appertaining or belonging (to); unconnected, unrelated; inconsonant. ? Obs.

\P c1380 WYCLIF  \textit{Serm. Sel. Wks.} II. 31 Many men in þis world ben impertinent to erþeli lordis, for neiþer þei ben servantis to hem, ne þes lordis þeir worldly lordis.    
\P 1526  \textit{Pilgr. Perf.} (W. de W. 1531) 166 Thynges that be eche to other impertynent \& dyuerse.    
\P 1666 \textit{Ormonde MSS.} in  \textit{10th Rep. Hist. MSS. Comm. App.} v. 23 His private affayres and business (impertinent to anything relating to the said Lord Archbishop).    
\P 1809-10 COLERIDGE  \textit{Friend} (1837) III. 118 The more distant, disjointed and impertinent to each other and to any common purpose, will they appear.

\itembf{2.} Not pertaining to the subject or matter in hand; not pertinent; not to the point; irrelevant. Now rare exc. in Law.

\P c1386 CHAUCER  \textit{Clerk's Prol.} 54 Trewely as to my Iuggement Me thynketh it a thyng impertinent Saue that he wole conuoyen his mateere.    
\P 1530 PALSGR. 7 As for w is no letter used in the frenche tong..therfore as impertinent I passe it over.    
\P a1571 JEWEL  \textit{Serm. bef. Queen} (1583) A iij b, Let no man thinke these things are impertinent or from the purpose.    
\P 1610 SHAKES.  \textit{Temp.} i. ii. 138 I'le bring thee to the present businesse Which now's vpon's: without the which, this Story Were most impertinent.    
\P 1642 JER. TAYLOR  \textit{Episc.} (1647) 84 The allegation of S. Timothy's being an Evangelist, is absolutely impertinent, though it had been true.    
\P 1768 BLACKSTONE  \textit{Comm.} III. xxvii. 443 The master is to examine the propriety of the bill: and, if he reports it scandalous or impertinent, such matter must be struck out.    
\P 1812 M. EDGEWORTH  \textit{Vivian} x. (1832) 196 He did not..digress to fifty impertinent episodes, before he came to the point.    
\P 1872 WHARTON  \textit{Law Lex.} (ed. 5) 467/1 The Court may..direct the costs occasioned by any impertinent matter in any proceeding, to be paid by the party introducing it.

\itembf{3.} Not suitable to the circumstances; incongruous, inappropriate, out of place; not consonant with reason; absurd, idle, trivial, silly.

\P 1590 P. BARROUGH  \textit{Meth. Physick} i. xxxiii. (1639) 53 Many ignorant practitioners..have endeavoured to cure this infirmity with many impertinent medicines.    
\P 1631 WEEVER  \textit{Anc. Fun.} Mon. 16 These superfluous and impertinent costs of funerall expenses.    
\P 1662 J. DAVIES tr.  \textit{Olearius' Voy. Ambass.} 80 The opinion the Muscovites have of themselves and their abilities, is sottish, gross, and impertinent.    
\P 1677 HALE  \textit{Prim. Orig. Man.} i. i. 13 In comparison of this, all other Knowledge is vain, light and impertinent.    
\P 1706 PHILLIPS,  \textit{Impertinent}, .. absurd, silly, idle.    
\P 1706 ESTCOURT  \textit{Fair Examp.} iv. i. 42 For my part, I think a Woman's Heart is the most impertinent part of the whole Body.    
\P 1849 RUSKIN  \textit{Sev. Lamps} iv. §21. 111 There never was a more flagrant nor impertinent folly than the smallest portion of ornament in anything concerned with railroads.

\itembf{b.} Unsuitable, unfitted for. Obs.

\P 1594 CAREW  \textit{Huarte's Exam. Wits} (1616) 177 A power impertinent for curing.    Ibid. 183 To make clockes, pictures, poppets, and other ribaldries..impertinent for mans seruice.

\itembf{c.} Of persons: Absurd, silly. Obs.

\P 1639 T. BRUGIS tr.  \textit{Camus' Mor. Relat.} 205 As soone as a man brags, he is taken to be impertinent.    
\P 1681 J. CHETHAM  \textit{Angler's Vade-m.} xxii. §1 (1689) 143, I suspect myself to be Impertinent in saying thus much of the Conger, and Lampery.    
\P 1711 STEELE  \textit{Spect.} No. 148 \phonetic{⁋}7 The Ladies whom you visit, think a wise Man the most Impertinent Creature living.

\itembf{4.} Const. to (unto): in senses 2 and 3.

\P 1532 MORE  \textit{Confut. Barnes} viii. Wks. 740/1 Beyng as it is impertinent to the principall purpose.    
\P 1564  \textit{Brief Exam.} C iij, I thynke it not impartinent vnto this matter.    
\P 1656 HOBBES  \textit{Lib., Necess. \& Chance} (1841) 5 All the places of Scripture that he allegeth..are impertinent to the question.    
\P 1733 NEAL  \textit{Hist. Purit.} II. 304 It is no impertinent story to our present purpose.    
\P 1849 W. FITZGERALD tr.  \textit{Whitaker's Disput.} 185 All the common disquisitions upon this place..however true in themselves, are foreign to the subject and impertinent to the matter in hand.

\itembf{5.} Of persons, their actions, etc.: Meddling with what is beyond one's province; intrusive, presumptuous; behaving without proper respect or deference to superiors or strangers; insolent or saucy in speech or behaviour. (The chief current sense in colloq. use.)

\P 1618 SIR D. CARLETON  \textit{Let.} 4 Dec. in Crt. \& Times Jas. I (1848) II. 111 They [the Armenians at the Synod of Dort] are decried from their impertinent boldness and impudence by all men.]    
\P 1681 NEVILE  \textit{Plato Rediv.} 32, I have been impertinent in interrupting you.    
\P 1716 LADY M. W. MONTAGU  \textit{Let. to Mrs. Thistlethwaite} 30 Aug., It is publicly whispered, as a piece of impertinent pride in me, that I have hitherto been saucily civil to everybody.    
\P 1725 DE FOE  \textit{Voy. round World} (1840) 91 A very useful, skilful fellow, but withal so impertinent and inquisitive that we knew not what to say to him.    
\P 1798 NELSON  \textit{Let. to French Commander at Malta} Oct., I feel confident that you will not attribute it either to insolence or impertinent curiosity.    
\P 1847 JAMES  \textit{Convict} iii, He thought the stranger's tone rather impertinent.    
\P 1888 M. E. BRADDON  \textit{Fatal Three} i. iv, Fay has been most impertinent to me.

\itembf{b.} transf. of things.

\P 1848 DICKENS  \textit{Dombey} iv, Fenced up behind the most impertinent cushions.    
\P 1860 SALA  \textit{Lady Chesterf.} v. 83 The Lowther Arcade is vulgar and impertinent.    
\P 1861 THACKERAY  \textit{Four Georges} iv. (1862) 221 Her fair hair, her blue eyes, and her impertinent shoulders.

\itembf{B.} n.

\itembf{1.} An impertinent or irrelevant matter.

\P 1628 FELTHAM  \textit{Resolves} i. Ep. Ded. A iij b, To apparell any more [of my thoughts] in these Paper vestments, I should multiply impertinents.

\itembf{2.} An impertinent person: see the adj.; now esp. a meddlesome, presumptuous, or insolent person; one who does or says that which he has no business to do or say, and which is considered a piece of presumption or insolence.

\P 1635 A. STAFFORD  \textit{Fem. Glory} (1869) 5 This curious Impertinent.    
\P 1678 R. L'ESTRANGE  \textit{Seneca's Mor.} (1702) 398 This Day I have had entire to my Self..For all the Impertinents were either at the Theatre..or at the Horse-match.    
\P 1682 A. BEHN  \textit{City Heiress} 39 Nay dear Impertinent, no more Complements, be gone!    
\P 1710 PALMER  \textit{Proverbs} 355 An inquisitive impertinent..medling where he has nothing to do.    
\P 1825 LAMB  \textit{Elia Ser.} ii. Stage Illusion, When the pleasant impertinent of comedy..worries the studious man with taking up his leisure, or making his house his home.    
\P 1846 W. P. SCARGILL  \textit{Purit. Grave} 52 Henry St. John..rebuked the young impertinents.

\noindent Hence \phonetic{imˈpertinentness}, impertinency.

\P 1670 PENN  \textit{Truth Rescued fr. Impost.} 66 The Frivolousness and Impertinentness of this Ribaldry to the Controversie in hand.
\end{myenumerate}


%%%%%%%%%%%%%%%%%%%%%%%%%%%%%%%%
\myitem{impudent} a. (n.)

\noindent \phonetic{(ˈɪmpjʊdənt)}

\noindent [ad. L. impudēns, impudēnt-em shameless, f. im- (im-2) + pudēns ashamed, modest, orig. pres. pple. of pudēre to make or feel ashamed. Cf. F. impudent (16th c. in Hatz.-Darm. and Godef. Compl.: but the latter has the adv. impudemment of 1461).]
\vspace{-0.3cm}

\begin{myenumerate}

\itembf{1.} Wanting in shame or modesty; shameless, unblushing, immodest; indelicate. (In quot. 1628, ‘without the means of decency’.) Obs.

\P c1386 CHAUCER  \textit{Pars. T.} \phonetic{⁋}323 Inpudent is he that for his pride hath no shame of hise synnes.    
\P 1533 UDALL  \textit{Floures} 90 Canis (sayth Donate) is a worde that menie vse to obiect vnto suche as be impudent shameles felowes.    
\P 1579 G. HARVEY  \textit{Letter-bk.} (Camden) 61 Setting the best and impudentist face of it that I can borrowe.    
\P 1611 BIBLE  \textit{Ecclus.} xix. 2 He that cleaueth to harlots will become impudent.    
\P 1628 HOBBES  \textit{Thucyd.} (1822) 101 Many for want of things necessary..were forced to become impudent in the funerals of their friends.    
\P 1632 LITHGOW  \textit{Trav.} i. 26 Their impudent Curtezans, the most lascivious harlots in the world.    
\P 1659 D. PELL  \textit{Impr. Sea} 76 With impudent fore-heads, and with brows rubbed on brass-pots.    
\P 1732 GAY  \textit{Achilles} iii, Then her bosom too is so preposterously impudent!

\itembf{2.} Possessed of unblushing presumption, effrontery, or assurance; shamelessly forward, insolently disrespectful.

\P 1563-87 FOXE  \textit{A. \& M.} (1684) III. 493 Thou art as impudent a Fellow as I have communed withal.    
\P 1583 FULKE  \textit{Defence} xix. 544 You are the most impudent advoucher, I think, that ever became a writer.    
\P 1638 BAKER tr.  \textit{Balzac's Lett.} (vol. III.) 123 Sufficient defence against the audaciousnesse of the most impudent.    
\P 1709-10 HEARNE in  \textit{Reliq.} (1857) I. 181 Some persons were so impudent (to speak in the canting phrase) as to huzza him.    
\P 1710-11 SWIFT  \textit{Lett.} (1767) III. 125 Oh faith, you're an impudent saucy couple of sluttekins for presuming to write so soon.    
\P 1829 LYTTON  \textit{Devereux} ii. iv, Thou art an impudent thing to jest at us.    
\P 1848 DICKENS  \textit{Dombey} viii, Wickam is a wicked, impudent, bold-faced hussy.

\itembf{b.} Of conduct, actions, etc.

\P 1597 SHAKES.  \textit{2 Hen. IV,} ii. i. 135 You call honorable Boldnes, impudent Sawcinesse.    
\P 1639 T. BRUGIS tr.  \textit{Camus' Mor. Relat.} 246 [She] disclosed..[his] impudent attempt against the reverence of his marriage.    
\P 1755 YOUNG  \textit{Centaur} ii. Wks. 1757 IV. 134  Our impudent folly puts nature out of countenance.    
\P 1862 MARSH  \textit{Eng. Lang.} i. 10 An impudent fabrication of the fourteenth century.    
\P 1873 HALE  \textit{In His Name} vi. 64 This was the impudent reply of the largest boy of the group.

\itembf{B.} n. A person of unblushing effrontery or insolence.

\P 1586 T. B. tr.  \textit{La Primaud. Fr. Acad.} (1589) 404 No beast (as they say) is so shamelesse as an impudent.    Ibid. 253.    
\P 1589 PUTTENHAM  \textit{Eng. Poesie} i. xxvii. (Arb.) 69 Defrauded of the reward, that an impudent had gotten by abuse of his merit.    
\P 1632 LITHGOW  \textit{Trav.} x. 434 Many dissembling impudents intrude themselves in this high calling of God.
\end{myenumerate}


%%%%%%%%%%%%%%%%%%%%%%%%%%%%%%%%
\myitem{crass} a.

\noindent \phonetic{(kræs)}

\noindent [ad. L. crass-us solid, thick, dense, fat, etc. Cf. F. crasse fem. adj. (16th c. in Littré); OF. had cras, now gras.]
\vspace{-0.3cm}

\begin{myenumerate}

\itembf{1.} Coarse, gross, dense, thick (in physical constitution or texture). Now somewhat rare.

\P 1545 T. RAYNALDE  \textit{Byrth Mankynde} 12 The bottome of the mother or wombe is more crasse, thycke, and flesshy.    
\P 1646 SIR T. BROWNE  \textit{Pseud. Ep.} ii. v. 91 A crasse and fumide exhalation.    
\P 1695 WOODWARD  \textit{Nat. Hist. Earth} (1723) 295 Particles, which are more crass and ponderous.    
\P 1715 tr.  \textit{Pancirollus' Rerum Mem.} I. i. ix. 23 Of all Unguents..the most crasse and thickest.    
\P 1866 \textit{Treas.  Bot.} s.v., The leaves of cotyledons, which are much more fleshy, have been called crass.    
\P 1884 J. COLBORNE  \textit{Hicks Pasha} 180 A crass, gluey substance.

\itembf{b.} Said of things material as opposed to immaterial or spiritual. Obs.

\P 1649 JER. TAYLOR  \textit{Gt. Exemp.} ii. Ad Sec. 12. 94 Dives had the inheritance of the earth, in the crasse materiall sense.    
\P 1653 H. MORE  \textit{Antid. Ath.} iii. vi. §7 Whatsoever is crass and external leaves stronger Impress upon the Phansie.    
\P 1664  \textit{ Synops. Proph.} 217 Bearing strongly upon the phancy by exhibiting crass and palpable objects.

\itembf{2.} Of personal qualities, ideas, and other things immaterial: Gross; grossly dull or stupid, ‘dense’.

\P 1660 R. COKE  \textit{Justice Vind.} 20 Where the phantasie..is crass and dull and moves slowly.    
\P 1664 H. MORE  \textit{Myst. Iniq.} 110 An undoubted and conspicuous piece of the crassest Anti-christianism.    
\P 1859  \textit{Times} 20 Aug. 8/3 A free Press..to..dispel the crass ignorance which weighs over the land.    
\P 1877 E. R. CONDER  \textit{Bas. Faith} iii. 108 The crass materialism which talks about the brain secreting thought as the liver secretes bile.    
\P 1881 W. R. SMITH  \textit{Old Test.} in \textit{Jew. Ch.} 291 The crasser forms of religion.

\itembf{b.} Of persons: Grossly stupid, ‘dense’; grossly insensitive or unrefined (rare).

\P 1861 THACKERAY  \textit{Philip} viii, Your..undeserved good fortune..has rendered you hard, cold, crass, indifferent.    
\P 1872 GEO. ELIOT  \textit{Middlem.} xvi, Crass minds..whose reflective scales could only weigh things in the lump.    
\P 1877 BLACK  \textit{Green Past.} xx. (1878) 161 This crass idiot.
\end{myenumerate}


%%%%%%%%%%%%%%%%%%%%%%%%%%%%%%%%
\myitem{peril} n.

\noindent \phonetic{(ˈpɛrɪl)}

\noindent [a. F. péril (10th c. in Littré) = Pr. peril, perilh, Cat. peril, It. periglio:—L. perīculum, perīclum experiment, trial, risk, danger, f. root of ex-perī-rī to try, make trial of + -culum, suffix naming instruments.]
\vspace{-0.3cm}

\begin{myenumerate}

\itembf{1. a.} The position or condition of being imminently exposed to the chance of injury, loss, or destruction; risk, jeopardy, danger.

\P a1225  \textit{Ancr. R.} 194 Gostlich fondunge..mei beon, uor þe peril, icleoped breoste wunde.    
\P 1297 R. GLOUC.  (Rolls) 2208 Of peril a se \& eke a lond.    
\P a1300  \textit{Cursor M.} 24852 (Cott.) Þe mariners..war neuer in parel [v.r. perel] mar.    
\P 1390 GOWER  \textit{Conf.} II. 168 Saturnus after his exil Fro Crete cam in gret peril.    ?
\P a1400 LYDG.  \textit{Chorle \& Byrde} 183 Who dredeth no paryll, in paryll he shall falle.    
\P a1533 LD. BERNERS  \textit{Huon} lxxxiii. 257 He was neuer in his lyfe in suche perell.    
\P 1575  \textit{Mirr. Mag., Dk. Somerset} xliv, Constant I was in my Princes quarel, To dye or liue and spared for no parel.    
\P 1595 SHAKES.  \textit{John} iii. i. 295 The perill of our curses light on thee So heauy, as thou shalt not shake them off.    
\P 1749 SMOLLETT  \textit{Regicide} ii. viii, Glory Is the fair child of peril.    
\P 1832 W. IRVING  \textit{Alhambra} II. 166 Having commanded at Malaga during a time of peril and confusion.    
\P 1875 JOWETT  \textit{Plato} (ed. 2) V. 128 In the hour of peril.

\itembf{b.} Const. (a) of that which is exposed to danger (chiefly with life); (b) of the evil fate that threatens, or (obs. or arch.) of the cause of danger; (c) to with inf. (obs.).

\P 1340 HAMPOLE  \textit{Pr. Consc.} 161 In grete perille of saul es þat man Þat has witt and mynde and na gude can.    
\P c1450  \textit{St. Cuthbert} (Surtees) 1740 In perill  of þair lyues þai stode.    
\P 1596 SHAKES.  \textit{Merch. V.} ii. ii. 173 To be in perill of my life with the edge of a featherbed.    
\P 1790 PALEY  \textit{Horæ Paul.} Wks. 1825 III.  174 He acquitted himself of this commission at the peril of his life.    
\P 1840 DICKENS  \textit{Barn. Rudge} ii, You were never in such peril of your life as you have been within these few moments.

\P c1375  \textit{Cursor M.} 26193 (Fairf.) Quen men is in perel [Cott. wath] of dede.    
\P 1377 LANGL.  \textit{P. Pl.} B. xiv. 301 Þorw þe pas of altoun Pouerte myȝte passe with-oute peril of robbynge.    
\P 1481 CAXTON  \textit{Myrr.} ii. vi. 76 Kynge Alysaundre..eschewed the parell and daunger of thise olyfauntes.    
\P 1553 BALE VOCACYON in  \textit{Harl. Misc.} (Malh.) I. 330 In parell of the sea, in parell of shypwrack.    
\P 1634 SIR T. HERBERT  \textit{Trav.} 5 The..ship-boyes were in perill of those Sharkes.    
\P 1876 GEO. ELIOT  \textit{Dan. Der.} xlviii, A vessel in peril of wreck.

\P c1385 CHAUCER \textit{L.G.W.} 1277 Dido, There as he was in paril for to sterue.
\P c1489 CAXTON  \textit{Blanchardyn} lii. 201 He was in pereyll to lose hym selfe and all his ooste.    
\P 1596 SHAKES.  \textit{Tam. Shr.} Induct. ii. 124 In perill to incurre your former malady.

\itembf{2.} (with a and pl.) A case or cause of peril; pl. dangers, risks.

   \textit{peril of the sea} (Marine Insurance): see quot. 1872.

\P a1300  \textit{Cursor M.} 4051 (Cott.) O perils [v.r. perelis] þat he fell in Sum-quat to tell i sal bigin.    
\P 1382 WYCLIF  \textit{2 Cor.} xi. 26 In perelis of flodis, in perels of theues, in perelis of kyn, in perels of hethen men [etc.].    
\P 1450-80 tr.  \textit{Secreta Secret.} 21 Pereylis and disesis that are to come of werres, pestilencis [etc.].    
\P a1548 HALL  \textit{Chron., Hen. IV} 15 b, To auenture themselfes on a newe chance and a doubtfull parell.    
\P 1774 GOLDSM.  \textit{Nat. Hist.} (1776) VI. 181 Scarce one in a thousand survives the numerous perils of its youth.    
\P 1817 W. SELWYN  \textit{Law Nisi Prius} (ed. 4) II. 893 It is the province of the jury to determine, whether the cause of the loss be a peril of the sea or not.    
\P 1872 \textit{Wharton's  Law Lex.} s.v., Perils of the sea..are strictly the natural accidents peculiar to the water, but the law has extended this phrase to comprehend events not attributable to natural causes, as captures by pirates, and losses by collision, where no blame is attachable to either ship, or at all events to the injured ship.    
\P 1875 JOWETT  \textit{Plato} (ed. 2) I. 93 Soldiers,..who are courageous in perils by sea.    
\P 1884  \textit{Manch. Exam.} 3 May 5/1 The certain perils of such an alliance.

\itembf{3.} Phrases. \textbf{a.} at all peril(s: at whatever risk; be the consequences what they may. by the (for, up) peril of my soul, upon my peril, etc.: used as asseverations. in peril of: at the risk of, under the penalty of (see also 1 b). Obs.

\P 13..  \textit{E.E. Allit. P.} C. 85 At alle peryles, quoth þe prophete, I aproche hit no nerre.    
\P 1362 LANGL.  \textit{P. Pl.} A. vi. 47 Nai, bi þe peril of my soule, quod pers.    
\P c1386 CHAUCER  \textit{Wife's Prol.} 561 My gaye scarlet gytes, Thise wormes ne thise Motthes ne thise mytes Vpon my peril frete hem neuer a deel.     
\P \textit{Merch. T.} 1127 Vp peril  of my soule I shal nat lyen.    
\P 1470-85 MALORY  \textit{Arthur} iv. i. 119 Ye lady, on my parel, ye shal see hit.    
\P 1607 SHAKES.  \textit{Cor.} iii. iii. 102 Wee..banish him our Citie In perill of precipitation From off the Rocke Tarpeian.    [
\P 1820 BYRON  \textit{Mar. Fal.} i. ii, That I speak the truth, My peril be the proof.]

\itembf{b.} \textit{at (on, to) your (his, etc.) peril}: you (etc.) taking the risk or responsibility of the consequences: esp. in commands, or warnings, referring to the risk incurred by disregard or disobedience.

\P 1433  \textit{Rolls of Parlt.} IV. 477/1 Such as they woll answere fore atte here perille.    
\P 1480 CAXTON  \textit{Chron. Eng.} ccxiv. 200 He sente hastely that they shold not fyght, and yf they dyd that they shold stonde to hir owne perylle.    ?
\P a1550 \textit{Freiris  of Berwik} 541 in Dunbar's Poems (1893) 303 Gif thow dois nocht, on thy awin perrel beid [= be it].    
\P 1590 SHAKES.  \textit{Mids. N.} iii. ii. 175 Disparage not the faith thou dost not know, Lest to thy perill thou abide it deare.    
\P 1632 MASSINGER  \textit{City Madam} iv. ii, Master Shrieve and Master Marshal, On your perils, do your offices.    
\P 1664 in  \textit{Buccleuch} MSS. (Hist. MSS. Comm.) I. 541 As they will answer the contrary at their perils.    
\P 1696 PHILLIPS  (ed. 5), Peril,..sometimes used by way of threatning. Do such a thing at your Peril.    
\P 1719 DE FOE  \textit{Crusoe} ii. xi, We..bade them keep off at their peril.    
\P 1832 H. MARTINEAU  \textit{Hill \& Valley} iii. 46 Shew yourselves at your peril.    
\P 1881 R. BUCHANAN  \textit{God \& Man} I. 141, ‘I must do my master's bidding.’ ‘At your peril! I have but to give the word, and they would duck you in the horsepond.’

\itembf{c.} without the peril of: beyond the (dangerous) reach or power of: cf. danger n. 1 b. Obs. rare.

\P 1590 SHAKES.  \textit{Mids. N.} iv. i. 158 To be gone from Athens, where we might be Without the perill of the Athenian Law.

\itembf{4.} A matter of danger; a perilous or dangerous matter. Const. it is peril, it is dangerous (to do something). Obs.

\P 1297 R. GLOUC.  (Rolls) 6786 Þe heiemen of þe lond wolde hom al day mene Þat hii nadde non eir of him \& þat gret peril it was Vor þer miȝte com to al þe lond gret wo uor such cas.    
\P c1386 CHAUCER  \textit{Wife's Prol.} 89 Peril is bothe fyr and tow tassemble.    
\P c1400 MANDEVILLE (Roxb.) xxvi. 123 It es grete peril to pursue þe Tartarenes.    
\P a1450  \textit{Knt. de la Tour} (1868) 60 Whedir it were perelle to do her counsaile or not.    
\P c1540  \textit{Pilgr. T.} 164 in Thynne's Animadv. 81 You know what perrele it is together to ley hyrdis fast vnto the fyer.

\itembf{5.} attrib. and Comb., as \textit{peril-proof, peril-daring} adjs.; \textit{peril point} U.S. Econ. (see quot. 1965).

\P 1605 SYLVESTER  \textit{Du Bartas} ii. iii. ii. Fathers 75 A broad thick breastplate..High peril-proof against affliction.    
\P 1807 MONTGOMERY  \textit{W. Indies} ii. 141 The valiant seized in peril$\sim$daring fight.    
\P 1948 \textit{Congress.  Rec.} 26 May 6503/2 No foreign trade agreement could be entered into until the Tariff Commission reports to the President its findings as to the so-called peril-point below which tariffs may not be cut.    
\P 1949 \textit{Sun}  (Baltimore) 11 July 10/2 The main innovation in the Republican program is the so-called ‘peril-point’ report which must be made to the President by the Tariff Commission.    
\P 1949  \textit{Economist} 17 Sept., Peril Points. This year's battle over American tariff policy opened just as the Administration was assuring Sir Stafford Cripps and Mr Bevin that the United States would pursue policies appropriate to a great creditor nation.    
\P 1961  \textit{Ibid.} 9 Dec. 1025/2 The President's authority to lower tariffs being renewed grudgingly but limited by ‘peril points’ and ‘escape clauses’.    
\P 1965 \textit{McGraw-Hill  Dict. Mod. Econ.} 376 Peril point, the maximum cut in a U.S. import duty which could be made for a given commodity without causing serious injury to domestic producers or to a similar commodity.

\noindent Hence \phonetic{ˈperilless} a., without or free from peril.

\P a1614 SYLVESTER  \textit{Litt. Bartas} 313 In their chamber pain$\sim$lesse, peril-lesse.



\end{myenumerate}

%%%%%%%%%%%%%%%%%%%%%%%%%%%%%%%%
\myitem{devout} a. and n.

\noindent \phonetic{(dɪˈvaʊt)}

\noindent [ME. devot, devout, a. OF. devot, devote (12th c. in Littré), = Pr.
devot, Sp. devoto, It. divoto, ad. L. dēvōt-us devoted, given up by vow, pa.
pple. of dēvovēre to devote. The close OF. ō became the vowel ou (\phonetic{uː}) in ME., whence the modern diphthong ou; but a form in ō, Sc. oi, was also in use: see devote a.]
\vspace{-0.3cm}

\begin{myenumerate}

\itembf{1.} Devoted to divine worship or service; solemn and reverential in religious exercises; pious, religious.

α \P a1225  \textit{Ancr. R.} 376 Þuruh aromaz, þet beoð swote, is understonden swotnesse of deuot heorte.    
\P c1325  \textit{E.E. Allit. P.} A. 406 Be dep deuote in hol mekenesse.    
\P c1400 MANDEVILLE (Roxb.) viii. 30 Þai er deuote men and ledez pure lyf.    
\P 1535 STEWART  \textit{Cron. Scot.} II. 567 Diuoit he wes with mony almous deid.    
\P 1549  \textit{Compl. Scot.} (1872) 4 The deuot Kyng, Numa pompilius.    
\P 1651 [see DEVOTE a.].

β \P 1297 R. GLOUC.  (1724) 369 In chyrche he was deuout ynou.    
\P 1382 WYCLIF  \textit{Ex.} xxxv. 29 Alle men and wymmen with a deuowt mynde offerden ȝiftis.    
\P c1440  \textit{Promp. Parv.} 120 Devowte, devotus.    
\P a1450  \textit{Knt. de la Tour} (1868) 7 A shorte orison, saide with good devouute herte.    
\P c1511 1ST  \textit{Eng. Bk. Amer.} (Arb.) Introd. 31/2 These people be very deuoute.    
\P 1530 PALSGR. 310/1 Devoute, holy disposed to praye, deuot.    
\P 1636 SIR H. BLOUNT  \textit{Voy. Levant} (1637) 87 All the devouter sort (which are not many) goe to Church, and say their prayers.    
\P 1732  \textit{Law Serious} C. i. (ed. 2) 1 He..is the devout Man who lives no longer to his own will..but to the sole will of God.    
\P 1865 M. ARNOLD  \textit{Ess. Crit.} ix. (1875) 398 The devoutest of your fellow Christians.    
\P 1883 FROUDE  \textit{Short Stud.} IV. ii. ii. 185 Keble was a representative of the devout mind of England.

\itembf{b.} gen. Devoted, religiously or reverently attached (to a person or cause). Obs.

\P c1380 WYCLIF  \textit{Serm.} Sel. Wks. I. 113 God wolle have oure herte devoute to him wiþouten ende.    
\P c1450  \textit{St. Cuthbert} (Surtees) 6953 To saint cuthbert he was deuoute.    
\P 1609 BIBLE (Douay)  \textit{Comm.} 201 Isaac was..devout to God.    
\P 1659 B. HARRIS  \textit{Parival's Iron Age} 205 Sir Thomas Wentworth..became the most devout friend of the Church.

\itembf{2.} Of actions and things: Showing or expressing devotion; reverential, religious, devotional.

α \P a1340 HAMPOLE  \textit{Psalter}, Cant. 502 Þe deuot ȝernyngis of his halighis.    
\P c1500 \textit{Blowbol's Test.} in  \textit{Halliwell Nugae Poet.} 3 He wold syng Foure devoite masses at my biryng.    
\P a1541 BARNES  \textit{Wks.} 318 (R.), To help mee wyth his deuote prayer.    
\P 1552 ABP. HAMILTON  \textit{Catech.} (1884) 8 Faithful and devoit prayar.    1625- [see devote a.].

β \P c1340 HAMPOLE  \textit{Prose Tr.} 24 Deuoute prayers, feruent desires, and gostely meditacions.    
\P 1526 (Title),  The Pylgrymage of Perfeccyon, a devoute Treatyse in Englysshe.    
\P 1603 KNOLLES  \textit{Hist. Turks} (1621) 78 The devout warre, taken in hand for the reliefe of the poore Christians in Syria.    
\P 1667 MILTON  \textit{P.L.} xi. 863 With uplifted hands, and eyes devout.    
\P 1763 JOHN  BROWN \textit{Poetry \& Mus.} xii. 214 Our parochial Music..is solemn and devout.    
\P 1841 ELPHINSTONE  \textit{Hist. Ind.} II. 347 In his writings, he affects the devout style usual to all Mussulmans.

\itembf{3.} Earnest, sincere, hearty.

\P 1828 WEBSTER  \textit{s.v.}, You have my devout wishes for your safety.    
\P 1880 MRS. E. LYNN  \textit{Linton Rebel of Family} I. v, The sanctity of caste, in which she..was so devout a believer.

\itembf{B.} as n.

\itembf{1.} A devotee. Obs.

\P [c1440 GESTA  \textit{Rom.} xcii. 419 (Add. MS.) This knyght had a good woman to wife, and a deuoute to oure ladie.]    
\P 1616 R. SHELDON  \textit{Miracles Antichrist} 247 (T.) Not..the ordinary followers of Antichrist, but..his special devouts.    
\P 1675 tr.  \textit{Machiavelli's Prince} xv. (Rtldg. 1883) 98 One a devout, another an atheist.

\itembf{2.} That which is devout; the devotional part.

\P 1649 MILTON  \textit{Eikon.} i. (1851) 344 This is the substance of his first Section, till we come to the devout of it, model'd into the form of a privat Psalter.
\end{myenumerate}


%%%%%%%%%%%%%%%%%%%%%%%%%%%%%%%%%
%\myitem{smother} v.
%
%\noindent \phonetic{(ˈsmʌðə(r))}
%
%\noindent [f. smother n.]
%\vspace{-0.3cm}
%
%\begin{myenumerate}
%
%\itembf{I.} trans.
%
%1. a.I.1.a To suffocate with smoke.
%
%\P 1560 J. DAUS tr.  \textit{Sleidane's Comm. 220 b, They were smothered with smoke and burnt all.    
%\P 1579 WALSINGHAM in  \textit{Victoria Co. Hist., Surrey (1902) I. 391 A fyre made..by hunters that had earthed a badger, and thought to have smouthered him.    
%\P 1624 CAPT. SMITH  \textit{Virginia (1629) 85 But the poore Salvage..was so smoothered with the smoake he had made..that we found him dead.    
%\P 1719 DE FOE  \textit{Crusoe} ii. (Globe) 496 The House, which was by this time all of a light Flame, fell in upon them, and they were smothered or burnt together.    
%\P 1848 BARTLETT  \textit{Dict. Amer. 314 That the inky stream may smother or drive away mosquitoes.
%
%\P 1589  \textit{Pappe w. Hatchet To Rdr., With the verie smoke the consciences of diuers are smothered.    
%\P a1704 T. BROWN  \textit{Sat. Persius imit. Wks.
%\P 1730 I. 54 BY  \textit{the thick fogs, which from his diet rise, His sense is smothered.    
%\P 1944 [see blanket n. 2 c].
%
%b.I.1.b To suffocate by the prevention of breathing; to deprive of life by suffocation. (Freq. in passive without implication of personal agency.) Also spec. of sheep, to suffocate others by falling on top of them, as during a round-up; to cause (sheep) to die in this manner (N.Z.).
%
%\P a1548 HALL  \textit{Chron., Hen. VIII, 55 [Certain criminals] the same Richarde Hun feloniously strangeled and smodered.    
%\P 1600 E. BLOUNT tr.  \textit{Conestaggio 51 The thirde was smothered in the water.    
%\P 1665 MANLEY tr.  \textit{Grotius' Low C. Wars 221 They that escaped slaughter..were smother'd in the Mud.    
%\P 1713 ADDISON  \textit{Cato ii. vi, The helpless traveller..smother'd in the dusty whirlwind dies.    
%\P 1745 POCOCKE  \textit{Descr. East II. i. vi. 27 Being surrounded, and almost smothered by the crowd.    
%\P 1819 SHELLEY  \textit{Cenci ii. i. 143 How just it were to..smother me when overcome by wine.    
%\P 1864 M. E. BRADDON  \textit{Aurora Floyd xviii, What does the chap in the play get for his trouble when the blackamoor smothers his wife?    
%\P 1871 M. A. BARKER  \textit{Christmas Cake in Four Quarters iv. iii. 290, I had to bring 'em [sc. the mob of sheep] down uncommon easy, for it was a nasty place, and I didn't want half of 'em to be smothered in the creek.    
%\P 1930 L. G. D.  \textit{Acland Early Canterbury Runs 1st Ser. vi. 128 They once smothered 5000 in the gully.    
%\P a1948  \textit{ Ibid. (1951) 397 Run sheep..are very easy to s[mother] on broken hill ground... They s[mothere]d 1,200 once..at Mount Peel.
%
%\P 1742 YOUNG  \textit{Nt. Th. i. 147 Is it in the flight of three$\sim$score years, To..smother souls immortal in the dust?    
%\P 1781 COWPER  \textit{Truth 316 He begs their flatt'ry,..And, smother'd in't at last, is prais'd to death!    ?
%\P 1813 SHELLEY  \textit{Falsehood \& Vice 50 She smothered Reason's babes in their birth.    
%\P 1897 M. KINGSLEY  \textit{W. Africa 472, I therefore used to smother those twins by leading the conversation off.
%
%\P 1817 SHELLEY  \textit{Rev. Islam vi. xlix, I am Pestilence... I flit about, that I may slay and smother.
%
%c.I.1.c Used hyperbolically to denote an effusive welcome, etc., or the gaining of a complete or overwhelming victory.
%
%\P 1676 WYCHERLEY  \textit{Pl. Dealer iv. i, She..smothered me with a thousand tasteless kisses.    
%\P 1873 HOLLAND  \textit{A. Bonnicastle v. 98 In a moment I was smothered with welcome.
%
%\P 1890  \textit{Pall Mall G. 1 Dec. 1/3 If there is one club more than another which Notts County would care to smother it is Aston Villa.    
%\P 1900  \textit{Westm. Gaz.} 30 Mar. 2/2 They have simply smothered every scratch that has rowed against them.
%
%\itembf{2.} a.I.2.a To conceal by keeping silent about; to suppress all mention of, to hush up (a matter, etc.). Obs. (Now with up: see 6 a.)
%
%\P 1579 W. WILKINSON  \textit{Confut. Fam. Love 70 b, I lyke not to smother sinnes.    
%\P 1591 GREENE  \textit{Maidens Dr. ix, Bribes could not make him any wrong to smother.    
%\P 1642 GAUDEN  \textit{3 Serm. 48 As much as we defalk or smother of an inquired Truth.    
%\P 1699 BENTLEY  \textit{Phalaris 203 Somebody's artifice in suppressing and smothering what he thinks makes against him.    
%\P 1704 HEARNE  \textit{Ductor Hist. (1714) I. 344 Great Care has been taken to smoother his Name, but Theopompus..tells us, he was called Erostratus.    
%\P 1752 YOUNG  \textit{Brothers i. i, [Her story was] Smother'd by the king; And wisely too.
%
%b.I.2.b To cover up, so as to conceal or cause to be forgotten.
%
%\P c1585 FAIRE  \textit{Em i. 295 Where neither envious eyes nor thought can pierce, But endless darkness ever smother it.    
%\P 1613 JACKSON  \textit{Creed ii. 357 It was in their hearts, though hid and smothered in the wrinkles of their crooked hearts.    
%\P 1643 BAKER  \textit{Chron., Eliz. 120 Richard Hooker,..who with too much meeknesse smoothered his great Learning.    
%\P 1722 STEELE  \textit{Conscious Lovers i. ii, I am afraid..there's some$\sim$thing I don't see yet, something that's smother'd under all this Raillery.    
%\P 1863 KINGLAKE  \textit{Crimea (1876) I. vii. 100 So he began to turn this way and that, in order that by turmoil he might smother the past.
%
%c.I.2.c To repress, retain from displaying, (feeling, etc.) by the exercise of self-control.
%
%\P 1591 SHAKES.  \textit{1 Hen. VI,} iv. i. 110 Your priuate grudge my Lord of York, will out, Though ne'er so cunningly you smother it.    
%\P 1593  \textit{ Lucr. Argt., Smoothering his passions for the present, [he] departed with the rest.    
%\P 1624 CAPT. SMITH  \textit{Virginia iii. iii. 52 Smothering his distast to avoyd the Saluages suspition.    
%\P 1662 J. DAVIES tr.  \textit{Mandelslo's Trav. 245 The Gentleman..was a little troubled at it, but smother'd his indignation.    
%\P 1712 STEELE  \textit{Spect. No. 263 \phonetic{⁋}6 Both your Sisters are crying to see the Passion which I smother.    
%\P 1813 SHELLEY  \textit{Q. Mab iii. 43 Smothering the glow of shame.    
%\P 1847 PRESCOTT  \textit{Peru iii. vii. (1850) II. 190 Almagro..had seemed willing to smother his ancient feelings of resentment towards his associate.    
%\P 1891 E. PEACOCK  \textit{N. Brendon II. 101 She smothered her own grief.
%
%3. a.I.3.a To cover up so as to prevent from having free play or development; to suppress or check in this way.
%
%\P 1590 SHAKES.  \textit{Com. Err.} iii. ii. 35 My earthie grosse conceit: Smothred in errors.    
%\P 1605  \textit{ Macb. i. iii. 141 Function is smother'd in surmise.    
%\P 1650 H. MORE OBSERV. in  \textit{Enthus. Tri., etc. (1656) 108 You..by your slubbering and barbarous translating..smother the fitnesse of the Sense.    
%\P 1762 COWPER  \textit{To Miss Macartney 7 Dwells there a wish..To smother in ignoble rest At once both bliss and woe?    
%\P 1780  \textit{Mirror} No. 71, These exertions..would soon have been smothered by cold political prudence.    
%\P 1823 SCOTT  \textit{Quentin D. xxiv, Ridicule..often checks what is absurd, and fully as often smothers that which is noble.    
%\P 1843 R. J. GRAVES  \textit{Lect. Clin. Med. 371 You may smother the disease while it is merely local.    
%\P 1882 W. BALLANTINE  \textit{Exper. i. 9 Ability..smothered by pomposity and vulgar pride.
%
%b.I.3.b To prevent (words, etc.) from having full utterance; to render indistinct or silent.
%
%\P 1601 HOLLAND  \textit{Pliny I. 164 The fore-teeth..yeeld a distinction and varietie in our words,..drawing them out at length, or smuddering and drowning them in the end.    1797-
%\P 1809 COLERIDGE  \textit{Three Graves iv. xiv, No power Had she the words to smother.    
%\P 1821 CLARE  \textit{Vill. Minstr. I. 161 Contented she smother'd her sighs on his breast.    
%\P 1832 BREWSTER  \textit{Nat. Magic vii. 176 Suddenly the voice seemed smothered.
%
%c.I.3.c To stop (a cricket ball) by placing the bat more or less over it. Also in Assoc. Football (see quot. 1954).
%
%\P 1845 N. WANOSTROCHT  \textit{Felix on Bat i. iv. 18 Should it be pitched an inch too far, be sure to get well out at it, and smother it.    
%\P 1889 BOY'S  \textit{Own Paper 4 May 496 How the twists should smothered be Before they reach the middle stump.    
%\P 1954 F. C. AVIS  \textit{Soccer Dict. 112 Smother, to put oneself in the way of an opponent's shot, especially by the goalkeeper advancing from his goal towards the opponent.    
%\P 1976 NORTHUMBERLAND  \textit{Gaz. 26 Nov., His shot was smothered as the final whistle went.
%
%d.I.3.d Rugby Football. To tackle with a bear-like hug embracing the body and arms, preventing one's opponent from releasing the ball or touching it down.
%
%\P 1920 W. CAMP  \textit{Football without a Coach vii. 132 Unless experience shows that there is a certain definite play to watch or a certain player to smother.    
%\P 1928 SUNDAY  \textit{Times 5 Feb. 24/7 He kicked well ahead on the slippery turf, and after Hunt had smothered the full-back, scored.
%
%4. a.I.4.a To deaden or extinguish (fire, etc.) by covering so as to exclude the air; to cause to smoulder. Also fig.
%
%\P a1591 H. SMITH  \textit{Serm. (1637) 727 Many have smothered their light so long that the dampe hath put out the candle.    
%\P 1627 CAPT. SMITH  \textit{Seaman's Gram. xiii. 61 Smother the fire with wet cloathes.    
%\P 1657 AUSTEN  \textit{Fruit Trees ii. 143 Heat pent up and smoothered for a time.    
%\P 1758 REID tr.  \textit{Macquer's Chym. I. 141 If care be taken to smother them, so as to prevent their flaming while they burn.    
%\P 1787 JEFFERSON  \textit{Writ. (1859) II. 322 A fire, which, though smothered of necessity for the present moment, will probably never be quenched but by signal revenge.    
%\P 1837 CARLYLE  \textit{Fr. Rev.} i. iv. iv, A fiery fuliginous mass, which could not be choked and smothered, but would fill all France with smoke.    
%\P 1856 KANE  \textit{Arctic Expl. I. xxxii. 444, I succeeded in smothering the fire.
%
%b.I.4.b To cook in a close vessel. (Cf. smore v. 4.)
%
%\P 1706-7 FARQUHAR  \textit{Beaux' Strat. i. i, They'll eat much better smothered with onions.    1748- [see smothered 3].
%
%\itembf{5.} To cover up, cover over, densely or thickly by some thing or substance. (Common in recent use.)
%
%\P 1598 E. GUILPIN  \textit{Skial. (1878) 21 To..shew good legs, spite of slops smothering thies.    
%\P 1840 R. H. DANA  \textit{Bef. Mast xxxi. 113 In a few minutes the sails [were] smothered and kept in by clewlines and buntlines.    
%\P 1851 MAYHEW  \textit{Lond. Lab. II. 34/2 When dry and finished, we take what is called a ‘soft-heel-ball’ and ‘smother’ it over.    
%\P 1872 BLACK  \textit{Adv. Phaeton xxi. 297 The small stations we passed were smothered in green foliage.
%
%\itembf{6.} With up: a.I.6.a To conceal, suppress, hush up (a matter, etc.). Cf. sense 2 a.
%
%\P 1589  \textit{Pappe w. Hatchet B iv b, Hee woulde not smoother vp sinne, and deale in hugger mugger against his Conscience.    
%\P 1649 MILTON  \textit{Eikon.} ix. Wks.
%\P 1851 III.  \textit{401 The suspected Poysoning of his Father, not inquir'd into, but smother'd up.    
%\P 1687 MIéGE  \textit{Gt. Fr. Dict. ii. s.v., The Business was smothered up.    
%\P 1827 SCOTT  \textit{Surg. Dau. Pref., It was thought best to smother it up at the time.    
%\P 1883 STEVENSON  \textit{Treas. Isl. xiii, He's as anxious as you and I to smother things up.
%
%b.I.6.b To cover up in a close, dense, or suffocating manner, etc.
%
%\P c1590 GREENE  \textit{Fr. Bacon xiv, A nunne?.. Twere injurie to me, To smother up such bewtie in a cell.    
%\P 1592 SHAKES.  \textit{Ven. \& Ad.
%\P 1035 AND  \textit{there [the snail] all smother'd up, in shade doth sit.    
%\P 1631 GOUGE  \textit{God's Arrows iv. §13. 391 This fire..lay..smothered up.    
%\P 1644 J. FARY  \textit{God's Severity (1645) 23 The Lords wrath lies long smothered up, but at last it kindles.    
%\P 1820 KEATS  \textit{Hyperion i. 106, I am smother'd up, And buried from all godlike exercise.
%
%\itembf{7.} With down, out (see quots.). rare.
%
%\P 1632 LITHGOW  \textit{Trav. viii. 371 The..ingeniosity of their best styles..is ecclipsed, and smothered downe.    
%\P 1863 GARDENER'S  \textit{Chron. 23 May 493 The next year it may be noticed that the wished for crop has been smothered out.
%
%\itembf{II.} intr.
%
%\itembf{8.} To be suffocated or stifled; to be prevented from breathing freely by smoke or other means.
%
%\P c1520 EVERYMAN  \textit{796 What, sholde I smoder here?    
%\P 1648 HEXHAM  \textit{ii, Ick Smoore van den roock, I Smoother with the smoake, or, I am Choaked with the Vapour.    
%\P 1871 B. TAYLOR  \textit{Faust v. iv. (1875) II. 283 Ah, the good old father, mother, Doomed among the smoke to smother.    
%\P 1895  \textit{Cent. Mag. Aug. 628/2 One opinion was that he would not go into his hole because he was too hot and would smother.
%
%9. a.II.9.a To smoulder; to burn slowly. Now dial.
%
%\P 1600 SURFLET  \textit{Countrie Farme 558 Set on fire a quantitie of haye, after quench it againe by and by,..and whiles it is smoothering and smoaking, spread it vpon a plate of iron.    
%\P 1667 PEPYS  \textit{Diary} 29 July, The fire..lies smothering a great while..before it flames.    
%\P 1729 G. ADAMS tr.  \textit{Sophocles, Antig. iv. i. II. 56 The Fire shone not from the Sacrifices, but in the Ashes the Flame smothered.    
%\P 1804  \textit{Naval Chron. XI. 79 She will burn and smother to the Water's edge.    
%\P 1825 E. HEWLETT  \textit{Cottage Comforts vi. 42 Let the fire be banked up..with turves, which will smother on for hours.    1881- in dialect use (Notts., Leic., Warw.).
%
%b.II.9.b fig. or in fig. context.
%
%\P 1579 L. TOMSON  \textit{Calvin's Serm. Tim. 447/1 He will not haue our sinns couered, and lie smothering so, yt they may not be knowen.    
%\P 1588 GREENE  \textit{Pandosto (1607) 4 These..thoughts a long time smothering in his stomacke, began at last to kindle..a secret mistrust.    
%\P 1621 LADY  \textit{M. Wroth Urania 357 Heere began the harme to smother like wet hay in fire.    
%\P 1679 MANSELL  \textit{Narr. Popish Plot 5 When their old animosity did yet smoother.    
%\P 1697 COLLIER  \textit{Ess. Mor. Subj. ii. (1709) 65 A Man had better talk to a Post, than let his Thoughts lie Smoking and Smothering in his Head.
%
%c.II.9.c To die out in smoulder. rare—1.
%
%\P 1621 T. WILLIAMSON tr.  \textit{Goulart's Wise Vieillard 63 The heate of passions in youth beginning to coole and smoother out in old men.
%
%\itembf{10.} Of smoke: To escape slowly.
%
%\P 1725 DE FOE  \textit{Voy. round World (1840) 262 We saw a smoke indeed in the house, rather than coming out of it; and the little that did, smothered through a hole in the roof instead of a chimney.
%
%1\itembf{1.}1 Boxing. (See quot. 1954.)
%
%\P 1916 [see infight v. 2].    
%\P 1954 F. C. AVIS  \textit{Boxing Dict. 103 Smother, to prevent, by clever positioning of the arms, the development of an opponent's attack.
%
%Hence ˈsmotherable a., that may be smothered.
%
%\P 1824  \textit{Blackw. Mag.} XVI. 664 A woman who is not over fastidious in all her personal arrangements..is to me the most justifiably smotherable.
%
%
%
%\end{myenumerate}


%%%%%%%%%%%%%%%%%%%%%%%%%%%%%%%%%
\myitem{demur} v.

\noindent \phonetic{(dɪˈmɜː(r))}

\noindent [a. F. demeurer, in OF. demorer, -mourer (= Pr. and Sp. demorare, It. dimorare):—pop. L. dēmorāre = cl. L. dēmorārī to tarry, delay, f. de- I. 3 + morārī to delay. The OF. demor-, demour-, proper to the forms with atonic radical vowel, was at length assimilated to the tonic form demeur-; the latter gave the ME. forms demeore, demere: cf. people, and the forms meve, preve (F. meuve, preuve) of move, prove.]
\vspace{-0.3cm}

\begin{myenumerate}

\itembf{1.} intr. To linger, tarry, wait; fig. to dwell upon something. Obs.

\P a1225  \textit{Ancr. R.} 242 Auh ȝif ich hie swuðe uorðward, demeore ȝe þe lengre.    
\P c1300 \textit{K. Alis.}  7295 He n'ul nought that ye demere [rime dere].    
\P 1550 NICOLLS  \textit{Thucyd.} 73 (R.), Yet durst they not demoure nor abyde vpon the campe.    
\P 1559 BALDWIN in  \textit{Mirr. Mag.} (1563) 39 b, Take hede ye demurre not vpon them.    
\P 1595 SOUTHWELL  \textit{St. Peter's Compl.} 19 But ô, how long demurre I on his eyes.    
\P 1604 T. WRIGHT  \textit{Passions} v. 213, I demurre too long in these speculative discourses.    
\P 1653 URQUHART  \textit{Rabelais} i. ii, If that our looks on it demurre.

\itembf{b.} To stay, remain, abide. Obs.

\P 1523 \textit{St. Papers  Hen. VIII}, IV. 34 She cannot demore there without extreme daunjur and peril.    
\P 1536  \textit{Act 28 Hen. VIII}, c. 10 Any person..dwellyng, demurryng, inhabitinge or resiant within this realme.    
\P 1550 NICOLLS  \textit{Thucyd.} 72 (R.) The sayde Peloponesyans demoured in the land.

\itembf{c.} To last, endure, continue. Obs.

\P 1547 HOOPER  \textit{Declar. Christ} iii. Wks. (Parker Soc.) 21 This defence..shall demour for ever till this church be glorified.

\itembf{2.} trans. To cause to tarry; to put off, delay.

\P 1613 PURCHAS  \textit{Pilgrimage} ii. xviii. 174 Whose judgement is demurred until the day of Reconciliation.    
\P 1635 QUARLES  \textit{Embl.} iv. x. (1818) 239 The lawyer..then demurs me with a vain delay.    
\P 1682 D'URFEY  \textit{Butler's Ghost} 69, I swear.. Henceforth to take a rougher course, And, what you would demur to force.

\itembf{3.} intr. To hesitate; to delay or suspend action; to pause in uncertainty. Obs.

\P 1641 MILTON  \textit{Ch. Govt.} vii. (1851) 135 This is all we get by demurring in Gods service.    
\P 1654 CODRINGTON tr.  \textit{Hist. Ivstine} 418 He found the King to demur upon it.    
\P 1655 FULLER  \textit{Ch. Hist.} ii. ii. §40 King Edwine demurred to embrace Christianity.    
\P 1699 BENTLEY  \textit{Phal.} 516 The Delphians demurring, whether they should accept it or no.    
\P 1743 J. DAVIDSON  \textit{Æneid} viii. 261 You need not demur to challenge.    
\P 1778 F. BURNEY  \textit{Evelina} li, You are the first lady who ever made me even demur upon this subject.    
\P 1818 W. TAYLOR in  \textit{Monthly Rev.} LXXXVII. 534 All the Yorkists could thus co-operate, without demurring between their rightful sovereigns.

\itembf{b.} To be of doubtful mind; to remain doubtful. Obs. rare.

\P 1612 T. TAYLOR  \textit{Comm. Titus} i. 3 And demurre with the Philistines, whether God or Fortune smite vs.    
\P a1628 F. GREVILLE  \textit{Sidney} (1652) 237 To have demurred more seriously upon the sudden change in his Sonne.

\itembf{c.} trans. To hesitate about. Obs. rare.

\P 1667 MILTON  \textit{P.L.} ix. 558 What may this mean? Language of Man pronounc't By Tongue of Brute, and human sense exprest? The first..I thought deni'd To Beasts..The latter I demurre, for in thir looks Much reason, and in thir actions oft appeers.    
\P a1730 E. FENTON  \textit{Hom. Odyss.} xi. Imit. (Seager), Let none demur Obedience to her will.

\itembf{4.} intr. To make scruples or difficulties; to raise objection, take exception to (occas. at, on). (The current sense; often with allusion to the legal sense, 5.)

\P 1639 FULLER  \textit{Holy War} ii. xxxvi. (1840) 98 The caliph demurred hereat, as counting such a gesture a diminution to his state.    
\P 1751 C. LABELYE  \textit{Westm. Br.} 93, I..gave my Directions..which being in some Measure demurred to, the Matter was brought before the Board.    
\P 1775 SHERIDAN  \textit{Rivals} ii. ii, My process was always very simple—in their younger days, 'twas ‘Jack, do this’—if he demurred, I knocked him down.    
\P 1807 SOUTHEY  \textit{Espriella's Letters} III. 29 They are so unreasonable as to demur at finding corn for them.    
\P 1855 BROWNING  \textit{Let. to Ruskin}, I cannot begin writing poetry till my imaginary reader has conceded licences to me which you demur at altogether.    
\P 1860 TYNDALL  \textit{Glac.} i. v. 40 My host at first demurred..but I insisted.    
\P 1875 MCLAREN  \textit{Serm.} Ser. ii. ix. 150 We can afford to recognise the fact, though we demur to the inference.

\itembf{b.} trans. To object or take exception to. rare.

\P 1827 H. H. WILSON  \textit{Burmese War} (1852) 25 As the demand was unprecedented, the Mugs, who were British subjects, demurred payment.    
\P 1876 GLADSTONE  \textit{Homeric Synchr.} 59, I demur the inference from these facts.

\itembf{5.} Law. (intr.) To put in a demurrer.

\P [a1481 LITTLETON  \textit{Tenures} §96 Et fuist demurre en iudgement en mesme le plee, le quel les xl. iours serront accompts de le primer iour del muster de host le Roy.]    
\P 1620 J. WILKINSON  \textit{Coroners \& Sherifes} 60 It was demurred on in Law.    
\P 1628 COKE  \textit{On Litt.} 70 a, And it was demured in iudgement in the same plea, whither the 40 dayes should bee accounted from the first day of the muster of the kings host.    Ibid. 72 a, He that demurreth in Law confesseth all such matters of fact as are well and sufficiently pleaded.    
\P 1641 in  Rushw. \textit{Hist. Coll.} iii. (1692) I. 334 To which Plea Mr. Attorney-General demurred in Law, and the said Samuel Vassall joyned in Demurrer with him.    
\P 1660  \textit{Trial of Regic.} 107, I must demur to your Jurisdiction.    
\P 1681  \textit{Trial S. Colledge} 10 And if so be matter of Law arises upon any evidence that is given against you..you may demurr upon that Evidence, and pray Counsel of the Court to argue that demurrer.    
\P 1848 MACAULAY  \textit{Hist. Eng.} II. 84 The plaintiff demurred, that is to say, admitted Sir Edward's plea to be true in fact, but denied that it was a sufficient answer.
\end{myenumerate}


%%%%%%%%%%%%%%%%%%%%%%%%%%%%%%%%%
%\myitem{rejoinder} n.
%
%\noindent \phonetic{(rɪˈdʒɔɪndə(r))}
%
%\noindent [a. F. rejoindre inf. used as n. The sense is prob. from AF. usage: see rejoin v.1]
%\vspace{-0.3cm}
%
%\begin{myenumerate}
%
%\itembf{1.} Law. The defendant's answer to the plaintiff's replication.
%
%\P 1482 in  \textit{I. S. Leadam Star Chamber Cas. (Selden Soc.) 14 This is the reioyner of John Attwyll..to the replicacion of John Tayllour.    
%\P 1540  \textit{Act 32 Hen. VIII, c. 30 §1 Replycacyons, reioynders, rebutters..and other pleadynges.    
%\P 1588 FRAUNCE  \textit{Lawiers Log. i. ii. 10 In every count, barre, replication, rejoynder, \&c.    
%\P 1649 W. M. WANDERING  \textit{Jew (1857) 48 She has Demurs, and Replications, and Rejoynders; but my case hangs.    
%\P a1683 SCROGGS  \textit{Courts-leet (1714) 168 If they proceed nor further by Replication, Rejoinder, Surrejoinder.    
%\P 1768 BLACKSTONE  \textit{Comm.} III. 310 The rejoinder must support the plea, without departing out of it.    
%\P 1885  \textit{Law Times Rep. LIII. 486/1 Rejoinder of issue was made.
%
%\itembf{2.} An answer to a reply (common in the titles of books and pamphlets); also simply, a reply.
%
%\P 1566 HARDING  \textit{(title) A rejoindre to Mr. Jewels Replie.    
%\P 1609 R. PARSONS  \textit{Quiet Reckoning title-p., In a large Preamble to a more ample Reioynder promised by him.    
%\P 1659 BP. WALTON  \textit{Consid. Considered 306, I shall promise to deal in like manner with him, if any rejoinder shall be found needful.    
%\P 1726 POPE  \textit{Odyss.} xx. 231 Rejoinder to the churl the King disdain'd.    
%\P 1759 FRANKLIN Ess. Wks.
%\P 1840 III.  \textit{232 The assembly took the governor's reply..into consideration, and prepared a suitable rejoinder.    
%\P 1877 FROUDE  \textit{Short Stud. (1883) IV. i. x. 125 An answer came in a form to which in that age no rejoinder was possible.
%
%\itembf{b.} Without article, in phr. in rejoinder.
%
%\P 1556 [see rejoin v.1 2].    
%\P 1844 DE QUINCEY in  \textit{‘H. A. Page’ Life (1877) I. xv. 332 In rejoinder to your note of Wednesday morning, I wrote an answer.
%
%So reˈjoinder v. intr., to reply. Obs. rare—1.
%
%\P a1660 HAMMOND  \textit{Serm. xix. Wks.
%\P 1683 IV. 604  \textit{When Nathan shall rejoynder with a Thou art the man,..then their hearts come to the touchstone.
%
%
%
%\end{myenumerate}
%
%
%%%%%%%%%%%%%%%%%%%%%%%%%%%%%%%%%
%\myitem{suffuse} v.
%
%\noindent \phonetic{(səˈfjuːz)}
%
%\noindent [f. L. suffūs-, pa. ppl. stem of suffundĕre, f. suf- = sub- 2, 26 + fundĕre to pour.]
%\vspace{-0.3cm}
%
%\begin{myenumerate}
%
%\itembf{1.} trans. To overspread as with a fluid, a colour, a gleam of light. \itembf{a.} of tears, moisture. Chiefly pass.
%
%\P 1590 [see suffused 1].    
%\P 1600 FAIRFAX  \textit{Tasso xii. lxxiv, His eies vnclos'd, with teares suffused.    
%\P 1754 HUME  \textit{Hist. Eng., Chas. I, x. I. 461 Hamilton long followed him with his eyes, all suffused in tears.    
%\P 1773-83 HOOLE  \textit{Orl. Fur. xviii.
%\P 1162 WHILE  \textit{tears his cheeks suffuse.    
%\P 1797 S. \& HT.  \textit{Lee Canterb. T. (1799) I. 352 His whole frame [was] suffused with a cold dew.    
%\P 1838 PRESCOTT  \textit{Ferd. \& Is. xiii. II. 115 Every eye was suffused with tears.
%
%\itembf{b.} of light, air, fire, colour. Often in fig. context.
%
%\P 1728-46 THOMSON Spring
%\P 1086 DARK  \textit{looks succeed; Suffus'd, and glaring with untender fire.    
%\P 1786 tr.  \textit{Beckford's Vathek (1883) 33 To hide the blush of mortification that suffused their foreheads.    
%\P 1813 SHELLEY  \textit{Q. Mab vi. 25 A kindling gleam of hope Suffused the Spirit's lineaments.    
%\P 1818 WORDSW.  \textit{Even. Volunt. ix. 45 Yon hazy ridges..Climbing suffused with sunny air.    
%\P 1860 TYNDALL  \textit{Glac. i. xxv. 184 The glorious light..suffused with gold and crimson the atmosphere itself.    
%\P 1877 BLACK  \textit{Green Past. xxxv. 283 The beautiful colour that for a second suffused her blushing face.    
%\P 1882 GARDEN  \textit{5 Aug. 119/1 Sepals and flowers white, suffused at base with rosy lilac.
%
%\itembf{c.} transf. and fig.
%
%\P 1813 COLERIDGE  \textit{Night-scene 43 Eyes suffused with rapture.    
%\P 1867 J. B. ROSE tr.  \textit{Virg. Æneid 160 The crowded ranks Of disembodied Shades suffused the banks.    
%\P 1868 HELPS  \textit{Realmah ii. (1876) 10 The most commonplace objects being suffused with beauty.    
%\P 1876 HOLLAND  \textit{Sev. Oaks xv. 234 The amused expression suffused the lawyer's face.    
%\P 1879 S. C. BARTLETT  \textit{Egypt to Pal. v. 101 The life and literature of the nation were suffused with these reminiscences.
%
%\itembf{2.} To pour (a liquid) over a surface. (Also refl.) Chiefly in fig. context.
%
%\P 1734 tr.  \textit{Rollin's Roman Hist. (1827) III. vii. 328 Suffusing over the study of philosophy the dye of rhetoric.    
%\P 1815  \textit{Ann. Reg., Chron. 92/2 Water, sugar, \&c. from the boiler and pans..suffused thickly upon the trees.    
%\P 1829 I. TAYLOR  \textit{Enthus. x. 282 The healing flood of Christian truth shall suffuse itself in all directions.    
%\P 1854  \textit{Jrnl. R. Agric. Soc. XV. ii. 427 Springs, suffused from higher grounds.
%
%
%
%\end{myenumerate}

%%%%%%%%%%%%%%%%%%%%%%%%%%%%%%%%
\myitem{imposition} n.

\noindent \phonetic{(ɪmpəʊˈzɪʃən)}

\noindent [ME. ad. L. impositiōn-em, n. of action from impōnĕre to place upon, impose, or a. OF. imposition, -icion (1317 in Godef.). First used in the special senses 1 b, 2, 5.]
\vspace{-0.3cm}

\begin{myenumerate}

\itembf{1.} The action of putting, placing, or laying on. Also concr. A layer over something. rare.

\P 1597 A. M. tr.  \textit{Guillemeau's Fr. Chirurg.} 38/2 The imposition of the fingers one the mouthes of the Veynes.    
\P 1599  tr. \textit{ Gabelhouer's Bk. Physicke} 54/2 On the sayed Straweberryes you must agayne strewe saulte, and agayne theron an impositione of strawberryes,..continuing the impositione of one on the other till the basen be repleate.    
\P 1833 MEDWIN  \textit{Shelley} (1847) II. 48 The imposition of my hand on his forehead, instantly put a stop to his spasms.    
\P 1888  \textit{Pall Mall G.} 6 Dec. 5/1 A Japanese lacquer box..in various stages of development, from the imposition of colour on the first stone to the last.

\itembf{b.} spec. The laying on of hands in blessing, ordination, confirmation, etc. [L. impositio, Vulgate, Acts viii. 18.]

\P 1382 WYCLIF  \textit{Bible Pref. Ep. Jerome} iii, The grace, the which is ȝouun to hym bi imposicoun [1388 puttyng  to] of the prestis hond.    
\P 1548 CRANMER  \textit{Catech.} 230 The ministration of Gods worde..was deryued from the Apostles vnto other after theim by imposition of handes, and gyuynge the holy ghost.    
\P 1597 HOOKER  \textit{Eccl. Pol.} v. lxvi. §1 With prayers of spiritual and personal benediction the manner hath been in all ages to use imposition of hands, as a ceremony betokening our restrained desires to the party, whom we present unto God by prayer.    
\P 1660 JER. TAYLOR  \textit{Worthy Commun.} i. iii. 59 Thus we find that the grace of God is given by the imposition of hands.    
\P 1796 MORSE  \textit{Amer. Geog.} II. 157 Ministers, or preaching presbyters..alone can..assist at the imposition of hands upon other ministers.    
\P 1885  \textit{Catholic Dict.} (ed. 3) s.v., In two instances (the imposition of hands in ordination and confirmation) it [the rite] has received a sacramental efficacy.

\itembf{c.} Print. The imposing or arranging of pages of type in the forme.

\P 1824 J. JOHNSON  \textit{Typogr.} II. vii. 144 A general outline for the imposition of whatever odd matter there may be at the conclusion of a work.    Ibid. xiv. 495 Pages..laid down for imposition, without folios or head lines, must be rectified by the person who has been slovenly enough to adopt this plan.

\itembf{2.} The action of attaching, affixing, or ascribing; bestowal (of a name, etc.).

\P 1387-8 T. USK  \textit{Test. Love} ii. iv. (Skeat) l. 141 Wel, quod I, this inpossession I wol wel understande.    
\P 1430-40 LYDG.  \textit{Bochas} i. i. (1544) 1 b, Adam made an imposicion..to those beastes all Of very reason what men should them call.    
\P 1599 HAKLUYT  \textit{Voy.} II. ii. 89 Termed Cantam, which is rather the common name of the prouince, then a word of their proper imposition.    
\P 1709 STEELE  \textit{Tatler} No. 49 \phonetic{⁋}1 The Imposition of honest Names and Words upon improper Subjects.    
\P 1870 J. H. NEWMAN  \textit{Gram. Assent} i. §2. 7 By our apprehension of propositions I mean our imposition of a sense on the terms of which they are composed.

\itembf{3.} Imputation, accusation, charge. Obs.

\P 1611 SHAKES.  \textit{Wint. T.} i. ii. 74 The Imposition clear'd, Hereditarie ours.

\itembf{4.} The action of imposing or laying as a burden, duty, charge, or task; the action of inflicting, levying, enjoining, or enforcing.

\P 1593 SHAKES. Lucr. 1697 At this  request..Each present Lord began to promise aide, As bound in Knighthood to her imposition.    
\P 1594 HOOKER  \textit{Eccl. Pol.} i. ii. §6 The Imposition of this Law upon himself is his own free and voluntary Act.    
\P 1621 BURTON  \textit{Anat. Mel.} i. ii. i. ii. (1651) 51 The superstitious impositions of fasts.    
\P 1841 MYERS  \textit{Cath. Th.} iii. §34. 123 Opinions..not derived from forcible external imposition.    
\P 1845 MCCULLOCH  \textit{Taxation} i. iv. 108 The effects that would result from the imposition of taxes.

\itembf{b.} The levying of a tax; taxation. Obs.

\P c1374 CHAUCER  \textit{Boeth.} i. pr. iv. 9 (Camb. MS.) Coempcion..þat weere estabelyssed vp on the poeple by swich a manere imposiscion as who so bowhte a bossel corn he moste yeue the kynge the fifte part.    
\P 1628 in  \textit{Clarendon Hist. Reb.} iii. §217 Any power of Imposition upon any Merchandizes.

\itembf{5.} Anything imposed, levied, or enjoined: \itembf{a.} An impost; tax, duty; spec. in pl. duties upon imports and exports imposed by the royal prerogative.

\P c1460 FORTESCUE  \textit{Abs. \& Lim. Mon.} x. (1885) 132 He takith certayn imposicions made by hym selff vppon euery oxe.    
\P 1483  \textit{Act 1 Rich. III}, c. 2 A new Imposition called a Benevolence.    
\P a1533 LD. BERNERS  \textit{Huon} lx. 210 He hath reissyd vp in all his londes new taylles \& gables \& inpossessyons.    
\P 1689 BURNET  \textit{Tracts} I. 44 Those who stay behind, can scarce live and pay those grievous Impositions that are laid upon them.    
\P 1839 KEIGHTLEY  \textit{Hist. Eng.} I. 83 The lands of the church were also subject to the ordinary impositions for the public service.    
\P 1863 H. COX  \textit{Instit.} iii. ii. 601 Prerogative impositions at the ports were dormant from the reign of Edward III. to that of Mary.

\itembf{b.} A command, charge, or ordinance imposed or laid upon one. Obs.

\P 1596 SHAKES.  \textit{Merch. V.} iii. iv. 33, I doe desire you Not to denie this imposition, The which my loue and some necessity Now layes vpon you.    
\P 1601 HOLLAND  \textit{Pliny} II. 513 In those capitulations of peace..I find this expresse article and imposition, that they should not vse yron, but only about tillage of the ground.    
\P 1637 R. HUMFREY tr.  \textit{St. Ambrose} i. 141 His imposition, ‘let those in Iudea flie to the mountaines’.    
\P 1664 H. MORE  \textit{Myst. Iniq.} iii. 7 The decrees and ceremonial impositions of men.

\itembf{c.} A literary exercise or task imposed as a punishment at school or college. (Colloquially abbreviated impo or impot.)

\P 1746 WARTON  \textit{Progr. Discontent} 121 When impositions were supplyd To light my pipe, or sooth my pride.    
\P 1785  \textit{ Minor Poems Milton} 422 note (Webster), Literary tasks called impositions.    
\P 1806-7 J. BERESFORD  \textit{Miseries Hum. Life} (1826) xii. Concl. 322, I have never forgotten the passage, since I once translated it at Oxford as an imposition.    
\P 1844 J. T. J.  \textit{Hewlett Parsons \& W.} xv, The penalty for transgressing this..was a long imposition—task some would call it.    
\P 1899  \textit{Punch} 22 Feb. 88/2, I..got an ‘impot’ for cribbing a Greek exercise.

\itembf{6.} The action of imposing upon or deceiving by palming off what is false or unreal; an instance of this, an imposture.

\P 1632 LITHGOW  \textit{Trav.} iii. 108 When the flat contrary of his abjured impositions, is infallibly knowne to be of undoubted trueth.    
\P 1708 SWIFT  \textit{Death Partridge}, The predictions you printed..were mere impositions on the people.    
\P 1749 FIELDING  \textit{Tom Jones} xvi. ix, He was afraid Miss Western would never agree to an imposition of this kind.    
\P 1875 JOWETT  \textit{Plato} (ed. 2) II. 83 He who would either impose on others or escape imposition must know the truth.
\end{myenumerate}

%%%%%%%%%%%%%%%%%%%%%%%%%%%%%%%%
\myitem{adjuvant} a. and n.

\noindent \phonetic{(ˈædʒ(j)uːvənt)}

\noindent [a. Fr. adjuvant (16th c. in Litt.), ad. L. adjuvant-em, pr. pple. of adjuvā-re to assist; f. ad to + juvā-re to help.]
\vspace{-0.3cm}

\begin{myenumerate}

\itembf{A.} adj. Assisting, aiding, helpful, auxiliary.

\P a1614 P. LILIE  \textit{2 Serm.} (1619) 3, I doe not say they are principall causes, but instrumentall, adjuvant, secundary, inferiour causes.    
\P 1650 GREENHILL  \textit{On Ezek.} (1874) Ded. 4 It is my unhappiness that I cannot be sufficiently adjuvant to such Princely beginnings.    
\P 1836 TODD  \textit{Cycl. Anat. \& Phys.} I. 645/2 Used as adjuvant respiratory organs.    
\P 1874 WEBSTER  \textit{Rep. Patent Congress at Vienna} iv. 355 An examination system which should be adjuvant and advisory to the applicant.

\itembf{B.} n. [The adj. used absol.] A person or thing helping or aiding; a help, helper, or assistant. spec. in Med. A substance added to a prescription to assist the action of the principal ingredient or ‘base.’

\P 1609 YELVERTON in  \textit{Archæol.} XV. 51 (T.) I have only been a careful Adjuvant, and was sorry I could not be the efficient.    
\P 1654 T. WHITAKER  \textit{Bl. of Grape} 2 (T.) These [plants] are adjuvants by reason of their cathartique quality.    
\P 1865 HUXLEY ETHNOLOGY in  \textit{Crit. \& Addr.} 1873, vii. 138 The value of philology as an adjuvant to ethnology.    
\P 1875 WOOD  \textit{Therap.} (1879) 83 Serpentaria, An elegant stimulant tonic, especially useful as an adjuvant to more powerful bitters.
\end{myenumerate}


%%%%%%%%%%%%%%%%%%%%%%%%%%%%%%%%%
\myitem{subservient} a. (n.)

\noindent \phonetic{(səbˈsɜːvɪənt)}

\noindent [ad. L. subserviens, -entem, pr. pple. of subservīre to SUBSERVE.]
\vspace{-0.3cm}

\begin{myenumerate}

\itembf{A.} adj.

\itembf{1.} Being of use or service as an instrument or means; serving as a means to further an end, object, or purpose; serviceable. Const. to a person or thing, a design, condition, process.

\P 1632 TATHAM  \textit{Love crowns the end} I. Dram. Wks. (1878) 19 If these eyes be my own, I fondly trust They may be more subservient to me.    
\P 1651 BAXTER  \textit{Inf. Bapt.} 144 If they do preach any wholsom Doctrine, it is usually but subservient to their great Design.    
\P 1656 RIDGLEY  \textit{Pract. Physick} 55 The spirits..subservient to the imagination in the Brain.    
\P 1690 LOCKE  \textit{Hum. Und.} ii. ix. §7 Ideas, which we may..suppose may be introduced into the Minds of Children in the Womb, subservient to the necessity of their Life..there.    
\P 1729 BUTLER \textit{Serm.} Wks. 1874 II. 150  Every particular affection..is subservient to self-love.    
\P 1781 GIBBON  \textit{Decl. \& F.} xviii. (1787) II. 99 The arts of fraud were made subservient to the designs of cruelty.    
\P 1873 SYMONDS  \textit{Grk. Poets} vii. 189 The drama renders all arts subservient to the one end of action.    
\P 1879 G. C. HARLAN  \textit{Eyesight} ii. 18 All the other structures of the eye may be considered subservient to this one [the retina].

\itembf{b.} Const. to with inf. or a prep. with gerund.

\P 1668 DRYDEN  \textit{Dram. Poesy} Wks. 1725 I. 43 They  dwell on him and his concernments, while the rest of the Persons are only subservient to set him off.    
\P 1714 R. FIDDES  \textit{Pract. Disc.} ii. 145 Persons who are subservient in this respect towards promoting the honour of God.    
\P 1719 YOUNG  \textit{Revenge} iii. i, This is a good subservient artifice, To aid the nobler workings of my brain.    
\P 1755 SMOLLETT  \textit{Quix.} (1803) II. 23 In making you subservient in facilitating our success.

\itembf{c.} without construction. Obs.

\P 1650 BULWER  \textit{Anthropomet.} 173 They are not in the number of them that perform an action, but of those that are subservient.    
\P 1661 J. FELL  \textit{Hammond} 112 Scarce ever reading any thing which he did not make subservient in one kinde or other.    
\P 1701 GREW  \textit{Cosmol. Sacra} ii. i. 36 While we are awake, we feel none of those Motions, which are continually made, in the disposal of the Corporeal Principles Subservient herein.

\itembf{2.} Acting or serving in a subordinate capacity; subordinate, subject. Const. to. 
\textbf{a.} of persons.

\P 1647 CLARENDON  \textit{Hist. Reb.} i. §140 That the Queen might have solely that Power, and he only be Subservient to her.    
\P 1667 DECAY  \textit{Chr. Piety} ii. \phonetic{⁋}13 Can we think he will be patient thus to be made subservient to his enemy?    
\P 1711 G. HICKES  \textit{Two Treat. Chr. Priesth.} (1847) II. 79 The deacons as subservient inferior ministers.    
\P 1721 PRIOR  \textit{Predest.} 63 Wks. 1907 II. 347  Is God subservient to his own Decree?    
\P 1873 HAMERTON  \textit{Intell. Life} vii. vi. 258 Women are by nature far more subservient to custom than we are.    
\P 1880 ‘VERNON  LEE’ \textit{Italy} iii. i. 73 They wanted the singer to remain subservient to the composer.

\itembf{b.} of things.

\P 1641 MILTON  \textit{Ch. Govt.} iii. Wks. 1851 III.  109 Copies out from the borrow'd manuscript of a subservient scrowl.    
\P 1656 TUCKER \textit{Rep.} in  \textit{Misc. Scott. Burgh Rec. Soc.} 19 The towne is a mercat towne, but subservient and belonging..to the towne of Lynlithquo.    
\P 1687 DRYDEN  \textit{Hind \& P.} i. 88 Superiour faculties are set aside, Shall their subservient organs be my guide?    
\P 1709 POPE  \textit{Ess. Crit.} 263 Most Critics, fond of some subservient art, Still made the Whole depend upon a Part.    
\P 1864 PUSEY  \textit{Lect. Daniel} ii. 88 Antiochus Epiphanes..directed against God what was to be subservient to God.    
\P 1870 DISRAELI  \textit{Lothair} xii, Assuming that religion was true..then religion should be the principal occupation of man, to which all other pursuits should be subservient.

\itembf{c.} Law. (Cf. SERVIENT and SERVITUDE 7.)

\P 1681 STAIR  \textit{Inst. Law Scot.} i. xvi. 327 Personal Servitudes are, whereby the property of one is subservient to the person of another.    
\P 1681 [see SERVITUDE 7].    
\P 1884  \textit{Law Rep.} 25 Chanc. Div. 580 The mortgagees of C, D, and E..acquiesced in those blocks being made subservient to the adjoining block B.

\itembf{3.} Of persons, their actions, etc.: Slavishly submissive; truckling, obsequious.

\P 1794 MRS. RADCLIFFE  \textit{Myst. Udolpho} xlviii, Emily was..disgusted by the subservient manners of many persons, who [etc.].    
\P 1819 SCOTT  \textit{Ivanhoe} xxi, The foreigner came here poor, beggarly, cringing, and subservient.    
\P 1839 JAMES  \textit{Louis XIV}, IV. 251 He contrived to ally this subservient flattery to a degree of intemperate vehemence towards Louis.    
\P 1874 GREEN  \textit{Short Hist.} viii. §2 (1882) 472 The lawyers had been subservient beyond all other classes to the Crown.

\itembf{B.} n. A subservient person or thing. rare.

\P 1867 D. PAGE  \textit{Man} 143 The primitive notion that this earth was the centre of the universe, and the sun, moon, and stars, formed merely to be its subservients.    
\P 1898 MEREDITH  \textit{Odes Fr. Hist.} 35 The fair subservient of Imperial Fact.
\end{myenumerate}


%%%%%%%%%%%%%%%%%%%%%%%%%%%%%%%%%
\myitem{efficacious} a.

\noindent \phonetic{(ɛfɪˈkeɪʃəs)}

\noindent [f. L. efficāci- (see prec.) + -ous: see -acious.]
\vspace{-0.3cm}

That produces, or is certain to produce, the intended or appropriate effect; effective. (Said of instruments, methods, or actions; not, in prose, of personal agents.)

\P 1528 ROY  \textit{Sat.} (1845) Goddis worde is so efficacious.    
\P 1651 BIGGS  \textit{New Disp.} 35 Lesse efficacious, that is, in plain English ineffectual.    
\P 1669 GALE  \textit{Crt. Gentiles} i. iii. iii. 39 He saies it is the first efficacious cause of the Being of althings.    
\P a1679 T. GOODWIN  \textit{Wks.} (1863) VII. 510 God..vouchsafeth..efficacious grace to overcome temptation.    
\P 1744 BERKELEY  \textit{Siris} §58 Soap, therefore, is justly esteemed a most efficacious medicine.    
\P 1830 LYELL  \textit{Princ. Geol.} (1875) II. iii. xli. 421 Variation and Natural Selection will be efficacious in forming distinct races in separate islands.    
\P 1860 MILL  \textit{Repr. Govt.} (1865) 51/2 To provide efficacious securities against this evil.    
\P 1873 BROWNING  \textit{Red. Cott. Nt.-Cap} 497 Be efficacious at the Council there.


%%%%%%%%%%%%%%%%%%%%%%%%%%%%%%%%%
\myitem{ennui} n.

%\noindent \phonetic{(‹fatatilde›nɥi)} % French
\noindent \phonetic{(ɑ̃nɥi)}

\noindent [a. Fr. ennui, OF. enui:—L. in odio: see ANNOY, ENNOY, which are older adoptions of the same Fr. word.

So far as frequency of use is concerned, the word might be regarded as fully naturalized; but the pronunciation has not been anglicized, there being in fact no Eng. analogy which could serve as a guide.]
\vspace{-0.3cm}

\begin{myenumerate}
\itembf{a.} The feeling of mental weariness and dissatisfaction produced by want of occupation, or by lack of interest in present surroundings or employments.

\P 1667 EVELYN  \textit{Mem.} (1857) III. 161 We have hardly any words that do..fully express the French naivete, ennui, bizarre, etc.    
\P 1732 BERKELEY  \textit{Alciphr.} ii. §17 They should prefer doing anthing to the ennui of their own conversation.]    
\P 1758 CHESTERFIELD  \textit{Lett.} IV. 117 In less than a month the man, used to business, found that living like a gentleman was dying of ennui.    
\P 1789 MRS. PIOZZI  \textit{Journ. France} II. 388 Muse! prepare some sprightly sallies To divert ennui at Calais.    
\P 1801 M. EDGEWORTH  \textit{Angelina} i. 10 She felt insupportable ennui from the want of books and conversation suited to her taste.    
\P 1871 DARWIN  \textit{Desc. Man} I. ii. 42 Animals manifestly enjoy excitement and suffer from ennui.

\itembf{b.} Personified. \textbf{c.} concr. A cause of ennui.

\P 1790 C. M. GRAHAM  \textit{Lett. Educ.} 290 It would entirely subdue the dæmon Ennui.    
\P 1812 H. \& J.  \textit{Smith Rej. Addr., Cui Bono} i, The fiend Ennui awhile consents to pine.    
\P 1847 W. E. FORSTER in  \textit{T. W. Reid Life} (1888) I. vii. 208 We drove to a first-class hotel..a stylish, comfortless temple of ennui.    
\P 1849 C. BRONTË  \textit{Shirley} vii. 87 Every stitch she put in was an ennui.
\end{myenumerate}


%%%%%%%%%%%%%%%%%%%%%%%%%%%%%%%%
\myitem{revere} v.

\noindent \phonetic{(rɪˈvɪə(r))}

\noindent [ad. F. révérer or L. reverērī, f. re- RE- + verērī to fear.]
\vspace{-0.3cm}

\begin{myenumerate}

\itembf{1.} trans. To hold in, or regard with, deep respect or veneration.

\P 1661 in  Blount \textit{Glossogr}.    
\P 1665 GLANVILL  \textit{Def. Van. Dogm.} 53 If Aristotle were vicious and immoral, there is much the less reason why we should revere his Authority.    
\P 1691 HARTCLIFFE  \textit{Virtues} 137 Sudden Anger reveres or stands in awe of no Man.    
\P 1717 POPE  \textit{Iliad} ix. 754 Revere thy roof, and to thy guests be kind.    
\P 1761 HUME  \textit{Hist. Eng.} liii. III. 174 The people..generally abhorred the Convocation as much as they revered the Parliament.    
\P 1837 WHEWELL  \textit{Hist. Induct. Sci.} (1857) I. 213 Works which were long revered as a code of science.    
\P 1864 BRYCE  \textit{Holy Rom. Emp.} v. (1875) 68 For all..had heard of Rome's glories, and revered the name of Cæsar.

absol. \P 1797 MRS. RADCLIFFE  \textit{Italian} xvii, Tremble, therefore, and revere.    
\P 1848 LYTTON  \textit{Harold} vii. v, The love that reveres.

\itembf{2.} With inf. To be reluctant to do something, through a feeling of respect. Obs. rare—1.

\P 1689 HICKERINGILL  \textit{Ceremony-Monger} v, If I did not revere to cast Dirt upon the Ashes of the Dead, I could [etc.].

\noindent Hence \phonetic{reˈvered} ppl. a.

\P 1787 BURNS  \textit{Addr. to W. Tytler} 1 Revered defender of beauteous Stuart.    
\P 1818 SHELLEY  \textit{Hymn Earth} 24 Such delights by thee Are given, rich Power, revered Divinity.    
\P 1836 THIRLWALL  \textit{Greece} xxiv. III. 311 The treaties were..preserved in the most revered sanctuaries.    
\P 1875 JOWETT  \textit{Plato} (ed. 2) I. 327 As I was saying, revered friend, the abundance of your wisdom makes you indolent.
\end{myenumerate}


%%%%%%%%%%%%%%%%%%%%%%%%%%%%%%%%%
\myitem{venerate} v.

\noindent \phonetic{(ˈvɛnəreɪt)}

\noindent [ad. L. venerāt-, ppl. stem of venerārī (also venerāre) to reverence, worship, adore; whence also It. venerare, Sp. and Pg. venerar, F. vénérer.]
\vspace{-0.3cm}

\begin{myenumerate}

\itembf{1.} trans. To regard with feelings of respect and reverence; to look upon as something exalted, hallowed, or sacred; to reverence or revere.

\P 1623 COCKERAM I \textit{Venerate}, to worship.    
\P 1656 BLOUNT  \textit{Glossogr.}, Venerate, to reverence, worship or honour.    
\P 1742 YOUNG  \textit{Nt. Th.} ii. 355 Who venerate themselves, the world despise.    
\P 1794 R. J. SULIVAN  \textit{View Nat.} I. 481 But there was a class of Alchymists, whose genius, probity, and conduct, we have reason to venerate.    
\P 1851 D. WILSON  \textit{Preh. Ann.} iv. iv. (1863) II. 293 The ruined chapels are still venerated.    
\P 1870 J. BRUCE  \textit{Life Gideon} iv. 70 [We] have learned to venerate the Word of God.

\itembf{2.} To pay honour to (something) by a distinct act of reverence.

\P 1844 LINGARD  \textit{Anglo-Sax. Ch.} (1858) I. v. 189 Thrice he venerated the sacred remains.

\noindent Hence \phonetic{ˈvenerated, ˈvenerating} ppl. adjs.

\P 1790 BURKE  \textit{Fr. Rev.} Wks. V. 84 You would have had..a reformed and *venerated clergy.    
\P 1818 COBBETT  \textit{Pol. Reg.} XXXIII. 169 In the Reports, the Resolutions, and in the venerated Acts, of your Honourable House.    
\P 1847 PRESCOTT  \textit{Peru} (1850) II. 143 It would be easier to govern under the venerated authority to which the homage of the Indians had been so long paid.    
\P 1873 BROWNING  \textit{Red Cotton Night-Cap Country} 272 Smiling and sighing had the same effect Upon the venerated image.

\P 1663 BOYLE  \textit{Usef. Exp. Nat. Philos.} I. iii. 55 The Queen of Sheba..then brake forth into pathetic and *venerating exclamations.    
\P 1828 MISS HIGGINSON in  Drummond \& Upton \textit{Life Martineau} (1902) I. iii. 50 [Her reply declines to accept from him a] venerating love.    
\P 1863 GEO. ELIOT  \textit{Romola} iii. xxxiv, He..saw the faces of men and women lifted towards him in venerating love.    
\P 1888 RUSKIN  \textit{Præterita} III. 8 Without..trouble to their venerating visitors in coming so far up hill.
\end{myenumerate}


%%%%%%%%%%%%%%%%%%%%%%%%%%%%%%%%%
\myitem{contemn} v.

\noindent \phonetic{(kənˈtɛm)}

\noindent [a. OF. contemner, contempner (cited 1453 in Godef.), ad. L. contem(p)n-ĕre, f. con- intensive + temnĕre to slight, scorn, disdain, despise: cf. Gr. τέµνειν to judge. Now chiefly a literary word.]
\vspace{-0.3cm}

\begin{myenumerate}

\itembf{1.} trans. To treat as of small value, treat or view with contempt; to despise, disdain, scorn, slight.

\P 1450-1530  \textit{Myrr. our Ladye} p. xlviii, They that do contempne me and forgette my charyte they do this to me.    
\P 1526  \textit{Pilgr. Perf.} (W. de W. 1531) 92 Who so contempneth you contempneth me.    
\P 1591 SHAKES.  \textit{Two Gent.} ii. iv. 129, I haue done pennance for contemning Loue.    
\P 1651 HOBBES  \textit{Leviath.} i. vi. 24 Those things which we neither Desire, nor Hate, we are said to Contemne.    
\P 1681 DRYDEN  \textit{Abs. \& Achit.} i. 381 Not that your Father's mildness I contemn.    
\P 1777 SHERIDAN  \textit{Trip Scarb.} ii. i, I did not start at his addresses as when they came from one whom I contemned.    
\P 1876 GEO. ELIOT  \textit{Dan. Der.} iv. xxxiii, It lay in Deronda's nature usually to contemn the feeble.

\itembf{b.} Const. with inf. To scorn or disdain to do.

\P 1609 BIBLE (Douay)  \textit{Deut.} xxi. 18 A stubbourne and froward sonne, that..contemneth to be obedient.    
\P 1622 WITHER  \textit{Mistr. Philar.} (1633) 738 Some..who do not contemne In his retyred walkes to visit him.

\itembf{2.} To treat (law, orders, etc.) with contemptuous disregard.

\P 1573 TUSSER  \textit{Husb.} (1878) 195 His benefites if we forget, or do contemne his lawe.    
\P 1579 SPENSER  \textit{Sheph. Cal.} Nov. 48 Let not my small demaund be so contempt.    
\P c1665 MRS. HUTCHINSON  \textit{Mem. Col. Hutchinson} (1846) 424 Mr. Cooper contemned my lords' order, and would not obey it.    
\P 1762 HUME  \textit{Hist. Eng.} (1806) III. xlvi. 667 This counsel is not to be contemned.    
\P 1818 JAS. MILL  \textit{Brit. India} II. v. ix. 689 They..contemned and violated the engagement of treaties.

absol. \P 1609 BIBLE (Douay)  \textit{Bel \& Dr.} i. 12 They contemned, because they had made under the table a secrete entrance [Vulg. contemnebant autem, quia, etc.].
\end{myenumerate}


%%%%%%%%%%%%%%%%%%%%%%%%%%%%%%%%%
\myitem{tinge} v.

\noindent \phonetic{(tɪndʒ)}

\noindent [ad. L. ting-ĕre to dye, colour.]
\vspace{-0.3cm}

\begin{myenumerate}

\itembf{1.} trans. To impart a trace or slight shade of some colour to; to tint; to modify the tint or colour of (const. with). Also absol.

\P 1477 RIPLEY  \textit{Comp. Alch.} xi. vi. in Ashm. Theat. Chem. Brit. (1652) 182 Saffron when yt ys pulveryzate, Tyngyth much more of Lycour.    
\P 1577 HARRISON  \textit{England} iii. viii. (1878) ii. 55 As their saffron is not so fine as that of Cambridge shire and about Walden, so it will not cake, ting, nor hold colour withall.    
\P 1577 HOLINSHED  \textit{Chron., Descr. Scot.} vii. 9/2 Theyr fleshe moreouer is redde as it were tynged with Saffron.    
\P 1658 A. FOX  \textit{Würtz' Surg.} iii. xvi. 265 Which will tinge the Aquavitæ to a redness.    
\P 1725 \textit{Bradley's  Fam. Dict.} s.v. Oak, A way of tinging Oak..so as it will resemble coarse Ebony.    
\P 1769 N. NICHOLLS  \textit{Corr. w. Gray} (1843) 99 Just when Autumn had begun to tinge the woods with a thousand beautiful varieties of colour.    
\P 1863 M. HOWITT  \textit{F. Bremer's Greece} II. xvi. 138 The summit of Parnassus was tinged with the red light of morning.

\itembf{b.} transf. To impart a slight taste or smell to; to affect slightly by admixture.

\P 1690 C. NESSE  \textit{O. \& N. Test.} I. 236 Fragrant flowers and fruits, the sweet odours whereof had likely ting'd those goodly garments.    
\P 1707 MORTIMER  \textit{Husb.} (1721) II. 353 Liquors tinged with the spirituous Flavour of other Fruits.    
\P c1826  \textit{Lond. Encycl. s.v. Barometer}, Common water, tinged with a sixth part of aqua regia.    
\P 1863 MRS. OLIPHANT  \textit{Salem Chapel} xiii, The sweet atmosphere was tinged with the perfumy breath which always surrounded her.

\itembf{2.} intr. To become modified in colour; to take a (specified or implied) tinge.

\P 1662 R. MATHEW  \textit{Unl. Alch.} §107. 174 Put on more Vinegar..till thou seest that it will ting no more.    
\P 1756 C. LUCAS  \textit{Ess. Waters} I. 15 The solution..upon the addition of new spirit of salt, tinges a kind of orange color.    
\P 1821 CLARE  \textit{Vill. Minstr.} I. 93 He [the oak] tinges slow with sickly hue.

\itembf{3.} fig. To affect in mind or feeling by intermixture, infusion, or association; to qualify, modify, or slightly vary the tone of.

\P 1674 N. FAIRFAX  \textit{Bulk \& Selv.} 47 Our souls are indeed so far ting'd with body.    
\P 1681 WOOD  \textit{Life} 14 Mar. (O.H.S.) II. 526 Fame tells us that he is tinged with presbyterian leven.    
\P 1702 C. MATHER  \textit{Magn. Chr.} iii. i. iii. (1852) 303 His exact education..tinged him with an aversation to vice.    
\P 1784 COWPER  \textit{Task} iv. 553 The town has ting'd the country.    
\P 1856 EMERSON  \textit{Eng. Traits, Lit.} Wks. (Bohn) II. 106 The influence of Plato tinges the British genius.    
\P 1884 JENNINGS  \textit{Croker Papers} I. vi. 182 This grief tinged the whole of Mr. Croker's subsequent life.

\itembf{4.} trans. Alch. To change by the action of a tincture: cf. TINCTURE v. 2 b, TINCT v. 3. Obs.

\P 1650 FRENCH  \textit{Distill.} (1651) Ded. A iv b, As men bring lead to Philosophers to be tinged into gold.    
\P 1660 tr.  \textit{Paracelsus' Archidoxis} i. v. 75 So likewise doth this Tincture tinge the Hydropical..Body into a sound State.

\itembf{5.} Trade. To mark with a tinge (TINGE n.3).

\P 1850 [see TINGE n. 3].

\noindent Hence \textbf{tinged} (\phonetic{tɪndʒd}) ppl. a.

\P 1658 A. FOX  \textit{Würtz' Surg.} iii. xvi. 265 This ting'd Aquavitæ is to be extracted per Balneum.    
\P 1774 M. MACKENZIE  \textit{Maritime Surv.} 110 With a smoked or tinged Glass before your Eye.    
\P 1867 DEUTSCH  \textit{Rem.} (1874) 23 To be dependent on the possibly tinged version of an interpreter.
\end{myenumerate}


%%%%%%%%%%%%%%%%%%%%%%%%%%%%%%%%%
\myitem{smack} n.1

\noindent \phonetic{(smæk)}

\noindent [OE. smæc, = OFris. smek, MDu. smac, MLG. smak (LG. smakk, schmakk; also Sw. smak, Da. smag), OHG. and MHG. smac, smach (G. dial. schmack; cf. G. geschmack). Slightly different in formation are OFris. smaka (WFris. smaek), MDu. smake (Kilian smaeck; Du. smaak), MLG. smake (LG. smâk, schmaak). See also SMATCH n.1]
\vspace{-0.3cm}

\begin{myenumerate}

\itembf{I. 1.} A taste or flavour; the distinctive or peculiar taste of something, or a special flavour distinguishable from this.

\P a1000 in  Wr.-Wülcker 225 Dulcis sapor, i. dulcis odor, swete smæc.    
\P c1050  \textit{Ibid.} 455 Nectar,..þone swetan smæc.    
\P c1200 ORMIN 1653 Forr  witt and skill iss wel inoh Þurrh salltess smacc bitacnedd.    Ibid. 14294 Swa summ þeȝȝ waterr wærenn, Off wikke smacc.    
\P 1340  \textit{Ayenb.} 112 Þet is kynges mete huerinne byeþ ech manyere lykinges and alle guode smackes.    
\P a1400 \textit{Stockh.  Medical MS.} ii. 608 in Anglia XVIII. 322 Of hennebane arn spycys iij..Alle wyll sauour an hidhows smak.    
\P c1475 HENRYSON  \textit{Poems} (S.T.S.) III. 152 It wilbe þe softar and sweittar of þe smak.    
\P a1536 \textit{Proverbs} in  \textit{Songs, Carols, etc.} (E.E.T.S.) 128 Thowgh peper be blak, it hath a good smak.    
\P 1578 LYTE  \textit{Dodoens} ii. lxxxv. 263 The leaues..are of a very strong and pleasant sauour, and good smacke or taste.    
\P 1606 J. CARPENTER  \textit{Solomon's Solace} xxviii. 118 Those vessels will long retaine and yeeld the smack of that liquor which was in them first steeped.    
\P 1675 EVELYN  \textit{Terra} (1729) 29 Every plant has a smack of the Root.    
\P 1710 T. FULLER  \textit{Pharm. Extemp.} 1 Midling Ale..that hath no burnt, musty, or otherwise ill smack.    
\P 1761 CHURCHILL \textit{Rosciad} Wks. 1763 I. 24 And  boniface, disgrac'd, betrays the smack..of Falstaff's sack.    
\P 1823 J. BADCOCK  \textit{Dom. Amusem.} 21 It possesses a dull, acidulous, offensive smack, and an empyreumatic smell.    
\P 1873 BROWNING  \textit{Red Cotton Night-Cap Country} 245 And now, for perfume, pour Distilment rare,..Till beverage obtained the fancied smack.

\itembf{b.} fig. or in fig. context.

\P 1340  \textit{Ayenb.} 177 Efterward me ssel lete þane smak of zenne.    
\P 1593 in  \textit{Lyly's Wks.} (1902) III. 451 Experience bids me..champe the bridle of a bitter smacke.    
\P 1690 DRYDEN  \textit{Amphitryon} i. i, He's constant to a handsome family; he knows when they have a good smack with them.    
\P 1850 THACKERAY  \textit{Pendennis} xli, There are works of all tastes and smacks.

\itembf{c.} Pleasant or agreeable taste or relish. Obs.

\P 1573 TUSSER  \textit{Husb.} (1878) 132 Least Doue and the cadow, there finding a smack, with ill stormie weather doo perish thy stack.    
\P 1600 TOURNEUR  \textit{Trans. Metam.} xxix. 202 If this sweet sinne still feedes him with her smacke.

\itembf{2.} Scent, odour, smell. Obs.

\P a1000 [see sense 1].    
\P c1250  \textit{Owl \& Night.} 823 Þenne is þes hundes smel fordo; he not þurh þe meynde smak hweþer he schal vorþ þe abak.    
\P 1549 E. ALLEN  \textit{Par. Rev.} 19 A cat of ye mountayne.., whiche with her smacke and savour, draweth many beastes unto her.

\itembf{3.} transf. A trace, tinge, or suggestion of something specified.

   Common c1570-1680, and in mod. use.

\P 1539 CROMWELL in  \textit{Merriman Life \& Lett.} (1902) II. 173 To powre in som smak of the pure lernying of Cristes doctrine amonges them.    
\P 1577 B. GOOGE  \textit{Heresbach's Husb.} iii. (1586) 138 b, Whatsoeuer commeth of an olde stocke, hath lightly a smack of his olde parentes imperfection.    
\P 1602 \textit{2nd  Pt. Return fr. Parnass.} ii. vi, Good faith, the boy begins to haue an elegant smack of my stile.    
\P 1639 FULLER  \textit{Holy War} iv. viii. (1840) 191 The others were suspected to have a smack of the imperial faction.    
\P 1688 HOLME  \textit{Armoury} iii. 233/1 The Orcadians..use the Gothish Language, which they derive from the Norwegians,..of whose qualities they still have a smack.

\P 1845 S. AUSTIN  \textit{Ranke's Hist. Reform.} II. 75 Graceful poems—not the less attractive for a slight smack of the workshop.    
\P 1874 BURNAND  \textit{My Time} xxix. 280 A smack of real earnestness in his tone.

\itembf{b.} A slight or superficial knowledge; a smattering. Chiefly in phr. \textbf{to have a smack of, at, or in} something. Obs.

(a) \P 1551 ROBINSON tr.  \textit{More's Utopia} (1895) 9 If it be one that hath a lytell smacke of learnynge.    
\P 1581 MULCASTER  \textit{Positions} xxxvii. (1887) 144 Bycause they haue some petie smake of their booke.    
\P c1618 MORYSON  \textit{Itin.} iv. 229 Hauing gott a smacke of the grownds of our lawe.    
\P 1685-90 J. COOD  \textit{Wonderful Provid.} (1849) 104 A very young man..who had got a smack of the Latin tongue.    
\P 1791 MRS. RADCLIFFE  \textit{Rom. Forest} (1820) I. 66, I learned a smack of boxing of that Englishman.

(b) \P 1579 LYLY  \textit{Euphues} (Arb.) 151 Whereby he may..haue in al sciences a smacke, whereby he may readily dispute of any thing.    
\P 1602 \textit{2nd  Pt. Return fr. Parnass.} iii. i, He hath also a smacke in poetry.    
\P 1679 M. MASON  \textit{Tickler Tickled} 2 For Padge hath a Smack at Latin, but let them English it that will.

\itembf{c.} A mere tasting, a small quantity, of liquor; a mouthful. Also fig.

\P 1693 DRYDEN  \textit{Persius} iv. 69 He 'says the wimble, often draws it back, And deals to thirsty servants but a smack.    
\P 1759 GARRICK  \textit{High Life below Stairs} ii, He has had a smack of every sort of wine.    
\P 1766 ANSTEY  \textit{New Bath Guide} (ed. 2) 135 May I venture to give Her a Smack of my Muse?    
\P 1824 W. IRVING  \textit{Tales Trav.} I. 18 A relish of the Marquis's well-known kitchen, and a smack of his superior Champagne and Burgundy.    
\P 1865 J. HATTON  \textit{Bitter Sweets} iii, We'll just have one smack of the liquor before you're off to Helswick.

\itembf{d.} A touch or suggestion of something having a characteristic odour or taste.

\P 1848 DICKENS  \textit{Dombey} vii, There was a smack of stabling in the air of Princess's Place.    
\P 1886 STEVENSON  \textit{Silverado Sq.} 34 A rough smack of resin was in the air.    
\P 1889 DOYLE  \textit{Micah Clarke} 320 A gentle breeze, sweet with the smack of the country.

\itembf{II. 4. a.}  The sense or faculty of taste. Obs.

   So OFris. smek, G. (ge)schmack, etc.

\P a1200 \textit{Vices \& Virtues} 17 Ȝesihthe, ȝeherhþe, smac, and smell, and tactþe.

\itembf{b.} fig. Delight or enjoyment; inclination, relish. Chiefly in phrases. Obs.

\P 1340  \textit{Ayenb.} 33 He..to-ualþ ine þa slacnesse þet he ne heþ smak, ne deuocion, wel to done.    
\P 1551 ROBINSON tr.  \textit{More's Utopia} ii. (1895) 254 So quyckelye they haue taken a smacke in couetesenes.    
\P 1580 LYLY  \textit{Euphues} (Arb.) 426 Philautus had taken such a smacke in the good entertainment.    
\P 1609 \textit{Ev. Woman in  Hum.} ii. i, I haue no appetite at all to live in the countrie.., now, as they say, I have got a smacke on the Cittie.    
\P 1620 SHELTON  \textit{Quix.} iii. xi. I. 231 She hath a very great Smack of Courtship, and plays with every one.
\end{myenumerate}


%%%%%%%%%%%%%%%%%%%%%%%%%%%%%%%%%
\myitem{leaven} n.

\noindent \phonetic{(ˈlɛv(ə)n)}

\noindent [a. F. levain (recorded from 12-13th c.) = Prov. levam:—L. levāmen means of raising (recorded only in the sense ‘alleviation, relief, comfort’), f. levāre (F. lever) to raise.]
\vspace{-0.3cm}

\begin{myenumerate}

\itembf{1.} A substance which is added to dough to produce fermentation; spec. a quantity of fermenting dough reserved from a previous batch to be used for this purpose (cf. sour-dough). In 16-18th c. often pl. Phrase, to lay, put leaven(s.

\P 1340  \textit{Ayenb.} 205 Ase þe leuayne zoureþ þet doȝ.    
\P 1390 GOWER  \textit{Conf.} I. 294 He is the levein of the brede, Which soureth all the past about.    
\P c1400 \textit{Lanfranc's  Cirurg.} 352 Take þe wombis of cantarides \& grinde him wiþ leueyne.    
\P c1425 \textit{Voc.} in  Wr.-Wülcker 663/21 Hoc leuamentum, lewan.    
\P 1471 RIPLEY  \textit{Comp. Alch.} ix. viii. in Ashm. (1652) 175 Lyke as flower of Whete made into Past, Requyreth Ferment whych Leven we call.    
\P a1483 \textit{Liber Niger} in  \textit{Househ. Ord.} (1790) 70 One yoman furnour..seasonyng the ovyn and at the making of the levayne at every bache.    
\P c1532 G. DU WES  \textit{Introd. Fr.} in Palsgr. 946 To put the levain, fermenter.    
\P 1533 ELYOT  \textit{Cast. Helthe} (1539) 27 b, Breadde of fyne floure of wheate, hauynge no leuyn, is slowe of digestion.    
\P 1541 R. COPLAND  \textit{Guydon's Quest. Chirurg.} N j, And yf ye veynes as yet appere nat wel, a day before he must haue a plaster of leueyne.    
\P 1573 TUSSER  \textit{Husb.} lxxxix. (1878) 179 Wash dishes, lay leauens.    
\P 1601 HOLLAND  \textit{Pliny} I. 566 The meale of Millet is singular good for Leuains.    
\P 1611 BIBLE  \textit{Exod.} xii. 15 Euen the first day yee shall put away leauen out of your houses.    
\P 1671 SALMON  \textit{Syn. Med.} iii. xxii. 430 Rie, the leaven is more powerfull than that of Wheat, in breaking all Aposthumes.    
\P 1699 EVELYN  \textit{Acetaria} 53 Add a Pound of Wheat-flour, fermented with a little Levain.    
\P 1747 H. GLASSE  \textit{Cookery} xvii. 151 The more Leaven is put to the Flour, the lighter and spongier the Bread will be.    
\P 1809 N. PINKNEY  \textit{Trav. France} 33 The bread is made of wheat meal, but in some cottages consisted of thin cakes without leven.    
\P 1876 tr.  \textit{Schützenberger's Ferment.} 10 The ancients used as leaven for their bread either dough that had been kept till it was sour, or beer-yeast.

\itembf{b.} In wider sense: Any substance that produces fermentation; = FERMENT n. 1; occasionally applied to the ‘ferment’ of zymotic diseases.

\P 1658 R. WHITE tr.  \textit{Digby's Powd. Symp.} (1660) 111 Oyl of tartar fermented by the levain of roses.    
\P 1689 HARVEY  \textit{Curing Dis. by Expect.} iv. 21 [The] humours..acquire a levain so pernicious, as to deprave and subvert the animal Faculty.    
\P 1747 tr.  \textit{Astruc's Fevers} 254 Moreover such a foreign levain is so disproportioned to our nature, that its effects will be the greater; nor must we admire, that this mortal ferment should be the product of some particular countries.    
\P 1758 J. S. \textit{Le Dran's Observ. Surg.} (1771) 137 Her Blood was loaded with a bad Leven.    
\P 1822-34 \textit{Good's  Study Med.} (ed. 4) I. 694 The activity of its [typhus'] leaven by which it assimilates all the fluids of the body to its own nature.

\itembf{2.} fig. \textbf{a.} Chiefly with allusion to certain passages of the gospels (e.g. Matt. xiii. 33, xvi. 6): An agency which produces profound change by progressive inward operation.

\P 1390 [see sense 1].    
\P 1555 PHILPOT  \textit{Apol.} (1599) B 8 b, What pharisaical leuen dothe they scatter abrode.    
\P 1641 MILTON  \textit{Reform.} ii. Wks. 1851 III. 49 The soure levin of humane Traditions mixt in one putrifi'd Masse with the poisonous dregs of hypocrisie in the hearts of Prelates.    
\P 1647 N. BACON  \textit{Disc. Govt. Eng.} i. iii. 7 And thus the Romans levened with the Gospell..insinuated that leven by degrees, which in the conclusion prevailed over all.    
\P 1725 BOLINGBROKE 24 JULY in  \textit{Swift's Lett.} (1767) II. 211 Lest so corrupt a member should come again into the house of lords, and his bad leaven should sour that sweet untainted mass.    
\P 1799 J. ADAMS  \textit{Wks.} (1854) IX. 8 There is a very sour leaven of malevolence in many English and in many American minds against each other.    
\P 1865 PARKMAN  \textit{Huguenots} ii. (1875) 17 To the utmost bounds of France, the leaven of the Reform was working.    
\P 1875 STUBBS  \textit{Const. Hist.} III. xxi. 542 The evil leaven of these feelings remained.

\itembf{b.} Used for: A tempering or modifying element; a tinge or admixture (of some quality).

\P 1576 FLEMING  \textit{Panopl. Epist.} 410 You have your fine walkes..and therewithall communication seasoned with the leven of learning.    
\P 1699 BENTLEY  \textit{Phal.} 406 Their Style had some Leaven from the Age that each of them liv'd in.    
\P 1740 J. CLARKE  \textit{Educ. Youth} (ed. 3) 124 The latter [Seneca]..has a Mixture of the Stoick Leaven.    
\P 1793 HOLCROFT  \textit{Lavater's Physiogn.} i. 13 Virtue unsullied by the leven of vanity.    
\P 1864 SWINBURNE  \textit{Atalanta} 318 Pleasure with pain for leaven.    
\P 1883 S. C. HALL  \textit{Retrospect} II. 185 A leaven of gaiety clung to her through life.    
\P 1884  \textit{Manch. Exam.} 23 June 6/1 We should remember their temptations and mix a large leaven of charity with our judgments.

\itembf{c.} Phrases. \textit{of the same leaven}: of the same sort or character. \textit{the old leaven}: after 1 Cor. v. 6, 7, the traces of the unregenerate condition; hence often applied to prejudices of education inconsistently retained by those who have changed their religious or political opinions.

\P 1598 B. JONSON  \textit{Ev. Man in Hum.} i. ii. 73 One is a Rimer, sir, o' your owne batch, your owne levin.    
\P 1650 TRAPP  \textit{Comm. Num.} 48 A loafe of the same leaven, was that resolute Rufus.    
\P 1653 MILTON \textit{Hirelings} Wks. 1738 I. 569  They quote Ambrose, Augustin, and some other ceremonial Doctors of the same Leven.    
\P 1722 SEWEL  \textit{Hist. Quakers} 4 The Prejudice of the old Leaven.    
\P 1727 SWIFT  \textit{To Very Yng. Lady} Wks. 1755 II. II.  42 Of the same leaven are those wives, who, when their husbands are gone a journey, must have a letter every post.    
\P 1839 STONEHOUSE  \textit{Axholme} 191 The old leaven of dissent, in which Wesley was brought up.

\itembf{3.} attrib.

\P 1547 BOORDE  \textit{Brev. Health} ccvii. 72 Rye breade, Levyn bread,..and all maner of crustes.    
\P 1880 KINGLAKE  \textit{Crimea} VI. vi. 134 The army of General Canrobert was often..able to provide itself with good leaven bread.
\end{myenumerate}



%%%%%%%%%%%%%%%%%%%%%%%%%%%%%%%%
\myitem{reprove} v.1

\noindent \phonetic{(rɪˈpruːv)}

\noindent [ad. OF. reprover (AF. also repruver; mod.F. réprouver):—L. reprobāre: see reprobate v. The β-forms are from those parts of the verb in which the stem had stress (AF. repreov-, OF. repreuv-): see prove v.]
\vspace{-0.3cm}

\begin{myenumerate}

\itembf{1.} trans. To reject. Obs.

\P a1340 HAMPOLE  \textit{Psalter} xx. 12 Amange þe deuels of hell, þe whilke þou has forsaken and reproued.    
\P 1382 WYCLIF  \textit{Luke} xx. 17 The stoon whom men bildinge reproueden [1388 repreueden],  this is maad in to the heed of the corner.    
\P c1450 MIROUR  \textit{Saluacioun} 3474 The stone whilk the biggers reproved in the heved is made angulere.    
\P 1526 TINDALE  \textit{Heb.} vi. 8 That grounde which beareth thornes and bryars is reproved and is nye vnto cursynge.    
\P 1582 BENTLEY  \textit{Mon. Matrones} 69 It seemeth to them God is parciall, bicause he hath elected some, and some reprooued.    
\P 1604 E. G[RIMSTONE]  \textit{D'Acosta's Hist. Indies} ii. xii. 109, I am almost ready to follow the opinion of such as reproove the qualities..which Aristotle gives vnto the Elements, saying they are but imaginations.

\itembf{b.} Sc. To set aside as invalid. Obs. rare.

\P 1480 \textit{Act.  Dom. Conc.} (1839) 52/1 Þat þe saidis provost, chanonis, \& chapelanis, sall brouke \& Joyse þe said landis..quhil þe said lettre be Repreifit \& declarit of na vale.

\itembf{2.} To express disapproval of (conduct, actions, beliefs, etc.); to censure, condemn. Now rare.

α \P 1340-70 \textit{Alex.  \& Dind.} 220 Þat non haþel..mihte alegge any lak our lif to reproue.    
\P 1432-50 tr.  \textit{Higden} (Rolls) III. 401 Thyne arte is to be reprovede that schewede not this to the before.    
\P 1483 CAXTON  \textit{Cato} F viij, Tho ben fooles that blamen and reprouen the tyme, sayeng that the tyme is cause of theyr sekenesse.    
\P 1579 GOSSON  \textit{Sch. Abuse} (Arb.) 54 If he come to our stall, and reprooue our ballance when they are faultie.    
\P 1615 J. STEPHENS  \textit{Satyr. Ess.} 20 Envy loves That humor best, which bitterly reproves All states.    
\P 1658 EVELYN  \textit{Fr. Gard.} (1675) 58, I do not utterly reprove the graffing of the wood, though but of one year.    
\P 1770 GOLDSM.  \textit{Des. Vill.} 169 He tried each art, reproved each dull delay.    
\P 1820 SHELLEY  \textit{Fiordispina} 40 Lulled by the voice they love, which did reprove The childish pity that she felt for them.

β \P c1380 WYCLIF  \textit{Wks.} (1880) 9 Ȝif þei haten..trewe men to techen frely holy writt and repreuen synne.    
\P c1450 tr.  \textit{De Imitatione} ii. ii. 42 Oþir men knowe oure defautes \& repreue hem.    
\P 1513 DOUGLAS  \textit{Æneis} i. Prol. 106 My werk or ȝe repreif Considdir it warlie, reid oftair than anis.    
\P 1567  \textit{Satir. Poems Reform.} vii. 82 Quhat preachour this repreif, I pray ȝow, durst?

\itembf{3.} To reprehend, rebuke, blame, chide, or find fault with (a person). Also const. for, of.

 α \P a1325 \textit{Prose  Psalter} xlix. 9 Y ne shal nouȝt repruue þe in þy sacrifices.    
\P 1340 HAMPOLE  \textit{Pr. Consc.} 5314 Alle þis sall he do þos openly To reprove þe synful men þar-by.    
\P c1400 MANDEVILLE (Roxb.) xv. 70 Me thoȝt grete schame þat Sarzenes..schuld þus reproue vs of oure inperfiteness.    
\P c1450 LOVELICH  \textit{Grail} xxxvi. 8 [For] On thyng that he dyde At Rome, Reproved he was be Clergies dome.    
\P 1568 GRAFTON  \textit{Chron.} II. 729 Reproouing and reuiling him with such yll wordes..that all the hearers abhorred it.    
\P 1667 MILTON  \textit{P.L.} x. 761 What if thy Son Prove disobedient, and reprov'd retort, Wherefore didst thou beget me?    
\P 1727 DE FOE  \textit{Syst. Magic} i. iv. (1840) 95 Others suggest, that Noah having reproved and reproached Canaan for some crime,..the Devil took hold of his resentment.    
\P 1855 TENNYSON  \textit{Maud} i. xx. i, Was it gentle to reprove her..?    
\P 1871 B. TAYLOR  \textit{Faust} (1875) II. i. iii. 27 You praise us—reprove us, It doesn't move us.

 β \P 1303 R.  BRUNNE \textit{Handl. Synne} 3722 Ȝyf þou for wraþþe madyst chydyng, Or repreuedyst a man of vyle þyng.    
\P 1377 LANGL.  \textit{P. Pl.} B. x. 261 God in þe gospel grymly repreueth Alle þat lakken any lyf.    
\P 1483 CAXTON  \textit{Cato} 4 Of Saynt Ambrose that repreuyed openly themperour of his synne.    
\P 1549  \textit{Compl. Scot.} xv. 123 Thou repreifis \& accusis me of the faltis that my tua brethir committis daly.    
\P 1596 SPENSER  \textit{F.Q.} v. vi. 24 Nor suffering the least twinckling sleepe to start Into her eye..; But if the least appear'd, her eyes she streight reprieved.

\itembf{b.} To accuse or convict. Obs. rare.

\P c1380 WYCLIF  \textit{Wks.} (1880) 30 Þer-for crist seiþ to þe iewis who of ȝou schal repreue me of synne.    
\P 1382   \textit{John} xvi. 8 He schal reproue the world of synne.    
\P c1440  \textit{York Myst.} xxxii. 241 Oure poynte expresse her reproues þe Of felonye falsely and felle.

\itembf{c.} To reproach, taunt. Const. of. Obs. rare—1.

\P c1330 R. BRUNNE  \textit{Chron. Wace} (Rolls) 1 1665 Þey  repreue vs of our auncessours Þat þey ouer-cam þem wyþ harde stours; Of pouerte þey make vmbreyd.

\itembf{4.} absol. To employ reprehension or rebuke.

\P a1340 HAMPOLE  \textit{Psalter} xiii. 6 Þaire mouth is ay redy to myssay and reproue.    
\P 1382 WYCLIF  \textit{Prov.} xxv. 10 Lest perauenture he asaile to thee, whan he shal heren, and to repreuen cese not.    
\P 1533 GAU  \textit{Richt Vay} 29 Al the writ quhilk is inspirit..is profetabil to tech, to reprw, to correk.    
\P 1611 BIBLE  \textit{2 Tim.} iv. 2 Reprooue, rebuke, exhort with all long suffering \& doctrine.    
\P 1766 FORDYCE  \textit{Serm. Yng. Wom.} (1767) I. i. 36 Reprove only when you must.    
\P 1821 SHELLEY  \textit{Epipsych.} 603 The troop which errs, and which reproves.    
\P 1876 M. E. BRADDON  \textit{J. Haggard's Dau.} I. 11 He came to the water-side tavern to reprove and exhort.

\itembf{5.} To disprove; to prove (an idea, statement, etc.) to be false or erroneous. Obs.

\P c1374 CHAUCER  \textit{Boeth.} v. met. iv. 130 (Camb. MS.), Whan it retorneth in to hym self it reproeueth and distroyet the false thinges by the trewe thinges.    
\P 1377 LANGL.  \textit{P. Pl.} B. x. 345 ‘Contra’, quod I, ‘bi cryste þat can I repreue’.    
\P c1430  \textit{Pilgr. Lyf Manhode} i. lxxxv. (1869) 49 For to assoile better þine argumentes þat seist j haue falsed and repreved þi gretteste principle.    
\P 1538 BALE  \textit{God's Promises} ii, All thys is true, Lorde, I cannot thy wordes reprove.    
\P 1593 SHAKES.  \textit{2 Hen. VI,} iii. i. 40 Reproue my allegation, if you can, Or else conclude my words effectuall.    
\P 1691 RAY  \textit{Creation} i. (1692) 25 This confident Assertion of DesCartes is fully examined and reproved by..Mr. Boyl.

\itembf{b.} To refute or confute (a person). Obs.

\P 1563 WINȝET  \textit{Four Scoir Thre Quest.} Wks. (S.T.S.) I. 101 Men in this vocatioun..suld..be..potent to repreue and conuict the gainsayaris of the samin.    
\P 1585 T. WASHINGTON tr.  \textit{Nicholay's Voy.} ii. ix. 42 b, Where he sayth the second to lye on the North part, he may by the view \& eisight onely be reproued, being in deed towards the East.    
\P 1601 HOLLAND  \textit{Pliny} xvi. xxxi, Deceived they are, and may be reproved by the instance of fig-trees.

\itembf{6.} To impair, diminish. Obs. rare.

\P 1450-80 tr.  \textit{Secreta Secret.} 9 Kepe euyr temperaunce in largete.., ne neuer repreue thi yeftis with ayentakyng.    
\P 1576 FLEMING  \textit{Panopl. Epist.} 403 Hee sheweth that his loue is so farre from being reproued, that it is augmented.    
\P 1590 GREENWOOD  \textit{Collect. Sclaund. Art.} G ij b, This is hit that..maketh all the syluer saints..to bestur them, least their portions should be reproued; They would gladly haue their portions improued.
\end{myenumerate}


%%%%%%%%%%%%%%%%%%%%%%%%%%%%%%%%
\myitem{reprimand} v.

\noindent \phonetic{(rɛprɪˈmɑːnd, -æ-)}

\noindent [ad. F. réprimander, reprimender (1642), f. réprimande: see prec.]
\vspace{-0.3cm}

\begin{myenumerate}

\itembf{1.} trans. To rebuke, reprove, or censure (a person) sharply or severely.

\P 1681 PRIDEAUX  \textit{Lett.} (Camden) 102 In the same manner he proceeded to repriman them for their unworthy behavior both to his Majesty and us.    
\P 1687 H. HOLDEN in  \textit{Magd. Coll. \& Jas. II} (O.H.S.) 124 The Bishop..in a large speech..reprimanded the Fellows of their disobedience.    
\P 1727 SWIFT  \textit{Poisoning E. Curll} Wks. 1755 III.  i. 149 This gentleman..reprimanded Mr. Curll for wrongfully ascribing to him the aforesaid poems.    
\P 1748  \textit{Anson's Voy.} i. iii. 30 The Boatswain immediately reprimanded them, and ordered them to be gone.    
\P 1770 JUNIUS  \textit{Lett.} xxxviii. (1788) 205 The lofty terms in which he was persuaded to reprimand the city.    
\P 1835 W. IRVING  \textit{Tour Prairies} 203 The Captain..reprimanded the sentinel for deserting his post, and obliged him to return to it.    
\P 1875 JOWETT  \textit{Plato} (ed. 2) I. 137 In such cases any man will be angry with another, and reprimand him.

absol. \P 1856 KANE  \textit{Arct. Expl.} I. xvi. 195 It was in vain that I..argued, jeered, or reprimanded: an immediate halt could not be avoided.

\itembf{b.} To censure, find fault with (an act). rare—1.

\P 1722 WATERLAND  \textit{Arian Subscript.} Suppl., Wks. 1823 II. 380  Lord Burghley..reprimanded the warm proceedings of the Heads against him.

\itembf{2.} To repress, restrain. Obs. rare—1.

\P 1710 T. FULLER  \textit{Pharm. Extemp.} 116 It [i.e. the electuary] reprimands the Animal Spirits when too furious.

\noindent Hence \phonetic{repriˈmander; repriˈmanding} vbl. n. and ppl. a.; \phonetic{repriˈmandingly} adv.

\P 1748 RICHARDSON  \textit{Clarissa} (1811) II. 315 Giving a hint, which perhaps..you will reprimandingly call, ‘Not being able to forego the ostentation of sagacity.’    
\P 1851 J. HAMILTON  \textit{Royal Preacher} xvii. (1854) 220 A long lecture of rough reprimanding and perverse faultfinding.    
\P 1867 \textit{Quiver}  II. 186 Then said the owl unto his reprimander—‘Fair sir, I have no enemies to slander.’    
\P 1899  \textit{Westm. Gaz.} 2 Aug. 10/3 The cleric found his Bishop in a reprimanding mood.
\end{myenumerate}


%%%%%%%%%%%%%%%%%%%%%%%%%%%%%%%%%
\myitem{reprobate} v.

\noindent \phonetic{(ˈrɛprəbeɪt)}

\noindent [f. L. reprobāt-, ppl. stem of reprobāre, f. re- RE- 2 d + probāre to PROVE: cf. REPROVE v.]
\vspace{-0.3cm}

\begin{myenumerate}

\itembf{1.} trans. To disapprove of, censure, condemn.

\P 1432-50 tr.  \textit{Higden} (Rolls) VI. 407 Sergius..beynge a cardinalle diacon, and reprobate by Formosus the pope, wente to Fraunce.    Ibid. VIII. 259 Gregory the xthe..approbate certeyne of the ordres of beggers..; somme he reprobate, as frers Saccines.    
\P 1607 J. CARPENTER  \textit{Plaine Mans Plough} 36 So those Scribes..were rejected..and their workes reprobated.    
\P 1671 [R. MACWARD]  \textit{True Nonconf.} 145 It was not only not introduced, but plainly reprobate by our Lord and his Apostles.    
\P 1752 LAW \textit{Spirit Love} ii. (1816) 129 For nothing is reprobated in Cain, but that very same which is reprobated in Abel.    
\P 1787 WINTER  \textit{Syst. Husb.} 205 His neighbours reprobated his method of proceeding.    
\P 1850 W. IRVING  \textit{Mahomet} vii. (1853) 36 He reprobated what he termed the heresies of his nephew.    
\P 1882 J. B. STALLO  \textit{Concepts Mod. Physics} 57 The ‘assumption’ of universal attraction is reprobated as an ‘absurdity’ by James Croll.

\itembf{b.} To abhor to do a thing. Obs. rare.

\P 1779 EARL MALMESBURY  \textit{Diaries \& Corr.} I. 236 His Prussian Majesty has..perhaps employed means we should reprobate to make use of.

\itembf{2.} Of God: To reject or cast off (a person or persons) from Himself; to exclude from participation in future bliss. (Cf. reprobation 3.)

\P 1526  \textit{Pilgr. Perf.} (W. de W. 1531) 24 b, For theyr synne they be reprobate \& forsaken of god.    
\P 1646 SIR T. BROWNE  \textit{Pseud. Ep.} 340 That the Thiefe on the right hand was saved, and the other on the left reprobated..we are ready to admit.    
\P a1711 KEN  \textit{Psyche Poet.} Wks. 1721 IV. 294  Paternal God, though it is just To reprobate infected Dust [etc.].    
\P 1751 G. LAVINGTON  \textit{Enthus. Meth. \& Papists} iii. (1754) 3 Persons of weak Spirits..will naturally..look upon themselves as reprobated, and forsaken of God.    
\P 1783 COWPER  \textit{Let. to Newton} 21 Apr., Such a man reprobated in the great day, would be the most melancholy spectacle.    
\P 1847 J. KIRK  \textit{Cloud Disp.} xi. 164 Proof that God has reprobated from eternity a certain part of mankind.

\itembf{3.} To reject, refuse, put away, set aside. (Sometimes with suggestion of sense 1.)

\P 1609 BIBLE (Douay)  \textit{Gen.} xxv. comm., The younger is elected, the elder reprobate.    
\P a1661 FULLER  \textit{Worthies} (1840) III. 130 Pole being reprobated, Julius the Third..was chosen in his place.    
\P 1773 JOHNSON  \textit{Let. to Mrs. Thrale} 20 Sept., I think the resolution both of my head and my heart engaged, and reprobate every thought of desisting from the undertaking.    
\P 1782 PRIESTLEY  \textit{Matt. \& Spir.} (ed. 2) I. Pref. 30 Mr. De Luc..will see this opinion..reprobated with contempt.    
\P 1850 NEALE  \textit{Med. Hymns} (1867) 116 Reprobated and rejected Was this Stone.

\itembf{b.} Law. To reject (an instrument or deed) as not binding on one. (Chiefly in Sc. Law, as opposed to approbate.) Also absol.

\P 1726 AYLIFFE  \textit{Parergon} 305 An Exception lies against the Tenor of an Instrument by other Proofs and Evidence in Writing: and this Method (among others) is the best way of reprobating an Instrument.    
\P a1768 ERSKINE  \textit{Inst. Law Scot.} iii. iii. §49 (1773) 465 The grantee does not in such case approbate and reprobate the same deed.    
\P 1836  \textit{Blackw. Mag.} XXXIX. 662 You cannot approbate and reprobate the same instrument.    
\P 1899 \textit{19th  Cent.} May 734 The clerical objector cleaves to the one set of laws and rejects the other. He seeks to approbate and reprobate.

\itembf{c.} To repudiate, cast off, disown. ? Obs.

\P 1748 RICHARDSON  \textit{Clarissa} (1811) I. xxv. 179, I beseech him not to reprobate his child for an aversion which it is not in her power to conquer.    
\P 1780 \textit{Newgate  Cal.} V. 154 The seduction was followed by very disagreeable consequences: the father reprobated his daughter.

\itembf{4.} intr. To employ reproaches. Obs. rare.

\P 1698 \textit{Christ  Exalted} 100 He reprobated exceedingly against Israel.

\noindent Hence \phonetic{ˈreprobated} ppl. a. Also absol.

\P 1535 JOYE  \textit{Apol. Tindale} (Arb.) 16 Where the state of the electe and of the reprobated immediately after their deth is described.    
\P 1647 WITHER  \textit{Carmen Expost.} B iij, God hath, for that offence, Expos'd you to a reprobated sense, Believing lies.    
\P 1668 CLARENDON  \textit{Contempl. Ps. Tracts} (1727) 571 It is not possible for the most reprobated sinner to believe [etc.].    
\P 1782 COWPER  \textit{Table-T.} 459 Callous and tough, The reprobated race grows judgment-proof.    
\P 1790 H. MORE  \textit{Relig. Fash. World} (1791) 197 This reprobated strictness therefore..is in reality the true cause of actual enjoyment.
\end{myenumerate}

%%%%%%%%%%%%%%%%%%%%%%%%%%%%%%%%
\myitem{contempt} n.

\noindent \phonetic{(kənˈtɛm(p)t)}

\noindent [ad. L. contempt-us (u stem) scorn, f. contempt- ppl. stem of contemnĕre to CONTEMN. Cf. OF. contemps ‘mépris’ (1346 in Godef.), contempt (Cotgr.), which was possibly the immediate source.]
\vspace{-0.3cm}

\begin{myenumerate}

\itembf{1.} The action of contemning or despising; the holding or treating as of little account, or as vile and worthless; the mental attitude in which a thing is so considered. (At first applied to the action, in modern use almost exclusively to the mental attitude or feeling.) Const. of, for; phrase in contempt of.

\P 1393 GOWER  \textit{Conf.} I. 217 He toke upon him alle thinge Of malice and of tirannie In contempte of regalie.    
\P a1400 \textit{Cov.  Myst.} 83 Contempt of veyn glory.    
\P 1526  \textit{Pilgr. Perf.} (1531) 16 b, Couetynge..the goodes of this worlde, to the contempte and despysynge of grace.    
\P 1581 MARBECK  \textit{Bk. of Notes} 249 Contempt consisteth chiefelie in three things: for either wee contemne onelie in minde..or lastlie when we adde words or deedes.    
\P 1605 SHAKES.  \textit{Lear} ii. iii. 8 The basest..shape That euer penury in contempt of man Brought neere to beast.    
\P 1611 BIBLE  \textit{Esther} i. 18 Thus shall there arise too much contempt [Coverdale despytefulnes] and wrath.    
\P 1614 BP. HALL  \textit{Medit. \& Vows} iii. §18. 72 Wee are soon cloyed..and have contempt bred in us through familiaritie.    
\P a1679 HOBBES  \textit{Rhet.} ii. ii. 46 Contempt, is when a man thinks another of little worth in comparison to himself.    
\P 1711 STEELE  \textit{Spect.} No. 148 \phonetic{⁋}1 New Evils arise every Day..in contempt of my Reproofs.    
\P 1732 BERKELEY  \textit{Alciphr.} i. §4 An outward contempt of what the public esteemeth sacred.    
\P 1845 M. PATTISON  \textit{Ess.} (1889) I. 21 This flimsy hypocrisy..inspired Gregory with a contempt which he could not dissemble.    
\P 1872 DARWIN  \textit{Emotions} xi. 254 Extreme contempt, or, as it is often called, loathing contempt, hardly differs from disgust.

\itembf{b.} (with a and pl.) Obs. except as in 4 b.

\P 1574 WHITGIFT  \textit{Def. Aunsw.} ii. Wks. 1851 I. 284,  I beseech God forgive you your outrageous contempts.    
\P c1665 MRS. HUTCHINSON  \textit{Mem. Col. Hutchinson} (1846) 34 All the contempts they could cast at him were their shame not his.    
\P 1733 WESLEY  \textit{Wks.} (1872) VII. 486 Our sins are so many contempts of this highest expression of his love.

\itembf{2.} The condition of being contemned or despised; dishonour, disgrace; esp. in to have, hold in contempt, bring, fall into, contempt.

\P c1450 CASTLE  \textit{Hd. Life St. Cuthb.} (Surtees) 3711 My teching eftir my dissese Sall noȝt be had in contempt.    
\P 1550 BALE  \textit{Sel. Wks.} (1849) 259 Having his verity in much more contempt than afore.    
\P 1560 BIBLE  (Genev.) \textit{Isa.} xxiii. 9 To bring to contempt [1611 into  contempt] all them that be glorious in the earth.    
\P 1594 SHAKES.  \textit{Rich. III,} i. iii. 80 My selfe disgrac'd, and the Nobilitie Held in contempt.    
\P c1645 HOWELL  \textit{Lett.} (1650) I. 473 She may be said to have..fallen to such a contempt that she dares scarce show her face.

\P 1837 W. IRVING  \textit{Capt. Bonneville} I. 219, I and my people will share the contempt you are bringing upon yourselves.    
\P 1875 JOWETT  \textit{Plato} (ed. 2) III. 189 He would like to bring military glory into contempt.

\itembf{3.} = Object of contempt. Obs. (Cf. similar use of joy, delight, aversion, etc.).

\P 1611 BIBLE  \textit{Gen.} xxxviii. 23 And Iudah said, Let her take it to her, lest we bee shamed [marg. become a contempt].    
\P 1746 W. HORSLEY  \textit{Fool} (1748) I. 101 The Companion of every Scoundrel, and the Contempt of every reasonable Creature breathing.    
\P c1832 BEDDOES  \textit{Poems}, Murderer's Haunted Couch, Thou shalt not dare to break All men's contempt, thy life, for fear of worse.

\itembf{4.} Law. Disobedience or open disrespect to the authority or lawful commands of the sovereign, the privileges of the Houses of Parliament or other legislative body; and, esp. action of any kind that interferes with the proper administration of justice by the various courts of law; in this connexion called more fully contempt of court. [OF. contemnement de justice.]

   \textit{Contempt of court} includes any disobedience to the rules, orders, or process of a court, whether committed by an inferior court, by the servants of the court or officers of the law, or by strangers, and any disrespect or indignity offered to the judges in their judicial capacity within or without the court.

\P [1552 HULOET,  \textit{Contempte} ..properlye agaynste the lawe.]    
\P 1621 H. ELSING  \textit{Debates Ho. Lords} (Camden) 78 Yf he had spoaken anything which doth touch the Kinge in his honour..Arundell. Difference betwene contempt and treason.    
\P 1625 in  \textit{Rymer Fœdera} XVIII. 144/1 Such further Paynes, Penalties, and Imprisonments, as..can or may be inflicted upon them for their Contempt and Breach of Our royall Commandment in this Behalfe.    
\P 1837 DICKENS  \textit{Pickw.} xxv, ‘Mr. Jinks,’ said the magistrate, ‘I shall commit that man for contempt.’    
\P 1866 CRUMP  \textit{Banking} iii. 82 An order restraining bankers from parting with money..must be obeyed at the risk of being committed for contempt of court.

\itembf{b.} (with a and pl.) An act of such disregard or disobedience.

\P 1621 H. ELSING  \textit{Debates Ho. Lords} (Camden) 78 The question whether Yelverton be not fytt to be censured of a greate contempt.    
\P a1626 BACON  \textit{Max. \& Uses Com. Law} (1636) 5 Contempts against the crowne, public annoyances against the people.    
\P 1722 SEWEL  \textit{Hist. Quakers} (1795) I. iv. 352 Imprisoned upon contempts (as the not putting off hats before the magistrates was called).    
\P 1768 BLACKSTONE  \textit{Comm.} III. 287 Not having obeyed the original summons, he had shewn a contempt of the court.    
\P 1862 BROUGHAM  \textit{Brit. Const.} xvii. 256 Both Houses claim to visit with severe punishment what are called contempts or breaches of their privileges.

\itembf{c.} \textit{in contempt}: in the position of having committed contempt, and not having purged himself.

\P 1768 BLACKSTONE  \textit{Comm.} III. 443 If the defendant, on service of the subpoena, does not appear..he is then said to be in contempt.    
\P 1766 ENTICK  \textit{London} IV. 265 It is a general court for debtors, and such as are in contempt of the Courts of Chancery and Common-pleas.    
\P 1845 STEPHEN  \textit{Laws Eng.} II. 177 note, On continuing to make default after having been ordered by the court to pay..he will be in contempt.
\end{myenumerate}


%%%%%%%%%%%%%%%%%%%%%%%%%%%%%%%%
\myitem{vitiate} v.

\noindent \phonetic{(ˈvɪʃɪeɪt)}

\noindent [f. L. vitiāt- (med.L. also viciāt-), ppl. stem of vitiāre (whence It. viziare, Sp. and Pg. viciar, F. vicier), f. vitium VICE n.1 Cf. prec.]
\vspace{-0.3cm}

\begin{myenumerate}

\itembf{1.} trans. To render incomplete, imperfect, or faulty; to impair or spoil.

\P 1534 MORE  \textit{Treat. Passion} Wks. 1303/1 Hym must we serue, though specially wyth the mynde (whych if it be not good, viciateth all together) yet..also wyth body and goodes and al.    
\P a1631 DONNE  \textit{Serm., Matt.} v. 16 (1640) 82 A superstitious end, or a seditious end vitiates the best worke.    
\P 1665 MANLEY  \textit{Grotius' Low C. Wars} 453 Other Advices were prefer'd, which..do many times vitiate, if not ruine, the most noble and valiant Undertakings.    
\P 1678 BARCLAY  \textit{Apol. Quakers} vii. §2. 197 This Doctrine of Justification hath been, and is greatly vitiated in the Church of Rome.    
\P 1711 ADDISON  \textit{Spect.} No. 25 \phonetic{⁋}5 A continual Anxiety for Life vitiates all the Relishes of it, and casts a Gloom over the whole Face of Nature.    
\P 1738 WARBURTON  \textit{Div. Legat.} I. 166 Time, which naturally and fatally viciates and depraves all things.    
\P 1794 HUTTON  \textit{Philos. Light}, etc. 124 It would only lead us into error, and thus vitiate the science or philosophy in which it were employed.    
\P 1808 J. HASLAM  \textit{Observ. Madness \& Mel.} i. (1809) 31 It might be urged, that in these instances, the perception was vitiated.    
\P 1851 NICHOL  \textit{Archit. Heav.} (ed. 9) 60 Considering that a deviation from truth by the fraction of a hairbreadth, would vitiate the figure.

\itembf{b.} To corrupt (a) literary works or (b) language by carelessness, arbitrary changes, or the introduction of foreign elements.

(a) \P 1659 BP. WALTON  \textit{Consid. Considered} 198 The Septuagint..which we now have is the same for substance with that anciently used, though..by the injury of time, and frequent transcriptions vitiated.    
\P 1788 REID  \textit{Aristotle's Logic} i. §i. 5 There is reason to doubt whether what [works] are his be not much vitiated and interpolated.

(b) \P 1690  \textit{Temple Ess.}, Poetry Wks. 1720 I. 243  Where$\sim$ever the Roman Colonies had remained, and their Language had been generally spoken, the common People used that still, but vitiated with the base Allay of their Provincial Speech.    
\P 1742 DE FOE'S  \textit{Tour Gt. Brit.} (ed. 3) III. 4 It is observable, that the Normans could not well pronounce Lincoln, but vitiated it to Nichol.    
\P 1756 JOHNSON  \textit{Dict. Pref.}, Many barbarous terms and phrases, by which other dictionaries may vitiate the style, are rejected from this.    
\P 1790 ‘CASSANDRA’  (J. Bruckner) \textit{Crit. Tooke's Purley} 55 Those who consider how much the language had been vitiated at the time they lived, by the importation of foreign words.

\itembf{2.} To render corrupt in morals; to deprave in respect of principles or conduct; to lower the moral standard of (persons).

\P 1534 MORE  \textit{Treat. Passion} Wks. 1311/2 We shulde note well and marke thereby, that the vice of a vicious personne, viciateth not the company or congregacion.    
\P 1658-9 in  \textit{Burton's Diary} (1828) IV. 59 This will not vitiate persons, but your nature and your posterity.    
\P 1682 BURNET  \textit{Rights Princes} Pref. 13 Mankind is not so vitiated with prejudice.    
\P 1751 JOHNSON  \textit{Rambler} No. 177 \phonetic{⁋}12 The suppression of those habits with which I was vitiated.    
\P 1770 JUNIUS  \textit{Lett.} xxxvii. (1788) 199 If any part of the representative body be not chosen by the people, that part vitiates and corrupts the whole.    
\P 1853 C. L. BRACE  \textit{Home Life Germany} 258 In 1806, the army had become thoroughly vitiated by luxury.    
\P 1880 E. KIRKE  \textit{Garfield} 55 In short, he had only one fault, but that was radical, and in the end, vitiated the whole man. He was thoroughly selfish.

\itembf{b.} Similarly with impersonal objects.

\P 1584 R. SCOT  \textit{Discov. Witchcr.} v. v. (1886) 80 He being a spirit, may with Gods leave and ordinance viciat and corrupt the spirit and will of man.    
\P 1598 MARSTON  \textit{Pygmal.}, Sat. ii, Many spots my mind doth vitiate.    
\P 1634 HABINGTON  \textit{Castara} Pref. (Arb.) 12, I encounter'd there..Innocencie,..not vitiated by conversation with the world.    
\P 1675 TRAHERNE  \textit{Chr. Ethics} 324 So doth one vice cherished and allowed corrupt and viciate all the vertues in the whole world.    
\P 1714 R. FIDDES  \textit{Pract. Disc.} ii. 93 Sufferings vitiate the best tempers.    
\P 1751 JOHNSON  \textit{Rambler} No. 172 \phonetic{⁋}2 Many vitiate their principles in the acquisition of riches.    
\P 1837 H. MARTINEAU  \textit{Soc. Amer.} III. 263 The encouragement of an amusement which does seem to be vitiated there.    
\P 1847 HAMILTON  \textit{Rewards \& Punishm.} viii. (1853) 362 One sin of youth vitiates a protracted life.    
\P 1861 MILL  \textit{Utilit.} i. 4 To what extent the moral beliefs of mankind have been vitiated..by the absence of any distinct recognition of an ultimate standard.

\itembf{c.} To pervert (the eye, taste, etc.), so as to lead to false judgements or preferences.

\P 1806 A. HUNTER  \textit{Culina} (ed. 3) 120 Stomachs may be so far vitiated as to lose all relish for plain roast, or boiled meat.    
\P 1821 CRAIG  \textit{Lect. Drawing}, etc. ii. 103 This practice has such a tendency to vitiate the eye and to mislead the mind.    
\P 1845 MCCULLOCH  \textit{Taxation} i. vi. (1852) 245 It had the mischievous effect of vitiating the public taste and stimulating the consumption of ardent spirits.

\itembf{3.} To deflower or violate (a woman). Obs.

\P 1547-50 [see \textit{Vitiating} vbl. n.].    
\P 1624 HEYWOOD  \textit{Gunaik.} i. 35 Till she returned into her owne naturall forme, in which he vitiated her, and of her begat Achilles.    
\P c1645 HOWELL  \textit{Lett.} (1650) I. 49 This beutious Maid [Venice] hath bin often attempted to be vitiated.    
\P 1675 BAXTER  \textit{Cath. Theol.} i. 107 Being not..moved by him (as David to murder Urias, and to vitiate his wife).    
\P 1710 STEELE  \textit{Tatler} No. 198 \phonetic{⁋}8 He confessed his Marriage, and his placing his Companion on Purpose to vitiate his Wife.    
\P 1769 BLACKSTONE  \textit{Comm.} IV. 81 It was a felony and attended with a forfeiture of the fief, if the vasal vitiated the wife or daughter of his lord.    
\P 1791 BURKE  \textit{Let. Member Nat. Assembly} Wks. VI. 36 Pedagogues, who betray the most awful family trusts, and vitiate their female pupils.

\itembf{4.} To corrupt or spoil in respect of substance; to make bad, impure, or defective.

\P 1572 J. JONES  \textit{Bathes Buckstone} 15 For blood is the treasure of lyfe, not viciated.    
\P 1599 SANDYS  \textit{Europæ Spec.} (1632) 103 As a dead Flie doth vitiate a whole boxe of sweet oyntment.    
\P 1608 TOPSELL  \textit{Serpents} 125 Euen as women in their monthly courses doe vitiat their looking-glasses.    
\P 1652 L. S. \textit{People's  Liberty} iii. 6 As much water cannot so soon be viciated as a lesser quantity.    
\P 1674 R. GODFREY  \textit{Inj. \& Ab. Physic} 33 The very texture of his Stomach and other vital bowels was vitiated.    
\P 1759 MILLS tr.  \textit{Duhamel's Husb.} i. xvi. 93 Farmers distinguish the wheat thus vitiated by saying that it is blacked in the point.    
\P 1789 W. BUCHAN  \textit{Dom. Med.} (1790) 465 When the saliva is vitiated,..the curing of the disorder is the cure of this symptom.    
\P 1863 GEO. ELIOT  \textit{Romola} xxxiv, The oncoming of a malady that has permanently vitiated the sight.    
\P 1882  \textit{Med. Temp. Jrnl.} No. 52. 177 As I shall endeavour to show you, it vitiates the blood.

\itembf{b.} esp. To render (air) impure and so inadequate for, or injurious to, life.

\P 1715 DESAGULIERS  \textit{Fires Impr.} 34 The ill Humours which go out of their Bodies..vitiate the Air more and more.    
\P 1793 BEDDOES  \textit{Consump.} 137 Only a very small portion of the air was vitiated, i.e. converted into fixed air.    
\P 1869 E. A. PARKES  \textit{Pract. Hygiene} (ed. 3) 118 The impurity of the air vitiated by respiration.    
\P 1878 HUXLEY  \textit{Physiogr.} 84 This gas would unduly accumulate, and..vitiate the entire bulk of the atmosphere.

\itembf{5.} To render of no effect; to invalidate either completely or in part; spec. to destroy or impair the legal effect or force of (a deed, etc.).

\P 1621 SANDERSON  \textit{Serm.} I. 170 An earthly judge is subject to misprision, mis-information, partiality, corruption, and sundry infirmities that may vitiate his proceedings.    
\P 1726 AYLIFFE  \textit{Parergon} 104 A Transposition of the Order of the Sacramental Words, does, in some Mens Opinion, vitiate Baptism.    
\P 1790 BURKE  \textit{Fr. Rev.} 37 If all the absurd theories of lawyers and divines were to vitiate the objects in which they are conversant, we should have no law, and no religion left in the world.    
\P 1827 JARMAN  \textit{Powell's Devises} II. 21 If an undefined portion of a bequest is to be applied to a purpose void by the statute, it vitiates the whole.    
\P 1853 LYTTON  \textit{My Novel} xii. xxvii, I told them flatly..that, as Mr. Egerton's agent, I would allow no proceedings that might vitiate the election.    
\P 1883  \textit{Law Rep.} 11 Q.B. Div. 568 The plaintiff is engaged in carrying out the illegal objects of the association;..and this circumstance alone vitiates the contract for repayment.

\itembf{b.} To render (an argument, etc.) inconclusive or unsatisfactory.

\P 1748 HARTLEY  \textit{Observ. Man} i. iii. §i. 308 This will not vitiate the foregoing Conjectures.    
\P 1846 MILL  \textit{Logic} i. v. §3 The theory of that intellectual process has been vitiated by the influence of these erroneous notions.    
\P 1866 HERSCHEL  \textit{Fam. Lect. Sci.} (1867) 73 His proof is vitiated by an enormous oversight: and the thing..is a physical impossibility.    
\P 1878 STEWART \& TAIT  \textit{Unseen Univ.} ii. §84. 94 It is this eternity of atom which vitiates the hypothesis.

\itembf{6. a.} To adulterate. Obs.

\P 1728 SHERIDAN tr.  \textit{Persius} ii. (1739) 35 It was Luxury first made us vitiate our Oyl with Cassia.

\itembf{b.} To alter feloniously. Obs.

\P 1753  \textit{Scots Mag.} Aug. 420/1 And William Taylor, for vitiating a bank-note.

\noindent Hence \textbf{\phonetic{ˈvitiating}} vbl. n. and ppl. a.

\P 1547 HOOPER  \textit{Declar. Christ \& Office} xii. L viij, The deathe of his chyldre, the conspyricie of Absolon, the uiciating of his wiues.    
\P a1550 LELAND  \textit{Itin.} (1769) V. 21 The Collegiate Chirch..was translatid to Aberguili for vitiating of a Maide.    
\P 1647 CLARENDON  \textit{Contempl. Ps. Tracts} (1727) 392 The yielding to every corrupt affection and passion is as great a vitiating and weakening of the mind.    
\P 1669 BOYLE  \textit{Certain Physiol. Ess.} (ed. 2) Absol. Rest Bodies 27 Finding its passage obstructed..by the vitiating of the Pores of the Glass.    
\P 1832 J. S. MILL in  \textit{Monthly Repos.} VI. 658 After all which has been done to break down these vitiating, soul-debasing prejudices,..where are we now?    
\P 1858 J. MARTINEAU  \textit{Stud. Chr.} 275 A certain vitiating unsoundness of mind.    
\P 1859 GEO. ELIOT  \textit{A. Bede} xxix, No man can escape this vitiating effect of an offence against his own sentiment of right.
\end{myenumerate}


%%%%%%%%%%%%%%%%%%%%%%%%%%%%%%%%%
\myitem{vituperate} v.

\noindent \phonetic{(vaɪˈtjuːpəreɪt, vɪ-)}

\noindent [f. L. vituperāt-, ppl. stem of vituperāre to censure, blame, disparage, find fault with, etc., f. vitu- for viti-, stem of vitium blemish, fault, VICE n.1 + parāre to prepare. See also VITUPER v.]
%\vspace{-0.3cm}

\noindent trans. To blame, speak ill of, find fault with, in strong or violent language; to assail with abuse; to rate or revile.

   Not in common use until the beginning of the 19th c.

\P 1542 BOORDE  \textit{Dyetary} xvi. (1870) 273 They louyth not porke nor swynes flesshe, but doth vituperat \& abhorre it.    
\P 1611 COTGR.,  \textit{Vituperer}, to vituperate, dispraise, discommend. [Hence in Cockeram, Blount, Bailey, etc.]    
\P 1638 PENKETHMAN  \textit{Artach.} C ij, Whatsoever transcends their sedulous apprehension..without any favourable expostulation..they will unworthily and unwittingly vituperate and reprehend.

\P 1819 SCOTT  \textit{Ivanhoe} xxxiii, The incensed priests..continued to raise their voices, vituperating each other in bad Latin.    
\P 1826 LAMB  \textit{Elia} Ser. ii. Pop. Fallacies iv, A speech from the poorest sort of people which always indicates that the party vituperated is a gentleman.    
\P 1860 FROUDE  \textit{Hist. Eng.} V. 477 He vituperated from the pulpit the vices of the court.    
\P 1883 A. FORBES in  \textit{Fortn. Rev.} 1 Nov. 671 Englishmen are not in the habit of vituperating Monk as a traitor.

refl. \P 1812 H. \& J. SMITH  \textit{Rej. Addr.} x. (1873) 96 Deviation from scenic propriety has only to vituperate itself for the consequences it generates.

\itembf{b.} absol. or intr. To employ abusive language.

\P 1856 R. A. VAUGHAN  \textit{Mystics} viii. v. 46 Vituperated and vituperating, he became a wanderer throughout Germany.    
\P 1877 MRS. OLIPHANT  \textit{Makers Flor.} vi. 168 He loses his temper and begins to vituperate.

\noindent Hence \textbf{\phonetic{viˈtuperated}} ppl. a.

\P 1841 EMERSON  \textit{Conservative Wks.} (Bohn) II. 272 You are yourself the result of this manner of living, this foul compromise, this vituperated Sodom.



%%%%%%%%%%%%%%%%%%%%%%%%%%%%%%%%%
\myitem{invective} a. and n.

\noindent \phonetic{(ɪnˈvɛktɪv)}

\noindent [a. F. invectif, -ive adj., invective n. (14-15th c. in Hatz.-Darm.), ad, late L. invectīvus ‘reproachful, abusive’, in med.L. invectīva (sc. ōrātio) as n., f. ppl. stem of invehĕre: see invect and -ive.]
\vspace{-0.3cm}

\begin{myenumerate}

\itembf{A.} adj.

\itembf{1.} Using or characterized by denunciatory or railing language; inclined to inveigh; expressing bitter denunciation; vituperative, abusive. Now rare.

\P 1430-40 LYDG.  \textit{Bochas} vi. xv. (MS. Bodl. 263) 336/2 He..Compiled hadde an Invectiff scripture Ageyn Antoyne.    
\P 1576 A. HALL  \textit{Acc. Quarrell} (1815) 35 Divers invective speeches..had passed in the same.    
\P 1591 GREENE  \textit{Disc. Coosnage} (1859) 58 What is the matter good wife (quoth I) that you use such invective words against the collier?    
\P a1661 FULLER  \textit{Worthies, Cambr.} i. (1662) 153 He was..always devoted to Queen Mary, but never invective against Queen Elizabeth.    
\P 1716 WODROW  \textit{Corr.} (1843) II. 120 They kept a fast to pray for success to the Pretender's arms, and a thanksgiving for his arrival..and were very invective and bitter.    
\P 1741 MIDDLETON  \textit{Cicero} I. vi. 471 Cicero..made a reply to him on the spot in an Invective speech, the severest perhaps, that was ever spoken by any man.    
\P 1866 Athenæum  No. 2001. 299/3 What we may call invective history.    
\P 1890 E. JOHNSON  \textit{Rise Christendom} 368 William, the invective opponent of the..friars.

\itembf{2.} Carried or borne in (against something). Obs.

\P 1603 FLORIO  \textit{Montaigne} ii. xii. (1632) 244 As hugh rocks doe regorge th' invective waves.

\itembf{B.} n.

\itembf{1.} A violent attack in words; a denunciatory or railing speech, writing, or expression.

\P 1523 SKELTON  \textit{Garl. Laurel} 96 Iuuenall was thret parde for to kyll For certayne enuectyfs, yet wrote he none ill.    
\P 1546 \textit{Supplic.  Poore Commons} (E.E.T.S.) 84 Theyr sermons were lytle other then inuectiues agaynst vsery.    
\P 1640 BP. HALL  \textit{Episc.} ii. xvii. 183 This it is that fills..Pamphlets with spightfull invectives.    
\P 1781 GIBBON  \textit{Decl. \& F.} xxvii. (1869) II. 82 Their satirical wit degenerated into sharp and angry invectives.    
\P 1839 JAMES  \textit{Louis XIV}, IV. 342 The duke, in going down stairs, poured forth volleys of invectives upon the Chief President.    
\P 1844 THIRLWALL  \textit{Greece} lxii. VIII. 177 Cleomenes..sent a letter to the assembly, containing bitter invectives against Aratus.

\itembf{2.} (Without pl.) Denunciatory or opprobrious language; vehement denunciation; vituperation.

\P 1602 W. FULBECKE  \textit{2nd Pt. Parall.} 26 Yet the Græcians did not alwaies suffer this licentious rage and inuectiue of Poets.    
\P a1770 JORTIN  \textit{Serm.} (1771) V. xix. 401 The book of Proverbs is full of invective and indignation against..those profligates.    
\P 1839 KEIGHTLEY  \textit{Hist. Eng.} II. 27 He burst out into a torrent of invective.
\end{myenumerate}


%%%%%%%%%%%%%%%%%%%%%%%%%%%%%%%%%
\myitem{obloquy} n.

\noindent \phonetic{(ˈɒbləkwɪ)}

\noindent [ad. late L. obloqui-um contradiction, f. obloquī to speak against, gainsay, contradict, f. ob- (ob- 1 b) + loquī to speak. (The early spelling obliq- may have arisen through confusion with oblique.)]
\vspace{-0.3cm}

\begin{myenumerate}

\itembf{1.} Evil-speaking directed against a person or thing; abuse, detraction, calumny, slander. Formerly also with an and pl., An abusive or calumnious speech or utterance (obs.).

\P 1460 J. CAPGRAVE  \textit{Chron.} 281 In this tyme cam oute a bulle..whech revokid alle the graces that had be graunted..of whech ros mech slaundir and obliqui ageyn the Cherch.    
\P 1502 W. ATKYNSON tr.  \textit{De Imitatione} iii. xl. 229 Infyrmytes, \& iniurye, oblyquies \& repreues..these thynges helpe to purches vertues.    
\P 1591 SHAKES.  \textit{1 Hen. VI,} ii. v. 49 He..did vpbrayd me with my Fathers death; Which obloquie set barres before my tongue.    
\P 1673  \textit{True Worsh. God} p. ii, I shall not much concern my self with the obloquies of such men.    
\P 1777 WATSON  \textit{Philip II} (1839) 375 It would be prudent perhaps not to expose himself again to the obloquy of his detractors.    
\P 1867 SMILES  \textit{Huguenots Eng.} viii. (1880) 137 They had to..hold their convictions in the face of obloquy, opposition.

\itembf{b.} Abuse or detraction as it affects the person spoken against; the condition of being spoken against; evil fame, bad repute; reproach, disgrace.

\P 1469  \textit{Paston Lett.} II. 380 They that be abut yow be in obloquy of all men.    
\P 1494 FABYAN \textit{Chron.} vii. 618 All was ruled by the quene \& her counsayll..to the great maugre \& oblyquy of the quene.    
\P 1513 MORE in  \textit{Grafton Chron.} (1568) II. 767 From the great obloquy that he was in so late before, he was..in so great trust that..he was made [etc.]    
\P 1602 MARSTON  \textit{Antonio's Rev.} iv. iii, The just revenge Upon the author of thy obloquies.    
\P 1647 CLARENDON  \textit{Hist. Reb.} vii. §337 And undergo the perpetual obloquy of having lost a Kingdom.

\itembf{2.} transf. A cause, occasion, or object of detraction or reproach; a reproach, a disgrace. Obs.

\P 1589 NASHE  \textit{Anat. Absurd.} 39 To shew what an obloquie these impudent incipients in Arts are vnto Art.    
\P 1601 SHAKES.  \textit{All's Well} iv. ii. 44 An honour longing to our house,..Which were the greatest obloquie i' the world, In me to loose.    
\P 1621 BURTON  \textit{Anat. Mel.} ii. iii. vii. (1651) 356, I have been..arraigned and condemned, I am a common obloquy.
\end{myenumerate}


%%%%%%%%%%%%%%%%%%%%%%%%%%%%%%%%%
\myitem{scurrility} n.

\noindent \phonetic{(skəˈrɪlɪtɪ)}

\noindent [a. F. scurrilité (15th c.), or ad. L. scurrīlitās, f. scurrīlis: see SCURRILE a. and -ITY.]
\vspace{-0.3cm}

\begin{myenumerate}

\itembf{a.} The quality of being scurrilous; buffoon-like jocularity; coarseness or indecency of language, esp. in invective and jesting.

\P 1508 DUNBAR  \textit{Flyting} 58 Scarth fra scorpione, scaldit in scurrilitie.    
\P 1526  \textit{Pilgr. Perf.} (W. de W. 1531) 90 b, Scurrilite or spekynge of fylthy wordes.    
\P 1588 SHAKES.  \textit{L.L.L.} v. i. 4 Your reasons at dinner haue beene..pleasant without scurrillity.    
\P 1654 GATAKER  \textit{Disc. Apol.} 3, I list not to contend with him in scurrilitie and bad language.    
\P 1759 SYMMER in  \textit{Ellis Orig. Lett.} Ser. ii. IV. 414 The hawkers..every day have some new piece of scurrility against him, to bawl about the streets.    
\P 1849 MACAULAY  \textit{Hist. Eng.} v. I. 650 He was, as usual, interrupted in his defence by ribaldry and scurrility from the judgment seat.    
\P 1874 GREEN  \textit{Short Hist.} vii. §1. 346 The Sacrament of the Mass..was attacked with a scurrility and profaneness, which passes belief.

\P a1566 R. EDWARDS  \textit{Damon \& Pithias} (1908) B j b, I came not yet to be the Kinges foole, Or to fill his eares with seruile squirilitie.    
\P 1577 STANYHURST  \textit{Descr. Irel.} ii. 6 b in Holinshed, The heathen misliked in an orature squirilitie.    
\P 1607 DEKKER \& WEBSTER  \textit{Westw. Hoe} ii. i. B 4 b, So long as your mirth bee voyde of all Squirrility.

\itembf{b.} Something scurrilous.

\P 1589 PUTTENHAM  \textit{Eng. Poesie} i. xxxi. (Arb.) 76 Such among the Greekes were called Pantomimi, with vs Buffons, altogether applying their wits to Scurrillities \& other ridiculous matters.    
\P 1733 POPE  \textit{Dunc.} ii. 299 note, Concanen..was author of several dull and dead scurrilities in the British and London Journals.    
\P 1830 D'ISRAELI  \textit{Chas. I}, III. xi. 244 Who could have imagined that the writers of these scurrilities were scholars.

\itembf{c.} Buffoon-like behaviour. Obs.

\P 1614 J. NORDEN  \textit{Labyrinth Mans Life} L 4, Heroicke acts, that make men honorable, Are only sweet and most inestimable; The rest are false, found mere scurrilitie, By which some loose, both fame and dignitie.    
\P 1624 HEYWOOD  \textit{Gunaik.} i. 24 We may as well say, Cats, Goates and Apes, are by chance given to voracitie, lust, and squirilitie.

\noindent Hence \textbf{\phonetic{scuˈrrilitiship}} nonce-wd., ? the state of persons who indulge in scurrility.

\P 1592 NASHE  \textit{Strange Newes} G 2 b, Maister Bird shall..meeter it mischieuously in maintenance of their scurrilitiship and ruditie.
\end{myenumerate}


%%%%%%%%%%%%%%%%%%%%%%%%%%%%%%%%%
\myitem{Billingsgate} n.

\noindent \phonetic{(ˈbɪlɪŋsgeɪt)}

\noindent [The proper name (presumably from a personal name Billing) of one of the gates of London, and hence of the fish-market there established. The 17th c. references to the ‘rhetoric’ or abusive language of this market are frequent, and hence foul language is itself called ‘billingsgate.’]
\vspace{-0.3cm}

\begin{myenumerate}

\itembf{1.} One of the gates of the city of London; the fish-market near it; the latter noted for vituperative language.

\P c1250 LAY. 15070 And ladde to Londene..bisides Bellinges-ȝate [c1205 BæLȝES-].    
\P 1585 PILKINGTON  \textit{Exp. Nehem.} (1841) 345 The gates of cities have their names..of them that builded them, as Lud-gate and Billings-gate, of Lud and Billinns.    
\P 1658 R. NEWCOURT  \textit{Title Map Lond.}, Billings gate Founded by Belen ye 23th Brittish Kinge.    
\P 1672 MARVELL  \textit{Reh. Transp.} i. 167 There is not a scold at Billins$\sim$gate but may defend herself.    
\P 1705 HICKERINGILL  \textit{Priest-cr.} i. (1721) 56 The Rhetorick of Billingsgate, viz. Lying and Slandering.    
\P 1795 WINDHAM  \textit{Speeches Parl.} (1812) I. 266 The scolding of a fishwoman in Billingsgate.    
\P 1848 THACKERAY  \textit{Van. Fair} xiii, Mr. Osborne..cursed Billingsgate with an emphasis worthy of the place.

\itembf{b.} attrib. (in reference to language.)

\P 1652 CULPEPPER  \textit{Eng. Physic.}, With down-right Billingsgate-Rhetoric.    
\P 1726 AMHERST  \textit{Terræ Fil.} x. 48, I know nothing that he is fit for, but Billinsgate sermons.    
\P 1750 WESLEY  \textit{Wks.} (1872) IX. 87 Low, Billingsgate invectives.

\itembf{2.} Scurrilous vituperation, violent abuse.

\P 1676 WYCHERLEY  \textit{Pl.-Dealer} iii. i. (1678) 35 With sharp Invectives  Wid. (Alias Belin'sgate).    
\P 1710 SHAFTESBURY  \textit{Charac.} (1737) III. ii. 15 Philosophers and Divines, who can be contented to..write in learned Billinsgate.    
\P 1799 T. JEFFERSON  \textit{Writ.} (1859) IV. 289 We disapprove the constant billingsgate poured on them officially.    
\P 1867 FREEMAN  \textit{Norm. Conq.} (1876) I. App. 625 This is mere Billingsgate.

\itembf{3.} A clamouring foul-mouthed person, a vulgar abuser or scold. Obs.

\P 1683 TRYON  \textit{Way to Health} 480 Neither have we any Billings-gates, all that sort of People are our hewers of VVood and drawers of Water.    
\P 1715 \textit{Bowes' Trag.} in  \textit{Yorksh. Anthol.} (1851) 18 Words not fit for a Billingsgate.    
\P 1721-90 BAILEY,  \textit{Billingsgate}, a scolding impudent Slut.

\noindent Hence \textbf{Billingsgate} v. rare. \textbf{Billingsgatry}, scurrilous language.

\P 1673 \textit{Remarks  upon Rem.} 56 (Boucher) A great deal of Billingsgatry against poets.    
\P 1715 A. LITTLETON  \textit{Lat. Dict.}, To Billingsgate it. Arripere maledictum ex trivio.
\end{myenumerate}


%%%%%%%%%%%%%%%%%%%%%%%%%%%%%%%%
\myitem{lascivious} a.

\noindent \phonetic{(ləˈsɪvɪəs)}

\noindent [ad. late L. lascīviōs-us (Isidore), f. L. lascīvi-a (n. of quality f. lascīvus sportive, in bad sense lustful, licentious): see -ous.]
\vspace{-0.3cm}

\begin{myenumerate}

\itembf{1.} Inclined to lust, lewd, wanton.

\P c1425 LYDG.  \textit{Assembly of Gods} 686 Lastyuyous [read lascyuyous] lurdeyns, \& pykers of males.    
\P 1494 FABYAN \textit{Chron.} vii. 402 Ye lassiuyous and wanton disposicions of the sayd Pyers of Gaueston.    
\P 1555 EDEN  \textit{Decades} 141 He chaunced to lyue in those lasciuious and wanton dayes.    
\P 1567 J. MAPLET  \textit{Gr. Forest} 88 The Gotebucke is verie wanton or lasciuious.    
\P 1601 SHAKES.  \textit{All's Well} iv. iii. 248, I knew the young Count to be a dangerous and lasciuious boy.    
\P 1601 HOLLAND  \textit{Pliny} II. 544 One picture there is of his doing, wherein he would seeme to depaint Lascivious [quoted in mod. Dicts. as ‘lascious’] wantonnesse.    
\P 1667 MILTON  \textit{P.L.} ix. 1014 Hee  on Eve Began to cast lascivious Eyes.    
\P 1781 COWPER  \textit{Anti-Thelyphthora} 199 The Fauns and Satyrs, a lascivious race, Shrieked at the sight.    
\P 1856 MRS. BROWNING  \textit{Aur. Leigh} iii. 767 Thin dangling locks, and flat lascivious mouth.

\P 1586 W. WEBBE  \textit{Eng. Poetrie} D iiij, He.. is wholy to bee reputed a laciuious disposed personne.

\itembf{b.} Inciting to lust or wantonness. Also in milder sense, voluptuous, luxurious. Obs.

\P 1589 PUTTENHAM  \textit{Eng. Poesie} ii. ix. [x.] (Arb.) 97 Carols and rounds and such light or lasciuious Poemes.    
\P 1594 SHAKES.  \textit{Rich. III,} i. i. 13 He capers nimbly in a ladies Chamber, To the lasciuious pleasing of a Lute.    
\P 1602 T. FITZHERBERT  \textit{Apol.} 36 b, How many are there..that..make no scruple to keep lasciuious pictures to prouoke themselues to lust?    
\P 1621 BURTON  \textit{Anat. Mel.} ii. ii. ii. (1651) 240 By Philters and such kinde of lascivious meats.    
\P 1660 F. BROOKE tr.  \textit{Le Blanc's Trav.} 155 Their garments are something lascivious, for being cut and open their skin is seen.    
\P 1671 L. ADDISON  \textit{W. Barbary} 150 That they should have Chaires there to sit in with as much lascivious ease, as at home.    
\P 1780 COWPER  \textit{Table T.} 462 To the lascivious pipe and wanton song, That charm down fear, they frolic it along.    
\P 1838 LYTTON  \textit{Leila} i. iv, Not thine the lascivious arts of the Moorish maidens.

\itembf{2.} Used for: Rank, luxuriant.

\P 1698 FRYER  \textit{Acc. E. India \& P.} 243 Forded several Plashes where flourished lascivious Shrubs.



\end{myenumerate}


%%%%%%%%%%%%%%%%%%%%%%%%%%%%%%%%
\myitem{voluptuous} a.

\noindent \phonetic{(vəˈlʌptjuːəs)}

\noindent [ad. OF. (also mod.F.) voluptueux, -euse (= Sp. and Pg. voluptuoso, It. voluttuoso), or L. voluptuōsus (Pliny, etc.), f. voluptas pleasure, VOLUPTY. Cf. VOLUPTEOUS a.]
\vspace{-0.3cm}

\begin{myenumerate}

\itembf{1.} Of or pertaining to, derived from, resting in, characterized by, gratification of the senses, esp. in a refined or luxurious manner; marked by indulgence in sensual pleasures; luxuriously sensuous: \textbf{a.} Of desires or appetites.

\P c1374 CHAUCER \textit{Troylus} iv. 1573 Love  ne drof yow nought to don this dede, But lust voluptuous, and cowarde drede.    
\P c1407 LYDG.  \textit{Reson \& Sens.} 4714 To soiourne in the Erbere..Oonly ordeyned for delyte And voluptuouse appetyte.    
\P 1491 CAXTON  \textit{Vitas Patr.} (W. de W. 1495) i. i. 5/1 This techith us our sauyour for to kepe us from voluptuous desyres.    
\P 1526  \textit{Pilgr. Perf.} (W. de W. 1531) 82 b, Abstynence from the carnall voluptuous appetyte of the flesshe.    
\P c1540 in  \textit{Prance Addit. Narr. Pop. Plot} (1679) 36 The supporters of our voluptuose and Carnal Appetite.    
\P 1697 SOUTH  \textit{Serm.} I. 32 God..has corrected the Boundlessness of his Voluptuous desires, by stinting his strengths, and contracting his Capacities.    
\P 1796 MORSE  \textit{Amer. Geog.} II. 546 [Dancing girls, who] communicate, by a natural contagion, the most voluptuous desires to the beholders.

\itembf{b.} Of pleasure or pleasurable sensations.

\P c1407 LYDG.  \textit{Reson \& Sens.} 2022 Venus..goddesse is of al plesaunce, Of lust, and fleshly appetyte, And of voluptuous delyte.    
\P 1603 KNOLLES  \textit{Hist. Turks} (1638) 242 Solyman..lay in great securitie,..passing his time in all voluptuous pleasure.    
\P 1663 S. PATRICK  \textit{Parab. Pilgr.} xiv, Because I believe you are desirous to know, how they receive and take in those voluptuous enjoyments.    
\P 1756 BURKE  \textit{Subl. \& B.} i. v, That smooth and voluptuous satisfaction which the assured prospect of pleasure bestows.    
\P 1820 SHELLEY  \textit{Prometh. Unb.} i. 426 If thou might'st dwell among the Gods the while Lapped in voluptuous joy?    
\P 1869 J. PHILLIPS  \textit{Vesuv.} i. 10 The long voluptuous dream came to a startling end.    
\P 1888 \textit{Buck's  Handbk. Med. Sci.} VI. 397/2 Excessive voluptuous sensations may be the result of peripheral or central causes.

transf. \P 1614 DONNE  \textit{Lett.} (1651) 173 Out of a voluptuous loathnesse to let that taste go out of my mouth.    
\P 1815 SHELLEY  \textit{Alastor} 11 Spring's voluptuous pantings when she breathes Her first sweet kisses, have been dear to me.

\itembf{c.} Of modes of life or conduct.

\P 1432-50 tr.  \textit{Higden} (Rolls) VI. 79 The luffe of the cuntre and elegancy voluptuous deceyvide his grevous labors.    
\P 1553 BRENDE  \textit{Q. Curtius} x. 209 Hauing in these and suche other like voluptuous vanities consumed a great part of the treasure.    
\P a1578 LINDESAY  (Pitscottie) \textit{Chron. Scot.} (S.T.S.) I. 82 They subornit him quyitlie to dissobedience,..for by it they thocht they had ane woluptous lyfe.    
\P 1582 BIBLE  (Genev.) \textit{2nd Alph. Direct., Voluptuous} liuing, one of the thornes that choke the worde.    
\P 1600 HOLLAND  \textit{Livy} xxxvi. ii. 925 The very souldiours were let loose and given over to take voluptuous waies.    
\P 1634 W. TIRWHYT tr.  \textit{Balzac's Lett.} 211 He as easily surmounteth all his voluptuous irregularities, as he doth his most violent revels.    
\P 1685 OTWAY  \textit{Windsor Castle} 124 The Priests who humble Temp'rance should profess, Sought silken Robes and fat voluptuous Ease.    
\P a1734 NORTH  \textit{Lives} (1826) II. 95 By his voluptuous unthinking course of life he ran in debt.    
\P 1809 W. IRVING  \textit{Knickerb.} (1861) 75 The gallant warrior starts from soft repose, from golden visions, and voluptuous ease.    
\P 1817 SHELLEY  \textit{Constantia} iv, The breath of summer night, Which..suspends my soul in its voluptuous flight.    
\P 1838 THIRLWALL  \textit{Greece} xxxviii. V. 29 A man of voluptuous habits, who desired power as an instrument of sensual indulgence.

\itembf{d.} Of fare or feasting.

\P 1544  \textit{Exhort.} in \textit{Priv. Prayers} (1851) 569 Wholesome abstinence..from all delicious living in voluptuous fare.    
\P 1585 LUPTON  \textit{Thous. Notable Th.} (1675) 77 Cleopatra, the last Queen of Egypt,..did drink one so voluptuous a draught as never any did before.    
\P 1638 PENKETHMAN  \textit{Artach.} K 3 Excessive consumption and abuse of Wheat and other Victuals in voluptuous Feasts.    
\P 1727 [DORRINGTON]  \textit{Philip Quarll} (1816) 14 These provisions being somewhat too voluptuous for an hermit.    
\P 1759 B. MARTIN  \textit{Nat. Hist.} I. 78 The most voluptuous Part of Cookery.    
\P 1796 MORSE  \textit{Amer. Geog.} II. 589 That dissolving jelly which is so voluptuous a rarity at European tables.

\itembf{e.} Of places.

\P 1687 A. LOVELL tr.  \textit{Thevenot's Trav.} i. 39 They tell a thousand other Fopperies of this voluptuous Paradise.    
\P 1820 SHELLEY  \textit{Prometh. Unb.} i. 171 Foodless toads Within voluptuous chambers panting crawled.    
\P 1832 W. IRVING  \textit{Alhambra} I. 4 A soft southern region, decked out with all the luxuriant charms of voluptuous Italy.    
\P 1839 THIRLWALL  \textit{Greece} l. VI. 227 The army was permitted to revel for some time in the enjoyments which the most splendid and voluptuous of Eastern cities offered in profusion.

\itembf{2.} Addicted to sensual pleasure or the gratification of the senses; inclined to ease and luxury; fond of elegant or sumptuous living.

\P c1440 \textit{Gesta  Rom.} xviii. 333 (Add. MS.), The voluptuous flessh, that bereth the fire of glotonye and lechery.    
\P 1577 tr.  \textit{Bullinger's Decades} (1592) 20 Voluptuous and daintie louers of this world..doo without any fruite at al heare Gods worde.    
\P 1594 T. B. \textit{La  Primaud. Fr. Acad.} ii. 121 Our Lord Iesus Christ himselfe, who was neither nice nor voluptuous.    
\P 1612 T. TAYLOR  \textit{Comm. Titus} ii. 12 The voluptuous person, is a louer of his pleasure more then of God.    
\P 1638 SIR T. HERBERT  \textit{Trav.} (ed. 2) 240 The poore are not so voluptuous: they content themselves with drie ryce, herbs, roots.    
\P 1670 CLARENDON  \textit{Ess. Tracts} (1727) 166 The lustful and voluptuous Person, who sacrifices the Strength and Vigour of his Body to the Rage and Temptation of his Blood.    
\P a1734 NORTH  \textit{Lives} (1826) II. 411 The bey was a merry fellow, and, like other voluptuous Turks, had his buffoons to divert him.    
\P 1783 JOHNSON  \textit{Lett.} (1788) II. 298 A friend of mine, who courted a lady of whom he did not know much, was advised to see her eat, and if she was voluptuous at table, to forsake her.    
\P 1838 THIRLWALL  \textit{Greece} II. 172 The voluptuous and unwarlike people were protected by impregnable walls.    
\P 1848 LYTTON  \textit{Harold} i. i, A large building that once had belonged to some voluptuous Roman.

absol. \P a1680 BUTLER  \textit{Characters} (1908) 266 The voluptuous is very hard to be pleas'd.    
\P 1682 BURNET  \textit{Rights Princes} v. 160 As if it had been the Rich and Voluptuous, and not the Poor and the Hungry.    
\P 1762 \textit{Charac.} in  \textit{Ann. Reg.} 13 His high relish of social enjoyment soon brought him into request with the voluptuous of all ranks.    
\P 1802  \textit{Gentl. Mag.} Jan. 3/1 To the..Splenetic—the Voluptuous—the Petulant—and the Proud.

transf. \P a1822 SHELLEY  \textit{Calderon} iii. 56 And, voluptuous Vine, O thou Who seekest most when least pursuing.

\itembf{3.} Imparting a sense of delicious pleasure; suggestive of sensuous pleasures, esp. of a refined or luxurious kind.

\P 1816 BYRON  \textit{Ch. Har.} iii. xxi, And when Music arose with its voluptuous swell, Soft eyes look'd love to eyes which spake again.    
\P 1820 HAZLITT  \textit{Lect. Dram. Lit.} 71 The poet succeeds less in the voluptuous and effeminate descriptions.    
\P 1844 LEVER  \textit{T. Burke} xli. 307 The seigneur..had..mixed in the voluptuous fascinations of the period.    
\P 1877 DOWDEN  \textit{Shaks. Primer} vi. 87 The voluptuous moonlit nights are only like a softer day.

\itembf{b.} Suggestive of sensuous pleasure by fulness and beauty of form.

\P 1839 HALLAM  \textit{Hist. Lit.} (1847) II. 101 We recognise his spirit in the sylvan shades and voluptuous forms of Albano and Domenichino.    
\P 1841 MACAULAY  \textit{Ess., Hastings} (1851) 649 There appeared the voluptuous charms of her to whom the heir of the throne had in secret plighted his faith.    
\P 1875 JOWETT  \textit{Plato} (ed. 2) III. 144 The voluptuous image of a Corinthian courtezan.    
\P 1891 FARRAR  \textit{Darkn. \& Dawn} xxvi, She was now twenty-six, but had lost none of her voluptuous loveliness.

transf. \P 1852 TENNYSON  \textit{Ode Wellington} 208 He shall find the stubborn thistle bursting Into glossy purples, which out$\sim$redden All voluptuous garden-roses.
\end{myenumerate}

%%%%%%%%%%%%%%%%%%%%%%%%%%%%%%%%
\myitem{ostensible} a. (n.)

\noindent \phonetic{(ɒˈstɛnsɪb(ə)l)}

\noindent [a. F. ostensible (1740 in Dict. Acad.), ad. L. type *ostensibil-is (med.L. in Laws Hen. I. c. 80 §11), f. ostens-, ppl. stem of ostendĕre: see ostend.]
\vspace{-0.3cm}

\begin{myenumerate}

\itembf{1.} That may be shown, exhibited, or presented to view, hence, presentable; also, made or prepared to be shown. Obs.

\P 1762-71 H. WALPOLE  \textit{Vertue's Anecd. Paint.} (1786) II. 140 [Rubens] was called to Paris by Mary de' Medici, and painted the ostensible history of her life in the Luxemburgh.    
\P 1783 LD. TEMPLE  \textit{Let.} 2 Apr. in \textit{Dk. Buckhm. Crt. Geo.} III (1853) I. 226, I wish you to write me an ostensible letter..upon the conduct of the Portuguese.    
\P 1798 BAY  \textit{Amer. Law Rep.} (1809) I. 92 B. was the only ostensible person in the country, P. having gone off, and C.'s estate not being sufficient to make good the loss.    
\P a1805 A. CARLYLE  \textit{Autobiog.} i. (1860) 31 He took great pains to make them (especially the first, for the second was hardly ostensible) appear among his best scholars.    
\P 1828 BENTHAM  \textit{Wks.} (1843) X. 591 You should..send me two letters—one confidential, another ostensible.

\itembf{2.} That presents itself to view or shows itself off; open to public view; conspicuous, ostentatious. Obs.

\P 1782 in  \textit{Ld. Macartney's Life} \&c. (1807) I. 144 Were we to adopt the ostensible and artificial language of that prudence which [etc.].    
\P 1803 MRQ.  WELLESLEY \textit{Let. to A. Wellesley} 26 June in Owen \textit{Desp.} (1877) 302 The most direct and even ostensible interposition of the British authority.    
\P 1809 MALKIN  \textit{Gil Blas} x. ii. \phonetic{⁋}12 He has been in an ostensible situation..and his father ought to be buried with all the forms of state.    
\P 1828 LD. GRENVILLE  \textit{Sink. Fund} 29 Which..can exhibit to us only the outward and ostensible workings of this complicated mechanism.

\itembf{3.} Declared, avowed, professed; exhibited or put forth as actual and genuine: often implicitly or explicitly opposed to ‘actual’, ‘real’, and so = merely professed, pretended.

\P 1771 JUNIUS  \textit{Lett.} liv. 289 The best of princes is not displeased with the abuse which he sees thrown upon his ostensible Ministers.    
\P 1786 BURKE  \textit{W. Hastings} Wks. 1842 II. 119  A party of British and other troops, with the nabob in the ostensible, and the British resident in the real, command.    
\P 1837 H. MARTINEAU  \textit{Soc. Amer.} III. 269 There will be less that is ostensible and more that is genuine, as they grow older.    
\P 1848 C. BRONTË  \textit{J. Eyre} x. (1873) 85 My ostensible errand on this occasion was to get measured for a pair of shoes.    
\P 1874 GREEN  \textit{Short Hist.} vii. §4. 381 Her ostensible demand was for English aid in her restoration to the throne.

\itembf{B.} as n. in pl. Ostensible matters.

\P 1861 J. PYCROFT  \textit{Agony Point} xxiii. (1862) 231 When all these positive essentials and ostensibles were so respectably witnessed.
\end{myenumerate}


%%%%%%%%%%%%%%%%%%%%%%%%%%%%%%%%
\myitem{specious} a.

\noindent \phonetic{(ˈspiːʃəs)}

\noindent [ad. L. speciōs-us fair, beautiful, fair-seeming, f. speciēs species. Hence also F. spécieux, -euse, It. spezioso, Sp. and Pg. especioso.]
\vspace{-0.3cm}

\begin{myenumerate}

\itembf{1.} Fair or pleasing to the eye or sight; beautiful, handsome, lovely; resplendent with beauty. ? Obs. \itembf{a.} Of persons, their parts, etc., or of things.

(a) \P a1400 MINOR  \textit{Poems fr. Vernon MS.} xxiii. 146 Heil ful of grace, eke Speciouse at al, Mayden wys and þerto Meke.    
\P c1425 \textit{St. Elizabeth  of Spalbeck} in \textit{Anglia} VIII. 115/45 Hir chere semiþ þen ful specyous and cleer \& gracyous.    
\P 1526  \textit{Pilgr. Perf.} (W. de W. 1531) 184 Specyous \& beautyfull is he aboue all the chylder of men.    
\P 1626 T. H[AWKINS]  \textit{Caussin's Holy Crt.} 45 Nicephorus relateth certaine lineaments of his stature, colour and proportion of his members,..in all parts louely and specious.    
\P 1652 GAULE  \textit{Magastrom.} 265 Yet the wise men of Greece were not ashamed to pursue specious boyes.    
\P a1670 HACKET  \textit{Cent. Sermons} (1675) 422 There is thy Saviour..looking like a specious Bridegroom.    
\P 1748 RICHARDSON  \textit{Clarissa} (1811) I. xvi. 109 Disagreeable only as another man has a much more specious person.    
\P 1791 COWPER  \textit{Odyss.} xvii. 547 Gods! how illiberal with that specious form!    
\P 1818 HAZLITT  \textit{Eng. Poets} i. (1870) 14 The Greek statues are little else than specious forms.

(b) \P 1402  \textit{Pol. Poems} (Rolls) II. 98 The pore man at the specious ȝate praiede to the apostlis to parten of her almes.    
\P c1440  \textit{Gesta Rom.} viii. 20 That oþer [way] specius and faire, sett aboute withe lileis and Rosis.    
\P 1582 N.T. (Rhem.) Acts iii. 10 He which sate for almes at the Specious gate of the temple.    
\P 1621 R. BRATHWAIT  \textit{Nat. Embassie} (1877) 188 Smooth to the touch, and specious to the sight.    
\P 1651 FRENCH  \textit{Distill.} vi. 192 So will the Spirit..be coloured with a very specious blue colour.    
\P 1697 AUBREY  \textit{Brief Lives} (1898) I. 77 The great Cardinal Richelieu, who lived both to designe and finish that specious towne of Richelieu.    
\P 1756 BURKE  \textit{Subl. \& B.} Wks. 1842 I. 57 When  any object partakes of the above mentioned qualities, or of those of beautiful bodies, and is withal of great dimensions, it is full as remote from the idea of mere beauty; I call it fine or specious.

transf.\P c1485 DIGBY  \textit{Myst.} (1882) iii. 628 To me itt is a Ioye most speceows.    
\P 1631 MASSINGER  \textit{Emperor East} i. ii, Your specious titles Cannot but take her.

\itembf{b.} Of flowers, birds or their feathers, etc. In later use, having brilliant, gaudy, or showy colouring. Also transf.

(a) \P 1513 BRADSHAW  \textit{St. Werburge} i. 3456 This rutilant gemme and specious floure [sc. the body of St. Werburge].    
\P a1637 B. JONSON  \textit{Underwoods, Epitaph Master Corbet} Wks. (1640) 178 And adde his Actions unto these, They were as specious as his Trees.    
\P a1682 SIR T. BROWNE  \textit{Misc. Tracts} (1684) 93 Successive acquists of fair and specious Plants.    
\P 1731 MILLER  \textit{Gard. Dict.} s.v. Saxifraga, The fourth Sort is propagated for the Sake of its specious Flowers.    
\P 1800 ANDREWS  \textit{Bot. Rep.} 87 This truly specious Ixia!    
\P 1812  \textit{New Botanic Gard.} I. 29 The corolla specious, and purple in colour.    
\P 1837 P. KEITH  \textit{Bot. Lex.} 265 The novice in botany, who is attracted, perhaps, only by what is specious in the plant or flower.

(b) \P 1688 HOLME  \textit{Armoury} ii. 287 It can set up specious feathers on the crown of its head like a crest.    
\P 1688  \textit{Phil. Trans.} XVII. 996 There be other sorts of Goldfinches variegated with red, orange and yellow Feathers, very specious and beautiful.    
\P 1786 S. GOODENOUGH in  \textit{Mem. Sir J. E. Smith} (1832) I. 184 Bees, several new ones, one very specious indeed.    
\P 1803 SHAW  \textit{Gen. Zool.} IV. ii. 603 Specious Mackrel, Scomber Speciosus.    
\P 1809  \textit{Ibid.} VII. ii. 364 Specious Jay, Corvus speciosus. Crested green Jay.

\itembf{2.} Having a fair or attractive appearance or character, calculated to make a favourable impression on the mind, but in reality devoid of the qualities apparently possessed.

In certain contexts passing into the sense ‘merely apparent’.

\P 1612 T. TAYLOR  \textit{Comm. Titus} i. 16 Their actions, although neuer so good in themselues, neuer so specious vnto others,..yet are abhominable vnto God.    
\P 1644 QUARLES  \textit{Judgm. \& Mercy} 144 Let not the specious goodness of the end encourage me to the unlawfulness of the means.    
\P 1681 DRYDEN  \textit{Abs. \& Achit.} 746 A smooth pretence Of specious love, and duty to their Prince.    
\P 1705 STANHOPE  \textit{Paraphr.} II. 264 The most specious Instances,..such as Martyrdom,..are no necessary Proofs of Charity.    
\P 1743 FRANCIS tr.  \textit{Hor., Odes} ii. i. 4 The specious Means, the private Aims,..how fatal to the Roman State!    
\P 1774 REID  \textit{Aristotle's Logic} iv. §2 (1788) 72 The friends of Aristotle have shown that this improvement of Ramus is more specious than useful.    
\P 1807 CRABBE  \textit{Birth Flattery} 67 What are these specious gifts, these paltry gains?    
\P 1849 MACAULAY  \textit{Hist. Eng.} v. I. 599 It appeared that this plan, though specious, was impracticable.    
\P 1873 W. H. DIXON  \textit{Two Queens} x. v. II. 179 What was done by him in Rome was merely specious.

absl. \P 1676 DRYDEN  \textit{Aurengz.} Ep. Ded. A ij, But somewhat of Specious they must have, to recommend themselves to Princes.

\itembf{b.} Of pretences, pretexts, etc.

\P 1611 SPEED  \textit{Hist. Gt. Brit.} ix. viii. 499/2 Traiterous requests..which he was now willing to maske with the specious pretext of iustice and deuotion.    
\P 1632 \textit{Galway Arch.} in  \textit{10th Rep. Hist. MSS. Comm.} App. V. 478 The specious pretences you made.    
\P 1734 \textit{Col. Records Pennsylv.} III. 546 Notwithstanding the specious and ample Professions made by the Governor of Maryland.    
\P 1769 ROBERTSON  \textit{Chas.} V, x. III. 254 The specious pretexts which had formerly concealed his ambitious designs.    
\P 1836 THIRLWALL  \textit{Greece} xvii. III. 4 Cimon seized this specious pretext for exterminating the people.

\itembf{c.} Of appearance, show, etc.

\P a1628 PRESTON  \textit{Effect. Faith} (1631) 74 There be many works that have a specious and faire shew in the view of men; But..God regards them not.    
\P 1647 CLARENDON  \textit{Hist. Reb.} iv. §172 The law..being neglected or disesteemed (under what specious shews soever).    
\P 1729 BUTLER \textit{Serm.} Wks. 1874 II. 65 A  discovery..which they..have found out through all the specious appearances to the contrary.    
\P 1735 SOMERVILLE  \textit{Chase} ii. 313 To rob, and to destroy, beneath the Name And specious Guise of War.    
\P a1827 WORDSW.  \textit{Sonn. Liberty} ii. vi. 10 Ere wiles and politic dispute Gave specious colouring to aim and act.    
\P 1849 MACAULAY  \textit{Hist. Eng.} vii. II. 231 A policy which had a specious show of liberality.    
\P 1870 MOZLEY  \textit{Univ. Serm.} iv. (1877) 74 We have even in the early Christian Church that specious display of gifts which put aside as secondary the more solid part of religion.

\itembf{d.} Of falsehood, bad qualities, etc.

\P 1665 J. GLANVILL  \textit{Scepsis Sci.} xiv. 79 Such an Infinite of uncertain opinions, bare probabilities, specious falshoods.    
\P 1682 DRYDEN  \textit{Abs. \& Achit.} ii. 955 Who Truth from specious falsehood can divide [etc.].    
\P 1728 YOUNG  \textit{Love Fame} ii. 68 If not to some peculiar end assign'd, Study's the specious trifling of the mind.    
\P 1748 W. MELMOTH  \textit{Fitzosborne Lett.} lii. (1749) II. 63 Religion without this sovereign principle [generosity], degenerates into slavish fear, and wisdom into a specious cunning.    
\P 1823 SCOTT  \textit{Quentin D.} xvii, In whose eyes the sincere devotion of a heathen is more estimable than the specious hypocrisy of a Pharisee.    
\P 1866 MRS. H. WOOD  \textit{St. Martin's Eve} i. (1874) 4 Be not ensnared by specious deceit.

\itembf{3.} Of language, statements, etc.: Fair, attractive, or plausible, but wanting in genuineness or sincerity.

\P 1651 HOBBES  \textit{Leviath.} ii. xxi. 110 It is an easy thing, for men to be deceived, by the specious name of Libertie.    
\P 1665 MANLEY  \textit{Grotius' Low C. Wars} 371 The Prince,..by an evident demonstration, confuting specious words.    
\P 1670 MARVELL  \textit{Corr.} Wks. (Grosart) II. 338 This motion seemed specious and welcome to the Committee.    
\P 1712 ADDISON  \textit{Spect.} No. 469 \phonetic{⁋}5 Gratifications, Tokens of Thankfulness, Dispatch Money, and the like specious Terms.    
\P 1798 S. \& HT.  \textit{Lee Canterb. T.} II. 230 She then imparted the specious tale of the Marquis's loss at the gaming-table.    
\P 1849 MACAULAY  \textit{Hist. Eng.} v. I. 568 The meaning latent under this specious phrase.    
\P 1855 MOTLEY  \textit{Dutch Rep.} v. v. (1866) 748 The specious language of Philip's former letters.

\itembf{b.} Of reasoning, arguments, etc.: Plausible, apparently sound or convincing, but in reality sophistical or fallacious.

\P 1651 HOBBES  \textit{Leviath.} i. xv. 73 This specious reasoning is neverthelesse false.    
\P 1656 tr.  \textit{Hobbes' Elem. Philos.} (1839) 415 For the establishing of vacuum, many and specious arguments and experiments have been brought.    
\P 1726 POPE  \textit{Odyss.} xix. 8 To sooth their fears a specious reason feign.    
\P 1788 GIBBON  \textit{Decl. \& F.} xliv. IV. 378 A specious theory is confuted by this free and perfect experiment.    
\P 1791 MACKINTOSH  \textit{Vind. Gall.} Wks. 1846 III.  107 Many subtle and specious objections are urged.    
\P 1856 \textit{N. Brit.  Rev.} XXVI. 23 Undoubtedly it is robust good sense which is here brought to bear upon a specious sophism.    
\P 1877 GEIKIE  \textit{Christ} xxvii. (1879) 308 He was not led away by such suggestions, however specious.

absol. \P a1850 J. C. CALHOUN  \textit{Wks.} (1874) III. 274 To this it may be traced, that the Senator prefers the specious to the solid, and the plausible to the true.

\itembf{4.} Apparent, as opposed to real. Obs.—1

\P 1617 MORYSON  \textit{Itin.} ii. 64 The Lord Deputie conceived the Earles surprise to bee an evill more spetious then materiall.

\itembf{5.} Of material things: Outwardly or superficially attractive or pleasing, but possessing little intrinsic worth; showy. rare.

\P 1816 SIR J. REYNOLDS  \textit{Charac. of Painters of Italy} 136 [Michael Angelo] has rejected all the false, though specious ornaments, which disgrace the works even of the most esteemed artists.    
\P 1825 MACAULAY  \textit{Ess., Milton} (1851) I. 23 We shall, like Bassanio in the play, turn from the specious caskets.., and fix on the plain leaden chest.

\itembf{6.} Of persons: Characterized by conduct, actions, or reasoning, of a specious nature; outwardly respectable.

\P 1740 RICHARDSON  \textit{Pamela} (1824) I. 83 But now I have found you out, you specious hypocrite!    
\P 1798 CANNING  \textit{New Morality} 84 in Poetry Anti-Jacobin (1799) 223 If Vice appal thee..Yet may the specious bastard brood, which claim A spurious homage under Virtue's name,..rouse thee!    
\P 1799 W. GILPIN  \textit{Serm.} v. 54, I propose next to describe that of the specious or decent man. By the decent man, I mean him, who governs all his actions by appearances.    
\P 1841 DICKENS  \textit{Barn. Rudge} xl, You are a specious fellow,..and carry two fans under your hood.    
\P 1884  \textit{Pall Mall G.} 14 May 5/1 If we were to sum up similarly in one word the chief characteristics of their German rival, we should say that Von Hartmann was specious.

\itembf{7.} Of algebra: = LITERAL a. 1 c. Obs. (Cf. SPECIES 8 b.)

\P 1670 COLLINS in  \textit{Rigaud Corr. Sci. Men} (1841) I. 154 A design to cause Diophantus to be turned into specious algebra.    
\P 1673 KERSEY  \textit{Algebra} I. i. 2 Algebra is by late Writers divided into two kinds; to wit, Numeral and Literal (or Specious).    
\P 1728 CHAMBERS  \textit{Cycl.} s.v. Algebra, In 1590, Vieta..introduc'd what he call'd his ‘Specious Arithmetick’, which consists in denoting the Quantities..by Symbols or Letters.

\itembf{8.} Psychol. Appearing to be actually known or experienced.

\P 1890 W. JAMES  \textit{Princ. Psychol.} I. 642 We are constantly conscious of a certain duration—the specious present—varying in length from a few seconds to probably not more than a minute.
\end{myenumerate}

%%%%%%%%%%%%%%%%%%%%%%%%%%%%%%%%
\myitem{opprobrious} a.

\noindent \phonetic{(əˈprəʊbrɪəs)}

\noindent [ad. OF. ob-, opprobrieux, or late L. opprobriōs-us, f. L. opprobrium: see opprobrium.]
\vspace{-0.3cm}

\begin{myenumerate}

\itembf{1.} Of words, language, etc.: Conveying opprobrium or injurious reproach; attaching, or intended to attach, disgrace; contumelious, vituperative, abusive. Rarely of persons: Using contumelious or abusive language.

\P 1387 TREVISA  \textit{Higden} (Rolls) VII. 167 Prayeng a opprobrious a reprevynge name unto þaym but if they drank.    
\P 1483 CAXTON  \textit{Gold. Leg.} (1892) 1079 After many obprobryes wordes..they ladde hym forthe vnto a tree.    
\P a1548 HALL  \textit{Chron., Edw. IV,} 198 b, A man contumelious, opprobrious, and an iniurious person.    Ibid., Hen. VIII, 144 These with many approbrious wordes, were spoken against the Cardinall.    
\P 1602 ROWLANDS  \textit{Greene's Ghost} 3 The name of Conicatchers is..vsed for an opprobrious name for euerie one that sheweth the least occasion of deceit.    
\P 1715-20 POPE  \textit{Iliad} vii. 108 Stern Menelaus first the silence broke, And, inly groaning, thus opprobrious spoke.    
\P 1831 MACAULAY  \textit{Ess., Hampden} (1887) 228 The multitude pressed round the King's coach, and insulted him with opprobrious cries.    
\P 1839 I. TAYLOR  \textit{Anc. Chr.} I. iv. 548 The opprobrious epithet, hypocrite..is the world's rough judgment.

\itembf{b.} Of actions, feelings, etc.: Offering or disposed to offer indignity; insulting, insolent. Obs.

\P 1630 QUARLES  \textit{Div. Poems, Sion's Sonn.} xi. iv, The Bridall bed, which Time, or Age Durst never warrant from th' opprobrious rage Of envious fate.    
\P 1701 ROWE  \textit{Amb. Step-Moth.} iv. iii, Whom that fell Dog..With most opprobrious Injuries has loaded.

\itembf{2.} Attended by or involving shame or infamy; held in dishonour; associated with disgrace; infamous, shameful, disgraceful. Now rare.

\P c1510 MORE  \textit{Picus} Wks. 15/2 The opprobriouse death of the crosse.    
\P 1597 HOOKER  \textit{Eccl. Pol.} v. lxxxi. §15 Neither did any thing seeme opprobrious out of which there might arise commoditie and profit.    
\P 1667 MILTON  \textit{P.L.} i. 403 The wisest heart Of Solomon he led..to build His Temple right against the Temple of God, On that opprobrious Hill.    
\P 1784 COWPER  \textit{Task} v. 379 Opprobrious more To France than all her losses and defeats,..Her house of bondage,..the Bastille.    
\P 1860 PUSEY  \textit{Min. Proph.} 81 The reproachful words of the enemies of God are but the echo of the opprobrious deeds of His unfaithful servants.

\itembf{b.} Subject to opprobrium. rare.

\P 1804 E. DE ACTON  \textit{Tale without Title} II. 133 To see their emoluments arise from some other source than tithes, the collection of which frequently renders them very opprobrious to their parishioners.
\end{myenumerate}


%%%%%%%%%%%%%%%%%%%%%%%%%%%%%%%%
\myitem{vituperative} a.

\noindent \phonetic{(vaɪˈtjuːpərətɪv, vɪ-)}

\noindent [ad. L. type *vituperātīv-us, f. vituperāt-, ppl. stem of vituperāre, or directly f. vituperate v. + -ive. Cf. obs. F. vituperativement adv. (Godef.), It. vituperativo.]
\vspace{-0.3cm}

\begin{myenumerate}

\itembf{1.} Of words, language, etc.: Containing, conveying, or expressing strong depreciation; violently abusive or fault-finding; contumelious, opprobrious. Also, of or pertaining to vituperation.
   Freq. in the 19th c.

\P 1727 POPE, etc. \textit{Art of Sinking} 115 The vituperative partition will as easily be replenished with a most choice collection [of arguments].    
\P 1759 STERNE  \textit{Tr. Shandy} i. xix, Tristram!—Melancholy dissyllable of sound! which, to his ears, was unison to Nincompoop, and every name vituperative under heaven.    
\P 1816 SCOTT  \textit{Antiq.} xxx, In utter despair at this vituperative epithet.    
\P 1856 KANE  \textit{Arct. Expl.} II. xii. 129 His eloquence becoming more and more licentious and vituperative.    
\P 1859 MILL  \textit{Liberty} ii. (1865) 32/1 It is far more important to restrain this employment of vituperative language than the other.

\itembf{b.} Const. of (a person). rare— 1.

\P 1823 SCOTT  \textit{Quentin D.} viii, Had I..heard by report that a question vituperative of my Prince had been asked by the King of France, I had..instantly mounted and returned.

\itembf{2.} Characterized or accompanied by vituperation or abuse.

\P 1754 CHESTERFIELD in  \textit{World} No. 101 \phonetic{⁋}3 The torrents of their [sc. female] eloquence, especially in the vituperative way, stun all opposition.    
\P 1844 DISRAELI  \textit{Coningsby} ii. i, The indignant, soon to become vituperative, secession of a considerable section of the cabinet.    
\P 1871 ‘HOLME Lee’ \textit{Miss Barrington} I. ix. 129 When they have been most in fault themselves, they are most prone to shower a general vituperative blame and condemnation on the other side.

\itembf{3.} Of persons: Given to vituperation; employing or uttering abusive language.

\P 1819  \textit{Blackw. Mag.} V. 90 A Whig is a vituperative animal.    
\P 1843 CARLYLE  \textit{Past \& Pr.} iii. v, Quietly hearing all manner of vituperative able editors speak.    
\P 1904 H. PAUL  \textit{Hist. Mod. Eng.} I. xii. 208 The violent and vituperative champion of the Protestant religion.

\noindent Hence \textit{vituperatively} adv., in a vituperative manner; with vituperation or abuse.

\P 1831 CARLYLE in Froude  \textit{First 40 Years} (1882) II. 159 The critical republic will cackle vituperatively, or perhaps maintain total silence.    
\P 1852  \textit{Fraser's Mag.} XLVI. 456 [He] continues his vituperatively shrill demands.    
\P 1884 J. PARKER  \textit{Apost. Life} III. 115 They would not speak their mother tongue if they did not speak vituperatively.
\end{myenumerate}


%%%%%%%%%%%%%%%%%%%%%%%%%%%%%%%%
\myitem{contumelious} a.

\noindent \phonetic{(kɒntjuːˈmiːlɪəs)}

\noindent [a. OF. contumélieus (mod.F. -eux), ad. L. contumēliōs-us, f. contumēlia contumely + -ous.]
\vspace{-0.3cm}

\begin{myenumerate}

\itembf{1.} Of words and actions: Of the nature of, or full of contumely; reproachful and tending to convey disgrace and humiliation; despiteful.

\P 1483 CAXTON  \textit{Gold. Leg.} 427/3 He sayd noo wordes tumelous ne contumelious ne other dysordynate wordes.    
\P 1526  \textit{Pilgr. Perf.} (1531) 13 Contumelyous and opprobryous blasphemes of the iewes.    
\P 1531 ELYOT  \textit{Gov.} iii. xii, Catullus..wrate agayne hym contumelyouse or reprocheable versis.    
\P 1591 SHAKES.  \textit{1 Hen. VI,} i. iv. 39 With scoffes and scornes, and contumelious taunts.    
\P 1701 SWIFT  \textit{Contests Nobles \& Com.} Wks. 1755 II. I. 31  The people frequently proceeded to rude contumelious language.    
\P 1884  \textit{Manch. Exam.} 29 Oct. 5/2 ‘Bonnet’..‘jackal’..‘badger’..are all contumelious terms.

\itembf{b.} Of persons: Dealing in or using contemptuous reproach or abuse; superciliously insolent.

\P 1548 HALL  \textit{Chron.} 198 b, Kyng Edward..is a man, contumelious, opprobrious.    
\P 1614 T. ADAMS  \textit{Divell's Banket} 229 He is not contumelious against vs, that haue been contumacious against him.    
\P 1855 TENNYSON  \textit{Maud} i. xiii. 2 Curving a contumelious lip.

\itembf{c.} Insolent. Obs.

\P 1561 T. N[ORTON]  \textit{Calvin's Inst.} (1634) Table Script. Quot., A contumelious and stubborne sonne, which will not be ruled by his Father or Mother.    
\P 1650 BULWER  \textit{Anthropomet.} viii. 100 In the contumelious despight of Nature [they] will have ears larger than Hounds.    
\P a1745 SWIFT  \textit{Wks.} (1841) II. 438 [Faction] was so universal that I observed the dogs in the streets much more contumelious and quarrelsome than usual.

\itembf{2.} Reproachful, shameful, disgraceful. Obs.

\P 1546 LANGLEY  \textit{Pol. Verg. De Invent.} iv. v. 89 a, It was a contumelious thing both emong the Romaines and the Lumbardes to be shauen.    
\P 1663 COWLEY  \textit{Verses \& Ess., Of Liberty} (1669) 82 If anything indeed ought to be called honorable, in so base and contumelious a condition.
\end{myenumerate}


%%%%%%%%%%%%%%%%%%%%%%%%%%%%%%%%
\myitem{scurrilous} a.

\noindent \phonetic{(ˈskʌrɪləs)}

\noindent [f. scurrile a. + -ous.]
%\vspace{-0.3cm}

\noindent ‘Using such language as only the licence of a buffoon can warrant’ (J.); characterized by coarseness or indecency of language, esp. in jesting and invective; coarsely opprobrious or jocular.

\P 1576 GASCOIGNE  \textit{Needles Eye} Wks. 1910 II. 419  What shall we thinke of skurulous, deceyptfull, byting, slanderous..wordes?    
\P 1597 HOOKER  \textit{Eccl. Pol.} v. Ded. §7 The scurrilous and more then Satyricall immodestie of Martinisme.    
\P 1611 SHAKES.  \textit{Wint. T.} iv. iv. 215 Forewarne him, that he vse no scurrilous words in's tunes.    
\P 1651 HOBBES  \textit{Leviath.} ii. xxi. 110 Sometimes a scurrilous Jester, as Hyperbolus.    
\P 1716 ADDISON  \textit{Freeholder} No. 23 \phonetic{⁋}1 They are grown scurrilous upon the Royal family.    
\P 1828 MACAULAY  \textit{Ess., Hallam} (1851) I. 56 They might be violent in innovation and scurrilous in controversy.    
\P 1874 GREEN  \textit{Short Hist.} vii. §2. 359 The old scurrilous ballads were heard again in the streets.

\noindent Hence \textit{scurrilously} adv., in a scurrilous manner; after the manner of a buffoon. Also \textit{scurrilousness}.

\P 1597 BEARD  \textit{God's Judgem.} ii. xxxvi. (1631) 431 Such as shamed not as soone as they had glutted their..heads with wine, to fall scurrilously a dauncing.    
\P 1666 PEPYS  \textit{Diary} 17 Oct., Heard the Duke discourse, which he did mighty scurrilously, of the French.    
\P 1727 BAILEY  vol. II, Scurrilousness, scandalous Language, saucy Drollery, Buffoonry.    
\P 1789 W. BELSHAM  \textit{Ess.} (1799) II. 369 He has been..scurrilously reviled as the genuine successor and counterpart of..Hugh Peters.

%%%%%%%%%%%%%%%%%%%%%%%%%%%%%%%%
\myitem{bellow} v.

\noindent \phonetic{(ˈbɛləʊ)}

\noindent [\phonetic{Of uncertain etymology. The equation of ME. belwen with the rare OE. bylᴁian suggests that the latter is late WSax. for *bięlᴁian, Anglian *bęlᴁian; but the origin of this is not evident, unless it be a parallel formation to the synonymous bellan, bell v.4, say from OTeut. *balligôjan: cf. OE. a-dílᴁian, OS. dîligôn, OTeut. *dîligôjan, parallel to *dîlôjan, in OHG. tîligôn and tîlôn to destroy.}]
\vspace{-0.3cm}

\begin{myenumerate}

\itembf{1.} prop. To roar as a bull, or as a cow when excited. (Ordinarily, a cow lows.)

\P c1000 \textit{Martyrol.}  17 Jan. (Cockayne Shrine 52) Hwilum þa deofol hine swungon..hwilum hi hine \phonetic{bylᴁedon} on swa fearras and ðuton eall swa wulfas.    
\P c1305  \textit{Leg. Rood} 145 Beestes gan belwe in eueri binne.    
\P 1377 LANGL.  \textit{P. Pl.} B. xi. 333 Þere ne was cow..Þat wolde belwe after boles.    
\P 1388 WYCLIF  \textit{Jer.} l. 11 And lowiden ether bellewiden, as bolis.    
\P 1580 NORTH  \textit{Plutarch} 358 (R.) Like wild beasts bellowing and roaring.    
\P 1611 SHAKES.  \textit{Wint. T.} iv. iv. 28 Iupiter Became a Bull, and bellow'd.    
\P 1784 BURNS  \textit{Lett.} x. Wks. (Globe) 302 A cow bellowing at the crib without food.    
\P 1868 \textit{Once a Week} No. 5. 99 The first bull advances bellowing fiercely.

\itembf{b.} trans.

\P 1868 \textit{Once a Week} No. 5. 99 A young bull bellows a challenge.

\itembf{2.} Applied to the roaring of other animals; used formerly in sense of BELL v.4 2.

\P 1486  \textit{Bk. St. Albans} E v, An hert belowys.    
\P 1575 TURBERV.  \textit{Venerie} 238 An harte belloweth.    
\P 1596 SHAKES.  \textit{Merch. V.} v. i. 73 Youthful and vnhandled Colts..bellowing and neighing loud.    
\P 1602   \textit{Ham.} iii. ii. 264 The croaking Rauen doth bellow for Reuenge.    
\P 1738-51 CHAMBERS  \textit{Cycl.} s.v. Hunting. The terms for their noise at rutting time..A hart belleth; a buck growns or troats; a roe bellows.    
\P 1766 \textit{Vacation} in  Dodsley \textit{Coll. Poems} III. 153 The master stag..Bellows loud with savage roar.    
\P 1875 B. TAYLOR  \textit{Faust} iii. I. 51 Poodle..Cease to bark and bellow.

\itembf{3.} Of human beings: To cry in a loud and deep voice; to shout, vociferate, roar (depreciative or humorous); also (seriously) to roar from pain.

\P 1602 SHAKES.  \textit{Ham.} iii. ii. 36 There bee Players..that..haue so strutted and bellowed.    
\P 1649 MILTON  \textit{Eikon.} Wks. (1738) I. 43 Not fit for that liberty which they cried out and bellowed for.    
\P 1709 STEELE  \textit{Tatler} No. 54 \phonetic{⁋}3 He is accustom'd to roar and bellow so terribly loud in the Responses.    
\P 1718 POPE  \textit{Iliad} v. 1053 Mars  bellows with the pain.    
\P 1824 W. IRVING  \textit{T. Trav.} II. 234 Like a bully bellowing for more drink.

\itembf{b.} trans. To utter (words or cries) in a loud and deep voice; frequently with out, forth.

\P 1581 NOWELL \& DAY in  \textit{Confer.} i. (1584) D iiij b, Beelzebub bellowed out most horrible blasphemies.    
\P 1603 KNOLLES  \textit{Hist. Turkes} (1621) 663 Bellowing out certaine superstitious charms.    
\P 1771 SMOLLETT  \textit{Humph. Cl.} (1815) 143 Noisy rustics bellowing ‘Green pease’ under my window.    
\P 1881 C. M. YONGE  \textit{Lads \& Lasses Langley} i. 41 Some used to bellow or screech out any familiar hymn in an irreverent way.

\itembf{c.} \textit{to bellow off}: to drive off by shouting, to shout down.

\P 1837 CARLYLE  \textit{Fr. Rev.} II. iii. iii. ix. 249 Fain would Reporter Rabaut speak his..last-words; but he is bellowed off.

\itembf{4.} Of thunder, cannon, wind, the sea, and other inanimate agents: To make a loud hollow noise; to roar.

\P 1384 CHAUCER  \textit{H. Fame} (Fairf.) 1803 A  soun As lowde as beloweth [v.r. belwith, bellyth, belleth] wynde in helle.    
\P 1596 SPENSER  \textit{F.Q.} i. vii. 7 A dreadfull sownd, Which through the wood loud bellowing did rebownd.    
\P 1653 HOLCROFT  \textit{Procopius} 36 Mount Vesuvius bellowed.    
\P 1727 THOMSON \textit{Summer} 1168 Thule  bellows through her utmost isles.    
\P c1800 WORDSW.  \textit{Sonn. Liberty} xii, And Ocean [should] bellow from his rocky shore.    
\P 1866 B. TAYLOR  \textit{Soldier \& Pard} 27 Our cannon bellowed round.

\itembf{b.} With obj.: To give forth, emit, utter, or proclaim with loud noise.

\P 1706 WATTS  \textit{Horæ Lyr.} ii. I. 236 Till the hollow brazen clouds Had bellow'd..Loud thunder.    
\P 1852 TENNYSON  \textit{Ode Wellington} 66 His captain's-ear has heard them boom, Bellowing victory, bellowing doom.    
\P 1858 HAWTHORNE  \textit{Fr. \& It. Jrnls.} I. 141 A large cannon-ball..rolling down..bellowing forth long thunderous echoes.
\end{myenumerate}


%%%%%%%%%%%%%%%%%%%%%%%%%%%%%%%%
\myitem{vociferous} a.

\noindent \phonetic{(və(ʊ)ˈsɪfərəs)}

\noindent [f. L. vōcifer-ārī (see VOCIFERATE v.) + -ous.]
\vspace{-0.3cm}

\begin{myenumerate}

\itembf{1.} Uttering loud cries or shouts; clamorous, bawling, noisy.

\P c1611 CHAPMAN  \textit{Iliad} ii. 83 Thrise three vociferous heralds rose to checke the rout, and get Eare to their Ioue-kept gouernors.    
\P 1700 T. BROWN tr.  \textit{Fresny's Amusem.} 121, I sailed into a Presbyterian Meeting..where the vociferous Holder-forth was as bold and saucy, as if the Deity and all Mankind had owed him Money.    
\P 1749 FIELDING  \textit{Tom Jones} ii. ix, Mr. Allworthy had been before silent, from the same cause which had made his sister vociferous.    
\P 1784 COWPER  \textit{Task} 1 299 The boorish driver leaning o'er his team Vocif'rous, and impatient of delay.    
\P 1816 SOUTHEY  \textit{Poet's Pilgr. Proem} xviii, The restless joy Of those glad girls, and that vociferous boy!    
\P 1834 JAMES  \textit{J. Marston Hall} vii, My companions were very vociferous.    
\P 1875 JOWETT  \textit{Plato} (ed. 2) V. 56 The whole audience instead of being mute became vociferous.

transf. \P 1850-1 LONGFELLOW  \textit{Gold. Leg.} Prol., Sp. iv, Hover downward! Seize the loud, vociferous bells, and..to the pavement Hurl them from their windy tower.

fig. \P 1883  \textit{Harper's Mag.} Sept. 565/1 Mr. Cody..could scarcely design a vulgar and vociferous work if he tried.

\itembf{b.} Applied to birds.

\P 1809 SHAW  \textit{Gen. Zool.} VII. 94 Vociferous Eagle, Falco Vocifer.    
\P 1809 W. IRVING  \textit{Knickerb.} iii. ii. (1820) 170 Flocks of vociferous geese cackled about the fields.

\itembf{2.} Of the nature of vociferation; uttered with or accompanied by clamour; characterized by loud declamation.

\P 1631 R. BRATHWAIT  \textit{Whimzies, Piper} 144 All he reedes, he puts into his pipe: which consisting of three notes breaks out into a most vociferous syllogisme.    
\P 1740 CIBBER  \textit{Apol.} (1756) II. 59 Though candour and benevolence are silent virtues, they are as visible as the most vociferous ill-nature.    
\P 1828 D'ISRAELI  \textit{Chas.} I, II. v. 126 Popular gratitude is as vociferous as it is sudden.    
\P 1837 W. IRVING  \textit{Capt. Bonneville} II. 283 Jealousy of their good name now prompted them to the most vociferous vindications of their innocence.    
\P 1873 BLACK  \textit{Pr. Thule} i, Showing by his answers that he was but vaguely hearing the vociferous talk of his companions.
\end{myenumerate}


%%%%%%%%%%%%%%%%%%%%%%%%%%%%%%%%%
%\myitem{blatant} a. (and n.)

%\noindent \phonetic{(ˈbleɪtənt)}

%\noindent [Apparently invented by Spenser, and used by him as an epithet of the thousand-tongued monster begotten of Cerberus and Chimæra, the ‘blatant’ or ‘blattant beast’, by which he symbolized calumny. It has been suggested that he intended it as an archaic form of bleating (of which the 16th c. Sc. was blaitand), but this seems rather remote from the sense in which he used it. The L. blatīre to babble, may also be compared. (The a was probably short with Spenser: it is now always made long.)]
%\vspace{-0.3cm}

%\begin{myenumerate}

%\itembf{1.} In the phrase ‘blat(t)ant beast’, taken from Spenser (cf. F.Q. v. xii. 37, 41; vi. i. 7, iii. 24, ix. 2, x. 1, xii. advt., xii. 2): see above.

%\P 1596 SPENSER  \textit{F.Q.} v. xii. 37 Unto themselves they [Envie and Detraction] gotten had A monster which the blatant beast men call, A dreadful feend of gods and men ydrad.    Ibid. vi. i. 7 ‘The blattant beast,’ quoth he, ‘I doe pursew.’    
%\P 1602  \textit{Return fr. Parnass.} v. iv. (Arb.) 69 The Ile of Dogges, where the blattant beast doth rule and raigne.    
%\P 1636 C. FITZGEFFREY  \textit{Bless. Birthd.} (1881) 128 That blatant beast So belched forth from his blaspheaming brest.    
%\P a1658 CLEVELAND  \textit{Gen. Poems} (1677) 60 Cub of the Blatant Beast.    
%\P 1768 TUCKER  \textit{Lt. Nat.} I. 596 The blatant beast..with his unbridled tongue.    
%\P 1812 BYRON  \textit{Ch. Har.} i. xxvi. (Orig. MS.), Then burst the blatant beast [note, a figure for the mob], and roar'd, and raged.    
%\P 1856 MISS MULOCH  \textit{J. Halifax} (ed. 17) 340 He was one of the most ‘blatant-beasts’ of the Reign of Terror.

%\itembf{2.} fig. \itembf{a.} Of persons or their words: Noisy; offensively or vulgarly clamorous; bellowing.

%\P 1656 BLOUNT  \textit{Glossogr.}, Blatant, babling, twatling.    
%\P 1674 MARVELL  \textit{Reh. Transp.} ii. 371 You are a Blatant Writer and a Labrant.    
%\P 1821 SOUTHEY  \textit{Vis. Judgem.} x. Wks. X. 223 Maledictions, and blatant tongues, and viperous hisses.    
%\P 1872 BAGEHOT  \textit{Physics \& Pol.} (1876) 92 Up rose a blatant Radical.    
%\P 1874 H. REYNOLDS  \textit{John Bapt.} viii. 515 A blatant, insolent materialism threatens to engulf moral distinctions.

%\itembf{b.} Clamorous, making itself heard.

%\P 1790 COWPER  \textit{Odyss.} vii. 267 Not the less Hear I the blatant appetite demand Due sustenance.    
%\P 1863 GEO. ELIOT  \textit{Romola} (1880) I. ii. xxix. 359 An orator who tickled the ears of the people blatant for some unknown good.    
%\P 1866 WHIPPLE  \textit{Char. \& Charac. Men} 166 All agree in a common contempt blatant or latent.    
%\P 1867 J. MACGREGOR  \textit{Voy. Alone} 65 A mass of human being whose want..misery, and filth are..patent to the eye, and blatant to the ear.

%\itembf{c.} In recent usage: obtrusive to the eye (rather than to the ear as in orig. senses); glaringly or defiantly conspicuous; palpably prominent or obvious.

%\P 1889 W. S. GILBERT  \textit{Gondoliers} ii, I write letters blatant On medicines patent.    
%\P 1903 G. GISSING  \textit{Private Papers H. Ryecroft} 274 The blatant upstart who builds a church, lays out his money in that way not merely to win social consideration.    
%\P 1912 G. B. SHAW  \textit{Let.} 19 Aug. in \textit{Shaw \& Mrs. P. Campbell} (1952) 38 You don't loathe the scenery for being prosy and mediocre in spite of its blatant picturesqueness as you do in Switzerland.    
%\P 1930 SAYERS \& EUSTACE  \textit{Documents in Case} li. 246 The blatant way in which he had marked his trail..[etc.] were actions entirely inconsistent with the carelessness of an innocent man.    
%\P 1937 H. NICOLSON  \textit{Helen's Tower} ix. 191 If they were kept in the Museum..their blatant lack of human interest had caused me to pass them by.    
%\P 1942  \textit{New Statesman} 11 July 26/1 Mankind, he said, is led by half-truths or blatant lies.    
%\P 1957 A. E. COPPARD  \textit{It's Me, O Lord!} v. 55 The colonel..clad in a suit of blatant check, spats, and a monocle.    
%\P 1957  \textit{Times} 19 Dec. 4/3 A blatant piece of late tackling.

%\itembf{3. a.} Bleating, bellowing (or merely, loud-voiced).

%\P 1791 COWPER  \textit{Iliad} xxiii 39 Many a sheep and blatant goat.    
%\P 1866 J. ROSE  \textit{Ecl. \& Georg. Virg.} 69 Rooks rejoicing, and the blatant herds.

%\itembf{b.} Noisily resonant, loud.

%\P 1816 SCOTT  \textit{Old Mort.} xiv, A blatant noise which rose behind them.    
%\P 1867 \textit{Cornh.  Mag.} Jan. 30 The vibrating and blatant powers of a hundred instruments.

%\itembf{B.} as n. One who has a blatant tongue. Obs.

%\P 1610 W. FOLKINGHAM  \textit{Art of Survey} Introd. Poem, Couch rabid Blatants, silence Surquedry.
%\end{myenumerate}


%%%%%%%%%%%%%%%%%%%%%%%%%%%%%%%%%
\myitem{strident} a. (and n.)

\noindent \phonetic{(ˈstraɪdənt)}

\noindent [ad. L. strīdentem, pr. pple. of strīd‹emacbreve›re, to creak. Cf. F. strident.]
\vspace{-0.3cm}

\begin{myenumerate}

\itembf{1. a.} Making a harsh, grating or creaking noise; loud and harsh, shrill.

\P 1656 BLOUNT  \textit{Glossogr.}, Strident, crashing or making a noise, creaking.    
\P 1721 BAILEY.    
\P 1848 THACKERAY  \textit{Van. Fair} li, ‘Brava! brava!’ old Steyne's strident voice was heard roaring over all the rest.    
\P 1860 FARRAR  \textit{Orig. Lang.} iv. 76 Strident consonants evidently formed from the hiss of certain serpents.    
\P 1875 H. JAMES  \textit{R. Hudson} xxv. (1879) III. 231 His strident accent.    
\P 1905 J. B. FIRTH  \textit{Highw. Derbysh.} xxvi. 394 The rush and rattle of strident wheels.

\itembf{b.} Phonetics. Of the articulation of a consonantal sound: characterized by friction that is comparatively turbulent. Also as n., a consonant articulated in this way.

\P 1956 JAKOBSON \& HALLE  \textit{Fundamentals of Lang.} 31 Strident/mellow: acoustically—higher intensity noise vs. lower intensity noise; genetically—rough-edged vs. smooth-edged.    Ibid. 42 Mellow constrictives, opposed to strident constrictives, or strident plosives (affricates) opposed to mellow plosives (stops proper) do not appear in child language before the emergence of the first liquid.    
\P 1965  \textit{Amer. Speech} XL. 9 T cannot follow a dental stop or S follow a strident (sibilant).    
\P 1968 CHOMSKY \& HALLE  \textit{Sound Pattern Eng.} 329 Strident sounds are marked acoustically by greater noisiness than their nonstrident counterparts.    
\P 1976 \textit{Word} 1971 XXVII.  220, s .. [and] f .. also embody the Strident vs. Mellow distinction and are both +Strident.

\itembf{2.} transf. and fig.

\P 1876 F. HARRISON  \textit{Choice Bks.} (1886) 413 All this is not to be disposed of by a somewhat strident scorn in the name of a somewhat mysterious gospel.    
\P 1907 Athenæum  25 May 641/1 The..picture..is free from the strident colour which he has sometimes fallen into of late.

\noindent Hence \textit{stridently} adv.

\P 1859 BOYD  \textit{Recreat. Country Parson} (1862) 36 There lies the large blue quarto,..there the massive foolscap,..then the ivory stridently cuts it through.    
\P a1894 STEVENSON  \textit{St. Ives} xxvi. (1908) 194 The whole enclosure continuously and stridently resounded with the rain.
\end{myenumerate}


%%%%%%%%%%%%%%%%%%%%%%%%%%%%%%%%%
\myitem{boisterous} a.

\noindent \phonetic{(ˈbɔɪstərəs)}

\noindent [Used in the same sense as the earlier boisteous, boistuous, boistous, of which it appears to be a variant modified by some obscure analogy.]
\vspace{-0.3cm}

\begin{myenumerate}

\itembf{I.} Rough or coarse in quality.

\itembf{1.} Rough, coarse, as e.g. food. Obs.

\P 1474 CAXTON  \textit{Chesse} iii. i, The labourer of the erth vseth grete and boistrous metis.

\itembf{2.} Of rough, strong, or stiff texture; stout, stiff, unyielding. Obs.

\P 1572 tr.  \textit{Buchanan's Detect. Mary} in H. Campbell \textit{Love-lett. Mary} (1824) 135 She could abide at the poop, and..handle the boisterous cables.    
\P 1577 HOLINSHED  \textit{Chron.} III. 915/1 Hauing vpon him a great gowne of boisterous veluet.    
\P 1586 WARNER  \textit{Alb. Eng.} ii. viii. (1612) 37 About his boistrous necke full oft their daintie armes they cast.    
\P 1594 T. B. \textit{La Primaud. Fr. Acad.} ii. 33 Hee hath not made the ligaments..nor the sinewes of any such boisterous or stiffe matter.    
\P 1700 DRYDEN  \textit{Sigismonda \& G.} 59 The leathern out-side, boistrous as it was, Gave way.

\itembf{3.} Roughly massive, bulky, big and cumbrous.

\P 1596 SPENSER  \textit{F.Q.} i. viii. 10 His boystrous club.    
\P 1633 J. FOSBROKE  \textit{Warre or Conflict} 30 Goliah, notwithstanding..his huge and boisterous armour, etc.    
\P 1641 R. BROOKE  \textit{Eng. Episc.} i. x. 59 The Pandects of the Civill Law are too boystrous, and of too great extent for any Civilian to comprehend.    
\P 1642 MILTON  \textit{Apol. Smect.} Wks. (1851) 292 If the work seeme more triviall or boistrous then for this discourse.

\itembf{4.} Rough to the feelings; painfully rough. Obs.

\P 1592 SHAKES.  \textit{Rom. \& Jul.} i. iv. 26 Is loue a tender thing? it is too rough, Too rude, too boysterous, and it pricks like thorne.    
\P 1595 \textit{John iv.} i. 95 Feeling what small things are boysterous there [in the eye].

\itembf{5.} Rough in operation; not skilful or delicate.

\P 1609 SIR G. PAULE  \textit{Abp. Whitgift} 28 This bishop was not so boysterous a surgeon.

\itembf{6.} Strong- or coarse-growing, rank. Obs.

\P 1622 WITHER PHILAR. in  \textit{Juv.} (1633) 590 [The pool] overgrowne with boystrous Sedge.    
\P 1671 MILTON  \textit{Samson} 1164 As Good  for nothing else, no better service, With those thy boysterous locks.

\itembf{II.} Acting roughly, violent.

\itembf{7.} Violent in action or properties. Obs.

\P 1544 PHAËR  \textit{Regim. Lyfe} (1560) N ii b, The saide venime is so swift, so fearce, and so boistrous of itselfe.    
\P 1645 MILTON  \textit{Colast.} Wks. (1851) 349 A boisterous and bestial strength.    
\P 1695 WOODWARD  \textit{Nat. Hist. Earth} vi. (1723) 294 The Heat becomes too powerful and boisterous for them.

\itembf{8.} Of wind, weather, waves, etc.: Rough, the opposite of ‘calm’.

\P 1576 THYNNE  \textit{Ld. Burghley's Crest} in \textit{Animadv. App.} iv. (1865) 113 In calme or boystrous tyde.    
\P 1596 DRAYTON  \textit{Leg.} iii. 488 The boyst'rous Seas.    
\P 1684 \textit{Contempl.  State of Man} i. ii. (1699) 20 A boystrous Wind had blown away the Leaves.    
\P 1726-7 BOLINGBROKE in  \textit{Swift's Lett.} (1766) II. lxxiii, This boisterous climate of ours.    
\P 1836 MACGILLIVRAY tr.  \textit{Humboldt's Trav.} xxi. 299 A boisterous passage of twenty-five days.    
\P 1843 PRESCOTT  \textit{Mexico} (1850) I. 194 Finding some difficulty in doubling a boisterous headland.

\itembf{9.} Of persons and their actions. a.II.9.a Full of rough violence to others, violently fierce, savage, truculent. Obs.

\P 1581 MARBECK  \textit{Bk. of Notes} 753 Those boysterous Nemrothes, that neuer will be satisfied with the slaughter of Innocents.    
\P 1593 SHAKES.  \textit{3 Hen. VI,} ii. i. 70 Oh..boyst'rous Clifford, thou hast slaine The flowre of Europe.    
\P 1681 E. SCLATER  \textit{Serm. Putney} 11 What care boisterous Enemies for what these can do unto them?    
\P 1713 POPE \textit{Frenzy J.D.} in  \textit{Swift's Wks.} (1755) III. i. 144 By your indecent and boisterous treatment of this man of learning, I perceive you are a violent sort of person.    
\P 1791 COWPER  \textit{Iliad} v. 370 Distant from the boisterous war.

\itembf{b.} Rough and violent in behaviour and speech, turbulent; too rough or clamorous. (Orig. in a distinctly bad sense, but gradually passing into c.)

\P 1568 T. HOWELL  \textit{Newe Sonets} (1879) 139 Feare not his boustrous vantinge worde.    
\P 1593 SHAKES.  \textit{Rich. II,} i. i. 4 Heere to make good ye boistrous late appeal.    
\P 1667 E. CHAMBERLAYNE  \textit{St. Gt. Brit.} i. i. iii. (1743) 8 The men are strong and boisterous, great wrestlers, and healthy.    
\P 1690 CROWNE  \textit{Eng. Frier} i. i. 3 Pox o' this boystrous fool.    
\P 1705 OTWAY  \textit{Orphan} v. xix. 2296 Stand off thou hot-brain'd boistrous noisy Ruffian.    
\P 1853 MARSDEN  \textit{Early Purit.} 55 Every form of church government..had for awhile its boisterous advocates.

\itembf{c.} Abounding in rough but good-natured activity bordering upon excess, such as proceeds from unchecked exuberance of spirits.

\P a1683 SIDNEY  \textit{Disc. Gov.} iii. §25 (1704) 334 That boisterous humor being gradually temper'd by disciplin.    
\P 1709 STEELE  \textit{Tatler} No. 45 \phonetic{⁋}8 Their boisterous Mirth.    
\P 1752 HUME  \textit{Ess. \& Treat.} (1777) I. 5 It renders the mind incapable of the rougher and more boisterous emotions.    
\P 1822 W. IRVING  \textit{Braceb. Hall} xix. 167 A rich, boisterous, foxhunting baronet.    
\P 1848 MACAULAY  \textit{Hist. Eng.} I. 213 Under the outward show of boisterous frankness.

\itembf{10.} quasi-adv. Boisterously. Obs.

\P 1595 SHAKES.  \textit{John} iv. i. 76 Alas, what neede you be so boistrous rough?
\end{myenumerate}


%%%%%%%%%%%%%%%%%%%%%%%%%%%%%%%%
\myitem{obstreperous} a.

\noindent \phonetic{(əbˈstrɛpərəs)}

\noindent [f. L. obstreper-us clamorous (f. obstrep-ĕre to make a noise against, shout at, oppose noisily or troublesomely) + -ous.]
\vspace{-0.3cm}

\begin{myenumerate}

\itembf{1.} Characterized by great noise or outcry, esp. in opposition; clamorous, noisy; vociferous.

Quot. 1922 is ellipt.  for ‘obstreperous mouth’.

\P c1600 \textit{Timon} i. ii. (1842) 6 Proceed'st thou still with thy ostreperous noyse.    
\P 1603 B. JONSON  \textit{Sejanus} v. iii, They [ravens] sate all night, Beating the ayre with their obstreperous beakes.    
\P a1661 FULLER  \textit{Worthies} (1840) II. 211 He..was very obstreperous in arguing the case for transubstantiation.    
\P 1748 SMOLLETT  \textit{Rod. Rand.} viii. (1804) 41, I heard him very obstropulous in his sleep.    
\P 1751 JOHNSON  \textit{Rambler} No. 89 \phonetic{⁋}11 The most careless and obstreperous merriment.    
\P 1824 J. WIGHT  \textit{Mornings at Bow St.} 155 They were forthwith conveyed to the watch-house, and there they conducted themselves so ‘obstropolously’, that the constable of the night found it necessary to have them put down below.    
\P 1856 R. A. VAUGHAN  \textit{Mystics} (1860) II. 51 The obstreperous rhetoricians will plague me with their big words.    
\P 1875 EMERSON  \textit{Lett. \& Soc. Aims} v. 131 Obstreperous roarings of the throat.    
\P 1922 JOYCE  \textit{Ulysses} 420 Hark! Shut your obstropolos.

\itembf{2.} Resisting control, management, advice, etc., in a noisy manner; turbulent or unruly in behaviour, esp. in resistance.

\P 1657 [see OBSTREPEROUSNESS].    
\P 1727 PHILIP  \textit{Quarll} 105 Fearing she would grow obstrepulous, they each of 'em took hold of one of her Arms.    
\P 1773 GOLDSM.  \textit{Stoops to Conq.} iii, I'm sure you did not treat Miss Hardcastle..in this obstropalous manner.    
\P 1806 T. S. SURR  \textit{Winter in Lond.} (ed. 3) III. 5 You have been quite obstropulous; no getting any food into your mouth but by force.    
\P 1827 SCOTT  \textit{Diary} 2 Oct. in Lockhart, We dined at Wooler, where an obstreperous horse retarded us for an hour at least.    
\P 1874 BURNAND  \textit{My time} i. 4 Generally having my own way..and becoming remarkably obstreperous when thwarted.    
\P 1881 \textit{Macm.  Mag.} Nov. 40 The most obstreperous and unmanageable of all young merlins.
\end{myenumerate}

%%%%%%%%%%%%%%%%%%%%%%%%%%%%%%%%
\myitem{whimsical} a. (n.)

\noindent \phonetic{(ˈhwɪmzɪkəl)}

\noindent [f. WHIMSY + -ICAL.]
\vspace{-0.3cm}

\begin{myenumerate}

\itembf{1.} Of persons, their actions, thoughts, etc.: Full of, subject to, or characterized by a whim or whims; actuated by or depending upon whim or caprice.

\P 1653 W. RAMESEY  \textit{Astrol. Rest.} To Rdr. 10 So they fell to words and at last (to end this Whimsical controversie) they resolved to kill one another.    Ibid. 11 Were not they better be..grave, sober, serious, then whymiscal, fickle and fantastical?    
\P 1690 C. NESSE  \textit{O. \& N. Test.} I. 251 So do the whimsical Enthusiasts..make long relations of strange dreams.    
\P 1703 EARL OF ORRERY  \textit{As you find it} iii. i. 35 A Man with a fantastical, whimsical Stomach may starve in the midst of Plenty, not for want of Food, but such as he likes.    
\P 1711 ADDISON  \textit{Spect.} No. 101 \phonetic{⁋}7 One Sir Roger de Coverley, a whimsical Country Knight.    
\P 1756 BURKE  \textit{Subl. \& Beaut.} iii. xi. (1759) 208 It has given rise to an infinite deal of whimsical theory.    
\P 1809 MALKIN  \textit{Gil Blas} iv. vii. \phonetic{⁋}2 One of those old codgers who have been a little whimsical or so in their youth.    
\P 1839 HALLAM  \textit{Lit. Eur.} ii. vii. §20 It would be rather whimsical to deny this to be a principal merit in a comparison.    
\P 1875 J. E. T.  ROGERS \textit{Protests of Lords} I. Pref. p. lvi, Two whimsical dissents from Lords Radnor and Abingdon.

\itembf{2.} Characterized by deviation from the ordinary as if determined by mere caprice; fantastic, fanciful; freakish, odd, comical.

\P 1675 E. WILSON  \textit{Spadacr. Dunelm.} Pref. B 5 b, Panacæa's, Universal Medicines, Secrets, and such like whimsical Remedies.    
\P 1687 T. BROWN SAINTS in  \textit{Uproar} Wks. 1730 I. 79 The  most whimsical scene of the farce is still behind.    
\P a1700 EVELYN  \textit{Diary} 29 Nov. 1644, A whimsical chayre, which folded into so many varieties as to turn into a bed, a bolster, a table, or a couch.    
\P 1710 SWIFT  \textit{Lett.} (1767) III. 57 Is it not whimsical that the dean has never once written to me?    
\P 1769 BURKE  \textit{Corr.} (1844) I. 165 Matters here are in a situation whimsical enough.    
\P 1773 WESLEY  \textit{Jrnl.} 29 Nov., Wks. 1830 IV. 5, I went..to Sheerness; over that whimsical ferry, where footmen and horses pay nothing.    
\P 1826 F. REYNOLDS  \textit{Life \& Times} I. 193 The Germans are whimsical animals in their appearance.    
\P 1836 BRANDE  \textit{Chem.} (ed. 4) 17 Alembics, stills, retorts, receivers, and a variety of whimsical and complex vessels.    
\P 1852 MRS. STOWE  \textit{Uncle Tom's C.} ix. 66 Our senator..looked after his little wife with a whimsical mixture of amusement and vexation.    
\P 1890 \textit{Science-Gossip} XXVI. 85 All these whimsical prescriptions gradually fell out of the Pharmacopœias.

\noindent absol.\P 1740 CIBBER  \textit{Apol.} (1756) I. 112 Who..delighted more in the whimsical than the natural.    
\P 1838 DICKENS  \textit{Nich. Nick.} xxiv, Hesitating between the respect he ought to assume, and his love of the whimsical.

\itembf{b.} Subject to uncertainty or the ‘caprice of fortune’. Obs.

\P 1654 WHITLOCK  \textit{Zootomia} 151 Must the bread of Life be ground only by the winde of every Doctrine? and whimsicall Wind-Mills?    
\P 1700 CONGREVE  \textit{Way of World} ii. vii, A Fellow that lives in a Windmill has not a more whimsical Dwelling than the Heart of a Man that is lodg'd in a Woman. There is no Point of the Compass to which they cannot turn.    
\P 1716 ADDISON  \textit{Freeholder} No. 18 \phonetic{⁋}3, I shall only take notice of the whimsical circumstances a people must lie under, who can be thus made poor or rich by an edict.    
\P 1748 RICHARDSON  \textit{Clarissa} (1768) III. 191 Poor man! he stands a whimsical chance between us.

\itembf{B.} n. (in pl.) A cant name for a section of the Tories in the reign of Queen Anne: see quots.

\P 1714 SWIFT  \textit{Pres. St. Aff.} Wks. 1841 I. 492/2  That race of politicians, who in the cant phrase are called the whimsicals.    
\P 1818 SCOTT  \textit{Br. Lamm.} xxvii, Many of the High Church party..affected to separate their principles from those of the Jacobites, and, on that account, obtained the denomination of Whimsicals.
\end{myenumerate}


%%%%%%%%%%%%%%%%%%%%%%%%%%%%%%%%
\myitem{quaint} a. (adv.)

\noindent \phonetic{(kweɪnt)}

\noindent [a. OF. cointe (quointe, cuinte, etc.), queinte:—L. cognitum known, pa. pple. of cognoscĕre to ascertain. The development of the main senses took place in OF., and is not free from obscurity (cf., however, COUTH and KNOWN).

 In its older senses the Eng. word seems to have been in ordinary use down to the 17th c., though in many 16–17th c. examples the exact meaning is difficult to determine. After 1700 it occurs more sparingly (chiefly in sense 6), until its revival in sense 8, which is very frequent after 1800.] 

\vspace{-0.3cm}

\begin{myenumerate}
\itembf{A.} adj. \textbf{I. 1.} Of persons: Wise, knowing; skilled, clever, ingenious. In later use chiefly with ref. to the employment of fine language (cf. sense 6). Obs.

\P a1250  \textit{Leg. Kath.} 580 (Cott. MS.) Hei! hwuch wis read Of se cointe [v.r. icudd] keiser.    
\P c1290 \textit{S. Eng.  Leg.} I. 381/165 Þe beste Carpenter And þe quoynteste þat ich euere i-knev.    
\P a1325  \textit{Prose Psalter} cxviii. 98 Thou madest me quainte [L. prudentem] vp myn enemis to þi comaundement.    
\P c1400  \textit{Destr. Troy} 1531 Wise  wrightis to wale..qwaint men of wit.    
\P 1501 DOUGLAS  \textit{Pal. Hon.} i. lxv, Ȝit clerkis bene in subtell wordis quent, And in the deid als schairp as ony snaillis.    
\P 1593 SHAKES.  \textit{2 Hen. VI,} iii. ii. 274 To shew how queint an Orator you are.    
\P 1596   \textit{Tam. Shr.} iii. ii. 149 Wee'll ouerreach..The quaint Musician.    
\P a1628 PRESTON  \textit{New Covt.} (1634) 273 If you would preach as other men do, and be curious and quaint of Oratory.    
\P 1697 DRYDEN  \textit{Æneid} xi. 698 Talk on ye quaint Haranguers of the Crowd.    
\P 1728 MORGAN  \textit{Algiers} I. vi. 176 The Arabs in general are quaint, bold, hospitable, and generous, excessive Lovers of Eloquence and Poesy.

\itembf{b.} In bad sense: Cunning, crafty, given to scheming or plotting. Obs.

\P a1225  \textit{Ancr. R.} 328 Þeos kointe harloz þet scheaweð forð hore gutefestre.    
\P c1340  \textit{Cursor M.} 739 (Fairf.) Þe nedder þat ys so quaynt of gyle.    
\P c1394 \textit{P. Pl.  Crede} 482 ‘Dere broþer’ quaþ Peres ‘þe devell is ful queynte’.    
\P 1402 HOCCLEVE  \textit{Letter of Cupid} 152 Sly, queynt, and fals in al vnthrift coupable.    
\P 1513 DOUGLAS  \textit{Æneis} ii. i. 59 Knaw ȝe nocht bettir the quent Vlexes slycht?    
\P 1674-91 RAY  \textit{N.-C. Words} (E.D.S.), ‘A wheint lad’, q. queint; a find lad: ironice dictum. Also, cunning, subtle.    
\P 1680 OTWAY  \textit{Orphan} iii. iv. 864 The quaint smooth Rogue, that sins against his Reason.

\itembf{2.} Of actions, schemes, devices, etc.: Marked by ingenuity, cleverness, or cunning. Now arch.

\P a1225  \textit{Ancr. R.} 294 Ure Louerd..brouhte so to grunde his kointe kuluertschipe.    
\P c1330 \textit{Arth. \& Merl.}  4447 (Kölbing) Morgein..þat wiþ hir queint gin Bigiled þe gode clerk Merlin.    
\P 1387 TREVISA  \textit{Higden} (Rolls) IV. 429 Iosephus..fonde up a queynte craft, and heng wete cloþes uppon þe toun walles.    
\P c1460 TOWNELEY  \textit{Myst.} xiii. 593 This was a qwantte gawde, and a far cast, It was a hee frawde.    
\P 1522  \textit{World \& Child} in Hazl. \textit{Dodsley} I. 245, I can many a quaint game.    
\P 1598 ROWLANDS  \textit{Betray. Christ} 10 When traitor meets, these quaint deceits he had.    
\P 1641  BROME \textit{Jovial Crew} ii. Wks. 1873 III.  378, I..over-heard you in your queint designe, to new create your selves.    
\P 1742 W. SHENSTONE  \textit{Schoolmistress} xii, With quaint arts the giddy crowd she sways.    
\P 1889 ‘MARK  TWAIN’ \textit{Yankee} iv. 37 This quaint lie was most simply and beautifully told.    
\P 1970 C. HAMPTON  \textit{Philanthropist} i. 13 John puts the revolver into his mouth and presses the trigger. Loud explosion. By some quaint device, gobs of brain and bright blood appear on the whitewashed wall.

\itembf{3.} Of things: Ingeniously or cunningly designed or contrived; made with skill or art; elaborate. Obs.

\P c1290 \textit{S. Eng.  Leg.} I. 88/62 He liet heom makien a quoynte schip.    
\P 1297 R. GLOUC. (Rolls) 1555 Hii  ȝeue him an quointe [v.r. koynte] drench, mid childe vor to be.    
\P c1384 CHAUCER  \textit{H. Fame} iii. 835 And evermo..This queynte hous aboute wente, That never-mo hit stille stente.    
\P a1400-50  \textit{Alexander} 4275 Have we no cures of courte ne na cointe sewes.    
\P 1627 DRAYTON  \textit{Nymphidia} lxix, He told the arming of each joint, In every piece how neat and quaint.    
\P 1631 SHIRLEY  \textit{Traitor} iv. ii, Who knows But he may marry her, and discharge his Duchess With a quaint salad?

\itembf{4.} Of things: Skilfully made, so as to have a good appearance; hence, beautiful, pretty, fine, dainty. Obs.

\P 13..  \textit{E.E. Allit. P.} B. 1382 With  koynt carneles aboue, coruen ful clene.    
\P 13..  \textit{Gaw. \& Gr. Knt.} 877 Whyssynes vpon queldepoyntes, þat koynt wer boþe.    ?
\P ?a1366 CHAUCER  \textit{Rom. Rose} 98 A sylvre nedle forth I droughe, Out of an aguler queynt ynoughe.    
\P c1400  \textit{Destr. Troy} 777 An ymage full nobill..þat qwaint was \& qwem, all of white siluer.    
\P 1596 SPENSER  \textit{F.Q.} iv. x. 22 Nor hart could wish for any queint device, But there it present was, and did fraile sense entice.    
\P 1671 MILTON  \textit{Samson} 1303 In his  hand A Scepter or quaint staff he bears.

\itembf{b.} Of dress: Fine, fashionable, elegant. Obs.

\P ?a1366 CHAUCER  \textit{Rom. Rose} 65 The ground..maketh so queynt his robe and fayr That it hath hewes an hundred payr.    
\P 1380 \textit{Lay  Folks Catech.} (Lamb. MS.) 1221 Ne worschipe  not men for here fayre cloþes, ne for here qweynte schappis þat sum men usen.    
\P 1501 DOUGLAS  \textit{Pal. Hon.} i. xlvi, In vestures quent of mony sindrie gyse.    
\P 1592 GREENE \textit{Upst. Courtier} in  \textit{Harl. Misc.} (Malh.) II. 223 Costly attire, curious and quaint apparell is the spur that prickes them forward.    
\P 1627 FLETCHER  \textit{Locusts} i. xiii, All lovely drest In beauties livery, and quaint devise.

\textit{5.} Of persons: Beautiful or handsome in appearance; finely or fashionably dressed; elegant, foppish. Obs.

\P a1300  \textit{Cursor M.} 28015 Yee leuedis..studis..hu to mak yow semle and quaint.    
\P a1310 in  \textit{Wright Lyric} P. 26 Coynte ase columbine, such hire cunde ys.    
\P 1362 LANGL.  \textit{P. Pl.} A. ii. 14 A wommon wonderliche clothed..Ther nis no qweene qweyntore.    
\P a1450  \textit{Knt. de la Tour} (1868) 40 Folke shulde not have thaire herte on the worlde, nor make hem queint, to plese it.    
\P 1590 GREENE  \textit{Never Too Late} Wks. 1882 VIII.  82 He made himselfe as neate and quaint as might be.    
\P 1598 SHAKES.  \textit{Merry W.} iv. vi. 41 Quaint in greene, she shall be loose en-roab'd.    
\P 1610   \textit{Temp.} i. ii. 317 Fine apparision: my queint Ariel, Hearke in thine eare.    
\P 1784 COWPER  \textit{Task} ii. 461 A body so fantastic, trim, And queint in its deportment and attire.

\itembf{6.} Of speech, language, modes of expression, etc.: Carefully or ingeniously elaborated; highly elegant or refined; clever, smart; full of fancies or conceits; affected. Obs. (now merged in 8).

\P 13.. GUY  \textit{Warw.} (A.) 346 To hir he spac..Wiþ a wel queynt steuen.    
\P c1386 CHAUCER  \textit{Can. Yeom. Prol. \& T.} 199 We semen wonder wise, Oure termes been so clergial and so queynte.    
\P 1513 DOUGLAS  \textit{Æneis} i. Prol. 255 The quent and curious castis poeticall.    
\P c1570  \textit{Pride \& Lowl.} (1841) 807 Pleasaunt songes..To queynt and hard for me to understand.    
\P 1655 E. TERRY  \textit{Voy. E. Ind.} XII. 232 The Persian there is spoken as their more quaint and Court-tongue.    
\P 1676 MARVELL  \textit{Mr. Smirke} K iv, A good life is a Clergy man's best Syllogism, and the quaintest Oratory.    
\P 1712 STEELE  \textit{Spect.} No. 450 \phonetic{⁋}1 A new Thought or Conceit dressed up in smooth quaint Language.    
\P 1783 BURKE  \textit{Rep. Aff. India} Wks. 1842 II. 76 A  style,..full of quaint terms and idiomatick phrases, which strongly bespeak English habits in the way of thinking.

\itembf{7.} Strange, unusual, unfamiliar, odd, curious (in character or appearance). Obs. (now merged in 8).

\P 13.. \textit{Coer  de L.} 216 Thou schalt se a queynte brayd.    
\P c1369 CHAUCER \textit{Dethe Blaunche} 1330 This  is so queynt a sweuyn.    
\P c1400  \textit{Destr. Troy} 7715 There come with this kyng a coynt mon of shappe.    
\P c1440 \textit{Ipomydon} 1637 Right  vnsemely on queynte manere He hym dight.    
\P 1513 DOUGLAS  \textit{Æneis} iii. Prol. 12 Now moist I write..Wyld auentouris, monstreis and qwent affrayis.    
\P 1579 SPENSER  \textit{Sheph. Cal.} Oct. 114 With queint Bellona in her equipage.    
\P 1629 MILTON  \textit{Nativity} 194 A drear, and dying sound Affrights the Flamins at their service quaint.    
\P 1714 POPE  \textit{Wife of Bath} 259 How quaint an appetite in woman reigns! Free gifts we scorn, and love what costs us pains.    
\P 1808 SCOTT  \textit{Marm.} iii. xx, Came forth—a quaint and fearful sight.

\itembf{8.} Unusual or uncommon in character or appearance, but at the same time having some attractive or agreeable feature, esp., having an old-fashioned prettiness or daintiness.

\P 1795 SOUTHEY  \textit{Joan of Arc} viii. 234 He for the wintry hour Knew many a merry ballad and quaint tale.    
\P 1808 SCOTT  \textit{Marm.} ii. iii, For this, with carving rare and quaint, She decked the chapel of the saint.    
\P 1824 W. IRVING  \textit{T. Trav.} I. 91 The streaks of light and shadow thrown among the quaint articles of furniture.    
\P 1862 STANLEY  \textit{Jew. Ch.} (1877) I. x. 202 The device is full of a quaint humour which marks its antiquity.    
\P 1884 J. T. BENT in  \textit{Macm. Mag.} Oct. 434/2 The herdsmen were much quainter and more entertaining than our city-born muleteers.

\itembf{b.} Of furniture: designed in the style of art nouveau.

\P 1897 \textit{Furnit. \& Decoration}  XXXIV. 197/1 That new style called ‘Quaint’, which seems to be carcase without the spirit of the new style promulgated by the Arts and Crafts and other societies.    
\P 1952 J. GLOAG  \textit{Short Dict. Furnit.} 377 A fashion in furniture design, corresponding with the New Art movement at the end of the 19th and the opening of the present century, was known as the quaint style.    
\P 1975 \textit{Country  Life} 2 Oct. 852/3 The spindly chairs and tables of the ‘quaint’ vogue.

\itembf{II. 9.} Proud, haughty. Obs. rare.

\P a1225  \textit{Ancr. R.} 140 Þet fleshs is her et home..ant for þui hit is cwointe \& cwiuer.    
\P 1340  \textit{Ayenb.} 89 Þo þet makeþ ham zuo quaynte of þe ilke poure noblesse þet hi habbeþ of hare moder þe erþe.    
\P c1430  \textit{Pilgr. Lyf Manhode} ii. cvii. 115, I hatte orgoill, the queynte [F. la bobanciere], the feerce hornede beste.    [
\P 1610 G. FLETCHER  \textit{Christ's Vict.} ii. liv, Queint Pride Hath taught her sonnes to wound their mother's side.]

\itembf{10.} Dainty, fastidious, nice; prim. Obs.

\P 1483 CAXTON  \textit{Gold. Leg.} 128 b/1 She chastyssed them that were nyce and queynte.    
\P 1579 G. HARVEY  \textit{Letter-bk.} (Camden) 73 The rest in a manner ar..overstale for so queynte and queasye a worlde.    
\P 1590 SPENSER  \textit{F.Q.} iii. vii. 10 She nothing quaint Nor 'sdeignfull of so homely fashion.    
\P 1640  \textit{Brome Sparagus Gard.} iii. vii. Wks. 1873 III.  167 Your new infusion of pure blood, by your queint feeding on delicate meates and drinks.    
\P 1678 R. L'ESTRANGE  \textit{Seneca's Mor.} To Rdr., Fabius..taxes him..for being too Queint and Finical in his Expressions.

\itembf{11.} \textit{to make it quaint}, to act quaintly, in various senses, esp. to behave proudly, disdainfully, or deceitfully. Obs.

\P c1369 CHAUCER  \textit{Dethe Blaunche} 531 Lo! how goodly spak this knight..He made hyt nouther tough ne queynte.    
\P 1390 GOWER  \textit{Conf.} v. 4623 (II. 282) O traiteresse..Thou hast gret peine wel deserved, That thou canst maken it so queinte.    
\P c1400  \textit{Rom. Rose} 2038, I..kneled doun with hondis Ioynt, And made it in my port ful queynt.    
\P c1422 HOCCLEVE  \textit{Jonathas} 642 He thoghte not to make it qweynte and tow.    
\P c1430  \textit{Pilgr. Lyf Manhode} ii. cvi. (1869) 115 With alle myne joyntes stiryinge and with alle my sinewes j make it queynte [F. je marche si fierement.]

\itembf{B.} adv. Skilfully, cunningly. Obs. rare.

\P c1340  \textit{Cursor M.} 5511 (Fairf.) Ȝou be-houys to wirke ful quaynte and in þaire dedis ham attaynt.    
\P c1384 CHAUCER  \textit{H. Fame} i. 245 What shulde I speke more queynte, Or peyne me my wordes peynte?    
\P 1552 LYNDESAY  \textit{Monarche} 180 Fresche flora spred furth hir tapestrie, Wrocht be dame Nature quent and curiouslie.

\itembf{C.} Comb., as \textit{quaint-carved, quaint-eyed, quaint-felt, quaint-looking, quaint-mouthed, quaint-shaped, quaint-sounding, quaint-stomached, quaint-witty, quaint-worded} adjs.

\P 1575 G. HARVEY  \textit{Letter-bk.} (Camden) 91 Thou arte so queyntefelt In thy rondelett.    
\P 1598 MARSTON  \textit{Pygmal.} i. 140 Like no quaint stomack't man [he] Eates vp his armes.    
\P 1603 FLORIO  \textit{Montaigne} i. xxxvi. (1632) 115 A quaint-wittie, and loftie conceit.    
\P 1744 AKENSIDE  \textit{Pleas. Imag.} iii. 250 Where'er the pow'r of ridicule displays Her quaint-ey'd visage.    
\P 1838 J. R. LOWELL  \textit{Class Poem} ix. 11 What quaint-mouthed sentences! and how profound!    
\P 1853 JAMES  \textit{Agnes Sorel} (1860) I. 2 This tall quaint-shaped window.    
\P 1859 J. G. WHITTIER  \textit{On Prayer Bk.} in \textit{Independent} (N.Y.) 15 Sept. 1/1 The quaint-carved, Gothic door.    
\P 1863 GROSART  \textit{Small Sins} (ed. 2) 17 Their quaint-worded dispositions and distinctions.    
\P 1922 R. LEIGHTON  \textit{Compl. Bk. Dog} xii. 178 Most people are well acquainted with the personal appearance of this quaint-looking dog.    
\P 1957 A. N. PRIOR  \textit{Time \& Modality} 55 ‘The True’ and ‘The False’ are certainly quaint-sounding objects to be named by phrases like ‘The conquest of Gaul by Caesar’.
\end{myenumerate}

%%%%%%%%%%%%%%%%%%%%%%%%%%%%%%%%
\myitem{frivolous} a.

\noindent \phonetic{(ˈfrɪvələs)}

\noindent [f. L. frīvol-us + -ous. Cf. frivol a.]
\vspace{-0.3cm}

\begin{myenumerate}

\itembf{1.} Of little or no weight, value, or importance; paltry, trumpery; not worthy of serious attention; having no reasonable ground or purpose.

\P 1549 BALE  \textit{Leland's N.Y. Gift} D iv, We fynde for true hystoryes, most fryuolouse fables and lyes.    
\P 1578 TIMME  \textit{Caluine on Gen.} 25 It is too frivolous and vaine to expound this worde.    
\P 1624 LD. KENSINGTON in Ellis  \textit{Orig. Lett.} Ser. i. III. 172 In their frivolous delayes, and in the unreasonable conditions which they propounded.    
\P 1648 GAGE  \textit{West Ind.} xx. 169 His answers seeming frivolous.    
\P c1670 WOOD  \textit{Life} (O.H.S.) I. 398 The warden..did put the college to unnecessary charges, and very frivolous expences.    
\P 1770 \textit{Junius'  Lett.} xxxix. 198 They voted his information frivolous.    
\P 1776 ADAM SMITH  \textit{W.N.} i. xi. (1869) I. 184 The other frivolous ornaments of dress and furniture.    
\P 1828 SCOTT  \textit{F.M. Perth} vii, The slight and frivolous complaints unnecessarily brought before him.    
\P 1871 DIXON  \textit{Tower} III. xxv. 280 He was arrested on a frivolous charge.

\itembf{b.} Law. In pleading: Manifestly insufficient or futile.

\P 1736 in  \textit{Swift's Lett.} (1766) II. 249 The decree was affirmed most unanimously, the appeal adjudged frivolous.    
\P 1883 SIR H. COTTON in  \textit{Law Rep.} 11 Q. Bench Div. 532 Unless the counter-claim is frivolous and unsubstantial.

\itembf{2.} Characterized by lack of seriousness, sense, or reverence; given to trifling, silly.

\P 1560 tr.  \textit{Fisher's Treat. Prayer} F ij, Eschewyng all vayne, friuolus, and vnfruitfull thoughtes.    
\P 1575 G. HARVEY  \textit{Letter-bk.} (Camden) 101 Frivolous boyishe grammer schole trickes.    
\P 1687 WOOD  \textit{Life} 21 Apr., The duke of Bucks is dead..many frivolous things extant—‘Bays’, a comedy.    
\P 1711 STEELE  \textit{Spect.} No. 156 \phonetic{⁋}6 From reading frivolous Books, and keeping as frivolous Company.    
\P 1783 JOHNSON 18 APR. in  \textit{Boswell}, He may be a frivolous man, and be so much occupied with petty pursuits, that he may not want friends.    
\P 1862 M. E. BRADDON  \textit{Lady Audley} ix. 63 Lady Audley amused herself in her own frivolous fashion.

\noindent absol. \P 1836 EMERSON  \textit{Nat., Idealism} Wks. (Bohn) II. 160 The frivolous make themselves merry with the Ideal theory, as if its consequences were burlesque.

\noindent Hence \textit{frivolously} adv., \textit{frivolousness}.

\P 1611 COTGR.,  \textit{Vainement}, vainely, friuolously, to no purpose.    
\P 1624 DONNE  \textit{Serm.} (Alford) V. cxxx. 330 If Abraham had any such doubts, of a Frivolousness in so base a Seal.    
\P 1712 STEELE  \textit{Spect.} No. 448 \phonetic{⁋}2 The frivolously false ones.    
\P 1768-74 TUCKER  \textit{Lt. Nat.} (1852) I. 119 To..judge of the weight or frivolousness of objections.    
\P 1812 G. CHALMERS  \textit{Dom. Econ. Gt. Brit.} 396 This argument..has been found to have, at least, the pertinacity of faction, if it have not the frivolousness of folly.    
\P 1885 LD. BLACKBURN in  \textit{Law Rep.} 10 Appeal Cases 223 The bankrupt being held to be acting frivolously and vexatiously.
\end{myenumerate}




\end{description}

