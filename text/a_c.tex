\chapter*{A}
%\markboth{VOCABULARY STUDY}{}
\markright{OED: A}{}
\addcontentsline{toc}{chapter}{OED: A}%
%\chapter[Oxford English Dictionary]{Vocabulary Study of Oxford English Dictionary}

%\setitemize{nosep}  % set no itemsep for itemize lists
%\leftmargini=7mm  % controls the relative spacing of description list

%%%%%%%%%%%%%%%%%%%%%%%%%%%%%%%%%%%%%%%%%%%%%%%%%%%%%%%%%%%%%%%%%%
\noindent The list of words is from Schur, \textit{1000 Most Important Words.}

%\begin{description}
\begin{description}[wide, labelwidth=!, labelindent=0pt] % noindent


%%%%%%%%%%%%%%%%%%%%%%%%%%%%%%%
%%\myitem{ }  
%%\noindent  \phonetic{}
%%\noindent  
%%\vspace{-0.3cm}
%%
%%\begin{myenumerate}
%%\itembf{ }
%%\end{myenumerate}


%%%%%%%%%%%%%%%%%%%%%%%%%%%%%%%
\myitem{abash } v. 

\noindent \phonetic{(əˈbæʃ)}

\noindent  
[ad. Anglo-Fr. abaïss- = OFr. ebaïss-, esbaïss-, lengthened stem (occurring in
pple. abaïss-ant, 3 pl. abaïss-ent, subj. abaïsse, etc.) of ésb-aïr, mod.Fr.
ébahir; f. es:—Lat. ex ‘out, utterly’ + baïr, bahir = Ital. baïre to astound,
regarded as formed on bah! a natural exclamation of astonishment. The OFr. -iss
here became -ish, as in perish, finish, punish, and the i was absorbed, as in
punch; in the north the -s remained, as in cheriss, fluriss, punyss; hence a
formal confusion between northern forms of abash, and the distinct vb. abase,
q.v.] 
\vspace{-0.3cm}

\begin{myenumerate}

\itembf{1.}
To destroy the self-possession or confidence of (any one), to put out of
countenance, confound, discomfit, or check with a sudden consciousness of shame,
presumption, error, or the like. \textbf{a.} active. 

%\begin{addmargin}[1em]{1em}
1375 BARBOUR \textit{Bruce} viii. 247 And thouch that thai be ma than we, That suld
abaiss ws litill thing.    
\P
1430 \textit{Pilg. Lyf of Man} 117 It is thilke bi whiche I
abashe alle the bestes of the cuntre.    
\P
1496 W. DE WORDE \textit{Dives \& Pauper} xiv. viii. 340/1 The lyon with his crye 
abassheth all other bestes.    
\P
1570 LEVINS \textit{Manipulus}, To Abashe Stupefacere.    
\P
1574 tr. \textit{Marlorats Apocalips} 26 For although lightning be bright, yet is 
it not chærefull, but rather abasheth men.
\P
1600 HEYWOOD \textit{1st Edw. IV}, iv. 27 To weaken and abash their fortitude.    
\P
1751 FIELDING \textit{Amelia} iii. ix. Wks. 1784 VIII. 304 A man whom no denial, no scorn
could abash.    
\P
1863 H. ROGERS \textit{Life of J. Howe} iii. 83 If not to convince, to
silence and abash the gainsayer.
%\end{addmargin}

\itembf{b.} refl. [mod.Fr. has only the refl. form s'ébahir.] To gape with surprise,
to stand confounded. Obs. 

c1450 LONELICH \textit{Holy Grail} xxi. 291 Thanne the Kyng Abasched him sore For þe
wordes he herde thore.    
\P
1485 CAXTON \textit{Paris \& Vienne} 62 Abasshe you not for thys derkenes.

\itembf{c.}
Most common in the pass.: to be, stand, or feel abashed; at an occasion,
of (obs.), by a cause. 

c1325 \textit{E.E. Allit. P.} 42. 149 Þat oþer burne watȝ abayst of his broþe wordeȝ.
\P
1366 MANDEVILLE xxix. 295 Alisandre was gretly astoneyed and abayst.    
\P
1382 WYCLIF \textit{Mark} v. 42 And thei weren abaischt [1388 abaischid] with greet stoneying.
\P
c1386 CHAUCER \textit{Clerk's T.} 955 Right nought was sche abaissht of her clothing
[v.r. abayst—2, abast, abayssht, abasshed, abassched].    
\P
1483 CAXTON \textit{G. Leg.} 70/3 Whan Dauid herd this he was sore abasshed.    
\P
1535 COVERDALE \textit{Is.} xiii. 8 One shall euer be abaszshed of another.    
\P
1667 MILTON \textit{P.L.} i. 331 They heard, and were abasht, and up they sprung.    
\P
1807 CRABBE \textit{Village} ii. 79 And while she stands abash'd, with conscious eye.    
\P
1876 GLADSTONE \textit{Homeric Synch.} 72 I might have been abashed by their authority.

\itembf{2.} intr. (by omission of refl. pron.) To stand dumb with confusion or
astonishment; to lose self-possession or confidence; to flinch or recoil with
surprise, shame, or sense of humiliation. Obs. 

1391 CHAUCER \textit{Boeth.} 146 (1868) No strong man ne semeþ nat to abassen or
disdaigner as ofte tyme as he hereþ þe noise of þe bataile.    
\P
1477 CAXTON \textit{Jason} 45 b, The herte of man sholde not abasshe in no thing.    
\P
1530 PALSGR. I abasshe, or am amazed of any thing, Je me esbahis.    
\P
1577–87 HOLINSHED \textit{Chron.} III. 1098/2 For she, notwithstanding all the 
fearefull newes that were brought to hir that daie, neuer abashed.    
\P
1585 JAMES I \textit{Essayes in Poesie} 44 She did shame The Sunne himself, her 
coulour was so bright, Till he abashit beholding such a light.
\end{myenumerate}

%%%%%%%%%%%%%%%%%%%%%%%%%%%%%%%
\myitem{abashed} ppl. a.  

\noindent \phonetic{ (əˈbæʃt) }

\noindent [abash v. + -ed.] 

\noindent
Put out of self-possession, stricken with surprise; confounded, discomfited,
disconcerted; checked with a sense of shame, presumption, or error. 
%\vspace{10pt}

1340 Hampole \textit{Pr. Consc. 1431} Swa þat man suld mare drede and be 
abayste, Over mykel in þe world here to trayste.    
\P 1534 Ld. Berners \textit{Golden Bk. of Marc. Aurel.} (1546) O iiii b, We 
holdyng downe our heddes abashed.    
\P 1718 Pope \textit{Iliad} viii. 540 The pensive goddesses, abash'd, controll'd.    
\P 1859 Tennyson \textit{Enid} 765 Enid, all abash'd she knew not why, Dared not to 
glance at her good mother's face.


%%%%%%%%%%%%%%%%%%%%%%%%%%%%%%%
\myitem{abate } v.  


\noindent \phonetic{(əˈbeɪt)}

\noindent  
[a. OFr. abat-re, abat-tre, f. à prep. to + batre, battre to beat:—late 
L. battĕre, batĕre, from cl. L. batuĕre. In the technical senses 18, 
19, the identity of the prefix is uncertain, and the relation to the 
other senses undetermined.] 
\vspace{-0.3cm}

\begin{myenumerate}
\itembf{I.} To beat down, demolish, destroy. 
\itembf{1.} trans. To beat down, throw down, demolish, level with the 
ground. Obs. exc. in Law. 

1366 MANDEVILLE viii. 95 (1839) Jerusalem hath often tyme ben distroyed, 
\& the Walles abated \& beten doun.    
\P c1420 \textit{Palladius on Husb.} ii. 5 Hem to desolate Of erthe, and 
all from every roote abate.    
\P 1494 FABYAN vii. 490 Ye gates of Bruges, of Ipre, of Courtray, and of 
other townes were abated and throwyn downe.    
\P 1576 LAMBARDE \textit{Peramb. Kent} 185 (1826) Bycause Apultre was not 
of sufficient strength for their defence and coverture they abated it to 
the ground.    
\P 1643 PRYNNE \textit{Doom of Cowardice \& Treach}. 4 And that night came 
a great party of them, and by fine force made an assault and abated the 
Baracadoes.    
\P 1664 EVELYN \textit{Kal. Hort.} 13 (1729) During the hottest months 
carefully abate the weeds.    
\P 1809 TOMLINS \textit{Law Dict.} s.v., To abate; to prostrate, break 
down, or destroy. In law to abate a castle or fort is to beat it down.    
\P 1864 \textit{Wandsw. Br. Act} 44 If any work made by the Company in, 
over, or across the River Thames‥be abandoned or suffered to fall into disuse 
or decay, the Conservators of the River Thames may abate and remove the same.

\itembf{2.} fig. To put down, put an end to, do away with (any state or 
condition of things). Obs. 

c1270 \textit{E.E.P., Old Age} 149 When eld blowid he is blode . his ble is 
sone abatid.    
\P 1340 HAMPOLE \textit{Pr. Consc.} 1672 Ded [= death], of al þat it comes to, 
abates And chaunges all myghtes and states.    
\P c1350 \textit{Will. Palerne} 1141 To abate þe bost of þat breme duke.    
\P 1413 LYDGATE \textit{Pylg. Sowle} v. xii. 103 (1843) And fynally abatid is 
the strif.    
\P 1585 ABP. SANDYS \textit{Serm}. 79 (1841) St. Paul abateth this opinion.    
\textit{Ibid}. 293 To abate the haughty conceit which naturally we have of 
ourselves.

\itembf{3.} Esp. Law. \textbf{a.} To put an end to, do away with (as a nuisance, 
or an action). 

1297 R. GLOUC. 447 And oþer monye luþer lawes, þat hys elderne adde ywroȝt, He 
behet, þat he wolde abate.    
\P 1768 BLACKSTONE \textit{Comm}. III. 168 The primitive sense is that of 
abating or beating down a nusance.    
\P 1780 BURKE \textit{Sp. on Econ. Ref.} Wks. III. 247 They abate the nuisance, 
they pull down the house.    
\P 1844 H. ROGERS \textit{Essays} I. ii. 88 He has not lived in vain who has 
successfully endeavoured to abate the nuisances of his own time.    
\P 1859 DE QUINCEY \textit{The Cæsars} Wks. X. 104 To put him down and abate 
him as a monster.

\itembf{b.} To render null and void (a writ). 

1580 BARET \textit{Alvearie}, His accusation or writte is abated or 
ouerthrowne when the Attorney by ignorance declareth not the processe in due 
forme, or the writte abateth.    
\P 1621 SANDERSON \textit{Serm.} Ad. Cl. ii. xxii. 30 (1674) And any one 
short Clause or Proviso, not legal, is sufficient to abate the whole Writ 
or Instrument.    
\P 1726 AYLIFFE \textit{Parergon} 266 This only suspends but does not 
abate the action.    
\P 1741 ROBINSON \textit{Gavelkind} vi. 109 The Writ was abated by the Court.    
\P 1809 TOMLINS \textit{Law Dict.} s.v., To abate a nuisance is to destroy, 
remove, or put an end to it.‥ To abate a writ is to defeat or overthrow it 
by shewing some error or exception.

\itembf{4.} intr. (through refl.) To be at an end, to become null or void; 
esp. of writs, actions, appeals. 

1602 W. FULBECKE \textit{First Part of Parallele} 62 In the summons A. 
was omitted, wherefore the writte abated.    
\P 1745 DE FOE \textit{Eng. Tradesm.} I. xvi. 148 Commissions shall not 
abate by the death of his majesty.    
\P 1768 BLACKSTONE \textit{Comm.} III. 247 The suit is of no effect, and 
the writ shall abate.    
\P 1809 TOMLINS \textit{Law Dict.} s.v. It is said an appeal shall abate, 
and be defeated by reason of covin or deceit.    
\P 1860 MASSEY \textit{Hist. Engl.} III. xxxi. 437 The Committee of 
Privileges resolved, that impeachments stood on the same footing as 
appeals and writs of error; consequently they did not abate.

\itembf{II.} To bring down, lower, depress. 

\itembf{5.} To bring down (a person) physically, socially, or mentally; 
to depress, humble, degrade; to cast down, deject. Obs. 

c1325 GROSSETESTE \textit{Castel of Loue} 1334 He was abated of his tour 
[= in his turn].    
\P c1386 CHAUCER \textit{Pars. T.} 118 The heyher that they were in this 
present lif, the more schuln thay ben abatid and defouled in helle.    
\P 1470–85 MALORY \textit{Morte Arthur} (1634 repr. 1816) I. 241 Then sir 
Beaumains abated his countenance.    
\P 1564 BAULDWIN \textit{Moral Phil.} (ed. Palfr.) iii. 4 Hee is to be 
honoured among them that be honoured, that fortune abateth without fault.    
\P 1618 RALEIGH \textit{Remains} (1644) 27 If any great person to be 
abated, not to deal with him by calumniation or forged matter.    
\P 1651 JER. TAYLOR \textit{Sermons} i. ix. 104 They were abated with 
humane infirmities and not at all heightened by the Spirit.

\itembf{6.} intr. To fall, be dejected, humbled. Obs. 

1306 \textit{Political Songs} (Camd. S.) 216 Ys continaunce abated eny 
bost to make.    
\P 1387 TREVISA \textit{Higden} Rolls Ser. II. 185 Þe bolde nolle abateþ 
[cervix deprimitur].    
\P c1460 \textit{Urbanitatis in Babees Book} (1868) 16 Lette not þy 
contynaunce also abate.    
\P 1642 ROGERS \textit{Naaman} 30 The naturall spirit of the hautiest‥
will abate and come downe.

\itembf{7.} To abate of; to bring down (a person) from; hence to deprive 
of, curtail of. Obs. 

c1430 \textit{Octouian Imperator} 1316 (Weber III. 212) He was abated of 
all hys hete.    
\P c1530 LD. BERNERS \textit{Arthur of Lytell Bryt.} 105 (1814) That she 
be not thereby abbated of her noblenesse and estate.    
\P 1605 SHAKES. \textit{Lear} ii. iv. 161 She hath abated me of halfe my 
Traine.    
\P 1637 LISLE tr. \textit{Du Bartas} 30 Mens bodies were abated of their 
bignesse.

\itembf{III.} To bring down in size, amount, value, force. 

\itembf{8.} To beat back the edge or point of anything; to turn the edge; 
to blunt. lit. and fig. Obs. 

1548 HALL \textit{Chron.} 689 Such wepons as the capitain of the Castle shall 
occupie, that is, Morrice pike sworde target, the poynt and edge abated.    
\P 1594 SHAKES. \textit{Rich. III}, v. v. 35 Abate the edge of Traitors, 
gracious Lord.    
\P 1613 W. BROWNE \textit{Brit. Past.} i. iv. (1772) 107 With plaints which 
might abate a tyrant's knife.    
\P 1625 BACON \textit{Essays} ix, To abate the edge of envy.    
\P 1634 HEYWOOD \textit{Maidenh. lost} xi. 120 The name of Childe Abates my 
Swords keene edge.    
\P 1699 EVELYN \textit{Acetaria} 145 (1729) Such as abate and take off the 
keeness.

\itembf{9.} To bring down in size; lower, lessen or diminish (things 
tangible). arch. 

1398 TREVISA \textit{Barth. De Pr. Rerum} (1495) xvii. lxxviii. 652 Gutta 
abatyth all swellynge and bolnynge.    
\P 1611 BIBLE \textit{Gen.} viii. 3 After the end of the hundred and fiftie 
dayes, the waters were abated.    
\P 1612 WOODALL \textit{Surgeon's Mate} Wks. (1653) 11 Small Files are used‥
to abate any end of a bone‥which is fractured.    
\P 1662 EVELYN \textit{Chalcog.} (1769) 59 In wood, which is a graving much 
more difficult; because all the work is to be abated and cut hollow.    
\P 1823 SCOTT \textit{Peveril} (1865) 241 A lucky accident had abated 
Chiffinch's party to their own number.

\itembf{10.} intr. To decrease in size or bulk. arch. 

1587 GOLDING \textit{Mornay's Chr. Relig.} xiv. 220 (1617) The more that the 
body abateth in flesh, the more workfull is the mind.    
\P 1597 WARNER \textit{Albion's Eng.} iii. xviii. 86 Their poyson, growing 
when it seemeth to abate.    
\P 1726 DE FOE \textit{Hist. Devil} i. x. 121 (1840) The arke rested, the 
waters abating.

\itembf{11.} trans. To bring down in value, price, or estimation. arch. 

1340 \textit{Ayenb.} 28 Vor þe guode los to abatye, and hire guodes to loȝy, 
þe envious agrayþeþ alle his gynnes.    
\P c1400 \textit{Rom. Rose} 286 She ne might all abate his prise.    
\P c1460 FORTESCUE \textit{Absol. \& Lim. Mon.} (1714) 116 Hou the Pricys 
of Merchaundises, growyn in this Lond, may be holdyn up, and encreasyd, and the 
Prycys of Merchaundise, brought into this Lond abatyd.    
\P 1651 HOBBES \textit{Leviathan} ii. xxii. 119 They raise the price of those, 
and abate the price of these.    
\P 1670 R. COKE \textit{Disc. of Trade} 33 If the Importation of Irish Cattel 
had abated the Rents of England one half.

\itembf{12.} intr. To fall in amount, value, or price, suffer reduction, be 
reduced. arch. exc. in Law. 

1745 DE FOE \textit{Eng. Tradesm.} II. xxxii. 101 As wages abate to the poor, 
provisions must abate in the market, and rents must sink and abate to the 
landlords.    
\P 1768 BLACKSTONE \textit{Comm.} II. 512 And in case of a deficiency of 
assets, all the general legacies must abate proportionably, in order to pay 
the debts.

\itembf{13.} trans. To lessen or lower in force or intensity (a quality, 
feeling, action, etc.); to diminish, lessen, lighten, relieve, mitigate. 

1330 R. BRUNNE \textit{Chron.} 269 His moder Helianore abated þer grete bale.    
\P 1340 HAMPOLE \textit{Pr. Consc.} 2840 For na thyng may abate þair pyne.    
\P 1574 tr. \textit{Marlorats Apocalips} 33 Charitie is lyke fyre, whyche is 
easyly put oute if it be abated.    
\P 1593 T. HILL \textit{Profitable Arte of Gard.} 137 The sauor of them 
[garlic] wilbe greatly abated.    
\P 1599 SHAKES. \textit{Hen. V,} iii. ii. 24 Abate thy Rage, abate thy manly 
Rage.    
\P 1611 BIBLE \textit{Deut.} xxxiv. 7 His eye was not dimme, nor his naturall 
force abated.    
\P 1670 WALTON \textit{Lives} iv. 288 Lord, abate my great affliction, or 
increase my patience.    
\P 1759 ROBERTSON \textit{Hist. Scot.} I. ii. 156 She shook the fidelity, or 
abated the ardour of some.    
\P 1859 MILL \textit{Liberty} ii. 68 To abate the force of these considerations.

\itembf{14.} intr. To fall off in force or intensity; grow less, calm down. 

c1400 \textit{Destr. Troy} xi. 4665 Sesit the wyndis; The bremnes abated.    
\P 1599 SHAKES. \textit{Hen. V,} iv. iv. 50 My fury shall abate, and I The 
Crownes will take.    
\P 1697 DRYDEN \textit{Virg. Georg.} i. 463 (1721) When Winter's Rage abates, 
when chearful Hours Awake the Spring.    
\P 1720 DE FOE \textit{Capt. Singleton} xvi. 274 Towards morning the wind 
abated a little.    
\P 1837 CARLYLE \textit{Fr. Rev.} I. vi. iii. 322 This conflagration of the 
South-East will abate.    
\P 1869 \textit{Echo} Oct. 9 The Foot and Mouth Disease which has been raging 
with some virulence is now beginning to abate.

\itembf{IV.} To strike off, deduct. 

\itembf{15.} trans. To strike off or take away a part, to deduct, subtract. 
\textit{a.} with of (out of, from obs.). 

\P c1391 CHAUCER \textit{Astrol.} 34 Abate thanne thees degrees And minutes 
owt of 90.    
\P 1413 LYDGATE \textit{Pylgr. Sowle} iv. viii. 62 (1483) He nele noo thynge 
abaten of the prys.    
\P 1551 RECORDE \textit{Pathway to Knowl.} ii. Introd., And if you abate euen 
portions from things that are equal, those partes that remain shall be equall 
also.    
\P 1570 DEE \textit{Math. Praef.} 9 If from 4. ye abate 1. there resteth 3.    
\P 1611 BIBLE \textit{Lev.} xxvii. 18 It shall be abated from thy estimation.    
\P 1679–88 \textit{Secret Service Moneys of Chas. \& Jas.} II, 126 (Camd. S. 
1851) To be abated out of the moneys that are or shall be due to him for work.    
\P 1741 \textit{Complete Family-Piece} i. ii. 192 Take‥9 eggs, abating 4 whites.    
\P 1745 DE FOE \textit{Eng. Tradesm.} I. xix. 178 Rather than abate a farthing 
of the price they had asked.    
\P 1866 ROGERS \textit{Agric. \& Prices} I. xx. 506 The merchant abating 
something of his morning price.

\itembf{b.} with obj. (orig. dat.) of the person. 

1465 \textit{Manners \& Househ. Exps.} 465 Roberd Thrope lente me l.s.‥and herof 
he moste a bate me [= to me] .xiiij.s.    
\P 1647 SANDERSON \textit{Sermons Ad Aul.} xv. 1 (1673) 209 He therefore 
sendeth for his Master's Debtors forthwith; abateth them of their several Sums, 
and makes the Books agree.    
\P 1671 J. FLAVEL \textit{Fount. of Life} iii. 6 When the Payment was making, 
he will not abate him one Farthing.    
\P 1771 FRANKLIN \textit{Autobiog.} Wks. 1840 I. 61 She would abate me two 
shillings a week for the future.

\itembf{c.} absol. To make an abatement. 

1530 PALSGR. 420, I alowe or abate upon a reckenyng or accompte made.    
\P 1745 DE FOE \textit{Eng. Tradesm.} I. xix. 179 He cannot abate without 
underselling the market, or underrating the value of his goods.    
\P 1817 JAS. MILL \textit{Brit. India} II. iv. iv. 134 Lacey offered to 
abate in his pecuniary demand.

\itembf{16.} fig. To omit, leave out of count; to bar or except. 

1588 SHAKES. \textit{L.L.L.} v. ii. 547 Abate [a] throw at Novum, and the 
whole world againe, Cannot pricke out five such.    
\P 1700 LAW \textit{Council of Trade} 253 (1751) Abating accidents which 
happen but seldom.    
\P 1772 JOHNSON in \textit{Boswell} (1816) II. 149 Abating his brutality, 
he was a very good master.    
\P 1865 SALA \textit{Diary in America} I. 307 Abating the gold and silver plate.

\itembf{17.} To abate of (a thing): to deduct something from, make an abatement 
from; to lower, or lessen in amount. arch. 

1644 BULWER \textit{Chirologia} 144 It falls short and abates of the perfection 
of the thing.    
\P 1645 BP. HALL \textit{Remedy of Discontent.} 27 Their fading condition 
justly abates of their value.    
\P 1653 IZAAK WALTON \textit{Compl. Angler} 2 [I shall] either abate of my 
pace, or mend it, to enjoy such a companion.  
\P 1765 TUCKER \textit{Lt. of Nat.} II. 635 Their own experience and the 
world they converse with will abate of this excess.    
\P 1810 SCOTT \textit{Lady of L.} v. iii. 22 The guide abating of his 
pace Led slowly through the pass's jaws.

\itembf{V.} Technical. 

\itembf{18.} Falconry. To beat with the wings, flutter. More commonly 
aphetized to bate. Obs. 

c1430 \textit{Bk. of Hawkyng in Rel. Antiq.} I. 297 If that she [the hawk] 
abate, let her flee, but be war that thou constreyne her not to flee.    
\P 1575 TURBERVILLE \textit{Booke of Falc.} 135 You shall keepe hir alwayes 
in best plighte and leaste daunger to abate.

\itembf{19.} In Horsemanship. ‘A Horse is said to Abate, when working upon 
Curvets, he puts his two hind Legs to the Ground, both at once, and observes 
the same Exactness at all Times.’ Bailey 1721; whence in J. and subseq. 
Dicts. Obs.
\end{myenumerate}


%%%%%%%%%%%%%%%%%%%%%%%%%%%%%%%
\myitem{abdicate } v. 

\noindent \phonetic{(ˈæbdɪkeɪt) }

\noindent  
[f. L. abdicāt-, ppl. stem of abdicā-re to renounce, disown, reject; f. 
\textit{ab} off, away + \textit{dicā-re} to proclaim, make known.] 
\vspace{-0.3cm}

\begin{myenumerate}
\itembf{1.} trans. To proclaim or declare to be no longer one's own, 
to disclaim, disown, cast off; esp. to disown or disinherit children. Now 
only as a tech. term of Rom. Law (L. abdicare filium, also patrem). 

1541 ELYOT \textit{Im. Gov.} 149 The father‥doeth abdicate nowe and then 
one, that is to saie, putteth them out of his familie.    
\P 1644 MILTON \textit{Jus Pop.} 34 Parents may not causelessly abdicate 
or disinherit children.    
\P 1697 POTTER \textit{Antiq. Greece} iv. xv. 351 (1715) Parents were 
allow'd to be reconcil'd to their children, but after that could never 
abdicate them again.    
\P 1763 SHENSTONE \textit{Essays} 117 Wherever I disesteemed, I would 
abdicate my first cousin.    
\P 1828 SEWELL \textit{Oxf. Pr. Essay} 70 Sons were exposed, abdicated, 
and sold by the laws of Solon.

\itembf{2.} To depose (from an office or dignity). Obs. 

1621 BURTON \textit{Anat. Mel.} i. 2. iii. xv. 127 (1651) The Turks 
abdicated Cernutus, the next heir, from the empire.

\itembf{3.} refl. To formally cut oneself off, sever, or separate oneself 
from anything; esp. to divest oneself of an office (L. abdicare se magistratu). 
Obs. 

1548 HALL \textit{Chron. Introd. Hist. Hen. IV.} 11 (1809) To perswade a man‥
to Abdicate himselfe from his empire and imperiall preheminence.    
\P 1689 EVELYN \textit{Mem.} (1857) II. 299 The great convention‥resolved that 
King James‥had by demise abdicated himself and wholly vacated his right.    
\P 1689 H. MORE \textit{Myst. Iniq.} 28 A Prince‥who, by transgressing 
against the Laws of the Constitution, hath abdicated himself from the 
Government, and stands virtually Deposed.

\itembf{4.} trans. To put away, cast off, discard (anything). Obs. 

1553–87 FOXE \textit{A. \& M.} (1596) 333/2 The King our souereigne lord and 
maister cannot abdicate from himselfe this right.    
\P 1633 BP. HALL \textit{Hard Texts} 343 Neither hast thou, O Cyrus, so 
well known me as to abdicate thine Idolatry.    
\P 1642 ROGERS \textit{Naaman} 527 If the Lord Jesus purposely would defile 
and abdicate the seventh day Sabbath of the Jew.    
\P 1688–9 LADY R. RUSSELL \textit{Letters} No. 84. II. 11 Accidents may 
abdicate your opinion.

\itembf{5.} To formally give up (a right, trust, office, or dignity); to 
renounce, lay down, surrender, abandon; at first implying voluntary 
renunciation, but now including the idea of abandonment by default. See 
the parliamentary discussions of 1688. 

   1633 BP. HALL \textit{Hard Texts} 41 Abdicating our just privileges.    
\P 1688 LD SOMERS \textit{Speech on the Vacation of the Throne} The word 
abdicate doth naturally and properly signify, entirely to renounce, to 
throw off, disown, relinquish any thing or person, so as to have no further 
to do with it; and that whether it be done by express words or in writing 
(which is the sense your Lordships put upon it, and which is properly called 
resignation or cession), or by doing such acts as are inconsistent with the 
holding and retaining of the thing, which the Commons take to be the present 
case.    
\P 1726 DE FOE \textit{Hist. Devil} (1840) i. i. 14 The thrones which the 
Devil and his followers abdicated and were deposed from.    
\P 1783 JOHNSON \textit{Club Rules in Boswell} (1816) IV. 277 Whoever shall 
for three months together omit to attend‥shall be considered as having 
abdicated the club.    
\P 1805 FOSTER \textit{Essays} i. vii. 90 To have abdicated the dignity of 
reason.    
\P 1857 PRESCOTT \textit{Philip} i. i. 10 The Regent Mary formally abdicated 
her authority.    
\P 1857 RUSKIN \textit{Pol. Econ. Art}, 5 A power not indeed to be envied‥but 
still less to be abdicated or despised.

\itembf{6.} Comm. Law. Said of the insurer surrendering his rights of ownership 
to the underwriters. 

1755 N. MAGENS \textit{Ess. Insur.} II. 36 The Owners of such Gold, Silver, 
or Pearls, cannot renounce or abdicate them to the Underwriters.

\itembf{7.} absol. (by ellipsis of the thing resigned, usually the throne or 
crown). To renounce or relinquish sovereignty, or its equivalent. 

   a1704 T. BROWN \textit{Epigr.} Wks. 1730 I. 121 Either he must abdicate 
or thou.    
\P 1726 DE FOE \textit{Hist. Devil} (1840) ii. i. 181 The Devil abdicated 
for awhile.    
\P 1837 CARLYLE \textit{Fr. Rev.} I. vii. xi. 399 Is it not strange so few 
kings abdicate; and none yet heard of has been known to commit suicide?    
\P 1879 GLADSTONE \textit{Gleanings} III. i. 5 The Majority have in virtue 
and effect abdicated, and their opponents are the true and genuine corporation.
\end{myenumerate}






%%%%%%%%%%%%%%%%%%%%%%%%%%%%%%%
\myitem{aberrant} a. 

\noindent \phonetic{(əˈbɛrənt) }

\noindent [ad. L. aberrant-em, pr. pple. of aberrā-re. See aberr.] 
\vspace{-0.3cm}

\begin{myenumerate}

\itembf{1.} lit. Wandering away or straying from a defined path; hence fig. diverging or
deviating from any moral standard. 

\P 1848 KINGSLEY \textit{Saint's Trag.} (1878) iv. ii. 123 Such a choice must argue
Aberrant senses, or degenerate blood.    
\P 1864 COCKRAN \textit{tr. Pressensé's Reply to Renan} 83 People see in it the signs 
of a diseased, aberrant genius.

\itembf{2.} Deviating widely from the ordinary or natural type, exceptional, irregular,
abnormal; especially in Nat. Hist.  \vspace{0.2cm} \\ 
   
\P 1830 LYELL \textit{Princ. Geol.} (1875) II. iii. xxxvii. 322 If there be such
proneness in each aberrant form to merge into the normal type.    
\P 1835 KIRBY \textit{Habits \& Inst. An.} II. xvi. 74 The usual oral organs, though a little aberrant
in their structure.    
\P 1839 HALLAM \textit{Lit. Eur.} i. viii. §28 These aberrant lines
are much more common in the dramatic blank verse of the seventeenth century.
\P 1857 H. MILLER \textit{Sch. \& Schoolm.} viii. 167 His mother, though of a devout
family of the old Scottish type, was an aberrant specimen.    
\P 1878 M. FOSTER \textit{Physiology} iv. v. 560 The events are much more characteristic in the typical
female than in the aberrant male.    
\P 1881 WESTCOTT \& HORT \textit{N.T. in Greek} II. 240
It would be‥difficult to derive the neutral reading from any coalescence of the
aberrant readings.

\noindent Hence \textbf{aberrantly} adv. 

\P 1878 G.G. SCOTT \textit{Recoll.} (1879) vii. 291 Skidmore followed my design, but
somewhat aberrantly.    
\P 1929 \textit{New Statesman} 31 Aug. 614/1 Most unfortunately we
have aberrantly accepted a mandate for Palestine.
\end{myenumerate}


%%%%%%%%%%%%%%%%%%%%%%%%%%%%%%%
\myitem{abominate} v. 

\noindent \phonetic{(əˈbɒmɪneɪt) }

\noindent  [f. L. abōmināt- ppl. stem of abōminā-ri: see abominable and -ate.] 
\vspace{-0.3cm}

\begin{myenumerate}
\itembf{1.} To feel extreme disgust and hatred towards; to regard with intense aversion;
to abhor, loathe.  \\
   
\P 1644 BULWER \textit{Chironomia} 53 Who refuse, abhor, detest or abominate some
execrable thing.    
\P 1649 MILTON \textit{Eikon.} i. 339 (1851) A Scotch Warr, condemn'd
and abominated by the whole kingdom.    
\P 1706 DE FOE \textit{Jure Divino Pref.} 4 Those
who Swore to him when he was King‥are all Perjur'd Rebels; abominable, and to be
abominated by all good Men.    
\P 1728 NEWTON \textit{Chronol. Amended} 9 The
Egyptians‥lived only on the fruits of the earth, and abominated flesh-eaters.
\P 1866 MOTLEY \textit{Dutch Rep.} iii. v. 437 Influential persons in Madrid had openly
abominated the cruel form of amnesty which had been decreed.

\itembf{2.} loosely. To dislike strongly. 

\P 1880 V. LEE\textit{Italy} iv. iii. 170 Steele‥had no musical sense, and abominated
operas.    
\P 1881 A. TROLLOPE \textit{Ayala's Angel} III. xlvi. 37 Then he spake again ‘I
do abominate a perverse young woman.’
\end{myenumerate}

%%%%%%%%%%%%%%%%%%%%%%%%%%%%%%%
\myitem{abrasive} a. and n.

\noindent \phonetic{(əˈbreɪsɪv) }

\noindent  [f. L. abrās-us: see abrase + -ive; as if from a L. *abrāsīvus.] 
\vspace{-0.3cm}

\begin{myenumerate}
\itembf{A.} adj. \textbf{a.} Having the property of abrading.  

\P 1875 URE \textit{Dict. Arts} s.v. Abrasion, The abrasive tool or grinder is exactly a
counterpart of the form to be produced.    
\P 1880 G.C. WALLICH in \textit{Athen.} 6 Mar. 316 To dispose of the supposition that 
the shape of the Pyrospores is due to any rolling or abrasive action at the sea bed.

\itembf{b.} fig. 

\P 1925 T. DREISER \textit{Amer. Trag.} (1926) I. ii. xxxiv. 387 His mind was troubled
with hard, abrasive thoughts.    
\P 1963 EDMUND WILSON in \textit{New Statesman} 8 Feb. 198/3 Abrasive is coming in, 
in application to literary qualities.

\itembf{B.} n. An abrasive substance or body. 

\P 1853 O. BYRNE \textit{Artisan's Handbk.} 17 To polish the tool upon the oil-stone, or
other fine abrasive for setting the edge.    
\P 1951 \textit{Good Housek. Home Encycl.} 11/1
Abrasives are useful for heavily soiled surfaces, when soap and water or
detergents are unsuccessful and some gentle friction is required.    
\P 1960 \textit{Jrnl. Iron \& Steel Inst.} CXIV. 406/1 A study of bonded abrasives.
\end{myenumerate}


%%%%%%%%%%%%%%%%%%%%%%%%%%%%%%%
\myitem{abrogate }  v.

\noindent \phonetic{(ˈæbrəgeɪt) }

\noindent [f. prec., or on analogy of vbs. so formed.] 
\vspace{-0.3cm}

\begin{myenumerate}
\itembf{1.} To repeal (a law, or established usage), to annul, to abolish
authoritatively or formally, to cancel. 

\P 1526 TINDALE \textit{Heb.} viii. 13 In that he sayth a new testament he hath abrogat
the olde.    
\P 1553 WILSON \textit{Rhetorique} 24b, They abrogate suche vowes as were
proclaimed to be kept.    
\P 1649 MILTON \textit{Eikon.} 46 Doubtless it repented him to
have establish'd that by Law, which he went about so soon to abrogat by the
Sword.    
\P 1666 FULLER \textit{Hist. Cambr.} (1840) 157 Thus was the pope's power fully
abrogated out of England.    
\P 1775 BURKE \textit{Sp. Concil. with Amer.} Wks. III. 60 We
wholly abrogated the ancient government of Massachuset.    
\P 1841 MYERS \textit{Cath. Thoughts} iv. §26. 305 The Law of the Jews‥was not rejected nor contradicted by
the Gospel‥but simply abrogated by being absorbed.    
\P 1862 LD. BROUGHAM \textit{Brit. Constitn.} i. 22 But the same power which formed these rules may abrogate or
suspend them.

\itembf{2.} To do away with, put an end to. 

\P 1588 SHAKES. \textit{L.L.L.} iv. ii. 55 Perge, good M. Holofernes, perge, so it shall
please you to abrogate scurilitie.    
\P 1634 T. HERBERT \textit{Travaile} 141 Others say
all the world was a paradice till sinne abrogated its glory.    
\P 1851 MRS. BROWNING \textit{Casa Guidi Wind.} 95 Pay certified, yet payers abrogated.    
\P 1855 OWEN \textit{Skel. \& Teeth} 86 In the whales the movements of these vertebræ upon one another
are abrogated.
 
\itembf{3.} Immunol. To suppress or prevent (a physiological process). 

\P 1959 [implied in *ABROGATION n. below].    
\P 1965 \textit{Science} 2 July 82/2 The
inhibition of cell growth in the hybrids‥, which is detected by tumor
transplantation into mice, could be abrogated by treatment of the recipient mice
with cortisone acetate.    
\P 1974 \textit{Nature} 10 May 161/1 (heading) Lymphocytes from
human newborns abrogate mitosis of their mother's lymphocytes.    
\P 1990 EMBO \textit{Jrnl.} IX. 3821/2 PT application abrogates interleukin-2 (IL-2) secretion from a
murine hybridoma.
\end{myenumerate}

%%%%%%%%%%%%%%%%%%%%%%%%%%%%%%%
\myitem{abstemious} a.

\noindent \phonetic{(æbˈstiːmɪəs) }

\noindent  
[f. L. abstēmi-us + -ous. Abstemius was considered by L. writers to be f. abs
away from + tēmētum intoxicating liquor; but even in L. was extended to
temperance in living generally. The verbal resemblance to abstain, absteine, has
in Eng. given it a still wider use, and also produced the forms absteinous,
abstenious.] 
\vspace{-0.3cm}

\begin{myenumerate}
\itembf{1.} Dispensing with wine and rich food; temperate or sparing in food;
characterized by or belonging to such temperance; sparing. \textbf{a.} Of persons,
their lives, or habits. 

\P 1624 HEYWOOD \textit{Gunaikeion} v. 226 To this absteinous life shee added the strict
vow of chastitie.    
\P 1718 POPE \textit{Iliad} xix. 328 Let me pay To grief and anguish
one abstemious day.    
\P 1832 CARLYLE \textit{Remin.} i. 26 Mother and father were
assiduous, abstemious, frugal without stinginess.    
\P 1878 BLACK \textit{Green Past. and Picc.} xxix. 234 They were remarkably abstemious at breakfast.

\itembf{b.} Of the food. 

\P 1776–88 GIBBON \textit{Decl. \& Fall} lviii, His [Peter the Hermit's] diet was
abstemious, his prayers long and fervent.    
\P 1832 SCOTT \textit{Talism.} ii. 26 The meal of the Saracen was abstemious.

\itembf{2.} Abstinent, refraining, sparing (with regard to other things than food).
rare. 

\P 1610 SHAKES. \textit{Temp.} iv. i. 53 Be more abstemious, Or else good night your vow.
\P 1632 MASSINGER \textit{Maid of Hon.} ii. ii. The king‥Is good and gracious‥Abstemious
from base and goatish looseness.    
\P 1823 LAMB \textit{Elia} (1865) i. xxi. 163 You advised an abstemious introduction 
of literary topics.
\end{myenumerate}


%%%%%%%%%%%%%%%%%%%%%%%%%%%%%%%
\myitem{abstinent} a. and n.

\noindent \phonetic{(ˈæbstɪnənt) }

\noindent  
[a. Fr. abstinent, refashioned on OFr. astenant:—L. abstinent-em, pr. pple. of
abstinē-re: see abstain.] 
\vspace{-0.3cm}

\begin{myenumerate}
\itembf{A.} adj. Holding back or refraining; esp. from indulgence of appetite;
continent, abstemious, temperate. 

\P c1386 CHAUCER \textit{Pars. T.} 873 Abstinent in etyng and drynkyng, in speche and in
dede.     
\P c1440 \textit{Prompt. Parv.} Abstynent, or absteynynge.    
\P 1588 A. KING \textit{Canisius' Catech.} 132 b, Bot he, quha is abstinent, sal prolonge his lyf.
\P 1603 HOLLAND \textit{Plutarch's Morals} 651 And he againe, who is too too sober, and
abstinent altogether, becommeth unpleasant and unsociable.    
\P 1713 \textit{Guardian} (1756) I. 16 She has passed several years in widowhood with that abstinent
enjoyment of life, which has done honour to her deceased husband.    
\P 1867 J. MARTINEAU \textit{Chr. Life} (ed. 4) 84 What abstinent integrity is‥demanded by many a
master.

\itembf{B.} n. One who abstains, an abstainer, a faster. In Eccl. Hist. the Abstinents
were a sect who appeared in the 3rd century. 

\P c1440 \textit{Prompt. Parv.} Abstynent‥or he that dothe abstynence.    
\P 1615 CHAPMAN \textit{Odyssey} xvii. 381 And this same harmful belly by no mean The greatest 
abstinent can ever wean.    
\P 1669 J. REYNOLDS \textit{Disc. in Harl. Misc.} (1745) iv. 48 Some of
these Abstinents were of melancholick complexions.    
\P 1753 CHAMBERS \textit{Cyc. Suppl.}
s.v., Some represent the Abstinentes‥that they particularly enjoined abstinence
from the use of marriage; others say, from flesh; and others, from wine.    
\P 1860 \textit{All Y. Round} No. 64. 322 There is also [in China] a female sect called the
Abstinents‥who make a vow to abstain from everything that has enjoyed life, and
to eat nothing but vegetables.
\end{myenumerate}

%%%%%%%%%%%%%%%%%%%%%%%%%%%%%%%
\myitem{abstruse }  a.

\noindent \phonetic{(æbˈstruːs) }

\noindent  
[ad. L. abstrūs-us thrust away, concealed, pa. pple. of abstrūd-ĕre: see prec.
Mentioned by P. Heylin as an ‘uncouth and unusual word’ in 1656.] 
\vspace{-0.3cm}

\begin{myenumerate}
\itembf{1.} Concealed, hidden, secret. Obs. 

\P 1602 THYNNE \textit{Chaucer} (1865) 107 The Abstruse skill, the artificiall veine; By
true Annalogie I ryhtly find.    
\P 1620 SHELTON \textit{Don Quixote} (1746) II. iv. xv. 194
Hidden in the most abstruse dungeons of Barbary.    
\P 1667 MILTON \textit{P.L.} v. 712 The
eternal eye, whose sight discerns Abstrusest thoughts.    
\P 1762 B. STILLINGFLEET\textit{Linn. Or. in Misc. Tracts} 9 That the abstruse forces 
of the elements, which otherwise would escape our senses, may be made manifest.

\itembf{2.} Remote from apprehension or conception; difficult, recondite. 

\P 1599 THYNNE \textit{Animadv.} (1865) 36 That abstruce scyence whiche Chaucer knewe
full well.    
\P 1671 MILTON \textit{Samson} 1064 Be less abstruse, my riddling days are
past.    
\P 1704 SWIFT \textit{Tale of a Tub} Wks. 1760 I. 13 Readers, who cannot enter into
the abstruser parts of the discourse.    
\P 1751 WATTS \textit{Improv. Mind} (1801) 107 Let
not young students apply themselves to search out deep, dark, and abstruse
matters, far above their reach.    
\P 1848 H. MILLER \textit{First Impr.} (1857) xix. 340
Men who had wrought their way‥into some of the abstrusest questions of the
schools.    
\P 1855 MILMAN \textit{Lat. Chr.} (1864) V. ix. viii. 380 But these were
solitary abstruse thinkers or minds which formed a close esoteric school.
\end{myenumerate}
 


%%%%%%%%%%%%%%%%%%%%%%%%%%%%%%%
\myitem{accolade}  n.

\noindent \phonetic{(ækəʊˈleɪd, akəʊˈlɑːd; now usu. ˈækəʊleɪd) }

\noindent  
[a. mod.Fr. accolade, ad. It. accollata, n. f. pa. pple. of accollare to embrace
about the neck; see accoll, and -ade. Introduced into Fr. in 16th c. superseding
the OFr. cogn. acolée; it has similarly superseded the earlier acolee in Eng.] 
\vspace{-0.3cm}

\begin{myenumerate}
\itembf{1.a.} Properly, an embrace or clasping about the neck; technical name of the
salutation marking the bestowal of knighthood, applied at different times to an
embrace, a kiss, and a slap on the shoulders with the flat blade of a sword.
   [Not in Cotgrave 1611 who has Accollade (Fr.) a colling, clipping, imbracing
about the necke; Hence, the dubbing of a Knight, or the ceremony used therein.] 

\P 1623 FAVINE \textit{Theat. Honour} i. vi. 51 Giuing him also the Accollade, that is to
say, Kissing him.    
\P 1706 PHILLIPS \textit{Accollade}, clipping and colling, embracing
about the Neck.    
\P 1753 CHAMBERS \textit{Cycl. Supp.} s.v. Antiquaries are not agreed,
wherein the Accolade properly consisted.    
\P 1817 SCOTT \textit{Wav.} I. x. 131 The
quantities of Scotch snuff which his accolade communicated.    
\P 1852 C.M. YONGE \textit{Cameos} I. xvi. 122. (1877) Henry conferred on him 
the accolade, or sword blow, which was the chief part of the ceremony.    
\P 1858 WISEMAN \textit{Last Four Popes} 511 Could he [the Pope] receive him 
[Czar Nicholas] with a bland smile and insincere
accollade?

\itembf{b.} fig. A supreme honour; a mark of approval or admiration; a bestowal of
praise, a plaudit; an acknowledgement of merit. 

\P 1852 P.J. BAILEY \textit{Festus} (ed. 5) 250, I would knight you on the spot, But,
really, I'm afraid, my sword's forgot. However, take my verbal accolade!    

\P 1906 \textit{‘O. Henry’ in Munsey's Mag.} Aug. 559/2 All this meant that Curly had won his
spurs, that he was receiving the puncher's accolade.    
\P 1940 W. FAULKNER \textit{Hamlet} ii. ii. 131 The impotent youths who‥had 
conferred upon them likewise blindly and unearned the accolade of success.    
\P 1955 E. BLISHEN \textit{Roaring Boys} i. 18
Improbable accolades. ‘Good old sir!’ ‘You're a sport, sir!’    
\P 1961 M. BEADLE \textit{These Ruins are Inhabited} (1963) ix. 113 A Nobel Prize 
is the top accolade a scientist can receive.    
\P 1974 ‘J. Herriot’ \textit{Vet in Harness} xii. 89 Once the long
process had been completed and the last piece of marzipan and icing applied she
dearly loved to have the accolade from an expert.    
\P 1984 \textit{Ann. Rep. Racal Electronics PLC} 7/1 The highest accolade in the 
engineering profession—election
to the Fellowship of Engineering—was bestowed in April 1984 on Geoffrey Lomer.

\itembf{2.} Music. A vertical line or brace, used to couple together two or more staves.
(Sometimes confined to a straight thick line so used, as distinguished from a
brace or double curve; but in mod.Fr. accollade = the brace or double curve
‹horizb›, used not merely in music but in ordinary printing, algebra,
classification, etc.) 

\P 1882 ROCKSTRO in \textit{Grove Dict. Mus. s.v. Score}, In Scores‥the Staves are
united, at the beginning of every page, either by a Brace, or by a thick line,
drawn, like a bar, across the whole, and called the Accolade.
\end{myenumerate}

%%%%%%%%%%%%%%%%%%%%%%%%%%%%%%%
\myitem{acerbic }  a.

\noindent \phonetic{(əˈsɜːbɪk) }

\noindent  [f. L. acerb-us + -ic.] 

\noindent
Of a sour, harsh, or severe character. Freq. of speech, manner, or temper:
sharp, cutting. 

\P 1865 \textit{North Brit. Daily Mail} 4 Dec., Exaggerated notions are entertained
now-a-days regarding the gloomy acerbic nature of Sabbath observance among the
ancient Jews.    
\P 1971 \textit{Times Lit. Suppl.} 21 May 582/1 The fury he aroused in the
acerbic breast of Karl Marx.    
\P 1976 \textit{Economist} 13 Mar. 33 As defeat in Florida
came closer‥, his speeches grew less polite and more acerbic.    
\P 1984 \textit{Washington Post} 3 Aug. b7 Although they borrow from Tom 
Lehrer and Mark Russell‥, they are
far less acerbic—perhaps because they are part of what they lampoon.

%%%%%%%%%%%%%%%%%%%%%%%%%%%%%%%
\myitem{acme }  n.

\noindent \phonetic{(ˈækmiː) }

\noindent  
[a. Gr. ἀκµή point. Long consciously used as a Gr. word, and written in Gr.
letters from Ascham 1570 to Goldsmith 1750, although spelt as Eng. by B. Jonson
1625, and commonly afterwards.] 
\vspace{-0.3cm}

\begin{myenumerate}
\itembf{1.} gen. The highest point or pitch; the culmination, or point of perfection, in
the career or development of anything. 

\P 1570 R. ASCHAM \textit{Scholem.} (1863) 93 The Latin tong, even whan it was, as the
Grecians say, in ἀκµῇ, that is, at the hiest pitch of all perfitenesse.
\P a1637 B. JONSON \textit{Discov.} So that he may be named, and stand as the mark and
ἀκµή of our language.    
\P 1641 W. CARTWRIGHT \textit{Lady Err.} ii. iv. (1651) 23 I' th'
heat and achme of devotion.    
\P 1655 FULLER \textit{Ch. Hist.} iii. 78 Date we from this
day, the achme or vertical height of Abbeys, which henceforward began to stand
still, \& at last to decline.    
\P 1659 LESTRANGE \textit{Alliance Div. Off.} ix. The
Liturgy and ceremonie of our Church, drawing nigh to its ἀκµή.    
\P 1675 OGILBY \textit{Brit. Ded.}, In the Achma of the Three Last Empires of the World.    
\P 1765 GOLDSM. \textit{Ess., Taste,} By the age of ten his genius was at the ἀκµή.    
\P 1790 BURKE \textit{Fr. Revol.} Wks. V. 236 The growth of population in France 
was by no means at its acmé in that year.    
\P 1800 WEEMS \textit{Washington} (1877) xi. 155 Having at length
attained the acme of all his wishes.    
\P 1817 MALTHUS \textit{Population} III. 57 No
country has ever reached, or probably ever will reach, its highest possible acme
of produce.    
\P 1835 I. TAYLOR \textit{Spir. Despotism} §5. 188 A position whence the
transition was easy to the acmé of unbounded despotism.    
\P 1868 GLADSTONE \textit{Juv. Mundi} (1870) xi. 421 It is however in Achilles 
that courtesy reaches to its acmè.    
\P 1880 \textit{Boy's Own Bk.} 240 The acme of bicycle riding.

\itembf{2.} esp. \textbf{a.} The period of full growth, the flower or full-bloom 
of life. Obs. 

\P 1620 VENNER \textit{Via Recta} viii. 174 They haue not attained vnto the Acme, or full
height of their growing.    
\P 1625 B. JONSON \textit{Staple of News Prol.} (1631) 5 He must
be one that can instruct your youth, And keepe your Acme in the state of truth.
\P 1650 BULWER \textit{Anthropometam.} §22. 245 [It] may be either in the achma or
declination of our age.    
\P 1660 T. STANLEY \textit{Hist. Philos.} (1701) 259/2 Youth is
the encrease of the first Refrigerative part, Age the decrease thereof, ἄκµη,
the constant and perfect Life which is betwixt both.    
\P 1664 EVELYN \textit{Sylva} 37
Every tree‥after each seven years improving twelve pence in growth, till they
arriv'd to their acme.    
\P 1844 STANLEY \textit{Arnold's Life \& Corr.} II. x. 314 The
thought that the forty-ninth year, fixed by Aristotle as the acme of the human
faculties, lay still some years before him.

\itembf{b.} The point of extreme violence of a disease, the crisis. arch. 

\P c1630 JACKSON \textit{Creed} viii. xiii. Wks. VII. 496 Christ Jesus‥in the very ἀκµὴ
of his agony‥did set the fairest copy of that obedience.    
\P 1676 GREW \textit{Plants, Lect.} ii. i. §26 (1682) 242 We may conceive the 
reason of the sudden access of
an acute Disease, and of its Crisis‥when the Cause is arrived unto such an ἀκµὴ.
\P 1752 in \textit{Phil. Trans.} XLVII. lxxiii. 586 From the beginning to the flatus or
acme of the disease, they almost all die.    
\P 1837 CARLYLE \textit{Fr. Rev.} (1872) I. v.
vi. 167 Paris wholly has got to the acme of its frenzy.
 
\itembf{3.} Used attrib. to designate (a screw having) a type of modified square
thread whose grooves have sides inclined at an angle of 29°. 

\P 1895 \textit{Amer. Machinist} XVIII. 2 Mr. Handy has named the new thread the ‘Acme
Standard’.    
\P 1920 \textit{Ibid.} LIII. 105 The Acme thread was not designed for any
particular ratio of pitch for a given diameter although W. S. Dix‥recommended
that the ratios of pitch to diameter be in the proportion of one half the number
of threads or twice the pitch of the U.S. standard screw.    
\P 1930 \textit{Engineering} 6 June 721/1 A narrow thread must throw up less 
metal in its thread than a Vee-thread, or an acme thread.    
\P 1964 S. CRAWFORD \textit{Basic Engin. Processes} (1969)
v. 116 Travel of the cross-slide is controlled by a screw, usually of square or
acme thread form.
\end{myenumerate}



%%%%%%%%%%%%%%%%%%%%%%%%%%%%%%%
\myitem{acolyte}  n.

\noindent \phonetic{(ˈækəlaɪt)}

\noindent  
[ad. med.L. acolitus, acolithus, acolythus, corrupt forms of acolūthus a. Gr. 
ἀκόλουθος following, attending upon, subst. an attendant. The normal form 
is acoluth, as written by some of the 16th c. scholars. Occ. aphetized to 
colet, and expanded to acolythist, acolouthite.] 
\vspace{-0.3cm}

\begin{myenumerate}
\itembf{1.} Eccl. An inferior officer in the church who attended the priests and deacons, and performed subordinate duties, as lighting and bearing candles, etc. 

\P c1000 ÆLFRIC \textit{Past. Ep. in Anc. Laws} II. 378 Acolitus is se þe leoht berð æt Godes þenungum.
\P c1315 SHOREHAM 45 The ferthe [degree in orders] acolyt hys to segge y-wys Tapres to bere wel worthe.    
\P 1382 WYCLIF \textit{Coloss.} Prol., Therfore the apostle, thennis boundyn, writith to hem fro Effecie bi Tyte, a dekene, and Honesym, acolite.    
\P 1460 J. CAPGRAVE \textit{Chron.} 74 He that schuld be mad a Bischop schuld first be a benet‥and then a colet; and then subdiacone, diacone, and prest.    
\P 1555 \textit{Fardle of Facions} ii. xii. 267 The Acholite, whiche we calle Benet or Cholet, occupieth the roume of Candle-bearer.    
\P 1561 T. N[ORTON] \textit{Calvin's Inst.} (1634) iv. 155 They play ye Philosophers about ye name of Acoluth, calling him a Ceroferar, a taper bearer with a worde‥wheras Acoluthos in Greke simply signifieth a folower.    
\P 1588 A. KING \textit{Canisius' Catech.} 109 Gif ony man deseruis to be ane Bishope lat him first be ostiar, secundlie lecteur, nixt ane Exorcist, efter ane Acolyt.    
\P 1594 HOOKER \textit{Eccl. Pol.} vii. xx. Wks. III. 347 The bishops attendants, his followers they were; in regard of which service the name of Acolythes seemeth plainly to have been given.    
\P 1637 GILLESPIE \textit{Eng. Popish Cerem.} iii. viii. 161 Exorcists, Monkes, Eremits, Acoluths, and all the whole rabble of Popish orders.    
\P 1649 SELDEN \textit{Laws of Eng.} i. x. (1739) 18 Acolites, which waited with the Taper ready lighted.    
\P 1824 SOUTHEY \textit{Bk. of the Ch.} I. 353 The candlestick, taper and urceole were taken from him as acolyte.    
\P 1849 W. FITZGERALD tr. \textit{Whitaker's Disput.} 505 The apostolic canons‥name only five orders,—the bishop, priest, deacon, reader, and chanter, omitting the exorcist, porter, and acolyth.    
\P 1855 tr. \textit{Labarte's Arts Mid. Ages \& Renaiss.} i. 15 Two acolythes carried the candlesticks.    
\P 1873 W.H. DIXON \textit{Two Queens} I. vi. x. 369 At every porch a priest came out with acolyte and choir.

\itembf{2.} In other senses: \textbf{a.} An attendant or junior assistant in any ceremony or operation; a novice. 

\P 1829 SCOTT \textit{Demonol.} vii. 213 Nor are such acolytes found to evade justice with less dexterity than the more advanced rogues.    
\P 1831 \textit{Kenilw.} xxxii. (1853) 296 To awaken the bounty of the acolytes of chivalry.    
\P 1865 DICKENS \textit{Mut. Fr.} i. 137 It was the function of the acolyte to dart at sleeping infants.

\itembf{b.} An attendant insect or other animal. 
   
\P 1876 BENEDEN \textit{An. Paras.} 4 Species at the mercy of others, and dependent on acolytes, which are in every respect inferior to themselves.

\itembf{c.} An attendant star. 
   
\P 1876 CHAMBERS \textit{Astron.} 910 Acolyte‥sometimes used to designate the smaller of two stars placed in close contiguity.
\end{myenumerate}

%%%%%%%%%%%%%%%%%%%%%%%%%%%%%
\myitem{acquiesce }  v.

\noindent  \phonetic{(ækwɪˈɛs) }

\noindent  
[a. MFr. acquiesce-r (16th c. in Littré), f. L. acquiēsc-ĕre; f. ac- = ad- to, at + quiēsc-ĕre to rest.] 
\vspace{-0.3cm}

\begin{myenumerate}
\itembf{1.} intr. To remain at rest, either physically or mentally; to rest satisfied (in a place or state). Obs. 

\P c1620 A. HUME \textit{Orthogr. Brit. Tongue} (1865) 9 But as now we sound it in quies and quiesco, the judiciouse ear may discern tuae soundes. But because heer we differ not, I wil acquiess.    
\P 1642 HOWELL \textit{For. Trav.} (1869) 88 Being safely returned to his Mother soile, he may very well acquiesse in her lap.    
\P 1756 BURKE \textit{Subl. \& B.} Wks. I. i. §9. 136 We were not made to acquiesce in life and health.    
\P 1788 PRIESTLEY \textit{Lect. on Hist.} v. li. 386 No situation—in which he can entirely acquiesce, so as to look out for no farther improvements.

\itembf{b.} To acquiesce from: To rest, or cease from. Obs. rare. 

\P 1659 LESTRANGE \textit{Alliance Div. Off.} (1846) 12, I resolved totally to acquiesce from such contests.

\itembf{c.} To acquiesce under: To remain in quiet subjection, to submit quietly, to remain submissive. Obs. 

\P 1680 in SOMERS \textit{Tracts} II. 90 For if he be innocent, and that the Right of Succession be his, all Men will quietly acquiesce under him.    
\P 1749 FIELDING \textit{Tom Jones} ix. vii. (1840) 137/2 Our readers may not so easily acquiesce under the same ignorance.    
\P 1771 \textit{Junius Lett.} xliv. 236 Privilege of parliament‥has hitherto been acquiesced under.    
\P 1781 T. JEFFERSON \textit{Corr. Wks.} 1859 I. 310 [It may] lead the minds of the people to acquiesce under those events which they see no human power prepared to ward off.

\itembf{2.} To agree tacitly to, concur in; to accept (the conclusions or arrangements of others). 

\P 1651 HOBBES \textit{Leviathan} i. vii. 32 Our Beleefe‥is in the Church; whose word we take, and acquiesce therein.    
\P 1672 MARVELL \textit{Rehearsal Transp.} i. 52 You are bound to acquiesce in his judgment, whatsoever may be your private Opinion.    
\P 1690 LUTTRELL \textit{Brief Rel.} (1857) II. 21 The said citty acquiesced, and wrote a submissive letter to the king.    
\P 1781 COWPER \textit{Lett.} 4 Oct. Wks. 1876, 85, I perfectly acquiesce in the propriety of sending Johnson a copy of my productions.    
\P 1831 SCOTT \textit{F.M. Perth} xi. (1874) 115 Douglas seemed to acquiesce in the necessity of patience for the time.    
\P 1877 MOZLEY \textit{Univ. Serm.} iv. 76 They speak with an air of men whose claims have been acquiesced in by others.

\itembf{b.} Const. to, with. Obs. 

\P 1651 HOBBES \textit{Gov. \& Soc.} xi. §6. 171 We must acquiese to their sayings, whom we have truly constituted to be Kings over us.    
\P 1685 LADY R. RUSSELL \textit{Lett.} 24. I. 64 The great thing is to acquiesce with all one's heart to the good pleasure of God.    
\P 1703 DE FOE \textit{Shortest way to Peace in Miscell.} I. 465 If they acquiesce with a Church of England Government.    
\P 1748 RICHARDSON \textit{Clarissa} (1811) V. 33 Clarissa had a double inducement for acquiescing with the proposed method.

\itembf{3.} trans. To bring to rest; to appease, satisfy, or harmonize. Obs. 

\P 1658–9 LOCKYER in Burton \textit{Diary} (1628) IV. 114 This union did most acquiesce all interests.
\end{myenumerate}


%%%%%%%%%%%%%%%%%%%%%%%%%%%%%
\myitem{acrimonious }  a.

\noindent  \phonetic{(ˌækrɪˈməʊnɪəs) }

\noindent  
[ad. Fr. acrimonieux, -euse, ad. med.L. ācrimōniōs-us, f. ācrimōnia; see acrimony and -ous.] 
\vspace{-0.3cm}

\begin{myenumerate}
\itembf{1.} 1 = acrid 1. arch. 

\P 1612 WOODALL \textit{Surgeon's Mate} Wks. 1653, 180 If it proceed of an acrimonious fretting humor, etc.    
\P 1646 SIR T. BROWNE \textit{Pseud. Ep.} 336 Artificiall copperose‥is a rough and acrimonious kinde of salt.    
\P 1664 H. POWER \textit{Exp. Philos.} i. 63 A sharp and acrimonious vapour that strikes our nostrils.    
\P 1732 ARBUTHNOT \textit{Rules of Diet} 298 All Substances that abound with an acrimonious Salt and Volatile Oil are hurtful.    
\P 1813 MARSHALL \textit{Gardening} §19, 328 (ed. 5) The sap is very (even dangerously) acrimonious.    
\P 1856 MILL \textit{Logic} iv. v. §4 (1868) II. 244 Natural substances which possessed strong and acrimonious properties.

\itembf{2.} Bitter and irritating in disposition or manner; bitter-tempered. 

\P 1775 JOHNSON \textit{Tax. no Tyr.} 69 Malignity thus acrimonious.    
\P 1831 SCOTT \textit{Abbot} i. 12 Engaged in a furious and acrimonious contest.    
\P 1833 I. TAYLOR \textit{Fanaticism} §1, 2 If‥his feelings are petulant and acrimonious.    
\P 1849 MACAULAY \textit{Hist. Eng.} I. 565 Only a single acrimonious expression escaped him.    
\P 1861 MAY \textit{Const. Hist. Eng.} I. i. 54 (1863) Political hostility had been embittered by the most acrimonious disputes.
\end{myenumerate}


%%%%%%%%%%%%%%%%%%%%%%%%%%%%%
\myitem{acronym }  n., orig. U.S.

\noindent  \phonetic{(ˈækrənɪm) }

\noindent [f. acro-(Gr. ἀκρο- highest) + -onym after homonym.] 

\noindent
A word formed from the initial letters of other words. Hence as v. trans., to convert into an acronym (chiefly pass. and as pa. pple.). 
Also acronymic a.; acronymically adv.; acronyming vbl. n.; acronymize v. trans. 

\P 1943 \textit{Amer. N. \& Q.} Feb. 167/1 Words made up of the initial letters or syllables of other words‥I have seen‥called by the name acronym.    
\P 1947 \textit{Word Study} 6 (title) Acronym Talk, or ‘Tomorrow's English’.    
\P 1947 \textit{Word Study} May 6/2 Some new forms combine the initial syllables (resembling blends) instead of initial letters, as in the case of Amvets (American Veterans' Association)‥but they still are in the spirit of acronyming.    Ibid. 7/2 There has definitely been a speed-up in ‘acronyming’.    
\P 1950 S. POTTER \textit{Our Language} 163 Acronyms or telescoped names like nabisco from National Biscuit Company.    
\P 1954 \textit{Britannica Bk. of Yr. 1954 } 638/1 Typical of acronymic coinages, or words based on initials, were‥mash (Mobile Army Surgical Hospital).    
\P 1956 R. WELLS in M. Halle et al. \textit{For Roman Jakobson} 665 Take the WE counterpart of the SE expression to be acronymized (North Atlantic Treaty Organization), and select from each word the first one or two or three letters in such a way that the selected letters, assembled and regarded as one word, will have a normal, pronounceable SE counterpart.    
\P 1967 \textit{Sci. News} 19 Aug. 177/1 The TacSatCom, as it is acronymed, is a small-scale system which should be in the field soon.    
\P 1971 \textit{Daily Tel.} 3 Feb. 12 Has the Establishment realised, inquires an acronymically-minded reader, that if the Industrial Relations Bill becomes law, it will not be only Ireland that is saddled with an IRA?    
\P 1972 \textit{Sat. Rev.} (U.S.) 3 June 30 Nitrogen oxide, acronymed NOx, is another of the plant's noxious by-products.    
\P 1981 \textit{Amer. Speech} LVI. 65 Byte is a fairly far-fetched way of acronymizing binary digit eight.    
\P 1981 \textit{Maledicta} V. 99 Who were the real ‘ethnics’, acronymically speaking?    
\P 1983 \textit{Verbatim} Spring 2/2 Paulies play puck (ice hockey) or hoop (basketball, also acronymed to b-ball).


%%%%%%%%%%%%%%%%%%%%%%%%%%%%%
\myitem{adage }  n.

\noindent  \phonetic{(ˈædɪdʒ) }

\noindent  
[a. Fr. adage, ad. L. adagium a proverb, f. ad to + *agi- root of ajo = agio I say. (Fick I. 481.) A by-form was adagy.] 

\noindent ‘A maxim handed down from antiquity; a proverb.’ J. 

\P 1548 HALL \textit{Chron. Edw. IV}, an. 9, 209 He forgat the olde adage, saynge in tyme of peace prouyde for warre.    
\P 1593 SHAKES. \textit{3 Hen. VI,} i. iv. 126 Vnlesse the Adage must be verifi'd, That Beggers mounted, runne their Horse to death.    
\P 1605 \textit{Macb.} i. vii. 45 Letting, I dare not, wait vpon I would, Like the poore Cat i'th'Addage.    
\P 1642 HOWELL \textit{For. Trav.} 25 Every Nation hath certain Proverbs and Adages peculiar to it selfe.
\P a1733 NORTH \textit{Lives of Norths} (1826) II. 355 According to the philosophic adage, omnes stulti insaniunt, all fools are out of their wits.    
\P 1847 BARHAM \textit{Ingol. Leg.} (1877) 6 That truest of adages—‘Murder will out.’    
\P 1872 JENKINSON \textit{Guide to Eng. Lakes} (1879) 189 Tourists in their anxiety to cut off a corner are sometimes induced to cross the valley, but‥discover the truth of the adage ‘most haste, least speed.’


%%%%%%%%%%%%%%%%%%%%%%%%%%%%%
\myitem{admonitory }  a.

\noindent  \phonetic{(ædˈmɒnɪtərɪ) }

\noindent  
[ad. L. admonitōri-us; see admonitor and -y.] 

\noindent Of or pertaining to an admonitor; giving or conveying admonition; warning. 

\P 1594 HOOKER \textit{Eccl. Pol.} i. §8 (J.) The sentence of reason is either mandatorie‥or else permissiue‥or thirdly, admonitorie.    
\P 1679 in Somers \textit{Tracts} I. 44 This little Admonitory Address.    
\P 1818 SCOTT \textit{Hrt. Midl.} 279 The clergyman‥fixed upon her a glance, at once steady, compassionate, and admonitory.    
\P 1865 DICKENS \textit{Mut. Fr.} xi. 254 A raised admonitory finger.

%%%%%%%%%%%%%%%%%%%%%%%%%%%%%
\myitem{adroit }  a.

\noindent  \phonetic{(əˈdrɔɪt) }

\noindent  
[a. Fr. adroit, orig. adv. phrase à droit according to right, rightly, properly, f. à to + droit right, OFr. dreit:—late L. drictum, dirictum:—cl. L. directum right: see direct. Subseq. used as adj., and in this sense adopted in Eng.] 

\noindent Possessing address or readiness of resource, either bodily or mental; having ready skill, dexterous, active, clever. 

\P 1652 EVELYN \textit{France} (R.) The best esteemed and most adroit cavalry in Europe.    
\P 1678 BUTLER \textit{Hudibras} iii. i. 365 He held his talent most adroit, For any mystical exploit.    
\P 1718 \textit{Free-thinker} No. 150, 326 The Right-Hand and Arm of most Men are‥more adroit than the Left.    
\P 1809 W. IRVING \textit{Knickerb.} xi. vii. (1849) 122 The adroit bargain by which the island of Manhattan was bought for sixty guilders.    
\P 1825 \textit{Br. Jonathan} I. 269 They played about one another now like adroit wrestlers.    
\P 1860 MOTLEY \textit{Netherl.} (1868) II. xiii. 139 Adroit intriguers burned incense to him as to a god.

%%%%%%%%%%%%%%%%%%%%%%%%%%%%%
\myitem{adulation }  n.

\noindent  \phonetic{(ˌædjuːˈleɪʃən) }

\noindent  
[a. OFr. adulacion, ad. L. adūlātiōn-em, n. of action f. adūlā-ri: see adulate.] 

\noindent  
Servile flattery or homage; exaggerated and hypocritical praise to which the bestower consciously stoops. 

\P c1380 CHAUCER \textit{Bal. Good Counsail} (R.) Men woll‥call faire speache adulacion.    
\P 1429 \textit{Pol. Poems} (1859) II. 145 Eschew flatery and adulacioun.    
\P 1538 BALE \textit{Thre Lawes} 964 By fayned flatterye, and by coloured adulacyon.    
\P 1582 N.T. (Rhem.) \textit{1 Thess.} ii. 5 For neither haue we been at any time in the word of adulation, as you know.    
\P 1599 SHAKES. \textit{Hen. V}, iv. i. 271 Thinks thou the fierie Feuer will goe out With Titles blowne from Adulation?    
\P 1766 GOLDSM. \textit{Vic. W.} iii. 18 Adulation ever follows the ambitious, for such alone receive pleasure from flattery.    
\P 1858 O.W. HOLMES \textit{Aut. Brkf. Table} xii. 115, I have two letters on file; one is a pattern of adulation, the other of impertinence.


%%%%%%%%%%%%%%%%%%%%%%%%%%%%%
\myitem{adversity }  n.

\noindent  \phonetic{(ædˈvɜːsɪtɪ) }

\noindent  
[a. MFr. adversité, refash. f. OFr. aversite:—L. adversitāt-em opposition, contrariety, f. adversus: see adverse and -ity.] 
\vspace{-0.3cm}

\begin{myenumerate}
\itembf{1.} The state or condition of being contrary or opposed; opposition, contrariety. Obs. 

\P 1382 WYCLIF \textit{Ps.} iii. 8 For thou hast smyte all doende adversite [1388 beynge adversaries] to me with oute cause.
\P a1420 HOCCLEVE \textit{De Reg. Princ.} 390, I was agast fulle sore of the, Leste thow thurghe thoughtfulle adversitee Not hadest stonden in the feithe aright.
\P 1450 LONELICH \textit{Grail} xviii. 174 One bone, sire kyng, þat thow graunte me Withowten lettynge owthir adversite.

\itembf{2.} The condition of adverse fortune; a state opposed to well-being or prosperity; misfortune, distress, trial, or affliction. (The earliest sense in Eng.) 

\P c1230 \textit{Ancren Riwle} 194 Þe uttre uondunge is mislicunge in aduersite.    
\P 1340 \textit{Ayenb.} 27 Kuead of aventure, ase povertie oþer adversitie.    
\P 1483 CAXTON \textit{Gold. Leg.} 399/4 Thenne late us praye‥that he so gouerne us bytwene welth \& aduersyte in this present lyf.    
\P 1535 COVERDALE \textit{Prov.} xvii. 17 In aduersite a man shall know who is his brother [1611 A brother is borne for aduersitie].    
\P 1570–87 HOLINSHED \textit{Scot. Chron.} (1806) I. 81 Adversitie findeth few friends.    
\P 1592 SHAKES. \textit{Rom. \& Jul.} iii. iii. 55 Aduersities sweete milke, Philosophie.    
\P 1600 \textit{A.Y.L.} ii. i. 12 Sweet are the vses of aduersitie.    
%\P 1750 JOHNSON \textit{Rambler} No. 150 ⁋5 He that never was acquainted with adversity has seen the world but on one side.    
\P 1750 JOHNSON \textit{Rambler} No. 150 \cardo{⁋}5 He that never was acquainted with adversity has seen the world but on one side.    
\P 1771 JUNIUS \textit{Lett.} xlix. 254 A virtuous man, struggling with adversity, [is] a scene worthy of the gods.
\P a1852 D. WEBSTER \textit{Wks.} 1877, III. 341 The discipline of our virtues in the severe school of adversity.

\itembf{3.} An adverse circumstance; a misfortune, calamity, trial. 

\P 1340 \textit{Ayenb.} 84 Þe kueades and þe aduersetes of þe wordle.
\P c1386 CHAUCER \textit{Clerk's T.} 551 Noon accident for noon aduersitee Was seyn in hire.    
\P 1483 CAXTON \textit{Cato} b ij. b, Strengthe for to resiste ageynst all aduersytees.    
\P 1526 TINDALE \textit{Acts} vii. 10 And God was with him, and delivered hym out off all his adversities.    
\P 1651 HOBBES \textit{Leviathan} ii. xxxi 188 The Prosperities and Adversities of this life.    
\P 1842 LONGFELLOW \textit{Sp. Stud.} ii. i. 1 Pray, tell me more of your adversities.

\itembf{4.} Contrariness of nature; perversity. (In Shak. = perverse one, quibbler.) Obs. 

\P 1489 CAXTON \textit{Faytes of Armes} iii. ix. 186 The felawes muste be chaunged by som aduersyte that is in them.    
\P 1606 SHAKES. \textit{Tr. \& Cr.} v. i. 14, P. Who keepes the Tent now? T. The Surgeons box, or the Patients wound. P. Well said aduersity.
\end{myenumerate}


%%%%%%%%%%%%%%%%%%%%%%%%%%%%%
%\myitem{ægis}  n.
\myitem{aegis}  n.

\noindent  \phonetic{(ˈiːdʒɪs) }

\noindent  
[L. ægis, a. Gr. αἰγίς, of uncert. etym.; see Liddell and Scott, s.v.] 
\vspace{-0.3cm}

\begin{myenumerate}
\itembf{1.} A shield, or defensive armour; applied in ancient mythology to that of Jupiter or Minerva. 

\P 1704 ROWE \textit{Ulysses} iii. i. 1128 She [Pallas] shakes her dreadful Ægis from the Clouds.    
\P 1760 HOME \textit{Siege of Aquileia} iv, His adamantine ægis Jove extends.    
\P 1812 BYRON \textit{Ch. Har.} ii. xiv, Where was thine Ægis, Pallas, that appalled Stern Alaric?

\itembf{2.} fig. A protection, or impregnable defence. Now freq. in senses ‘auspices, control, etc.’, esp. in phr. under the ægis (of). 

\P 1793 HOLCROFT \textit{Lavater's Physiog.} xxix. 137 Feeling is the ægis of enthusiasts and fools.    
\P 1836 THIRLWALL \textit{Greece} III. xviii. 83 They were sheltered by the ægis of the laws.    
\P 1865 LECKY \textit{Rationalism} (1878) II. 323 He cast over them the ægis of his own mighty name.    
\P 1910 \textit{Encycl. Brit.} III. 936/2 Under the aegis of the Billiard Association a tacit understanding was arrived at that the position must be broken up, should it occur.    
\P 1958 P. GAMMOND \textit{Duke Ellington} i. 18 They make their valuable individual contributions, but under the Ellington aegis they find themselves constantly enriched musically.    
\P 1963 \textit{B.S.I. News} May 14/2 These basic criteria and recognized methods of assessment, drawn up under the aegis of BSI.

\itembf{3.} attrib. and Comb., ægis-bearing, ægis-orb. 

\P 1793 WORDSWORTH \textit{Even. Walk} 69 The broadening sun appears; A long blue bar its ægis orb divides.    
\P 1877 BRYANT \textit{Odyss.} v. 128 The purposes Of Ægis-bearing Jove.
\end{myenumerate}

%%%%%%%%%%%%%%%%%%%%%%%%%%%%%
\myitem{affable }  a.

\noindent  \phonetic{(ˈæfəb(ə)l) }

\noindent  
[a. Fr. affable (14th c. in Litt.) ad. L. affābilis easy to be spoken to; f. affāri or adfāri to address; f. ad to + fāri to speak.] 
\vspace{-0.3cm}

\begin{myenumerate}
\itembf{a.} Easy of conversation or address; civil and courteous in receiving and responding to the conversation or address of others—especially inferiors or equals; accostable, courteous, complaisant, benign. (Const. to comparatively recent.) 

\P 1540 WHITTINTON \textit{Tullyes Offyce} I. 50 Ulysses‥wolde shewe hym selfe to all persones effable and gentyll to speake vnto.    
\P 1545 JOYE \textit{Expos. Dan.} xi. (R.) He was prudent, comely, princely, affable, ientle and amiable.    
\P 1596 SHAKES. \textit{1 Hen. IV}, iii. i. 168 Valiant as a Lyon, and wondrous affable.    
\P 1610 B. JONSON \textit{Alchem.} ii. iii. (1616) 628 [She is] the most affablest creatur, sir! so merry!    
\P 1667 MILTON \textit{P.L.} vii. 42 Raphaël, The affable archangel.    
\P 1723 J. SHEFFIELD (Dk. Buckhm.) \textit{Wks.} (1753) I. 53 Gentle his look, and affable his mien.    
\P 1876 FREEMAN \textit{Norm. Conq.} II. vii. 27 When not stirred up by passion he was gentle and affable to all men.

\itembf{b.} Formerly used more loosely. Obs. 

\P 1622 MALYNES \textit{Anc. Law-Merch.} 501 The judiciall and affable judgements of this age.    
\P 1641 MILTON \textit{Ch. Govt.} ii. (1851) 148 The learned and affable meeting of frequent Academies.    
\P 1709 STEELE \textit{Tatler} No. 101 \cardo{⁋}5 A Country Foxhunter‥shall in a Week's Time look with a courtly and affable Paleness.
\end{myenumerate}

%%%%%%%%%%%%%%%%%%%%%%%%%%%%%
\myitem{aficionado }  n.

\noindent  \phonetic{(afiθioˈnaðo; anglicized as əfɪsɪəˈnɑːdəʊ)}

\noindent  
[Sp. = amateur, f. pa. pple. of aficionar to become fond of, f. afición affection.] 

\noindent  
A devotee of bull-fighting; by extension an ardent follower of any hobby or activity. 

\P 1845 R. FORD \textit{Handbk. Travellers in Spain} i. ii. 178 This sham fight is despised by the torero and aficionado, who aspire only to be in at the death.    
\P 1902 W.D. HOWELLS \textit{Lit. \& Life} iii. 58 The last [bull] was uncommonly fierce, and when his hindquarters came off or out, his forequarters charged joyously among the aficionados on the prisoners' side.    
\P 1957 \textit{Times} 12 Oct. 7/6 The bull-fight is the most Spanish of spectacles.‥ Some old aficionados will go so far as to say that it is dying.
   
\P 1882 C.G. LELAND \textit{Gypsies} 25 The aficionados, or Romany ryes, by whom I mean those scholars who are fond of studying life and language from the people themselves.    
\P 1928 F.O. LINDLEY \textit{Diplomat off Duty} iv. 64 All amateurs, or to use a much more expressive Spanish word, aficionados, of bathing agree that the full flavour of the pastime is only tasted in beautiful surroundings.    
\P 1948 J. STEINBECK \textit{Russian Jrnl.} (1949) iii. 37 A little swing band was led by Ed Gilmore, who is a swing aficionado.    
\P 1957 \textit{Technology} Apr. 70/3 The aficionados of science fiction and golf.    
\P 1959 J. WAIN \textit{Trav. Woman} 41, I didn't tell you I had a son who was an aficionado of railways, did I?


%%%%%%%%%%%%%%%%%%%%%%%%%%%%%
\myitem{affinity }  n.

\noindent  \phonetic{(əˈfɪnɪtɪ) }

\noindent  
[a. Fr. afinité, affinité, ad. L. affīnitāt-em, n. of state f. affīn-is: see affine n.] 
\vspace{-0.3cm}

\begin{myenumerate}
\itembf{I.} Affinity by position. 

\itembf{1. a.} Relationship by marriage; opposed to consanguinity. Hence collect. Relations by marriage. 

\P 1303 R. BRUNNE \textit{Handl. Sinne} 7379 Or ȝyf he wyþ a womman synne Þat
sum of hys kyn haþ endyde ynne‥He calleþ hyt an affynyte.
\P c1315 SHOREHAM 70 Alle here sybbe affinitè.    
\P 1483 CAXTON \textit{G. de la Tour} C viij b, Be he of his parente his affynyte or other.    
\P 1509 FISHER \textit{Wks.} 1876, 293 What by lygnage what by affinite she had xxx. kinges \& quenes within the iiii. degre of maryage vnto her.    
\P 1649 SELDEN \textit{Laws Eng.} i. lv. (1739) 98 Many‥ that by affinity and consanguinity were become English-men.    
\P 1726 AYLIFFE \textit{Parergon} 326 Affinity is a Civil Bond of Persons, that are ally'd unto each other by Marriage or Espousals.    
\P 1849 MACAULAY \textit{Hist. Eng.} I. 172 He was closely related by affinity to the royal house. His daughter had become, by a secret marriage, Duchess of York.

\itembf{b.} In R.C. Ch.: The spiritual relationship between sponsors and their godchild, or between the sponsors themselves, called in older English gossip-red (cf. kin-red). 

\P c1440 \textit{Relig. Pieces fr. Thornton MS.} (1867) 13 His sybb frendes or any oþer þat es of his affynyte gastely or bodyly.    
\P 1751 CHAMBERS \textit{Cycl.} s.v., The Romanists talk of a spiritual Affinity, contracted by the sacrament of baptism and confirmation.    
\P 1872 FREEMAN \textit{Hist. Ess.} (ed. 2) 23 When he has succeeded in placing the bar of spiritual affinity between the King and his wife.

\itembf{2.} Relationship or kinship generally between individuals or races. collect. Relations, kindred. 

\P 1382 WYCLIF \textit{Ruth} iii. 13 If he wole take thee bi riȝt of affynyte the thing is wel doo.    
\P 1440 J. SHIRLEY \textit{Dethe of K. James} (1818) 7 With many other of thare afinite.    
\P 1494 FABYAN iv. lxx. 49 He therfore with helpe of his affynyte and frendes, withstode the Romaynes.    
\P 1677 GALE \textit{Crt. Gentiles} i. i. ix. 47 The great Identitie, or at least, Affinitie that was betwixt the old Britains, and Gauls.    
\P 1794 G. ADAMS \textit{Nat. \& Exp. Philos.} III. xxxii. 316 The labour of individuals‥weaves into one web the affinity and brotherhood of mankind.    
\P 1872 YEATS \textit{Growth \& Viciss.} Comm. 37 The affinities of the people which connected them‥with the Semitic races of Arabia.

\itembf{3.} Philol. Structural resemblance between languages arising from and proving their origin from a common stock. 

\P 1599 THYNNE \textit{Animadv.} (1865) 66 The latyne, frenche, and spanyshe haue no doble W, as the Dutche, the Englishe, and suche as have affynytye with the Dutche.    
\P 1659 PEARSON \textit{Creed} (1839) 245 We know the affinity of the Punic tongue with the Hebrew.    
\P 1796 MORSE \textit{Amer. Geog.} I. 80 Between some of these languages, there is indeed a great affinity.    
\P 1859 JEPHSON \textit{Brittany} xx. 313 To trace the affinities of words in different languages.

\itembf{4.} Nat. Hist. Structural resemblance between different animals, plants, or minerals, suggesting modifications of one primary type, or (in the case of the two former) gradual differentiation from a common stock. 

\P 1794 SULLIVAN \textit{View of Nat.} I. 458 Thus we shall find that antimony has an affinity with tin.    
\P 1830 LYELL \textit{Princ. Geol.} (1875) II. iii. xxxiv. 250 The species are arranged‥with due regard to their natural affinities.    
\P 1862 DARWIN \textit{Orchids} iii. 115 In the shape of the labellum we see the affinity of Goodyera to Epipactis.    
\P 1872 NICHOLSON \textit{Palæont.} 353 The true Reptiles and the Birds‥are nevertheless related to one another by various points of affinity.

\itembf{5.} fig. Causal relationship or connexion (as flowing the one from the other, or having a common source), or such agreement or similarity of nature or character as might result from such relationship if it existed; family likeness. 

\P 1533 ELYOT \textit{Castel of Helth} (1541) 35 By reason of the affinitie whiche it hath with mylke, whay is convertible in to bloude and fleshe.    
\P 1540 MORYSINE tr. \textit{Vives Introd. Wysdome} C iiij, Vyces and their affynities, as foolyshnes, ignorancy, amased dulnesse.    
\P 1642 R. CARPENTER \textit{Experience} iii. v. 46 What is the reason that Grace hath such marvellous affinity with Glory?    
\P 1795 MASON \textit{Ch. Mus.} i. 76 The sound of every individual instrument bears a perfect affinity with the rest.    
\P 1855 H. REED \textit{Lect. Eng. Lit.} ii. (1878) 74 Philosophy and poetry are for ever disclosing affinities with each other.    
\P 1861 TULLOCH \textit{Eng. Purit.} iv. 421 This spiritual affinity between Luther and Bunyan is very striking.

\itembf{6.} Neighbourhood, vicinity. [OFr. afinité.] Obs. 

\P 1678 R. RUSSELL tr. \textit{Geber} iv. ii. 242 The third Property is Affinity (or Vicinity) between the Elixir and the Body to be transmuted.    
\P 1770 HASTED in \textit{Phil. Trans.} LXI. 161 Some kinds of wood‥decay by the near affinity of others.

\itembf{II.} Affinity by inclination or attraction. 

\itembf{7.} Voluntary social relationship; companionship, alliance, association. Obs. 

\P 1494 FABYAN v. ciii. 78 Gonobalde‥promysed ayde to his power. Lotharius, of this affynyte beyng warned, pursued the sayde Conobalde.    
\P 1580 NORTH \textit{Plutarch} (1676) 4 That so many good men would have had affinity with so naughty and wicked a man.    
\P 1611 BIBLE \textit{2 Chron.} xviii. 1 Now Jehosaphat‥ioyned affinitie with Ahab.

\itembf{8.} Hence fig. A natural friendliness, liking, or attractiveness; an attraction drawing to anything. 

\P 1616 SURFLET \& MARKH. \textit{Countrey Farme} 322 For this dung, by a certaine affinitie, is gratefull and well liked of Bees.    
\P 1652 FRENCH \textit{Yorksh. Spa} viii. 71 With this hath the spirit of the Spaw water great affinity.    
\P 1832 H. MARTINEAU \textit{Each \& All} iv. 61 Natural affinities are ever acting, even now, in opposition to circumstance.    
\P 1860 MAURY \textit{Phys. Geog. Sea} ii. §70 So sharp is the line, and such the want of affinity between those waters.

\itembf{9.} esp. Chemical attraction; the tendency which certain elementary substances or their compounds have to unite with other elements and form new compounds. 

\P 1753 CHAMBERS \textit{Cycl. Supp.} s.v., M. Geoffroy has given [in 1718] a table of the different degrees of affinity between most of the bodies employed in chemistry.    
\P 1782 KIRWAN in \textit{Phil. Trans.} LXXIII. 35 Chymical affinity or attraction is that power by which the invisible particles of different bodies intermix and unite with each other so intimately as to be inseparable by mere mechanical means.    
\P 1831 T.P. JONES \textit{Convers. Chem.} i. 22 Elective affinity, or elective attraction, you will find spoken of in every work upon chemistry.
\P c1860 FARADAY \textit{Forces of Nat.} iii. 93 This new attraction we call chemical affinity, or the force of chemical action between different bodies.

\itembf{10.} A psychical or spiritual attraction believed by some sects to exist between persons; sometimes applied concretely to the subjects or objects of the ‘affinity.’ 

\P 1868 DIXON \textit{Spir. Wives} I. 99 All these Spiritualists accept the doctrine of special affinities between man and woman; affinities which imply a spiritual relation of the sexes higher and holier than that of marriage.    Ibid. II. 204 Such natures as, on coming near, lay hold of each other, and modify each other, we call affinities.

\itembf{III.} Special Comb. affinity group U.S., a group or association of people sharing a common purpose or interest; spec. one allowed certain privileges when chartering an aeroplane. 

\P 1970 \textit{Hearings Subcomm. Transportation of Comm. Interstate \& Foreign Commerce} (91st. U.S. Congress 2 Sess.) 8 Legitimate *affinity groups, the American Legion, the American Bar Association, the Knights of Columbus [etc.].    
\P 1976 \textit{Time} 19 Jan. 62 No longer does the traveler have to belong to a so-called affinity group, such as a club or union, to qualify for the reduced rates.    
\P 1984 \textit{Amer. Banker} 22 June 4 Insurance companies increasingly look to third-party channels for marketing their products. They include sponsored markets, such as employers and associations; affinity groups in banks and real estate enterprises; [etc.].
\end{myenumerate}

%%%%%%%%%%%%%%%%%%%%%%%%%%%%%
\myitem{aggrandize }  v.

\noindent  \phonetic{(ˈægrændaɪz) }

\noindent  
[f. Fr. agrandiss- extended stem of agrand-ir (16th c. aggr-), prob. ad. It. aggrandire; f. ag- = ad- to + grandire, L. grandīre to make great; f. grandis large. The ending is assimilated to words of Gr. origin with -ize.] 
\vspace{-0.3cm}

\begin{myenumerate}
\itembf{1.} trans. To enlarge, increase, magnify, or intensify (a thing). 

\P 1634 T. HERBERT \textit{Trav.} 7 (T.) The devil has infused prodigious idolatry into their hearts, enough to relish his palate and aggrandize their tortures.    
\P 1656 EARL OF MONMOUTH  \textit{Advt. fr. Parnass.} 48 Making use of the calamities of others, as an instrument thereby to agrandize his authority.    
\P 1748 ANSON \textit{Voy.} i. viii. (ed. 4) 110 That no circumstance might be wanting which could aggrandize our distress.    
\P 1855 BAIN \textit{Senses \& Intell.} iii. ii. §11 The whole soul, passing into one sense, aggrandizes that sense and starves the rest.    
\P 1868 RUSKIN \textit{Pol. Econ. Art} i. 80 The selfish and tyrannous means they commonly take to aggrandize or secure their power.

\itembf{2.} To increase the power, rank, or wealth of (a person or a state). Often refl. 

\P 1682 BURNET \textit{Rights of Princes Pref.} 3 For the aggrandizing or maintaining his nephews and kindred.    
\P 1780 W. COXE \textit{Russ. Discov.} 22 Every circumstance which contributes to aggrandize the Russian empire.    
\P 1800 WELLINGTON in \textit{Gen. Desp.} I. 207 If we aggrandize ourselves at the expense of the Mahrattas.    
\P 1872 YEATS \textit{Growth \& Viciss. Comm.} 96 Venice was aggrandised by this traffic.

\itembf{3.} To make (a thing) appear greater; to give a character of grandeur to; to embellish, exaggerate. 

\P 1687 \textit{Death's Vis.} (1713) Pref. 2 'Tis pleaded, that Religion aggrandizes a Poem.    
\P 1775 T. WARTON \textit{Hist. Eng. Poetry} I. 53 Nothing could aggrandise Fingal's heroism more highly.    
\P 1779 JOHNSON \textit{L.P., Pope} Wks. 1787 IV. 119 The ship-race, compared with the chariot-race, is neither illustrated nor aggrandised.    
\P 1848 H. MILLER \textit{First Impr.} ix. (1857) 144 The scene, though small, is yet aggrandized with much art.

\itembf{4.} To make (a person) appear greater; to exalt. 

\P 1753 RICHARDSON \textit{Grandison} (1781) III. xviii. 161 Your pretty imagination is always at work to aggrandize the man, and to lower the babies.    
\P 1823 LAMB \textit{Elia Ser.} ii. xxiv. (1865) 433 The first thing to aggrandise a man in his own conceit, is to conceive of himself as neglected.

\itembf{5.} intr. To become greater; to increase. Obs. Cf. Fr. s'agrandir. 

\P 1646 HALL \textit{Poems} 8 Follies continued till old age, do aggrandize and become horrid.    
\P 1704 \textit{Lond. Gaz.} mmmmlxxiv/2 Could not but with Horrour see him aggrandize in Power.
\end{myenumerate}

%%%%%%%%%%%%%%%%%%%%%%%%%%%%%
\myitem{alacrity } n. 

\noindent  \phonetic{(əˈlækrɪtɪ) }

\noindent  
[ad. L. alacritāt-em, n. of quality f. alacer brisk (also in It. alacrità): see -ty.] 

\noindent  
Briskness, cheerful readiness, liveliness, promptitude, sprightliness. 

\P c1510 MORE \textit{Picus} Wks. 1557, 8/1 That meruelouse alacritee languished.    
\P 1594 SHAKES. \textit{Rich. III,} v. iii. 73, I haue not that alacrity of spirit, Nor cheere of Minde that I was wont to haue.    
\P 1687 T. BROWN \textit{Saints in Uproar} Wks. 1730 I. 79 With what wonderful alacrity you scamper'd over the Alps.    
\P 1710 STEELE \textit{Tatler} No. 34 \cardo{⁋}2 It immediately gives an Alacrity to the Visage and new Grace to the whole Person.    
\P 1791 COWPER \textit{Il.} v. 145 She wing'd him with alacrity divine.    
\P 1820 SCOTT \textit{Monast.} xv. 98 He accepted with grateful alacrity.

%%%%%%%%%%%%%%%%%%%%%%%%%%%%%
\myitem{allay }  v.

\noindent  \phonetic{(əˈleɪ) }

\noindent  
\phonetic{
[f. a- prefix 1 + lay, OE. lęcᴁan, causal of licᴁan to lie. OE. alęcᴁan (cogn. w. Goth. uslagjan, OHG. irleccan, mod.G. erlegen) was inflected : Imper. aleᴁe, alecᴁað; Ind. pres. ic alecᴁe, þú aleᴁest, he aleᴁ(e)þ, we alecᴁað; pa. tense aleᴁde, aléde; Pa. pple. aleᴁd, aléd; whence ME. aleggen (əˈlɛdʒən); aleye, alaye (əˈlɛɪə, əˈleɪə, əˈleɪ), aleggeþ; I alegge, þou aleyest, he aleyeþ, we aleggeþ or aleggen; aleyde; aleyd, -eid, -ayd, -aid; levelled c 1400, by substitution of aleye for alegge all through; as inf. to aleyen, alaye(n, alay(e; subsequently mis-spelt allay, after words from L. in all- (see ad- 2). In its two forms, alegge and aleye, this vb. was formally identical with 4 other vbs. of Romance origin; viz. 1. alegge, allege v.1:—L. alleviāre; 2 alaye, allay v.2:—L. alligāre; 3. aleye, allay v.3:—L. allēgāre; 4. alegge, allege v.2 = OFr. alléguer, L. allēgāre, a learned form of allay v.3 Amid the overlapping of meanings that thus arose, there was developed a perplexing network of uses of allay and allege, that belong entirely to no one of the original vbs., but combine the senses of two or more of them. Those in allay are placed at the end of this word.] 
}
\vspace{-0.3cm}

\begin{myenumerate}
\itembf{I.} Unmixed senses: To lay from one, lay aside or down; put down; put down the proud, pride, tumult, violence; to quell, abate. 

\itembf{1.} To lay, lay down, lay aside. Obs. 

\P c970 \textit{Canons of K. Edgar in Anc. Laws} II. 286 \phonetic{Alecᴁe} þonne his wæpna.
\P c1000 \textit{Ags. G.} Luke ii. 16 \phonetic{Hiᴁ} \phonetic{ᴁemetton}‥ðæt cild on binne aléd.
\P c1160 \textit{Hatton G.} ibid., Gemetton þæt chyld on binne \phonetic{aleiᴁd}.

\itembf{2.} To lay aside (a law, custom, practice); hence, to set aside, annul, abolish, destroy the legal force of (anything). Obs. 

\P c1175 \textit{Lamb. Hom.} 91 Þenne beoð eowre sunnen aleide.    Ibid. 115 He scal wicche creft aleggan.    
\P 1205 LAYAM 7714 Þurh þa luue of þan feo feond-scipe aleggen.    
\P 1297 R. GLOUC. 144 Gode lawes, þat were aleyd, newe he lette make.    
\P c1350 \textit{Will. Palerne} 5240 Þan william wiȝtli‥a-leide alle luþer lawes.    
\P 1413 LYDG. \textit{Pylgr. Sowle} iv. xxxvi. (1483) 84 Worshyp is aleyde and neuer shal retourne.

\itembf{3.} To abandon, give up (a course of action). Obs. 

\P a1330 \textit{Sir Otuel} 38 Bi me he sente the to segge, Thou sscoldest Christendom alegge.    
\P c1380 \textit{Sir Ferumb.} 3300 Hot þat þyn assaut be noȝt aled ‹revsc› and let by-gynne hit newe.

\itembf{4.} To put down, bring low, quell (a person). Obs. 

\P c1000 ÆLFRIC \textit{Josh.} x. 13 \phonetic{Hiᴁ} aledon heora fynd.    
\P c1175 \textit{Lamb. Hom.} 91 Ic alegge þine feond under þine fot-sceomele.    
\P c1300 in Wright \textit{Lyric P.} xxxvii. 105 Alle thre shule ben aleyd, with huere foule crokes.    
\P 1387 TREVISA \textit{Higden Rolls} Ser. III. 237 [The Greeks] schulle be aleyde [obruentur] wiþ the multitude of Perses.

\itembf{5.} To put down or overthrow (a principle or attribute of men). Obs. 

\P a1000 \textit{Sec. Laws of Cnut} (Thorpe I. 380) Unriht \phonetic{alecᴁan}.    
\P c1200 \textit{Trin. Coll. Hom.} 11 Unbileue is aiware aleid, and rihte leue arered.
\P c1300 \textit{Beket} 1928 Forto awreke ous wel of him ·and alegge his prute.    
\P c1440 \textit{Arthur} 219 Thy pryde we wolle alaye.    
\P 1593 SHAKES. \textit{2 Hen. VI,} iv. i. 60, I, and alay this thy abortiue Pride.    
\P 1642 ROGERS \textit{Naaman} 205 Wherby carnall reason is somewhat alaied and abated.    
\P 1659 PEARSON \textit{Creed} (1839) 88 Sufficiently refuting an eternity, and allaying all conceits of any great antiquity.

\itembf{6.} To put down by argument, confute, overthrow. Obs. rare. 

\P a1250 \textit{Owl \& Night.} 394 Heo ne miȝte noȝt alegge That the hule hadde hire i-sed.

\itembf{7.} To cause to lie, to lay (dust, etc.). Obs. rare. 

\P 1642 FULLER \textit{Holy \& Prof. St.} v. xiv. 413 That in Noahs floud the dust was but sufficiently allayed.

\itembf{8.} To put down or repress (any violence of the elements, as heat, wind, tempest); to calm, assuage, ‘lay’ a storm. (This and the next sense are perhaps influenced by allege v.1: see 11 below.) 

\P 1488 CAXTON \textit{Chastys. Goddes Chyld.} 12 Hete is thenne ful colde and alayed.    
\P 1580 BARET \textit{Alv.} A 282 The tempest is alaied.    
\P 1610 SHAKES. \textit{Temp.} i. ii. 2 If by your Art (my deerest father) you haue Put the wild waters in this Rore; alay them.    
\P 1781 J. MOORE \textit{Italy} (1790) I. ii. 23 One of the virtues of the holy water [is] that of allaying storms.    
\P 1847 DISRAELI \textit{Tancred} iii. iv. (1871) 183 The fervour of the air was allayed.    
\P 1862 TRENCH \textit{Mirac.} iv. 147 Having allayed the tumult of the outward elements.

\itembf{9.} To quell or put down (any disturbance in action or any tumult of the passions); to appease. 

\P c1380 \textit{Sir Ferumb.} 1373 Y-blessed mot þou be, For aled þow hast muche debate.    
\P 1387 TREVISA \textit{Higden Rolls} Ser. IV. 293 Forto alegge þe outrage of þe kyngdom of Jewes.    
\P 1600 FAIRFAX \textit{Tasso} xix. xx. 340 Tancred‥Asswag'd his anger and his wrath alaid.    
\P 1623 BINGHAM \textit{Xenophon} 35 To allay, if he could, these distrusts, before they broke out into open hostilitie.    
\P 1697 DRYDEN \textit{Virg. Georg.} iv. 131 This deadly Fray, A Cast of scatter'd Dust will soon allay.    
%\P 1711 ADDISON \textit{Spect.} No. 16 ⁋4 If I can any way asswage private Inflammations, or allay publick Ferments.    
\P 1711 ADDISON \textit{Spect.} No. 16 \cardo{⁋}4 If I can any way asswage private Inflammations, or allay publick Ferments.    
\P 1855 PRESCOTT \textit{Philip II,} I. ii. xi. 265 The best means of allaying the popular excitement.    
\P 1863 KINGLAKE \textit{Crimea} (1876) I. xiv. 236 Words tending to allay suspicion.    
\P 1880 MCCARTHY \textit{Own Time} III. xxxii. 48 Various efforts were made to allay the panic.

\itembf{10.} intr. (for refl.) To subside, sink, abate, cease; to become mild. Obs. 

\P 1526 TINDALE \textit{Mark} iv. 39 And the wynde alayed. [So 1557 (Genev.).]    
\P 1561 HOLLYBUSH \textit{Hom. Apoth.} 33 a, For as $\sim$ sone as the stomake perceyveth the savoure of the bread, then doth the wambling alaye.    
\P 1593 SHAKES. \textit{3 Hen. VI,} i. iv. 146 And, when the Rage allayes, the Raine begins.    
\P 1723 WODROW \textit{Corr.} (1843) III. 78 If there were any room to hope that your hearts were allaying.

\itembf{II.} Confused with allege v. to lighten or alleviate, both verbs being in 14th c. alegge, and both used of pains, etc., so that alegge peine was in the one sense = quell pain, in the other = alleviate pain. Both senses might be expressed by abate, and they came to be regarded as the same word, so that from c 
1400 alaye was used for alegge in both (cf. Caxton's ‘t’ alegge thurste,' see allege v. 2, Gower's ‘to allay thurst’); and finally alegge became obs., and allay remained with the combined meaning. 

\itembf{11.} To subdue, quell (any trouble, as care, pain, thirst); to abate, assuage, relieve, alleviate. 

\P c1220 \textit{Ureisun Ure Lefdi} 133 Þu miht lihtliche‥al mi sor aleggen.    
\P 1250 LAY 25684 Al þis lond he wole for-fare ‹revsc›bote þou alegge oure care.
\P 1393 GOWER \textit{Conf.} III. 11 Which may his sory thurst allay.    Ibid. III. 273 If I thy paines mighte alaie.    
\P 1578 LYTE \textit{Dodoens} 341 The roote Rhodia‥alayeth head ache.    
\P 1667 MILTON \textit{P.L.} x. 566 Fondly thinking to allay Thir appetite.    
\P 1681 WYNDHAM \textit{King's Concealm.} 76 The pleasantness of the Host‥allayed and mitigated the weariness of the Guests.    
\P 1768 BEATTIE \textit{Minstrel} ii. xxxii, I would allay that grief.    
\P 1836 MACGILLIVRAY tr. \textit{Humboldt's Trav.} xix. 283 These Indians swallow quantities of earth for the purpose of allaying hunger.

\itembf{III.} Confused with allay v.2, to alloy, mix, temper, qualify. The two verbs were from the 15th c. completely identical in form, and thus in appearance only different uses of the same word. (The earlier of the following senses are more closely related to the next vb. than to this; but it is, on the whole, more convenient to place them here, than under a word which is obs. or arch. in its own proper sense.) 

\itembf{12.} To temper (iron, steel, etc.) Obs. 

\P 1409 \textit{Roll for Building Durham Cloisters,} Pro alayng secur', chyselle, wegges.    
\P 1486 \textit{Bk. St. Albans} (1810) h iij, Ye shall put the quarell in a redde charkcole fyre tyll that it be of the same colour that the fyre is. Thenne take hym oute and lete hym kele, and ye shall find him well alayd for to fyle.

\itembf{13.} To temper or abate (a pleasure or advantage) by the association of something unpleasant. 

\P 1514 BARCLAY \textit{Cyt. \& Uplondyshm.} 48 Because one service of them continuall Allayeth pleasure.
\P a1670 HACKET in Wolcott \textit{Life} (1865) 175 If the comfort of our joy be not allayed with some fear.    
\P 1759 JOHNSON \textit{Rasselas} xxvi. (1787) 71 Benefits are allayed by reproaches.    
\P 1796 MORSE \textit{Amer. Geog.} I. 310 The principal circumstance that allayed the joys of victory.    
\P 1839 HALLAM \textit{Hist. Lit.} III. iii. iii. §131. 115 But this privilege is allayed by another, i.e. by the privilege of absurdity.

\itembf{14.} To dilute, qualify (wine with water, etc.). Obs. 

\P c1450 J. RUSSELL \textit{Bk. Nurt. in Babees Bk.} (1868) 132 Watur hoot \& cold, eche oþer to alay.    
\P 1470 HARDING \textit{Chron.} lxxii, He vsed the water ofte to alaye His drynkes.    
\P 1533 ELYOT \textit{Cast. Helth} (1541) 32 White wyne alayd with moche water.    
\P 1655 CULPEPER \textit{Riverius} xv. v. 419 Clysters‥made of Vinegar allaied with Water.    
\P 1676 HOBBES \textit{Odyss.} ix. 212 Which when he drank, he usually allaid With water pure.

\itembf{15.} fig. Obs. 

\P 1586 T.B. tr. \textit{La Primaudaye's Fr. Acad.} Ded., To alay the strength of the word of Christ with the waterish sayings and fables of men.    
\P 1650 FULLER \textit{Pisgah Sight} iv. vii. 125 God‥allaying the purity of his nature, with humane Phrases.

\itembf{16.} To abate, diminish, weaken, mitigate. 

\P 1603 FLORIO \textit{Montaigne} (1634) 624 To allay or dim the whitenesse of paper.    
\P 1628 PRYNNE \textit{Cens. Cozens} 96 This pretence‥will not mittigate nor allay his Crime.    
\P 1748 CHESTERFIELD \textit{Lett.} 166 II. 111 Neither envy, indignation, nor ridicule, will obstruct or allay the applause which you may really deserve.    
\P 1805 FOSTER \textit{Ess.} ii. iv. 169 They must allay their fire of enterprise.    
\P 1842 H. ROGERS \textit{Introd. Burke's Wks.} 59 To allay and temper its splendour down to that sober light which may enable his audience to see his argument.
\end{myenumerate}

%%%%%%%%%%%%%%%%%%%%%%%%%%%%%
\myitem{allegory }  n.

\noindent  \phonetic{(ˈælɪgərɪ) }

\noindent  
[ad. L. allēgoria, a. Gr. ἀλληγορία, lit. speaking otherwise than one seems to
speak, f. ἄλλος other + -ἄγορία speaking; cf. ἀγορεύω to speak, orig. to
harangue, f. ἀγορά the public assembly. Cf. Fr. allégorie, perh. the direct
source of the Eng. The L. allegoria was occas. used unchanged in 16th c.] 
\vspace{-0.3cm}

\begin{myenumerate}
\itembf{1.} Description of a subject under the guise of some other subject of aptly
suggestive resemblance. 

\P 1382 WYCLIF \textit{Gal.} iv. 24 The whiche thingis ben seid by allegorie, or goostly
vndirstondinge [Vulg. per allegoriam].    
\P 1477 EARL RIVERS (Caxton) \textit{Dictes} 66 The sayd Platon dide teche his sapyence by allegorye.    
\P 1589 PUTTENHAM \textit{Eng. Poesie} (1869) 196 Properly and in his principall vertue Allegoria is when we do
speake in sence translatiue and wrested from the owne signification,
neuerthelesse applied to another not altogether contrary, but hauing much
conueniencie with it.    
\P 1712 PARNELL \textit{Spect.} No. 501 \cardo{⁋}1 Some of the finest
compositions among the ancients are in allegory.    
\P 1840 CARLYLE \textit{Heroes} (1858)
207 Allegory and Poetic Delineation, as I said above, cannot be religious Faith.

\itembf{b.} attrib. 

\P 1532 MORE \textit{Confut. Tindale} Wks. 1557, 415/1 These heretikes nowe not onely rob
the churche in an allegorye sense. Answ. Frith 835/1 The wordes of Chryste
might beside the lyttarall sence bee vnderstanden in an allegorye.

\itembf{2.} An instance of such description; a figurative sentence, discourse, or
narrative, in which properties and circumstances attributed to the apparent
subject really refer to the subject they are meant to suggest; an extended or
continued metaphor. 

\P 1534 MORE \textit{On the Passion} Wks. 1557, 1340/1 It might be taken for an allegory
or some other trope or figure.    
\P 1577 T. VAUTROLLIER tr. \textit{Luther's Ep. Gal.} 149
The allegorie of the two sonnes of Abraham, Isaacke and Ismael.    
\P 1611 BIBLE \textit{Gal.} iv. 24 Which things are an Allegorie.    
\P 1751 JOHNSON \textit{Rambl.} No. 176 \cardo{⁋}11
They discover in every passage‥some artful allegory.    
\P 1846 T. WRIGHT \textit{Mid. Ages}
II. xix. 257 The spirited and extremely popular political allegory of the
‘Vision of Piers Ploughman.’

\itembf{3.} An allegorical representation; an emblem. 

\P a1639 W. WHATELY \textit{Protot.} i. xi. (1640) 154 These two mothers and the
children borne of them were allegories, that is, figures of some other thing
mystically signified by them.    
\P 1769 BURKE \textit{State Nat.} Wks. II. 134
Procrustes‥with his iron bed, the allegory of his government.    
\P 1882 MRS. PITMAN \textit{Mission Life in Greece} 30 That Hercules is only an allegory of the sun.
\end{myenumerate}

%%%%%%%%%%%%%%%%%%%%%%%%%%%%%
\myitem{alleviate }  v.

\noindent  \phonetic{(əˈliːvɪeɪt) }

\noindent  
[f. prec. ‘Reckoned by Heylin, in 1656, among uncouth and unusual words.’ Todd.] 
\vspace{-0.3cm}

\begin{myenumerate}
\itembf{1.} To make lighter, diminish the weight of. Obs. 

\P 1665-6 \textit{Phil. Trans.} I. 157 Such as have exact Wheel-Barometers may try whether Odors or Fumes do alleviate the Air.

\itembf{2.} To lighten, or render more tolerable, or endurable; to relieve, mitigate. Also absol. 

\P 1528 PAYNELL tr. \textit{Salernes Regiment} 22 Milk‥alleviateth the griefes of the breast.
\P a1656 BP. HALL \textit{Balm of Gil.} i. §ii. (1863) 6 To alleviate the sorrows of their heavy partners.    
\P 1712 STEELE \textit{Spect.} No. 450 \cardo{⁋}3, I‥found means to alleviate, and at last conquer my Affliction.    
\P 1871 G.H. NAPHEYS \textit{Prevent. Dis.} iii. ii. 619 To alleviate the sufferings of the invalid.    
\P 1876 MOZLEY \textit{Univ. Serm.} v. 120 Hope alleviates the sorrow of that home.    
\P 1888 MRS. H. WARD \textit{R. Elsmere} xli, The constant effort to serve and to alleviate.

\itembf{3.} To lighten the gravity of (an offence); to extenuate, palliate. Obs. 

\P 1768 BLACKSTONE \textit{Comm.} IV. 15 The violence of passion, or temptation, may sometimes alleviate a crime.    
\P 1777 R. WATSON \textit{Philip II} (1793) II. xiv. 181 They began to alleviate the outrages of the soldiers.
\end{myenumerate}

%%%%%%%%%%%%%%%%%%%%%%%%%%%%%
\myitem{alliteration}  n.

\noindent  \phonetic{(əˌlɪtəˈreɪʃən) }

\noindent  
[n. of action f. alliterate v.: see -ation.] 
\vspace{-0.3cm}

\begin{myenumerate}
\itembf{1.} gen. The commencing of two or more words in close connexion, with the same letter, or rather the same sound. 

\P 1656 BLOUNT \textit{Glossogr., Alliteration,} a figure in Rhetorick, repeating and playing on the same letter.    
\P 1749 POWER \textit{Pros.} Numbers 71 That which some call Alliteration, i.e. beginning several Words with the same Letter, if it be natural, is a real Beauty.    
\P 1763 CHURCHILL \textit{Proph. Famine Poems} I. 101 Apt Alliteration's artful aid.    
\P 1831 MACAULAY \textit{Johnson} 126 Taxation no Tyranny‥was‥nothing but a jingling alliteration which he ought to have despised.    
\P 1871 R.F. WEYMOUTH \textit{Euph.} 4 ‘Delightful to be read, and nothing hurtfull to be regarded; wherein there is small offence by lightnes given to the wise, and lesse occasion of loosenesse profferred to the wanton.’ Lilie's favourite form of alliteration is well marked in this sentence.

\itembf{2.} The commencement of certain accented syllables in a verse with the same consonant or consonantal group, or with different vowel sounds, which constituted the structure of versification in OE. and the Teutonic languages generally. Thus from the beginning of Langland's Piers Ploughman, text C.:

\begin{verse}
In a somere seyson · whan softe was þe sonne,  \\
Y shop me into shrobbis · as y a shepherde were;   \\
In abit as an ermite · vnholy of werkes,  \\
Ich wente forth in þe worlde · wonders to hure,   \\
And sawe meny cellis · and selcouthe þynges.
\end{verse}
   
\P 1774 T. WARTON \textit{Eng. Poetry} (1840) I. Diss. i. 38 The Islandic poets are said to have carried alliteration to the highest pitch of exactness.    
\P 1846 T. WRIGHT \textit{Ess. Mid. Ages} I. i. 14 The form of Saxon poetry is alliteration—not rhyme.    
\P 1871 EARLE \textit{Philol. Eng. Tong.} §626 Alliteration did not necessarily act on the initial letter of the word.
\end{myenumerate}

%%%%%%%%%%%%%%%%%%%%%%%%%%%%%
\myitem{amanuensis }  n.

\noindent  \phonetic{(əˌmænjuːˈɛnsɪs) }

\noindent  
[L. (in Suetonius) adj. used subst., f. denominative phrase a manu a secretary, short for servus a manu + -ensis belonging to.] 

\noindent  
One who copies or writes from the dictation of another. 

\P 1619 SCLATER \textit{Expos. Thess.} (1627) I. To Reader 6 An Amanuensis to take my Dictates.    
\P 1621 BURTON \textit{Anat. Mel.} Democr. 11 Allowing him six or seven amanuenses to write out his dictates.    
\P 1714 \textit{Spect.} No. 617 \cardo{⁋}4 Our Friend‥by the help of his Amanuensis, took down all their Names.    
\P 1765 TUCKER \textit{Lt. Nat.} II. 446 Cæsar could dictate to three amanuenses together.    
\P 1860 SMILES \textit{Self-Help} ii. 38 For many years after their marriage, she acted as his amanuensis.

%%%%%%%%%%%%%%%%%%%%%%%%%%%%%
\myitem{ambience }  

\noindent  \phonetic{(ˈæmbɪəns) }

\noindent  
[f. ambient a.: see -ence; cf. F. ambiance.] 
\vspace{-0.3cm}

\begin{myenumerate}
\itembf{1.} Environment, surroundings; atmosphere. 
   The Fr. form ambiance is used in Art for the arrangement of accessories to support the main effect of a piece. 

\P 1889 \textit{Harper's Mag.} Sept. 500/2 The form which we discern in the dreamy ambience is of supreme elegance.    
\P 1902 W. WATSON \textit{Ode on Coronation of King Edward VII} 5 Slowly in the ambience of this crown Have many crowns been gathered.    
\P 1923 R.H. MYERS \textit{Mod. Music} iv. 47 No other composer has ever reproduced in music with such complete success the very perfume and ambiance of a literary text.    
\P 1944 \textit{Burlington Mag.} June 156/1 But the present picture was never meant to be microscopically dissected thus, for it is‥an impression, a single figure in its ambiance, which is vaguely suggested as reflections in a mirror.    
\P 1952 \textit{Ballet Ann.} VI. 25 The costumes and sets‥have such a suggestion of space that they give the Sadler's Wells stage the ambiance of Covent Garden.    
\P 1957 \textit{London Mag.} Jan. 52 For some writers the urban ambience may provide just the kind of stimulus they need.    
\P 1961 \textit{Listener} 5 Oct. 527/2 The Zoo provides a colourful ambience for this Administrative Novel [sc. Angus Wilson's ‘Old Men at the Zoo’].    
\P 1965 \textit{N. \& Q.} CCX. 15/1 The way in which the poet by the use of the traditional vocabulary gives the impression that he was introducing his heroine into a Germanic ambience.
 
\itembf{2.} Audio. The acoustic quality of a particular environment, as reproduced in a recording; spec. a sense of some specific or individual atmosphere, esp. an impression of live performance, created or enhanced by recording techniques (such as added reverberation), or by the presence of background noise. 

\P 1961 G.A. BRIGGS \textit{A to Z in Audio} 15 For domestic use a reasonable amount of ambience in most records is desirable to give the listener a sensation of being in the concert hall, but too much blurs the fine detail.    
\P 1971 \textit{Hi-Fi Sound} Feb. 71/3 The shape and the furnishing of the listening room, modifying the ambience that is built-in by the recording engineer, can broaden and smudge the stereo image.    
\P 1977 \textit{Gramophone} Sept. 512/2 In quadraphony the back speakers make the contribution‥of added ambience, to add a subtle extra dimension to the realism of the orchestral image at the front.    
\P 1986 \textit{Electronic Musician} Aug. 29/1 How would you like to beef up the sounds you already have by adding a software-controllable dose of ambience or punch?    
\P 1993 \textit{Rolling Stone} 14 Oct. 54/2 We were starting to lose trust in the conventional sound of rock and roll‥, those big beautiful pristine vocal sounds with all this lush ambience and reverb.

\itembf{3.} = ambient music s.v. *ambient a. 3 c. 

\P 1991 \textit{Vox} Sept. 66/3 If we were to talk Brian Eno, The Blue Nile and ambience, the picture would be clearer‥. This is dance music without the wild agitation.    
\P 1995 \textit{Face} Jan. 47/1 Ambience doesn't have its immediate roots in the chill-out rooms of danceterias: its connection to the original vision of Cage and Eno is far more explicit.
\end{myenumerate}


%%%%%%%%%%%%%%%%%%%%%%%%%%%%%
\myitem{ambiguous }  a.

\noindent  \phonetic{(æmˈbɪgjuːəs) }

\noindent  
[f. L. ambigu-us doubtful, driving hither and thither (f. ambig-ĕre, f. amb- both ways + ag-ĕre to drive) + -ous.] 

\noindent  
The objective meanings, though second in Latin, seem earliest in Eng. 
\vspace{-0.3cm}

\begin{myenumerate}
\itembf{I.} Objectively. 

\itembf{1.} Doubtful, questionable; indistinct, obscure, not clearly defined. 

\P 1528 MORE \textit{Heresyes} iv. Wks. 1557, 247/2 If it wer nowe doutful \& ambiguous whether the church of Christ wer in the right rule of doctrine or not.    
\P 1573 MURRAY \textit{Let. in Wodrow Soc. Misc.} (1844) 289 Cairfull for the gude ordour of the Kirk in thingis ambiguouss.
\P c1800 K. WHITE \textit{Contempl.} 133 Faint ambiguous shadows fall.    
\P 1851 RUSKIN \textit{Mod. Paint.} I. ii. 2. v. §10 Even the most dexterous distances of the old masters‥are ambiguous.

\itembf{2.} Of words or other significant indications: Admitting more than one interpretation, or explanation; of double meaning, or of several possible meanings; equivocal. (The commonest use.) 

\P 1532 MORE \textit{Confut. Tindale} Wks. 1557, 437/1 This englishe word knowledge is ambiguous and doubtfull.    
\P 1589 PUTTENHAM \textit{Eng. Poesie} (1869) 267 The ambiguous, or figure of sence incertaine, as if one should say Thomas Tayler saw William Tyler dronke, it is indifferent to thinke either th'one or th'other dronke.    
\P 1671 MILTON \textit{P.R.} i. 435 Answers‥dark, Ambiguous, and with double sense deluding.    
\P 1752 JOHNSON \textit{Rambl.} No. 192 \cardo{⁋}8 The gentlemen‥irritated me with ambiguous insults.    
\P 1853 MAURICE \textit{Proph. \& Kings} xvii. 288, I do not rest anything upon tenses. Every reader of the prophets must feel how ambiguous they are.    
\P 1867 A.J. ELLIS \textit{E.E. Pronunc.} i. i. 25 The Welsh alphabet‥having only one ambiguous letter, y.

\itembf{3.} Of doubtful position or classification, as partaking of two characters or being on the boundary line between. 

\P 1603 FLORIO \textit{Montaigne} (1634) 294 Mungrell and ambiguous shapes.    
\P 1667 MILTON \textit{P.L.} vii. 473 Ambiguous between sea and land The river-horse and scaly crocodile.    
\P 1756 HUME \textit{Hist. Eng.} II. xx. 20 His character became fully known. and was no longer ambiguous to either faction.    
\P 1839 MURCHISON \textit{Silur. Syst.} 418 Stratified rocks of ambiguous character.

\itembf{II.} Subjectively. 

\itembf{4.} Of persons: Wavering or uncertain as to course or conduct; hesitating, doubtful. Obs. 

\P 1550 NICOLS \textit{Thucyd.} 175 (R.) People that be ambiguous or doubtefulle.    
\P 1649 MILTON \textit{Eikon.} 239 Thus shall they be too and fro, doubtfull and ambiguous in all thir doings.

\itembf{5.} Of things: Wavering or uncertain in direction or tendency; of doubtful or uncertain issue. 

\P 1612 SHELTON \textit{Don Quix.} I. ii. v. 90 That she do favour and protect him in that ambiguous Trance which he undertakes.    
\P 1813 SCOTT \textit{Rokeby} i. xii, The eddying tides of conflict wheeled Ambiguous.    
\P 1850 MRS. BROWNING \textit{Prometh. Bound} Poems I. 184 Do not cast Ambiguous paths, Prometheus, for my feet.

\itembf{6.} Hence, Insecure in its indications; not to be relied upon. 

\P 1756 BURKE \textit{Subl. \& B.} Wks. 1842 I. 26 The taste, that most ambiguous of the senses.

\itembf{7.} Of persons, oracles, etc.: Using words of doubtful or double meaning. 

\P 1566 KNOX \textit{Hist. Ref.} Wks. 1846 I. 370 To no point wald sche answer directlie; bot in all thingis sche was‥ambigua.
\P a1700 DRYDEN (J.) Th' ambiguous god, who rul'd her lab'ring breast.
\P a1725 POPE \textit{Odyss.} i. 490 Antinous‥Constrain'd a smile and thus ambiguous spoke.    
\P 1864 SWINBURNE \textit{Atalanta} 1500 What mutterest thou with thine ambiguous mouth.
\end{myenumerate}

%%%%%%%%%%%%%%%%%%%%%%%%%%%%%
\myitem{ambivalent }  a.

\noindent  \phonetic{(æmˈbɪvələnt) }

\noindent  
[f. ambivalence, after equivalent a.] 

\noindent  
Of, pertaining to, or characterized by ambivalence; having either or both of two contrary or parallel values, qualities or meanings; entertaining contradictory emotions (as love and hatred) towards the same person or thing; acting on or arguing for sometimes one and sometimes the other of two opposites; equivocal. (a) In Psychology. 

\P 1916 C.E. LONG tr. \textit{Jung's Analytical Psychol.} vi. 200 Tendencies, under the stress of emotions, are balanced by their opposites—thus giving an ambivalent character to their expression.    
\P 1920 P.M. BLANCHARD \textit{Adolescent Girl} (1921) v. 125 A second case where the falsehoods were‥the result of ambivalent desire for and fear of the erotic life.    
\P 1922 J. RIVIERE tr. \textit{Freud's Introd. Lect. Psycho-Analysis} ii. xv. 194 The coincidence of opposites in the dream-work is analogous to what is called the antithetical sense of primal words in the oldest languages. The philologist, R. Abel‥begs us not‥to imagine that there was any ambiguity in what one person said to another by means of ambivalent words of this sort.    
\P 1924 A.A. BRILL tr. \textit{Bleuler's Textbk. Psychiatry} ii. 126 It is chiefly ambivalent complexes that influence pathology.    
\P 1954 \textit{Listener} 30 Sept. 523/2 Our deeper urges are strangely ambivalent, ready to spend themselves on love or hate, altruism or destruction.

\noindent (b) In literary and general use. 

\P 1929 B. RUSSELL \textit{Marriage \& Morals} xiii. 140 Christianity‥has always had an ambivalent attitude towards the family.    
\P 1939 L. TRILLING \textit{M. Arnold} iv. 123 The story of ambivalent love is a characteristic one of the 19th century.    
\P 1947 C.S. LEWIS \textit{Miracles} xiv. 151 Death is‥what some modern people would call ‘ambivalent’. It is Satan's great weapon and also God's great weapon; it is holy and unholy; our supreme disgrace and our only hope.    
\P 1957 D.J. ENRIGHT \textit{Apothecary's Shop} 196 Where Rilke is concerned‥Auden's attitude in his poetry is ambivalent. He cannot help disapproving the application, but‥he cannot help praising the technique.    
\P 1958 A.E. DYSON in \textit{Ess. \& Stud.} 53 Irony is‥the most ambivalent of modes, constantly changing colour and texture.    
\P 1958 J. PRESS \textit{Chequer'd Shade} v. 93 Some readers obviously derive from poetry which they do not comprehend a peculiar, ambivalent pleasure.    
\P 1963 \textit{Times Lit. Suppl.} 15 Feb. 103/2 Ambivalent-seeming relations with his brilliant Eton tutor.    
\P 1965 \textit{Camb. Rev.} 20 Feb. 273/1 A Ph.D. is a somewhat ambivalent acquisition: it is not always clear whether it is mentioned as a positive desideratum or a last resort.

%%%%%%%%%%%%%%%%%%%%%%%%%%%%%
\myitem{ambulatory }  a.

\noindent  \phonetic{(ˈæmbjʊlətərɪ) }

\noindent  
[ad. L. ambulātōri-us of or pertaining to a walker, f. ambulātor, q.v.; cf. Fr. ambulatoire.] 
\vspace{-0.3cm}

\begin{myenumerate}
\itembf{1.} Of or pertaining to a walker, or to walking. 

\P 1622 HEYLYN \textit{Cosmogr.} iii. (1682) 129 Being at his ambulatory Exercise.    
\P 1796 MORSE \textit{Amer. Geog.} II. 83 The ambulatory life of herdsmen and shepherds.    
\P 1874 HELPS \textit{Soc. Press.} iv. 63 When that man has an object, it is astonishing what ambulatory powers he can develop.

\itembf{2.} Adapted or fitted for walking. 

\P 1835 KIRBY \textit{Habits \& Inst. An.} II. xvi. 84 The thoracic legs‥become also its ambulatory legs.    
\P 1852 DANA \textit{Crustacea} i. 10 Feet ambulatory or prehensile.    
\P 1877 W. THOMSON \textit{Voy. Challenger} I. ii. 133 Leaf-like sacs‥which fringe the ambulatory disk.

\itembf{3.} Moving from place to place, having no fixed abode; movable. 

   
\P 1622 HOWELL \textit{Lett.} 5 Mar., His council of state went ambulatory always with him.    
\P 1649 JER. TAYLOR \textit{Gt. Exemp. Pref.} \cardo{⁋}25 They served the ends of God‥by their ambulatory life.    a 
\P 1703 BURKITT \textit{On N.T. Acts} vii. 50 The tabernacle was an ambulatory temple.    
\P 1845 R. HAMILTON \textit{Pop. Educ.} 191 Many [schools] are ambulatory, and‥are held only during four or five months in farm houses.    
\P 1858 GEN. P. THOMPSON  \textit{Audi Alt. Part.} I. xxv. 96 While the ambulatory guillotine was doing its work in the provinces.

\itembf{4.} fig. Shifting, not permanent, temporary, mutable. (So in L. and Fr.) ambulatory will: one capable of revocation. 

\P 1621–31 \textit{Laud Serm.} (1847) 73 Nor is this ceremony Jewish or ambulatory, to cease with the law.    
\P 1651 W.G. \textit{Cowel's Instit.} 133 A mans will‥according to the Civill Law is ambulatory, or alterable, untill Death.    
\P 1789 MRS. PIOZZI \textit{Fr. \& It.} II. 387 They learn to think virtue and vice ambulatory.    
\P 1832 J. AUSTIN \textit{Jurispr.} I. xxi. 452 Every intention‥which regards the future is ambulatory or revocable.

\itembf{5.a.} Path. and Med. = ambulant a. sense 3. 

\P 1857 DUNGLISON \textit{Med. Lex.} s.v., A morbid affection is said to be ‘ambulatory’‥when it skips from one part to another.    
\P 1882 QUAIN \textit{Dict. Med.} I. 38/1 Ambulatory, a term given to typhoid fever, showing that the patient is able to walk about during the attack.    
\P 1903 \textit{Westm. Gaz.} 21 Feb. 6/1 That the cause of death was ambulatory typhoid.    
\P 1947 L.K. FERGUSON \textit{Surg. Ambulatory Patient} (ed. 2) p. ix, Surgery of the ambulatory patient is the surgery performed more often by the younger men and general practitioners.    Ibid. i. 1 (heading) A survey of the field of ambulatory surgery.
 
\itembf{b.} Of places or apparatus: intended or suitable for ambulant patients. 

\P 1890 BILLINGS \textit{Med. Dict.} I. 47/1 Ambulatory clinic, clinic for persons able to walk about; a dispensary.    
\P 1973 \textit{Sci. Amer.} Sept. 29/3 The vast bulk of care is provided by physicians in ambulatory settings.    
\P 1978 B. PYM \textit{Very Private Eye} (1984) 317 Had an ambulatory electro-cardiogram attached to me for 24 hours.    
\P 1981 \textit{Times} 1 Dec. 15/7 The Tracker ambulatory recorder uses a standard C-90 tape cassette running at slow speeds to record a continuous electrocardiograph.    
\P 1990 \textit{Brain} CXIII. 1584 Subcutaneous administration of apomorphine by ambulatory minipump.
\end{myenumerate}

%%%%%%%%%%%%%%%%%%%%%%%%%%%%%
\myitem{ameliorate }  

\noindent  \phonetic{(əˈmiːlɪəreɪt) }

\noindent  
[a recent formation (not in Johnson 1773), after the earlier meliorate q.v., on Fr. améliorer, refashioned from OFr. ameillorer to make better, f. à to + meillorer:—L. meliōrāre, f. melior better.] 
\vspace{-0.3cm}

\begin{myenumerate}
\itembf{1.} trans. To make better; to better, improve. 

\P 1767 [See AMELIORATING].    
\P 1779 SWINBURNE \textit{Trav. Spain} xxxvi. (T.) The probability of their lot being so much ameliorated.    
\P 1813 SIR H. DAVY \textit{Agric. Chem.} 203 A sterile soil‥may be ameliorated by the application of quick lime.    
\P 1849 MACAULAY \textit{Hist. Eng.} I. 279 In every human being there is a wish to ameliorate his own condition.    
\P 1879 \textit{Quatrefages' Hum. Spec.} 70 Gardeners and breeders‥ameliorate‥the plants and animals in which they are interested.

\itembf{2.} intr. To grow better. 

\P 1789–96 Morse \textit{Amer. Geog.} I. 626 The state of things is rapidly ameliorating.    
\P 1882 GEIKIE in \textit{Macm. Mag.} Mar. 365/2 [Man]‥would find his way back as the climate ameliorated.
\end{myenumerate}


%%%%%%%%%%%%%%%%%%%%%%%%%%%%%
\myitem{amenable }  a.

\noindent  \phonetic{(əˈmiːnəb(ə)l) }

\noindent  
[apparently a. AFr. amenable (not in Godef.), f. amener to bring to or before, f. à to + mener to lead:—L. mināre to threaten, hence to drive cattle with minatory shouts. Cf. Sc. ca' = call and drive. The spelling amesnable is quite artificial, influenced by mesne, demesne, etc.] 
\vspace{-0.3cm}

\begin{myenumerate}
\itembf{1.} Of persons: Liable to be brought before any jurisdiction; answerable, liable to answer, responsible (to law, etc., or absol.). 

\P 1596 SPENSER \textit{State of Irel.} 100 Not amesnable to Law.    
\P 1662 FULLER \textit{Worthies} ii. 74 The inferiour sort of the Irish were‥not Amesnable by Law.    
\P 1691 BLOUNT \textit{Law Dict.,} Amenable, others write it amainable, from the Fr. main, a hand‥is applied in our Law Books to a Woman that is supposed governable by her Husband.    
\P 1769 JUNIUS \textit{Lett. Pref.} 12 The sovereign of this country is not amenable to any form of trial.    
\P 1810 COLERIDGE \textit{Friend} (ed. 3) II. 5 The sufficiency of the conscience to make every person a moral and amenable being.    
\P 1876 GRANT \textit{Burgh. Sch. Scotl.} i. i. 6 The Abbots of Dunfermline, to whom only he was amenable.

\itembf{2.} Of things: Liable to the legal authority of. 

\P 1768 BLACKSTONE \textit{Comm.} III. 413 Personal property, which is‥always amesnable to the magistrate.    
\P 1817 JAS. MILL \textit{Brit. India} II. v. ix. 697 All offences against the act were rendered amenable to the courts of law.

\itembf{3.} Hence loosely. Liable (to a charge, claim, etc.). 

\P 1863 MRS. C. CLARKE \textit{Shaks. Char.} xvii. 431 He is amenable to the charge of a host of vices.    
\P 1876 E. MELLOR \textit{Priesth.} vii. 312 The next witness‥is amenable to the same imputation of uncandid‥quotation.    
\P 1844 DICKENS \textit{Mart. Chuz.} (C.D. ed.) 270 Your property‥being amenable to all claims upon the company.

\itembf{4.} fig. Answerable at the bar of (any critical instrument): capable of being tested by. Const to. 

\P 1828 MILL \textit{Autobiogr.} (1924) 298 Make them amenable to the general tribunal of the public at large.    
\P 1843 \textit{Logic} I. ii. i. 216 Such of them [sc. assertions]‥as, not being amenable to direct consciousness or intuition, are appropriate subjects of proof.    
\P 1845 \textit{Ess.} II. 220 Historical facts are hardly yet felt to be‥amenable to scientific laws.    
\P 1867 BUCKLE \textit{Civilis.} III. v. 369 Amenable to the touch, but invisible to the eye.

\itembf{5.} Of persons and things: Disposed to answer, respond, or submit (to influence); responsive, tractable; capable of being won over. 

\P 1803 WELLINGTON in \textit{Gen. Disp.} II. 417 A high spirited people‥by no means amenable to discipline.    
\P 1861 MILL \textit{Utilitar.} iv. 60 Will‥is amenable to habit.    
\P 1874 SPURGEON \textit{Treas. David} lxxxii. i. IV. 40 Oriental judges are frequently‥amenable to bribes.    
\P 1878 E. WHITE \textit{Life in Christ} v. xxix. 496 Perplexed but amenable spirits whom sorrow and fear‥are drawing back to their Father.
\end{myenumerate}

%%%%%%%%%%%%%%%%%%%%%%%%%%%%%
\myitem{amenity }  n.

\noindent  \phonetic{(əˈmiːnɪtɪ, əˈmɛnɪtɪ) }

\noindent  
[? a. Fr. amenité (in Cotgr. 1611), or perh. direct ad. of its original L. amœnitāt-em, f. amœn-us pleasant: see amene and -ity.] 
\vspace{-0.3cm}

\begin{myenumerate}
\itembf{1.} The quality of being pleasant or agreeable: \textbf{a.} of places, their situation, aspect, climate, etc. 

\P 1432–50 tr. \textit{Higden} (1865) I. 77 That place hath also amenite.    
\P 1611 CORYAT \textit{Crudities} 448 For amenity of situation‥it doth farre excel all other cities.    
\P 1683 \textit{Brit. Spec.} 17 The amœnity and Utility of its Seas, Rivers and Ponds.    
\P 1832 J. AUSTIN \textit{Jurispr.} (1879) II. l. 858 The fiar may also cut and sell timber, so as not to injure the amenity.    
\P 1846 PRESCOTT \textit{Ferd. \& Is.} I. ii. 120 The superior amenity of the climate.

\itembf{b.} of persons, their habits, actions, etc. 

\P 1815 M. EDGEWORTH \textit{Patron.} xvii. 279 His manners wanted amenity, gaiety, and frankness.    
\P 1824 DIBDIN \textit{Libr. Comp.} 90 Who does not love the amenity of Erasmus?    
\P 1873 DIXON \textit{Two Queens} I. i. vii. 46 In amenity of life, his Court had been a Moorish rather than a Gothic Court.

\itembf{c.} In mod. use (freq. in pl.) applied to the more ‘human’ and pleasurable environmental aspects of a house, factory, town, etc., as distinguished from the features of the house, etc., considered in or by itself. Also concr. (usu. in sing.), a particular advantageous or convenient feature of this kind. Also attrib. Also, amenity bed (see quots.). (See also sense 3 b.) 

\P 1908 \textit{Royal Comm. Care Feeble-Minded, Min. Evid.} II. 63/1 Social Amenities. The experience we have gained emphasises the desirability of organised recreation.    
\P 1928 \textit{Britain's Industr. Future (Liberal Ind. Inquiry)} iv. xxiv. §9. 336 Amenity woodland definitely uneconomic.    
\P 1929 \textit{Oxford Times} 8 Feb. 13/4 The payment of £88 for the purchase of the land; the payment of £250 as compensation for the loss of amenities and disturbance of existing garden and grounds.    
\P 1936 \textit{Times} 2 Apr. 10/3 Repairable cottages of amenity value‥could be acquired and sympathetically repaired by the local authorities.    
\P 1951 \textit{Brit. Med. Jrnl.} 13 Oct. Suppl. 146/2 Amenity beds, for which the maintenance charge is almost negligible and medical service is given under the National Health Service Act.    
\P 1951 B.J. COLLINS \textit{Devel. Plans Explained} 42 Amenity, the quality which makes a desirable residence desirable, a favoured locality favoured, or enchanting views in all directions enchanting.    
\P 1952 \textit{Lancet} 2 Aug. 229/1 No privileges can be bought within the service (except the amenity bed, for which a relatively small charge has been made).    
\P 1957 \textit{Times} 12 Dec. 18/4 Arrangements were made to provide an amenity centre for the labour on Effingham and Seventh Mile Estates, the centre comprising a clubroom with cinema and a playing field.    
\P 1958 \textit{Times} 1 July i/3 There are, of course, many holdings below 20 acres, especially those that are part-time or amenity holdings where the earning of an income from the holding is not of great importance.    
\P 1958 \textit{Listener} 11 Sept. 368/2 Where the people themselves want a new amenity—a school, a meeting house, a road to link up with the outside world, [etc.].    
\P 1964 G.L. COHEN \textit{What's Wrong with Hospitals?} i. 23 ‘Amenity beds’‥were designated under the Act for patients who want more privacy and will pay extra for it.

\itembf{2.} Joyousness, exhilaration. Obs. rare. 

\P 1627 FELTHAM \textit{Resolves} ii. lxx. (1677) 307 The Amœnity and Floridness of the warm and spirited bloud.

\itembf{3.} concr. in pl. \textbf{a.} Pleasant places or scenes. (Cf. pleasance.) Obs. 

\P 1664 EVELYN \textit{Silva} (1776) 604 Arboreous Amenities and plantations of woods.    
\P 1671 \textit{Diary} (1827) II. 354 The suburbs are large, the prospects sweete, with other amenities.    
\P 1762 H. WALPOLE \textit{Vertue's Anecd. Paint.} (1786) IV. 140 A country so profusely beautified with the amænities of nature.

\itembf{b.} Pleasant ways or manners; pleasant pursuits, pleasures, delights, agreeable relations, civilities. 

\P 1841 D'ISRAELI \textit{(title)} Amenities of authors.    
\P 1860 MOTLEY \textit{Netherl.} (1868) I. v. 234 This interchange of dainties led the way to the amenities of diplomacy.    
\P 1866 \textit{Cornh. Mag.} Aug. 157 All the amenities of home life are wanting.    
\P 1883 \textit{Scotsman} 12 May 9/7 Talking amenities with Sir Stafford Northcote.
\end{myenumerate}


%%%%%%%%%%%%%%%%%%%%%%%%%%%%%
\myitem{amorous }  a.

\noindent  \phonetic{(ˈæmərəs) }

\noindent  
[a. OFr. amorous (mod.Fr. amoureux):—L. amōrōs-um, f. amōr love: see -ous.] 
\vspace{-0.3cm}

\begin{myenumerate}
\itembf{I.} actively. 

\itembf{1.} Of persons: Inclined to love; habitually fond of the opposite sex. Also fig. of things: Loving, fond. 

\P 1303 R. BRUNNE \textit{Handl. Synne} 7988 Þys was a prest ryȝt amerous, And amerous men are leccherous.    
\P 1393 GOWER \textit{Conf.} I. 304 Whiche of the two more amorous is Or man or wife.    
\P 1483 CAXTON \textit{Gold. Leg.} 90/1 Therfore saith the holy ghoost to the sowle that is amerouse.    
\P 1607 TOPSELL \textit{Four-footed Beasts} (1673) 341 The hairs layed to Womens lips, maketh them amorous.    
\P 1610 J. GUILLIM \textit{Displ. Herald.} iii. vii. (1660) 133 The Woodbine is a loving and amorous plant, which embraceth all that it growes near unto.    
\P 1616 R.C. \textit{Times' Whistle} vi. 2583 Doth captive the hart Of amarous ladies.    
\P 1728 YOUNG \textit{Odes to King} Wks. 1757 I. 177 Beneath them lies, With lifted eyes, Fair Albion, like an amorous maid.    
\P 1807 CRABBE \textit{Par. Reg.} ii. 405 Sir Edward Archer is an amorous knight.

\itembf{b.} with unto. Obs. rare. 
\P c1400 \textit{Destr. Troy} viii. 3926 Troilus was‥amirous vnto Maidens \& mony hym louyt.

\itembf{2.} Affected with love towards one of the opposite sex; in love, enamoured, fond. Also fig. of things (both as subject and object of love). \textbf{a.} absol. 
\P c1314 \textit{Guy Warw.} 37 Namore wostow of armes loue‥So amerous thou were anon right.    c 
\P 1385 CHAUCER \textit{L.G.W. } 1189 This amerous quien.
\P c1440 \textit{Gesta Rom.} ii. v. 285 The thirde knyght is wondir amerous, and lovethe you passyng well.    
\P 1596 SHAKES. \textit{Tam. Shr.} iii. i. 63 Our fine Musitian groweth amorous.    
\P 1647 COWLEY \textit{Bathing} iii. in \textit{Mistress} (1669) 79 The amorous Waves would fain about her stay.    
\P 1711 STEELE \textit{Spect.} No. 78 \cardo{⁋}4 The young Lady was amorous, and had like to run away with her Father's Coachman.    
\P 1822 W. IRVING \textit{Braceb. Hall} xix. 164 The amorous frog piped from among the rushes.

\itembf{b.} with on. Obs. 

\P c1386 CHAUCER \textit{Frankl. T.} 764 This squier On Dorigen that was so amorus.    
\P 1477 EARL RIVERS \textit{(Caxton) Dictes} 146 He was amerous on somme noble lady.    
\P 1599 SHAKES. \textit{Much Ado} ii. i. 161 Sure my brother is amorous on Hero.    
\P 1625 MILTON \textit{Death Fair Inf.} i, Being amorous on that lovely dye That did thy cheek envermeil.

\itembf{c.} with of. 

\P a1450 \textit{Knt. de la Tour} (1868) 168 There came another knyght which was also amerous of that lady.    
\P 1606 SHAKES. \textit{Ant. \& Cl.} ii. ii. 202 And made The water to follow faster, As amorous of their strokes.    
\P 1692 DRYDEN \textit{St. Euremont's Ess.} 212 One must be very amorous of a Truth, to search after it at that Price.    
\P 1821 KEATS \textit{Isabel} xix, Thy roses amorous of the moon.

\itembf{d.} with in: Delighting in. Obs. rare. 

\P a1674 CLARENDON \textit{Hist. Reb.} II. viii. 392 He was amorous in Poetry, and Musick, to which he indulged the greatest part of his time.

\itembf{3.} Of action, expression, etc.: Showing love or fondness; fond, loving. \textbf{a.} (sexual.) 

\P c1385 CHAUCER \textit{L.G.W.} 1102 Many an Amorouse [v.r. amorous, amorows] lokynge \& devys.    
\P 1493 \textit{Petronylla} (Pynson) 123 Nightyngalys with amerous notys clere Salueth Esperus.    
\P 1525 LD. BERNERS \textit{Froiss.} II. xxvi. 72 His eyen gray and amorous.    
\P 1605 SHAKES. \textit{Lear} i. i. 48 France \& Burgundy, Great Riuals in our yongest daughters loue, Long in our Court, haue made their amorous soiourne.    
\P 1750 JOHNSON \textit{Rambl.} No. 182 \cardo{⁋}7 Not being accustomed to amorous blandishments.    
\P 1863 B. TAYLOR \textit{Poet's Jrnl.} (1866) 54 Earth in amorous palpitation Receives her bridgegroom's kiss.

\itembf{b.} (general): Loving, affectionate, devoted, ardent. 

\P 1677 GALE \textit{Crt. Gentiles} II. iii. 64 Those amorose impetuosities that are in men and tend to pietie or impietie.    Ibid. 145 An amorous vehemence against sin.    
\P 1784 J. BARRY \textit{Lect. Art} v. (1848) 187 With attention and amorous assiduity.    
\P 1856 R. VAUGHAN \textit{Ho. w. Mystics} (1860) I. 65 The amorous quest of the soul after the Good.

\itembf{4.} Of or pertaining to (sexual) love. 

\P c1385 CHAUCER \textit{L.G.W.} 2616 Fful is the place‥Of songis amerous, of maryage.    
\P 1483 CAXTON \textit{Gold. Leg.} 31/2 The holy institucion of this amerous sacrament shold be the more honourably halowed.    
\P 1567 DRANT \textit{Horace Ep.} To Reader, So greate a scull of amarouse Pamphlets.    
\P 1592 SHAKES. \textit{Rom. \& Jul.} iii. ii. 8 Louers can see to doe their Amorous rights, And by their owne Beauties.    
\P 1635 SWAN \textit{Spec. Mundi} vi. §4 (1643) 266 Sow-bread‥is a good amorous medicine, and will make one in love.    
\P 1741 H. WALPOLE \textit{Lett. to H. Mann} 7 (1834) I. 23 The poor Princess and her conjugal and amorous distresses.    
\P 1809 W. IRVING \textit{Knickerb.} 75 To manhood roused, he spurns the amorous flute.    
\P 1846 PRESCOTT \textit{Ferd. \& Is.} I. viii. 373 Offered up his amorous incense on the altar of the Muse.

\itembf{II.} passively, Of persons and things: Lovable, lovely. Obs. 

\P c1400 \textit{Rom. Rose} 2901 It is thyng most amerous, For to aswage a mannes sorowe, To sene his lady by the morowe.    
\P 1535 STEWART \textit{Cron. Scot.} II. 37 His wyfe‥buir to him ane virgin amorus.    
\P 1557 \textit{Primer Sarum} D iij, O mother of God moste glorious, and amorous.    
\P 1567 \textit{Trial of Treas.} in Hazl. \textit{Dodsley} III. 288 O she is a minion of amorous hue.    
\P 1611 DEKKER \textit{Roaring Girle} 213, J. Here's most amorous weather, my Lord. Omnes. Amorous weather! J. Is not amorous a good word?

\itembf{b.} quasi-n. A lover; one in love. Obs. 

\P a1440 \textit{Sir Degrev.} 655 Sir Degrivaunt that amerus Had joye of that syȝth.    
\P 1491 CAXTON \textit{Vitas Patr.} (W. de Worde) i. xli. 62/2 How ofte she hath‥made fayre herself for to playse her amourouse or loues.
\end{myenumerate}

%%%%%%%%%%%%%%%%%%%%%%%%%%%%%
\myitem{amorphous }  a.

\noindent  \phonetic{(əˈmɔːfəs) }

\noindent  
[f. mod.L. amorphus, a. Gr. ἄµορϕ-ος shapeless (f. ἀ priv. + µορϕή form) + -ous. Cf. mod.Fr. amorphe.] 

Not in J. 
\begin{myenumerate}

\itembf{1.} Having no determinate shape, shapeless, unshapen; irregularly shaped, unshapely. 
   
\P 1731 BAILEY \textit{Amorphous}, without form or shape, ill-shapen.    
\P 1791 D'ISRAELI \textit{Cur. Lit.} (1866) 148/1 An amorphous hat, very much worn.    
\P 1831 CARLYLE \textit{Sart. Res.} (1858) 178 The enormous, amorphous Plum-pudding, more like a Scottish Haggis.    
\P 1870 LOWELL \textit{Among my Bks.} Ser. i. (1873) 203 That quality in man which‥gives classic shape to our own amorphous imaginings.    
\P 1878 BLACK \textit{Green Past.} xxxviii. 301 All three wore heavy and amorphous garments.

\itembf{b.} Belonging to no particular type or pattern; anomalous, unclassifiable. 

\P 1803 \textit{Phil. Trans.} XCIV. 38 This kind of attraction is either regular, irregular, or amorphous.    
\P 1845 CARLYLE \textit{Cromwell} (1871) I. 63 A morose, amorphous, cynical Law-pedant.

\itembf{2.} Min. \& Chem. Not composed of crystals in physical structure; uncrystallized, massive. 

\P 1801 BOURNON \textit{Arseniates} in \textit{Phil. Trans.} XCI. 171 The matrix‥siliceous; sometimes crystalline; and sometimes in an amorphous mass.    
\P 1842 W. GROVE \textit{Corr. Phys. Forces} (ed. 6) 84 An opaque amorphous state, as graphite or charcoal.    
\P 1870 TYNDALL \textit{Heat} xiii. §639 A fragment of almost black amorphous phosphorus.    
\P 1879 RUTLEY \textit{Stud. Rocks} x. 123 Augite often contains inclosures of amorphous glass.

\itembf{3.} Geol. Occurring in a continuous mass, without stratification, cleavage, or other division into similar parts. 

\P 1830 LYELL \textit{Princ. Geol.} I. 346 An amorphous mass passing downwards into lava, irregularly prismatic.    
\P 1853 PHILLIPS \textit{Rivers, etc. Yorksh.} iv. 124 These perishing cliffs show at the bottom the amorphous boulder-clay.

\itembf{4.} Biol. Without the definite shape or organization found in most higher animals and plants. 

\P 1848 DANA \textit{Zoophytes} 711 The structure was completely amorphous.    
\P 1868 WRIGHT \textit{Ocean W.} iv. 74 A sort of animated jelly, amorphous and diaphanous.    
\P 1877 ROBERTS \textit{Handbk. Med.} I. 51 Coagulated fibrin, either amorphous or fibrillated.

\itembf{5.} fig. Ill-assorted, ill-digested, unorganized. 

\P 1837 CARLYLE \textit{Fr. Rev.} (1872) III. iii. v. 121 An amorphous Sansculottism taking form.    
\P 1869 LECKY \textit{Europ. Mor.} I. i. 247 [Rome's] population soon became an amorphous, heterogeneous mass.
\end{myenumerate}

%%%%%%%%%%%%%%%%%%%%%%%%%%%%%
\myitem{anachronism }  n.

\noindent  \phonetic{(əˈnækrənɪz(ə)m) }

\noindent  
[a. Fr. anachronisme, ad. L. anachronism-us, a. Gr. ἀναχρονισµ-ός, n. of action f. ἀναχρονίζ-ειν to refer to a wrong time, f. ἀνά up, backwards + χρόν-ος time.] 
\vspace{-0.3cm}

\begin{myenumerate}
\itembf{1.} An error in computing time, or fixing dates; the erroneous reference of an event, circumstance, or custom to a wrong date. Said etymologically (like prochronism) of a date which is too early, but also used of too late a date, which has been distinguished as parachronism. 
\P a1646 J.G[REGORY] \textit{De Æris et Ep.} (1650) 174 An error committed herein [in a Synchronism] is called Anachronism.    
\P 1669 GALE \textit{Crt. Gentiles} i. iii. viii. 85 This error sprang from Anachronisme, and confusion of Histories.    
\P 1704 HEARNE \textit{Duct. Hist.} (1714) I. 7 Virgil making Dido and Æneas Co⁓temporaries, whereas they lived at Three Hundred Years distance‥committed an Anachronism.    
\P 1798 FERRIAR \textit{Eng. Histor.} 249 An anachronism of thirty or forty years‥is easily overlooked.    
\P 1856 MRS. STOWE \textit{Dred} (1856) I. Pref., Some anachronisms with regard to the time of the session of courts have been allowed.    
\P 1876 E. MELLOR \textit{Priesth.} iv. 172 The so-called literal interpretation involves an anachronism, inasmuch as it antedates the death of our Lord upon the cross.

\itembf{2.} Anything done or existing out of date; hence, anything which was proper to a former age, but is, or, if it existed, would be, out of harmony with the present; also called a practical anachronism. Also transf. of persons. 

\P 1816 COLERIDGE \textit{Lay Serm.} 329 If this one-eyed experience does not seduce its worshipper into practical anachronisms.    
\P 1859 JEPHSON \textit{Brittany} ix. 145 A pilgrimage now seems an anachronism.    
\P 1864 \textit{Round Table} 18 June 4/3 She gives them phrases and words which‥had their beginning long since that period, and are in fact linguistic anachronisms.    
\P 1871 \textit{Daily News} 15 Apr. 2 [The Benchers] would be living anachronisms in this age of progress, were it not that they are extremely fond of good eating.    
\P 1899 B. HARRADEN \textit{Fowler} i. vii, ‘Sentiment,’ she repeated. ‘It is absurd to try and hustle sentiment off the scenes.’‥‘You are always an anachronism,’ he said, quietly.    
\P 1952 M. MCCARTHY \textit{Groves of Academe} iii. 37 She herself was a smoldering anachronism, a throwback to one of those ardent young women of the Sixties, Turgenev's heroines.
\end{myenumerate}

%%%%%%%%%%%%%%%%%%%%%%%%%%%%%
\myitem{anagram }  n.

\noindent  \phonetic{(ˈænəgræm) }

\noindent  
[a. Fr. anagramme, or ad. mod.L. anagramma (16th c.), f. Gr. ἀνα-γράϕ-ειν, to write up, write back or anew. Ἀνάγραµµα was not in Greek, though the grammarians had ἀναγραµµατίζ-ειν to transpose the letters of a word, and ἀναγραµµατισµός transposition of letters.] 
\vspace{-0.3cm}

\begin{myenumerate}
\itembf{1.} A transposition of the letters of a word, name, or phrase, whereby a new word or phrase is formed. 

\P 1589 PUTTENHAM \textit{Eng. Poesie} (Arb.) 115 Of the Anagrame, or poesie transposed.    
\P 1609 B. JONSON \textit{Silent Wom.} iv. iii. (1616) 572 Who will‥make anagrammes of our names.    
\P 1632 HOWELL \textit{Lett.} (1650) I. 261 This Gustavus (whose anagram is Augustus) was a great Captain.    
\P 1705 HICKERINGILL \textit{Priest-Cr.} ii. iii. 36 The true Anagram of Jesuita, is Sevitia, Cruelty.    
\P 1865 CARLYLE \textit{Fredk. Gt.} II. vi. ii. 14 Monsieur Arouet Junior (le Jeune, or l. j.), who, by an ingenious anagram‥writes himself Voltaire ever since.

\itembf{2.} loosely or fig. A transposition, a mutation. Obs. 

\P 1634 HEYWOOD \textit{Maidenh. well Lost} xi. 119 What meane these strange Anagrams?
\P a1659 CLEVELAND \textit{Comm. Place} (1677) 167 Heaven descends into the Bowels of the Earth, and, to make up the Anagramm, the Graves open and the Dust ariseth.    
\P 1678 BUTLER \textit{Hudibr.} iii. i. 772 His body, that stupendous frame, Of all the world the anagram.
\end{myenumerate}

%%%%%%%%%%%%%%%%%%%%%%%%%%%%%
\myitem{analogy }  n.

\noindent  \phonetic{(əˈnælədʒɪ) }

\noindent  
[ad. L. analogia, a. Gr. ἀναλογία equality of ratios, proportion (orig. a term of mathematics, but already with transf. sense in Plato), f. ἀνάλογ-ος adj.: see analogon. Cf. mod.Fr. analogie.] 
\vspace{-0.3cm}

\begin{myenumerate}
\itembf{1.} Math. Proportion; agreement of ratios. 

\P 1557 RECORDE \textit{Whetst.} C ij, If any one proportion be continued in more then 2 nombers, there maie be then a conference also of these proportions‥that conference or comparison is named Analogie.    
\P 1570 BILLINGSLEY \textit{Euclid} v. Introd. 126 This booke‥entreateth of proportion and Analogie, or proportionalitie.    
\P 1660 BARROW \textit{Euclid} v. def. 4 That which is here termed Proportion is more rightly called Proportionality or Analogy.    
\P 1742 BAILEY\textit{Analogy} [in the Mathematicks] the Comparison of several Ratio's of Quantities or Numbers one to another.    
\P 1855 H. SPENCER \textit{Psychol.} (1872) II. vi. viii. 112 An analogy is ‘an agreement or likeness between’ two ratios in respect of the quantitative contrast between each antecedent and its consequent.

\itembf{2.} Hence, Due proportion; correspondence or adaptation of one thing to another. Obs. 

\P 1577 tr \textit{Bullinger's Decades} 1018 Analogie is an aptnes, proportion and a certaine conuenance of the signe to ye thing signified.
\P a1626 BP. ANDREWES \textit{Serm.} (1856) I. 429 If there be an analogy of faith, so is there of hearing also.    
\P 1684 tr. \textit{Bonet's Merc. Compit.} vi. 204 This bastard Pleurisie‥arose from a pituitous matter gathered in the Bloud through Analogy with Winter.    
\P 1774 GOLDSM. \textit{Nat. Hist.} I. 143 Some philosophers have perceived so much analogy to man in the formation of the ocean, that they have not hesitated to assert its being made for him alone.

\itembf{3.} Equivalency or likeness of relations; ‘resemblance of things with regard to some circumstances or effects’ (J.); ‘resemblance of relations’ (Whately); a name for the fact, that, the relation borne to any object by some attribute or circumstance, corresponds to the relation existing between another object and some attribute or circumstance pertaining to it. Const. to, with, between.
   This is an extension of the general idea of proportion from quantity to relation generally, and is often expressed proportionally, as when we say ‘Knowledge is to the mind, what light is to the eye.’ The general recognition of this analogy makes light, or enlightenment, or illumination, an analogical word for knowledge. 

\P 1550 VERON \textit{Godly Sayings} (1846) 28 Marke well, good reader, the analogye of the old and new sacramentes.    
\P 1605 BACON \textit{Adv. Learn.} ii. viii. §3 (1873) 122 Which three parts active [experimental, philosophical, magical] have a correspondence and analogy with the three parts speculative.    
\P 1658 PHILLIPS, \textit{Analogy}, Like Reason, Relation, Proportion, Agreement, Correspondency.    
\P 1675 BAXTER \textit{Cath. Theol.} ii. i. 13 We can think no otherwise of the Divine Conceptions and Volitions, but as we are led by the analogy of humane acts.    
\P 1765 TUCKER \textit{Lt. Nat.} II. 466 Analogy is the similitude or correspondence of particulars between things.    
\P 1785 REID \textit{Intell. Powers} 65 Some conceived analogy between body and mind.    
\P 1833 BREWSTER \textit{Nat. Magic} viii. 195 There is still one property of sound, which has its analogy also in light.    
\P 1860 TYNDALL \textit{Glac.} ii. 10. 285 The analogy between a river and a glacier moving through a sinuous valley is therefore complete.    
\P 1879 LUBBOCK \textit{Sci. Lect.} iv. 137 There seem to be three principal types [of ants] offering a curious analogy to the three great phases: the hunting, pastoral, and agricultural stages, in the history of human development.

\itembf{4.} more vaguely, Agreement between things, similarity. 

\P 1605 TIMME \textit{Quersit.} i. iv. 18 A great analogie or conuenience is found in this contrarietie of beginnings.
\P a1682 SIR T. BROWNE \textit{Tracts} 45 Who from some analogy of name conceive the Ægyptian Pyramids to have been built for granaries.    
\P 1712 ADDISON \textit{Spect.} No. 416 \cardo{⁋}1 Places, Persons, or Actions in general which bear a Resemblance, or at least some remote Analogy, with what we find represented.    
\P 1806 SYD. SMITH \textit{Elem. Mor. Phil.} (1850) 359 There is a certain analogy to this in drunkenness.    
\P 1839 MURCHISON \textit{Silur. Syst.} i. xxvii. 358 The trilobites‥bear so strong an analogy to those described by M. Brongniart.

\itembf{5.} As a figure of speech: The statement of an analogy, a simile or similitude. Obs. 

\P a1536 TINDALE \textit{Wks.} 473 (R.) Fetching his analogie and similitude at the naturall bodie.    
\P 1570 DEE \textit{Math. Præf.} 21 Parables and Analogies of whose natures, etc.    
\P 1651 HOBBES \textit{Leviath.} iii. xxxiv. 213 According to the same Analogy, the Dove, and the Fiery Tongues‥might also be called Angels.

\itembf{6.} = analogue. 

\P 1646 SIR T. BROWNE \textit{Pseud. Ep.} 158 Many have nostrills which have no lungs, as fishes, but none have lungs or respiration, which have not some shew, or some analogy of nostrills.    
\P 1661 in Heath \textit{Grocers' Comp.} (1869) 486 Man‥is the worlds analogy, And hath with it a Co-existency.    
\P 1837 LYTTON \textit{Athens} I. 296 The child is the analogy of a people yet in childhood.    
\P 1877 W. LYTTEIL \textit{Landm.} i. iii. 28 We readily find many analogies to such a name as Kairguin.

\itembf{7.} Logic. \textbf{a.} Resemblance of relations or attributes forming a ground of reasoning. \textbf{b.} The process of reasoning from parallel cases; presumptive reasoning based upon the assumption that if things have some similar attributes, their other attributes will be similar. 

\P 1602 in \textit{Thynne's Animadv.} Pref. 107 By true Annalogie I rightly find.    
\P 1692 BENTLEY \textit{Boyle Lect.} iv. 127 He hath made out from Example and Analogy.    
\P 1736 BUTLER \textit{Anal.} Introd. 4 Analogy is of weight‥towards determining our Judgment.    
\P 1832 J. AUSTIN \textit{Jurispr.} (1879) II. 1040 Analogy denotes an inference or a reasoning or argumentation, whereof an analogy of objects is mainly the cause or ground.    
\P 1843 MILL \textit{Logic} iii. xx. §1 The word Analogy as the name of a mode of reasoning is generally taken for some kind of argument supposed to be of an inductive nature but not amounting to a complete induction.    
\P 1853 ROBERTSON \textit{Serm.} Ser. iv. xxx. (1863) 231 Analogy is probability from a parallel case. We assume that the same law which operates in the one case will in another, if there be a resemblance between the relations of the things compared.    
\P 1871 C. DAVIES \textit{Metric Syst.} iii. 176 The analogy of all experience warrants the conjecture.    
\P 1875 STUBBS \textit{Const. Hist.} I. i. 11 Analogy, however, is not proof, but illustration.

\itembf{8.} Language. Similarity of formative or constructive processes; imitation of the inflexions, derivatives, or constructions of existing words, in forming inflexions, derivatives, or constructions of other words, without the intervention of the formative steps through which these at first arose.
   Thus the new inflexion bake, baked, baked (instead of the historical bake, book, baken) is due to analogy with such words as rake, raked, raked, etc. When the formative steps are not only absent, but could not have been present, the process is often called False Analogy; as when starvation was formed to bear the same relation to starve, that vexation does to vex. Vexation being historically due to the existence of vexāt- the ppl. stem of a L. vb. vexā-re, whence through Fr. vexe-r we have vex, there could be no such formative steps in the case of the Teut. vb. starve. But as all mere analogy, even that of vex-es, vex-ed, vex-ing, is in this sense ‘false,’ the term form-association is now commonly used of an analogical process which considers the mere forms of existing words, apart from their history. 

\P 1659 B. WALTON \textit{Consid. Considered} 264 There [is]‥a particular Grammar analogy in each particular tongue, before it be reduced into rules.    
\P 1706 PHILLIPS, \textit{Analogy}‥in Grammar, the Declining of a Noun, or Conjugating of a Verb, according to its Rule or Standard.    
\P 1747 JOHNSON \textit{Plan of Dict.} Wks. 1787 IX. 178 To our language may be with great justness applied the observation of Quintilian, that speech was not formed by an analogy sent from heaven.    
\P 1751 CHAMBERS \textit{Cycl.} s.v. Analogy, In matters of language, we say, new words are formed by Analogy.    
\P 1874 MORRIS \textit{Hist. Eng. Gram.} 95 The th in farther has crept in from false analogy with further.    
\P 1878 SWEET in \textit{Trans. Philol. Soc.} (1877–9) 391 Paul goes on to protest against the epithet ‘false’ analogy, remarking that it is really ‘correct,’ working as it does with unerring psychological instinct.

\itembf{9.} Nat. Hist. Resemblance of form or function between organs which are essentially different (in different species), as the analogy between the tail of a fish and that of the whale, the wing of a bat and that of a bird, the tendril of the pea and that of the vine. 

\P 1814 SIR H. DAVY \textit{Agric. Chem.} 62 Linnæus, whose lively imagination was continually employed in endeavours to discover analogies between the animal and vegetable systems, conceived ‘that the pith performed for the plant the same functions as the brain and nerves in animated beings.’    
\P 1854 WOODWARD \textit{Man. Mollusca} 55 Resemblances of form and habits without agreement of structure‥are termed relations of‥analogy.    
\P 1857 BERKELEY \textit{Cryptog. Bot.} §25 We understand by analogy those cases in which organs have identity of function, but not identity of essence or origin.    
\P 1870 HOOKER \textit{Stud. Flora} 13 Nymphæaceæ‥Affinities. With Papaveraceæ, but not close; presents analogies with Hydrocharideæ and Villarsia.
\end{myenumerate}

%%%%%%%%%%%%%%%%%%%%%%%%%%%%%
\myitem{anathema }  n. and adj.

\noindent  \phonetic{(əˈnæθɪmə) }

\noindent  
[a. L. anathema an excommunicated person, also the curse of excommunication, a. Gr. ἀνάθεµα, orig. ‘a thing devoted,’ but in later usage ‘a thing devoted to evil, an accursed thing’ (see Rom. ix. 3). Orig. a var. of ἀνάθηµα an offering, a thing set up (to the gods), n. of product f. ἀνατιθέναι to set up, f. ἀνά up + τιθέναι (stem θε-) to place. Cf. prec., and anatheme.] 
\vspace{-0.3cm}

\begin{myenumerate}
\itembf{I.} From eccl. Greek and Latin. 

\itembf{1.} Anything accursed, or consigned to damnation. Also quasi-adj. Accursed, consigned to perdition. 

\P 1526 [See \textbf{ANATHEMA MARANATHA}].    
\P 1625 BACON \textit{Ess., Goodness} (Arb.) 207 He would wish to be an Anathema from Christ, for the Salvation of his Brethren.    
\P 1634 CANNE \textit{Necess. Separ.} (1849) 162 Delivered over unto Satan, proclaimed publicans, heathens, anathema.    
\P 1765 TUCKER \textit{Lt. Nat.} II. 299 Saint Paul wished to become anathema himself, so he could thereby save his brethren.

\itembf{2.} The formal act, or formula, of consigning to damnation. \textbf{a.} The curse of God. \textbf{b.} The great curse of the church, cutting off a person from the communion of the church visible, and formally handing him over to Satan; or denouncing any doctrine or practice as damnable. Hence \textbf{c.} Any denunciation or imprecation of divine wrath against alleged impiety, heresy, etc. \textbf{d.} A curse or imprecation generally.
   (The weakening of the sense has accompanied the free use of anathemas as weapons of ecclesiastical rancour.) 

\textbf{a.}    
\P 1619 DONNE \textit{Biathan.} (1644) 192 Which Anathema‥was utter damnation, as all Expositors say.    
\P 1756 BURKE \textit{Vind. Nat. Soc.} Wks. I. 64 The divine thunders out his anathemas.    
\P 1877 MOZLEY \textit{Univ. Serm.} ii. 37 To strike with His anathema those who made a gain of their virtues.

\textbf{b.}    
\P 1590 SWINBURN \textit{Testaments} 60 Vnlesse he be excommunicate with that great curse, which is called Anathema.    
\P 1642 FULLER \textit{Holy \& Prof. St.} v. xi. 404 The Donatists, whilest blessing themselves, cared not for the Churches Anathema's.    
\P 1726 AYLIFFE \textit{Parerg.} 256 An Anathema‥differs from an Excommunication only in respect of a greater kind of Solemnity.    
\P 1769 ROBERTSON \textit{Charles V,} III. viii. 71 Against all who disclaimed the truth of these tenets, anathemas were denounced.    
\P 1844 GLADSTONE \textit{Gleanings} V. xlv. 114 The Pope‥has condemned the slave trade—but no more heed is paid to his anathema than to the passing wind.

\textbf{c.}    
\P 1782 PRIESTLEY \textit{Nat. \& Rev. Relig.} II. 80 The Mohammedans denounce anathemas against unbelievers.    
\P 1850 GLADSTONE \textit{Gleanings} V. xiv. 182 To deliver over to anathema the memories of our forefathers in the Church.

\textbf{d.}    
\P 1691 NORRIS \textit{Pract. Disc.} 90 Willing rather to err with the Multitude‥than incur the great Censure, the heavy Anathema of Singularity. 
\P a1757 CIBBER in \textit{Dilworth Pope} 16 How then could you thunder out such anathema's on your own enemies?    
\P 1827 LYTTON \textit{Pelham} lxvii. (1840) 294 ‘Confound the man!’ was my mental anathema.    
\P 1867 L.M. CHILD \textit{Romance Repub.} xx. 237 The Signor‥succeeded in smothering his half-uttered anathemas.

\itembf{II.} From the earlier sense of ἀνάθεµα or ἀνάθηµα. (In this sense better 
pronounced \phonetic{ænəˈθiːmə}) 

\itembf{3.} A thing devoted or consecrated to divine use. 

\P 1581 MARBECK \textit{Bk. of Notes} 39 Anathema (saith Chrisostome) are those things which being consecrated to God, are laied up from other things.    
\P 1608 TOPSELL \textit{Serpents} 779 Will not permit a [spider's] web—the very pattern, index, and anathema of supernaturall wisdome—to remain untouched.    
\P 1857 BIRCH \textit{Anc. Pottery} (1858) I. 178 The little figures, in the shape of animals‥may have been votive offerings to the gods, such anathemata being offered by the poor.
 
Draft partial entry February 2007

\itembf{III.} adj. In predicative use: loathsome, repugnant, or extremely objectionable to. 

\P 1648 R. HERRICK \textit{Hesperides} sig. S7, Who read'st this Book that I have writ, And can'st not mend, but carpe at it: By all the muses! thou shalt be Anathema to it, and me.    
\P 1862 \textit{Littell's Living Age} 5 Apr. 22/1 Apple-green papers in bedrooms have long been anathema to nervous men.    
\P 1880 \textit{Littell's Living Age} 6 Mar. 617/2 Glory such as Rajah Brooke has won was ‘anathema’ to him.    
\P 1919 R. FIRBANK \textit{Valmouth} xi. 189 A book is anathema to her.    
\P 1944 \textit{Sun (Baltimore)} 19 Oct. 21/2 Defeats are anathema to gridsters with January 1 [i.e. the day on which post-season bowl games are played] on their minds.    
\P 1970 M. TORMÉ \textit{Other Side of Rainbow} (1971) iii. 50 While lip-syncing is anathema to most singers, it was Judy's particular teacup.    2005 Gay Times Dec. 150/1 The idea of paying to simply go into a bar is anathema to us Northerners.
\end{myenumerate}

%%%%%%%%%%%%%%%%%%%%%%%%%%%%%
\myitem{ancillary }  a. and n.

\noindent  \phonetic{(ˈænsɪlərɪ, ænˈsɪlərɪ) }

\noindent  
[ad. L. ancillāri-us (more correctly ancillār-is) of or pertaining to a
handmaid, f. ancilla: see prec.] 
\vspace{-0.3cm}

\begin{myenumerate}
\itembf{A.} adj. 

\itembf{1.} Subservient, subordinate, ministering (to). 

\P 1667 WATERHOUSE \textit{Fire of Lond.} 60 God makes every thing ancillary hereunto.
\P 1768 BLACKSTONE \textit{Comm.} iii. vii. (R.) It is beneath the dignity of the king's
courts to be merely ancillary to other inferior jurisdictions.    
\P 1836 H. TAYLOR \textit{Statesm.} viii. 49 It will be rather ancillary than essential.    
\P 1848 ARNOULD \textit{Mar. Insur.} II. ii. v. 652 Warlike stores‥directly ancillary to warlike
purposes.    
\P 1869 RAWLINSON \textit{Anc. Hist.} 8 Geography, the other ancillary science
to History.

\itembf{2.} lit. (after L.) Of or pertaining to maid-servants. rare and affected. 

\P 1852 THACKERAY \textit{Esmond} iii. ix. (1876) 404 The ancillary beauty was the one
whom the Prince had selected.    
\P 1854 BADHAM \textit{Halieut.} 399 Ancillary reformation
has not yet begun to be thought of; cats are not more detrimental to mice‥than
these smashing wenches to‥Sèvres teacups.

\itembf{3.} Designating activities and services that provide essential support to the
functioning of a central service or industry; also, of staff employed in these
supporting roles. Now used esp. of non-medical staff and services in hospitals. 

\P 1948 B. NEWMAN \textit{Baltic Background} vi. 139 Sixty-five per cent of the Estonians
are directly engaged in agriculture, and many more in its ancillary occupations.
\P 1955 \textit{Times} 10 May 9/2 There were inadequate ancillary services such as
laundries, kitchens, bathrooms, and lavatory accommodation.    
\P 1957 \textit{Encycl. Brit.} XVIII. 948/1 The Transport act‥nationalized the railways, together with
their ancillary services—docks, steamers, road vehicles, hotels and canals.
\P 1962 \textit{Lancet} 26 May 1114/1 Ancillary workers.—An ample complement of
ancillaries is essential. We would suggest one psychiatric social worker for
each consultant team.    
\P 1976 \textit{Daily Tel.} 20 July 2/4 Ancillary and other staff
from five trade unions are to stage a 24-hour strike from midnight tonight.
\P 1982 \textit{Financial Times} 7 July 12/6 The Government was not prepared to improve
on its latest offer of 7.5 per cent for nurses and 6 per cent for ancillary
staff and other grades.

\itembf{B.} n. 

\itembf{1.a.} One who acts as an assistant or servant. Obs.

\P 1867 G. MEREDITH in \textit{Fortn. Rev.} 1 Sept. 294 They were yoked before the glad
youth by his sister-ancillaries.

\itembf{b.} An ancillary worker. See sense 3 of the adj. 

\P 1962 [see sense A. 3 above].    
\P 1982 \textit{Financial Times} 17 Aug. 12/1 Bank staff
can hardly expect‥the kind of public support enjoyed by the low-paid hospital
ancillaries.    
\P 1985 \textit{Ibid.} 15 Nov. 24/8 Local authority manual workers have
settled‥; health service ancillaries are expected to secure a similar deal.

\itembf{2.} Something which is ancillary; an auxiliary or accessory. 

\P 1929 \textit{Morning Post} 2 Oct. 10/4 Aircraft must be regarded only as a very useful
and necessary ancillary to the main fleet.    
\P 1942 W.S. CHURCHILL \textit{Secret Session Speeches} 
(1946) 63 He had expected to meet the three Kongos and perhaps
two aircraft carriers together with ancillaries.    
\P 1972 \textit{Proc. Inst. Electr. Engineers} CXIX 189 A design of great 
simplicity has been developed in which the
vacuum-interrupter circuit-breakers and all ancillaries are housed in one
modular enclosure.    
\P 1980 \textit{Daily Tel.} 23 Apr. 3 (Advt.), Cave Tab are the
specialists in ancillaries, equipment and supplies for all DP and WP operations.
\P 1986 \textit{New Yorker} 27 Jan. 47/1, I thought I might as well do some air tests.
That involves two stages: first the airframe and its ancillaries, then the
engine.
\end{myenumerate}

%%%%%%%%%%%%%%%%%%%%%%%%%%%%%
\myitem{animus }  n.

\noindent  \phonetic{(ˈænɪməs) }

\noindent  
[a. L. animus (1) soul, (2) mind, (3) mental impulse, disposition, passion.] 

\noindent  
No pl. 

\begin{myenumerate}
\itembf{1.} Actuating feeling, disposition in a particular direction, animating spirit
or temper, usually of a hostile character; hence, animosity. 

\P [1818 Not in Todd.]    
\P 1820 \textit{Ann. Reg.} 1819 74/2 The original design‥was demonstrative of the animus of the projectors.    
\P 1831 GEN. P. THOMPSON \textit{Exerc.}
I. 424 The animus is to impress upon the British soldiery the duty of putting
down the liberties of their country.    
\P 1840 THACKERAY \textit{Paris Sk.-bk.} (1872) 212
The animus with which the case has been conducted.    
\P 1863 I. TAYLOR \textit{Pentateuch} 16 Almost every page‥affords an instance‥of an 
intense feeling, or, as we say,
animus; this is the word we use when a speaker or writer, who is labouring to
substantiate a defamation, finds it more than he can do to repress emotions,
that are not of the most amiable sort, and which he does not choose to avow.
\P 1864 LOWELL \textit{Biglow P.} Wks. 1879, 264/2 The animus that actuates the policy of
a foreign country.    
\P 1953 L. EDEL \textit{Henry James, Untried Years} iv. 195 Henry
James expressed‥the starch that congealed the blood of some New Englanders, with
an often ill-concealed‥animus.

\itembf{2.} Psychol. Jung's term for the masculine component of a female personality.
Cf. anima. 

\P 1923 [see anima].    
\P 1943 \textit{Horizon} VIII. 262 The dominating animus peeping
through the light-heartedness of the young girl.    
\P 1943 H. READ \textit{Educ. through Art} iv. 95 According to Jung, the conscious aspect of the individual's
personality‥is balanced‥by a contra-sexual counterpart—that individual's animus
(the male counterpart in the case of a woman) or anima (the female counterpart
in the case of a man).    
\P 1962 J. JACOBI \textit{Psychol. of C. G. Jung} (ed. 6) iii. 111
The animus-possessed woman, opinionated and argumentative, the female
know-it-all, who reacts in a masculine way and not instinctively.
\end{myenumerate}

%%%%%%%%%%%%%%%%%%%%%%%%%%%%%
\myitem{annals}  n.pl.

\noindent  \phonetic{(ˈænəlz) }

\noindent  
[ad. L. annāl-es the historical record of the events of each year, prop. masc.
pl. (sc. libri) of annālis yearly, f. annus year. Occas. used in sing.] 
\vspace{-0.3cm}

\begin{myenumerate}
\itembf{1.} A narrative of events written year by year. 

\P 1563 GRAFTON \textit{Epist. to Cecil} (R.) Short notes in maner of Annales commonly
called Abridgementes.    
\P 1607 SHAKES. \textit{Cor.} v. vi. 114 If you haue writ your
Annales true, 'tis there.    
\P 1622 HEYLIN \textit{Cosmogr.} Introd. (1674) 17/2 Annals‥are
a bare recital only of the Actions happening every year.    
\P 1759 ROBERTSON \textit{Hist. Scotl.} I. i. 1 Everything beyond that period 
to which well-attested annals reach is obscure.    
\P 1867 STUBBS \textit{Benedict's Chron.} Pref. I. 12 The difference between
chronicles and annals was‥that the former have a continuity of subject and
style, whilst the latter contain the mere jottings down of unconnected events.

\itembf{b.} sing. The record or entry of a single year, or a single item, in a
chronicle. 

\P 1699 BENTLEY \textit{Phal.} 282 Diodorus in the Annal of that year, says Phæon was
Archon.    
\P 1814 SIR R. WILSON \textit{Pr. Diary} II. 309 A modest inscription to record
the act of restoration‥an annal which the greatest anti-Buonapartist ought to
respect.    
\P 1865 EARLE \textit{Sax. Chron.} Introd. 10 Here and there may be seen an
annal, expressed in riper language, which must be marked as the interpolation of
a later Editor.

\itembf{c.} attrib. quasi-adj. 

\P 1670 MILTON \textit{Hist. Eng.} iv. Wks. 1851, 175 Huntingdon, as his manner is to
comment upon the annal Text, makes a terrible description of that fight.

\itembf{2.} Historical records generally. 

\P a1581 CAMPION \textit{Hist. Irel., Ep. Ded.} (1633) 1 Containing Annales and other
worthy memorialls.
\P a1687 PETTY \textit{Pol. Anat.} Ded., An Adventure that shall
shine in the Annals of Fame.    
\P 1706 ADDISON \textit{Rosamond} iii. i, Whatever glorious
and renowned In British annals can be found.    
\P 1750 GRAY \textit{Elegy} viii, The short
and simple annals of the poor.    
\P 1844 DISRAELI \textit{Coningsby} vi. ii. 226 The
glorious annals of their great country.    
\P 1878 C. STANFORD \textit{Symb. Christ} i. 5
The first war recorded in the annals of the human race.

\itembf{3.} Masses said for the space of a year. 

\P 1536 LATIMER \textit{2nd Serm. bef. Conv.} I. 52 No priest should sell his saying of
tricennals or annals.    
\P 1726 AYLIFFE \textit{Parerg.} 190 Annals are Masses said in the
Romish Church for the Space of a Year, or for any other Time, either for the
Soul of a Person deceas'd, or for the Benefit of a Person living.
\end{myenumerate}

%%%%%%%%%%%%%%%%%%%%%%%%%%%%%
\myitem{anomaly }  n.

\noindent  \phonetic{(əˈnɒməlɪ) }

\noindent  
[ad. L. anōmalia, a. Gr. ἀνωµαλία, n. of quality f. ἀνώµαλ-ος: see anomal.] 
\vspace{-0.3cm}

\begin{myenumerate}
\itembf{1.} Unevenness, inequality, of condition, motion, etc. 

\P 1571 DIGGES \textit{Pantom.} (1591) 178 The excesse wherby the Semidiameter of the
Ringe or Cornice of the Head dooth exceed the Cornice of the Coyle [of cannon] I
call the Anomalye.    
\P 1684 T. BURNET \textit{Th. Earth} II. 98 The great shakings and
concussions of our globe at that time, affecting some of the neighbouring
orbs‥may cause anomalies and irregularities in their motions.    
\P 1837 WHEWELL \textit{Hist. Induct. Sc.} I. iii. ii. 175 The motions of the sun and moon‥had other
anomalies or irregularities.

\itembf{2. a} Irregularity, deviation from the common order, exceptional condition or
circumstance. concr. A thing exhibiting such irregularity; an anomalous thing or
being. 

\P 1664 POWER \textit{Exp. Philos.} i. 78 To admire Nature's Anomaly‥in the number of
Eyes, which she has given to several Animals.    
\P 1722 WOLLASTON \textit{Relig. Nat.} ix. 217 Support him under all the anomalies of life.    
\P 1818 HALLAM \textit{Mid. Ages} (1872) II. 213 Time changes anomaly into system.    
\P 1852 GLADSTONE \textit{Gleanings} IV. xvi.
152 The intolerable anomaly of a state obeying in the civil sphere the dictates
of the Church.    
\P 1870 DISRAELI \textit{Lothair} l. 274 A capital without a country is an
apparent anomaly.

\itembf{b.} Nat. Sci. Deviation from the natural order. 

\P 1646 SIR T. BROWNE \textit{Pseud. Ep.} 135 They confound the generation of perfect
animalls with imperfect‥and erect anomalies, disturbing the lawes of Nature.
\P 1859 DARWIN \textit{Orig. Spec.} v. (1873) 108 There is no greater anomaly in nature
than a bird that cannot fly.    
\P 1860 MAURY \textit{Phys. Geog.} xv. §669 A low
barometer‥was considered an anomaly peculiar to the regions of Cape Horn.

\itembf{c.} Gram. Irregularity, exception to the prevailing form of inflexion, etc. 

\P 1612 BRINSLEY \textit{Lud. Lit.} xx. (1627) 224 Most exceptions or Anomalies may be
learned after.    
\P 1751 WATTS \textit{Improvem. Mind} (1801) 57 Let but few of the
anomalies or irregularities of the tongue be taught‥to young beginners.    1874
Blackie Self-Culture 34 Some anomalies, as in the conjugation of a few irregular
verbs.

\itembf{3.} Astr. The angular distance of a planet or satellite from its last perihelion
or perigee: so called because the first irregularities of planetary motion were
discovered in the discrepancy between the actual and the computed distance. 

\P 1669 FLAMSTEAD in \textit{Phil. Trans.} IV. 1109 The moons mean Anomaly is 0 s. 15 d. 10 m. 37 sec.    
\P 1706 PHILLIPS, \textit{Anomaly} of the Orbit is the Arch, or Distance of
a Planet from its Aphelion.    
\P 1867 E. DENISON \textit{Astron.} 32 The distance of a
planet from perihelion, or of the moon from perigee‥is called its true anomaly;
and the distance it would have gone in the same time if it moved uniformly, or
in a circle instead of an ellipse, is its mean anomaly; and their difference is
called the equation of the centre.    
\P 1868 \textit{Chambers's Encycl.} I. 280 The anomaly
was formerly measured from the aphelion; but from the fact that the aphelia of
most of the comets lie beyond the range of observation, the perihelion is now
taken as the point of departure for all planetary bodies.

\itembf{4.} Mus. A small deviation from a perfect interval, in tuning instruments with
fixed notes; a temperament. Ed. Encycl. 1830. 

\itembf{5. a.} Meteorol. (See quots.) 

\P 1853 E.J. SABINE tr. \textit{Dove's Distribution of Heat} 20 We require‥to exhibit
the relation of the actual temperature of each place to the mean or normal
temperature of its geographical latitude. I call the difference between the
actual and normal temperature the ‘thermic anomaly’.    
\P 1922 W.G. KENDREW \textit{Clim. Cont.} i. i. 3 The ‘anomaly of temperature’ for that 
place, a positive anomaly if
the place is warmer than the mean, a negative anomaly if it is colder.

\itembf{b.} Geogr. A local departure from the normal pull of gravity. 

\P 1924 H. JEFFREYS \textit{Earth} 121 This anomaly is always negative. In other words,
the gravity on a mountain top is less than elsewhere.    
\P 1944 A. HOLMES \textit{Princ. Physical Geol.} xviii. 404 This band of what 
are called ‘negative anomalies of
gravity’ implies that there is a corresponding deficiency of density in the
materials of the crust beneath.
\end{myenumerate}

%%%%%%%%%%%%%%%%%%%%%%%%%%%%%
\myitem{antecedent }  n.

\noindent  \phonetic{(æntɪˈsiːdənt) }

\noindent  
[a. Fr. antécédent (see next), subst. use of the adj. Already in L. antecēdens
was used subst. as a term of philosophy, and in this technical sense it first
appeared in the modern languages.] 
\vspace{-0.3cm}

\begin{myenumerate}
\itembf{1.} A thing or circumstance which goes before or precedes in time or order;
often also implying causal relation with its consequent. \textbf{a.} generally. 

\P 1612 T. TAYLOR \textit{Comm. Titus} i. ii (1619) That there may be full content with
it selfe, the antecedents and consequents.
\P 1680 in Somers \textit{Tracts} II. 548
Consider the Antecedents to the calling the Convention.
\P a1716 SOUTH (J.) It is‥the necessary antecedent‥of a sinner's return to God.    
\P 1824 COLERIDGE \textit{Aids to Refl.} (1848) I. 92 Conscience is the ground 
and antecedent of human (or
self-) consciousness, and not any modification of the latter.
\P 1862 BUCKLE \textit{Civilis.} (1869) III. iii. 130 Circumstances‥governed by a long chain of
antecedents.

Hence, in various special applications, of which the logical and grammatical are
the earliest uses of the word in Eng. 

\itembf{b.} Logic. (Opposed to consequent.) The statement upon which any consequence
logically depends; hence †(a) The premisses of a syllogism (obs.); (b) The part
of a conditional proposition on which the other depends. †(c) By some early
logicians the subject and predicate were called antecedent and consequent. 

\P c1400 \textit{Test. Love} ii. (1560) 284 b/1 The consequence is false, needes the
antecedent mote beene of the same condition.
\P 1425 WYNTOUN \textit{Cron.} viii. iii.
67 [I] grantis‥þe Antecedens Bot I deny þe consequens.    
\P 1587 FLEMING \textit{Contn. Holinshed} III. 324/1 You have shewn us the antecedent, now let us have the ergo.
\P 1628 T. SPENCER \textit{Logick} 161 Ramus doth call the subiect, and the
predicate‥antecedent, and consequent: but very vnduely.
\P 1665 J. GOODWIN \textit{Filled w. Spirit} (1867) 191 Let the word person in the antecedent of the
proposition be supposed to signify either something or nothing.    
\P 1870 BOWEN \textit{Logic} v. 128 All Hypothetical Judgments obviously consist of two parts, the
first of which is called the Condition or Antecedent.

\itembf{c.} Gram. (a) The noun to which a following pronoun refers, and to avoid the
repetition of which it is used. (b) esp. The substantive (word, clause, or
sentence) to which a relative pronoun or adverb points back, and to which the
relative clause stands in an attributive or adjective relation. 

\P 1393 LANGL. \textit{P. Pl.} C. iv. 364 Adjectif and substantif Acordeþ in alle Kyndes · with his antecedent. 
\P 1523 WHITTINTON \textit{Vulg.} 2 The relatyue of substaunce
shall accorde with his antecedent.    
\P 1655 GOUGE \textit{Comm. Hebr.} i. 10 This relative
‘Thou’ must have an antecedent.    
\P 1765 W. WARD \textit{Eng. Gram.} 128 The connexion of
a personal pronoun with its antecedent is very different from that of a relative
pronoun.    
\P 1876 MASON \textit{Eng. Gram.} 51 In the nominative and objective cases, what
is never preceded by an antecedent.

\itembf{d.} Math. The first of two numbers or magnitudes between which a ratio is
expressed; the first and third in a series of four proportionals. 

\P 1570 BILLINGSLEY \textit{Eucl.} v. def. 3 The first Terme, namely, that which is
compared, is called the antecedent.    
\P 1695 W. ALINGHAM \textit{Geom. Epit.} 14 In the
Comparison of 7 to 3, 7 is named the Antecedent, and 3 the Consequent.    
\P 1862 TODHUNTER \textit{Euclid} vi. iv, Those [sides] which are opposite to the equal angles
are homologous sides, that is, are the antecedents or the consequents of the ratios.

\itembf{e.} Music. (See quot.) 

\P 1869 OUSELEY \textit{Counterp.} xv. 95 The leading part [in a Canon] is called the
antecedent, the following part the consequent.

\itembf{2.} pl. The events of a person's bygone history (usually, as affecting the
postion now to be accorded him); also used of institutions, etc. 

\P 1841 GEN. THOMPSON \textit{Exerc.} VI. 237 They will‥sift what the French call their
antecedents, with the most scrupulous nicety.    
\P 1854 DE QUINCEY \textit{Selections} ii. 86 What modern slang denominates his antecedents.    
\P 1864 J.H. NEWMAN \textit{Apol.} 106 Froude and I were nobodies; with‥no antecedents to fetter us.    
\P 1868 M. PATTISON \textit{Academ. Organ.} §4. 111 Young fellows unacquainted with the antecedents
of the estates.

\itembf{3.} concr. A predecessor in the chain of development; an earlier form. rare. 

\P 1865 LECKY \textit{Rational.} (1878) I. 254 A wind instrument which some have placed
among the antecedents of the organ.

\itembf{4.} lit. A person that walks in front; an usher, an anteambulo. Obs. 

\P 1608 DAY \textit{Hum. out of Br.} ii. ii, Boy. I say a seruingman is an antecedent.
Oct. Because he sits before a cloakebag.    
\P 1632 MASSINGER \textit{City Madam} ii. ii, My
antecedent, or my gentleman-usher.
\end{myenumerate}

%%%%%%%%%%%%%%%%%%%%%%%%%%%%%
\myitem{anthropology }  n.

\noindent  \phonetic{(-ˈɒlədʒɪ) }

\noindent  
[f. Gr. ἄνθρωπο-ς man + -logy. Gr. had ἀνθρωπολόγος (Aristotle) treating of man,
of which *ἀνθρωπολογία was analogically the abst. n. Anthropologia occurs as
mod.L. in 1595, and anthropologie as mod.Fr.] 
\vspace{-0.3cm}

\begin{myenumerate}
\itembf{1.} The science of man, or of mankind, in the widest sense.
   This seems to have been the original application of the word in Eng. but for
two and a half cent., to c 1860, the term was commonly confined to the
restricted sense b. Since that date, it has sometimes been limited, by reaction,
to c. 

\P 1593 R. HARVEY \textit{Philad.} 15 Genealogy or issue which they had, Artes which they
studied, Actes which they did. This part of History is named Anthropology.
   
\P 1656 BLOUNT \textit{Glossogr.,} Anthropology, a speaking or discoursing of men.

\itembf{b.} The science of the nature of man, embracing Human Physiology and
Psychology and their mutual bearing.
   The sense in which ἀνθρωπολόγος was used by Aristotle, and Anthropologia by
Otto Casmann 1594–5 in his Psychologia Anthropologica, sive Animæ Humanæ
Doctrina; and Anthropologia: Pars II. hoc est de fabrica Humani Corporis. This
author seems to have first used the term. 

\P [1706 J. DRAKE \textit{(title)} Anthropologia Nova; or, A new System of Anatomy.]
\P 1706 PHILLIPS, \textit{Anthropology,} a Discourse or Description of Man, or of a Man's
Body.    
\P 1727–51 CHAMBERS \textit{Cycl.,} Anthropology includes the consideration both of
the human body and soul, with the laws of their union, and the effects thereof,
as sensation, motion, etc.    
\P 1810 COLERIDGE \textit{Taste} in \textit{Lect. Shaks.} II. 223 The
analysis of our senses in the commonest books of anthropology.    
\P 1834 PENNY \textit{Cycl.} II. 97 Anthropology‥considers man as a citizen of the world, and has
nothing properly to do with the varieties of the human race.

\itembf{c.} The ‘study of man as an animal’ (Latham). The branch of the science which
investigates the position of man zoologically, his ‘evolution,’ and history as a
race of animated beings. 

\P 1861 HULME \textit{Moquin-Tandon} Pref. 8 Natural History, or Anthropology‥the
principal characters of our species, its perfection, its accidental
degradations, its unity, its races, and the manner in which it has been
classified.    
\P 1881 FLOWER in \textit{Nature} No. 619. 437 The aim of zoological
anthropology is to discover a natural classification of man.

\itembf{2.} A speaking after the manner of men; anthropomorphic language. [The sense in
which ἀνθρωπολογέ-ειν was used by Philo.] Obs. 

\P 1727–51 CHAMBERS \textit{Cycl.,} Anthropology is particularly used in theology, for a
way of speaking of God, after the manner of men, by attributing human parts and
passions to him.
\end{myenumerate}

%%%%%%%%%%%%%%%%%%%%%%%%%%%%%
\myitem{antic }  a. and n.

\noindent  \phonetic{(ˈæntɪk) }

\noindent  
[app. ad. It. antico, but used as equivalent to It. grottesco, f. grotta, ‘a
cauerne or hole vnder grounde’ (Florio), orig. applied to fantastic
representations of human, animal, and floral forms, incongruously running into
one another, found in exhuming some ancient remains (as the Baths of Titus) in
Rome, whence extended to anything similarly incongruous or bizarre: see
grotesque. Cf. Serlio Architettura (Venice 1551) iv. lf. 70 a: ‘seguitare le
uestigie de gli antiqui Romani, li quali costumarono di far‥diuerse bizarrie,
che si dicono grottesche.’ Apparently, from this ascription of grotesque work to
the ancients, it was in English at first called antike, anticke, the name
grotesco, grotesque, not being adopted till a century later. Antic was thus not
developed in Eng. from antique, but was a distinct use of the word from its
first introduction. Yet in 17th c. it was occas. written antique, a spelling
proper to the other word.] 
\vspace{-0.3cm}

\begin{myenumerate}
\itembf{A.} adj. 

\itembf{1.} Arch. and Decorative Art. Grotesque, in composition or shape; grouped or
figured with fantastic incongruity; bizarre. 

\P 1548 HALL \textit{Chron. Hen. VIII} an. 12 (R.) A fountayne of embowed
woorke‥ingrayled with anticke woorkes.    
\P 1589 \textit{Hawkins' 2nd Voy.} in Arber \textit{Eng. Garner} V. 126 To paint their 
bodies with curious knots or antike work, as every
man, in his own fancy deviseth.    
\P 1598 FLORIO, \textit{Grottesca,} a kind of rugged
vnpolished painters worke, anticke work.    
\P 1603 \textit{Montaigne} i. xxvii. (1632) 89
All void places‥he filleth up with antike Boscage or Grotesko workes.    
\P 1623 COCKERAM \textit{Anticke Worke,} a worke in painting or caruing of diuers shapes of
Beasts, Birds, Flowers, etc., vnperfectly mixt, and made one of another.    1624
Wotton Archit. 97 Whether Grotesca (as the Italians) or Antique worke (as wee
call it) should be receiued.    
\P 1703 \textit{City \& Country Build.} 5 Antick, or
Antique-work‥a confused Composure of Figures of different Natures, and Sexes,
etc. As of Men, Beasts, Birds, Flowers, Fishes, etc. And such like Fancies as
are not in Rerum Natura.‥ This Work which we call Antick, the Italians call
Grotesca‥and the French Grotesque.    
\P 1826 J. ELMES \textit{Dict. Fine Arts, Antick,} Odd, ridiculously wild.

\itembf{2.} Absurd from fantastic incongruity; grotesque, bizarre, uncouthly
ludicrous: \textbf{a.} in gesture. 

\P 1590 MARLOWE \textit{Edw. II,} i. i. 167 My men, like satyrs,‥Shall with their
goat-feet dance the antic hay.    
\P 1602 SHAKES. \textit{Ham.} i. v. 172 How strange or
odde so ere I beare myselfe‥To put an Anticke disposition on.    
\P 1603 DRAYTON \textit{Her. Epist.} xi. 13 A Satyres Anticke parts he play'd.    
\P 1645 MILTON \textit{Colast.}
Wks. 1851, 365 No antic hobnaile at a Morris, but is more hansomly facetious.
\P 1660 H. MORE \textit{Myst. Godl.} iii. ix. 77 Their religious Rites and Ceremonies
being uncouth and antick.    
\P 1719 DE FOE \textit{Crusoe} 183 He came running to me‥making
a many antic gestures.    
\P 1805 WORDSWORTH \textit{Prel.} vii. (1850) 178 An antic pair Of
monkeys on his back.    
\P 1878 G. MACDONALD \textit{Phantastes} x. 149 Performing the most
antic homage.

\itembf{b.} in shape. 

\P 1642 R. CARPENTER \textit{Exper.} iii. v. 53 To appeare in strange and antick shapes.
\P 1788 \textit{New Lond. Mag.} 17 Several antic figures in shapes of boys danced.
\P 1861 \textit{Tannhäuser} 20 The twilight troop'd with antic shapes.

\itembf{c.} in dress or attire. 

\P 1642 MILTON \textit{Apol. Smect.} Wks. 1738 I. 125 It had no Rubric to be sung in an
antic Cope upon the Stage of a High Altar.    
\P 1665 GLANVILL \textit{Sceps. Sci.} 96 Their antick deckings with feathers.    
\P 1727 SWIFT \textit{Gulliver} iii. vii. 223 Two rows of guards‥dressed after a very antic manner.    
\P 1776 \textit{Chron.} in \textit{Ann. Reg.} 155/2 An ass‥with a fellow in an antick dress riding upon it.    
\P 1858 HAWTHORNE \textit{Fr. \& It. Jrnls.} I. 80 The papal guards, in the strangest 
antique and antic costume that was ever seen.

\itembf{3.} Having the features grotesquely distorted like ‘antics’ in architecture;
grinning. Obs. 

\P 1594 DRAYTON \textit{Idea} 424 Making withall some filthy Antike Face.    
\P 1611 COTGR, \textit{Gargouille,} The mouth of a Spowt, representing a Serpent, or the Anticke face of
some other ouglie creature.    
\P 1620 QUARLES \textit{Jonah} (1638) 41 Your mimick mouthes,
your antick faces. 
\P a1631 DONNE \textit{Elegies} (R.) Name not these living death-heds
unto me, For these not ancient but antique be.
\P 1659 CLEVELAND \textit{Wks.} (1687)
31 The Antick heads which plac'd without The Church, do gape and disembogue a
Spout.    
\P 1697 W. DAMPIER \textit{Voy.} (1729) III. i. 406 The little Tame-Owl‥making
divers antick faces.

\itembf{4.} Comb., as †antic-faced (see 3). 

\P 1635 J. TAYLOR (Water P.) \textit{Parr} in \textit{Harl. Misc.} (Malh.) IV. 205 An antick-faced
fellow, called Jack, or John the Fool.

\itembf{B.} n. 

\itembf{1.} Arch. and Decorative Art. An ornamental representation, purposely
monstrous, caricatured, or incongruous, of objects of the animal or the
vegetable kingdom, or of both combined. a.B.1.a Fantastic tracery or sculpture.
Obs. 

\P 1548 HALL \textit{Chron. Hen. VIII} an. 18 (R.) Aboue the arches were made many sondri
antikes and diuises.    
\P 1596 SPENSER \textit{F.Q.} ii. vii. 4 Woven with antickes and
wyld ymagery.    
\P 1645 EVELYN \textit{Mem.} (1857) I. 146 The walls and roof are painted,
not with antiques and grotesques, like our Bodleian.    
\P 1653 URQUHART \textit{Rabelais}
i. viii, A faire Cornucopia or Horne of abundance, such as you see in Anticks.
\P 1725 BRADLEY \textit{Fam. Dict.,} Grotesque or Grotesc, a work, the same with what is
sometimes called Antick.    
\P 1830 R. STUART \textit{Dict. Archit.: Antics,} In
architecture, Fancies having no foundation in nature, as sphinxes, centaurs,
syrens, representations of different sorts of flowers growing on the same stem;
grotesque ornaments of all kinds, as lions and pards with acanthus' tails, or
any other tails but their own proper ones; human forms with similar ridiculous
appendages. Ornaments, although strictly natural, in an unnatural situation; as,
caryatidæ of all kinds‥The villa Palagonia, in Sicily, is an antic, from
entrance gate to chimney top.

\itembf{b.} A caryatid, or (sculptured) human figure represented in an impossible
position. 

\P c1590 MARLOWE \textit{Faustus} (2nd vers.) 715 To make his monks‥stand like apes, And
point like antics at his triple crown.    
\P 1615 BP. HALL \textit{Contempl.} (1837) I.
xviii. iii. 395 Like some antic statue, in a posture of impotent endeavour.
\P 1638 CHILLINGWORTH \textit{Relig. Prot.} i. vi. §54, 374 Those crouching Anticks which
seeme in great buildings to labour under the weight they beare.    
\P 1640 BP. HALL \textit{Chr. Moder.} 20/1 Those antics of stone‥carved out under the end of great beams
in vast buildings, which seem‥as if they were hard put to it with the weight.
\P 1656 HALES \textit{Gold. Rem.} (1688) 167 Those that build houses make anticks that
seem to hold up the beams.    
\P 1830 [See sense a].

\itembf{c.} A grotesquely figured representation of a face, such as are used in
gargoyles. 

\P 1601 HOLLAND \textit{Pliny} (1634) II. 552 To set vp Gargils or Antiques at the top of
a Gauill end, as a finiall to the crest tiles.    
\P 1683 \textit{Lond Gaz.} mdccclix/8
Three Gold Seals, one with an Old Man's Head, another with a Woman's Head, and
the other with an Antick.

\itembf{2.} A grotesque or ludicrous gesture, posture, or trick; also fig. of
behaviour. (Commonly in pl.) 

\P 1529 FOXE in \textit{Supplic.} (1871) Introd. 9 In sothe it maketh me to laugh, to see
ye mery Antiques of M. More.    
\P 1572 SIR T. SMITH in Ellis \textit{Orig. Lett.} ii. 191
III. 20 Vaulting with notable supersaltes and through hoopes, and last of all
the Antiques, of carrying of men one uppon an other which som men call labores
Herculis.    
\P 1633 FORD \textit{Love's Sacr.} iii. iv, A pox upon your outlandish feminine
anticks.    
\P 1823 LAMB \textit{Elia} ii. v. (1865) 266 This mortal frame, while thou didst
play thy brief antics amongst us.    
\P 1843 LEVER \textit{Jack Hinton} xxvii. 189
Performing more antics than Punch in a pantomine.

\itembf{3.} A grotesque pageant or theatrical representation. Obs. 

\P 1588 SHAKES. \textit{L.L.L.} v. i. 119 Some delightfull ostentation, or show, or
pageant, or anticke, or fire-worke.    Ibid. v. i. 154 We will haue, if this
fadge not, an Antique.    
\P 1633 FORD \textit{Love's Sacr.} iii. ii, Performed by knights
and ladies of his court, In nature of an antick.    
\P 1673 \textit{Ladies Call.} ii. iii.
§26 How preposterous is it for an old woman‥to be at masks and dancings, when
she is only fit to act the antics.

\itembf{b.} Hence, A grotesque or motley company. rare. 

\P 1589 WARNER \textit{Alb. Eng.} (1612) 345 Heards-men, Sheap⁓heards, Plow-men, and
Hinds: this Anticke of Groomes.

\itembf{4.} A performer who plays a grotesque or ludicrous part, a clown, mountebank,
or merry-andrew. 

\P 1564 \textit{Cap} in \textit{Thynne's Animadv.} App. 130 Thou wearest me‥sometime lyke a
Royster, sometime like a Souldiour, sometime lyke an Antique.    
\P 1592 GREENE in \textit{Shaks. Cent. Praise} 2 Those Anticks garnisht in our colours.    
\P 1618 BP. HALL \textit{Serm.} v. 113 Are they Christians, or Antics in some Carnival?    
\P 1671 MILTON \textit{Samson} 1325 Jugglers and dancers, antics, mummers, mimics.    
\P 1719 DE FOE \textit{Crusoe} (1858) 341 Dancing and hallooing like an antic.    
\P 1827 HOOD \textit{Mids. Fairies} liv,
How Puck, the antic‥Had blithely jested with calamity.

\itembf{b.} transf. and fig. 

\P 1593 SHAKES. \textit{Rich. II}, iii. ii. 162 There [death] the Antique sits, Scoffing
his state, and grinning at his Pompe. [Cf. a 1631 in A3. A death's head grins
like an ‘antic.’]    
\P 1606 G. W[OODCOCKE] \textit{Hist. Justine} 10 b, There flocked a
great throng of souldiers about him, wondering at this so mishapen an Anticke.
\P 1823 LAMB \textit{Elia} ii. xxiv. (1865) 409 [A pun] is an antic which does not stand
upon manners, but comes bounding into the presence.    
\P 1864 DICKENS \textit{Mut. Fr.} ii. i. 172 A little crooked antic of a child.

\itembf{c.} phr. to dance antics. Obs. 

\P 1544 R. ASCHAM \textit{Toxoph.} (Arb.) 47 Myght be thought to daunce Anticke very
properly.    Ibid. 147 Menne that shoulde daunce antiques.    
\P 1602 DEKKER \textit{Satirom.} 245 Yet must we Dance Antickes on your Paper.
[\P 1635 AUSTIN \textit{Medit.}
208 Will Herod reward the Dance of an Antique with the Head of a Prophet?
\P 1687 CONGREVE \textit{Old Bachelor} iii. x. Stage Direct., After the song a dance of
Antics.]

\itembf{5.} Comb., as antick-cutter, a carver of grotesques. 

\P 1660 H. BLOOME \textit{Archit.} (title-page), Antick-Cutters.
\end{myenumerate}

%%%%%%%%%%%%%%%%%%%%%%%%%%%%%%%%
\myitem{antipathy} n.

\noindent \phonetic{(ænˈtɪpəθɪ)}

\noindent [ad. L. antipathīa, a. Gr. ἀντιπάθεια, n. of state f. ἀντιπαθής opposed in feeling, f. ἀντί against + πάθος, πάθε-, feeling. Cf. Fr. antipathie, in Cotgr. 1611.]
\vspace{-0.3cm}

\begin{myenumerate}
\itembf{1.} Contrariety of feeling, disposition, or nature (between persons or things); natural contrariety or incompatibility. The opposite of sympathy. Obs.

\P 1601 HOLLAND  \textit{Pliny} (1634) II. 430 The repugnancie and contrariety in nature which the Greeks call antipathie.    
\P 1605 SHAKES.  \textit{Lear} ii. ii. 93 No contraries hold more antipathy, Then I, and such a knaue.    
\P 1692 BENTLEY  \textit{Boyle Lect.} 97 When occult quality, and sympathy and antipathy were admitted for satisfactory explications of things.

\itembf{b.} Const. with a thing; between things. Obs.

\P 1601 HOLLAND  \textit{Pliny} (1634) II. 227 Such a contrarietie in nature or Antipathie there is‥between them and this herb.    
\P 1626 BACON  \textit{Sylva} §983 The Sea Hare hath an Antipathy with the Lungs‥ and erodeth them.    
\P 1655 W. GURNALL  \textit{Chr. in Arm.} ix. §2 (1669) 348/1 An Antipathy betwixt sinning and praying.

\itembf{2.} Feeling against, hostile feeling towards; constitutional or settled aversion or dislike.

\P 1606 WARNER  \textit{Alb. Eng.} xiv. lxxxii. 344 Were other Rankes not free of Publique-weales Antipathie.    
\P 1663 BUTLER  \textit{Hud.} i. i. 208 A Sect, whose chief Devotion lies In odd perverse Antipathies; In falling out with that or this.    
\P 1734 tr.  \textit{Rollin's Anc. Hist.} (1827) I. 144 Mutual hatred and antipathy.    
\P 1853 C. BRONTË  \textit{Villette} viii. (1876) 67 To attempt to touch her heart was the surest way to rouse her antipathy.

\itembf{b.} Const. against, to; between persons.

\P 1618 WITHER  \textit{Nec Habeo} Wks. 1633, 517, I no Antipathy (as yet) have had Twixt me and any Creature God hath made.    
\P 1667 PRIMATT  \textit{City \& Count. Build.} 28 A kind of Antipathy against the thriving of any but themselves.    
\P 1712 ADDISON  \textit{Spect.} No. 440 \cardo{⁋}5 Having the same Natural Antipathy to a Pun, which some have to a Cat.    
\P 1858 MAX MÜLLER  \textit{Chips} (1880) II. xxvii. 324 A mutual antipathy between the white and the black man.

\itembf{3.} concr. \textbf{a.} That which is contrary in nature (obs.). \textbf{b.} The object of antipathy or settled dislike.

\P 1622 MASSINGER \& DEKKER  \textit{Virg. Mart.} iv. iii, To go Where all antipathies to comfort dwell.    
\P 1691 NORRIS  \textit{Pract. Disc.} 205 Evil is the great antipathy of Human Nature‥her great and general Abhorrence.    
\P 1777 SHERIDAN  \textit{Trip to Scarb.} xi. i, Men that may be called the beau's antipathy, for they agree in nothing but walking upon two legs.
\end{myenumerate}


%%%%%%%%%%%%%%%%%%%%%%%%%%%%%%%%
\myitem{antiquity} n.

\noindent \phonetic{(ænˈtɪkwɪtɪ)}

\noindent [a. Fr. antiquité, 11th c. antiquitet, ad. L. antīquitāt-em, n. of quality f. antīqu-us: see antique and -ity.]
\vspace{-0.3cm}

\begin{myenumerate}

\itembf{I.} As abstract n.

\itembf{1.} The quality of being old (in the world's history) or ancient; long standing, oldness, ancientness.

\P 1450 \textit{Court  of Love} lxxii, This statute was of old antiquite.    
\P 1532 MORE  \textit{Confut. Tindale} Wks. 1557, 707/1 Then be you Jewes of more antiquitie then they.    
\P 1687 T. BROWN  \textit{Saints in Upr.} Wks. I. 73 A rusty spear, and a cloak of antiquity.    
\P 1752 JOHNSON  \textit{Rambl.} No. 192 ⁋2 Every Man boasted the antiquity of his family.    
\P 1851 D. WILSON  \textit{Preh. Ann.} II. iii. vi. 153 The geological antiquity of man.

\itembf{2.} Old age (of human life); seniority. Obs.

\P 1596 SHAKES.  \textit{2 Hen. IV,} i. ii. 208 Is not your voice broken?‥and euery part about you blasted with Antiquity.    
\P 1618 BOLTON  \textit{Florus} i. i. 7 Who for their authoritie should be called Fathers, and for their antiquitie, Senators, or Aldermen.    
\P 1677 \textit{Act} in  Marvell \textit{Growth Popery} 30 Three‥to be placed in such Order as the said Prelates‥think fit, without regard to dignity, antiquity, or any other form.

3.I.3 Ancient character or style.

\P 1850 LYNCH  \textit{Theoph. Trin.} ix. 164 There is much novelty without hope, much antiquity without sacredness.    
\P 1867 MAX MÜLLER  \textit{Chips} (1880) III. xiii. 248 The air of antiquity which pervades that county [Cornwall].

\itembf{II.} Elliptical senses.

4.II.4 The time of antiquity, olden time. a.II.4.a generally.

c\P 1380 \textit{Sir Ferumb.} \P 1316 An old  for-sake ȝeate‹revsc› of þe olde antiquytee.    
\P 1580 BARET  \textit{Alv.} A 421 Historie is the reporter of antiquitie, or of things done in olde tyme.    
\P 1605 BACON  \textit{Adv. Learn.} ii. ii. §7 Antiquity is like fame, caput inter nubila condit, her head is muffled from our sight.    
\P 1664 H. MORE  \textit{Myst. Iniq.} 473 The errours and Mistakes of dark Antiquity.    
\P 1712 \textit{Spect.} No. 548 \cardo{⁋}4, I cannot think of one real hero in all antiquity so far raised above human infirmities.
c\P 1854 STANLEY  \textit{Sinai \& Pal.} ii. (1858) 119 To what an antiquity does this carry us back! Ruins before the days of those who preceded the Philistines!

\itembf{b.} spec. The period before the middle ages, the time of the ancient Greeks and Romans.

\P 1450 \textit{Songs  \& Poems Costume} 53 Famous poetis of antyquyté, In Grece and Troye.    
\P 1594 T. B. \textit{La Primaud. Fr. Acad.} II. 535 The writings of al antiquity.
a\P 1704 T. BROWN  \textit{Comic. View} Wks. I. 157 Galen and other reverend blockheads of antiquity.    
\P 1874 BLACKIE  \textit{Self-Cult 73} The coolest and most practical thinker of all antiquity‥Aristotle.

\itembf{c.} The early ages of the Christian era; the early centuries of the Church; more explicitly Christian antiquity.

\P 1564 HARDING  \textit{Answ. Jewel} 173 To see antiquitie for proufe hereof‥Let him reade [etc.].    
\P 1574 BRISTOW  \textit{Brief Treat. Diuerse Plain Waies} (1599) 54 All Antiquity is full of such practise.    
\P 1753 CHALLONER  \textit{Cath. Chr. Instr.} 77 This Custom‥is as ancient as Christianity, as appears from the most certain Monuments of Antiquity.    
\P 1850 NEWMAN  \textit{Difficulties of Anglicans} ii. 34 He would‥have given up the Establishment, rather than have rejected antiquity.    
\P 1860 A. P. DE  LISLE in E. Purcell \textit{Life} (1900) I. x. 185 Christian Antiquity.

\itembf{5.} The people (or writers, etc.) of ancient times collectively; ‘the Ancients.’

\P 1538 STARKEY  \textit{England} iii. 78 Aftur the opynyon of the wyse and auncyent antyquyte.    
\P 1598 BARRET  \textit{Theor. Warres} v. iii. 152 This manner of marching‥we reade antiquitie to have vsed.    
\P 1641 MILTON  \textit{Prel. Episc.} (1851) 73 That indigested heap, and frie of Authors, which they call Antiquity.    
\P 1726 DE FOE  \textit{Hist. Devil} ii. vi. (1840) 246 We have Antiquity on our side, we have this truth confirmed by the testimony of many ages.    
\P 1876 MOZLEY  \textit{Univ. Serm.} i. 3 We think we have excelled all antiquity. We have excelled a later antiquity, but not the earliest and first.

\itembf{6.} (Now pl. or collect., formerly often sing.) Matters, customs, precedents, or events of earlier times; ancient records.

\P 1557 NORTH  \textit{Diall of Princes} A ij b, Paulus Diaconus‥sheweth an antiquitie right worthy to remember.    
\P 1629 COKE  \textit{On Litt.} 69 a, Which Antiquity I cite for that it concurreth with the act of Parliament.    
\P 1660 H. BLOOME  \textit{Archit.} Title-page, Gathered with great diligence‥out of Antiquities.    
\P 1782 PRIESTLEY  \textit{Corrupt. Chr.} I. i. 107 Whiston‥was certainly well read in Christian antiquity.    
\P 1876 DIGBY  \textit{Real Prop.} ii. §8. 94 The subject belongs entirely to the antiquities of our law.

\itembf{7.} (Now usually pl.; formerly sing. or collect.) Remains or monuments of antiquity; ancient relics.

\P 1513 MORE  \textit{Hist. Edw. V,} Ded. 1 The great care‥that hath alwaies been observed‥for the preservation of antiquities.    
\P 1605 BACON  \textit{Adv. Learn.} ii. ii. §1 Antiquities are history defaced, or some remnants of history which have casually escaped the shipwreck of time.    
\P 1622 PEACHAM  \textit{Compl. Gentl.} xii. (1634) 112, I come to the last of our select Antiquities, Coynes.    
\P 1676 D'URFEY  \textit{Mad. Fickle} iii. i, Rust adds to an Antiquity, 'tis our Friend.    
\P 1728 STUKELEY in \textit{Phil. Trans.} XXXV. 430 At Paunton‥I have heard of much Antiquity being found.    
\P 1787 T. JEFFERSON  \textit{Writ.} (1859) II. 133 The Pont du Gard, a sublime antiquity, and well preserved.    
\P 1869 RAWLINSON  \textit{Anc. Hist.} 2 Antiquities, or the actual extant remains of ancient times.

\itembf{8.} Comb. or attrib., as antiquity-hunting, antiquity piece.

\P 1860 \textit{Vac. Tour.} 119 The bishop of Ossory, who was antiquity-hunting in Sutherland.    
\P 1711 \textit{London  Gaz.} mmmmdccclv/4 A small Gold Ring, with an Antiquity Piece hanging to it.
\end{myenumerate}


%%%%%%%%%%%%%%%%%%%%%%%%%%%%%%%%
\myitem{antithesis} n.

\noindent \phonetic{(ænˈtɪθɪsɪs)}

\noindent [a. L. antithesis, a. Gr. ἀντίθεσις opposition, n. of action f. ἀντιτιθέναι, f. ἀντί against + τιθέναι (stem θε-) to place; already in Gr. a term of Logic and Rhetoric.]
\vspace{-0.3cm}

\begin{myenumerate}

\itembf{1.} Rhet. An opposition or contrast of ideas, expressed by using as the corresponding members of two contiguous sentences or clauses, words which are the opposites of, or strongly contrasted with, each other; as ‘he must increase, but I must decrease,’ ‘in newness of spirit, not in the oldness of the letter.’

\P 1529 FRITH  \textit{(title)} Antithesis; wherein are compared togeder Christes actes and oure holye Father the Popes.    
\P 1674 \textit{Gove. Tongue} iii. §17. 115 These are miserable antithesis's.    
\P 1728 POPE  \textit{Dunc.} i. 254 All arm'd with points, antitheses and puns.    
\P 1748 J. MASON  \textit{Elocution} 29 In an Antithesis, one contrary must be pronounced louder than the other.    
\P 1872 W. MINTO  \textit{Eng. Lit.} Introd. 9 When the balanced clauses stand in antithesis, it lends emphasis to the opposition.

\itembf{2.} The second of two such opposed clauses or sentences; a proposition opposed to a thesis; a counter-thesis or -proposition.

\P 1533 FRITH  \textit{Answ. More} F ij, As the contrarye antithesis doth euidently expresse.    
\P 1677 GALE  \textit{Crt. Gentiles} III. Pref., Impossible‥to discusse such an hypothesis without some opposition against such as defend the antithesis.    
\P 1678 OWEN  \textit{Mind of God} iii. 91 Given to disputing, or the maintaining of Antitheseses, or oppositions unto the Truth    
\P 1833 COLERIDGE  \textit{Table T.} 264 The style of Junius is a sort of metre, the law of which is a balance of thesis and antithesis.

\itembf{3.} By extension: Direct or striking opposition of character or functions (between two things); contrast. Const. of, between (with obs.).

\P 1631 PRESTON  \textit{Effec. Faith} 40 That Antithesis, that opposition that is made in that withdrawing of a mans selfe from God.    
\P 1850 KINGSLEY  \textit{Alt. Locke} xxxviii. (1879) 410 The antithesis of natural and revealed religion.    
\P 1872 DARWIN  \textit{Emotions} i. 5 Movements, so clearly expressive of affections‥being in complete opposition or antithesis to the attitude and movements which are expressive of anger.

\itembf{4.} The direct opposite, the contrast. Const. of, to.

\P 1831 MACAULAY  \textit{Moore's Byron, Ess.} I. 161 The reverse of a great dramatist, the very antithesis to a great dramatist.    
\P 1857 H. REED  \textit{Lect. Brit. Poets} vii. 244 Rhyme is sometimes taken as the antithesis of reason.    
\P 1879 FARRAR  \textit{Paul} II. 327 Is not the Pharisaic spirit‥the antithesis of the Christian?

\itembf{5.} (See quot.) Obs.

\P 1591 PERCIVALL  \textit{Sp. Dict.} B ij a, Antithesis, or Antistœchon: where if l follows immediately after r‥they change r into l, to make the sound the pleasanter, as for Dexarle, dexalle.    
\P 1657 J. SMITH  \textit{Myst. Rhet.} 172 Antithesis is sometimes a figure, whereby one letter is put for another; and then it is the same with Antistoichon.
\end{myenumerate}


%%%%%%%%%%%%%%%%%%%%%%%%%%%%%%%%%
\myitem{aphorism} n.

\noindent \phonetic{(ˈæfərɪz(ə)m)}

\noindent [a. Fr. aphorisme, afforisme, ad. med.L. aphorism-us, aforismus, a. Gr. ἀϕορισµός a distinction, a definition, f. ἀϕορίζ-ειν; see aphorize. From the ‘Aphorisms of Hippocrates,’ transferred to other sententious statements of the principles of physical science, and at length to statements of principles generally.]
\vspace{-0.3cm}

\begin{myenumerate}

\itembf{1.} A ‘definition’ or concise statement of a principle in any science.

\P 1528 PAYNELL  \textit{Salerne Regim.} B iv b, Galen saythe in the glose of this aphorisme, qui crescunt, etc.    
\P 1541 R. COPLAND  \textit{Guydon's Quest. Cyrurg.,} Of this vtylyte Arnolde of vylle maketh an afforysme.    
\P 1605 BACON  \textit{Adv. Learn.} i. v, Knowledge, while‥in aphorisms and observations‥is in growth.    
\P 1664 POWER  \textit{Exp. Philos.} iii. 190 The old and uncomfortable Aphorism of our Hippocrates.    
\P 1879 \textit{De Quatrefages' Hum. Spec.} 50 The aphorism‥which was formulated by Linnæus in regard to plants.

\itembf{2.} Any principle or precept expressed in few words; a short pithy sentence containing a truth of general import; a maxim.

\P c1590 MARLOWE  \textit{Faustus} i. 19 Is not thy common talk sound aphorisms?    
\P 1642 HOWELL  \textit{For. Trav.} (Arb.) 37 'Tis an old Aphorisme Oderunt omnes quem metuunt.    
\P 1687 H. MORE  \textit{App. Antidote} (1712) 191 That sensible Aphorism of Solomon, Better is a living Dog than a dead Lion.    
\P 1750 JOHNSON  \textit{Rambl.} No. 68 \cardo{⁋}10 Oppression, according to Harrington's aphorism, will be felt by those that cannot see it.    
\P 1880 GOLDW.  SMITH in \textit{Atl. Month.} No. 268. 201 The suggestive aphorism, ‘The want of belief is a defect that ought to be concealed when it cannot be overcome.’

\itembf{3.} abstractly, The essence or pith. Obs. rare.

\P 1594 J. KING  \textit{Jonah} (1864) 184 The aphorism and juice of the whole song.
\end{myenumerate}


%%%%%%%%%%%%%%%%%%%%%%%%%%%%%%%%%
\myitem{aplomb} n.

\noindent \phonetic{(aˈplʌm, əˈplɒm)}

\noindent [Fr. aplomb perpendicular position, steadfastness, assurance, f. the phr. à plomb ‘according to the plummet.’]
\vspace{-0.3cm}

\begin{myenumerate}

\itembf{1.} ‘The perpendicular’; perpendicularity.

\P 1872 C. KING  \textit{Sierra Nev.} iii. 69 We sprang on, never resting long enough to lose the aplomb.    
\P 1880 MRS. WHITNEY  \textit{Odd or Even} iii. 23 The girl jumped, with clean aplomb, from the wagon-wheel to the broad door-stone.

\itembf{2.} Assurance, confidence, self-possession, coolness.

\P 1828 GEN. P. THOMPSON  \textit{Exerc.} (1842) IV. 548 They never present themselves with any aplomb; but always with some lurking recognition of the power of their adversaries.    
\P 1849 C. BRONTË  \textit{Shirley} xi. 162 Impatience of her chilly ceremony and annoyance at her want of aplomb.

\itembf{3.} attrib. quasi-adj. Self-possessed, confident.

\P 1865 \textit{Gayworthys}  II. 29 Her ordinary aplomb fashion of speech.
\end{myenumerate}


%%%%%%%%%%%%%%%%%%%%%%%%%%%%%%%%%
\myitem{apocalypse} n.

\noindent \phonetic{(əˈpɒkəlɪps)}

\noindent [ad. L. apocalypsis, a. Gr. ἀποκάλυψις, n. of action f. ἀποκαλύπτειν to uncover, disclose, f. ἀπό off + καλύπτειν to cover.]
\vspace{-0.3cm}

\begin{myenumerate}

\itembf{1.} (With capital initial.) The ‘revelation’ of the future granted to St. John in the isle of Patmos. The book of the New Testament in which this is recorded.

\P [c1175 \textit{Lamb. Hom.} 81 Herof seid Seint Johan þe ewangeliste in apocalipsi.]
\P c1230 \textit{Ancr. R.} 94 ‘Hit is a derne halewi,’ seið sein Johan ewangeliste in þe Apocalipse.
\P c1400 \textit{Rom. Rose} 7395 That sallow horse of hewe, That in the Apocalips is shewed.
\P a1440 \textit{Sir Degrev.} 1437 The  Pocalyps of Ion.    
\P 1581 WALKER in \textit{Confer.} iv. (1584) Z iiij b, The Laodicean Councill omitteth Lukes Gospel \& the Apocalyps.    
\P 1667 MILTON  \textit{P.L.} iv. 2 That warning voice which he who saw Th' Apocalyps, heard cry in Heaven aloud.    
\P 1870 DISRAELI  \textit{Lothair} xliv. 230 The long-controverted point whether Rome in the great Apocalypse was signified by Babylon.

\itembf{2.} By extension: Any revelation or disclosure.

\P 1382 WYCLIF  \textit{1 Cor.} xiv. 26 He hath techinge, he hath apocalips, or reuelacioun, he hath tunge.    
\P 1621 BURTON  \textit{Anat. Mel.} 677 (L.) Interpret apocalypses, and those hidden mysteries to private persons.    
\P 1704 SWIFT  \textit{T. Tub} i. (1750) 31 The Revelation or rather the Apocalypse of all State⁓arcana.    
\P 1831 CARLYLE  \textit{Sart. Res.} ii. v, The new apocalypse of Nature unrolled to him.


\itembf{3.a.} Christian Church. The events described in the revelation of St John; the Second Coming of Christ and ultimate destruction of the world.

\P 1862 \textit{R.I. Schoolmaster} (Rhode Island Commissioner Public Schools) 8 22/2 There are those who‥think they already behold its fearful apocalypse terminating in darkness and in blood.    
\P 1947 N. FRYE  \textit{Fearful Symmetry} (1990) iii. 67 The apocalypse will necessarily begin with a slaughter of tyrants, and Christ came, Blake says, to deliver those bound under the knave.    
\P 2008 \textit{Washington Post} (Electronic ed.) 28 Jan. c3 Eddy sends an e-mail to thousands of like-minded Christians announcing: ‘The End Days have arrived. The Apocalypse and the Rapture are at hand.’

\itembf{b.} More generally: a disaster resulting in drastic, irreversible damage to human society or the environment, esp. on a global scale; a cataclysm. Also in weakened use.

\P 1894 J. SWINTON  \textit{Striking for Life} 357 Comrades of Chicago!‥ In these times there are‥prophecies of approaching apocalypse.‥ It will surely come.    
\P 1940 \textit{Common  Sense} Mar. 4/2 Washington is preoccupied with the threat of apocalypse across the Atlantic.    
\P 1980 \textit{Bookseller}  26 Jan. 316/2 Although most people are saddened by the enforced abandonment of some titles, no one is prepared to interpret it as the publishers' apocalypse.    
\P 1994 \textit{Time}  24 Oct. 33 While the poor are bewitched by dreams of peace and plenty, the rich are preparing for an apocalypse.
\end{myenumerate}


%%%%%%%%%%%%%%%%%%%%%%%%%%%%%%%%%
\myitem{apocryphal} a. and n.

\noindent \phonetic{(əˈpɒkrɪfəl)}

\noindent [f. as prec. + -al1.]
\vspace{-0.3cm}

\begin{myenumerate}

\itembf{A.} adj. Of doubtful authenticity; spurious, fictitious, false;
fabulous, mythical. \textbf{a.} orig. of a writing, statement, or story.

\P 1590 J. GREENWOOD  \textit{Sland. Art} B ij b, We hold them‥not only a babling, but apochriphall \& Idolatrous.    
\P 1678 BUTLER  \textit{Hud.} iii. i. 492 If but one word be true‥In all th' apocryphal romance.    
\P 1868 FREEMAN  \textit{Norm. Conq.} II. App. 569 The tale has a somewhat apocryphal sound.

\itembf{b.} spec. Of or belonging to the Jewish and early Christian uncanonical literature.

\P 1615 \textit{Curry-C for Coxe-C.} ii. 93 Peremptory‥against the Canonizing of these Apogriphall bookes.    
\P 1711 ADDISON  \textit{Spect.} No. 28 \cardo{⁋}6 Our Apocryphal Heathen God [Bel]‥in conjunction with the Dragon.    
\P 1865 LECKY  \textit{Ration.} (1878) I. 210 The apocryphal gospels‥were for the most part of Gnostic origin.

\itembf{c.} gen. Unreal, counterfeit, sham, ‘imitation.’

\P 1610 B. JONSON  \textit{Alchemist} i. i, A whoreson, upstart, apocryphal captain.    
\P 1649 C. WALKER  \textit{Hist. Indep.} ii. 226 This Agreement was‥complained of in the apocryphal House of Commons.    
\P 1843 JERROLD  \textit{Punch's Lett.} xx. Wks. I. 473 He lived by putting off pencils, with apocryphal lead in them.

\itembf{B.} n. An apocryphal writing. Obs. rare.

\P 1661 \textit{Grand  Debate} 13 Some Psalm or Scripture Hymn‥instead of that Apocryphal [the Benedicite].    
\P 1677 J. HANMER  \textit{View of Antiq.} 419 (T.) Nicephorus and Anastasius‥did rank these epistles in the number of apocryphals.
\end{myenumerate}


%%%%%%%%%%%%%%%%%%%%%%%%%%%%%%%%%
\myitem{apogee} Astr.

\noindent \phonetic{(ˈæpəʊdʒiː)}

\noindent [a. Fr. apogée (in Cotgr. 1611), f. L. apogæum, a. Gr. ἀπόγαιον (also ἀπόγειον), adj. neut. ‘away from the earth,’ (f. ἀπό off, from + γάϊος, γεῖος of the earth, f. γαῖα, γῆ the earth), but used absol. by Ptolemy (sc. διάστηµα distance) in the modern astronomic sense. Formerly used in Gr. or L. form apogeon, -gæum, -geum.]
\vspace{-0.3cm}

\begin{myenumerate}

\itembf{1.} The point in the orbit of the moon, or of any planet, at which it is at its greatest distance from the earth; also, the greatest distance of the sun from the earth when the latter is in aphelion. (A term of the Ptolemaic Astronomy, which viewed the earth as the centre of the universe; in modern astronomy strictly used in reference to the moon, and popularly said of the sun in reference to its apparent motion.)

\P 1594 J. DAVIS  \textit{Seamans Secr.,} Her Slowe Motion is in the point of Auge or apogeo.    
\P 1656 tr.  \textit{Hobbes' Elem. Philos.} (1839) 443 The apogæum of the sun or the aphelium of the earth.    1727–51 Chambers Cycl., Apogee is a point in the heavens at the extreme of the line of the apsides.    
\P 1812 WOODHOUSE  \textit{Astron.} xix. 206 Apogee, if the Sun be supposed to revolve, Aphelion, if the Earth.    
\P 1868 LOCKYER  \textit{Heavens} (ed. 3) 130 The greatest distance of the Moon from the Earth is about 643/4 the equatorial radius of our globe. When the Moon is at this distance, it is said to be in apogee.

\itembf{2.} The greatest altitude reached by the sun in his apparent course; his meridional altitude on the longest day. Obs.

\P 1605 BACON  \textit{Adv. Learn.} (1640) 146 The Apogée or middle point; and Perigée or lowest point of heaven.    
\P 1646 SIR T. BROWNE  \textit{Pseud. Ep.} vi. v. (1686) 242 In the Apogeum or highest point it is not so hot under that Tropick.

\itembf{3.} Hence fig. \textbf{a.} The most distant or remote spot. b.3.b The highest point, climax, culmination.

\P 1600 FAIRFAX  \textit{Tasso} ii. lxvii. 33 Thy Sunne is in his Apogæon placed, And when it moueth next, must needes descend.    
\P 1642 H. MORE  \textit{Song of Soul} ii. iii. ii. xii, She [the Soul] doth ascend, Unto her circles ancient Apogie.    
\P 1670 EACHARD  \textit{Contempt Clergy} 54 Sometimes he withdraws himself into the apogæum of doubt, sorrow, and despair.    
\P 1858 MOTLEY  \textit{Dutch Rep.} vi. Introd. 33 The trade of the Netherlands‥had however by no means reached its apogee.

\itembf{4.} The point in the trajectory of a missile, rocket, or the like at which it is at its greatest distance from the earth.

\P 1958 in \textit{Aero-Space Terms.}
\P 1961 \textit{Flight}  LXXX. 756/1 When the satellite reaches the 22,300 mile apogee of the trajectory‥the solid-propellant apogee motor will be used to inject the satellite into a circular, near-synchronous orbit.    
\P 1962 J. GLENN in \textit{Into Orbit} 6 The apogee or highest point of the capsule's orbit was over eight times that altitude.
\end{myenumerate}


%%%%%%%%%%%%%%%%%%%%%%%%%%%%%%%%%
\myitem{apostate} n. and a.

\noindent \phonetic{(əˈpɒstət)}

\noindent [a. Fr. apostate and L. apostata, ad. Gr. ἀποστάτ-ης, n. of agent f.
ἀποστα- (see apostasy). The L. apostata was by far the commoner form from \P 1350 to 1650, with pl. apostata(e)s.]
\vspace{-0.3cm}

\begin{myenumerate}

\itembf{A.} n.

\itembf{1.} One who abjures or forsakes his religious faith, or abandons his moral allegiance; a pervert.

\P 1340 \textit{Ayenb.} 19 Þe heretike and þe apostate þet reneyeþ hire bileaue.
\P c1380 WYCLIF  \textit{Wycket} 1 Infideles papistes and apostates.
\P c1400 \textit{Apol. Loll.} 93 To haue brokyn þe cristun feiþ‥\& to be paynims \& apostatais.    
\P 1491 CAXTON  \textit{Vitas Patr.} (W. de W.) ii. 309 a/1 Julyan thappostata.    
\P 1583 GOLDING  \textit{Calvin on Deut.} cc. 1246 For if we play ye Papistes‥we shall be apostataes.    
\P 1622 MASSINGER  \textit{Virg. Mart.} iii. i, In hopes to draw back this apostata‥Unto her father's faith.    
\P 1667 MILTON  \textit{P.L.} vi. 100 High in the midst exalted as a God Th' Apostate in his Sun-bright Chariot sate.    
\P 1728 YOUNG  \textit{Love Fame} i. (1757) 80 Polite apostates from God's Grace to Wit.    
\P 1808 SCOTT  \textit{Marm.} ii. iv, For inquisition stern and strict On two apostates from the faith.

\itembf{b.} R.C. Ch. A member of a religious order who renounces the same without legal dispensation.

\P c1387 TREVISA  \textit{Higden} vii. iv. Rolls Ser. VII. 309 An apostata þat brekeþ his ordre þey fongeþ nevere aȝen.    
\P 1401 \textit{Pol. Poems} II. 19 If you leave your habite a quarter of a yeare, ye should be holden apostataes.    
\P 1577 HOLINSHED  \textit{Chron.} III. 1239/1 One Rafe sometime a moonke of Glastenburie, and now become an apostata.    
\P 1855 MILMAN  \textit{Lat. Chr.} (1864) IX. xiv. i. 26 The renegade who pursued his private interests by sacrificing those of his order‥stood alone a despised and hated apostate.

\itembf{2.} One who deserts his party, or forsakes his allegiance or troth; a turncoat, a renegade.

\P 1362 LANGL.  \textit{P. Pl.} A. i. 102 He þat passeþ þat poynt is apostata in þe ordre. 
[\P 1393 Ys apostata  of knyȝt-hod.]    
\P 1608 J. DAY  \textit{Hum. out Breath} (1881) 53 Should he proue Apostata, denie Loue which he first enforcd vs to profes.
\P a1687 PETTY  \textit{Pol. Arith.} iii. (1691) 58 Apostates, to their own Country, and Cause.    
\P 1769 \textit{Junius  Lett.} i. (1804) I. 5 We see him, from every honourable engagement to the public, an apostate by design.    
\P 1826 DISRAELI  \textit{Viv. Grey} vii. ii. 388 No one is petted so much as a political apostate, except, perhaps a religious one.

\itembf{B.} adj.

\itembf{1.} Unfaithful to religious principles or creed, or to moral allegiance; renegade, infidel; rebellious.

\P 1382 WYCLIF  \textit{Ezek.} ii. 2 Folkis apostataas, that han broken her religioun.   
\P c1486 \textit{Bk. St. Albans Arms} C j a, The maruellis deth of Julian thappostita Emproure.    
\P 1590 H. BARROW in J. Greenwood \textit{Confer.} 6 All the parish‥were generally apostate.    
\P 1592 NASHE  \textit{P. Peniless} 33 b, Those Apostata spirits that rebelled with Belzebub.    
\P 1667 MILTON  \textit{P.L.} i. 125 So spake th' Apostate Angel.    
\P 1758 JORTIN  \textit{Erasmus} I. 176 Eggs of heresy, which the apostata Fryer Luther had before laid.    
\P 1878 C. STANFORD  \textit{Symb. Christ} i. 7 The last witness left for God in the midst of an apostate land.

\itembf{2.} gen. Deserting principles or party; perverted.

\P 1671 MARVELL  \textit{Corr.} 198 Wks.1872 II. 394 The apostate patriots, who were bought off.    
\P 1712 STEELE  \textit{Spect.} No. 516 \cardo{⁋}7 Those apostate abilities of men.
\end{myenumerate}


%%%%%%%%%%%%%%%%%%%%%%%%%%%%%%%%%
\myitem{apotheosis} n.

\noindent \phonetic{(æpəʊˈθiːəsɪs, əˌpɒθiːˈəʊsɪs)}

\noindent [a. L. apotheōsis (Tertull.), a. Gr. ἀποθέωσις, n. of action f. ἀποθεό-ειν to deify, f. ἀπό off, (in comb.) completely + θεό-ειν to make a god of, f. θεός god. The great majority of orthoepists, from Bailey and Johnson downward, give the first pronunciation, but the second is now more usual.]
\vspace{-0.3cm}

\begin{myenumerate}

\itembf{1.} The action of ranking, or fact of being ranked, among the gods; transformation into a god, deification; divine status.

[\P 1577 tr.  \textit{Bullinger's Decades} (1592) 759 Truely Aurelius Prudentius in his Apotheosis‥saith.]    
\P 1605 BACON  \textit{Adv. Learn.} i. 32 That which the Grecians call Apotheosis‥was the supreme honour, which a man could attribute unto man.    
\P 1677 HALE  \textit{Prim. Orig. Man.} ii. ii. 137 The Apotheoses or Inaugurations of many of the Heathenish Deities.    
\P 1879 FARRAR  \textit{Paul} I. 664 The early Emperors rather discouraged‥this tendency to flatter them by a premature apotheosis.

\itembf{2.} By extension: The ascription of extraordinary, and as it were divine, power or virtue; glorification, exaltation; the canonization of saints.

\P [1553–87 FOXE \textit{A. \& M.} I. 662/2 You‥affirm, that in this my Calendar, I make
an αποθεωσιν, or Canonization of false Martyrs.]   
\P 1651 HOBBES  \textit{Govt. \& Soc.} xviii. §14. 362 The canonization of Saints which the Heathen called Apotheosis.    
\P 1739 \textit{Gentl. Mag.} (title) The Apotheosis of Milton.    
\P 1758 JORTIN  \textit{Erasmus} I. 305 He promises‥to send him the apotheosis of his friend Reuchlin.    
\P 1879 O'CONNOR  \textit{Beaconsfield} 73 The meeting developed into an apotheosis of the Marquis of Chandos.

\itembf{3.} The deification, glorification, or exaltation of a principle, practice, etc.; a deified ideal.

\P 1651 BIGGS  \textit{New Disp.} \cardo{⁋}211 Because in the Apotheosis of phlebotomy they will have good bloud emitted.    
\P 1810 COLERIDGE  \textit{Friend} (1865) 143 The apotheosis of familiar abuses‥is the vilest of superstitions.    
\P 1846 PRESCOTT  \textit{Ferd. \& Is.} I. Introd. 35 The apotheosis of chivalry, in the person of their apostle and patron, St. James.    
\P 1852 A. JAMESON  \textit{Leg. Madonna} (1857) 47 Here all is spotless grace, etherial delicacy‥the very apotheosis of womanhood.

\itembf{4.} In loose usage: Ascension to glory, departure or release from earthly life; resurrection.

\P 1649 C. WALKER  \textit{Hist. Indep.} ii. 111 His Majesties Speech upon the Scaffold, and His Death or Apotheosis.    
\P 1680 H. MORE  \textit{Apocal. Apoc.} Pref. 17 The most assured argument‥of the apotheosis of Christ.    
\P 1684 T. BURNET  \textit{Th. Earth} I. 326 The general apotheosis; when death and hell shall be swallowed up in victory.    
\P 1850 CARLYLE  \textit{Latter-d. Pamph.} i. (1872) 25 Let us hope the Leave-alone principle has now got its apotheosis; and taken wing towards higher regions than ours.    
\P 1858 R. VAUGHAN  \textit{Ess. \& Rev.} I. 8 The philosophical school of Alexandria had become extinct, and there was no apotheosis.
\end{myenumerate}


%%%%%%%%%%%%%%%%%%%%%%%%%%%%%%%%%
\myitem{appellation} n.

\noindent \phonetic{(æpəˈleɪʃən)}

\noindent [a. Fr. appellation (13th c.), ad. L. appellātion-em, n. of action f. appellāre: see appeal v. and -tion.]
\vspace{-0.3cm}

\begin{myenumerate}

\itembf{I.} Appealing, appeal. [from OFr. apeler.] Obs.

\itembf{1.} The action of appealing to a higher court or authority against the decision of an inferior one; the appeal so made; = appeal n. 3. Obs.

\P 1494 FABYAN  vii. 479 In iugement vpon the appellacions before made by the erle of Armenak‥agayne prynce Edwarde.    
\P 1538 STARKEY  \textit{England} 125 Another grete mysordur, in appellatyon of such as be callyd spiritual causys.    
\P 1547 \textit{Homilies}  i. ix. (1859) 92 The condemnation both of body and soul, without either appellation or hope of redemption.    
\P 1609 SKENE  \textit{Reg. Maj.} 65 In Ecclesiasticall causes appellation is admitted within fourtie dayes.    
\P 1669 HONYMAN  \textit{Surv. Naphtali} II. 105 Pauls appellation to Cæsar, Acts xxv. ii.    
\P 1679 FILMER  \textit{Freeholder} 66 There might be Appellation made to the Kings Person.

\itembf{b.} Ground of appeal, title, claim. Obs. rare.

\P 1630 NAUNTON  \textit{Fragm. Reg.} (Arb.) 26 He could not find out any appellation to assume the Crown in his own Person.

\itembf{2.} gen. The action or process of appealing or calling on; entreaty, or earnest address. Obs.

\P 1587 M. GROVE  \textit{Pelops \& Hipp.} (1878) 18 No god there was but him they had in appellation.    
\P 1589 \textit{Hay any Work} 43 His appellation to the obedient cleargie.    
\P 1671 \textit{True  Non-Conf.} 399 Master Knox his reasoning‥in his appellation and admonition to the commonalty.

\itembf{II.} Calling, designation. [from later Fr. appeler, or L. appellāre.]

\itembf{3.} The action of calling by a name; nomenclature.

\P 1581 CAMPION in \textit{Confer.} iii. (1584) U iiij, Euery piece of bread is called bread‥because it was bread by appellation.    
\P 1630 PRYNNE  \textit{Anti-Armin,} 126 If it be grace in truth, as well as in appellation.    
\P 1742 HUME  \textit{Ess.} (1817) I. 36 The government, which in common appellation receives the appellation of free.    
\P 1875 WHITNEY  \textit{Life Lang.} ii. 27 They must be carefully distinguished in appellation.

\itembf{4.} A designation, name, or title given: a.II.4.a to a particular person or thing.

\P 1447 O. BOKENHAM  \textit{Lyvys of Seyntys} 44 Anne is as myche to seyn as grace And worthyly thys appellacyoun To hyr pertenyth.    
\P 1610 \textit{Histriom.}  i. 136 Seri. Your appellations? Post. Your names he meanes. The man's learn'd. 
\P a1674 CLARENDON  \textit{Hist. Reb.} I. i. 15 Stenny, an appellation he allways used of and towards the Duke.    
\P 1774 PRIESTLEY  \textit{Observ. Air} 178 By the common appellation of phlogisticated air.    
\P 1833–48 H. COLERIDGE \textit{North. Worth.} (1852) I. 69 Which entitles him to the appellation of a prose Juvenal.

\itembf{b.} to a class: A descriptive or connotative name.

\P 1581 MARBECK  \textit{Bk. of Notes} 665 Manes the Hereticke, whereof the Maniches haue their appellation.    
\P 1651 HOBBES  \textit{Govt. \& Soc.} vii. §3. 112 If he‥Rule well‥they afford him the appellation of a King; if not, they count him a Tyrant.    
\P 1709 SWIFT  \textit{T. Tub.} iii. 50 These men seem‥to have understood the appellation of critic in a liberal sense.    
\P 1841 BORROW  \textit{Zincali} I. vi. §1. 102 If not sorcerers, they have always done their best to merit that appellation.
\end{myenumerate}


%%%%%%%%%%%%%%%%%%%%%%%%%%%%%%%%%
\myitem{apposite} a.

\noindent \phonetic{(ˈæpəzɪt)}

\noindent [ad. L. apposit-us, pa. pple. of app-, adpōnĕre, f. ad to + -pōnĕre to place, put.]
\vspace{-0.3cm}

\begin{myenumerate}

\itembf{1.} Put or applied to. Obs. rare—0.

\P 1656 in BLOUNT,    
\P 1706 in PHILLIPS, etc.

\itembf{2.} Well put or applied; appropriate, suitable (to).

\P 1621 BURTON  \textit{Anat. Mel.} ii. ii. ii. (1651) 239 A most apposite remedy.    
\P 1634 HABINGTON  \textit{Castara} (1870) 15 Her language is not copious but apposit.    
\P 1709 SWIFT  \textit{T. of Tub} §3. 54 The types are so apposite.    
\P 1849 GROTE  \textit{Greece} ii. lv. (1862) V. 31 Mastery of apposite and homely illustrations.    
\P 1869 GOULBURN  \textit{Purs. Holiness} i. 6 The truth most apposite to the whole argument.

\itembf{3.} Of persons: Ready with appropriate remarks, apt. Obs.

\P 1703 POMFRET  \textit{Poet. Wks.} (1833) 31 In all discourse she's apposite and gay.    
\P 1788 H. WALPOLE in \textit{Reader} 7 Oct. 1865, 392/3 Qualified to talk on any subject; easy, agreeable, and apposite in their observations.

\itembf{4.} absol. or as n. That which is placed beside or in apposition. Obs.

\P 1677 GALE  \textit{Crt. Gent.} II. iv. 516 The negation of it implies a contradiction in the Adject or an Opposite in an Apposite.

\itembf{5.} See OPPOSITE.
\end{myenumerate}


%%%%%%%%%%%%%%%%%%%%%%%%%%%%%%%%%
\myitem{apprehend} v.

\noindent \phonetic{(æprɪˈhɛnd)}

\noindent [a. Fr. appréhende-r (15th c. in Godef.), ad. L. app-, adprehend-ĕre to lay hold of, seize, f. ad to + prehend-ĕre to seize. In the contracted form apprend-ĕre, the word survived in the Romance langs. in the fig. sense ‘lay hold with the mind, comprehend, learn,’ whence also later ‘teach, inform’: cf. Fr. apprendre, and Eng. apprise. Subsequently, the full apprehend-ĕre was taken into Fr. and Eng. in its orig. form and sense. apprend is occas. in 16–17th c.]
\vspace{-0.3cm}

\begin{myenumerate}

\itembf{I.} Physical.

\itembf{1.} To lay hold upon, seize, with hands, teeth, etc. Also said of fire, and fig. of trembling, fear, etc. Obs. or arch.

\P 1572 J. BOSSEWELL  \textit{Armorie} iii. 5 A great quakyng and tremblyng dyd apprehende hys hande.    
\P 1607 TOPSELL  \textit{Four-f. Beasts} 124 His dogs‥apprehending the garments of passengers.    
\P 1613 \textit{Life  William I} in \textit{Harl. Misc.} (1793) 28 A fire began‥which apprehending certain shops and warehouses, etc.
\P c1643 \textit{Maximes  Unf.} 8 Fury and affrightment apprehend the desperate.    
\P 1645 RUTHERFORD  \textit{Tryal \& Tri. Faith} (1845) 63 A lame hand that cannot apprehend.    
\P 1843 E. JONES  \textit{Sensat. \& Event} 122 While those two lips his brow did apprehend.

\itembf{b.} transf. To seize upon, take down, in writing. fig. To seize upon (points of a subject). Obs.

\P 1611 CORYAT  \textit{Crudities} 480, I apprehended it [an epitaph] with my pen while the Preacher was in his pulpit.    
\P 1615 T. ADAMS  \textit{Spir. Navig.} 24, I will only apprehend so much as may serve to exemplify this dangerous world.

\itembf{2.} To seize (a person) in name of law, to arrest.

\P 1548 UDALL, etc. \textit{Erasm. Par. John} vii. 1 (R.) To fynde sum occasion‥to attache and apprehende him.    
\P 1642 ROGERS  \textit{Naaman} 44 Paul‥going like a Pursivant‥to Damascus, to apprehend the Saints there.    
\P 1768 BLACKSTONE  \textit{Comm.} IV. 287 A justice of the peace cannot issue a warrant to apprehend a felon upon bare suspicion.    
\P 1855 MACAULAY  \textit{Hist. Eng.} III. 328 Troops had been sent to apprehend him.

\itembf{3.} To seize upon for one's own, take possession of. Also fig. Obs.

\P 1513 DOUGLAS  \textit{Æneis} xi. vii. 70 Ellis quhare‥to wend, Thayre dwelling place for ay to apprehend.    
\P 1611 BIBLE  \textit{Phil.} iii. 12 If that I may apprehend that for which also I am apprehended of Christ Iesus.    
\P 1652 NEEDHAM tr. \textit{Selden's Mare Cl.} 21 That Vacancies are his who apprehend's them first by occupation.

\itembf{4.} To seize or embrace (an offer or opportunity).

\P 1586 T. B. \textit{La Primaud. Fr. Acad.} 750 If we apprehend not that great grace and mercy of the Father offered to all.
\P a1619 DONNE  \textit{Biathan.} (1644) 126 If he apprehend not an opportunity to escape.    
\P 1633 BP. HALL  \textit{Hard Texts} 56 His faith, whereby he did firmely apprehend the‥aid of his eternal Father.

\itembf{II.} Mental.

\itembf{5.} gen. To learn, gain practical acquaintance with. Also absol. (The earliest use in Eng.; cf. Fr. apprendre.) Obs.

\P 1398 TREVISA  \textit{Barth. De P.R.} ii. ii. (1495) 28 He holdeth in mynde‥without foryetynge, all that he apprehendyth.    
\P 1531 ELYOT  \textit{Governour} (1834) 215 Thereby they provoke many men to apprehend virtue.
\P a1680 BUTLER  \textit{Rem.} (1759) I. 204 Children‥Improve their nat'ral Talents without Care, And apprehend, before they are aware.

\itembf{6.} To become or be conscious by the senses of (any external impression).

\P 1635 AUSTIN  \textit{Medit.} 60 When this Light shone in darkenesse, and our darkenesse, though it apprehended, yet it comprehended it not.    
\P 1651 HOBBES  \textit{Leviath.} iii. xxxiv. 212 That caused Agar supernaturally to apprehend a voice from heaven.    
\P 1855 BAIN  \textit{Sens. \& Int.} iii. i. §37 If I see‥two candle flames, I apprehend them as different objects.

\itembf{7.} To feel emotionally, be sensible of, feel the force of. Obs.

\P 1592 NASHE  \textit{P. Penilesse} 29 b, The‥soules of them that haue no power to apprehend such felicitie.    
\P 1605 B. JONSON  \textit{Volpone} ii. i, Dead. Lord! how deeply, sir, you apprehend it.    
\P 1670 WALTON  \textit{Lives,} That [kindness] was so gratefully apprehended by M. Hooker.

\itembf{8.} To lay hold of with the intellect: \textbf{a.} to perceive the existence of, recognize, see.

\P 1577 T. VAUTROLLIER  \textit{Luther's Ep. Gal.} 5 Who so doth not understand or apprehend this righteousness in afflictions and terrors of conscience.    
\P 1609 \textit{C. Butler's Fem. Mon.} Ad Auth. 16 There is not half that worth in Mee Which I have apprehended in a Bee.    
\P 1743 J. MORRIS  \textit{Serm.} vii. 184 We shall apprehend reason to conclude, that‥they were not so very young.    
\P 1872 BROWNING  \textit{Fifine} lxxi. 7 Each man‥avails him of what worth He apprehends in you.

\itembf{b.} to catch the meaning or idea of; to understand.

\P 1631 HEYWOOD  \textit{Lond. Jus Hon.} 279 As soone known as showne, and apprehended as read.    
\P 1755 B. MARTIN  \textit{Mag. Arts \& Sc.} i. xiii. 87 This is all so plain, that I can't but apprehend it.    
\P 1849 MACAULAY  \textit{Hist. Eng.} I. 463 The nature of the long contest between the Stuarts and their parliaments, was indeed very imperfectly apprehended by foreign statesmen.    
\P 1871 C. DAVIES  \textit{Metric Syst.} ii. 24 To apprehend distinctly the signification of a number, two things are necessary.

\itembf{c.} absol. or with subord. clause.

\P 1599 SHAKES.  \textit{Much Ado} ii. i. 84 Cousin, you apprehend passing shrewdly.    
\P 1660 STANLEY  \textit{Hist. Philos.} 46/1 Periander‥immediately apprehended that he advised him to put the most eminent in the City to death.    
\P 1712 STEELE  \textit{Spect.} No. 532 \cardo{⁋}2, I cannot apprehend where lyes the trifling in all this.    
\P 1785 REID  \textit{Intell. Powers} i. i, No one can explain by a Logical Definition what it is to think, to apprehend.

\itembf{9.} To understand (a thing to be so and so); to conceive, consider, view, take (it) as.

\P 1639 FULLER  \textit{Holy War} iv. ix. (1840) 193 They apprehended it a great courtesy done unto them.    
\P 1736 WESLEY \textit{Wks.} 1830 I. 100, I apprehended myself to be near death.    
\P 1858 GLADSTONE  \textit{Homer} III. 393 The eternal laws, such as the heroic age apprehended them.

\itembf{b.} absol. or with subord. clause.

\P 1614 B. JONSON  \textit{Barth. Fair} i. iv. 8 If hee apprehend you flout him once, he will flie at you.    
\P 1775 J. LYON  in \textit{Sparks Corr. Amer. Rev.} (1853) I. 101, I apprehend that secrecy is as necessary now as ever it was.    
\P 1839 HALLAM  \textit{Hist. Lit.} iv. vi. §17 In general, I apprehend, the later French critics have given the preference to Racine.

\itembf{10.} To anticipate, look forward to, expect (mostly things adverse).

\P 1603 SHAKES.  \textit{Meas. for M.} iv. ii. 149 A man that apprehends death no more dreadfully, but as a drunken sleepe.    
\P 1749 FIELDING  \textit{Tom Jones} (1836) I. iii. iii. 100 A triumphant question, to which he had apprehended no answer.    
\P 1879 TOURGEE  \textit{Fool's Errand} ii. 11 Love had taught her with unerring accuracy to apprehend the evil which impended.

\itembf{11.} To anticipate with fear or dread; to be fearful concerning; to fear. a.II.11.a with obj.

\P 1606 SHAKES.  \textit{Tr. \& Cr.} iii. ii. 80 Oh let my Lady apprehend no feare.    
\P 1643 SIR T. BROWNE  \textit{Relig. Med.} i. §54 Which makes me much apprehend the ends of those honest Worthies.    
\P 1702 \textit{Eng. Theophr.} 53 He apprehends every breath of air as much as if it were a Hurricane.    
\P 1832 H. MARTINEAU  \textit{Hill \& Valley} xiii. 125 No one‥could think‥that any further violence was to be apprehended.

\itembf{b.} with subord. clause. To be apprehensive, to fear.

\P 1868 HAWTHORNE  \textit{Our Old Home} (1879) 186, I sometimes apprehend that our institutions may perish.
\end{myenumerate}


%%%%%%%%%%%%%%%%%%%%%%%%%%%%%%%%%
\myitem{appropriate} ppl. a. and n.

\noindent \phonetic{(əˈprəʊprɪət)}

\noindent [ad. L. appropriāt-us pa. pple. of appropriā-re: see appropre.]
\vspace{-0.3cm}

\begin{myenumerate}

\itembf{A.} pple. or adj.

\itembf{1.} Annexed or attached (to), as a possession or piece of property; appropriated. spec. in Eccl. Annexed as a benefice to a religious corporation.

\P 1599 SANDYS  \textit{Europ. Spec.} (1637) 145 The Parish Priests in Italy‥have‥certeine Farmes as Gleabland appropriate.    
\P 1652 NEEDHAM  tr. \textit{Selden's Mare Cl.} Pref., The Sea's now made appropriate, And yields to all the Laws of state.    
\P 1751 CHAMBERS  \textit{Cycl.} s.v., There are computed to be in England 3845 churches appropriate and impropriate.

\itembf{2.} Belonging to oneself; private; selfish. Obs.

\P 1627 FELTHAM  \textit{Resolves} i. lxxxiii. Wks. 1677, 127 Policy‥works ever for appropriate ends; Love euer takes a partner into the Benefit.

\itembf{3.} Assigned to a particular person; special, individual. Obs. rare.

\P 1796 F. BURNEY  \textit{Camilla} viii. x, The end, therefore, of her deliberation was to show general gaiety, without appropriate favour.

\itembf{4.} Attached or belonging as an attribute, quality or right; peculiar to, own. a.A.4.a absol.

\P 1538 STARKEY  \textit{England} ii. i. §25. 162 We notyd‥in‥the hede, an appropryat dysease, wych we callyd then a frencey.    
\P 1794 SULLIVAN  \textit{View Nat.} I. 174 That the sun darts out light and heat to the limits of its appropriate system.    
\P 1809 COLERIDGE  \textit{Friend} (1837) I. i. 9 To charm away‥Ennui, is the chief and appropriate business of the poet.

\itembf{b.} with \textit{to}.

\P 1525 TINDALE  \textit{Par. Wicked Mamm.} Wks. I. 50 The forgiveness of sins and justifying is appropriate unto faith only.    
\P 1651 HOBBES  \textit{Leviath.} ii. xxx. 177 Honour, appropriate to the Soveraign onely.    
\P 1812 SOUTHEY  \textit{Lett.} (1856) II. 307 Coronet‥is [a word] appropriate to rank and heraldry.

\itembf{5.} Specially fitted or suitable, proper. Const. to, for.

\P 1546 PHAËR  \textit{Regim. Lyfe} B j, Remedies‥appropriat to every membre throughout the body.    
\P 1594 PLAT  \textit{Sorts of Soyle} 56 Salts‥most appropriate for the nature of mortar.    
\P 1661 BOYLE  \textit{Style H. Script.} Wks. 1744 II. 101/2 The Bible's being appropriate‥to make wise the simple.    
\P 1809 COLERIDGE  \textit{Friend} (1865) 29 Two mottos equally appropriate.    
\P 1869 FREEMAN  \textit{Norm. Conq.} III. xi. 47 Prayers and collects appropriate for the great solemnity.

\itembf{B.} n. [the adj. used absol.] A thing appropriated or appropriate; a property, attribute. Obs.

\P 1618 CHAPMAN  \textit{Hesiod} ii. 551 To prophane The Gods' Appropriates.    
\P 1642 JER. TAYLOR  \textit{Episc.} (1647) 102 The appropriates of their office so ordain'd by the Apostles.
\end{myenumerate}


%%%%%%%%%%%%%%%%%%%%%%%%%%%%%%%%%
\myitem{apt} a.

\noindent \phonetic{(æpt)}

\noindent [ad. L. apt-us fitted, suited, appropriate, pa. pple. of *ap-ĕre to fasten, attach.]

\noindent Const. to, for, or inf.
\vspace{-0.3cm}

\begin{myenumerate}
\itembf{1.} Fitted (materially), fitting. rare.

\P 1791 COWPER  \textit{Iliad} iii. 393 His brother's corslet‥apt to his own shape and size.

\itembf{2.} Suited, fitted, adapted (to (obs.) or for a purpose); having the requisite qualifications; fit. a.2.a of things. arch.

\P 1398 TREVISA  \textit{Barth. De P.R.} xvii. clvii. (1495) 707 Stoble is apt to many dyuerse vses.    
\P 1432–50 tr. \textit{Higden Rolls Ser.} I. 163 Thei toke places apte to make cites.    
\P 1526 TINDALE  \textit{N.T.} Addr., To make it more apte for the weake stomakes.    
\P 1625 BACON  \textit{Ess.} (Arb.) 471 States‥apt to be the Foundations of Great Monarchies.    
\P 1677 MOXON  \textit{Mech. Exerc.} (1703) 181 The Workman chuses such sizes as are aptest for his Work.    
\P 1858 CARLYLE  \textit{Fredk. Gt.} I. ii. ii. 54 Tracts of Preussen are‥frugiferous, apt for the plough.

\itembf{b.} of persons: Fit, prepared, ready. arch.

\P 1474 CAXTON  \textit{Chesse} 27 Whiche of hem‥was most apte for to sende to gouerne and juge the contre of spayn.    
\P 1526 TINDALE  \textit{Luke} ix. 62 No man that‥loketh backe is apte to the kyngdom of God.    
\P 1601 SHAKES.  \textit{Jul. C.} iii. i. 160 Liue a thousand yeeres, I shall not finde my selfe so apt to dye.
\P a1700 MRS. HUTCHINSON  \textit{Mem. Hutchinson} 22 He was apt for any bodily exercise.    
\P 1870 MORRIS  \textit{Earthly Par.} I. i. 20 Tall was he, slim, made apt for feats of war.

\itembf{3.a.} ellipt. Suited to its purpose; suitable, becoming, appropriate.

\P 1563 \textit{Myrr. Mag., Blacksmith} xix, The Plowman fyrst his land doth dresse and torne And makes it apte.    
\P 1597 MORLEY  \textit{Introd. Mus.} Annot., [Musicke is] a disposition of proportionable soundes deuided by apt distances.    
\P 1630 DEKKER  \textit{Honest Wh.} ii. Wks. 1873 II. 99 Pray the good woman take some apter time.    
\P 1710 STEELE  \textit{Tatler} No. 8 \cardo{⁋}1 Recommending the apt Use of a Theatre as the most agreeable‥Method of making a‥moral Gentry.    
\P 1807 WORDSW.  \textit{Resol. \& Indep.} xvi, To give me human strength, by apt admonishment.

\itembf{b.} esp. of language: Suitable or appropriate to express ideas; apposite, expressive.

\P 1590 SHAKES.  \textit{Mids. N.} v. i. 65 In all the play There is not one word apt.    
\P 1688 LD. DELAMERE  \textit{Wks.} 20 Apt words and quaint Phrases are very good adornments of Speech.    
\P 1865 MILL  \textit{Liberty} v. 57/1 What in the apt language of Bentham is called pre-appointed evidence.

\itembf{c.} of thoughts, remarks, etc. Appropriate to the occasion, apposite.

\P 1844 DISRAELI  \textit{Coningsby} v. vii. 216 The prompt reply or the apt retort.    
\P 1849 W. IRVING  \textit{Mahom. \& Succ.} xiv. (1853) 63 The smoke was an apt thought, and saved his camp from being sacked.    
\P 1877 SPARROW  \textit{Serm.} xxi. 284 The apt reply of the little Sunday-school scholar, who, when asked what eternity was, replied, ‘The life-time of God.’

\itembf{4.} Having a habitual tendency or predisposition (to do something).

\P 1570 LEVINS  \textit{Manip.} (1867) 28 Apte, aptus, idoneus‥is also the signe of verballes in -bilis, and participials in -dus: Apt to be taught, docilis; Apt to be red, legibilis.

\itembf{a.} of things: Calculated, likely; habitually liable, ready.

\P 1528 MORE  \textit{Heresyes} iv. Wks. 248/2 Yet be such workes‥apte to corrupt and infect the reder.    
\P 1678 BUTLER  \textit{Hud.} iii. i. 1048 For fat is wondrous apt to burn.    
\P 1784 COWPER  \textit{Lett.} Feb. 29 Wks. 1876, 161 Nothing is so apt to betray us into absurdity as too great a dread of it.    
\P 1868 FREEMAN  \textit{Norm. Conq.} II. vii. 12 Any kind of taxation is apt to be looked on as a grievance.

\itembf{b.} of persons: Customarily disposed, given, inclined, prone.

\P c1550 LUSTY  \textit{Juv.} in Hazl. \textit{Dodsley} II. 53 That I may be apt thy holy precepts to fulfil.    
\P 1592 SHAKES.  \textit{Rom. \& Jul.} iii. i. 34 So apt to quarell.    
\P 1718 POPE  \textit{Iliad} xxiv. 530 For apt is youth to err.    
\P 1771 FRANKLIN \textit{Autobiog.} Wks. 1840 I. 85, I perceive I am apt to speak in the singular number.    
\P 1857 RUSKIN  \textit{Pol. Econ. Art} 26 We are apt to act too immediately on our impulses.

\itembf{c.} Inclined, disposed (in a single instance).

\P 1677 R. CARY  \textit{Palæol. Chron.} ii. ii. i. iv. 195, I am apt to think, that‥Vashti is meant.    
\P 1706 HEARNE  \textit{Rem. \& Collect.} (1885) I. 297, I am apt to think he has not consulted Books enough upon this occasion.    
\P 1899 E. E. HALE  \textit{Lowell} 126, I am apt to think that this modest man was the first person‥to recognize [etc.].

\itembf{5.} Susceptible to impressions; ready to learn; intelligent, quick-witted, prompt. Mod. const. at.

\P 1535 COVERDALE  \textit{Ecclus.} xxxvii. 22 Some man is apte and well instructe in many thinges.    
\P 1601 SHAKES.  \textit{Jul. C.} v. iii. 68 O hatefull error‥Why do'st thou shew to the apt thoughts of men The things that are not.    
\P 1660 PEPYS  \textit{Diary} 28 Aug., Beginning to teach my wife some scale in musique, and found her apt beyond imagination.    
\P 1719 DE FOE  \textit{Crusoe} (1858) 220 He was the aptest scholar that ever was.    
\P 1832 H. MARTINEAU  \textit{Life in Wilds} vi. 77 Men‥are‥apt at devising ways of easing their toils.

\vspace{0.2cm}
\noindent
Quasi-adv., as in apt-deceiving, apt-divided.

\P 1597 DANIEL  \textit{Civ. Wars} i. lxx, Intestine strife‥The apt-divided state entangle would.    Ibid. (1717) 213 Such apt-deceiving Clemency And seeming Order.
\end{myenumerate}


%%%%%%%%%%%%%%%%%%%%%%%%%%%%%%%%%
\myitem{arbiter} n.

\noindent \phonetic{(ˈɑːbɪtə(r))}

\noindent [a. L. arbiter (? f. ar- = ad- to + bētĕre, bītĕre, to go, ‘one who goes to see,’ hence, who looks into or examines) a judge in equity, a supreme ruler. Cf. arbitrator, arbitrer. Arbiter was the orig. L. word, still extant in F. as arbitre; arbitrātor was a later L. n. of agent from arbitrāri to act as arbiter; of this the OF. descendant was arbitreor, -our, by the side of which arbitrateur, -our, was also adopted as a technical term by the jurists. In Eng., arbitrour seems to have been the earliest, then arbitratour, and in 16th c. arbiter from L., though arbitre may well have existed in ME. (The 16th c. spelling arbitour, -or, was, as in ancestor, merely imitative of words properly in -our.)]
\vspace{-0.3cm}

\begin{myenumerate}

\itembf{1.} gen. One whose opinion or decision is authoritative in a matter of debate; a judge.

\P 1502 ARNOLD  \textit{Chron.} (1811) 160 Abdalazys‥most iust arbiter and juge of trouth.    
\P 1601 HOLLAND  \textit{Pliny} II. 151 As a deputed judge or arbiter delegat to determin of mans health, and the preseruation thereof.    
\P 1790 COWPER  \textit{Odyss.} viii. 314 Nine arbiters appointed to intend The whole arrangement of the public games.    
\P 1824 DIBDIN  \textit{Libr. Comp.} 520 The late Mr. Fox (no mean arbiter in literary taste).

\itembf{2.a.} spec. One who is chosen by the two parties in a dispute to arrange or decide the difference between them; an arbitrator, an umpire. (See note to ARBITRATOR 1.)

\P 1549 HOOPER  \textit{Ten Commandm.} x. Wks. 1843–52, 348 To solicitate the same by honest arbiters and godly friends.    
\P 1609 SKENE  \textit{Reg. Maj.} 20 Ane Judge haueand ane ordinar jurisdiction, may nocht be ane Arbitour.    
\P 1754 ERSKINE  \textit{Princ. Sc. Law} (1809) 492 The power of arbiters is wholly derived from the consent of parties.    
\P 1852 GLADSTONE  \textit{Gleanings} IV. xiv. 150 Beyond the Atlantic‥things civil and things spiritual move in their separate spheres, without any need for an arbiter between them.    
\P 1873 DIXON  \textit{Two Queens} I. iv. i. 179 Appointed arbiter of the dispute.

\itembf{b.} transf. or fig.

\P a1568 COVERDALE  \textit{Hope of Faithf.} xii. (1574) 83 Christ‥the arbiter and mediator betwene God and men.    
\P 1580 SIDNEY  \textit{Arcadia,} The sun [at the equinox]‥indifferent arbiter between the night and the day.    
\P 1667 MILTON  \textit{P.L.} ix. 50 Twilight‥short Arbiter 'Twixt Day and Night.

\itembf{3.} One who has power to decide or ordain according to his own absolute pleasure; one who has a matter under his sole control. Also fig.

\P 1628 SIR R. LE GRYS tr. \textit{Barclay's Argenis} 286 Thou sittest as it were the arbiter of the fortune of thy neighbour Kings.    
\P 1652 NEEDHAM  \textit{Selden's Mare Cl.} 19 Absolute Lord or Arbiter of the whole world.    
\P 1785 REID  \textit{Int. Powers} i. i. §11 Use‥which is the arbiter of language.    
\P 1814 BYRON  \textit{To Napoleon,} The arbiter of others' fate, A suppliant for his own.    
\P 1874 MOTLEY  \textit{Barneveld} I. i. 61 The proud‥position of arbiter of Europe.

\itembf{4.} arbiter elegantiarum, arbiter elegantiæ [L., lit. ‘judge of elegance’: Petronius Arbiter was the elegantiæ arbiter of Nero's court (Tacitus Ann. xvi. 18)], a judge of matters of taste, an authority on etiquette.

\P 1818 LADY  MORGAN \textit{Fl. Macarthy} II. iii. 175 He looked up to Lord Frederick Eversham, as the arbiter elegantiarum of that system.    
\P 1841 CRAIK \& MACFARLANE  \textit{Pict. Hist. Eng. Geo.} III I. 651/1 Derrick‥succeeded Nash as arbiter elegantiarum at Bath.    
\P 1933 BALMER \& WULIE \textit{When Worlds Collide} v. 49 A connoisseur of life and living—an arbiter elegantiae.    
\P 1957 R. N. C. HUNT \textit{Guide Communist Jargon} xxi. 76 Zhdanov was appointed Stalin's arbiter elegantiarum in the late 'forties.
\end{myenumerate}


%%%%%%%%%%%%%%%%%%%%%%%%%%%%%%%%%
\myitem{arcane} a.

\noindent \phonetic{(ɑːˈkeɪn)}

\noindent [ad. L. arcānus, f. arcē-re to shut up, arca chest; cf. F. arcane.]

\noindent
Hidden, concealed, secret.

\P 1547 BOORDE  \textit{Brev. Health} Pref. 2 The eximiouse and Archane science of physicke.    
\P 1595 \textit{Locrine}  v. iv. 187 Have I bewrayed thy arcane secrecy?    
\P 1678 CUDWORTH  \textit{Intell. Syst.} Pref., To Reveal the Arcane Mysteries of Atheism.    
\P 1876 E. GOSSE in \textit{Academy} 9 Dec. 557 Walking in the arcane world of wonder.


%%%%%%%%%%%%%%%%%%%%%%%%%%%%%%%%%
\myitem{arch} a.

\noindent \phonetic{(ɑːtʃ)}

\noindent [ARCH- prefix used as a separate word: see next.]
\vspace{-0.3cm}

\begin{myenumerate}

\itembf{A.} adj.

\itembf{1.} Chief, principal, prime, pre-eminent. (Now rarely used without the hyphen.)

\P 1547 \textit{Life Abp. Canterb.} Pref. D viij b, The fauour off any thoughe neuer so arch a Prelate.    
\P 1594 SHAKES.  \textit{Rich. III,} iv. iii. 2 The most arch deed of pittious massacre.    
\P 1613 \textit{Hen. VIII,} iii. ii. 102 An Heretique, an Arch-one.    
\P 1647 WARD  \textit{Simp. Cobler} 88 We cannot helpe it though we can, which is the Arch infirmity in all morality.    
\P 1649 PRYNNE  \textit{Vind. Lib. Eng.} 45 And proclaim them the Archest Impostors under Heaven.    
\P 1678 [SEE 2].    
\P 1834 LYTTON  \textit{Pompeii} (1877) 231 Thou mayest have need of thy archest magic to protect thyself.

\itembf{2.} [Arising from prec. sense, in connexion with wag, knave, rogue, hence with fellow, face, look, reply, etc.] Clever, cunning, crafty, roguish, waggish. Now usually of women and children, and esp. of their facial expression: Slily saucy, pleasantly mischievous.

\P 1662 MORE  \textit{Antid. Ath.} i. viii. (1712) 151 That arch wag‥ridiculed that solid argument.    
\P 1678 BUNYAN  \textit{Pilgr.} ii. 147 Greath. Above all that Christian met‥By-ends was the arch one. Hon. By-ends; What was he? Greath. A very arch Fellow, a downright Hypocrite.    
\P 1710 \textit{Tatler}  No. 193 \cardo{⁋}1 So arch a leer.    
\P 1775 WESLEY  \textit{Wks.} (1872) IV. 41 Some arch boys gave him such a mouthful of dirt.    
\P 1810 CRABBE  \textit{Borough} xv, Arch was her look and she had pleasant ways.    
\P 1872 BLACK  \textit{Adv. Phaeton} xxiii. 324 Her arch ways, and her frank bearing.    
\P 1877 M. ARNOLD  \textit{Poems} I. 27 The archest chin Mockery ever ambush'd in.

\itembf{b.} Const. at, upon. Obs.

\P 1670 EACHARD  \textit{Contempt Clergy,} Lads that are arch knaves at the nominative case.    
\P 1712 STEELE  \textit{Spect.} No. 432 \cardo{⁋}5 A Templar, who was very arch upon Parsons.    
\P 1741 RICHARDSON  \textit{Pamela} (1824) I. 135 ‘Sir Simon‥you are very arch upon us.’

\itembf{B.} absol. quasi-n. A chief (one). Obs.

\P 1605 HEYWOOD  \textit{If you know not} Wks. (1874) 239 Poole that Arch, for truth and honesty.    
\P 1605 SHAKES.  \textit{Lear} ii. i. 61 The Noble Duke my Master, My worthy Arch and Patron.
\end{myenumerate}


%%%%%%%%%%%%%%%%%%%%%%%%%%%%%%%%%
\myitem{arduous} a.

\noindent \phonetic{(ˈɑːdjuːəs)}

\noindent [f. L. ardu-us high, steep, difficult + -ous.]
\vspace{-0.3cm}

\begin{myenumerate}

\itembf{1.} Lofty, high, steep, difficult to climb; also fig.

\P 1709 POPE  \textit{Ess. Crit.} 95 Those arduous paths they trod.    
\P 1713 STEELE  \textit{Guard.} No. 20 \cardo{⁋}1 To forgive is the most arduous pitch human nature can arrive at.    
\P 1831 MACAULAY  \textit{Boswell, Ess.} (1854) I. 174/2 Knowledge at which Sir I. Newton arrived through arduous and circuitous paths.

\itembf{2.} Hard to accomplish or achieve; requiring strong effort; difficult, laborious, severe.

\P 1538 STARKEY  \textit{England} 27 A mater‥of grete dyffyculty and harduos.    
\P 1718 POPE  \textit{Iliad} xiv. 523 An arduous battle rose around the dead.    
\P 1775 HARRIS  \textit{Philos. Arrangem.} (1841) 259 A task too arduous for unassisted philosophy.    
\P 1849 MACAULAY  \textit{Hist. Eng.} I. 206 Such an enterprise would be in the highest degree arduous and hazardous.

\itembf{3.} By transference to the activity required for the task: Strenuous, energetic, laborious.

\P 1753 [SEE ARDUOUSLY].    
\P 1860 TYNDALL  \textit{Glac.} i. §22. 160 Less than two good ones [guides]‥an arduous climber ought not to have.    
\P 1873 BURTON  \textit{Hist. Scot.} VI. lxxiii. 376 Montrose made arduous efforts to reconstruct his army.
\end{myenumerate}


%%%%%%%%%%%%%%%%%%%%%%%%%%%%%%%%%
\myitem{argot} n.

\noindent \phonetic{(argo)}

\noindent [Fr. Of unknown origin.]

\noindent
The jargon, slang, or peculiar phraseology of a class, orig. that of thieves and rogues.

\P 1860 FARRAR  \textit{Orig. Lang.} vi. 134 Leaves an uninviting argot in the place of warm and glowing speech.    
\P 1869 \textit{Fam. Speech} ii. (1873) 78 The argots of nearly every nation.


%%%%%%%%%%%%%%%%%%%%%%%%%%%%%%%%%
\myitem{arid} a.

\noindent \phonetic{(ˈærɪd)}

\noindent [ad. L. ārid-us, f. ārē-re to be dry, parched with heat. Perh. directly from F. aride, 15th c. refashioning of OF. are, arre.]
\vspace{-0.3cm}

\begin{myenumerate}

\itembf{1.} Dry, without moisture, parched, withered. †a.1.a of substances: Dry; anhydrous. Obs.

\P 1652 L. S. \textit{People's Lbty.} ix. 17 Aride and liquide fruicts.    
\P 1742 SHENSTONE  \textit{Schoolmistr.} 106 Lavender‥in arid bundles bound.    
\P 1803 \textit{Phil. Trans.} XCIII. 14 Arid white salt‥Arid, may be appropriated to express the state of being devoid of combined water.

\itembf{b.} Med. of the skin. Obs.

\P 1704 SWIFT  \textit{Batt. Bks.} (1711) 248 Her Body grew white and arid.    
\P 1727 ARBUTHNOT \& POPE (J.) My complexion is become adust, and my body arid.

\itembf{c.} of the ground or climate. Hence, barren, bare.

\P 1656 BLOUNT  \textit{Glossogr.,} Arid, dry, barren, withered, unfruitful.    
\P 1730 THOMSON  \textit{Autumn} 147 Without him summer were an arid waste.    
\P 1849 DICKENS  \textit{Barn. Rudge} (1866) I. lviii. 265 The dry, arid look of the dusty square.    
\P 1872 BAKER  \textit{Nile Tribut.} Pref. 7 Arid sands and burning deserts.

\itembf{2.} fig. Dry, uninteresting, barren, jejune.

\P 1827–39 DE QUINCEY \textit{Murder} Wks. IV. 26 An old arid and adust metaphysician.
\P 1846 LYTTON  \textit{Lucretia} (185
\P 1863 GEO. ELIOT \textit{Romola} lxxi, Arid of all good.
\end{myenumerate}


%%%%%%%%%%%%%%%%%%%%%%%%%%%%%%%%%
\myitem{Armageddon} n.

\noindent \phonetic{(ɑːməˈgɛdən)}

\noindent [See Rev. xvi. 16 (A.V.)]
\vspace{-0.3cm}

\noindent
The place of the last decisive battle at the Day of Judgement; hence used allusively for any ‘final’ conflict on a great scale. Also attrib.

\P 1811 SHELLEY  \textit{Let.} 12 Jan. (1964) I. 45 Do we not now see Superstition decaying‥except where Faber‥and several others of the Armageddon-Heroes maintain their posts.    
\P 1886 SUFFOLK \& CRAVEN  \textit{Racing} 247 As long as we have racing we shall have betting—that ceaseless war between layers and backers will still be waged.‥ At present we see no sign of a final Armageddon.    
\P 1896 KIPLING  \textit{England's Answer} in \textit{Poems} (1919) I. 237 In the day of Armageddon, or the last great fight of all.    
\P 1910 \textit{Encycl. Brit.} II. 561/1 From the application of the word Armageddon to the great battle of the End of Time comes the use of the phrase ‘an Armageddon’ to express any great slaughter or final conflict.    
\P 1917 F. M. FORD  \textit{Let.} 5 Jan. (1965) 83, I am sure you could not have done a better ‘bit’ during Armageddon.    
\P 1928 W. DEEPING  \textit{Old Pybus} ii. §2 Mr. Pybus had been able to speak of the war as Armageddon without cribbing an obvious bleat from the popular press.


%%%%%%%%%%%%%%%%%%%%%%%%%%%%%%%%%
\myitem{arrant} a.

\noindent \phonetic{(ˈærənt)}

\noindent [A variant of errant, ‘wandering, vagrant, vagabond,’ which from its frequent use in such expressions as arrant thief, became an intensive, ‘thorough, notorious, downright,’ especially, from its original associations, with opprobrious names. For the vowel-change cf. arrand= errand, Harry=Herry, Henry, far=earlier fer, etc.]
\vspace{-0.3cm}

\begin{myenumerate}

\itembf{1.} Wandering, itinerant, vagrant; esp. in knight arrant, bailiff arrant; in which the etymological errant is now alone used.

[\P c1400 \textit{Circumcis.} (Turnb. 1843) 97 To bryng the lost schepe ageyn‥That was errawnt, ydyl, and in vayne.]    
\P 1550 CROWLEY  \textit{Epigr.} (1872) 12 Title, Of Baylife Arrantes.    
\P 1557 \textit{K. Arthur} (Copland) vii. x, With that knyght wyll I juste, for I see that he is a knyght arraunt.    
\P 1602 WARNER  \textit{Alb. Eng.} ix. xlvi. 217 Arrant Preachers, humming out a common-place or two.    
\P 1647 HOWARD  \textit{Crown Rev.} 18 Bayliffe arrant. Fee.—4l. 11s. 3d.    
[\P 1691 BLOUNT  \textit{Law Dict.}, Bailiffs Errant are those whom the Sheriff appoints to go up and down the County to serve Writs, etc.]

\itembf{2.} In thief errant, arrant thief [= robber]: orig. an outlawed robber roving about the country, a freebooter, bandit, highwayman; hence, a public, notorious, professed robber, a ‘common thief,’ an undisguised, manifest, out-and-out thief.

\P 1386 CHAUCER  \textit{Manciple's T.} 120 An outlawe or a thef erraunt. [See the whole passage.]    
\P 1553 BALE  \textit{Vocacyon} in \textit{Harl. Misc.} (Malh.) I. 362 The most errande thefe and mercilesse murtherer.    
\P 1563 GRAFTON  \textit{Chron. Hen. IV,} an. 1 (R.) There is not so ranke a traytor, nor so arrant a thefe.    
\P 1637 J. POCKLINGTON  \textit{Sund. no Sabb.} 13 The arrantest Pharisee theefe in Jerusalem.
\P a1744 SWIFT \textit{Wks.} 1841 II. 79 Every servant an arrant thief as to victuals and drink.    
\P 1822 W. IRVING  \textit{Braceb. Hall} xxvii. 247 Who, like errant thieves, could not hold up their heads in an honest house.

\itembf{3.} Hence: Notorious, manifest, downright, thorough-paced, unmitigated. Extended
from thief to traitor, knave, rebel, coward, usurer; after 
1575 widely used as an opprobrious intensive, with fool, dunce, ass, idiot, hypocrite, Pharisee, Papist, Puritan, infidel, atheist, blasphemer, and so on through the whole vocabulary of abuse.

\P 1393 LANGL.  \textit{P. Pl.} C. vii. 307 An erraunt vsurer.    
\P 1494 FABYAN  v. lxxx. 58 Beyng a errant Traytoure.    
\P 1538 TUNSTALL in Strype \textit{Eccl. Mem.} I. i. xliv. 338 Reginald Pole, comen of a noble blood, and thereby the more errant traitor.    
\P 1553 \textit{Puocl.} ibid. III. App. vi. 10 The most arrande traytour Syr John Dudley.
\P c1588 GREENE  \textit{Fr. Bacon} v. 26 Why, thou arrant dunce, shall I never make thee a good scholar?    
\P 1596 DRAYTON  \textit{Legends} i. 112 Which she to Sots and arrant Ideots threw.    
\P 1602 SHAKES.  \textit{Ham.} i. v. 124 Hee's an arrant knaue.    
\P 1621 BURTON  \textit{Anat. Mel.} ii. iii. ii. (1651) 316 A nobleman therefore in some likelihood‥is‥a proud fool, an arrant asse.    
\P 1660 H. MORE  \textit{Myst. Godl.} v. xiii. 168 Either an arrant Infidel or horrid Blasphemer.    
\P 1679 MANSELL  \textit{Narr. Popish Plot} Addr., Who may prove good tools, though errant Fools.    
\P 1719 DE FOE  \textit{Crusoe} 482 They are errant cowards.    
\P 1749 FIELDING  \textit{Tom Jones} xiv. iii. (1840) 205 The arrantest villain that ever walked upon two legs.    
\P 1824 W. IRVING  \textit{T. Trav.} II. 34 As arrant a crew of scapegraces as ever were collected together.    
\P 1837 HOWITT  \textit{Rur. Life} ii. v. (1862) 141 The inhabitants of solitary houses are often most arrant cowards.

\itembf{b.} transf. of things, i.e. opprobrious deeds and qualities, theft, presumption, lie, device, etc.

\P 1639 FULLER  \textit{Holy War} v. xxx (1840) 301 It were arrant presumption for flesh to prescribe God his way.    
\P 1692 BENTLEY  \textit{Boyle Lect.} i. 9 They cover the most arrant Atheism under the mask and shadow of a Deity.    
\P 1753 RICHARDSON  \textit{Grandison} (1781) IV. xxxiv. 241, I am afraid I have written arrant nonsense.    
\P 1776 PENNANT  \textit{Tour Scot.} ii. 327 This hill, till about the year 995, was an errant desert‥and uninhabitable.    
\P 1858 BUCKLE  \textit{Civilis.} (1869) III. v. 480 Little better than arrant trifling.

\itembf{4.} Without opprobrious force: Thorough, downright, genuine, complete, ‘regular.’

\P 1570 LEVINS  \textit{Manip.} 25 Arrant, grandis, magnus.    
\P 1575 TURBERV.  \textit{Venerie} 193 Good and arrant Terriers‥to make the foxe or Badgerd start the soner.    
\P 1664 EVELYN  \textit{Sylva} 95 He that shall behold its grain‥will never scruple to pronounce it arrant wood.    
\P 1704 ROWE  \textit{Ulysses} Epil. 15 They Like arrant Huswives, rise by Break of Day.    
\P 1820 W. IRVING  \textit{Sketch Bk.} II. 59 A tight brisk little man, with the air of an arrant old bachelor.

\itembf{5.} With the opprobrious force transferred to the adj.: Thoroughly bad, good for nothing, rascally.

\P 1581 B. RICH  \textit{Farewell} (1846) 25 Her beautie had so entangled her arrant hoste.    
\P 1592 G. HARVEY  \textit{Pierce's Superer.} 6 So forward to accuse, debase, revile‥as the arrantest fellow in a Country?    
\P 1676 WYCHERLEY  \textit{Plain-Dealer} iii. i, Mine's as arrant a Widow-Mother, to her poor Child, as any's in England.    
\P 1708 POPE \textit{Lett.} Wks. 1736 V. 61 You are not so arrant a critic of the modern Poets as‥to damn them without a hearing.    
\P 1761 SMOLLETT  \textit{Gil Blas} vii. iii, It was easy to see through all his piety that he was an arrant author at the bottom.

\itembf{b.} as pred.

\P 1641 MILTON  \textit{Animadv. Def. Smectymn.} ii, The authority of some synodal canons which are now arrant to us.

\itembf{6.} as n. A person of no reputation, a good-for-nothing.

\P 1605 BRETON  \textit{Be not angry} 8 Her good-man who should be sent of errands, while she were with her arrants.
\end{myenumerate}


%%%%%%%%%%%%%%%%%%%%%%%%%%%%%%%%%
\myitem{arrogate} v.

\noindent \phonetic{(ˈærəʊgeɪt)}

\noindent [f. L. arrogāt- ppl. stem of adr-, arrogā-re to ask or claim for oneself, to adopt one whose consent may legally be asked, f. ad- to + rogāre to ask. Modern writers on Roman Law have appropriated the form adrogate to the specific legal sense.]
\vspace{-0.3cm}

\begin{myenumerate}

\itembf{1.} Rom. Law. To adopt as a child. (See adrogate.)

\P 1649 JER. TAYLOR  \textit{Gt. Exemp.} iii. §15. 89 He did arrogate John‥into Maries kindred.

\itembf{b.} transf. To adopt (that which is proper to another). Obs.

\P 1530 \textit{Epit. Barnes Wks.} 371 (R.) The Byshops‥doe arrogate vnto themselues some thyng of the Phariseis pride.

\itembf{2.} To claim and assume as a right that to which one is not entitled; to lay claim to and appropriate (a privilege, advantage, etc.) without just reason or through self-conceit, insolence, or haughtiness. a.2.a with to and refl. pron.

\P 1537 LATIMER  \textit{Serm.} (1844) 43 How much soever we arrogate these holy titles unto us.    
\P 1671 MILTON  \textit{P.R.} iv. 315 To themselves all glory arrogate, to God give none.    
\P 1777 WATSON  \textit{Philip II} (1793) II. xiii. ii. 154 The Spaniards‥had arrogated to themselves every important branch of the administration.    
\P 1844 BROUGHAM  \textit{Brit. Const.} ix. §2 (1862) 120 They arrogated to themselves the right of approving or rejecting all that was done.

\itembf{b.} with simple obj. only.

\P 1593 BILSON  \textit{Govt. Christ's Ch.} 18 Yet may they not arrogate any parte of Christes honour.    
\P 1667 MILTON  \textit{P.L.} xii. 26 Will arrogate Dominion undeserv'd Over his brethren.    
\P 1702 ROWE  \textit{Tamerlane} i. ii. 575 And arrogate a Praise that is not ours.    
\P 1858 DORAN  \textit{Crt. Fools} 92 The liberty arrogated by the professor of wit.

\itembf{3.} To lay claim, without reason or through self-conceit, to the possession of (some excellence); to assert without foundation that one has; to assume. a.3.a with to and refl. pron.

\P 1563 \textit{Homilies}  ii. xvi. ii. (1859) 461 Whether all men doe justly arrogate to themselves the Holy Ghost, or no?
\P a1638 MEDE  \textit{Wks.} iv. xii. 757 Nor do I arrogate so much ability to myself.    
\P 1789 BELSHAM  \textit{Ess.} II. xl. 501 They arrogate‥all wisdom, knowledge, and even honesty, to themselves.    
\P 1872 BLACK  \textit{Adv. Phaeton} xxix. 384 She arrogated to herself a certain importance.

\itembf{b.} with simple obj. only.

\P 1598 R. BARCKLEY  \textit{Felic. Man} Ded., One that arrogateth superioritie over all.    
\P 1660 STANLEY  \textit{Hist. Philos.} (1701) 428/2 Thus Pythagoras might arrogate the soul of Euphorbus.    
\P 1768 \textit{Phil. Trans.} LVIII. 149, I can arrogate no merit in the discovery.    
\P 1848 H. ROGERS  \textit{Ess.} I. vi. 321 Arrogating the exclusive possession of wisdom.

\itembf{c.} with inf. or absol. Obs.

\P 1628 WITHER  \textit{Brit. Rememb.} v. 203 Doe falsly arrogate to be inspired.    
\P 1648 C. WALKER  \textit{Relat. \& Obs.} i. 29 They arrogate to be the peculiar people of God.    
\P 1648 MILTON  \textit{Tenure Kings} 13 Surely no Christian Prince would arrogate so unreasonably above human condition.

\itembf{4.} To lay similar claim to (something) on behalf of another; to ascribe or attribute to, or demand for, without just reason.

\P 1605 TIMME  \textit{Quersit.} i. vi. 24 We deny that those inset and naturall qualities‥are to be arrogated to hotte, moist, and drie.    
\P 1810 COLERIDGE  \textit{Friend} i. iv. (1867) 12 To antiquity we arrogate many things, to ourselves nothing.    
\P 1863 COX  \textit{Inst. Eng. Govt.} i. viii. 111 An attempt was made‥to arrogate to the Crown the privilege of issuing writs.
\end{myenumerate}


%%%%%%%%%%%%%%%%%%%%%%%%%%%%%%%%%
\myitem{arsenal} n.

\noindent \phonetic{(ˈɑːsɪnəl)}

\noindent [a. It. arze- arsenale, Sp. Pg. F. arsenal, earlier forms of which are It. arzenà (Dante), arzanà (still in use), 16–17th c. F. arsena, arsenac (see Littré), all in the current sense; cf. It. and Sp. darsena, Sicilian tirzanà (Diez), Pg. taracena, tercena, F. darse, darsine, ‘a dock’; also Sp. atarazána, atarazanál, ‘arsenal, factory, wine-cellar, etc.’ The original is the Arab. 
dar as-sina'ah "workshop," literally "house of manufacture," from dar "house" +
sina'ah "art, craft, skill," from sana'a "he made.",
which is directly represented by the Romance darsena, taracena; atarazana is prob. a Sp. Arab. form with article al-, ad- prefixed; arsena is either (as Diez thinks) from darsena, with d dropped (perh. by assoc. with de, d', preposition, cf. dante, ante n.1), or (as Defréméry and others hold) from as-sina'ah 
alone. See Dozy, and Devic in Littré's Supp. The final -ale, -al was added in It. or Sp. The wider sense of the Arabic is retained in Sp.; the other languages have narrowed it to dock and armoury. The earliest forms in Eng. were from It., but the existing one is that common to Fr., Sp. and Pg.]
%\noindent [a. It. arze- arsenale, Sp. Pg. F. arsenal, earlier forms of which are It. arzenà (Dante), arzanà (still in use), 16–17th c. F. arsena, arsenac (see Littré), all in the current sense; cf. It. and Sp. darsena, Sicilian tirzanà (Diez), Pg. taracena, tercena, F. darse, darsine, ‘a dock’; also Sp. atarazána, atarazanál, ‘arsenal, factory, wine-cellar, etc.’ The original is the Arab. dār aççināﻋah, workshop, factory (i.e. dār house, place of, al the, çināﻋah, art, mechanical industry, f. çanaﻋa to make, fabricate), which is directly represented by the Romance darsena, taracena; atarazana is prob. a Sp. Arab. form with article al-, ad- prefixed; arsena is either (as Diez thinks) from darsena, with d dropped (perh. by assoc. with de, d', preposition, cf. dante, ante n.1), or (as Defréméry and others hold) from aç-çināﻋah alone. See Dozy, and Devic in Littré's Supp. The final -ale, -al was added in It. or Sp. The wider sense of the Arabic is retained in Sp.; the other languages have narrowed it to dock and armoury. The earliest forms in Eng. were from It., but the existing one is that common to Fr., Sp. and Pg.]
\vspace{-0.3cm}

\begin{myenumerate}

\itembf{1.} A dock possessing naval stores, materials, and all appliances for the reception, construction, and repair of ships; a dockyard. Obs. exc. Hist.

\P 1506 SIR R. GUYLFORDE  \textit{Pilgr.} (1851) 7 At the Archynale there be closed within‥an .C. galyes.    
\P 1549 THOMAS  \textit{Hist. Italy} (1561) 74 b, The Arsenale [at Venice] in myne eye excedeth all the rest: For there they haue well neere two hundred galeys.    
\P 1580 NORTH  \textit{Plutarch} (1676) 372 Set up an arsenal or store-house to build gallies in.    
\P 1601 HOLLAND  \textit{Pliny} I. 175 Making the Arsenall at Athens, able to receiue 1000 SHIPS.    
\P 1611 CORYAT  \textit{Crudities} 216, I was at the Arsenall which is so called quasi ars naualis, because there is exercised the Art of making tackling and all other necessary things for shipping.    
\P 1693 URQUHART  \textit{Rabelais} iii. lii, Carricks, Ships‥and other vessels of his Thalassian arsenal.    
\P 1838 ARNOLD  \textit{Hist. Rome} (1846) I. xxi. 461 Building ships, and arsenals to receive and fit them out properly.

\itembf{2.} A public establishment for the manufacture and storage, or for the storage alone, of weapons and ammunition of all kinds, for the military and naval forces of the country.

\P 1579 FENTON  \textit{Guicciard.} viii. (1599) 317 A fire kindled‥in their storre house called the Arzenale‥where was their saltpeter.    
\P 1625 BACON  \textit{Ess.} (Arb.) 473 Stored Arcenalls and Armouries.    
\P 1660 HOWELL  \textit{Let. Ital. Prov.} in \textit{Dict.}, The whole Arsenal of Venice is not able to arm a Coward.    
\P 1676 BULLOKAR,  \textit{Arcenel}, an Armoury, Storehouse of Armour or Artillery.    
\P 1727 CHAMBERS  \textit{Cycl.} s.v., The Arsenal at Paris is that where the cannon or great guns are cast.    
\P 1781 GIBBON  \textit{Decl. \& F.} II. 53 Offensive weapons of all sorts, and military engines, which were deposited in the arsenals.    
\P 1811 D. LYSONS  \textit{Environs Lond.} I. 594 The gun-wharf at Woolwich‥is now called the Arsenal, or Royal Arsenal. This Arsenal is the grand depôt of the ordnance belonging to the navy.    
\P 1876 J. THORNE  \textit{Environs Lond.} II. 742/1 The Royal Arsenal [Woolwich] stretches for a mile along the Thames E. of the Dockyard. It is the only arsenal in the kingdom; the smaller establishments at the other dockyards are called gun-wharfs, and receive their supplies from Woolwich.

\itembf{b.} fig.

\P 1598 SYLVESTER  \textit{Du Bartas} i. (1633) 24 Of changefull chances common Arcenal.    
\P 1604 T. WRIGHT  \textit{Passions Mind} v. §4. 185 Their arcinall or storehouse of persuasiue prouission.    
\P 1643 FEATLY  \textit{Pref. Newman's Concord.} 1 Scripture is‥the spirituall arsenall of munition.    
\P 1857 H. REED  \textit{Lect. Brit. Poets} ix. 300 Weapons from the arsenal of poetic satire.
\end{myenumerate}


%%%%%%%%%%%%%%%%%%%%%%%%%%%%%%%%
\myitem{artful} a.

\noindent \phonetic{(ˈɑːtfʊl)}

\noindent [f. art n. + -ful.]
\vspace{-0.3cm}

\begin{myenumerate}

\itembf{I.} Of persons or agents.

\itembf{1.} Versed in the (liberal) arts; learned, wise.

\P 1613 HEYWOOD  \textit{Braz. Age} ii. ii. Wks. III. 213 A beauteous Lady, art-full wise.    
\P 1681 JORDAN  \textit{Lond. Joy} in \textit{Heath Grocers' Comp.} (1869) 544 A piece worthy of an artful man's Examination.

\itembf{2.} Having practical, operative, or constructive skill; dexterous, clever. arch.

\P 1697 DRYDEN  \textit{Life Virgil} (R.) Too artful a writer to set down events in exact historical order.    
\P 1710 SHAFTESBURY  \textit{Charac.} iii. i. (1737) II. 385 Subtile Threds spun from their artful Mouths!    
\P 1718 POPE  \textit{Iliad} xiv. 204 Her artful hands the radiant tresses tied.

\itembf{3.} Skilful in adapting means to ends, so as to secure the accomplishment of a purpose, adroit; passing gradually into: Skilful in taking an unfair advantage; using stratagem, wily; cunning, crafty, deceitful.

\P 1739 T. SHERIDAN  \textit{Persius} i. 23 Horace was more artful, and in a merry Way touched upon his Friends' Faults without putting them out of Humour.    
\P 1760 MITCHELL in ellis \textit{Orig. Lett.} ii. 480 IV. 419 Make use of the artful pen of Voltaire to draw secrets from the King of Prussia.    
\P 1797 T. BEWICK  \textit{Brit. Birds} I. 73 Made use of by artful and designing men.    
\P 1857 BOHN  \textit{Handbk. Prov.} 67 An artful fellow is the devil in a doublet.

\itembf{II.} Of things, actions, etc.

\itembf{4.} Displaying or characterized by technical skill; performed or executed in accordance with the rules of art; artistic. arch.

\P 1615 \textit{Latham's Falconry} Pref. Verses, To‥force her to your voice and luring fall, Is strangely artfull.    
\P 1637 MILTON  \textit{Comus} 494 Thyrsis! whose artful strains have oft delayed The huddling brook.    
\P 1718 J. CHAMBERLAYNE  \textit{Relig. Philos.} I. vi. §8 So artful a Machine as every Man is.    
\P 1799 G. SMITH  \textit{Laboratory} I. 41 It would not be deemed an artful performance to fire one cartouch after another.

\itembf{5.} Produced by art, as opposed to what is natural; artificial, imitative, unreal.

\P 1706 ADDISON  \textit{Rosamond} ii. i, In yon cool grotto's artful night.    
\P 1779 J. MOORE  \textit{View Soc. Fr.} viii. (1789) I. 55 The artful distresses of a romance.    
\P 1857 EMERSON  \textit{Poems} 114 Smite the chords‥That they may render back Artful thunder.

\itembf{6.} Skilfully adapted for the accomplishment of a purpose; ingenious, clever; passing gradually into: Cunning, crafty, deceitful.

\P 1705 STANHOPE  \textit{Paraphr.} I. 217 Artful Reasonings, and most moving Eloquence.    
\P 1712 STEELE  \textit{Spect.} No. 400 \cardo{⁋}2 Artful Conformity to the Modesty of a Woman's Manners.    
\P 1843 MILL  \textit{Logic} ii. iv. §4 The marks, by an artful combination of which men have been able to discover and prove all that is proved in geometry.    
\P 1865 DICKENS  \textit{Mut. Fr.} xv, This is a very artful dodge.
\end{myenumerate}


%%%%%%%%%%%%%%%%%%%%%%%%%%%%%%%%
\myitem{artless} a.

\noindent \phonetic{(ˈɑːtlɪs)}

\noindent [f. art n. + -less.]
\vspace{-0.3cm}

\begin{myenumerate}

\itembf{1.} Devoid of art or skill: \textbf{a.} Unpractised, inexperienced, unskilful; unskilled, ignorant.

\P 1589 NASHE  \textit{Anat. Absurd.} 40 The artlesse tongue of a tedious dolt.    
\P 1628 WITHER  \textit{Brit. Rememb.} vii. 1184 Such artlesse riders, that they cannot sit them.    
\P 1747 JOHNSON  \textit{Plan Eng. Dict.} Wks. IX. 165 The work in which I engaged is generally considered‥as the proper toil of artless industry.    
\P 1847 LD. LINDSAY  \textit{Chr. Art} I. 124 The artless artists seem to have worked on, from arch to arch‥without a thought‥ of economising their space.

\itembf{b.} Devoid of the fine or liberal arts; having no desire for or endeavour after artistic effect; uncultured.

\P 1599 MARSTON  \textit{Sco. Villanie} ii. Proem 192 Seeking conceits to sute these Artlesse times.    
\P 1636 BALLARD in \textit{Ann. Dubrensia} (1877) 35 The rugged Poem of an Art-lesse Muse.    
\P 1774 J. BRYANT  \textit{Mythol.} I. 46 The most dry and artless historians are in general the most authentic.    
\P 1860 RUSKIN  \textit{Mod. Paint.} V. ix. ii. 216 A shadowy life—artless, joyless, loveless. No devices in that darkness of the grave.

\itembf{2. a.} Constructed without art or skill, rude, clumsy. \textbf{b.} Designed without art, inartistic, crude.

\P 1695 WOODWARD  \textit{Nat. Hist. Earth} iii. i. (1723) 166 That there is any thing incommodious and Artless‥in the Globe.    
\P 1774 JOHNSON  \textit{West. Isl.} Wks. X. 373 Brogues, a kind of artless shoes.    
\P 1782 WARTON  \textit{Hist. Kiddington} (T.) Assemblages of artless and massy pillars.    
\P 1878 LUBBOCK  \textit{Preh. Times} v. 141 They enclose an artless stone vault.

\itembf{3.} Free from art (as opposed to nature); unartificial, natural, simple.

\P 1672 DRYDEN in \textit{Shaks. C. Praise} 348 Such Artless beauty lies in Shakespears wit.    
\P 1752 C. LENNOX  \textit{Fem. Quix.} I. i. ii. 8 Curls, which had so much the appearance of being artless, that all but her maid‥imagined they were so.    
\P 1754 SHERLOCK  \textit{Disc.} (1759) I. iv. 169 The Doctrines of the Gospel were artless and plain.    
\P 1852 A. JAMESON  \textit{Leg. Madonna} 152 The same artless grace, the same dramatic grouping.

\itembf{4.} Simple-minded, sincere, guileless, ingenuous.

\P 1714 BUDGELL  \textit{Spect.} No. 605 \cardo{⁋}9 Imitation is a kind of artless Flattery.    
\P 1766 WESLEY  \textit{Wks.} (1872) III. 247 The artless people drank in every word.    
\P 1822 W. IRVING  \textit{Braceb. Hall} v. 43 The delightful blushing consciousness of an artless girl.    
\P 1868 STANLEY  \textit{Westm. Ab.} i. 34 His artless piety and simple goodness.
\end{myenumerate}


%%%%%%%%%%%%%%%%%%%%%%%%%%%%%%%%%
\myitem{ascetic} a. and n.

\noindent \phonetic{(əˈsɛtɪk)}

\noindent [ad. Gr. ἀσκητικός adj., f. ἀσκητής a monk or hermit, f. ἀσκέ-ειν to exercise: see -ic.]
\vspace{-0.3cm}

\begin{myenumerate}

\itembf{A.} adj.

\itembf{1.} Of or pertaining to the Ascetics, or to the exercise of extremely rigorous self-discipline; severely abstinent, austere.

\P 1646 SIR T. BROWNE  \textit{Pseud. Ep.} viii. 126 This ascetic rule, which held that a saint was disgraced by the very society which his mild Master sought and loved.    
\P 1682 \textit{Chr. Morals} (1756) 97 The old Ascetick christians found a paradise in a desert.    
\P 1757 BURKE  \textit{Abridgm. Eng. Hist.} Wks. X. 276 A monastery which had acquired great renown for‥the severity of its ascetick discipline.    
\P 1850 TENNYSON \textit{In Mem.} cix, High nature amorous of the good, But touch'd with no ascetic gloom.

\itembf{2.} = ascetical 1.

\P 1822 BURROWES  \textit{Cycl., Ascetic,} the title of certain books on devout exercises.    
\P 1868 PATTISON  \textit{Academ. Org.} §5. 122 The knowledge to be cultivated is not ascetic divinity.

\itembf{B.} n.

\itembf{1.} Eccl. Hist. (Freq. with cap. initial.) One of those who in the early church retired into solitude, to exercise themselves in meditation and prayer, and in the practice of rigorous self-discipline by celibacy, fasting, and toil.

\P 1673 CAVE  \textit{Prim. Chr.} iii. ii. 253 One of the primitive Asceticks.    
\P 1776 GIBBON  \textit{Decl. \& F.} xxxvii. (R.) The Ascetics, who obeyed and abused the rigid precepts of the gospel.    
\P 1861 A. BERESFORD-HOPE  \textit{Eng. Cathedr. 19th C.} v. 165 The deserts of the Thebaïd had been peopled by troops of sturdy and gaunt but God-fearing ascetics.

\itembf{2.} gen. One who is extremely rigorous in the practice of self-denial, whether by seclusion or by abstinence from creature comforts.

\P 1660 JER. TAYLOR  \textit{Ductor Dubit.} ii. iii. 8. §4 The primitive Christians were generally such ascetics in this instance of fasting.    
\P 1862 STANLEY  \textit{Jewish Ch.} (1877) I. i. 17 He is not an ascetic‥but full of the affections and interests of family and household.

\itembf{3.} pl. An ascetical treatise.

\P 1751 CHAMBERS  \textit{Cycl.} s.v., Books of spiritual exercises. As the ascetics, or devout treatises of St. Basil.
\end{myenumerate}


%%%%%%%%%%%%%%%%%%%%%%%%%%%%%%%%%
\myitem{asperity} n.

\noindent \phonetic{(əˈspɛrɪtɪ)}

\noindent [a. OF. asprete (mod. âpreté):—L. asperitātem, f. asper rough: see -ty. Subseq. assimilated to the L. word.]
\vspace{-0.3cm}

\begin{myenumerate}

\itembf{1.} Unevenness of surface, roughness, ruggedness; concr. in pl. sharp, rough, or rugged excrescences.

\P 1491 CAXTON  \textit{Vitas Patr.} (W. de W.) i. xxxvii. 50 a/1, Fewe people wente for to see him, for the grete asprete or sharp⁓nesse of the place.    
\P 1578 LYTE  \textit{Dodoens} 246 Iuyce of Mynte‥taketh away the asperitie, and roughnesse of the tongue.    
\P 1662 H. MORE  \textit{Antid. Ath.} ii. xii. (1712) 84 To view the Asperities of the Moon through a Dioptrick-glass.    
\P 1743 tr.  \textit{Heister's Surg.} 396 If any splinters or Asperities of Bones present themselves.    
\P 1830 LINDLEY  \textit{Nat. Syst. Bot.} 25 Almost all Delimaceæ have the leaves covered with asperities.

\itembf{2.} Roughness of savour, tartness, acridity, acrimony. arch.

\P 1620 VENNER  \textit{Via Recta} v. 87 Very good for the asperity and siccity of the stomacke.    
\P 1667 \textit{Phil. Trans.} II. 512 Esteeming the Mass of bloud by reason of its asperity‥unfit for nutrition.    
\P 1747 BERKELEY  \textit{Siris} §86 (T.) The asperity of tartarous salts.

\itembf{3.} Harshness of sound, grating quality. arch.

\P 1664 H. MORE  \textit{Myst. Iniq.} 239 The shrilness and asperity of the noise they make.    
\P 1750 JOHNSON  \textit{Rambl.} No. 88 \cardo{⁋}12 Our language, of which the chief defect is ruggedness and asperity.    
\P 1774 J. BRYANT  \textit{Mythol.} I. 167 A place in Egypt, which he could not specify on account of its asperity.

\itembf{4.} Of literary style: Ruggedness, lack of polish, inelegance. arch.

\P 1779 JOHNSON  \textit{Cowley} Wks. II. 66 Avoids with very little care either meanness or asperity.     \textit{Philips} ibid. II. 293 Those asperities that are venerable in the Paradise Lost are contemptible in the Blenheim.

\itembf{5.} fig. Harshness to the feelings, rigour, severity; hence, hardship, difficulty. (The earliest sense; arch. exc. in \itembf{b.} Bitter coldness, rigour, bleakness.)

\P c1230 \textit{Ancr. R.} 354 Vilte and asprete‥scheome and pine‥beoð þe two leddre stalen þet beoð upriht to þe heouene.
\P a1535 MORE \textit{Wks.} 1218 (R.) To‥minysh the vygour and asperite of the paynes.    
\P 1659 HARDY  \textit{Serm.} 1 John xlix. (1865) 318/1 This oil [of gladness]‥mitigateth the asperity of affliction.    
\P 1750 JOHNSON  \textit{Rambl.} No. 80 \cardo{⁋}4 The nakedness and asperity of the wintry world.    
\P 1866 \textit{Daily Tel.} 16 Jan. 7/5 The great asperity of the climate in winter.

\itembf{6.} Harshness or sharpness of temper, esp. when displayed in tone or manner; crabbedness, bitterness, acrimony; in pl. harsh, embittered feelings.

\P 1664 H. MORE  \textit{Myst. Iniq. Apol.} 554 Animosities, and asperities of mind about toys and trifles.    
\P 1757 JOHNSON  \textit{Rambl.} No. 176 \cardo{⁋}8 Quickness of resentment and asperity of reply.    
\P 1838 DICKENS  \textit{Nich. Nick.} iii. (C.D. ed.) 13 Demanded with much asperity what she meant.
\end{myenumerate}


%%%%%%%%%%%%%%%%%%%%%%%%%%%%%%%%%
\myitem{aspersion} n.

\noindent \phonetic{(əˈspɜːʃən)}

\noindent [ad. L. aspersiōn-em, n. of action f. aspers-: see asperse and -ion1.]
\vspace{-0.3cm}

\begin{myenumerate}

\itembf{1.} The action of besprinkling (a person or thing), or of sprinkling or scattering (liquid, dust, etc.).

\P 1553-87 FOXE  \textit{A. \& M.} I. 497/1 By the aspersion of the bloud of Jesus Christ.    
\P 1699 BURNET  39 \textit{Articles} xx. (1700) 193 Aspersion may answer the true end of Baptism.    
\P 1782 PRIESTLEY  \textit{Corrupt. Chr.} II. viii. 109 They make many aspersions of holy water.    
\P 1846 W. MASKELL  \textit{Mon. Rit.} I. 209 St. Peter‥baptized five thousand on one day; but this must have been by aspersion.

\itembf{2.} That which is sprinkled; a shower or spray.

\P 1610 SHAKES.  \textit{Temp.} iv. i. 18 No sweet aspersion shall the heauens let fall To make this contract grow.    
\P 1845 \textit{Blackw. Mag.} LVII. 584 An aspersion of cold water was dashed‥in the impassioned faces of the pair.

\itembf{3.} The sprinkling in of an ingredient. Obs.

\P 1605 BACON  \textit{Adv. Learn.} i. 29 There is to bee found besides the Theologicall sence, much aspersion of Philosophie.    Ibid. ii. 79 Divinity Morality and Policy, with great aspersion of all other artes.    a 
\P 1656 HALES  \textit{Golden Rem.} (1688) 34 Without any Aspersion of Severity.

\itembf{4.} Bespatterment with what soils; soil, stain. Obs.

\P 1614 T. ADAMS in Spurgeon \textit{Treas. Dav.} Ps. vi. 6 (1870) I. 70 Whatsoever aspersion the sin of the day has brought upon us.

\itembf{5.} The action of casting damaging imputations, false and injurious charges, or unjust insinuations; calumniation, defamation.

\P 1633 G. HERBERT  \textit{Charms \& Knots} in Temple 89 Who by aspersions throw a stone At th' head of others, hit their own.    
\P 1781 COWPER  \textit{Friendship} xvii, Aspersion is the babbler's trade, To listen is to lend him aid.    
\P 1873 GOULBURN  \textit{Pers. Relig.} iv. xi. 347 Imperious aspersion of God.

\itembf{6.} A damaging report; a charge that tarnishes the reputation; a calumny, slander, false insinuation. Esp. in the phr. to cast aspersions upon.

\P 1596 SPENSER  \textit{State Irel.} Pref. 2 Which may seeme to lay‥any particular aspersion upon some families.    
\P 1662 FULLER  \textit{Worthies} (1840) III. 120 As false is the aspersion of his being a great usurer.    
\P 1692 JAMES II \textit{Royal Tracts} * * G iv, Malicious Aspertions.    
\P 1749 FIELDING  \textit{Tom Jones} (1775) II. 209, I defy all the world to cast a just aspersion on my character.    
\P 1859 GEO. ELIOT  \textit{A. Bede} 53 Vindicating myself from the aspersions.
\end{myenumerate}


%%%%%%%%%%%%%%%%%%%%%%%%%%%%%%%%
\myitem{assiduous} a.

\noindent \phonetic{(əˈsɪdjuːəs)}

\noindent [f. L. assidu-us (f. assidē-re to sit by: see assess v.; lit. ‘sitting down to,’ hence ‘closely applying to’) + -ous.]
\vspace{-0.3cm}

\begin{myenumerate}

\itembf{1.} Of persons or agents: Constant in application to the business in hand, persevering, sedulous, unwearyingly diligent.

\P 1660 JER.  TAYLOR \textit{Duct. Dubit.} ii. ii. vii. §3 Christ‥commands us to be perfect, that is‥to be assiduous in our prayers.    
\P 1711 ADDISON  \textit{Spect.} No. 311 \cardo{⁋}5 Those assiduous Gentlemen who employ their whole Lives in the Chace.    
\P 1876 GREEN  \textit{Short Hist.} iii. §7 (1882) 148 He was assiduous in his attendance on religious services.

\itembf{2.} Constantly endeavouring to please, obsequiously attentive. arch.

\P 1725 POPE  \textit{Odyss.} vi. 89 The queen, assiduous, to her train assigns The sumptuous viands.    
\P 1750 JOHNSON  \textit{Rambl.} No. 104 \cardo{⁋}13 Few can be assiduous without servility.

\itembf{3.} Of actions: Unremitting, persistent, constant.

\P 1538 LELAND  \textit{Itin.} I. Introd. 20 By infinite Variete of Bookes and assiduus reading of them.    
\P 1667 MILTON  \textit{P.L.} xi. 310 To wearie him with my assiduous cries.    
\P 1711 ADDISON  \textit{Spect.} No. 10 \cardo{⁋}1 Follies that are only to be killed by a constant and assiduous Culture.    
\P 1849 MACAULAY  \textit{Hist. Eng.} I. 491 Baxter's life was chiefly passed‥in the assiduous discharge of parochial duties.

\itembf{4.} Of things: Constant, regular. Obs.

\P 1661 EVELYN  \textit{Fumifug. Misc. Writ.} (1805) 1 217 The Election of this constant and assiduous food, should something concerne us.
\end{myenumerate}


%%%%%%%%%%%%%%%%%%%%%%%%%%%%%%%%
\myitem{assuage} v.

\noindent \phonetic{(əˈsweɪdʒ)}

\noindent [a. OF. a(s)souage-r, -agier, Pr. a(s)suaviar, f. L. type *assuāviāre, f. ad to + suāvis sweet, agreeable. Cf. abridge, aggrege, allege (L. abbreviāre, aggraviāre, alleviāre).]
\vspace{-0.3cm}

\begin{myenumerate}

\itembf{I.} trans.

\itembf{1.} To soften, mitigate, calm, appease, allay (angry or excited feelings).

\P 1330 R. BRUNNE  \textit{Chron.} 300 His wrath forto asuage.
\P c1420 \textit{Pallad  on Husb.} iv. 883 But yf he bite hir in his rage, Let labouryng his melancoly swage.    
\P 1513 MORE  \textit{Rich. III,} Wks. 35/2 The displeasure of those that bare him grudge‥was well asswaged.    
\P 1642 ROGERS  \textit{Naaman} 32 God hath asswaged his pride, and tamed him.    
\P 1777 WATSON  \textit{Philip II} (1793) II. xiv. 229 They omitted nothing in their power to assuage his resentment.    
\P 1857 BUCKLE  \textit{Civilis.} viii. 500 That secular spirit which, in every country, has assuaged religious animosities.

\itembf{2.} To pacify, appease, calm (the excited person).

\P 1325 \textit{E.E. Allit. P.} C. 3 When heuy herttes ben hurt wyth heþyng‥Suffraunce may aswagen hem.    
\P 1596 SPENSER  \textit{F.Q.} v. ii. 47 But Artegall him fairely gan asswage.    
\P 1598 FLORIO,  \textit{Propitiare}‥to asswage God with sacrifice.    
\P 1706 ADDISON  \textit{Rosamond} ii. vi, Kindling pity, kindling rage At once provoke me, and asswage.    
\P 1763 SIR W. JONES  \textit{Caissa} Poems (1777) 33 So may thy prayers assuage the scornful dame.    
\P 1858 HAWTHORNE  \textit{Fr. \& It. Jrnls.} I. 295, I shall‥assuage and mollify myself a little after that uncongenial life of the consulate.

\itembf{3.} To relax, modify, moderate (a harsh law, etc.).

\P c1300 BEKET 1454 That the King wolde‥aswagi the lithere lawes.    
\P 1483 CAXTON  \textit{Gold. Leg.} 287/1, I pray the‥that thou asuage uppon hym the sentence of dampnacion.

\itembf{4.} To mitigate, alleviate, soothe, relieve (physical or mental pain); to lessen the violence of (disease).

\P 1393 GOWER  \textit{Conf.} I. 267 That shulde assuage The leper.  
\P c1400 \textit{Rom. Rose} 2815 Thus Swete-Thenkyng shalle aswage The peyne of lovers.    
\P 1561 T. N[ORTON]  \textit{Calvin's Inst.} iii. 206 Then were there ministred other plaisters to asswage such peines.    
\P 1605 BACON  \textit{Adv. Learn} ii. xxii. §1 They need medicine‥to assuage the disease.    
\P 1725 POPE  \textit{Odyss.} ii. 29 The rest with duteous love his griefs asswage.    
\P 1868 MILMAN  \textit{St. Paul's} xix. 481 Perhaps no man has assuaged so much human misery as John Howard.

\itembf{5.} To appease, satisfy (appetites, desires).

\P 1430 LYDG.  \textit{Venus-Mass} in \textit{Lay Folk's Mass-Bk.} 394 Water or wyne‥asswage the grete dryhnesse of ther gredy thruste.    
\P 1697 DRYDEN  \textit{Virg. Georg.} ii. 791 The good old God his Hunger did asswage With Roots and Herbs.    
\P 1812 COMBE (Dr. Syntax) \textit{Picturesque} x. 57 His thirst assuage With tea that's made of balm or sage.    
\P 1856 MRS. STOWE  \textit{Dred} II. xxvii. 278 So the fearful craving of his soul for justice was assuaged.

\itembf{6.} gen. To abate, lessen, diminish (esp. anything swollen). arch. or Obs.

\P c1430 LYDG.  \textit{Min. Poems} 64 His olde gyltis bothe to asoft and swage.    
\P 1494 FABYAN  VII. ccxxxvi. 273 Short of body, and therwith fatte; the whiche to aswage he toke ye lesse of metis.
\P 1525 SKELTON  \textit{El. Rummyng} 10 For her visage It would aswage A mannes courage.    
\P 1667 PEPYS  \textit{Diary} 20, 21 Dec., My poor wife is in mighty pain, and her face miserably swelled‥My wife is a little better, and her cheek asswaged.    
\P 1774 J. BRYANT  \textit{Mythol.} II. 284 The Dove‥brought the first tidings that the waters of the deep were asswaged.

\itembf{II.} intr.

\itembf{7.} Of passion, pain, appetite, etc. (from senses 1, 2, 4, 5): To become less violent, to abate. Obs.

\P 1330 R. BRUNNE  \textit{Chron.} 78 Of his crueltes he gynnes forto assuage.    c 
\P 1386 CHAUCER  \textit{Merch. T.} 838 His sorwe gan aswage.    
\P 1509 HAWES  \textit{Past. Pleas.} xviii. xvi, The great payne of love May not aswage tyl death it remove.    
\P 1607 TOPSELL  \textit{Four-f. Beasts} 57 Their lust asswageth till another time.    
\P 1722 DE FOE  \textit{Plague} 191 The plague being come to a crisis, its fury began to assuage.

\itembf{8.} gen. To grow less, diminish, decrease, fall off, die away; to abate, subside. arch. or Obs.

\P 1430 \textit{Hymns  to Virg.} (1867) 79 Take hede‥How fast ȝoure ȝouþe dooþ asswage.    
\P 1523 LD. BERNERS  \textit{Froiss.} I. xxviii. 42 Kyng Phylippes enterprise of ye sayd Croysey beganne to asswage and waxe cold.    
\P 1611 BIBLE  \textit{Gen.} viii. 1 And the waters asswaged.    
\P 1677 MOXON  \textit{Mech. Exerc.} 242 The Fire in Lime burnt, Asswages not, but lies hid.    
\P 1858 MOTLEY  \textit{Dutch Rep.} Introd. v. 17 As the deluge assuaged.
\end{myenumerate}


%%%%%%%%%%%%%%%%%%%%%%%%%%%%%%%%
\myitem{atavistic} a.

\noindent \phonetic{(ætəˈvɪstɪk)}

\noindent [f. prec.: see -istic.]

Of or pertaining to atavism; atavic; of or pertaining to a remote ancestor.

\P 1875 \textit{N. Amer. Rev.} CXX. 275 The social and the atavistic influence.    
\P 1915 W. S. MAUGHAM  \textit{Of Human Bondage} xxvi. 108 Some atavistic inheritance of the cave-dweller.    
\P 1922 JOYCE  \textit{Ulysses} 676 The sporadic reappearance of atavistic delinquency.    
\P 1932 E. WAUGH  \textit{Black Mischief} v. 168 Was it some atavistic sense of a caste, an instinct of superiority, that held him aloof?

Hence \textbf{atavistically} adv.

\P 1884 \textit{N. Amer. Rev.} Sept. 253 The ancient types crop out atavistically.    
\P 1897 E. P. EVANS  \textit{Evol. Ethics} i. 33 The lower classes‥reflect atavistically the ideas and passions of primitive man.    
\P 1926 \textit{Blackw. Mag.} Apr. 446/2 Some of them bolted atavistically up the nearest tree.


%%%%%%%%%%%%%%%%%%%%%%%%%%%%%%%%
\myitem{atrophy} n.

\noindent \phonetic{(ˈætrəfɪ)}

\noindent [a. F. atrophie, ad. L. atrophia, Gr. ἀτροϕία, n. of state f. ἄτροϕος ill-fed, not nourished, f. ἀ priv. + τροϕή nourishment.]
\vspace{-0.3cm}

\begin{myenumerate}

\itembf{1.} A wasting away of the body, or any part of it, through imperfect nourishment: emaciation.

\P 1620 VENNER  \textit{Via Recta} viii. 189 Which‥bringeth the body into a deformed Atrophie or consumption.    
\P 1667 MILTON  \textit{P.L.} xi. 486 Moon-struck madness, pining atrophy.    
\P 1862 TRENCH  \textit{Mirac.} xix. 323 A partial atrophy, showing itself in a gradual wasting of the size of the limb.

\itembf{2.} fig.

\P 1653 JER. TAYLOR  \textit{Serm. Year} Ded., We‥fear the people will fall to an Atrophy, then to a loathing of holy food.    
\P 1782 J. TRUMBULL  \textit{M'Fingal} iv. (1795) 102 By fatal atrophy of purse.    
\P 1840 CARLYLE  \textit{Heroes} (1858) 315 For the Scepticism‥is‥a chronic atrophy and disease of the whole soul.
\end{myenumerate}

%%%%%%%%%%%%%%%%%%%%%%%%%%%%%%%%
\myitem{attenuate} v.

\noindent \phonetic{(əˈtɛnjuːeɪt)}

\noindent [f. L. attenuāt- ppl. stem of attenuāre, f. at- = ad- to + tenuāre to make thin, f. tenuis thin. Cf. F. atténuer, 12th c.]
\vspace{-0.3cm}

\begin{myenumerate}

\itembf{1.} To make thin or slender in girth or diameter (e.g. by natural or artificial shaping, drawing out, wearing down, starving, physical decay).

\P 1530 PALSGR.  440/1, I attenuate, I make thynne, Jattenue.    
\P 1621 BURTON  \textit{Anat. Mel.} i. ii. iii. x. (1651) 111 They crucifie the soul of man, attenuate our bodies.    
\P 1668 CULPEPPER \& COLE tr. \textit{Barthol. Anat.} i. xvii. 47 The Ureters in their progress are not attenuated within, as other Vessels are.    
\P 1794 SULLIVAN  \textit{View Nat.} I. 47 This shell also being attenuated‥the surface of the earth will tumble in.    
\P 1848 A. JAMESON  \textit{Sacr. \& Leg. Art} (1850) 203 The wasted unclad form is seen attenuated by vigils.    
\P 1876 BANCROFT  \textit{Hist. U.S.} III. iii. 344 To attenuate them by gently drawing them out.

\itembf{2. a.} To make thin in consistency, to separate the particles of a substance, to diminish density, rarefy.

\P 1594 PLAT  \textit{Jewell-ho.} i. 40 Earth beeing attenuated becommeth water.    
\P 1691 E. TAYLOR  \textit{Behmen's Theos. Phil.} 187 The Suns lustre attenuateth the gross air.    
\P 1756 C. LUCAS  \textit{Ess. Waters} I. 48 Burning spirits‥are oils attenuated and subtilised by the action of fermentation.    
\P 1762 tr. \textit{Duhamel's Husb.} i. iii. 5 Salt, for example, may attenuate earth.    
\P 1874 [SEE ATTENUATED 2.]

\itembf{b.} spec. in Med. To render thinner (the humours or concretions of the body).

\P 1533 ELYOT  \textit{Cast. Helth} ii. xiv. (R.) Dry figges‥havinge power to attenuate or make humours currant.    
\P 1605 TIMME  \textit{Quersit.} i. xiii. 64 O[y]le of pepper doth attenuat‥tartarus matters in the body.    
\P 1797 DOWNING  \textit{Disord. Horn. Cattle} 13 These medicines‥powerfully attenuate the cloggy disposition of the blood.

\itembf{3.} fig. To weaken or reduce in force, effect, amount; in value, estimation; (obs.) to extenuate.

\P 1530 PALSGR.  440/1 He hath attenuat my power.    
\P 1579 LYLY  \textit{Euphues} (Arb.) 49 The delightfulnesse of the one will attenuate the tediousnesse of the other.    c 
\P 1645 HOWELL  \textit{Lett.} (1650) I. 335 The Mahometans‥attenuated their numbers in Asia.    
\P 1660 A. SIDNEY in \textit{Four C. Eng. Lett.} 119 To aggravate that, which he doth intend to attenuate.    
\P 1850 \textit{Q. REV.}  June 15 Some Notes‥intended to attenuate the authority of the Christian philosopher.    
\P 1869 LECKY  \textit{Europ. Mor.} I. i. 117 To attenuate‥his own appetites and emotions.

\itembf{4.} intr. To become slender, thinner, or weaker.

\P a1834 COLERIDGE  (in Webster), The attention attenuates as its sphere contracts.

\itembf{5.} Electr. To introduce attenuation; in pass., to be subjected to attenuation. Cf. attenuation 4.

\P 1886 LORD RAYLEIGH  in \textit{Phil. Mag.} 5th Ser. XXII. 490 If we had the means of observing the passage of signals at various points of a long cable, we should find them not merely retarded‥as we recede from the sending end, but also attenuated.    
\P 1892 HEAVISIDE  \textit{Electr. Papers} II. 133 The act of reflection attenuates.    Ibid. 346 During transmission along the circuit, the vibrations are attenuated.    
\P 1959 \textit{Chambers's Encycl.} VII. 696/2 The lower part of the Heaviside layer is of particular importance‥because it is in this region that radio waves used for long-distance communication are attenuated.
\end{myenumerate}

%%%%%%%%%%%%%%%%%%%%%%%%%%%%%%%%
\myitem{augment} v.

\noindent \phonetic{(ɔːgˈmɛnt)}

\noindent [a. F. augmente-r (14th c.), earlier aumenter, cogn. with It. aumentare, Sp. aumentar:—L. augmentā-re to increase, f. augment-um: see prec.]
\vspace{-0.3cm}

\begin{myenumerate}

\itembf{1.} trans. To make greater in size, number, amount, degree, etc.; to increase, enlarge, extend.

\P 1460 FORTESCUE  \textit{Abs. \& Lim. Mon.} (1714) 116 Hou our Navye may be mayntenyd, and augmentyd.    
\P 1561 T. N[ORTON]  \textit{Calvin's Inst.} iv. xiv. (1634) 634 marg., The power which Sacraments have in augmenting Faith.    
\P 1601 HOLLAND  \textit{Pliny} I. 58 [The Tiber] is augmented with two and forty riuers.    
\P 1763 J. BROWN  \textit{Poetry \& Mus.} §5. 66 The Chords of the Lyre were augmented gradually from four to forty.    
\P 1816 SCOTT  \textit{Old. Mort.} 217 The insurgents were intent upon augmenting and strengthening their forces.

\itembf{2.} intr. To become greater in size, amount, degree, intensity, etc.; to increase, grow, swell.

\P c1400 \textit{Rom. Rose} 5600 For to encrese, and not to lesse, For to aument and multiplie.    
\P 1475 CAXTON  \textit{Jason} 51 The bruit of preu Jason augmentid and encresid from day to day.    
\P 1589 GREENE  \textit{Menaph.} (Arb.) 39 The grasse hath his increase, yet never anie sees it augment.    
\P 1697 DRYDEN  \textit{Virg. Georg.} i. 466 The Winds redouble, and the Rains augment.    
\P 1869 TYNDALL  \textit{Light} §436 The polarizing angle augments with the refractive index of the medium.

\itembf{3.} trans. To increase or add to the resources of; to enhance in circumstances. Obs.

\P c1460 FORTESCUE  \textit{Abs. \& Lim. Mon.} (1714) 93 To augment his Realme in Rycesse, Welth, and Prosperyte.    
\P 1529 WOLSEY in \textit{Four C. Eng. Lett.} 11 Aggmentyng my lyvyng, and appoyntyng such thyngs as shuld be convenient for my furniture.    
\P 1601 CORNWALLYES  \textit{Essayes} ii. xxxvi. (1631) 117 Thou augmentest their state purchasing a blessing upon their house and life.

\itembf{4.} trans. and refl. To raise (a person) in estimation or dignity; to exalt. Obs.

\P 1567 \textit{Trial  Treas.} in Hazl. \textit{Dodsley} III. 273 Labour yourself to advance and augment.    
\P 1655 FULLER  \textit{Ch. Hist.} iii. ii. §43 II. 84 Theobald‥was augmented with the title of Legatus natus.

\itembf{b.} intr. To rise in estimation or dignity. Obs.

\P 1534 LD. BERNERS  \textit{Gold. Bk. M. Aurel.} I v b, With a littell fauour ye wyll exalt, augement, and grow into gret prid.

\itembf{5.} Her. (trans.) To make an honourable addition to (a coat of arms).

\P 1655 FULLER  \textit{Ch. Hist.} iv. II. 357 The Armes of London were augmented with the addition of a Dagger.    
\P 1864 BOUTELL  \textit{Heraldry Hist. \& Pop.} xiii. 95 The Scottish Baronets‥were authorized to augment their own arms.

\itembf{6.} To multiply (mathematically). Obs.

\P 1571 DIGGES  \textit{Pantom.} iii. iii. Q ij, The Solide content of a Cylinder is gotten by augmenting the base in his altitude.    
\P 1593 T. FALE  \textit{Dialling} 31 Augment the Sine of the Complement repeated, by the Sine of the doubtfull Arke: an the product arising thereof‥shall be the distance, etc.
\end{myenumerate}



%%%%%%%%%%%%%%%%%%%%%%%%%%%%%%%%
\myitem{augment} n.

\noindent \phonetic{(ˈɔːgmənt)}

\noindent [a. F. augment (14th c.), ad. L. augmentum increase, f. augēre to increase: see -ment.]
\vspace{-0.3cm}

\begin{myenumerate}

\itembf{1.} Increase, extension, augmentation. Obs.

\P 1430 LYDG.  \textit{Chron. Troy} i. v, In augment of thy wo.    
\P 1501 DOUGLAS  \textit{Pal. Hon. Prol.} i. x, In the is rute and agment of curage.    
\P 1599 THYNNE  \textit{Animadv.} 71 To seeke the augmente and correctione of Chawcers Woorkes.    
\P 1677 PLOT  \textit{Oxfordsh.} 132 That though indeed there be an augment in some petrifications, yet that it is not so in all.    
\P 1696 PHILLIPS, Augment‥an encreasing.

\itembf{2.} Gram. The prefixed vowel (in Sanskrit ă, in Greek ε) which characterizes the past tenses of the verb in the older Aryan languages. (Sometimes applied to any prefix supposed to be of analogous use, e.g. the ge- of past participles in German.)
(In Greek, when the ε remains separate, it is called the syllabic augment; when it forms, with a following vowel, a long vowel or diphthong, the temporal augment.) Hence augmentless a., wanting the verbal augment.

\P 1771 GRAY in \textit{Corr.} (1843) 226 The y which we often see prefixed to participles passive, ycleped, yhewe, etc.‥is the old Anglo-Saxon augment.    
\P 1861 JELF  \textit{Grk. Gram.} I. §171 The augment is employed in the indicative mood only of all the historic tenses.    
\P 1879 WHITNEY  \textit{Skr. Gram.} §585 The augment is a short a, prefixed to a tense stem‥The augment is a sign of past time.    Ibid. §587 The accentuation of the augmentless forms.
\end{myenumerate}


%%%%%%%%%%%%%%%%%%%%%%%%%%%%%%%%
\myitem{augur} v.

\noindent \phonetic{(ˈɔːgə(r))}

\noindent [f. prec. n.; or a. F. augure-r (14th c.), ad. L. augurāri, f. augur; see prec.]
\vspace{-0.3cm}

\begin{myenumerate}

\itembf{1.} trans. To prognosticate from signs or omens; to divine, forebode, anticipate.

\P 1601 B. JONSON  \textit{Poetaster} i. i, I did augur all this to him beforehand.    
\P 1775 BURKE  \textit{Sp. Conc. Amer.} Wks. III. 56 They augur misgovernment at a distance and snuff the approach of tyranny.    
\P 1827 SCOTT  \textit{Surg. Dau.} i. 25 The Docter‥hastened down stairs, auguring some new occasion for his services.    
\P 1852 D. MITCHELL  \textit{Bat. Summer} 70 Who augured from the very fact, a state of quietude.

\itembf{b.} Of things: to betoken, portend, give promise of.

\P 1826 SCOTT  \textit{Mal. Malagr.} i. 54 It seems to augur genius.    
\P 1843 LYTTON  \textit{Last Bar.} i. i. 32 Whose open, handsome, hardy face augured a frank and fearless nature.

\itembf{2.} intr. (or with subord. clause). To take auguries; to conjecture from signs or omens; to have foreknowledge or foreboding.

\P 1808 SCOTT  \textit{Marm.} iii. xv, Not that he augur'd of the doom, Which on the living closed the tomb.    
\P 1840 GEN. P. THOMPSON  \textit{Exerc.} (1842) V. 119 What have the cock-sparrows to do with it; do we augur from them, as the Romans did from chickens?    
\P 1877 SPARROW  \textit{Serm.} xxiii. 308 He may augur the gust is coming, but cannot prevent it.

\itembf{3.} esp. (with well or ill) \itembf{a.} Of persons: to have good or bad anticipations or expectations of, for.

\P 1803 WELLINGTON in \textit{Gurwood Disp.} II. 275, I augur well from this circumstance.    
\P 1849 MACAULAY  \textit{Hist. Eng.} I. 544 Fletcher, from the beginning, had augured ill of the enterprise.    
\P 1859 JEPHSON  \textit{Brittany} vi. 69 As I looked at his good-natured face I augured well for my reception.

\itembf{b.} Of things: to give good or bad promise. [Perh. ill was orig. a n. = evil.]

\P 1788 T. JEFFERSON  \textit{Writ.} (1859) II. 506 One vote, which augurs ill to the rights of the people.    
\P 1810 SCOTT  \textit{Lady of L.} iii. vii, All augured ill for Alpine's line.    
\P 1855 PRESCOTT  \textit{Philip II} (1857) 68 A reverential deference, which augured well for the success of his mission.

\itembf{4.} trans. (also with in) To induct into office or usher in with auguries; to inaugurate.

\P 1549 LATIMER  \textit{Serm. bef. Edw. VI} (Arb.) 46 Numa Pompilus, who was augured and created king [of] the Romaynes next after Romulus.    
\P 1865 \textit{Reader}  11 Feb. 157 Profuse promises have augured in its birth.
\end{myenumerate}




%%%%%%%%%%%%%%%%%%%%%%%%%%%%%%%%
\myitem{augur} n.

\noindent \phonetic{(ˈɔːgə(r))}

\noindent [a. L. augur, earlier auger; perh. f. av-is bird + -gar, connected with garrire to talk, garrulus talkative, and Skr. gar to shout, call, show, make known; but Fick would derive it from augēre to increase, promote, etc.; cf. auctor author n.]
\vspace{-0.3cm}

\begin{myenumerate}

\itembf{1.} A religious official among the Romans, whose duty it was to predict future events and advise upon the course of public business, in accordance with omens derived from the flight, singing, and feeding of birds, the appearance of the entrails of sacrificial victims, celestial phenomena, and other portents.

\P 1549 HOOPER  \textit{Commandm.} vi. Wks. (1852) 327 There were some called augures, that by observation of the birds of the air‥made men believe they knew things to come.    
\P 1719 D'URFEY  \textit{Pills} (1872) III. 78 Having like an Augur watched, Which way he took his flight.    
\P 1879 FROUDE  \textit{Cæsar} iii. 21 The College of Augurs could declare the auspices unfavourable, and so close all public business.

\itembf{2.} Hence extended to: A soothsayer, diviner, or prophet, generally; one that foretells the future.

\P 1593 DRAYTON  \textit{Eclogues} i. 7 Philomel, the augure of the Spring.    
\P 1647 R. STAPYLTON  \textit{Juvenal} 115 The Phrygians, Cilicians, and Arabians were very skilfull augurs, or diviners by the flight of birds.    
\P 1718 POPE  \textit{Iliad} i. 131 Augur accursed! denouncing mischief still, Prophet of plagues, for ever boding ill!
\end{myenumerate}


%%%%%%%%%%%%%%%%%%%%%%%%%%%%%%%
\myitem{auspicious} a.

\noindent [f. as prec. + -ous.] 
\vspace{-0.3cm}

\begin{myenumerate}
\itembf{1.} Ominous, esp. of good omen, betokening success, giving promise of a
favourable issue.  

\P 1614 SELDEN \textit{Titles Hon. 155} An auspicious flight of an Eagle towards him.
\P 1742 YOUNG \textit{Nt. Th. viii. 202} Beneath auspicious planets born. 
\P 1823 J. THACHER \textit{Mil. Jrnl. Amer. Rev. 155} The splendid achievement 
of General Gates is auspicious to his preferment.

\itembf{b.} Of persons: Predicting or prognosticating good. 

1702 Rowe \textit{Ambit. Step-Moth. ii. ii. 662} Auspicious Sage, I trust 
thee with my Fortune. 
\P 1879 C. ROSSETTI \textit{Seek \& Find 239} The aspect of jubilant auspicious
angels.

\itembf{2.} Favourable, favouring, conducive to success. 

1610 SHAKES. \textit{Temp. v. i. 314} I'le..promise you calme Seas, 
auspicious gales.
\P 1858 SEARS \textit{Athan. ii. xii. 248} The results..have a direct and 
auspicious bearing on the great subject.

\itembf{b.} Of persons: Showing favour, propitious, kind. 

1601 SHAKES. \textit{All's Well iii. iii. 8} And fortune play vpon thy prosperous
helme As thy auspicious mistris. 
\P 1756 C. LUCAS \textit{Ess. Waters I. Ded.}, Auspicious Heaven saw our distresses 
and dangers. 
\P 1871 ROSSETTI \textit{Poems 10} Fair with honorable eyes, Lamps of an auspicious 
soul.

\itembf{3.} Favoured by fortune, prosperous, fortunate. 

1616 BULLOKAR, Auspicious, lucky, fortunate. 
\P 1664 H. MORE \textit{Myst. Iniq. 241} But Harvest sometimes has a more 
auspicious sense. 
\P 1804 in Gurwood \textit{Disp. III.} 419 We.. have reposed for five auspicious 
years under the shadow of your protection.
\end{myenumerate}





%%%%%%%%%%%%%%%%%%%%%%%%%%%%%%%%
\myitem{autonomous} a.

\noindent \phonetic{(ɔːˈtɒnəməs)}

\noindent [f. Gr. αὐτόνοµ-ος making or having one's own laws, independent (f. αὐτο- self, own + νόµος law) + -ous.]
\vspace{-0.3cm}

\begin{myenumerate}

\itembf{1.} Of or pertaining to an autonomy.

\P 1800 W. TAYLOR  in \textit{Month. Mag.} VIII. 600 With an autocratic, not an autonomous, constitution.    
\P 1861 C. KING  \textit{Antique Gems} (1866) 237 The autonomous coins of Sybaris.

\itembf{2.} Possessed of autonomy, self-governing, independent. In Metaph.: see autonomy 1 c.

\P 1804 W. TAYLOR in \textit{Ann. Rev.} II. 244 If the [Irish] nation was to become autonomous.    
\P 1851 D. WILSON  \textit{Preh. Ann.} (1863) I. ii. i. 313 The autonomous Greek cities in Asia Minor.    
\P 1868 BAIN  \textit{Ment. \& Mor. Sc.} 736 The absolutely good Will must be autonomous—i.e., without any kind of motive or interest.

\itembf{3.} Biol. \itembf{a.} Conforming to its own laws only, and not subject to higher ones. \itembf{b.} Independent, i.e. not a mere form or state of some other organism.

\P 1861 H. MACMILLAN  \textit{Footn. Page Nat.} 158 Some of these productions may not be autonomous, some may seem to pass into each other by intermediate forms.    
\P 1881 \textit{Syd. Soc. Lex.} s.v. Autonomy, Anatomy and physiology are autonomous, since the phenomena presented by animals and plants are not at present referable to chemical, physical, or other laws.    
\P 1882 T. DYER in \textit{Nature} 23 Feb. 391 The view that they [lichens] are autonomous organisms.
\end{myenumerate}


%%%%%%%%%%%%%%%%%%%%%%%%%%%%%%%%%
\myitem{avuncular} a.

\noindent \phonetic{(əˈvʌŋkjʊlə(r))}

\noindent [f. L. avuncul-us maternal uncle, dim. of avus grandfather + -ar.]
\vspace{-0.3cm}

\begin{myenumerate}

\itembf{a.} Of, belonging to, or resembling, an uncle.

\P 1831 LANDOR \textit{Rupert} Wks. 1846 II. 571 Love‥Paternal or avuncular.    
\P 1854 THACKERAY  \textit{Newcomes} I. v. 50 Clive in the avuncular gig is driven over the downs.

\itembf{b.} (humorously) Of a pawnbroker: see uncle. Also absol.

\P 1832 \textit{Fraser's Mag.} V. 85 My only good suit is at present under the avuncular protection.    
\P 1859 SALA  \textit{Gaslight \& D.} iii. 37 If you enter one of these pawnshops‥you will observe these peculiarities in the internal economy of the avuncular life.    
\P 1922 JOYCE  \textit{Ulysses} 417 Avuncular's got my timepiece.

\vspace{0.2cm} \noindent
Hence \textbf{avuncularism} (joc.), recourse to a pawnbroker; \textbf{avuncularity}, the state of being an uncle; \textbf{avuncularly} adv., in the manner of an uncle.

\P 1859 D. G. ROSSETTI  \textit{Let.} 15 Feb. (1965) I. 348, I have only been saved from further ‘avuncularism’ by a visit of old Plint, who has bought two‥drawings.    
\P 1937 A. L. ROWSE  \textit{R. Grenville} ii. 28 The pleasures of avuncularity.    
\P 1957 \textit{Economist}  7 Sept. 824/1 The classical picture here is of Lord Woolton avuncularly presiding over the rapidly growing Young Conservatives.
\end{myenumerate}



%%%%%%%%%%%%%%%%%%%%%%%%%%%%%%%%
\myitem{awry} adv. and a.

\noindent \phonetic{(əˈraɪ)}

\noindent [f. a prep.1 + wry; cf. aright, awrong.]
\vspace{-0.3cm}

\begin{myenumerate}

\itembf{A.} adv.

\itembf{1.} Away from the straight (position or direction); to one side, obliquely; unevenly, crookedly, askew.

\P 1375 BARBOUR  \textit{Bruce} iv. 705 As thair bemys strekit air Owthir all evin, or on wry.    
\P 1490 CAXTON  \textit{Eneydos} xiv. 50 The stones of the walles appyeren alle awry sette.    
\P 1590 \textit{Pasquil's  Apol.} i. D b, The case standing as it dooth I cannot but draw my mouth awrie.    
\P 1607 DEKKER  \textit{Westw. Hoe} Wks. 1873 II. 294  They say Charing-crosse is falne downe‥but thats no such wonder, twas old, and stood awry.    
\P 1650 BULWER  \textit{Anthropomet.} xi. 115 Lest‥some crum (as we use to say) should go awry.    
\P 1714 POPE  \textit{Rape Lock} iv. 8 Not Cynthia when her manteau's pinned awry, E'er felt such rage.    
\P 1838 MARRYAT  \textit{Jac. Faithf.} ii. 9, I held my spoon awry, and soiled my clothes.

\itembf{b.} to look awry: to look askance or asquint. (Cf. the senses under these words.)

\P 1400  \textit{Rom. Rose} 291 Envy‥ne looked but awrie.    
\P 1573 G. HARVEY  \textit{Letter-bk.} (1884) 5, I passing bi him‥he hath lookd awri an other wai.    
\P 1609 ROWLANDS  \textit{Crew of Gossips} 6 When he speakes‥I'll hold my peace, and (frowning) looke awry.    
\P 1709 CHANDLER  \textit{Effort agst. Bigotry} 28 When a Church-man therefore shall in scornful Pride look awry upon‥a Dissenter.    
\P 1845 DARWIN  \textit{Voy. Nat.} x. (1852) 206 Some of our party began to squint and look awry.

\itembf{2.} fig. Out of the right course or place; in a wrong manner; improperly, erroneously, amiss.

\P 1494 FABYAN  2 To me it semyth so ferre sette a wrye In tyme of yeres.    
\P 1671 MILTON  \textit{P.R.} iv. 313 Much of the Soul they talk, but all awrie.    
\P 1850 MRS. BROWNING  \textit{Aur. Leigh} iii. 543 Those who think Awry, will scarce act straightly.

\itembf{b.} esp. in phr. to go awry, run awry, step awry, tread awry, walk awry: (of persons) to fall into error, do wrong; (of things) to turn out badly or untowardly, ‘go wrong.’

\P 1524 \textit{State Papers Hen. VIII,} I. 152 To wryng and wreste the maters in to bettre trayne, if they walke a wrye.    
\P 1570 B. GOOGE  \textit{Pop. Kingd.} iv. (1880) 56 b, The very Spouse and Church of Christ, that cannot runne awry.
\P a1625 BOYS  in Spurgeon \textit{Treas. Dav.} Ps. xv. 2 Aristides was so just‥that he would not tread awry.    
\P 1745 DE FOE  \textit{Eng. Tradesm.} I. ix. 65 If a tradesman but once ventures to step awry.    
\P 1858 CARLYLE  \textit{Fredk. Gt.} (1865) I. ii. xi. 116 Far worse, the marriage itself went awry.

\itembf{c.} to tread the shoe awry: to fall from virtue, break the law of chastity. Cf. F. faux pas.

\P 1520-41 WYATT  \textit{Poet. Wks.} (1861) 96 Farewell all my welfare! My shoe is trod awry.    
\P 1600 HEYWOOD  \textit{2nd Edw. IV,} Wks. 1874 I. 143  King Edward's children not legitimate‥Their mother hapt to tread the shoe awry.    
\P 1662 FULLER  \textit{Worthies} (1840) III. 130 He would not stick to tell where he trod his holy sandals awry.

\itembf{B.} adj. (usually pred., rarely attrib. Cf. wry.)

\itembf{1.} Out of the right course or position; displaced, disordered, disarranged; crooked, distorted.

\P 1658 W. BURTON  \textit{Itin. Anton.} 178 The journey will prove enormiously awry.    
\P 1728 YOUNG  \textit{Love Fame} vi. (1757) 149 What pity 'tis her shoulder is awry!    
\P 1847 BARHAM  \textit{Ingol. Leg.} (1877) 172 His features and phiz awry Show'd so much misery.    
\P 1883 \textit{Daily News} 9 Nov. 2/1 Blinds‥very different from the awry, dingy, imitation Venetians of his neighbour.

\itembf{2.} fig. Turned from the right course, wide of the mark, perverted, wrong. awry from: opposed to.

\P 1581 SIDNEY  \textit{Astr. \& Stella} xxvii, With dearth of words, or answers quite awrie.    
\P 1670 MILTON  \textit{Hist. Eng.} i. Wks. (1851) 23 Nothing more awry from the Law of God‥then that a Woman should give Laws to Men.    
\P 1872 BROWNING  \textit{Fifine} 1, If so succeed hand-practice on awry Preposterous art-mistake.

\itembf{C.} ellipt. quasi-v. To turn awry or aside.

\P 1613 R. C. \textit{Table Alph., Swarue,} awry, erre.    
\P 1653 BROME  \textit{Mad Couple} iii. i, High heeld shooes, that will awry sometimes with any Women.
\end{myenumerate}


%%%%%%%%%%%%%%%%%%%%%%%%%%%%%%%%
\myitem{axiom} n.

\noindent \phonetic{(ˈæksɪəm)}

\noindent [a. F. axiome, ad. L. axiōma, a. Gr. ἀξίωµα that which is thought worthy or fit, that which commends itself as self-evident, f. ἀξιόειν to hold worthy, f. ἄξιος worthy.]
\vspace{-0.3cm}

\begin{myenumerate}

\itembf{1.} A proposition that commends itself to general acceptance; a well-established or universally-conceded principle; a maxim, rule, law.

\P 1485 CAXTON  \textit{Paris \& V.} Prol., An axiom which in Latin expressed, hoc crede quod tibi verum esse videtur.    
\P 1579 LYLY  \textit{Euphues} (Arb.) 100 The Axiomaes of Aristotle.    
\P 1604 DEKKER  \textit{Honest Wh.} Wks. 1873 II. 63 That's an Axiome, a Principle.    
\P 1651 HOBBES  \textit{Govt. \& Soc.} i. §2. 3 Which Axiom, though received by most, is yet certainly false.    
\P 1757 JOHNSON  \textit{Rambl.} No. 175 \cardo{⁋}1 The axioms of wisdom which recommend the ancient sages to veneration.    
\P 1837 J. HARRIS  \textit{Gt. Teacher} 389 The axiom known by the name of the golden rule.    
\P 1875 H. E. MANNING  \textit{Mission H. Ghost} ii. 33 It is an axiom of the human reason that God is everywhere.

\itembf{b.} Specially restricted by Bacon to: An empirical law, a generalization from experience. Obs.

\P 1626 BACON  \textit{Sylva} §2 Led by great Judgement, and some good Light of Axioms.    
\P 1627 RAWLEY in \textit{Bacon's Ess.} (Arb.) Introd. 26 True Axiomes must be drawne from plaine Experience, and not from doubtful.    
\P 1838 SIR W. HAMILTON  \textit{Logic} xxvi. II. 47 Empirical rules (Bacon would call them axioms.)

\itembf{2.} Logic. A proposition (whether true or false).

\P 1588 FRAUNCE  \textit{Lawiers Log.} ii. i. 86 b, An axiom or proposition‥hath two partes, the bande, and the partes bound.    
\P 1656 STANLEY  \textit{Hist. Philos.} viii. Zeno xx. 43 Universally negative axioms are those, which consist of an universall negative particle, and a Categorem; as, no man walketh.    
\P 1664 H. MORE  \textit{Myst. Iniq. Apol.} 533 Otherwise no man might dispute or pronounce a false Axiome.    
\P 1742 IN BAILEY.

\itembf{3.} Logic and Math. ‘A self-evident proposition, requiring no formal demonstration to prove its truth, but received and assented to as soon as mentioned’ (Hutton).

\P 1600 HOOKER (J.) Axioms, or principles more general, are such as this, that the greater good is to be chosen before the lesser.    
\P 1660 R. COKE  \textit{Justice Vind.} 16.    
\P 1785 REID  \textit{Int. Powers} i. ii, Nor are they necessary truths, as mathematical axioms are.    
\P 1807 BYRON  \textit{Hours Idlen., College Exam.,} Happy the youth in Euclid's axioms tried.    
\P 1851 H. SPENCER  \textit{Soc. Stat.} ii. ix. §6 The axiom that the whole is greater than its part.
\end{myenumerate}

\end{description}




%%%%%%%%%%%%%%%%%%%%%%%%%%%%%%%%%%%%%%%%%%%%%%%%%%%
\chapter*{B}
%\markboth{VOCABULARY STUDY}{}
\markright{OED: B}{}
\addcontentsline{toc}{chapter}{OED: B}%

\begin{description}[wide, labelwidth=!, labelindent=0pt] % noindent

%%%%%%%%%%%%%%%%%%%%%%%%%%%%%%%%%
\myitem{badinage} n.

\noindent \phonetic{(badɪˈnaʒ, ˈbædɪnɪdʒ)}

\noindent [a. F. badinage, f. badiner (see below) and -age.]

\noindent
Light trifling raillery or humorous banter.

\P 1658 in PHILLIPS.    
\P 1740 CIBBER  \textit{Apol.} (1756) II. 74 The frivolous charms or playful badinage of a king's mistress.    
\P 1880 DISRAELI  \textit{Endym.} xxxvii, Men destined to the highest places should beware of badinage.



%%%%%%%%%%%%%%%%%%%%%%%%%%%%%%%%
\myitem{bailiwick} n.

\noindent \phonetic{(ˈbeɪlɪwɪk)}

\noindent [f. bailie + -wick: see also bailiffwick.]
\vspace{-0.3cm}

\begin{myenumerate}

\itembf{1. a.} A district or place under the jurisdiction of a bailie or bailiff. Used in Eng. Hist. as a general term including sheriffdom; and applied to foreign towns or districts under a vogt or bailli.

\P 1460 FORTESCUE  \textit{Abs. \& Lim. Mon.} (1714) 123 A mean Bayliff may do more in his Bayly-Weke.    
\P 1574 tr. \textit{Littleton's Tenures} 51 a, By the othe of xii true men of hys bayliwike.    
\P 1596 SPENSER  \textit{State Irel.} Wks. (1862) 553/2 The sheriffe of the shire, whose peculiar office it is to walke up and downe his bayli-wicke.    
\P 1678 T. JONES  \textit{Heart \& Right Sov.} 88 Our British Isles, which never were within the diocess or bayliwick of Rome.    
\P 1759 B. MARTIN  \textit{Nat. Hist. Eng.} II. 355 A fair Bailiwick and Town corporate.    
\P 1796 MORSE  \textit{Amer. Geog.} II. 305 Berne. This Canton contains 72 bailiwicks.    
\P 1862 ANSTED  \textit{Channel Isl.} iv. xxiii. 519 Guernsey, Alderney, and Sark, together with Herm‥composing the Bailiwick of Guernsey.    
\P 1884 \textit{Law Rep. Chanc. Div.} XXV. 341 The sheriff‥made a return‥that Mr. S. had no lay fee within his bailiwick.

\itembf{b.} transf. ‘One's natural or proper place or sphere’ (D.A.). Chiefly U.S.

\P 1843 \textit{Knickerbocker}  XXI. 589 A friend‥inside the southern division of Mason and Dixon's ‘bailiwick’.    
\P 1892 \textit{Outing}  (U.S.) Apr. 16/1 The baggage-man stared a little when we piled our ‘truck’ into his bailiwick.    
\P 1911 R. D. SAUNDERS  \textit{Col. Todhunter} ix. 119 I'm skeered to the marrow,‥because I'm out o' my bailiwick.    
\P 1940 \textit{S.P.E. Tract} lvi. 216 Bailiwick‥has given rise to the common phrase ‘out of one's bailiwick’, i.e. outside of one's natural sphere or function.

\itembf{2.} The office or jurisdiction of a bailie or a bailiff. (Now only Hist.)

\P 1494 FABYAN vii. 528 The offyce of ballywyke.
\P a1649 DRUMMOND of HAWTHORNDEN \textit{Jas. V} Wks. (1711) 88 A suit‥about the ballywick of Jedburgh-forrest.    
\P 1687 N. JOHNSTON  \textit{Assur. Abbey Lands} 69 Other Ecclesiastical Benefices, Provost-ships, Baly-wicks, Commendams, Canon-ships, etc.    
\P 1875 STUBBS  \textit{Const. Hist.} II. xvii. 557 No gift of land, franchise‥or bailiwick should be made.

\itembf{3.} Stewardship. (Cf. bailieship.) Obs.

\P 1550 CROWLEY \textit{Epigr.} 1257 Christe shall saie at the laste daye, Geve accounts of your baliwickes.    
\P 1601 DENT  \textit{Pathw. Heaven} (1603) 171 To give an account of our bailywicke.

\itembf{4.} Comb. bailiwick-town, a town under the jurisdiction of a bailiff; the chief town of a hundred.

\P 1675 OGILBY  \textit{Brit.} 172 Hexham‥is at present a well-built Bailiwick Town.    
\P 1724 DE FOE, etc. \textit{Tour Gt. Brit.} (1769) III. 241 The Bailiwick-town of Hexham.
\end{myenumerate}


%%%%%%%%%%%%%%%%%%%%%%%%%%%%%%%%
\myitem{baleful} a.

\noindent \phonetic{(ˈbeɪlfʊl)}

\noindent [OE. bealu-full, f. bealu bale n.1 + full. Until recent times almost exclusively poetic; still chiefly literary.]
\vspace{-0.3cm}

\begin{myenumerate}

\itembf{1.} Full of malign, deadly, or noxious influence; pernicious,
destructive, noxious, injurious, mischievous, malignant: \textbf{a.} physically or generally.

\P a1000 \textit{Crist}  (Grein) 259 Se bealofulla [= the devil] hyneþ heardlice.
\P c1220 \textit{St. Marher.}  10 To beoren me into his balefule hole.    
\P 1230 \textit{Ancr.} R. 114 So baluhful \& so bitter!
\P c1400 \textit{Destr. Troy} i. 167 These balfull bestes were‥ffull flaumond of fyre.    
\P 1592 SHAKES.  \textit{Rom. \& Jul.} ii. iii. 8 Balefull weedes, and precious Iuiced flowers.    
\P 1676 \textit{Black  Prince} in \textit{Harl. Misc.} (1793) 51 Great flocks of ravens, and other baleful birds of prey.    
\P 1712 SWIFT  \textit{Wond. Proph.} Wks. 1755 III. I. 173 This baleful dog-star.    
\P 1800-24 CAMPBELL  \textit{To Sir F. Burdett} v, His hate is baleful, but his love is worse.    
\P 1862 RAWLINSON  \textit{Anc. Mon.} I. i. 32 The baleful simoon sweeps across the entire tract.

\itembf{b.} morally.

\P c1175 \textit{Lamb.  Hom.} 215 Tend mine heorte and uorbern al þat is baluful þer inne.
\P c1300 \textit{Lay-Folks  Mass-Bk.} B. 404 Þat may lese alle baleful bandes.    
\P 1589 GREENE  \textit{Menaph.} (Arb.) 22 The baleful laborinth of despaire.    
\P 1597 LOK in Farr \textit{S.P.} (1845) I. 138 Through baleful lust of gold.    
\P 1751 SMOLLETT  \textit{Per. Pic.} (1779) III. lxxxi. 109 O baleful Envy! thou self-tormenting fiend.    
\P 1863 W. PHILLIPS  \textit{Speeches} xvi. 362 The potent and baleful prejudice of color.

\itembf{2.} subjectively: \textbf{a.} Full of pain or suffering, painful. Obs.

\P c1200 \textit{Trin. Coll. Hom.} 181 On þisse liue we beð on balfulle swinche for adames gulte.    
\P 1579 SPENSER  \textit{Sheph. Cal.} Jan., Such stormie stoures do breede my balefull smart.

\itembf{b.} Unhappy, wretched, miserable; distressed, sorrowful, mournful. arch.

\P 1325 \textit{E.E. Allit. P.} C. 979 Þe balleful burde [Lot's wife], þat neuer bode keped.
\P c1420 \textit{Anturs  of Arth.} xlii, The balefulle birde blenked on his blode.    
\P 1535 STEWART  \textit{Cron. Scot.} I. 124 The ȝoutting, ȝouling, and the bailfull beir Tha maid.    
\P 1596 DRAYTON  \textit{Legends} iii. 14 That Baleful sounds immovably do'st breathe.    
\P 1812 J. WILSON  \textit{Isle of Palms} i. 533 Baleful spirits barr'd from realms of bliss.
\end{myenumerate}


%%%%%%%%%%%%%%%%%%%%%%%%%%%%%%%%%
\myitem{banal} a.

\noindent \phonetic{(bəˈnɑːl, older ˈbeɪnəl)}

\noindent [a. F. banal, in Cotgr. bannal, f. ban:—med.L. bannum: see BAN n., and -AL.]

\vspace{-0.3cm}

\begin{myenumerate}
\itembf{1.} Of or belonging to compulsory feudal service.

\P 1753 CHAMBERS  \textit{Cycl. Supp.}, Bannal-Mill, a kind of feudal service, whereby the tenants of a certain district are obliged to carry their corn to be ground at a certain mill, and to be baked at a certain oven for the benefit of the lord.    
\P 1864 SIR F. PALGRAVE  \textit{Norm. \& Eng.} IV. 281 A bannal-oven of which the lord enjoyed the monopoly.

\itembf{2.} (From the intermediate sense of, Open to the use of all the community): Commonplace, common, trite; trivial, petty.

\P [1837 \textit{Athenæum}  No. 504. 453 These bannales personages are ‘much of a muchness.’]    
\P 1840 \textit{New Monthly Mag.} LIX. 458 All that her late companions can draw from her is the banal declaration, that she ‘never knew what happiness was before’.    
\P 1864 \textit{N. \& Q.}  Ser. iii. VI. 480 Facetious fools‥set up the banal laugh.    
\P 1868 BROWNING  \textit{Ring \& Bk.} x. 820 You must show the warrant, just The banal scrap, clerk's scribble.    
\P 1883 R. BURTON \& CAM. \textit{Gold Coast} I. iii. 54 Prizes were banal as medals after a modern war.
\end{myenumerate}


%%%%%%%%%%%%%%%%%%%%%%%%%%%%%%%%%
\myitem{bastion} n.

\noindent \phonetic{(ˈbæstɪən)}

\noindent [a. F. bastion, 16th c., ad. It. bastione, f. bastire to build, construct, late L. or common Romanic, of uncertain origin; generally referred to the same root as baston, baton.]
\vspace{-0.3cm}

\begin{myenumerate}

\itembf{1.} A projecting part of a fortification, consisting of an earthwork, faced with brick or stone, or of a mass of masonry, in the form of an irregular pentagon, having its base in the main line, or at an angle, of the fortification; its ‘flanks’ are the two sides which spring from the base, and are shorter than the ‘faces’ or two sides which meet in the acute ‘salient angle.’

\textbf{cut bastion}: one with its salient angle cut off and replaced by an
inward angle. 
\textbf{detached bastion}: one constructed apart from the
fortification, also called a lunette. 
\textbf{double bastion}: two bastions, one placed inside the other. 
\textbf{empty bastion}: one in which the interior surface is lower than the rampart. 
\textbf{flat bastion}: one placed in front of a ‘curtain.’ 
\textbf{Full} or \textbf{solid bastion}: one in which the interior surface is level with the rampart. 
\textbf{tower bastion}: a tower built like a bastion and provided with casemates.

\P 1598 BARRET  \textit{Theor. Warres} v. iii. 135 Baskets to cary earth to the bastion.    
\P 1693 \textit{Mem. Ct. Teckely} i. 14 This small City, flanked with five good Bastions.    
\P 1703 MAUNDRELL  \textit{Journ. Jerus.} (1732) 54 Bastions faced with hewn stone.    
\P 1812 WELLINGTON in \textit{Gurwood Disp.} IX. 27 To breach the face of Bastion at the south east angle of the fort.    
\P 1851 RUSKIN  \textit{Stones Ven.} I. v. 58 Sharp as the frontal angle of a bastion.

\itembf{2.} transf. and fig. Rampart, fortification, defence.

\P 1679 \textit{Est. Test.} 27 The frontier and Bastion of the Protestant Religion.    
\P 1781 COWPER  \textit{Convers.} 688 They build each other up‥As bastions set point-blank against God's will.    
\P 1858 LONGFELLOW  \textit{Ladder St. Aug.} ix, The distant mountains, that uprear Their solid bastions to the skies.
\end{myenumerate}


%%%%%%%%%%%%%%%%%%%%%%%%%%%%%%%%%
\myitem{bathos} n.

\noindent \phonetic{(ˈbeɪθɒs)}

\noindent [a. Gr. βάθος depth. First made Eng. in sense 2 by Pope's treatise, the title being a parody on Longinus's περὶ ὕψους; subseq. in the more etymological sense 1.]
\vspace{-0.3cm}

\begin{myenumerate}

\itembf{1.} Depth; lowest phase, bottom.

\P [1638 SANDERSON  \textit{Serm.} II. 101 There is such a height, and depth, and length, and breadth in that love; such a βάθος in every dimension of it.]    
\P 1758 JOHNSON  \textit{Idler} No. 79 \cardo{⁋}7 Declining‥to the very bathos of insipidity.    
\P 1840 MARRYAT  \textit{Olla Podr.} (Rtldg.) 276, I am at the very bathos of stupidity.

\itembf{2.} Rhet. Ludicrous descent from the elevated to the commonplace in writing or speech; anticlimax.

\P 1727 POPE  \textit{Bathos} 71 While a plain and direct road is paved to their ὕψος, or sublime; no track has been yet chalked out to arrive at our βάθος, or profund.    
\P 1787 J. ANDREWS  \textit{Anecdotes} s.v. Bathos, Had Ovid introduced this supper of Niobé between the death of her children and her own metamorphosis into stone, he would have furnished us, with a compleat instance of the Bathos.    
\P 1875 MCLAREN  \textit{Serm.} Ser. ii. xii. 211 It is as absurd bathos as to say, the essentials of a judge are integrity, learning, and an ermine robe!

\itembf{3.} Hence gen. A ‘come-down’ in one's career.

\P 1814 T. JEFFERSON  \textit{Writ.} (1830) IV. 240 How meanly has he closed his inflated career! What a sample of the bathos will his history present!    
\P 1841 MARRYAT  \textit{Poacher} xxviii, It was rather a bathos‥to sink from a gentleman's son to an under usher.
\end{myenumerate}


%%%%%%%%%%%%%%%%%%%%%%%%%%%%%%%%%
\myitem{behemoth} n.

\noindent \phonetic{(bɪˈhiːməθ, -ɔːθ)}

\noindent [Heb. b'hēmōth, used in Job xl. 15. In form the word is the plural of b'hēmāh ‘beast,’ and might be interpreted ‘great or monstrous beast’ (plural of dignity). But most moderns take it as really an Egyptian word p-ehe-mau, which would mean ‘water-ox,’ assimilated in Hebrew mouths to a Hebrew form.]
\vspace{-0.3cm}

An animal mentioned in the book of Job; probably the hippopotamus; but also used in modern literature as a general expression for one of the largest and strongest animals. Cf. LEVIATHAN.

\P 1382 WYCLIF  \textit{Job} xl. 10 Lo! bemoth [1388 behemot, 1611 behemoth] that I made with thee.    
\P 1430 LYDG.  \textit{Chron. Troy} ii. xvii, Whom the Hebrues‥call Bemoth that doth in latin playne expresse A beast rude full of cursednesse.    
\P 1667 MILTON  \textit{P.L.} vii. 471 Behemoth biggest born of earth.    
\P 1727 THOMSON  \textit{Summer} 710 The flood disparts: behold! in plaited mail, Behemoth rears his head.    
\P 1818 KEATS  \textit{Endym.} iii. 134 Skeletons of man, Of beast, behemoth, and leviathan.    
\P 1820 SHELLEY  \textit{Prometh. Unb.} iv. i. 310 The might Of earth-convulsing behemoth.    
\P 1857 EMERSON  \textit{Poems} 306 Be swift their feet as antelopes, And as behemoth strong.

\noindent fig. \P 1592 G. HARVEY  \textit{Pierces Super.}, Will soone finde the huge Behemoth of conceit to be the sprat of a pickle herring.    
\P 1850 MRS. STOWE  \textit{Uncle Tom's C.} xv. 140 He's a perfect behemoth.


%%%%%%%%%%%%%%%%%%%%%%%%%%%%%%%%%
\myitem{beleaguer} v.

\noindent \phonetic{(bɪˈliːgə(r))}

\noindent [a. Du. belegeren, f. be- + leger camp; cf. mod.G. belagern: see LEAGUER.]

\vspace{-0.3cm}

\begin{myenumerate}

\itembf{1.} To surround (a town, etc.) with troops so as to prevent ingress and egress, to invest, besiege.

\P 1590 SIR J. SMYTHE  \textit{Weapons} 4 These‥haue so affected the Wallons, Flemings, and base Almanes discipline, that‥they will not‥affoord to say that such a towne is besieged, but that it is belegard.    
\P 1598 BARRET  \textit{Theor. Warres} v. iii. 134 Antwerpe,‥then by him beleaguered.    
\P 1648 EVELYN  \textit{Mem.} (1857) III. 26 The castle of Dover, which some say is beleagured.    
\P 1846 PRESCOTT  \textit{Ferd. \& Is.} I. ix. 392 He reflected that the Castilians would soon be beleaguered.    
\P 1856 LONGFELLOW  \textit{Beleag. City} vii, That an army of phantoms vast and wan, Beleaguer the human soul.

\itembf{2.} transf. To surround, beset (generally with some idea of hostility or annoyance). Cf. besiege.

\P 1589 NASHE  \textit{Almond for P.} 5 a, A whole hoast of Pasquils‥will so beleaguer your paper walles.    
\P 1614 LODGE  \textit{Seneca} 4 Beleager him on euery side by thy bountie.    
\P 1741 RICHARDSON  \textit{Pamela} (1824) I. iv. 239 The girl is‥beleaguering, as you significantly express it, a worthy gentleman.    
\P 1822 W. IRVING  \textit{Braceb. Hall} xxvii. 253 It [the house] has been beleaguered by gipsy women.
\end{myenumerate}


%%%%%%%%%%%%%%%%%%%%%%%%%%%%%%%%%
\myitem{bellicose} a.

\noindent \phonetic{(ˌbɛlɪˈkəʊs)}

\noindent [ad. L. bellicōs-us: see -ose.]

\noindent Inclined to war or fighting; warlike.

\P 1432-50 tr. \textit{Higden} (1865) I. 321 Germanye, the peple of whom was‥bellicose.    
\P 1535 STEWART  \textit{Cron. Scot.} (1858) I. 134 Our godis aboue‥In Albione hes plantit‥The perfite pepill, bald and bellicois.    
\P 1706 MAULE  \textit{Hist. Picts} in \textit{Misc. Scot.} I. 32 The bellicose Romans.    
\P 1880 KINGLAKE  \textit{Crimea} VI. iii. 13 Their bellicose names were deceptive.


%%%%%%%%%%%%%%%%%%%%%%%%%%%%%%%%
\myitem{belligerent} a. and n.

\noindent \phonetic{(bɛˈlɪdʒərənt)}

\noindent [The earlier belligerant (cf. F. belligérant) was ad. L. belligerānt-em, pr. pple. of belligerāre to wage war: see belligerate, -ous. The current spelling, if due to imitation of L. gerentem, is etymologically erroneous, since the word is not derived from gerĕre; but cf. magnific-ent.]
\vspace{-0.3cm}

\begin{myenumerate}

\itembf{A.} adj.

\itembf{1.} Waging or carrying on regular recognized war; actually engaged in hostilities; formerly also said of warlike engines, and the like.

\P 1577 DEE  \textit{Relat. Spir.} i. (1659) 171 Four‥belligerant Castles, out of the which sounded Trumpets thrice.    
\P 1765 TUCKER  \textit{Lt. Nat.} II. 408 Religion and reason are so far from being belligerent powers‥that they join in alliance.
\P a1773 CHESTERFIELD  (T.) The belligerent and contracting parties.    
\P 1775 JOHNSON,  \textit{Belligerant}, waging war. Dict. [i.e. from some dictionary.]    
\P 1846 PRESCOTT  \textit{Ferd. \& Is.} I. iv. 213 A truce of six months between the belligerent parties.

\itembf{2.} fig. or transf. to other hostilities.

\P 1809 W. IRVING  \textit{Knickerb.} (1861) 117 He assumed a most belligerent look.    
\P 1812 \textit{Examiner}  11 May 290/2 The belligerent journalists‥are unanimously for the military.    
\P 1850 THACKERAY  \textit{Pendennis} xlvi (1884) 458 Costigan called for a ‘waither’ with such a belligerent voice.

\itembf{3.} attrib. from the n.: Of or pertaining to belligerents.

\P 1865 BRIGHT  \textit{Canada}, Sp. 13 Mar. (1876) 68 The acknowledgment of the belligerent rights of the South.    
\P 1881 J. WESTLAKE in \textit{Academy} 15 Jan. 41/2 Controversies‥concerning the capture of private belligerent property at sea.

\itembf{B.} n.

\itembf{1.} A nation, party, or person waging regular war (recognized by the law of nations).

\P 1811 \textit{Hist. Eur.} in Ann. Reg. 75/2 The common rules between civilized belligerents.    
\P 1839 HALLAM  \textit{Hist. Lit.} II. ii. iv. §86 War itself‥even for the advantage of the belligerents, had its rules.    
\P 1864  \textit{Times} 22 Dec., Deprived the blockaded Power of its rights as a maritime belligerent.

\itembf{2.} fig. or transf. to other hostile agents.

\P 1839 DICKENS  \textit{Nich. Nick.} ii, A loud shout attracted the attention of even the belligerents [i.e. policemen].    
\P 1849 MACAULAY  \textit{Hist. Eng.} xviii, Out of Parliament‥the belligerents were by no means scrupulous about the means which they employed.
\end{myenumerate}


%%%%%%%%%%%%%%%%%%%%%%%%%%%%%%%%%
\myitem{bemuse} v.

\noindent \phonetic{(bɪˈmjuːz)}

\noindent [f. BE- 2 + MUSE v.: cf. amuse.]

\noindent
trans. To make utterly confused or muddled, as with intoxicating liquor; to put
into a stupid stare, to stupefy. Hence \textbf{bemused, bemusing} ppl. a.

\P 1735 POPE  \textit{Prol. Sat.} 15 A parson much be-mus'd in beer.    
\P 1771 J. FOOT  \textit{Penseroso} iv. 196 [With] fairy tales bemused the shepherd lies.    
\P 1847 H. MILLER  \textit{First Impr.} xix. (1861) 265 The bad metaphysics with which they bemuse themselves.    
\P 1880 MCCARTHY  \textit{Own Times} xxx. III. 2 A Prussian was regarded in England as a dull beer-bemused creature.

\vspace{0.1cm} \noindent 
\textit{humorously}, To devote entirely to the Muses.

\P 1705 POPE  \textit{Let. H. Cromwell} Wks. 1735 I. 15 When those incorrigible things, Poets, are once irrecoverably Be-mus'd.


%%%%%%%%%%%%%%%%%%%%%%%%%%%%%%%%%
\myitem{bestow} v.

\noindent \phonetic{(bɪˈstəʊ)}

\noindent [ME. bistowen, f. bi-, be- 2 + stowen to place, stow.]
\vspace{-0.3cm}

\begin{myenumerate}

\itembf{1.} trans. To place, locate; to put in a position or situation, dispose of (in some place). arch.

\P c1374 CHAUCER  \textit{Troylus} i. 967 The god of love hath the bystowid In place digne unto thy worthines.    
\P 1528 MORE  \textit{Conf. agst. Trib.} iii. Wks. 228/1 As rowmes and liuinges fal voyde to bestowe them in.    
\P 1567 DRURY  \textit{Let.} in \textit{Tytler Hist. Scot.} (1864) III. 412 Bills bestowed upon the church doors.    
\P 1598 SHAKES.  \textit{Merry W.} iv. ii. 48 How should I bestow him? Shall I put him into the basket againe?    
\P 1610 J. GUILLIM  \textit{Heraldry} iii. i. (1660) 96 Under what heads each peculiar thing must be bestowed.    
\P 1713 POPE  \textit{Iliad} ix. 284 Glittering canisters‥Which round the board Menœtius' son bestow'd.    
\P 1873 BROWNING  \textit{Red Cotton Night-Cap Country} 116 The white domestic pigeon‥does mere duty by bestowing egg In authorized compartment.

\itembf{2.} To stow away; to place or deposit (anywhere) for storage, to store up. arch.

\P 1393 GOWER  \textit{Conf.} II. 84 The leed after Satorne groweth, And Jupiter the brass bestoweth.    
\P 1494 FABYAN vii. 466 Lancastre‥bestowed suche ordenaunce as the Frenshemen for haste lafte behynde.    
\P 1526 TINDALE  \textit{Luke} xii. 17, I have noo roume where to bestowe my frutes.    
\P 1590 SHAKES.  \textit{Com. Err.} i. ii. 78 In what safe place you have bestowed my money,
\P 1630 J. TAYLOR  \textit{Gt. Eater Kent} 13 His store-house, into which he would stow and bestow any thing that the house would afford.    
\P 1853 KANE  \textit{Grinnell Exp.} xxix. (1856) 247 Bestowing away my boots in a snugly-lashed bundle.

\itembf{3.} To lodge, quarter, put up; to provide with a resting- or sleeping-place. Also refl. arch.

\P 1577 HOLINSHED  \textit{Chron.} III. 813 They were all bestowed aboord in Spanish ships.    
\P 1605 SHAKES.  \textit{Macb.} iii. vi. 23 Sir, can you tell, Where he bestowes himselfe?    
\P 1665 MANLEY  \textit{Grotius' Low-C. Wars} 295 To bestow the wearied men into Garrisons.    
\P 1821 BYRON  \textit{Sardan.} iii. i. 121 See that the women are bestow'd in safety In the remote apartments.    
\P 1851 LONGFELLOW  \textit{Gold. Leg.} iv. iv, Shall the Refectorarius bestow Your horses and attendants for the night.

\itembf{b.} To bring to bed, confine. Obs. rare.

\P 1320 \textit{Sir Beves}  (Halliw.) 132 And Iosiane, Christ here be milde! In a wode was bestoude of childe.

\itembf{4.} To settle or give in marriage. Also refl. Obs.

\P c1386 CHAUCER  \textit{Reeve's T.} 61 To bystow hir hye Into som worthy blood of ancetrye.    
\P 1530 PALSGR. 452/1 He hath bestowed his doughter well.
\P c1550 CHEKE  \textit{Matt.} xxiv. 38 Eating and drinking, marijng, and bestowing yeer childern.    
\P 1600 SHAKES.  \textit{A.Y.L.} v. iv. 7 You will bestow her on Orlando heere.
\P c1670 MRS. HUTCHINSON  \textit{Mem. Col. Hutchinson} (1806) 9 Only three daughters who bestowed themselves meanly.    
\P 1714 T. ELLWOOD  \textit{Autobiog.} (1765) 100 He bestowed both his Daughters there in Marriage.

\itembf{5.} To apply, to employ (in an occupation); to devote (to, of obs.) for a specific purpose.

\P c1315 SHOREHAM 95 Thenche thou most wel bysyly, And thy wyȝt thran by-stowe.
\P c1386 CHAUCER  \textit{Wife's Prol.} 113, I wol bystowe the flour of myn age In the actes and in the fruytes of mariage.    
\P 1530 PALSGR. Introd. 2 Many‥shall also herafter bestowe theyr tyme in such lyke exercise.    
\P 1541 R. COPLAND  \textit{Guydon's Quest. Cyrurg.}, Howe to bestowe his remedyes to the body of man.    
\P 1580 BARET  \textit{Alv.} B 580 Thou haste well bestowed thy paynes.    
\P 1653 WALTON  \textit{Angler} i. 39 Bestow one day with me and my friends in hunting the Otter.    
\P 1655 FULLER  \textit{Ch. Hist.} vi. 279 These‥onely bestowed themselves in prayer.    
\P 1851 DIXON  \textit{W. Penn} xv. (1872) 125 How he intended to bestow his day.

\itembf{b.} esp. To apply money to a particular purpose; to lay out, expend, spend. Obs.

\P 1377 LANGL.  \textit{P. Pl.} B. ii. 75 In þe stories he techeth To bistowe þyn almes.    
\P 1526 TINDALE  \textit{2 Cor.} xii. 15, I will very gladly bestowe, and wilbe bestowed for youre soules.    
\P 1583 STUBBES  \textit{Anat. Abus.} 56 But nowe it is a small matter to bestowe‥a hundred pounde of one payre of Breeches. (God be mercifull unto us!)    
\P 1590 SHAKES.  \textit{2 Hen. IV} v. v. 11, I would haue bestowed the thousand pound I borrowed of you.    
\P 1611 BIBLE  \textit{Deut.} xiv. 26 Thou shalt bestow that money for whatsoeuer thy soule lusteth after.    
\P 1631 WEEVER  \textit{Anc. Fun. Mon.} 225 He bestowed much in building.

\itembf{c.} refl. To acquit oneself. Obs.

\P 1591 SHAKES.  \textit{Two Gent.} iii. i. 87. How and which way I may bestow myself to be regarded in her sun-bright eye.
\P 1600 \textit{A.Y.L.} iv. iii. 87 The boy is faire, Of femall fauour, and bestowes himselfe Like a ripe sister.    
\P 1606 SYLVESTER  \textit{Du Bartas} (1633) 320 He all assayls and him so brave bestowes, in his Fight, etc.

\itembf{6.} trans. (and absol.). To confer as a gift, present, give.

\P 1580 BARET  \textit{Alv.} B 580 To bestowe and giue his life for his country.    
\P 1583 STANYHURST  \textit{Æneis} ii. (Arb.) 45 Thee Greeks bestowing theyre presents Greekish I feare mee.    
\P 1613 SHAKES.  \textit{Hen. VIII}, iv. ii. 56 In bestowing, madam, He was most princely.    
\P 1632 BROME  \textit{Novella} ii. i, To brag of benefits one hath bestowne Doth make the best seeme lesse.    
\P 1750 JOHNSON  \textit{Rambl.} No. 38 \cardo{⁋}11 You here pray for water, and water I will bestow.    
\P 1802 M. EDGEWORTH  \textit{Moral T.} I. i. 7 The importance that wealth can bestow.    
\P 1870 BRYANT  \textit{Iliad} I. iii. 83 Whatever in their grace the gods bestow.

\itembf{b.} Const. on, upon (of obs.) a person.

\P 1535 COVERDALE  \textit{2 Chron.} xxiv. 7 All that was halowed for the house of the Lorde, haue they bestowed on Baalim.    
\P 1601 SHAKES.  \textit{Twel. N.} iii. iv. 2 How shall I feast him? What bestow of him?    
\P 1628 WITHER  \textit{Brit. Rememb.} Pref. 112 What freedomes on the Muses are bestowne.    
\P 1817 JAS. MILL  \textit{Brit. India} II. iv. v. 205 The steadiness‥of the English‥bestowed upon them a complete and brilliant victory.    
\P 1876 GREEN  \textit{Short Hist.} vi. §4 (1882) 301 He bestowed on him a pension of a hundred crowns a year.

\itembf{c.} (rarely) to or dat. pron. (Cf. 1541 in 5.)
\P 1588 SHAKES.  \textit{Tit. A.} iv. ii. 163 You must needs bestow her funerall.    
\P 1605 \textit{Lear} ii. i. 128 Bestow Your needfull counsaile to our businesses.
\end{myenumerate}


%%%%%%%%%%%%%%%%%%%%%%%%%%%%%%%%%
\myitem{bibliophile} n.

\noindent \phonetic{(ˈbɪblɪəʊfɪl)}

\noindent [a. F. bibliophile, f. BIBLIO- + Gr. ϕίλος friend.]

\noindent
A lover of books; a book-fancier; also as adj. \textbf{bibliophilic} a., of or
pertaining to a bibliophile. 
\textbf{bibliophilism} (\phonetic{bɪblɪˈɒfɪlɪz(ə)m}), the principles and practice of a bibliophile. 
\textbf{bibliophilist}, a bibliophile. 
\textbf{bibliophilistic} a., of or befitting a bibliophilist. 
\textbf{bibliophilous} (\phonetic{bɪblɪˈɒfɪləs}), a., addicted to bibliophily. 
\textbf{bibliophily} [F. bibliophilie], love of books, taste for books.

\P 1824 DIBDIN  \textit{Libr. Comp.} 780 The work‥has been reprinted by the Society of *Bibliophiles at Paris.

\P 1883 \textit{Pall Mall} G. 12 Oct. 5/1 A *bibliophil, an autograph and print collector.

\P 1883 \textit{American}  VI. 25 A *bibliophilic rarity and treasure.

\P 1824 DIBDIN  \textit{Libr. Comp.} 4 Manias which sometimes‥bring disgrace upon the good old cause of *bibliophilism.

\P 1883  \textit{Daily News} 1 Mar. 5/1 This quaint rule of *bibliophilistic morality, ‘no harm in stealing a book if he does not mean to sell it, but to keep it.’

\P 1882 STEVENSON  \textit{Men \& Bks.} 277 Odd commissions for the *bibliophilous Count.

\P 1877 SWINBURNE  \textit{Let.} 9 Oct. (1960) IV. 24, I have lately had two noble windfalls in the way of dramatic *bibliophily (if there is such a word).    
\P 1883 \textit{Athenæum}  2 June 702/2 The old reputation of France as the true home of elegant bibliophily.


%%%%%%%%%%%%%%%%%%%%%%%%%%%%%%%%%
\myitem{bibulous} a.

\noindent \phonetic{(ˈbɪbjʊləs)}

\noindent [f. L. bibul-us freely or readily drinking (f. bibĕre to drink) + -ous.]
\vspace{-0.3cm}

\begin{myenumerate}

\itembf{1.} Absorbent of moisture.

\P 1675 EVELYN  \textit{Terra} (1729) 18 If the Soil be exceeding bibulous.    
\P 1790 COWPER  \textit{Odyss.} i. 138 With bibulous sponges those Made clean the tables.    
\P 1827 FARADAY  \textit{Chem. Manip.} ii. 43 Remove the excess by bibulous paper.

\itembf{2.} Addicted to drinking or tippling.

\P 1861 THORNBURY  \textit{Turner} I. 116 The‥irregular hours of a careless bibulous age, had undermined Girtin's health.

\itembf{3.} Relating to drink.

\P 1825  \textit{Blackw. Mag.} XVII. 322 Unskilled in bibulous lore, if he knows not the value set upon the claret of Ireland.

\vspace{0.1cm} \noindent
Hence \textbf{bibulously} adv.

\P 1858 DE QUINCEY  \textit{Goldsm.} Wks. VI. 226 The arid sands that bibulously absorbed all the perennial gushings of German enthusiasm.
\end{myenumerate}


%%%%%%%%%%%%%%%%%%%%%%%%%%%%%%%%%
\myitem{blandishment} n.

\noindent \phonetic{(ˈblændɪʃmənt)}

\noindent [f. as prec. + -ment: cf. OF. blandissement.]
\vspace{-0.3cm}

\begin{myenumerate}

\itembf{1.} Gently flattering speech or action; cajolery.

\P 1591 SPENSER M. Hubberd 1274 He gan enquire‥of the Foxe, and his false blandishment.    
\P 1622 BACON  \textit{Henry VII}, Wks. (1860) 477 He‥would use strange sweetness and blandishments of words.    
\P 1711 ADDISON  \textit{Spect}. No. 128 \cardo{⁋}4 Nature has given all the Arts of Soothing and Blandishment to the Female.    
\P 1880 L. STEPHEN  \textit{Pope} iv. 96 He was not‥inaccessible to aristocratic blandishments.

\itembf{2.} fig. Attraction, allurement. concr. Anything that pleases or allures.

\P 1594 GREENE  \textit{Look. Glasse} (1861) 142 Bear hence these wretched blandishments of sin (Taking off his crown and robe).    
\P 1660 STANLEY  \textit{Hist. Philos.} (1701) 609/1 If any external blandishments happen, they increase not the chief good.    
\P 1875 J. BENNET  \textit{Winter Medit.} ii. xi. 369 His thoughts‥were ever on the blandishments of imperial Rome.
\end{myenumerate}


%%%%%%%%%%%%%%%%%%%%%%%%%%%%%%%%%
\myitem{blatant} a. (and n.)

\noindent \phonetic{(ˈbleɪtənt)}

\noindent [Apparently invented by Spenser, and used by him as an epithet of the thousand-tongued monster begotten of Cerberus and Chimæra, the ‘blatant’ or ‘blattant beast’, by which he symbolized calumny. It has been suggested that he intended it as an archaic form of bleating (of which the 16th c. Sc. was blaitand), but this seems rather remote from the sense in which he used it. The L. blatīre to babble, may also be compared. (The a was probably short with Spenser: it is now always made long.)]
\vspace{-0.3cm}

\begin{myenumerate}

\itembf{1.} In the phrase ‘blat(t)ant beast’, taken from Spenser (cf. F.Q. v. xii. 37, 41; vi. i. 7, iii. 24, ix. 2, x. 1, xii. advt., xii. 2): see above.

\P 1596 SPENSER  \textit{F.Q.} v. xii. 37 Unto themselves they [Envie and Detraction] gotten had A monster which the blatant beast men call, A dreadful feend of gods and men ydrad.    Ibid. vi. i. 7 ‘The blattant beast,’ quoth he, ‘I doe pursew.’    
\P 1602 \textit{Return fr. Parnass.} v. iv. (Arb.) 69 The Ile of Dogges, where the blattant beast doth rule and raigne.    
\P 1636 C. FITZGEFFREY  \textit{Bless. Birthd.} (1881) 128 That blatant beast So belched forth from his blaspheaming brest.
\P a1658 CLEVELAND  \textit{Gen. Poems} (1677) 60 Cub of the Blatant Beast.    
\P 1768 TUCKER  \textit{Lt. Nat.} I. 596 The blatant beast‥with his unbridled tongue.    
\P 1812 BYRON  \textit{Ch. Har.} i. xxvi. (Orig. MS.), Then burst the blatant beast [note, a figure for the mob], and roar'd, and raged.    
\P 1856 MISS MULOCH  \textit{J. Halifax} (ed. 17) 340 He was one of the most ‘blatant-beasts’ of the Reign of Terror.

\itembf{2.} fig. \textbf{a.} Of persons or their words: Noisy; offensively or vulgarly clamorous; bellowing.

\P 1656 BLOUNT  \textit{Glossogr.}, Blatant, babling, twatling.    
\P 1674 MARVELL  \textit{Reh. Transp.} ii. 371 You are a Blatant Writer and a Labrant.    
\P 1821 SOUTHEY  \textit{Vis. Judgem.} x. Wks. X. 223 Maledictions, and blatant tongues, and viperous hisses.    
\P 1872 BAGEHOT  \textit{Physics \& Pol.} (1876) 92 Up rose a blatant Radical.    
\P 1874 H. REYNOLDS  \textit{John Bapt.} viii. 515 A blatant, insolent materialism threatens to engulf moral distinctions.

\itembf{b.} Clamorous, making itself heard.

\P 1790 COWPER  \textit{Odyss.} vii. 267 Not the less Hear I the blatant appetite demand Due sustenance.    
\P 1863 GEO. ELIOT  \textit{Romola} (1880) I. ii. xxix. 359 An orator who tickled the ears of the people blatant for some unknown good.    
\P 1866 WHIPPLE  \textit{Char. \& Charac. Men} 166 All agree in a common contempt blatant or latent.    
\P 1867 J. MACGREGOR  \textit{Voy. Alone} 65 A mass of human being whose want‥misery, and filth are‥patent to the eye, and blatant to the ear.

\itembf{c.} In recent usage: obtrusive to the eye (rather than to the ear as in orig. senses); glaringly or defiantly conspicuous; palpably prominent or obvious.

\P 1889 W. S. GILBERT  \textit{Gondoliers} ii, I write letters blatant On medicines patent.    
\P 1903 G. GISSING  \textit{Private Papers H. Ryecroft} 274 The blatant upstart who builds a church, lays out his money in that way not merely to win social consideration.    
\P 1912 G. B. SHAW  \textit{Let.} 19 Aug. in Shaw \& Mrs. P. Campbell (1952) 38 You don't loathe the scenery for being prosy and mediocre in spite of its blatant picturesqueness as you do in Switzerland.    
\P 1930 SAYERS  \& EUSTACE \textit{Documents in Case} li. 246 The blatant way in which he had marked his trail‥[etc.] were actions entirely inconsistent with the carelessness of an innocent man.    
\P 1937 H. NICOLSON  \textit{Helen's Tower} ix. 191 If they were kept in the Museum‥their blatant lack of human interest had caused me to pass them by.    
\P 1942 \textit{New Statesman} 11 July 26/1 Mankind, he said, is led by half-truths or blatant lies.    
\P 1957 A. E. COPPARD  \textit{It's Me, O Lord!} v. 55 The colonel‥clad in a suit of blatant check, spats, and a monocle.    
\P 1957  \textit{Times} 19 Dec. 4/3 A blatant piece of late tackling.

\itembf{3. a.} Bleating, bellowing (or merely, loud-voiced).

\P 1791 COWPER  \textit{Iliad} xxiii 39 Many a sheep and blatant goat.    
\P 1866 J. ROSE  \textit{Ecl. \& Georg. Virg.} 69 Rooks rejoicing, and the blatant herds.

\itembf{b.} Noisily resonant, loud.

\P 1816 SCOTT  \textit{Old Mort.} xiv, A blatant noise which rose behind them.    
\P 1867 \textit{Cornh.  Mag.} Jan. 30 The vibrating and blatant powers of a hundred instruments.

\itembf{B.} as n. One who has a blatant tongue. Obs.

\P 1610 W. FOLKINGHAM  \textit{Art of Survey} Introd. Poem, Couch rabid Blatants, silence Surquedry.
\end{myenumerate}


%%%%%%%%%%%%%%%%%%%%%%%%%%%%%%%%%
\myitem{bovine} a.

\noindent \phonetic{(ˈbəʊvaɪn)}

\noindent [ad. L. bovīnus, f. bōs, bov- ox; cf. F. bovine.]

\noindent
Belonging to, or characteristic of, the ox tribe. Also ellipt. = bovine animal.

\P 1817 G. S. FABER  \textit{Eight Dissert.} (1845) I. 405 The worship of the bovine Apis.    
\P 1865 \textit{Athenæum}  No. 1969. 103/3 No wild bovine is now known in Syria.    
\P 1877 J. ALLEN  \textit{Amer. Bison} 468 Particularly bovine, also, is the satisfaction they take in rubbing themselves against trees.

\itembf{2.} fig. Inert, sluggish; dull, stupid; cf. bucolic.

\P 1855 O. W. HOLMES  \textit{Poems} 235 Where bovine rustics used to doze and dream.    
\P 1879 \textit{Contemp.  Rev.} 291 Neither in the ranks of bovine Toryism nor of rabid Radicalism.



%%%%%%%%%%%%%%%%%%%%%%%%%%%%%%%%%
\myitem{bravado} n.

\noindent \phonetic{(brəˈveɪdəʊ, -ˈvɑːdəʊ)}

\noindent [ad. Sp. bravada and F. bravade: see bravade and -ado2.]

\vspace{-0.3cm}

\begin{myenumerate}

\itembf{1.} Boastful or threatening behaviour; ostentatious display of courage or boldness; bold or daring action intended to intimidate or to express defiance; often, an assumption of courage or hardihood to conceal felt timidity, or to carry one out of a doubtful or difficult position.
   
Now usually in the singular, without a: less commonly a bravado or in pl.

\P 1599 HAKLUYT  \textit{Voy.} II. i. 287 It was not that Spanish brauado.    
\P 1626 \textit{Caussin's  Holy Crt.} 62 To sound vain $\sim$ glorious Brauado's.    
\P 1630 R. BRATHWAIT  \textit{Eng. Gentl.} (1641) 110 These Gamesters, who in a bravado will set their patrimonies at a throw.    
\P 1645 MILTON  \textit{Colast.} Wks. (1851) 362 Hee retreats with a bravado, that it deservs no answer.    
\P 1678 BUNYAN  \textit{Pilgr.} i. 128 Notwithstanding all his Bravadoes, he [Shame] promoteth the Fool, and none else.  
\P a1707 BP. PATRICK  \textit{Serm. 1 Sam.} xvii. 8 To have been done out of a bravado.    
\P 1800 WEEMS  \textit{Washington} x. (1877) 119 To hear their bravadoes, one would suppose, etc.    
\P 1816 JANE AUSTEN  \textit{Emma} ii. viii. 181 A sort of bravado—an air of affected unconcern.    
\P 1824 SCOTT  \textit{Redgauntlet} Introd., A series of idle bravadoes.    
\P 1853 ROBERTSON  \textit{Serm.} Ser. iii. xvii. 214 We may do it in bravado or in wantonness.

\itembf{b.} to make or give a bravado: to make a display in the face of the enemy, to offer battle. Obs.

\P 1600 HOLLAND  \textit{Livy} iii. lx. 128 When they made bravadoes, and challenged them to come forth and fight, not one Romane would answer them again.    
\P 1617 MORYSON  \textit{Itin.} ii. ii. ii. 164 That some foote should bee drawne out of the Campe, to give the Spaniards a brauado.    
\P 1688 \textit{Lond. Gaz.} No. 2361/3 A Party of the Moors making a Bravado.

\itembf{c.} attrib.

\P 1583 STUBBES  \textit{Anat. Abus.} ii. 50 The barbers‥haue one maner of cut called the French cut‥one of the brauado fashion.    
\P 1844 DISRAELI  \textit{Coningsby} v. iv. 204 It is a day‥of hopes and fears‥bravado bets and secret hedging.

\itembf{2.} A swaggering fellow, a hector, a bravo. Obs. [app. after Sp. masculines in -ado already used in Eng., as desperado, renegado, etc. Cf. bravo.]

\P 1653 A. WILSON  \textit{Jas.} I 28 Roaring Boys, Bravadoes, Roysters, \&c. commit many insolencies.    
\P 1668 PEPYS  \textit{Diary} 28 Feb., The Hectors \& bravadoes of the House.    
\P 1817 COLERIDGE  \textit{Biog. Lit.} II. xxi. 121 But idlers and bravadoes‥must beware.    
\P 1825 KNAPP \& BALDW. \textit{Newgate Cal.} III. 397/2 Webb‥was the greatest bravado.

\vspace{0.1cm} \noindent
Hence \textbf{bravadoism}. rare.

\P 1833  \textit{Fraser's Mag.} VIII. 527 Was‥his apparent strength and defiance, real weakness and bravadoism?
\end{myenumerate}


%%%%%%%%%%%%%%%%%%%%%%%%%%%%%%%%%
\myitem{brickbat} n.

\noindent \phonetic{(ˈbrɪkbæt)}

\noindent [See BRICK n.1 and BAT n.2]
\vspace{-0.3cm}

\begin{myenumerate}

\itembf{1. a.} A piece or fragment of a brick; properly, according to Gwilt, less than one half of its length. It is the typical ready missile, where stones are scarce. Also attrib.

\P 1563-87 FOXE  \textit{A. \& M.} III. 329 She sent a brickbat after him, and hit him on the back.    
\P 1597 S. FINCHE in \textit{Hist. Croydon App.} (1783) 153 They have filled up that trenche with‥brickbatts, and rubbushe.    
\P 1726 AMHERST  \textit{Terræ Fil.} l. 269 A very numerous mob‥assaulted the room‥with brickbats and stones.    
\P 1823 P. NICHOLSON  \textit{Pract. Build.} 355 The three-quarter brick, or brick-bat, is called a closer.    
\P 1871 DIXON  \textit{Tower} IV. xxvii. 288 Mud and brick-bats greeted the returning guards.    
\P 1890 W. JAMES  \textit{Princ. Psychol.} I. vii. 196 The continuous flow of the mental stream is sacrificed, and in its place an atomism, a brickbat plan of construction, is preached.

\itembf{b.} comb. \textbf{brickbat-cheese}.

\P 1784 J. TWAMLEY  \textit{Dairying} 59 To make brick bat Cheese‥put it into a wooden mould in the shape of a brick, press it a little, then dry it.    
\P 1861 MRS. BEETON  \textit{Bk. Househ. Management} 809 Brickbat cheese has nothing remarkable except its form.

\itembf{c.} fig. An uncomplimentary remark; adverse criticism.

\P 1642 MILTON  \textit{Apol. Smect.} (1851) 275, I beseech ye friends, ere the brick-bats flye, resolve me and yourselves, is it blasphemy‥for me to answer a slovenly wincer.    
\P 1929 \textit{Daily  Express} 7 Nov. 17/5 And now for the brickbats.    
\P 1955 [SEE  \textbf{BOUQUET} 1 b].    
\P 1966 \textit{Listener}  30 June 960/3 There were some much-needed brickbats thrown at our hero's wife.

\itembf{2.} Astr. (See quots.) colloq.

\P 1892 RANYARD  \textit{Proctor's Old \& New Astr.} 640 Clerk Maxwell used to describe the matter of the rings [of Saturn] as a shower of brickbats, amongst which there would inevitably be continual collisions taking place.    
\P 1898 A. M. CLERKE et. al. \textit{Astr.} 340 It may be that collisions are infrequent in this conglomeration of ‘brickbats’.    
\P 1926 H. C. MACPHERSON  \textit{Mod. Astr.} 78.
\end{myenumerate}


%%%%%%%%%%%%%%%%%%%%%%%%%%%%%%%%%
\myitem{bromide} Chem.

\noindent \phonetic{(ˈbrəʊmaɪd)}

\noindent [f. BROM-INE + -IDE.]
\vspace{-0.3cm}

\begin{myenumerate}

\itembf{1. a.} A primary compound of bromine with an element or organic radical. Several bromides (esp. those of ammonium, iron, and potassium) are in common medicinal use.

\P 1836 PENNY  \textit{Cycl.} V. 461/1 Carbon and Bromine form a liquid bromide of carbon.    
\P 1871 B. STEWART  \textit{Heat} §58 The same law holds good for the Bromides‥of ethyle and methyle.    
\P 1876 HARLEY  \textit{Mat. Med.} 204 Bromide of Iron acts as an energetic tonic.    
\P 1881 G. M. BEARD  \textit{Sea-Sickness} 36 The great value of the bromides in very large doses, as harmless and powerful sedatives.

\itembf{b.} familiarly for \textbf{bromide of potassium} (KBr).

\P 1883 \textit{Harper's  Mag.} Jan. 241/1 A little bromide completed the relief that put her asleep.

\itembf{c.} attrib.

\P 1886 FAGGE  \textit{Princ. Med.} II. 806 Bromide Rash.

\itembf{2.} A dose of potassium bromide taken as a sedative.

\P 1903 \textit{Smart  Set} IX. 14/1 I'll give you a bromide when you're ready for bed.

\itembf{3.} fig. A person whose thoughts and conversation are conventional and commonplace. Also, a commonplace saying, trite remark, conventionalism; a soothing statement. slang (orig. U.S.).

\P 1903 [\textit{Daily  Chron.} 9 May 4/5 Literature is resentful at being mistaken for bromide.]    
\P 1906 G. BURGESS  \textit{(title)} Are you a Bromide?    
\P 1909 W. RALEIGH  \textit{Lett.} (1926) II. 340 Bromides are dull partly because everyone pretends to understand them.    
\P 1924 R. HICHENS  \textit{After the Verdict} ii. xvii, For once Mrs. Baratrie gave way to a bromide. She said: ‘How good little Clive was!’    
\P 1925 \textit{Contemp.  Rev.} Oct. 469 There is the rise of slums which ‘ought not to be in a new country’, but which, in spite of this oft-quoted bromide, certainly existed in still earlier days.    
\P 1926 \textit{Publishers' Weekly} 20 Feb. 563 The old bromide that poetry never sells is once again proved to be wrong.    
\P 1950 \textit{Manch.  Guardian} Weekly 29 June 2/3 The Republicans would have to fall back on the old bromide about the incurable quarrelsomeness of ‘old, sick Europe’.    
\P 1961 B. FERGUSSON  \textit{Watery Maze} i. 15 These two bromides‥were quoted by the faithful‥until they were worn as thin as a Queen Victoria bun penny.

\itembf{4.} Photogr. \itembf{a.} \textbf{bromide developer}, a developer suitable for bromide paper; 
\textbf{bromide emulsion}, a gelatine emulsion impregnated with a bromide, esp. silver bromide; 
\textbf{bromide paper}, a paper coated with gelatino-bromide emulsion, used for contact printing and enlargements; 
also \textbf{bromide print, printer, printing} (of or with reference to bromide paper).

\P 1885 \textit{Amateur  Photographer} 27 Mar. 409 Britannia Bromide Paper, specially for enlargements.    
\P 1892 A. BROTHERS  \textit{Photogr.} 78 Opal glass and paper are coated with silver bromide emulsion.    Ibid., Bromide-Printing Process.    
\P 1902 \textit{Bromide  Monthly} Jan. 10 One well-known Bromide printer we know of makes his exposures in contact printing to the light of an ordinary candle from preference.    
\P 1904 GOODCHILD  \& TWENEY \textit{Technol. \& Sci. Dict.} 71/1 Bromide Prints‥are developed and fixed like dry plates.    
\P 1923 S. E. SHEPPARD in \textit{Photography} 165 Characteristic Curves for Bromide Papers.    
\P 1971 \textit{Ann. Rep. Curators Bodl. Libr. 1969-70} 46 Photography from Library material‥consisted of‥3,544 bromide prints.

\itembf{b.} A reproduction or proof on bromide paper; a bromide print.

\P 1967 F. J. M. WIJNEKUS  \textit{Elsevier's Dict. Printing} 45/2 Bromide, brief for bromide print.    
\P 1977  \textit{Economist} 5 Mar. 116 Work combining original artwork, illustrations, line or screen bromides‥and type matter.    
\P 1979  \textit{Times} 20 Nov. 4/4 Bromides, or photographic proofs, of individual reports have to be cut and pasted up in the standard way.    
\P 1983 H. EVANS  \textit{Good Times, Bad Times} ix. 182 The computer system‥was designed to translate keystrokes‥so that they emerged in the form of a photographic bromide ready for insertion.
\end{myenumerate}


%%%%%%%%%%%%%%%%%%%%%%%%%%%%%%%%
\myitem{brook} v.

\noindent \phonetic{(brʊk)}

\noindent [OE. brúcan (pa. tense bréac, brucon, pple. \phonetic{ᴁebrocen}), a Com. Teut. verb, but found in the other langs. with weak conjugation: OFris. brûka, OS. brûcan (MDu. brûken, Du. bruiken), LG. brûken, OHG. brûhhan (MHG. brûchen, Ger. brauchen), Goth. brukjan:—OTeut. stem *bruk- ‘to make use of, have the enjoyment of, enjoy’:—Aryan *bhrug-, whence also L. fru-i (:—frugv-i), fruct-us in same sense. The strong pa. tense and pple. occur in OE., but no certain instance of either is known in ME.; 16th c. Scotch has the weak brooked, brooket, bruikit.

   The phonetic history is unusual; the OE. brúcan, ME. bruken, brouke, would normally have given mod. browk; while the mod. brook, and Sc. bruik normally answer to a ME. brōken, found already, as a by-form, in Layamon.]
\vspace{-0.3cm}

\begin{myenumerate}
\itembf{1.} trans. To enjoy the use of, make use of, profit by; to use, enjoy, possess, hold. Obs. except Sc. in some legal phrases, and arch. in literature.

\P  \textit{Beowulf} 894 Þæt he beah-hordes brucan moste.
\P a1000 \textit{Wanderer}  44 (in Sweet Ags. Reader) Swa he‥giefstoles breac.
\P 1175 \textit{Lamb.  Hom.} 111 Þu ane ne brukest naut þinra welena.
\P c1205 LAY.  30308 Ne scal he nauere‥kinehelme broken [c 1275 BROUKE].
\P a1225 \textit{St. Marher},  19 Thu schalt aa buten ende bruken blisse.
\P a1300 \textit{Cursor  M.} 2589 To bruke þair heritage in pais.    Ibid. 2427 (Fairf.) Take here þi wife and brok [v.r. brouk, -e] hir wele.
\P c1440 BONE Flor. 1183 Syr Emere comawndyd every man To brooke wele the tresur that they wan.    
\P 1548 \textit{Compl.  Scot.} 86 Ihone kyng of ingland‥brukit the realme tuenty ȝeirs.    
\P 1603 JAS. I in \textit{Calderwood Hist. Kirk} 256 I, as long as I brook my life, shall maintain the same.    
\P 1637 RUTHERFORD  \textit{Lett.} cxl. (1862) I. 334 Long may He brook it!    
\P 1707 DUKE OF ATHOL in \textit{Vulpone} 21 To retain, enjoy or bruik and exerce all their Rights.    
\P 1828 SCOTT  \textit{F.M. Perth} xi, No man shall brook life after he has passed an affront on Douglas.    Mod. Sc. The langest leiver bruiks a' (= the survivor has possession of everything).

\itembf{b.} Formerly in asseverations: so (or as) brouke I my chyn, eyes, heid, etc.: so may I (or as I wish to) have the use of my eyes, etc.

\P c1175 \textit{Cott. Hom.} 233 Swa ibruce ic mine rice ne scule ȝie mine mete ibite.
\P a1300 \textit{Havelok} 311 He shal [ben] king‥So brouke I euere mi blake swire!    
\P 1384 CHAUCER  \textit{H. Fame} 273 For al-so browke I wel myn hede Ther may be vnder godelyhede Keuered many a shrewde vice.
\P c1386 \textit{Nun's Pr. T.} 480 So mot I brouke wel myn yen tway, Save ye, I herde never man so synge.
\P c1400 \textit{Gamelyn}  567 Than seyde the porter, ‘so brouke I my chyn, Ȝe schul sey your erand er ȝe comen in’.
\P c1460 \textit{Towneley  Myst.} 12 As browke I thise two shankys, It is full sore myne unthankys.    
\P 1591 \textit{Troub.  Raigne K. John} (1611) 29 Ill may I thriue, and nothing brooke with me, If shortly I present it not to thee.

\itembf{c.} to brook a name (well): to bear it appropriately, do credit to it, act consistently with it. Obs.

\P 1587 HARRISON  \textit{England} ii. v. (1877) 127 Would to God they might once brooke their name, Sans reproche.
\P a1600 \textit{Robin  Hood} (Ritson) ii. xvi. 30 ‘Simon,’ said the good wife, ‘I wish thou mayest well brook thy name’.    
\P 1622 R. HAWKINS  \textit{Voy. S. Sea} (1847) 11 Henceforth shee should be called the Daintie; which name she brooked as well for her proportion and grace, as for the many happie voyages.    
\P 1655 FULLER  \textit{Ch. Hist.} i. i. §8 And well did he brook his Name.

\itembf{2.} To make use of (food); in later usage, to digest, retain, or bear on the stomach.

\P c950 \textit{Lindisf. Gosp.} John iv. 32 Ic mett hafo to bruccanne ðone \phonetic{ᴁie} ne uutton.
\P a1000 ÆLFRIC  \textit{Gen.} iii. 19 On swate ðines andwlitan ðu bricst ðines hlafes.
\P c1175 \textit{Cott. Hom.} 221 Ælra þara þing þe on paradis beoð þu most bruce.
\P c1440 \textit{Promp. Parv.} 53 Brooke mete or drynke‥retineo vel digerendo retinere.    
\P 1540 T. RAYNALDE  \textit{Byrth Man} ii. ix. (1634) 142 If she refuse or cannot brooke meat.    
\P 1561 HOLLYBUSH  \textit{Hom. Apoth.} 32 Geue him a good draught of ye same‥as hote as he can brouke it.    
\P 1598 W. PHILLIP  \textit{Linschoten's Trav. Ind.} in Arb. \textit{Garner} III. 26 So fat that men can hardly brook them.

\itembf{b.} absol. Obs.

\P 1473 MARG. PASTON  \textit{Lett.} III. 79 Water of mynte‥were good for my cosyn to drynke for to make hym to browke.

\itembf{c.} fig. To digest mentally.

\P 1548 HALL  \textit{Chron.} (1809) 178 After the letter twise redde \& wisely brooked.

\itembf{3.} To put up with, bear with, endure, tolerate [a fig. sense of ‘to stomach’ in 2]. Now only in negative or preclusive constructions.

\P 1530 PALSGR. 471/2 He can nat brooke me of all men.    
\P 1583 STUBBES  \textit{Anat. Abus.} ii. 30 They cannot at any hand brooke or digest them that would counsel them to that.    
\P 1624 CAPT. SMITH  \textit{Virginia} iv. 115, I would deter such from comming here, that cannot well brooke labour.    
\P 1667 MILTON  \textit{P.L.} vi. 274 Heav'n‥Brooks not the works of violence and War.    
\P 1752 YOUNG  \textit{Brothers} ii. i, Such insults are not brook'd by royal minds.
\P c1815 JANE AUSTEN  \textit{Northang. Abb.} (1833) II. xv. 208 The General could ill brook the opposition of his son.
\P c1854 STANLEY  \textit{Sinai \& Pal.} v. (1858) 230 That haughty spirit that could brook no equal or superior.

\itembf{b.} intr. To put up with. Obs.

\P 1658 A. FOX tr. \textit{Wurtz' Surg.} ii. i. 49 The Wound cannot brook with the Medicine.

\itembf{c.} To find it agreeable to do something. Obs.

\P 1604 E. HAKE  \textit{No Gold, No G.} in Farr \textit{S.P.} (1848) 256 Few men brooke To helpe a man that is in need.

\itembf{4.} to brook up. [perh. a different word.] Obs.

\P 1691 RAY  \textit{S. \& E.C. Wds.} 91 To brook up, spoken of Clouds; when they draw together and threaten rain.    [Also 1721 IN Bailey.]

\vspace{0.1cm} \noindent
Here probably an error for busked.

\P 1300 \textit{Cursor  M.} 25282 Þe bodi has nede of bath to bruked be wid mete and clath.
\end{myenumerate}


%%%%%%%%%%%%%%%%%%%%%%%%%%%%%%%%%
\myitem{brouhaha} n.

\noindent \phonetic{(ˈbruːhɑːhɑː)}

\noindent [a. F. brouhaha (15th c. in Littré).]

\noindent
A commotion, a to-do, a ‘sensation’; hubbub, uproar.

\P 1890 O. W. HOLMES  \textit{Over Teacups} v. 94, I enjoy the brouhaha‥of all this quarrelsome menagerie of noise-making machines.    
\P 1931 C. MORLEY  \textit{John Mistletoe} ii. 95 He was immediately captivated by the jargon and brouhaha of the sales department.    
\P 1937 WYNDHAM  Lewis in L. Russell \textit{Press Gang!} 276 The peculiar esoteric brouhaha of the New York underworld.    
\P 1946 ‘BRAHMS’  \& SIMON \textit{Trottie True} vii. 186, I shall never forget the brou-ha-ha‥when Cousin Geraldine married into Trade.    
\P 1964  \textit{Times} 24 July 16/2 Whenever there is a City brouhaha of this kind, all sorts of voices are raised for all kinds of official inquiries.


%%%%%%%%%%%%%%%%%%%%%%%%%%%%%%%%%
\myitem{brusque} a.

\noindent \phonetic{(brʌsk, brʊsk)}

\noindent [a. F. brusque, according to Littré, etc., adapted in 16th c. from Italian brusco ‘soure, tarte, eagre, briske, vnripe; also soure- or grim-looking’ (Florio); cf. Sp. and Pg. brusco ‘rude, peevish, ill-tempered, roughly hasty’. The ulterior history is uncertain: one conjecture refers it to the Celtic words mentioned under brisk, which is hardly likely, if the Romanic word appeared first in Italian. See Diez and Littré. Commonly spelt brusk in the 17th c., but now usually spelt and often pronounced as French. (Cf. also brussly.)]
\vspace{-0.3cm}

\begin{myenumerate}

\itembf{1.} Tart. (= It. brusco.) Obs.

\P 1601 HOLLAND  \textit{Pliny} II. 152 The thin and bruske harsh wine nourisheth the body lesse.    [
\P 1752 LADY  M. W. MONTAGUE \textit{Lett.} lxxvi. IV. 23 A sort of wine they call brusco.]

\itembf{2.} Somewhat rough or rude in manner; blunt, ‘offhand’.

\P 1651 \textit{Reliq.  Wotton.} (1685) 582 The Scotish Gentlemen‥lately sent to that King, found‥but a brusk welcome.    
\P 1757 H. WALPOLE  \textit{Corr.} (1837) I. 370 This sounds brusque, but I will explain it.    
\P 1826 DISRAELI  \textit{Viv. Grey} ii. xv. 80 Yes, lively enough, but I wish her manner was less brusque.    
\P 1870 \textit{Lothair} xlvi. 243 He was brusk, ungracious, scowling, and silent.    
\P 1879 MCCARTHY  \textit{Own Times} II. xxii. 123 His blunt, brusque ways of speaking and writing.
\end{myenumerate}


%%%%%%%%%%%%%%%%%%%%%%%%%%%%%%%%%
\myitem{bucolic} a. and n.

\noindent \phonetic{(bjuːˈkɒlɪk)}

\noindent [ad. L. būcolic-us, a. Gr. βουκολικ-ός, f. βουκόλος herdsman.]
\vspace{-0.3cm}

\begin{myenumerate}

\itembf{A.} adj.

\itembf{1.} Of or pertaining to herdsmen or shepherds; pastoral.

\P 1613 R. C. \textit{Table Alph.} (ed. 3) Bucolike, pertaining to beasts or heardsmen.    
\P 1750 JOHNSON  \textit{Rambl.} No. 37 \cardo{⁋}10 The Pollio of Virgil‥is a composition truly bucolick.    
\P 1803 SYD.  SMITH \textit{Wks.} (1867) I. 50 He goes on, mingling bucolic details and sentimental effusions.    
\P 1863 M. HOWITT tr. \textit{F. Bremer's Greece} II. xvii. 167 The shepherds and shepherdesses‥milk the cattle, and compose bucolic poems.    
\P 1873 SYMONDS  \textit{Grk. Poets} x. 308 Bucolic poetry.

\itembf{2.} Pertaining to country life; rural, rustic, countryfied. (Somewhat humorous.)

\P 1846 LYTTON  \textit{Lucretia} (1853) 247 The second [partner] had a bucolic turn.    
\P 1859 GEO. ELIOT  \textit{A. Bede} 67 The keenest of bucolic minds felt a whispering awe at the sight of the gentry.    
\P 1875 A. R. HOPE  \textit{Schoolboy Fr.} 308 A sturdy-looking bucolic individual.    
\P 1878 M. E. HERBERT  \textit{Hübner's Ramble} ii. xii. 212 In its happy, bucolic isolation.

\itembf{3.} bucolic cæsura, a cæsura after the fourth foot in a dactylic hexameter.

\P 1887 G. M. HOPKINS  \textit{Let.} 20 Feb. (1938) 130 The ‘bucolic caesura’ (between fourth and fifth foot, pause or no pause in sense).    Ibid., The rarity of spondees before the bucolic caesura.    
\P 1957 \textit{Encycl.  Brit.} XXII. 56/1 A feature in his [sc. Theocritus'] versification‥is the so-called bucolic caesura. The rule is that, if there is a pause at the end of the fourth foot, this foot must be a dactyl.

\itembf{B.} n. [cf. L. Būcolica, Gr. βουκολικά in same use.]

\itembf{1.} pl. Pastoral poems: rarely sing. a single poem.

\P 1531 ELYOT  \textit{Gov.} i. x. (1883) I. 62 What thinge can be more familiar than his [Virgil's] bucolikes.
\P a1560 ROLLAND  \textit{Crt. Venus} iii. 103 His Georgiks and Bucolikis.    
\P 1656 BLOUNT  \textit{Glossogr., Bucolicks,} pastoral songs, or songs of Heardsmen.    
\P 1870  \textit{Daily News} 16 Apr., The manufacture of maple sugar, of which I may sing you a bucolic when the season arrives.

\itembf{2.} = Bucolic poet.

\P 1774 T. WARTON  \textit{Hist. Eng. Poetry} xxxix. III. 59 Spenser, who is erroneously ranked as our earliest English bucolic.

\itembf{3.} A rustic, peasant. (humorous.)

\P 1862 \textit{Sat. Rev.} No. 351. 72/1 It is a satisfaction to make the personal acquaintance of so worthy a bucolic.

\itembf{4.} pl. Agricultural pursuits. rare.

\P 1865  \textit{Times} 15 Apr., A fancy farm steading‥for any special branch of bucolics that may most delight the proprietor.
\end{myenumerate}


%%%%%%%%%%%%%%%%%%%%%%%%%%%%%%%%
\myitem{burgeon} v.

\noindent \phonetic{(ˈbɜːdʒən)}

\noindent [f. prec. n. Cf. F. bourgeonner.]

\vspace{-0.3cm}

\begin{myenumerate}

\itembf{1.} intr. To bud or sprout; to begin to grow.

\P c1325 \textit{E.E. Allit. P.} B. 1042 Þay borgounez \& beres blomez ful fayre.    
\P 1382 WYCLIF  \textit{Numb.} xvii. 8 The ȝerde of Aaron‥hadde buriowned.    
\P 1483 CAXTON  \textit{Gold. Leg.} 391/3 To burgene and brynge forth fruyte more plenteously.    
\P 1584 PEELE  \textit{Arraignm. Paris} i. iii. (1829) 10 The watery flowers burgen all in ranks.    
\P 1650 BP. HALL  \textit{Balm. Gil.} 79 When the Sun returnes‥ it burgens out afresh.    
\P 1721 BAILEY,  \textit{Burgeon,} to grow big about or gross, to bud forth.    
\P 1775 ASH,  \textit{Burgein, Burgeon} (v. intr. obsolete).    
\P 1810 SCOTT  \textit{Lady of L.} ii. xix, Earth lend it sap anew, Gaily to bourgeon, and broadly to grow.    
\P 1814 CARY  \textit{Dante} (Chandos) 209 Our plants then burgein.    
\P 1850 TENNYSON  \textit{In Mem.} cxv. 2. Now fades the last long streak of snow,/ Now burgeons every maze of quick/ About the flowering squares, and thick/ By ashen roots the violets blow.  

\itembf{b.} transf. Of the limbs or appendages of animals. Formerly also of animals and diseases.

\P 1382 WYCLIF  \textit{Lev.} xiii. 29 Man or womman, in whos heed or beerde boriouneth a lepre.    
\P 1536 BELLENDEN  \textit{Cron. Scot.} (1821) II. 326 Thir eddaris‥burgeon with mair plentuous nowmer than evir was sene.    
\P 1566 W. ADLINGTON  \textit{Apuleius} 31, I perceaved a plume feathers did burgen out.    
\P 1774 GOLDSMITH  \textit{Nat. Hist.} (1862) II. i. ii. 380 Two small feet are seen beginning to bourgeon near the tail.    
\P 1827 SCOTT  \textit{Napoleon} (1835) II. 390 A hydra whose heads bourgeoned‥as fast as they were cut off.

\itembf{c.} fig. To bud, burst forth; to grow, flourish.

\P 1382 WYCLIF  \textit{Prov.} xiv. 11 The tabernaclis of riȝtwis men shal burioune.    
\P 1531 ELYOT  \textit{Gov.} i. xiii. (1883) I. 132 Learning‥sowen in a childe‥springeth and burgeneth.    
\P 1641 MILTON  \textit{Animadv.} (1851) 195 The Prelatism of Episcopacy‥began then to burgeon.    
\P 1848 KINGSLEY  \textit{Saint's Trag.} iii. i. 33 Beneath whose fragrant dews all tender thoughts Might bud and burgeon.

\itembf{2.} trans. To shoot out, put forth as buds. Also with out, forth. Also transf. and fig.

\P 1382 WYCLIF  \textit{Gen.} iii. 18 It shal buriown to thee thornes and brembles.
\P c1400 \textit{Beryn}  692 The busshis burgyn out blosomis, \& flouris.    
\P 1596 LODGE  \textit{Marg. Amer.} 22 Love‥had newe burgend his wings.
\P c1820 SURTEES in Taylor \textit{Life} (1852) 288 This goodly graft‥bourgeon'd forth its flowers and leaf.
\end{myenumerate}

\end{description}



%%%%%%%%%%%%%%%%%%%%%%%%%%%%%%%%%%%%%%%%%%%%%%%%%%%
\chapter*{C}
%\markboth{VOCABULARY STUDY}{}
\markright{OED: C}{}
\addcontentsline{toc}{chapter}{OED: C}%

\begin{description}[wide, labelwidth=!, labelindent=0pt] % noindent

%%%%%%%%%%%%%%%%%%%%%%%%%%%%%%%%
\myitem{cabal} n.

\noindent \phonetic{(kəˈbæl)}

\noindent [a. F. cabale (16th c. in Littré), used in all the English senses, ad. med.L. cab(b)ala (It., Sp., Pg. cabala), cabbala, q.v. In 17th c. at first pronounced cabal (whence the abridged cab n.5); the current pronunciation was evidently reintroduced from Fr., perh. with sense 5 or 6.]
\vspace{-0.3cm}

\begin{myenumerate}

\itembf{1.} = CABBALA 1: The Jewish tradition as to the interpretation of the Old Testament. Obs.

\P 1616 BULLOKAR,  \textit{Cabal}, the tradition of the Jewes doctrine of religion.    
\P 1660 HOWELL  \textit{Lex. Tetragl.}, Words do involve the deepest Mysteries, By them the Jew into his Caball pries.    
\P 1663 BUTLER  \textit{Hud.} i. i. 530 For Mystick Learning, wondrous able In Magick, Talisman, and Cabal.

\itembf{2.} = CABBALA 2: \itembf{a.} Any tradition or special private interpretation. \itembf{b.} A secret. Obs.

\P a1637 B. JONSON  (O.) The measuring of the temple, a cabal found out but lately.    
\P 1635 D. PERSON  \textit{Varieties} I. Introd. 3 An insight in the Cabals and secrets of Nature.    
\P 1660-3 J. SPENCER \textit{Prodigies} (1665) 344 If the truth‥had been still reserved as a Cabbal amongst men.    
\P 1663 J. HEATH  \textit{Flagellum or O. Cromwell} 192 How the whole mystery and cabal of this business was managed by the‥Committee.
\P a1763 SHENSTONE  \textit{Ess.} 220 To suppose that He will regulate His government according to the cabals of human wisdom.

\itembf{3.} A secret or private intrigue of a sinister character formed by a small body of persons; ‘something less than conspiracy’ (J.).

\P 1663 J. HEATH  \textit{Flagellum or O. Cromwell}, He was no sooner rid of the danger of this but he was puzzled with Lambert's cabal.    
\P 1707 FREIND  \textit{Peterboro's Cond. Sp.} 171 The contrivances and cabals of others have too often prevail'd.    
\P 1824 W. IRVING  \textit{T. Trav.} II. 30 There were cabals breaking out in the company.    
\P 1876 BANCROFT  \textit{Hist. U.S.} VI. xlvi. 299 The cabal against Washington found supporters exclusively in the north.

\itembf{b.} as a species of action; = caballing.

\P 1734 tr. \textit{Rollin's Anc. Hist.} (1827) III. 22 To advance themselves‥by cabal, treachery and violence.    
\P 1791 BURKE  \textit{Th. on Fr. Affairs} VII. 74 Centres of cabal.    
\P 1876 BANCROFT  \textit{Hist. U.S.} III. 261 Restless activity and the arts of cabal.

\itembf{4.} A secret or private meeting, esp. of intriguers or of a faction. arch. or Obs.

\P 1649 BP. GUTHRIE  \textit{Mem.} (1702) 23 The Supplicants‥met again at their several Caballs.    1656-7 Cromwell in Burton Diary (1828) I. 382 He had never been at any cabal about the same.    
\P 1715 BENTLEY  \textit{Serm.} x. 356 A mercenary conclave and nocturnal Cabal of Cardinals.    
\P 1738 WARBURTON  \textit{Div. Legat.} I. 169 Celebrate the Mysteries in a private Cabal.    
\P 1822 W. IRVING  \textit{Braceb. Hall} iii. 23 To tell the anecdote‥at those little cabals, that will occasionally take place among the most orderly servants.

\itembf{b.} phrase. in cabal. arch. or Obs.

\P 1678 MARVELL  \textit{Poems} Wks. I. Pref. 8 Is he in caball in his cabinett sett.    
\P 1725 DE FOE  \textit{Voy. round World} (1840) 28 The gunner and second mate were in a close cabal together.    
\P 1807 CRABBE  \textit{Par. Reg.} i. (1810) 55 Here, in cabal, a disputatious crew Each evening meet.

\itembf{5.} A small body of persons engaged in secret or private machination or intrigue; a junto, clique, côterie, party, faction.

\P 1660 \textit{Trial  Regic.} 175 You were‥of the cabal.    
\P 1670 MARVELL  \textit{Corr.} cxlvii. Wks. 1872-5 II. 326 The governing cabal are Buckingham, Lauderdale, Ashly, Orery, and Trevor. Not but the other cabal [Arlington, Clifford, and their party] too have seemingly sometimes their turn.    
\P 1732 BERKELEY  \textit{Alciphr.} v. §21 A gentleman who has been idle at college, and kept idle company, will judge a whole university by his own cabal.    
\P 1767 G. CANNING  \textit{Poet. Wks.} (1827) 56 Should Fat Jack and his Cabal Cry ‘Rob us the Exchequer, Hal!’    
\P 1859 GULLICK  \& Timbs \textit{Paint.} 183 In Naples, where a cabal of artists was formed.

\itembf{6.} Applied in the reign of Charles II to the small committee or junto of the Privy Council, otherwise called the ‘Committee for Foreign Affairs’, which had the chief management of the course of government, and was the precursor of the modern cabinet.

\P 1665 PEPYS  \textit{Diary} 14 Oct., It being read before the King, Duke, and the Caball, with complete applause.    
\P 1667 \textit{Ibid.} 31 Mar., Walked to my Lord Treasurer's, where the King, Duke of York, and the Cabal, and much company withal.    
\P 1667 \textit{Ibid.} (1877) V. 128 The Cabal at present, being as he says the King, and the Duke of Buckingham, and Lord Keeper, the Duke of Albemarle and privy seale.

\itembf{b.} in Hist. applied spec. to the five ministers of Charles II, who signed the Treaty of Alliance with France for war against Holland in 1672: these were Clifford, Arlington, Buckingham, Ashley (Earl of Shaftesbury), and Lauderdale, the initials of whose names thus arranged chanced to spell the word cabal.

This was merely a witticism referring to sense 6; in point of fact these five men did not constitute the whole ‘Cabal’, or Committee for Foreign Affairs; nor were they so closely united in policy as to constitute a ‘cabal’ in sense 5, where quot. 
1670 shows that three of them belonged to one ‘cabal’ or clique, and two to another. The name seems to have been first given to the five ministers in the pamphlet of
1673 ‘England's  Appeal from the private Cabal at White-hall to the Great Council of the nation‥by a true lover of his country.’ Modern historians often write loosely of the Buckingham-Arlington administration from the fall of Clarendon in
1667 to 1673 as the  \textit{Cabal Cabinet} or \textit{Cabal Ministry}.

\P 1673 \textit{England's  Appeal} 18 The safest way not to wrong neither the cabal nor the truth is to take a short survey of the carriage of the chief promoters of this war.    
\P 1689 \textit{Mem. God's 29 Years Wonders} §25. 72 The great Ahitophel, the chiefest head-piece‥of all the Cabal.    
\P 1715 BURNET  \textit{Own Time} (1766) I. 430 This junta‥being called the cabal, it was observed that cabal proved a technical word, every letter in it being the first letter of those five, Clifford, Ashley, Buckingham, Arlington and Lauderdale.
\P a1734 NORTH  \textit{Exam.} iii. vi. \cardo{⁋}41. 453 The‥Promoters of Popery, supposed to rise by the Misfortunes of the Earl of Clarendon, were the famous CABAL.    
\P 1762 HUME  \textit{Hist. Eng.} (1806) V. lxix. 163 When the Cabal entered into the mysterious alliance with France.    
\P 1848 MACAULAY  \textit{Hist. Eng.} (1864) I. 101 It happened by a whimsical coincidence that, in 1671, the Cabinet consisted of five persons the initial letters of whose names made up the word Cabal‥These ministers were therefore emphatically called the Cabal; and they soon made that appellation so infamous that it has never since their time been used except as a term of reproach.

\itembf{7.} attrib. or in obvious comb.

\P 1673 R. LEIGH  \textit{Transp. Reh.} 36 By this time, the Politick Cabal-men were most of 'um set.    
\P 1674 R. LAW  \textit{Mem.} (1818) 61 The parliament was jealous of their caball lords.    
\P 1678 \textit{Trans Crt. Spain} 189 They maintain themselves only by a Cabal-genius, without any foundation of justice or fidelity.    
\P 1700 CONGREVE  \textit{Way of W.} i. i, Last night was one of their cabal nights.    
\P 1871 W. CHRISTIE  \textit{Life Shaftesbury} II. xii. 81 The heavy indictment of History against the so-called Cabal Ministry.
\end{myenumerate}


%%%%%%%%%%%%%%%%%%%%%%%%%%%%%%%%%
\myitem{cachet} n.

\noindent \phonetic{(kaʃɛ)}

\noindent [Fr.; f. cacher to conceal: in 18th c. treated as English.]
\vspace{-0.3cm}

\begin{myenumerate}

\itembf{1.} A seal. \textit{letter of cachet} (F. lettre de cachet): a letter under the private seal of the French king, containing an order, often of exile or imprisonment.

\P 1639 SPOTTISWOOD  \textit{Hist. Ch. Scotl.} iv. (1677) 193 She had appointed, in stead of his hand, a Cachet to be used in the signing of Letters.    
\P 1754 ERSKINE  \textit{Princ. Sc. Law} (1809) 177 On the accession of James VI. to the crown of England, a catchet or seal was made, having the King's name engraved on it, with which all signatures were to be afterwards sealed.    
\P 1753 \textit{Scots  Mag.} XV. 62/2 He obtained a letter of cachet.

\itembf{2.} fig. Stamp, distinguishing mark, ‘sign manual’.

\P 1840 THACKERAY  \textit{Paris Sk.-bk.} (1885) 69 All his works [pictures] have a grand cachet: he never did anything mean.    
\P 1882 C. PEBODY  \textit{Eng. Journalism} xxii. 176 The journal in which the cachet of fashionable life is to be distinguished.

\itembf{3.} attrib. Done under letter of cachet; privy, secret.

\P 1837  \textit{Fraser's Mag.} XVI. 293 Abominators of all close, cachet, muffled‥proceedings.

\itembf{4.} A covering of paste, gelatine, or other digestible material, enclosing (nauseous) medicine; = capsule 5.

\P 1884 \textit{Pharmac.  Jrnl.} XV. 42/2 Cachets are‥sheets of unleavened bread cut to a round or oval shape with a‥concave towards the centre,‥intended to receive the powder to be taken.    
\P 1898 Q. HOGG  in Ethel M. Hogg \textit{Biography} (1904) 349 My experience and cachets were of use to him.    
\P 1901 \textit{Contemp.  Rev.} Mar. 405 One cachet‥to be taken with the midday meal and one in the evening.

\itembf{4.} Prestige, high status; the quality of being respected or admired.

\P 1882 \textit{Daily  Advocate} (Newark, Ohio) 27 Apr. 3/1 The Dorsey levite‥is very stylish; it is difficult to make and still more difficult to wear, and will consequently retain its cachet and not become common.    
\P 1900 A. BLUNT  \textit{Jrnl.} 8 May (1986) x. 280 Certainly there is a ‘cachet’ about the Abbas Pasha descent even though the other 6 [horses] are about as highly bred as possible.    
\P 1952 H. WOUK  \textit{Caine Mutiny} i. ii. 15 Then it became a mere racial quirk of a lower social group, and lost its cachet.    
\P 1988 J. BURCHILL  \textit{Sex \& Sensibility} (1992) 55 And there is a certain cachet in not telling.    
\P 2006 \textit{Direct} Feb. 30/4 Porche's cachet makes its showrooms ‘destination dealerships’.
\end{myenumerate}


%%%%%%%%%%%%%%%%%%%%%%%%%%%%%%%%%
\myitem{cacophony} n.

\noindent \phonetic{(kæˈkɒfənɪ)}

\noindent [a. F. cacophonie, in 16th c. cacofonie, ad. (through mod.L.) Gr. κακοϕωνία, f. κακόϕωνος; see above. Formerly used in latinized form cacophonia.]
\vspace{-0.3cm}

\begin{myenumerate}

\itembf{1.} The quality of having an ill sound; the use of harsh-sounding words or phrases. (The opposite of euphony.)

\P 1656 BLOUNT  \textit{Glossogr.}, Cacophony, an ill, harsh, or unpleasing sound, (in words) a vitious utterance or pronunciation.    
\P 1733 SWIFT  \textit{Let.} lxvi. Wks. 1761 VIII. 154 Alter rhymes, and grammar, and triplets, and cacophonies of all kinds.
\P a1745 \textit{Wks.} (1841) II. 419 To allow for the usual accidents of corruption, or the avoiding a cacophonia.    
\P 1753 \textit{Chesterf.  Lett.} cclxvii, Avoid cacophony, and make your periods as harmonious as you can.    1847-8 De Quincey Protestantism Wks. VIII. 140 My labours in the evasion of cacophony.

\itembf{2.} Music. A discordant combination of sounds, dissonance. Also fig. Moral discord.

\P 1789 BURNEY  \textit{Hist Mus.} (ed. 2) I. viii. 133 What a cacophony would a complete chord occasion!    
\P 1831 MACAULAY  \textit{Let.} in Trevelyan \textit{Life \& Lett.} (1876) I. iv. 223 The oppressive privileges which had depressed industry would be a horrible cacophony.    
\P 1880 MADAME  A. GODDARD in \textit{Girl's Own Paper} 13 Mar. 166 The continual holding down of the loud pedal produces unutterable cacophony.

\itembf{3.} Med. Old term for a harsh, grating, or discordant state of the voice (Mayne Exp. Lex.).
\end{myenumerate}


%%%%%%%%%%%%%%%%%%%%%%%%%%%%%%%%%
\myitem{cadaverous} a.

\noindent \phonetic{(kəˈdævərəs)}

\noindent [ad. F. cadavéreux, -euse, ad. L. cadāverōs-us corpse-like, f. cadāver: see above.]
\vspace{-0.3cm}

\begin{myenumerate}

\itembf{a.} Of or belonging to a corpse; such as characterizes a corpse, corpse-like.

\P 1627 FELTHAM  \textit{Resolves} ii. xxxiv, A cadauerous man, composed of Diseases and Complaints.    
\P 1643 SIR T. BROWNE  \textit{Relig. Med.} i. (1656) §38 By continuall sight of Anatomies, Skeletons, or Cadaverous reliques.    
\P 1651 BIGGS  \textit{New Disp.} §26 Cadaverous dissection of bodies.    
\P 1713 DERHAM  \textit{Phys.-Theol.} iv. xi. 205 Some cadaverous smell those Ravens discover in the Air.    
\P 1776 WITHERING  \textit{Bot. Arrangem.} (1796) IV. 374 Cadaverous smell of the Phallus impudicus.    
\P 1855 BAIN  \textit{Senses \& Int.} ii. ii. §11 (1864) 172 The cadaverous odour is of the repulsive kind.    
\P 1848 DICKENS  \textit{Dombey} 36 The strange, unusual‥smell, and the cadaverous light.

\itembf{b.} esp. Of corpse-like or deadly pallor.

\P 1662 FULLER  \textit{Worthies} iii. 67 His eye was excellent at the instant discovery of a cadaverous face‥this made him at the first sight of sick Prince Henry, to get himself out of sight.
\P a1713 T. ELLWOOD  \textit{Life} 246 He found John Milton sitting in an Elbow Chair‥pale, but not cadaverous.    
\P 1820 W. IRVING  \textit{Sk. Bk.} II. 145 He has a cadaverous countenance, full of cavities and projections.    
\P 1835 WILLIS  \textit{Pencillings} I. vi. 38.
\end{myenumerate}


%%%%%%%%%%%%%%%%%%%%%%%%%%%%%%%%
\myitem{cadge} v.

\noindent \phonetic{(kædʒ)}

\noindent [Derivation and original meaning uncertain: in some early passages it varies with cache, cacche catch, of which in branch I it may be a variant: cf. the pairs botch, bodge; grutch, grudge; smutch, smudge. Branch II may also be connected with catch or ONF. cacher in other senses; but it may be a distinct word: the whole subject is only one of more or less probable conjecture. Connexion of ME. caggen with cage n. is phonetically impossible.]
\vspace{-0.3cm}

\begin{myenumerate}

\itembf{I.} Early senses.

\itembf{1.} trans. ? To fasten, tie: cf. cadgel v. (The early passages are obscure, and for one or other the senses drive, toss, shake, draw, have been proposed.) Obs.

\P c1325  \textit{E.E. Allit.} P. A. 511 For a pene on a day \& forth þay [labourers in the vineyard] gotz‥Keruen \& caggen \& man [= maken] hit clos.    Ibid. B. 1254 þay  wer cagged and kaȝt on capeles al bare.
\P a1400 \textit{Alexander} 1521 And þen he caggis [v.r. cachez] vp on cordis as curteyns it were.
\P 1400 \textit{Destr.  Troy} 3703 Hit sundrit þere sailes \& þere sad ropis; Cut of þere cables were caget to gedur.    
\P 1627 DRAYTON  \textit{Agincourt} 180 Whilst they are cadg'd contending whether can Conquer, the Asse some cry, some cry the man.    
\P 1875 \textit{Lanc.  Gloss.} (E.D.S.) Cadge, to tie or bind a thing.

\itembf{2.} To ‘bind’ the edge of a garment. Cf. cadging vbl. n. I. Obs.

\P 1530 PALSGR.  473/1, I cadge a garment, I set lystes in the lynyng to kepe the plyghtes in order.    Ibid. 596/1, I kadge the plyghtes of a garment. Je dresse des plies dune lisiere. This kote is yll kadged: ce sayon a ses plies mal dressés dune lisiere.

\itembf{3.} (See quots.) ? To tie or knot. Still dial.

\P 1703 THORESBY  \textit{Let. to Ray} (E.D.S.) To cadge, a term in making bone-lace.

\itembf{II.} To carry about, beg, etc.

\itembf{4.} trans. To carry about, as a pedlar does his pack, or a \textbf{CADGER} his stock-in-trade. Obs. exc. dial.

\P 1607 T. WALKINGTON  \textit{Opt. Glass} 154 Another Atlas that will cadge a whole world of iniuries without fainting.    
\P 1691 RAY  \textit{N.C. Wds.} (E.D.S.) Cadge, to carry.    
\P 1718 RAMSAY  \textit{Contn. Christ's Kirk} iii. xii, They gart him cadge this pack.    
\P 1788 MARSHALL  \textit{E. Yorksh.} Gloss. (E.D.S.) Cadge, to carry.    
\P 1858 M. PORTEOUS  \textit{Souter Johnny} 11 Weary naigs, that on the road Frae Carrick shore cadged monie a load.    
\P 1875 F. K. ROBINSON  \textit{Whitby Gloss.} (E.D.S.) Cadge, to carry; or rather, as a public carrier collects the orders he has to take home for his customers.

\itembf{5.} To load or stuff the belly. dial.

\P 1695 KENNETT  \textit{Par. Antiq.} Gloss. s.v. Cade, Hence‥cadge-belly, or kedge-belly, is a full fat belly.
\P c1746 COLLIER  (T. Bobbin) \textit{View Lanc. Dial.} Wks. (1862) 68 While I'r busy cadging mey Wem.    
\P 1854 BAMPTON  \textit{Lanc. Gloss., Cadge,} to stuff the belly.

\itembf{6.} intr. To go about as a cadger or pedlar, or on pretence of being one; to go about begging. dial. and slang.

\P 1812 J. H. VAUX  \textit{Flash Dict., Cadge,} to beg.    
\P 1846 LYTTON  \textit{Lucretia} ii. xii, ‘I be's good for nothin' now, but to cadge about the streets, and steal, and filch’.    
\P 1855 \textit{Whitby  Gloss.}, To Cadge about, to go and seek from place to place, as a dinner-hunter.    
\P 1859 H. KINGSLEY  \textit{G. Hamlyn} xv. (D.) ‘I've got my living by casting fortins, and begging, and cadging, and such like’.    
\P 1875 \textit{Lanc.  Gloss.} (E.D.S.) Cadge, to beg; to skulk about a neighbourhood.    
\P 1879 \textit{Print. Trades Jrnl.} xxix. 32 Cadging for invitations to the Mansion House.

\itembf{b.} trans. To get by begging.

\P 1848 E. FARMER  \textit{Scrap Book} (ed. 6) 115 Let each ‘cadge’ a trifle.    
\P 1878 BLACK  \textit{Green Past.} xi. 86 Where they can cadge a bit of food.
\end{myenumerate}


%%%%%%%%%%%%%%%%%%%%%%%%%%%%%%%%
\myitem{cajole} v.

\noindent \phonetic{(kəˈdʒəʊl)}

\noindent [a. F. cajoler, in same sense, of uncertain origin and history.
   Paré c 1550 has ‘cageoller comme un gay’ to chatter like a jay. Littré has 16th c. examples of cajoler, cajoller, cageoller, in the senses ‘to chatter like a jay or magpie’, and ‘to sing’, also, in the modern sense ‘to cajole’. Cotgr. 1611 has cajoler, cageoler ‘to prattle or jangle like a jay (in a cage), to bable or prate much to little purpose’. Most etymologists taking cageoler as the original form, have inferred its derivation from cage cage, through an assumed dim. *cageole. This is doubtful both in regard to sense and form; the early meaning ‘to chatter like a jay’ does not very obviously arise from cage, and does not clearly give rise to the modern sense. The Fr. dim. of cage is not *cageole but geôle ‘gaol’, whence F. enjôler (OF. engaioler, engauler, Sp. enjaular) ‘to put in gaol, imprison’, also ‘to inveigle, entice, allure, enthrall by fair words, cajole’. In Namur, cajoler has the sense enjoliver, to make joli, whence Grandgagnage would refer it to the stem jol- of joli, with ‘prefix ca- frequent in Walloon with an iterative force’. It is possible that two or even three words are here confused; in the modern sense, F. cajoler is synonymous with enjôler above, and if not cognate with that word, its sense has probably at least been taken over from it by form-association of cageoler or cajoler with enjôler. But the working out of the history must be left to French etymologists.] 
\vspace{-0.3cm}

\begin{myenumerate}
\itembf{1.} trans. To prevail upon or get one's way with (a person) by delusive flattery, specious promises, or any false means of persuasion. (‘A low word’ J.)

\P 1645 \textit{King's Cabinet Open.} Pref. 2 How the Court has been Caiolde (thats the new authentick word now amongst our Cabalisticall adversaries) by the Papists.    Ibid. 46 He‥gives avisoes to Caiole the Scots and Independents.    
\P 1649 MILTON  \textit{Eikon.} xxi, That the people might no longer be abused and cajoled, as they call it, by falsities and court-impudence.    
\P 1678 BUTLER  \textit{Hud.} iii. i. 1526 'Tis  no mean part of civil State-Prudence, to cajoul the Devil.    
\P 1723 SHEFFIELD  (Dk. Buckhm.) \textit{Wks.} (1753) II. 137 Cajoling a proud Nation to change their Master.    
\P 1735 POPE  \textit{Donne Sat.} iv. 90 You Courtiers so cajol us.    
\P 1823 LINGARD  \textit{Hist. Eng.} VI. 196 They sometimes cajoled, sometimes threatened the pontiff.    
\P 1863 W. PHILLIPS  \textit{Speeches} iii. 36 Leading statesmen have endeavored to cajole the people.

\itembf{b.} Const. into, from an action or state.

\P 1663 PEPYS  \textit{Diary} 17 Mar., Sir R. Ford‥cajoled him into a consent to it.
\P a1853 ROBERTSON  \textit{Lect.} ii. 55 Nor to cajole or flatter you into the reception of my views.    
\P 1862 TRENCH  \textit{Mirac.} xxviii. 310 He could neither be cajoled nor terrified from his‥avowal of the truth.

\itembf{c.} Const. out of: (a) to do (a person) out of (a thing) by flattery, etc.; (b) to get (a thing) out of a person by flattery, etc.

\P 1749 FIELDING  \textit{Tom Jones} xi. ix. (1840) 165/1 Everybody would not have cajoled this out of her.    
\P 1833 MARRYAT  \textit{P. Simple} (1863) 33 The stockings which she cajoled him out of.    
\P 1839 W. IRVING  \textit{Wolfert's R.} (1855) 247 The populace‥are not to be cajoled out of a ghost story by any of these plausible explanations.

\itembf{2.} intr. or absol. To use cajolery. †\textit{to cajole with}:—sense 1 (cf. persuade with).

\P 1665 PEPYS  \textit{Diary} 12 Oct., He hath cajolled with Seymour, who will be our friend.    
\P 1789 BELSHAM  \textit{Ess.} I. iii. 40 [Elizabeth] knew how to cajole, how to coax, and to flatter.    
\P 1870 L'ESTRANGE  \textit{Miss Mitford} I. vi. 210 The well-fee'd lawyers have ceased to browbeat or to cajole.
\end{myenumerate}


%%%%%%%%%%%%%%%%%%%%%%%%%%%%%%%%%
\myitem{callow} a. and n.

\noindent \phonetic{(ˈkæləʊ)}

\noindent [OE. calu (def. calw-e):—WGer. kalwo-, whence also MLG. kale, MDu. cāle (calu, gen. caluwes), OHG. chalo (def. chalwe, chalawe), MHG. kal (kalwe), Ger. kahl, by Kluge thought to be cognate with Lith. gŏlŭ naked, blank; but not improbably an adoption of L. calv-us bald. Cf. Ir. and Gael. calbh bald.]
\vspace{-0.3cm}

\begin{myenumerate}

\itembf{A.} adj.

\itembf{1.} Bald, without hair. Obs.

\P a1000 \textit{Prov.}  (Kemble) 42 (Bosw.) \phonetic{Moniᴁ} man weorþ færlice caluw.
\P a1000 \textit{Riddles}  xli. 99 (Gr.) Ic eom wide calu.
\P c1375 CATO  \textit{Major} ii. xxix, Þat forehed is lodly Þat is calouh \& bare.    
\P 1388 WYCLIF  \textit{Lev.} xiii. 40 A man of whos heed heeris fleten awei, is calu [1382 ballid].

\itembf{2.} Of birds: Unfledged, without feathers.

\P 1603 HOLLAND  \textit{Plutarch's Mor.} 63 Yoong callow birds which are not yet fethered and fledg'd.    
\P 1728 THOMSON  \textit{Spring} 667 The callow young‥Their brittle bondage break.    
\P 1801 SOUTHEY  \textit{Thalaba} v. iii. Poems IV. 180 Her young in the refreshing bath, Dipt down their callow heads.    
\P 1822 HAZLITT  \textit{Table-t.} II. xiv. 329 The callow brood are fledged.

\itembf{b.} Applied to the down of unfledged birds; and so, to the down on a youth's cheek and chin.

\P 1604 DRAYTON  \textit{Owle} 245 His soft and callow downe.    
\P 1697 DRYDEN  \textit{Virg. Past.} viii. 57 The callow Down began to cloath my Chin.    
\P 1735 SOMERVILLE  \textit{Chase} ii. 457 Prove‥their Valour's Growth Mature, e'er yet the callow Down has spread Its curling Shade.

\itembf{3.} fig. Inexperienced, raw, ‘unfledged’.

\P 1580 HARVEY in \textit{Spenser's Wks.} (Grosart) I. 40 Some, that weene themselves as fledged as the reste, being‥as kallowe.    
\P 1651 CLEVELAND  \textit{Poems} 31 Blasphemy unfledg'd, a callow curse.
\P a1797 H. WALPOLE  \textit{Mem. Geo.} II (1847) I. xii. 410 Teaching young and callow orators to soar.    
\P 1823 LAMB  \textit{Elia Ser.} ii. xvii. (1865) 343 The first callow flights in authorship.    
\P 1849 C. BRONTË  \textit{Shirley} xxxiii. 474 In all the voluptuous ease of a yet callow pacha.

\itembf{4.} Of land: a.A.4.a Bare; b.A.4.b (Ireland.) Low-lying and liable to be submerged.

\P 1677 PLOT  \textit{Oxfordsh.} 243 When these Lands are not swardy enough to bear clean tillage, nor callow or light enough to lie to get sward.    
\P 1878 LEVER  \textit{J. Hinton} xx. 138 Broad tracts of bog or callow meadow-land.    
\P 1882 \textit{Science  Gossip} Mar. 51 If a callow meadow is flooded all the winter.

\itembf{5.} Comb. †callow-mouse, a bat.

\P 1340 \textit{Ayenb.}  27 Þe enuious ne may ysy þet guod of oþren nanmore þanne þe oule oþer þe calouwe mous þe briȝtnesse of þe zonne.

\itembf{B.} n.

\itembf{1.} One who is bald; a bald-pate. Obs.

\P 1305 \textit{Life  St. Dunstan} 89 in E.E.P. (1862) 37 Out, what haþ þe calewe [St. Dunstan] ido: what haþ þe calewe ido.

\itembf{2.} A callow nestling; fig. a raw youth. Obs.

\P 1667 JER. TAYLOR  \textit{Serm.} (1678) 310 Such a person‥de$\sim$plumes himself to feather all the naked Callows that he sees.    
\P 1670 A. BEHN  \textit{Widow Rant.} iv. iii, She‥that can prefer such a callow as thou before a man.

\itembf{3.} The stratum of vegetable soil lying above the subsoil; the top or rubble bed of a quarry, which has to be removed to reach the rock. dial.

\P 1863 MORTON  \textit{Cycl. Agric.} II. Gloss. (E.D.S.) Callow (Norf., Suff.), the soil covering the subsoil.    
\P 1875 URE  \textit{Dict. Arts} I. 673 Callow, the top or rubble bed of a quarry. This is obliged to be removed before the useful material is raised.

\itembf{4.} A low-lying damp meadow by the banks of an Irish river.

\P 1862 H. COULTER  \textit{West of Ireland} 8 The extensive Callows lying along the banks of the Suck.    
\P 1865 \textit{Gard. Chron. \& Agric. Gaz.} 15 July 663/2 The callows consist of low flat land near a river, and liable to be overflowed, as well as being always in a damp state in the driest seasons.    
\P 1883 DUNDEE  \textit{Advert.} 25 Aug. 6/1 All the callows on the banks [of the Shannon] to Lusmagh‥are submerged.

\vspace{0.1cm} \noindent
Hence \textbf{callowness, callowy} a.

\P 1855 DE QUINCEY  in \textit{Page Life} (1877) II. xviii. 90 Such advantage‥as belongs to callowness or freshness.    
\P 1823 \textit{Monthly  Mag.} LV. 240 Like to a bird, who bestows on her callowy nestlings the morsel.
\end{myenumerate}


%%%%%%%%%%%%%%%%%%%%%%%%%%%%%%%%%
\myitem{calumny} n.

\noindent \phonetic{(ˈkæləmnɪ)}

\noindent [ad. L. calumnia and F. calomnie (15th c. in Littré).]

\vspace{-0.3cm}

\begin{myenumerate}

\itembf{1.} False and malicious misrepresentation of the words or actions of others, calculated to injure their reputation; libellous detraction, slander.

\P 1564 QUEEN  ELIZABETH in Froude \textit{Hist. Eng.} (1863) VIII. 103 Calumny will not fasten on me for ever.    
\P 1602 SHAKES.  \textit{Ham.} iii. i. 141 Be thou as chast as Ice, as pure as Snow, thou shalt not escape Calumny.    
\P 1611 \textit{Wint. T.} ii. i. 72 The Shrug, the Hum, or Ha (these Petty-brands That Calumnie doth vse).    
\P 1751 JOHNSON  \textit{Rambl.} No. 144 \cardo{⁋}6 Calumny is diffused by all arts and methods of propagation.    
\P 1838 THIRLWALL  \textit{Greece} V. xl. 118 His conduct‥had given a handle for calumny.

\itembf{2.} A false charge or imputation, intended to damage another's reputation; a slanderous report.

\P 1611 CHAPMAN  \textit{Iliad} xx. (R.) What then need we vie calumnies, like women that will weare Their tongues out.    
\P 1675 BAXTER  \textit{Cath. Theol.} ii. i. 108 The Synod of Dort rejecteth your accusation as a Calumny.    
\P 1751 JOHNSON  \textit{Rambl.} No. 183 \cardo{⁋}7 To spread suspicion, to invent calumnies, to propagate scandal, requires neither labour nor courage.    
\P 1836 GILBERT  \textit{Chr. Atonem.} vi. (1852) 168 A calumny against the revealed character of God.
\end{myenumerate}


%%%%%%%%%%%%%%%%%%%%%%%%%%%%%%%%%
\myitem{canard} n.

\noindent \phonetic{(kanar, kəˈnɑːd)}

\noindent [Fr.; lit. ‘duck’; also used in sense 1: see note there.]
\vspace{-0.3cm}

\begin{myenumerate}

\itembf{1.} An extravagant or absurd story circulated to impose on people's credulity; a hoax, a false report.

   Littré says Canard for a silly story comes from the old expression ‘vendre un canard à moitié’ (to half-sell a duck), in which à moitié was subsequently suppressed. It is clear that to half-sell a duck is not to sell it at all; hence the sense ‘to take in, make a fool of’. In proof of this he cites bailleur de canards, deliverer of ducks, utterer of canards, of date 1612: Cotgr., 1611, has the fuller vendeur de canards a moitié ‘a cousener, guller, cogger; foister, lyer’. Others have referred the word to an absurd fabricated story purporting to illustrate the voracity of ducks, said to have gone the round of the newspapers, and to have been credited by many. As this account has been widely circulated, it is possible that it has contributed to render the word more familiar, and thus more used, in English. [I saw the word in print before 1850 (J.A.H.M.).]

\P 1864 in WEBSTER.    
\P 1866 \textit{Even. Standard} 13 July 6 A silly canard circulated by the Owl, about England having joined France and Russia in ‘offering’ their mediation to the belligerents.    
\P 1880 W. DAY  \textit{Racehorse in Train.} xix. 185 The canards so industriously circulated as to the real cause of the deadly opposition he had met with.

\itembf{2.} A smaller surface on an aeroplane or hydrofoil providing stability or a means of control and placed forward of the main lifting surface; also (and orig.) an aeroplane with its wings so placed. Also attrib.

\P 1916 H. BARBER  \textit{Aeroplane Speaks} 137 Canard, literally ‘duck’, the name which was given to a type of aeroplane of which the longitudinal stabilizing surface (empennage) was mounted in front of the main lifting surface.    
\P 1928 C. F. S.  GAMBLE \textit{North Sea Air Station} Introd. 11 These monoplanes were of the ‘Canard’ (or ‘tail first’) type.    
\P 1931 \textit{Flight}  2 Jan. 4/1 His brother experimented with canard models.    
\P 1961 \textit{New Scientist} 16 Nov. 416/3 Most tentative designs for a Mach 3 liner provide for a canard form with the main wing at the rear.    
\P 1964 \textit{Sci. Amer.} June 27/3 SCAT 17 is a delta-wing design with a canard, or balancing surface, at the nose.    
\P 1967 \textit{Jane's Surface Skimmer Systems} 1967-68 95/2  The foils have been arranged‥in a canard configuration, with one foil forward and two foils aft.

\itembf{3.} A bright, deep blue, like the colour which is found on a duck's wing.

\P [1902  \textit{Daily Chron.} 13 Dec. 8/4 The peculiar bright, yet deep, blue known in Paris as ‘canard’.]    
\P 1908 \textit{Westm.  Gaz.} 22 Feb. 13/2 Canard—a new shade of blue inspired by the lovely patch of iridescent greeny blue that occurs on a duck's wing.    
\P 1923  \textit{Daily Mail} 21 June 1 Over 40 shades including Ivory,‥Apricot, Canard.
\end{myenumerate}


%%%%%%%%%%%%%%%%%%%%%%%%%%%%%%%%
\myitem{candour} n.

\noindent \phonetic{(ˈkændə(r))}

\noindent [17th c. candor, a. L. candor (-ōrem) dazzling whiteness, brilliancy, innocency, purity, sincerity, f. root cand- of candēre to be white and shining, ac-cendĕre to set alight, kindle: cf. candid, candle. F. candeur (16th c. in Littré) may have aided; the 14th c. example is properly Latin.]
\vspace{-0.3cm}

\begin{myenumerate}

\itembf{1.} Brilliant whiteness; brilliancy. Obs.

\P [1398 TREVISA  \textit{Barth. De P.R.} xix. xi. (1495) 871 Candor is passynge whytnesse.]    
\P 1634 SIR T. HERBERT  \textit{Trav.} 91 This nights travaile was bettered by Cynthias candor.    
\P 1692 TRYON  \textit{Good House-w.} ii. 25 Milk‥the Emblem of Innocence, deriving that aimable and pleasant Candor from a Gleam of the divine Light.

\itembf{2.} Stainlessness of character; purity, integrity, innocence. Obs.

\P a1610 B. JONSON  \textit{Alch.} v. v. (1616) 676 Helpe his fortune, though with some small straine Of his owne candor.    
\P 1675 TRAHERNE  \textit{Chr. Ethics} xxv. 388 If afterwards he comes to see the candor of his abused friend.    
\P 1703 ROWE  \textit{Fair Penit.} i. i. 376 Pure native Truth And Candour of the Mind.
\P a1704 T. BROWN  \textit{Eng. Sat.} Wks. 1730 I. 29 My lord Dorsets morals and integrity, his candor and his honour.

\itembf{3.} Freedom from mental bias, openness of mind; fairness, impartiality, justice.

\P a1637 B. JONSON  \textit{Epigr.} cxxiii. (R.) Writing thyselfe, or judging others writ, I know not which th' hast most, candor or wit.    
\P 1653 \textit{Hales'  Dissert. Peace} in \textit{Phenix} (1708) II. 388 If thou hast but a grain of Candor in thy heart, and wilt pass Sentence according to the Prescript of Truth.    
\P 1702 \textit{Clarendon's  Hist. Reb.} I. Pref. 2 The candor, and impartiality of what he relates.    
\P 1794 PALEY  \textit{Evid.} iii. ii. (1817) 282 A species of candour which is shown towards every other book, is sometimes refused to the Scriptures.    
\P 1836 WHATELY  \textit{Chr. Evid.} v, To exercise candour in judging fairly of the evidences.    
\P 1857 H. REED  \textit{Lect. Brit. Poets} xv. 202 In criticism candour with its comprehensive sympathies, is as rare, as bigotry is frequent.

\itembf{4.} Freedom from malice, favourable disposition, kindliness; ‘sweetness of temper, kindness’ (J.). Obs.

\P 1653 WALTON  \textit{Angler} To Rdr., If he [the Reader] bring not candor to the reading of this Discourse, he shall‥injure me‥by too many Criticisms.    
\P 1666 DRYDEN  \textit{Ann. Mirab.} Ded. (Globe ed.) 42 Your candour in pardoning my errors.    
\P 1751 JOHNSON  \textit{Cheynel} Wks. IV. 508 He shews himself sincere, but without candour.    
\P 1765 \textit{Pref. Shaks.} Wks. IX. 252 That bigotry which sets candour higher than truth.    
\P 1802 \textit{Med.  Jrnl.} VIII. 226 A gentleman of unbounded candor, and a most benevolent disposition.

\itembf{5.} Freedom from reserve in one's statements; openness, frankness, ingenuousness, outspokenness.

\P 1769 \textit{Lett.  Junius} ii. 11 This writer, with all his boasted candour, has not told us the real cause of the evils.    
\P 1836 HOR.  SMITH \textit{Tin Trump.} (1876) 72 Candour in some people may be compared to barley sugar drops, in which the acid preponderates over the sweetness.    
\P 1876 J. H. NEWMAN  \textit{Hist. Sk.} I. ii. iv. 257 Openness and candour are rare qualities in a statesman.
\end{myenumerate}


%%%%%%%%%%%%%%%%%%%%%%%%%%%%%%%%
\myitem{canon} n.

\noindent \phonetic{(ˈkænən)}

\noindent [Found in OE. as canon, a. L. canon rule, a. Gr. κανών rule. Early ME. had canon, prob. from OE., and canun, canoun, a. OF. canun, canon, the Fr. descendant of the L. Senses 12-14 are of obscure origin; some or all may belong to cannon, in F. spelt canon.]
\vspace{-0.3cm}

\begin{myenumerate}

\itembf{1. a.} A rule, law, or decree of the Church; esp. a rule laid down by an ecclesiastical Council. the canon (collectively) = canon law: see b.

   \textbf{The Canons}, in Ch. of Engl. = ‘The Constitutions and Canons Ecclesiastical’ agreed upon by Convocation, and ratified by King James I under the Great Seal in 1603.

\P c890 K. ÆLFRED \textit{Bæda} iv. xxiv. (Bosw.) Canones boc.    
\P a900 \textit{Laws of Ælfred} xxi. in Thorpe II. 376 (Bosw.) Ða canonas openlice beodaþ.   
\P a1300  \textit{Cursor M.} 26290 Als þe hali canon [v.r. -oun] vs sais þat scrift on sere-kin sines lais.    
\P 1451 \textit{Treaty  w. Scotl.} in Rymer \textit{Foedera} (1710) XI. 288 Maister Robert Dobbes, Doctor of Canon.    
\P 1489 CAXTON  \textit{Faytes of A.} iv. ix. 254 The canon deffendeth expresly al manere of bataille and violent hurt.    
\P 1597 HOOKER  \textit{Eccl. Pol.} v. lxi. §2 A sacred canon of the sixth reverend synod.    
\P 1601 SHAKES.  \textit{All's Well} i. i. 158 Selfe-loue, which is the most inhibited sinne in the Cannon.    
\P 1658 BRAMHALL  \textit{Consecr. Bps.} vii. 171 The Papall Canons were never admitted for binding Lawes in England.    
\P 1827 HALLAM  \textit{Const. Hist.} (1876) I. vi. 303 A code of new canons had recently been established in convocation with the King's assent.    
\P 1859 JEPHSON  \textit{Brittany} viii. 131 A priest is expressly forbidden by the canons‥to enter a public inn.

\itembf{b.} canon law (formerly law canon: cf. F. droit canon): ecclesiastical law, as laid down in decrees of the pope and statutes of councils. (See Gratian, Dist. iii. §2.)

\P c1340  \textit{Cursor M.} 26290 (Fairf.) Squa sais lagh Canoun þat is wise, þat shrift on mani synnis lise.    
\P 1387 TREVISA  \textit{Higden} (1865) II. 117 (Mätz.) By dome of lawe canoun.
\P c1400 \textit{Apol.  Loll.} 73 Law canoun is callid law ordeynid of prelats of the kirk.    
\P 1494 FABYAN  vii. 526 They sent ye estudyauntys of ye lawe, canon \& cyuyle.    
\P 1511 in W. H. TURNER  \textit{Select. Records Oxford} 7 John Prynne, bachiller of Canon.    
\P 1552 ABP. HAMILTON  \textit{Catech.} (1884) 1 Doctours of Theologie and Canon law.
\P a1586 \textit{Answ.  Cartwright} 3 The common Lawes are against the cannon Lawes in many hundreth poyntes.    
\P 1765 BLACKSTONE  \textit{Comm.} i. Introd. 82 The canon law is a body of Roman ecclesiastical law, relative to such matters as that church either has, or pretends to have, the proper jurisdiction over. This is compiled from the opinions of the antient Latin fathers, the decrees of general councils, the decretal epistles and bulles of the holy see.    
\P 1850 A. JAMESON  \textit{Leg. Monast. Ord.} (1863) 331 Where he made himself master of civil and canon law.

\itembf{2.} gen. \textbf{a.} A law, rule, edict (other than ecclesiastical). \textbf{b.} A general rule, fundamental principle, aphorism, or axiom governing the systematic or scientific treatment of a subject; e.g. canons of descent or inheritance; a logical, grammatical, or metrical canon; canons of criticism, taste, art, etc.

\P 1588 FRAUNCE  \textit{Lawiers Log.} i. ii. 7 b, Such rules, maximaes, canons, axioms‥or howsoever you tearme them.    
\P 1602 SHAKES.  \textit{Ham.} i. ii. 132 Or that the Euerlasting had not fixt His Cannon 'gainst Selfe-slaughter.    
\P 1607 \textit{Cor.} i. x. 26 Against the hospitable Canon.    
\P 1628 MILTON  \textit{Vac. Exerc.}, Substance with his Canons; which Ens‥explains.    
\P 1788 REID  \textit{Aristotle's Log.} v. ii. 113 They have reduced the doctrine of the topics to certaine axioms or canons.    
\P 1806 \textit{Med. Jrnl.} XV. 134 The canons of pathology.    
\P 1869 ROGERS  \textit{Pref. Adam Smith's W.N.} I. 17 The indirect taxation of France violated every canon of financial prudence and equity.    
\P 1874 SAYCE  \textit{Compar. Philol.} i. 58 The canons of taste and polite literature.    
\P 1879 FARRAR  \textit{St. Paul} I. 613 We may assume it as a canon of ordinary criticism that a writer intends to be understood.

\itembf{c.} A standard of judgement or authority; a test, criterion, means of discrimination.

\P 1601 HOLLAND  \textit{Pliny} II. 497 Moreouer, he made that which workmen call Canon, that is to say, one absolute piece of worke, from whence artificers do fetch their draughts, simetries, and proportions.    
\P 1651 HOBBES  \textit{Govt. \& Soc.} xvii. §16. 313 The sacred Scripture is‥the Canon and Rule of all Evangelicall Doctrine.    
\P 1869 GOULBURN  \textit{Purs. Holiness} vii. 65 This Lord's Prayer, what a canon does it supply for testing and correcting our spiritual state.    
\P 1874 W. WALLACE  \textit{Hegel's Logic} §52. 93 [Reason] is a canon, not an organon of truth, and can furnish only a criticism of knowledge.

\itembf{3.} Math. A general rule, formula, table; esp. a table of sines, tangents, etc. Obs.

\P 1391 CHAUCER  \textit{Astrol.} ii. §32 Lok how many howres thilke coniunccion is fro the Midday of the day precedent, as shewith by the canoun of thi kalender.    
\P 1594 BLUNDEVIL  \textit{Exerc.} ii. (ed. 7) 130 If you shall not finde in the Canon, the Sine which by your calculation is found.    
\P 1656 tr. \textit{Hobbes' Elem. Philos.} (1839) 292 The straight line BV‥if computed by the canon of signs.    
\P 1706 PHILLIPS, In Mathematicks, Cannon is an infallible Rule to resolve all things of the same Nature with the present Inquiry.    
\P 1751 CHAMBERS  \textit{Cycl.} s.v. Canon, Natural Canon of Triangles is a table of sines, tangents, and secants together‥Artificial Canon of Triangles is a table wherein the logarithms of sines and tangents are laid down.    
\P 1798 HUTTON  \textit{Course Math.} (1807) II. 3 A Trigonometrical Canon, is a table.

\itembf{4.} The collection or list of books of the Bible accepted by the Christian Church as genuine and inspired. Also transf., any set of sacred books; also, those writings of a secular author accepted as authentic.

\P 1382 WYCLIF  \textit{Apoc. Prol.}, In the bigynnyng of canon, that is, of the bok of Genesis.    
\P 1591 T. NORTON  \textit{Calvin's Inst.} i. 13 b, What reuerence is due to the Scripture, and what bookes are to be reckened in the canon therof.    
\P 1641 J. JACKSON  \textit{True Evang. T.} ii. 116 S. Andrew the Apostle‥added nothing to the Canon of Scripture.    
\P 1870 MAX  MÜLLER \textit{Sc. Relig.} (1873) 29 The process by which a canon of sacred books is called into existence.    
\P 1882 FARRAR  \textit{Early Chr.} I. 98 The Epistle to the Hebrews is not a work of St. Paul, but it is pre-eminently worthy of its honoured place in the Canon.    
\P 1885 \textit{Encycl. Brit.} XIX. 211/1 The dialogues forming part of the ‘Platonic canon’.    
\P 1953 C. J. SISSON  \textit{Shakespeare: Compl. Works} p. xviii (heading) The canon and the text.

\itembf{5.} A canonical epistle. See canonical 3.

\P 1483 CAXTON  \textit{Gold. Leg.} 25/3 Saynt John that saith in his canone, We have, etc.    
\P 1502 \textit{Ord. Crysten Men} ii. i. (W. de W. 1506) 84 Wherfore sayth well saynt Iames in his canon.

\itembf{6.} The portion of the Mass included between the Preface and the Pater, and containing the words of consecration.

\P a1300  \textit{Cursor M.} 21190 ÞE  first mess þat sent petre sang, Was þar þan na canon lang Bot pater-noster in þaa dais, Na langer canon was, it sais.    
\P 1395 PURVEY  \textit{Remonstr.} (1851) 42 After the sacringe, in the canoun of the masse.
\P a1450 \textit{Knt. de la Tour} (1868) 40. 
\P 1532 MORE  \textit{Confut. Tindale} Wks. 490/2 Luter himself casting away the holy canon of ye masse.    
\P 1656 BP. HALL  \textit{Tracts} (1677) 43 It was the farther solemnizing and beautifying that holy action which brought the Canon in.    
\P 1781 GIBBON  \textit{Decl. \& F.} II. xlv. 695 He officiated in the canon of the mass.    
\P 1868 HOOK  \textit{Lives Abps.} II. ii. iii. 284 note, The canon or rule was the part of the service containing the actual consecration.

\itembf{7.} Mus. \textbf{a.} A species of musical composition in which the different parts take up the same subject one after another, either at the same or at a different pitch, in strict imitation.

   A passage in Burney's Hist. Music (1781) 480 suggests as an earlier meaning: ‘The rule by which a composition (in canon-form), which is only partially indicted in the score, can be read out by the performers in full.’ Cf. quot. 1609.

\P 1597 MORLEY  \textit{Introd. Mus.} 104 Of how manie parts the Canon is, so manie Cliefes do they set at the beginning of the verse.    
\P 1609 DOULAND  \textit{Ornith. Microl.} 48 A Canon‥is an imaginarie rule, drawing that part of the Song which is not set downe out of that part which is set downe. Or it is a Rule, which doth wittily discouer the secret of a Song.    
\P 1795 MASON  \textit{Ch. Mus.} i. 54 Such Organists as were Masters of Canon, Fugue, and Counterpoint.    
\P 1869 OUSELEY  \textit{Counterp.} xxiii. §13 The closest stretto should be reserved for the end‥especially if it be introduced in canon.

\itembf{b.} A long hymn, used in the Eastern Church, consisting of eight odes, each of many stanzas.

\P 1862  \textit{Q. Rev.} Apr. 338 If we might venture‥to name the characteristics of these canons, we should say richness and repose, and a continuous thread of Holy Scripture‥woven into them.

\itembf{8. a.} ‘In old Records, a Prestation, Pension, or Customary payment upon some religious Account’ (Phillips 1706). From Roman Law.

\P 1633 CAVE  \textit{Ecclesiastici} Introd. 51 He restor'd the Corn-Canon, (as they call'd it) the yearly Allowance of Corn, which Constantine had settled upon the Church.    
\P 1726 AYLIFFE  \textit{Parerg.} 139 Which Allowance was, by the ancient Lawyers, called a Canon, and not a Prebend, as now it is.    
\P 1847-79 HALLIWELL,  \textit{Canon}, a portion of a deceased man's goods exacted by the priest.

\itembf{b.} A quit-rent. [cf. Littré, Canon 10.]

\P 1643 PRYNNE  \textit{Power Parl.} App. 164 Therefore to sustaine the burthens of Peace, the demesne was instituted, (which among the Lawyers is called Canon).    
\P 1774 S. HALLIFAX  \textit{Anal. Rom. Law} (1795) 69 On condition that the Tenant shall improve the Lands, and pay a yearly Canon or Quit-Rent to the Proprietor.

\itembf{9. a.} A chief epoch or era, serving to date from (Gr. κανὼν χρονικός); a basis for chronology. Cf. canon monument in 15.

\P 1833 CRUSE  \textit{Eusebius} vi. xxii. 242 A certain canon comprising a period of sixteen years.    
\P 1876 BIRCH  \textit{Rede Lect. Egypt} 14 The Turin papyrus, the canon of history, a list of all the kings.

\itembf{b.} paschal canon: the rule for finding Easter, to which was often appended a table of the dates of Easter and the feasts varying with it for a series of years.

\P 1727-51 CHAMBERS  \textit{Cycl.} s.v. Canon, Paschal Canon, a table of the moveable feasts, shewing the day of Easter, and the other feasts depending on it, for a cycle of nineteen years.

\itembf{10. a.} (See quot.)

\P 1727-51 CHAMBERS  \textit{Cycl.}, Canon, in monastic orders, a book wherein the religious of every convent have a fair transcript of the rules of their order, frequently read among them as their local statutes.

\itembf{b.} ‘The list of saints acknowledged and canonized by the Church’ (Chambers Cycl. 1727-51).

\itembf{11.} Printing. A size of type-body equal to 4-line Pica; the largest size of type-body that has a specific name.

   So called perhaps as being that used for printing the canon of the Mass; but Tory is said by Reed (op. cit. 36) to have used the term Canon for letter cut according to rule—lettres de forme—as distinguished from lettres bastardes.

\P 1683 MOXON  \textit{Mech. Exerc.}, French Canon 17½ [types] to a foot.    
\P 1688 R. HOLME  \textit{Armoury} iii. iii. 119/2 Canon, the great Canon is the name of the largest Letter for Printing that is used in England.    
\P 1721 BAILEY,  \textit{Canon}, (with Printers) a large sort of Printing Letter.    
\P 1887 T. B. REED  \textit{O. Eng. Lett. Foundries}, 36 The Canon of the Mass was‥printed in a large letter, and it is generally supposed that this size of letter being ordinarily employed in the large Missals, the type-body took its name accordingly; a supposition which is strengthened by its German name of Missal.

\itembf{12.} (See quot.)

\P 1696 PHILLIPS,  \textit{Canon}‥a Surgeon's Instrument, made use of for the sewing up of Wounds.    
\P 1721 IN Bailey;
\P 1755 in JOHNSON;  and in mod. Dicts.    Not in Syd. Soc. Lex.

\itembf{13.} (See quot.)

\P 1847-78 HALLIWELL,  \textit{Canons}, the first feathers of a hawk after she has mewed. [Perh. the same as cannon: cf. Sp. cañon a quill.]

\itembf{14.} A metal loop or ‘ear’ at the top of a bell, by which it is hung. Also written cannon (n.1 5).

\P 1688 R. HOLME  \textit{Armoury} iii. 461/2 This is called a St. Bell, because it hath not Canons on the head to fasten it to the stock.    
\P 1878 \textit{Grove  Dict. Mus.} I. 219 [Bells] are first carefully secured by iron bolts and braces through the ears or ‘canons’ to the stock.    
\P 1882 \textit{School  Guardian} No. 315. 12 The height of the bell from the lip to the top of the canons is 8 ft.

\itembf{15.} attrib. and Comb., as \textbf{canon law} (see 1b), \textbf{canon-lawyer, canon-making, canon monument} (cf. 9), \textbf{canon rule, canon type} (cf 11): canon-like, canon-wise adjs.

\P 1601 BP. BARLOW  \textit{Defence} 99 We acknowledge it *Canon-like, but not Canonicall.

\P 1659 BAXTER  \textit{Key Cath.} xxv. 147 This is a cheaper way of *Canon-making in a corner.

\P 1631 R. BYFIELD  \textit{Doctr. Sabb.} 149 You finde nothing‥in any‥*cannon monument, and register of Antiquitie.

\P 1603 HOLLAND  \textit{Plutarch's Mor.} 33 The very *Canon rule, and paterne of all vertue.

\P 1641 MILTON Reform. Wks. 1738 I. 7  An insulting and only *Canon-wise Prelate.

\itembf{16. a.} Literary Criticism. A body of literary works traditionally regarded as the most important, significant, and worthy of study; those works of esp. Western literature considered to be established as being of the highest quality and most enduring value; the classics (now freq. in the canon). Also (usu. with qualifying word): such a body of literature in a particular language, or from a particular culture, period, genre, etc.

\P 1929 \textit{Amer.  Lit.} 1 95 Those who read bits of Mather with pleasure will continue to feel that those bits cannot be excluded from the canon of literature until much excellent English ‘utilitarian’ prose is similarly excluded.    
\P 1953 W. R. TRASK tr. E. R. Curtius \textit{European Lit. \& Lat. Middle Ages} xiv. 264 Of the modern literatures, the Italian was the first to develop a canon.    
\P 1989  \textit{Times Lit. Suppl.} 7 July 739 My Secret History‥alludes to half the modernist canon, from Eliot to Hemingway to Henry Miller.    
\P 1999 \textit{N.Y. Rev. Bks.} 4 Nov. 29/2 The canon was under attack from feminists and social historians who saw it as the preserve of male and bourgeois dominance.

\itembf{b.} In extended use (esp. with reference to art or music): a body of works, etc., considered to be established as the most important or significant in a particular field. Freq. with qualifying word.

\P 1977 R. MACKSEY in \textit{Compar. Lit.} 92 1188 The  author concentrates on six major works in the operatic canon, masterpieces by two towering figures in the history of Western music.    
\P 1985  \textit{Washington Post} 5 July x12/1 What looks like spaghetti Bolognese and keeps fresh on the shelf for 50 years? Japanese plastic food, the real-as-life models that restaurants in Japan use for the prosaic business of window display, and that visitors have gleefully added to the canon of pop art.    
\P 1995 \textit{Independent}  (Nexis) 10 Dec. 2 Mick taught himself to play the guitar and spent ‘a great deal of time’ studying songwriting; not just the soul and R'n'B legends‥but the whole rock canon—the Rolling Stones and Led Zeppelin and the Velvet Underground, but especially The Beatles.    
\P 1998 \textit{Herald  (Glasgow)} 3 Sept. 22 The concept has settled comfortably into the canon of accepted biological theory.
\end{myenumerate}


%%%%%%%%%%%%%%%%%%%%%%%%%%%%%%%%
\myitem{cant} n.

\noindent \phonetic{(kænt)}

\noindent [This and its accompanying vb. presumably represent L. cant-us singing, song, chant (Pr. and NFr. cant, Fr. chant), cantā-re NFr. canter) to sing, chant; but the details of the derivation and development of sense are unknown.

   Cantare and its Romanic representatives were used contemptuously in reference to the church services as early as 1183, when according to Rigord (c 1200) Gest Philip. August. (1818) 11, the Cotarelli of the Bourges country ‘sacerdotes et viros religiosos captos secum ducentes, et irrisoriè cantores ipsos vocantes, in ipsis tormentis subsannando dicebant: Cantate nobis, cantores, cantate; et confestim dabant eis alapas, vel cum grossis virgis turpiter cædebant’. So far as the evidence shows, the vb. appears in Eng. first applied to the tones and language of beggars, ‘the canting crew’: this, which according to Harman was introduced c 1540, may have come down from the religious mendicants; or the word may have been actually made from Lat. or Romanic in the rogues' jargon of the time. The subsequent development assumed in the arrangement of the verb is quite natural, though not actually established. Some have however conjectured that cant is the Irish and Gaelic cainnt (pronounced \phonetic{kaɲtj}, or nearly \phonetic{kantʃj}) ‘language’. And as early as 1711 the word was asserted to be derived from the name of Andrew Cant or his son Alexander Cant, Presbyterian ministers of the 17th c. This perhaps means that the surname of the two Cants was occasionally associated derisively with canting. The arrangement of the n. here is tentative, and founded mainly on that of the vb., which appears on the whole earlier.] 
\vspace{-0.3cm}

\begin{myenumerate}
\itembf{I.} (Sporadic uses, from L. cantus or its representatives; not directly related to II.)

\itembf{1.} Singing, musical sound. cant organ: app. a technical term in music. Obs.

\P 1501 DOUGLAS  \textit{Pal. Hon.} i. xlii, Fabourdoun, pricksang, discant, countering, Cant organe, figuratioun, and gemmell.    
\P 1704 SWIFT  \textit{T. Tub} Wks. 1760 I. 100  Cant and vision are to the ear and the eye the same that tickling is to the touch.    
\P 1708 \textit{Brit.  Apollo} No. 79. 2/2 That shrill Cant of the Grasshoppers.

\itembf{2.} Accent, intonation, tone. Obs.

\P 1663 \textit{Aron-bimn.}  110 It depends not upon the cant and tone, or the wording of the Minister.    
\P 1763 \textit{Ann. Reg.} 307/2 If these lines want that sober cant which is necessary to an epitaph.

\itembf{II.} The speech or phraseology of beggars, etc., and senses connected therewith.

\itembf{3.} ‘A whining manner of speaking, esp. of beggars’; a whine.

\P 1640 CLEVELAND in Wilkins \textit{Polit. Ballads} I. 28 By lies and cants, [they] Would trick us to believe 'em saints.    
\P 1705 HICKERINGILL  \textit{Priest-cr.} iv. (1721) 227 With a Cant like a Gypsie, a Whine like a beaten Spaniel.

\itembf{4.} The peculiar language or jargon of a class: \textbf{a.} The secret language or jargon used by gipsies, thieves, professional beggars, etc.; transf. any jargon used for the purpose of secrecy.

\P 1706 in PHILLIPS.    
\P 1707 J. STEVENS tr. \textit{Quevedo's Com. Wks.} (1709) 226 They talk'd to one another in Cant.    
\P 1715 KERSEY,  \textit{Cant}, Gibberish, Pedler's French.    
\P 1734 NORTH  \textit{Exam.} ii. v. \cardo{⁋}110. 383 To avoid being understood by the Servants, they framed a Cant, and called the Design of a general Rising the Lease and Release.    
\P 1865 DICKENS  \textit{Mut. Fr.} xvi. 127 The ring of the cant.

\itembf{b.} The special phraseology of a particular class of persons, or belonging to a particular subject; professional or technical jargon. (Always depreciative or contemptuous.)

\P 1684 T. BURNET  \textit{Th. Earth} I. 214 There is heat and moisture in the body, \& you may call the one ‘radical’ and the other ‘innate’ if you please; this is but a sort of cant.    
\P 1712 ADDISON  \textit{Spect.} No. 421 \cardo{⁋}3 In the Cant of particular Trades and Employments.    
\P 1750 JOHNSON  \textit{Rambl.} No. 128 \cardo{⁋}4 Every class of society has its cant of lamentation, which is understood by none but themselves.    
\P 1839 DICKENS  \textit{Nich. Nick.} xxxiv, All love—bah! that I should use the cant of boys and girls—is fleeting enough.    1841–4 Emerson Ess. xiii. Poet Wks. (Bohn) I. 156 Criticism is infested with a cant of materialism.    
\P 1861 HOLLAND  \textit{Less. Life} viii. 119 Repeating the cant of their sect and the cant of their schools.

\itembf{c.} The peculiar phraseology of a religious sect or class. (Cf. 5 b.) Obs.

\P 1681 DRYDEN  \textit{Abs. \& Achit.} 521 Hot Levites‥Resum'd their cant, and with a zealous cry Pursued their old beloved theocracy.    
\P 1696 C. LESLIE  \textit{Snake} in Gr. (1698) Introd. 46 Really to understand the Quaker-Cant is learning a new Language.    
\P 1709 SACHEVERELL  \textit{Serm.} 15 Aug. 15 Diabolical Inspiration, and Non-sensical Cant.    
\P 1711  \textit{Spect.} No. 147 ⁋3 Cant is by some people derived from one Andrew Cant who, they say, was a Presbyterian minister‥who by exercise \& use had obtained the Faculty, alias Gift, of talking in the Pulpit in such a dialect, that it's said he was understood by none but his own Congregation, and not by all of them.

\itembf{d.} Provincial dialect; vulgar slang.

\P 1802 M. EDGEWORTH  \textit{Irish Bulls} (1832) 226 The cant of Suffolk, the vulgarisms of Shropshire.    
\P 1852 GLADSTONE  \textit{Glean.} IV. lxxxii. 122 The coarse reproduction of that unmitigated cant or slang.

\itembf{e.} attrib.

\P 1727 SWIFT  \textit{Let. Eng. Tongue} Wks. 1755 II. I. 185  To introduce and multiply cant words is the most ruinous corruption in any language.    
\P 1824 W. IRVING  \textit{T. Trav.} I. 273 Slang talk and cant jokes.    
\P 1841 BORROW  \textit{Zincali} (1843) II. 150 The first Vocabulary of the ‘Cant Language’‥appeared in the year 1680 appended to the life of ‘The English Rogue’.

\itembf{5.} A form of words, a phrase: †a.II.5.a A set form of words repeated perfunctorily or mechanically. Obs.

\P 1681 SEJANUS in \textit{Bagford Ballads} (1878) 758 note, A young Scribe is copying out a Cant, Next morn for to be spoke in Parliament.    
\P 1704 STEELE  \textit{Lying Lover} i. i. 7 Sure‥you talk by Memory, a Form or Cant which you mistake for something that's gallant.    
\P 1712 ADDISON  \textit{Spect.} No. 291 §6 With a certain cant of words.

\itembf{b.} A pet phrase, a trick of words; esp. a stock phrase that is much affected at the time, or is repeated as a matter of habit or form. (Formerly with a and pl.) arch.

\P 1681 \textit{Country-man's  Compl. \& Advice to King}, Gods! to be twice cajol'd by cants and looks.    
\P 1691 WOOD  \textit{Ath. Oxon.} II./450 Enamour'd with his obstreporousness and undecent cants.    
\P 1692 BENTLEY  \textit{Boyle Lect.} 200 That ordinary cant of illiterate‥atheists, the fortuitous or casual concourse of atoms.    
\P 1710 HEARNE  \textit{Collect.} (1886) II. 365 The late happy Revolution, (so he calls it, according to the common Cant).    
\P 1769 JUNIUS  \textit{Lett.} xxvi. 119 note, Measures, and not men, is the common cant of affected moderation.    c 
\P 1815 JANE AUSTEN  \textit{Northang. Abb.} (1833) I. v. 22 It is really very well for a novel‥is the common cant.

\itembf{c.} attrib.

\P 1712 ADDISON  \textit{Spect.} No. 530 \cardo{⁋}3 Enlivened with little cant-phrases.    
\P 1753 \textit{Stewart's Trial App.} 130 It was a cant word through the country, That the tenants might sit, since the worst of it would be paying the violent profits.    
\P 1774 GOUV. MORRIS in \textit{Sparks Life \& Writ.} (1832) I. 23 The belwethers‥roared out liberty, and property, and a multitude of cant terms.    
\P 1790 PALEY  \textit{Horæ Paul.} (1849) 396 There is such a thing as a peculiar word or phrase cleaving, as it were, to the memory of a writer or speaker and presenting itself to his utterance at every turn. When we observe this we call it a cant word or a cant phrase.    
\P 1855 PRESCOTT  \textit{Philip II} (1857) I. v. 79 To borrow a cant phrase of the day, like ‘a fixed fact’.    
\P 1868 HELPS  \textit{Realmah} xvii. (1876) 465 He‥can—to use the cant phrase—afford to support the dignity of the peerage.

\itembf{6.} As a kind of phraseology: a.II.6.a Phraseology taken up and used for fashion's sake, without being a genuine expression of sentiment; canting language.

\P 1710 BERKELEY  \textit{Princ. Hum. Knowl.} §87 All this sceptical cant follows from our supposing, etc.    
\P 1783 JOHNSON in \textit{Boswell} 15 May, My dear friend, clear your mind of cant‥you may talk in this manner; it is a mode of talking in society; but don't think foolishly.    
\P 1809 SYD.  SMITH \textit{Wks.} (1867) I. 174 The pernicious cant of indiscriminate loyalty.    
\P 1870 LOWELL  \textit{Study Wind.} 157 Enthusiasm, once cold, can never be warmed over into anything better than cant.    
\P 1875 SMILES  \textit{Thrift} ii. 20 In fact there is no greater cant than can't.    
\P 1883 J. PARKER  \textit{Tyne} Ch. 320 There is a cant of infidelity as certainly as there is a cant of belief.

\itembf{b.} esp. Affected or unreal use of religious or pietistic phraseology; language (or action) implying the pretended assumption of goodness or piety.

\P 1709 STRYPE  \textit{Ann. Ref.} I. lv. 609, I set down this letter at large, that men may see the cant of these men.    
\P 1716 ADDISON  \textit{Freeholder} No. 37 (J.) That cant and hypocrisy, which had taken possession of the people's minds in the times of the great rebellion.    
\P 1789 MRS. PIOZZI  \textit{Journ. France} I. 256 Hypocritical manners, or what we so emphatically call cant.    
\P 1849 ROBERTSON  \textit{Serm.} Ser. i. x. (1866) 182 Religious phraseology passes into cant.    
\P 1875 HAMERTON  \textit{Intell. Life} vi. iii. 211 He had a horror of cant, which‥gave him a repulsion for all outward show of religious observances.    
\P 1879 FROUDE  \textit{Cæsar} i. 6 The whole spiritual atmosphere was saturated with cant.

\itembf{c.} attrib.

\P 1747 CARTE  \textit{Hist. Eng.} I. 601 To make up what was wanting in the justice of their cause‥by a cant and sophistical way of expression.

\itembf{7.} One who uses religious phrases unreally.

\P 1725  \textit{New Cant. Dict.}, Cant, an Hypocrite, a Dissembler, a double-tongu'd, whining Person.    
\P 1824 MRS. CAMERON  \textit{Pink Tippet} iii. 16 Lest she should be called a cant.    
\P 1873 E. BERDOC  \textit{Adv. Protestant} 132 He was not a cant, but really felt what he said.
\end{myenumerate}



%%%%%%%%%%%%%%%%%%%%%%%%%%%%%%%%%
\myitem{cantankerous} a. colloq.

\noindent \phonetic{(kænˈtæŋkərəs)}

\noindent [Said by Grose, who spells it contankerous, to be a Wiltshire word. This spelling gives some support to the conjecture that the word was formed on ME. contak, conteke, contention, quarrelling, contekour, conteckour one who raises strife, whence *conteckerous, *contakerous would be a possible deriv. like traitorous, which might subseq. be corrupted under influence of words like cankerous, rancorous. Its oddly appropriate sound, and perh. some assoc. with these words, have given it general colloquial currency.]

\noindent
Showing an ill-natured disposition; ill-conditioned and quarrelsome, perverse, cross-grained.

\P 1772 GOLDSM.  \textit{Stoops to Conq.} ii, There's not a more bitter cantanckerous road in all christendom.    
\P 1775 SHERIDAN  \textit{Rivals} v. iii, I hope, Mr. Faulkland‥you won't be so cantanckerous.    
\P 1842 MISS MITFORD in \textit{L'Estrange Life} (1870) III. ix. 142 As cantankerous and humorous as Cassius himself.    
\P 1865 LIVINGSTONE  \textit{Zambesi} ix. 195 A crusty old bachelor or‥a cantankerous husband.    
\P 1873 \textit{St. Paul's Mag.} i. 533 A cantankerous element in his nature.

\vspace{0.1cm} \noindent
Hence \textbf{cantankerously} adv., \textbf{cantankerousness}.

\P 1868 A. K. H. BOYD  \textit{Lessons Mid. Age} 217 One impracticable, stupid, wrongheaded, and cantankerously foolish person of the twelve.    
\P 1876 MRS. H. WOOD  \textit{Orville Coll.} 411 You have behaved cantankerously to him.    
\P 1881 A. R. HOPE in \textit{Boy's Own Paper} 10 Sept. 794 The roller had crushed the cantankerousness right out of him.    
\P 1886 \textit{Chr. Life} 2 Jan. 2/6 A member‥expelled for general cantankerousness.





%%%%%%%%%%%%%%%%%%%%%%%%%%%%%%%%%
\myitem{capitulate} v.

\noindent \phonetic{(kəˈpɪtjʊleɪt)}

\noindent [f. prec. or on analogy of vbs. so formed: see -ate3.]
\vspace{-0.3cm}

\begin{myenumerate}

\itembf{1.} trans. To draw up in chapters, or under heads or articles; to specify, enumerate. Obs.

\P 1593 LODGE  \textit{Wm. Longbeard} E ij b, The lawes‥which we capitulate at sea are not‥ used on lande.    
\P 1608 TOPSELL  \textit{Serpents} 600 The places of serpents abode being thus generally capitulated.
\P c1645 HOWELL  \textit{Lett.} (1678) 116.    
\P 1678 MARVELL  \textit{Def. Howe} Wks. 1875 IV. 182  The Discourse‥capitulates that Mr. Howe should by efficacious intend infallibility, etc.

\itembf{b.} intr.

\P 1596 NASHE  \textit{Saffron Walden} 81 For an assay‥of his pen, he capitulated on the births of monsters.

\itembf{2.} intr. To draw up articles of agreement; to arrange or propose terms; to treat, bargain, parley.

\P 1596 SHAKES.  \textit{1 Hen. IV,} iii. ii 120 Percy, Northumberland,‥Mortimer, Capitulate against vs.    
\P 1618 SIR T. LAKE in \textit{Fortescue Papers} 38 He did not intend to capitulate with his Majesty.    
\P 1669 BAXTER  \textit{Call Unconv.} 247 Think not to capitulate with Christ, and divide your heart betwixt him and the world.    
\P 1697 W. DAMPIER  \textit{Voy.} (1729) I. 220 The Spaniards‥capitulated day after day to prolong time.    
\P 1748 RICHARDSON  \textit{Clarissa} (1811) VII. 344 It had the appearance of meanly capitulating with you.    
\P 1815 WELLINGTON in \textit{Gurw. Disp.} XII. 355 We must not capitulate with mutiny in any shape.    
\P 1816 SOUTHEY  \textit{Ess.} (1832) I. 322 Those magistrates‥who capitulated with the‥agricultural rioters, and‥acceded to the demands of a mob.

\itembf{b.} With various constructions: To make conditions, stipulate, agree. Obs.

\P 1580 NORTH  \textit{Plutarch} (1676) 965 Plemminius‥did capitulate with Lepidus to render up the Town.    
\P 1580 SIDNEY  \textit{Arcadia} iv. (1590) 432 To capitulate what tenements they should have.    
\P 1602 SEGAR  \textit{Hon. Mil. \& Civil} iii. xiii. 126 Two gentlemen capitulate to fight on horseback.    
\P 1715 DE FOE  \textit{Hon. \& Just.} (1841) 16, I capitulate for so much justice as to explain myself.    
\P 1818 M. W. SHELLEY  \textit{Frankenst.} iv, The man who thus capitulated for his safety.

\itembf{3.} trans. \textbf{a.} To make terms about, agree upon the terms of; to
formulate, arrange for, conclude. \textbf{b.} To make the subject of negotiation. Obs.

\P 1593 LODGE  \textit{Wm. Longbeard} F ij b, A peace lately capitulated betwixt Dagobert, kinge of France and Grimoald.
\P a1649 CHAS I. \textit{Wks.} 230 He had no Commission‥to capitulate anything concerning Religion.    
\P 1661 WEBSTER  \textit{Thracian Wonder} ii. i, How dare you, sir, capitulate the cause?

\itembf{4.} intr. To make terms of surrender; to surrender or yield on stipulated terms, in opposition to surrendering at discretion. The ordinary use; said of a general, force, garrison, fortress, town, etc.

\P 1689 LUTTRELL  \textit{Brief Rel.} (1857) I. 547 The 12th, the duke of Gourdon beat a parly, and desired to capitulate.    
\P 1705 \textit{Lond.  Gaz.} 4160/3 The Castle of Mittau began to capitulate the 14th instant.    
\P 1769 ROBERTSON  \textit{Chas.} V, V. v. 439 Want of provisions quickly obliged Trevulci to capitulate.    
\P 1874 BANCROFT  \textit{Footpr. Time} iii. 160 Washington‥after defending himself one day, capitulates.

\P 1714  \textit{Spect.} No. 566 \cardo{⁋}8, I still pursued, and, about two o'clock this afternoon, she thought fit to capitulate.    1841-4 Emerson Wks. (Bohn) I. 21, I am ashamed to think how easily we capitulate to badges and names.

\itembf{b.} trans. To surrender upon terms.

\P 1847 R. HAMILTON  \textit{Rew. \& Punishm.} vi. (1853) 264 We cannot capitulate the premises.    
\P 1870  \textit{Daily Tel.} 22 Sept., The new Minister‥seems‥disposed to the policy of capitulating France.

\vspace{0.1cm} \noindent
Hence \textbf{capitulated} ppl. a. \textbf{capitulating} vbl. n. and ppl. a.

\P 1586 J. FERNE  \textit{Blaz. Gentrie} 331 A Combate capitulated, that is to wit, a Combate, wherin are set downe‥diuers Articles or conditions, as to the manner of the battaile.    
\P 1654 EARL  OF ORRERY \textit{Parthenissa} (1676) 281 This capitulating Traytor.    
\P 1753 SMOLLETT  \textit{Ct. Fathom} (1784) 154/1 He put on his capitulating face.
\end{myenumerate}


%%%%%%%%%%%%%%%%%%%%%%%%%%%%%%%%%
\myitem{capricious} a.

\noindent \phonetic{(kəˈprɪʃəs)}

\noindent [ad. F. capricieux, ad. It. capriccioso (= Sp. caprichoso): see above. The by-form caprichious belongs to the corresp. forms of the n.]
\vspace{-0.3cm}

\begin{myenumerate}

\itembf{1.} Characterized by play of wit or fancy; humorous, fantastic, ‘conceited’. Obs.

\P 1594 CAREW  \textit{Huarte's Exam. Wits} 153 (L.) The inventive wits are termed in the Tuscan tongue capricious (capriciuso) for the resemblance they bear to a goat, who takes no pleasure in the open and easy plains, but loves to caper along the hill-tops.    
\P 1600 SHAKES.  \textit{A.Y.L.} iii. iii. 8, I am heere with thee, and thy Goats, as the most capricious Poet honest Ouid was among the Gothes.    
\P 1710 SHAFTESBURY  \textit{Charac.} (1737) III. 142 The capricious Point, and Play of Words.

\itembf{2.} Full of, subject to, or characterized by caprice; guided by whim or fancy rather than by judgement or settled purpose; whimsical, humoursome.

\P 1605 CAMDEN  \textit{Rem.} 57 A friend of his that knew him to be Caprichious.    
\P 1644 \textit{Eng. Tears} in \textit{Harl. Misc.} (Malh.) V. 450 The monstrous exorbitant liberty, that almost every capricious mechanick takes to himself.    
\P 1753 JOHNSON  \textit{Adventurer} No. 111 \cardo{⁋}6 Our estimation of birth is arbitrary and capricious.    
\P 1833 J. RENNIE  \textit{Alph. Angling} 49 We have known the salmon‥so capricious as often to prefer a fancy fly.    
\P 1884 \textit{Law Times Rep.} 10 May 325/1 The defendants' refusal was not capricious, but a bonâ fide exercise of their judgment.

\itembf{3.} transf. Of things: Subject to change or irregularity, so as to appear ungoverned by law.

\P 1823 LAMB  \textit{Elia Ser.} ii. vii. (1865) 283 The capricious hues of the sea, shifting like the colours of a dying mullet.    
\P 1830 LYELL  \textit{Princ. Geol.} (1875) II. ii. xlix. 617 The capricious distribution of coral reefs.    
\P 1874 HELPS  \textit{Soc. Press.} vi. 75 The vicissitudes of a capricious climate.    
\P 1875 TAIT  \& STEWART \textit{Unseen Univ.} iv. §118 To give to the atoms a perfectly arbitrary and capricious side movement.
\end{myenumerate}


%%%%%%%%%%%%%%%%%%%%%%%%%%%%%%%%%
\myitem{captious} a.

\noindent \phonetic{(ˈkæpʃəs)}

\noindent [ad. F. captieux or L. captiōs-us fallacious, sophistical, f. captiōn-em (see caption n.).]
\vspace{-0.3cm}

\begin{myenumerate}

\itembf{1.} Apt to catch or take one in; fitted to ensnare or perplex in argument; designed to entrap or entangle by subtlety; fallacious, sophistical.

\P 1447 O. BOKENHAM  \textit{Seyntys} 7 At Caimbrygge‥Where wyttys be manye ryht capcyows And subtyl.    
\P 1530 PALSGR.  307/1 Capcious, crafty in wordes to take one in a trap, captieux.    
\P 1548 UDALL, etc. \textit{Erasm. Par. Mark} ii. 23 a, Wherfore they went vnto Iesus, \& moued vnto hym this capcious question.    
\P 1677 GALE  \textit{Crt. Gentiles} II. iii. 31 Verbal, Captiose, Sophistic Questions.    
\P 1784 COWPER  \textit{Tirocinium} 903 A captious question, sir, and yours is one, Deserves an answer similar, or none.    
\P 1871 BLACKIE  \textit{Four Phases} i. 113 By captious questions to worm answers out of other people.

\itembf{b.} Crafty. Obs.

\P 1590 SWINBURN  \textit{Testaments} 147 This former kinde of disposition which by reason of the cunning condition appeareth to be made in hope of gaine, and is therefore properlie tearmed captious.    
\P 1608 TOPSELL  \textit{Serpents} 779 Spiders‥have given themselves‥to captious taking at advantage, watching and espying their prey.

\itembf{2.} Apt to catch at faults or take exception to actions; disposed to find fault, cavil, or raise objections; fault-finding, cavilling, carping.

\P 1380 WYCLIF  \textit{Serm.} Sel. Wks. II. 13 Þes wordis ben soþeli seid aȝens alle capcious men.    
\P 1538 COVERDALE  \textit{N.T.} Prol., The world is captious, and many there be that had rather find twenty faults, than to amend one.    
\P 1561 EDEN tr. \textit{Cortes' Arte de Navigar} Pref. ad fin., Enemies to vertue \& captious of other mens doinges.    
\P 1655 FULLER  \textit{Ch. Hist.} Pref., To cut off all occasions of Cavill from captious persons.    
\P 1804 \textit{Med. Jrnl.} XII. 359 The objections of the captious.    
\P 1865 TROLLOPE  \textit{Belton Est.} vi. 60 He was captious, making little difficulties, and answering him with petulance.

\itembf{3.} In various nonce-uses. \textbf{a.} Able to take in or contain, capacious. Obs.

\P 1601 SHAKES.  \textit{All's Well} i. iii. 208 Yet in this captious, and intenible Siue, I still poure in the waters of my loue And lacke not to loose still.

\itembf{b.} Alluring, taking, plausible. Obs.

\P 1776 SIR P. FRANCIS in \textit{Mem.} (1867) II. 55 The proposition was captious, and if made at an earlier period, might have been listened to by some of us.

\itembf{c.} humorous. ?

\P 1808 W. IRVING  \textit{Knickerb.} (1861) 134 Little captious short pipes, two inches in length, which‥could be stuck in one corner of the mouth.
\end{myenumerate}


%%%%%%%%%%%%%%%%%%%%%%%%%%%%%%%%%
\myitem{carnal} a.

\noindent \phonetic{(ˈkɑːnəl)}

\noindent [ad. L. carnāl-is fleshly (in Tertullian and other Christian writers), and frequent in med.L. as an attribute of relationship, as frater or soror carnalis, brother or sister by blood, in which use it appears in Eng. in 15th c. The theological sense appears equally early, but app. not in Wyclif. The Fr. repr. is charnel: see charnel.]
\vspace{-0.3cm}

\begin{myenumerate}

\itembf{1.} Of or pertaining to the flesh or body; bodily, corporeal. Obs.

\P c1470 HENRY Wallace xi. 1348 Bot  Inglissmen him seruit of carnaill fud.    
\P 1555 in Strype  \textit{Eccl. Mem.} III. App. xliv. 125 Look not you for it with carnal eyes.    
\P 1579 FULKE  \textit{Refut. Rastel} 745 The Lutheranes admitte the carnall presence.    
\P 1658 SIR T. BROWNE  \textit{Hydriot.} i. 22 Carnal Interrment or burying.    
\P 1847 tr. \textit{St. Aug. on Psalm} xlv. III. 240 The Church which coming from the Gentiles did not consent to carnal circumcision.

\itembf{2.} Related ‘in blood’, ‘according to the flesh’.

\P c1450 \textit{Merlin}  vii. 117 Noble knyghtes‥many of hem carnell frendes.    
\P 1490 CAXTON  \textit{How to Die} 8 His wyf, his chyldren, \& his frendes carnall.    
\P 1509 BARCLAY  \textit{Ship of Fooles} (1570) 181 Christ our Sauiour‥His carnall mother benignly did honour.    
\P 1598 HAKLUYT  \textit{Voy.} I. 66 Two carnall brothers.

\itembf{3.} Pertaining to the body as the seat of passions or appetites; fleshly, sensual.

\P 1400 \textit{Cov. Myst.} (1841) 84 Myghty soferauns of carnal temptacion.    
\P 1526 \textit{Pilgr.  Perf.} (W. de W. 1531) 148 b, Blynded with sensualite \& carnall pleasure.    
\P 1670 WALTON  \textit{Hooker} 33 The visible carnal sins of gluttony and drunkenness, and the like.    
\P 1829 SOUTHEY  \textit{All for Love} iv, To carnal wishes would it [Heaven] turn The mortified intent?

\itembf{b.} Sexual.

\P 1450 \textit{Merlin}  i. 17 That myght haue childe with-owte carnall knowynge of man.    
\P 1533 T. WILSON  \textit{Rhet.} 25 b, Without wedlocke and carnal copulation.    
\P 1667 MILTON \textit{P.L.} ix. 1013 That  false fruit‥Carnal desire inflaming.    
\P 1686 \textit{Col. Rec. Penn.} I. 176 He was accused of having Carnall Knowledge of his Brother in Law's woman Servant.

\itembf{4.} Not spiritual, in a negative sense; material, temporal, secular. arch.

\P 1483 [See CHARNEL].
\P c1510 BARCLAY  \textit{Mirr. Good Mann.} (1570) D ij a, Suche one in carnell troubles can no displeasour finde.    
\P 1611 BIBLE  \textit{Rom.} xv. 27 Their duetie is also to minister vnto them in carnall things.    
\P 1781 GIBBON  \textit{Decl. \& F.} xxviii. §5 III. 80 Judge whether Martin was supported by the aid of miraculous powers, or of carnal weapons.    
\P 1839 STONEHOUSE  \textit{Axholme} 207 [Wesley] began to doubt the utility, and even the lawfulness of carnal studies.

\itembf{b.} as n. in pl. ‘Carnal things’, temporal or worldly goods. [Rendering τὰ σαρκικά, or Vulg. carnalia, in Rom. xv. 27. 1 Cor. ix. 11.] Obs.

\P 1607 S. COLLINS  \textit{Serm.} (1608) 89 They haue aduanced‥the spirtualls of other men, with the loss‥of their own carnalls.    
\P 1625 BURGES  \textit{Pers. Tithes} 10 Euery man‥that is made partaker of the Minister's Spirituals, must render Carnals.    Ibid. 14 Spirituals doe well deserue carnals.

\itembf{5.} Not spiritual, in a privative sense; unregenerate, unsanctified, worldly.

\P 1510 MORE  \textit{Picus} Ded., All faithfull people are rather spirituall then carnall.    
\P 1526 TINDALE  \textit{Rom.} vii. 14 The lawe is spirituall, but I am carnall [Wyclif fleischli].    
\P 1611 BIBLE  \textit{Rom.} viii. 7 The carnall minde is enmitie against God.    
\P 1667 MILTON  \textit{P.L.} xi. 212 Had not doubt And carnal fear that day dimm'd Adams eye.    
\P 1712 ADDISON  \textit{Spect.} No. 494 \cardo{⁋}1 To abstain from all Appearances of Mirth and Pleasantry, which were looked upon as the Marks of a Carnal Mind.    
\P 1865 MOZLEY  \textit{Mirac.} iii. 65 To a carnal imagination an invisible world is a contradiction in terms—another world besides the whole world.

\itembf{6.} Carnivorous; fig. bloody, murderous. Obs.

\P 1594 SHAKES.  \textit{Rich III}, iv. iv. 56 This carnall curre Preyes on the issue of his mothers body.

\itembf{7.} Comb., as \textbf{carnal-minded} adj., \textbf{carnal-mindedness};
\textbf{carnal securitan} [f. carnal security; sense 5], etc.

\P 1664 H. MORE  \textit{Antid. Idol.} x. 123 Abusing the credulous and *carnal-minded.

\P 1607 HIERON  \textit{Wks.} I. 105 This must needes condemne our *carnall mindednesse.    
\P 1849 HARE  \textit{Par. Serm.} (1849) II. 30 Spiritual pride‥is apt to settle down into carnalmindedness.

\P 1627 BERNARD  \textit{Isle of Man} 18 One Mr. Outside, in the inside a *carnall Securitan, a fellow that will come to his Church.

\P 1655 FULLER  \textit{Ch. Hist.} ix. 112 A most *carnall-spirituall exposition.

\P 1818 SCOTT  \textit{Hrt. Midl.} xii, This *carnal-witted scholar, as he had in his pride termed Butler.
\end{myenumerate}


%%%%%%%%%%%%%%%%%%%%%%%%%%%%%%%%%
\myitem{carnivorous} a.

\noindent \phonetic{(kɑːˈnɪvərəs)}

\noindent [f. L. carnivor-us (f. carni- flesh + -vorus devouring) + -ous.]
\vspace{-0.3cm}

\begin{myenumerate}

\itembf{1.} Eating or feeding on flesh; applied to those animals which naturally prey on other animals, and spec. to the order Carnivora.

\P 1646 SIR T. BROWNE  \textit{Pseud. Ep.} iv. x, Many there are‥which eate no salt at all, as all carnivorous animals.    
\P 1664 POWER  \textit{Exp. Philos.} i. 6 In all Flyes, more conspicuously in Carnivorous or Flesh-Flyes.    
\P 1797 T. BEWICK  \textit{Brit. Birds} (1847) I. Introd. 9 Birds may be distinguished, like quadrupeds, into granivorous and carnivorous.    
\P 1833 MRS. BROWNING  \textit{Prometh. Bound}, Poems (1850) I. 187 Zeus's winged hound, The strong carnivorous eagle.    
\P 1845 DARWIN  \textit{Voy. Nat.} i. (1852) 34 The carnivorous beetles or Carabidæ.    
\P 1879 WALLACE  \textit{Australasia} iii. 56 Carnivorous marsupials preying upon the other groups.

\itembf{2.} Bot. Applied to those plants which absorb and digest animal substances as food.

\P 1868 \textit{Sci. Opinion} i. 16 The highly interesting carnivorous plants.    
\P 1878 MCNAB  \textit{Bot.} iv. (1883) 95 Some plants‥obtain a part of [their nitrogenous food] in a peculiar manner. These are the so-called carnivorous plants.

\itembf{3.} Med. Applied to caustics as destructive of flesh.

\P 1881 in \textit{Syd. Soc. Lex.}

\vspace{0.1cm}\noindent
Hence \textbf{carnivorously} adv., \textbf{carnivorousness.}

\P 1837 MARRYAT  \textit{Dog-Fiend} xxxviii, The sow‥was carnivorously inclined.    
\P 1858 HOGG  \textit{Life Shelley} II. 446 He dined carnivorously.    
\P 1856 \textit{Chamb. Jrnl.} V. 133 Carnivorousness is an aberration of humanity, and a semi-return to the diet of beasts.
\end{myenumerate}


%%%%%%%%%%%%%%%%%%%%%%%%%%%%%%%%
\myitem{carp} v.1

\noindent \phonetic{(kɑːp)}

\noindent [Senses 1–3, chiefly in northern poetry (especially in alliterative verse), were probably a. ON. karpa to brag; but the later prose senses 4–6 appear to be derived from, or influenced by, L. carpĕre to pluck, fig. to slander, calumniate. The ulterior history of the ON. word is uncertain.]
\vspace{-0.3cm}

\begin{myenumerate}

\itembf{1.} intr. To speak, talk. Obs.

\P a1240 WOHUNGE in \textit{Cott. Hom.} 287 Carpe toward ihesu and seie þise wordes.
\P a1300  \textit{Cursor M.} App. Resurrect. 388 Als þai come narre þe castelle, to-geder carpand.
\P c1400  \textit{Destr. Troy} 829 The Kyng þan full curtesly karpes agayne.    
\P 1420 \textit{Siege Rouen} 1235 in \textit{Archæol.} XXII. 381 Vnnethe thay myȝt brethe or carpe.    
\P 1470 HARDING  \textit{Chron. Proem.} x, Leonell‥that wedded‥The erles daughter of Vister, as man do Karpe.    
\P 1570 LEVINS  \textit{Manip.} 33/3 To carpe, talke, colloqui, confabulari.    
\P 1575 TURBERV.  \textit{Bk. Falconrie Epil.} Aa iij, To carpe it fine with those that haue no guile.

\itembf{b.} To discourse of, in speech or writing. Obs.

\P 1350 \textit{Will.  Palerne} 216 Þe kowherdes bestes i carped of bi-fore.    
\P 1393 LANGL.  \textit{P. Pl.} C. xxii. 199 Thus conscience of crist and of þe croys carpede.  
\P c1425 WYNTOUN  \textit{Cron.} iii. Prol. 26 (Jam.) Of thame‥Carpe we bot lityl.
\P a1605 MONTGOMERIE  \textit{Flyting} 575 Of his conditions to carp for a while.

\itembf{2.} trans. To speak, utter, say, tell. Obs.

\P 1350 \textit{Will.  Palerne} 503 To karp þe soþe.    
\P 1393 GOWER  \textit{Conf.} III. 325 To carpe Proverbes and demaundes sligh.
\P c1400  \textit{Destr. Troy} 4610 When Calcas his counsell had carpit to the end.    
\P 1515 \textit{Sc. Field} 73 in Furniv. Percy Folio I. 216 Our Knight full [of] courage carpeth these words.

\itembf{3.} intr. To sing or recite (as a minstrel); to sing (as a bird). Obs.

\P c1425 \textit{Thomas  of Erceld.} 313 ‘To harpe or carpe, whare$\sim$so þou gose, Thomas, þou sall hafe þe chose sothely’: And he saide ‘harpynge kepe I none, For tonge es chefe of mynstralsye’.    
\P 1515 BARCLAY  \textit{Egloges} iv. (1570) C iv/2 In goodly ditie or balade for to carpe.
\P a1528 SKELTON  \textit{Agst. comely Coyst.} 13 In his gamut carp he can.
\P c1570 THYNNE  \textit{Pride \& Lowl.} (1841) 8 Many was the bird did sweetly carpe Among the thornes.    
\P 1802 LOCHMABEN  \textit{Harper} vii. in Scott Minstr. Scott. Bord. (1869) 94 Then aye he harped, and aye he carped Till a' the lordlings footed the floor.

\itembf{4.} Vituperatively: To talk much, to prate, chatter. Cf. carper. Obs.

\P 1377 LANGL.  \textit{P. Pl.} B. x. 69 Clerkes‥carpen of god faste, and haue [him] moche in þe mouthe.
\P a1528 SKELTON  \textit{Col. Cloute} 549 Some‥Clatter \& carpe Of that heresy.    
\P 1530 PALSGR. 476/1, I carpe (Lydgate), Je carquette‥This is a farre northen verbe.    
\P 1557 \textit{Praise  Maistr. Ryce} in \textit{Tottel's Misc.} (Arb.) 202 Came Curiousness and carped out of frame.

\itembf{5.} spec. To talk querulously, censoriously, or captiously; to find fault, cavil. (The current sense.)
   (Certain examples of this before the 16th c. are wanting: the early ones may have merely the sense of 1 with contextual colouring. Cf. carper.)

\P [1377 LANGL.  \textit{P. Pl.} B. x. 286 Abasshed To blame yow or to greve, And carpen noght as they carpe now, Ne calle yow dumbe houndes.    
\P 1401 \textit{Pol. Poems} (1859) II. 77 Thou carpist also of oure coveitise, and sparist the sothe.    
\P 1515 BARCLAY  \textit{Egloges} i. (1570) A j, Some in Satyres against vices dare carpe.]    
\P 1548 \textit{Soul  John-Nobody} in \textit{Strype Cranmer} (1694) App. 139 They will currishly carp.    
\P 1561 T. NORTON  \textit{Calvin's Inst.} i. xiii. (1634) 49 Servetto carpeth, that God did beare the person of an Angell.    
\P 1655 DIGGES  \textit{Compl. Ambass.} 377 The King‥carpeth upon the marriage.
\P a1677 BARROW  \textit{Serm. Malice of Soc.}, In carping and harshly censuring‥their neighbours.    
\P 1785 BURNS  \textit{2nd Ep. Lapraik}, Ne'er grudge an' carp, Tho' fortune use you hard an' sharp.    
\P 1863 MRS. C. CLARKE  \textit{Shaks. Char.} xv. 386 The bulk of society did not assemble to carp and to cavil.

\itembf{b.} Const. at.

\P 1586 THYNNE  \textit{Contn. Holinshed} Pref., Curiouslie carping at my barrennes in writing.    
\P 1794 BURKE  \textit{Corr.} IV. 235 That faction and malice may not be able to carp at it.    
\P 1879 M. ARNOLD  \textit{Falkland Mixed Ess.} 207 We will not carp at this great writer.

\itembf{6.} trans. To find fault with, reprehend, take exception to. Obs.

\P 1550 CRANMER  \textit{Sacrament} 100 a, Whiche my saiyng diuers ignorant persones‥did carpe and reprehende.    
\P 1582 N. T. (RHEM.)  \textit{Luke} vii. marg., The Pharisees did alwaies carpe Christ.    
\P 1598 R. GRENEWEY  \textit{Tacitus Ann.} v. ii. (1622) 117 Couertly carping the Consull Fufius.    
\P 1605 CAMDEN  \textit{Rem.} (1637) 230 Carping whatsoever hath been done or said heretofore.    
\P 1678 R. BARCLAY  \textit{Apol. Quakers} iii. §vii. 87 Our Adversaries shall have nothing from thence to carp.

\itembf{7.} intr. (?) To censure; to judge, discriminate.

\P 1591 \textit{Troub.  Raigne K. John} (1611) 21 Any one that knoweth how to carpe, Will scarcely iudge us both one countrey borne.

\itembf{8.} (?) To contend, fight. Obs. rare.

\P 1535 STEWART  \textit{Cron. Scot.} I. 606 With brandis bricht that scherand wer and scharp So cruellie togidder did tha carp.

\vspace{0.1cm} \noindent
Associated with CARK, q.v.

\P 1465 \textit{Chevy  Chace} ii. 135 Tivydale may carpe off care.    
\P 1522 \textit{World  \& Child} in Hazl. \textit{Dodsley} I. 267 Ever he is carping of care.    
\P 1670 G. H.   \textit{Hist. Cardinals} i. ii. 49 Poor drudgeing‥Priests that carp and moyl all day long.    
\P 1702 \textit{Eng.  Theophrast.} 312 Carping for the unprofitable goods of this world.
\end{myenumerate}


%%%%%%%%%%%%%%%%%%%%%%%%%%%%%%%%
\myitem{carrion} n. (and a.)

\noindent \phonetic{(ˈkærɪən)}

\noindent [ME. caronye, caroine, a. ONF. \phonetic{caˈronië}, later caroine, caroigne, in central OF. charoigne (mod.F. charogne, and in other sense carogne, Picard carone, carongne) = Pr. caronha, It. carogna, Sp. carroña, pointing to a Romanic type *carōnia, supposed to be a deriv. of caro flesh, but not regularly formed on the stem carn-. The phonetic history of the English β. and δ. forms is obscure.]
\vspace{-0.3cm}

\begin{myenumerate}

\itembf{A.} n.

\itembf{1.} a.A.1.a A dead body; a corpse or carcass. Obs.

\P a1225 \textit{Ancr.  R.} 84 Þe bacbitare‥bekeð mid his blake bile o cwike charoines as þe þet is þes deofles corbin of helle.    
\P 1297 R. GLOUC.  265 [They] slowe‥eyȝte hondred \& fourty men, \& her caronyes [v.r. caroines] to drowe.
\P a1300  \textit{Cursor M.} 22906 Ded þar gun his [a lion's] caroigne [v.r. carion, caroyne, careyn] li.
\P c1308 POL.  \textit{Songs} (1839) 203 A vilir caraing nis ther non.    
\P 1382 WYCLIF  \textit{Hebr.} iii. 17 Whos careyns ben cast down in desert.
\P c1386 CHAUCER \textit{Knt.'s T.} 1157 The  careyne [v.r. careyn, caroyne, karoigne, caroigne] in the busk with throte ycorue. 
\P c1440  \textit{Promp. Parv.} 61 Caranye or careyn, cadaver.    
\P 1494 FABYAN v. cxxiv. 102 Ye cource of the riuer was let by the multitude of the caryens or dede bodyes.    
\P 1590 L. LLOYD  \textit{Diall Daies} Oct. 51 The raven‥returned not, but fed upon the carrens.
\P c1645 HOWELL  \textit{Lett.} I. i. xx, Dogs which‥eat the Carrens.    
\P 1718 \textit{Free-thinker}  No. 47. 342 The Raven‥stay'd to prey upon the Carrions of the Dead.    
\P 1763 C. JOHNSTON  \textit{Reverie} II. 235 They all flocked about him, croaking like so many ravens about a carrion.

\itembf{b.} = Applied to a dead man or corpse that ‘walks’ or returns to earth. Obs.

\P 1430 LYDG.  \textit{Min. Poems} (1840) 143 Blissid Austyn the careyn gan compelle, ‘In Jhesu name‥What that thu art trewly for to telle’.    
\P 1483 CAXTON  \textit{Gold. Leg.} 174/3 Thenne the caryon broughte hym thyder to the graue.

\itembf{2. a.} Dead putrefying flesh of man or beast; flesh unfit for food, from putrefaction or inherently.

\P 1297 R. GLOUC.  (Rolls) 6544 Þo ne vond he atte laste Noȝt of hom bote caroyne.
\P a1340 HAMPOLE  \textit{Psalter} cxlvi. 10 Þe deuyl‥fedis þaim wiþ karyun.
\P c1400  \textit{Destr. Troy} 1972 Caste  vnto curres as caren to ete.    
\P 1430 LYDG.  \textit{Chron. Troy} i. vii, Whan a beast is tourned to careine.
\P c1510 MORE  \textit{Picus} Wks. 25 Vile carein and wretched wormes meate.    
\P 1557 NORTH  \textit{Gueuara's Diall Pr.} (1619) 698/2 The wormes in carring.    
\P 1791 WOLCOTT  (P. Pindar) \textit{Remonstr.} Wks. 1812 II. 457  Like flies in Carrion.    
\P 1837 M. DONOVAN  \textit{Dom. Econ.} II. 127 The vulture‥feeds on putrid carrion.

\itembf{b.} ? = Death. Obs.

\P 1387 TREVISA  \textit{Higden} iv. xxxiii, Þerof cometh tweie manere of careyns, for we beeþ i-slowe wiþ wepoun, oþer we beeþ adreent. [Hence 1494 in Fabyan.]    
\P 1481 CAXTON  \textit{Myrr.} i. v. 18 They come the sooner to their ende and to carayne.

\itembf{3.} transf. \textbf{a.} Used (contemptuously) of a living human body; cf. CARCASS (? obs.). \textbf{b.} The fleshly nature of man, ‘the flesh’ in the Pauline sense (obs.).

\P 1377 LANGL.  \textit{P. Pl.} B. xiv. 331 Ne noyther sherte ne shone‥To keure my caroigne.
\P 1450  \textit{Knt. de la Tour} xxvii. (1868) 39 To aorne suche a carion as is youre body.    
\P 1491 CAXTON  \textit{Vitas Patr.} (W. de W.) i. xxxv. 31 a, To leue thy careyne and folowe Ihesu Cryste.    
\P 1549 \textit{Compl.  Scotl.} xvii. 154 Our carions ande corporal natur‥is baytht vile ande infekkit.    
\P 1596 SHAKES.  \textit{Merch. V.} iii. i. 38 Shy. My owne flesh and blood to rebell. Sol. Out vpon it old carrion, rebels it at these yeeres.    
\P 1832 H. MARTINEAU  \textit{Demerara} ii. 27 Much good may your tender mercies do your carrion.

\itembf{4.} Used (contemptuously) of a living person, as no better than carrion. Obs.

\P 1601 SHAKES.  \textit{Jul. C.} ii. i. 130 Priests and Cowards, and men Cautelous, Old feeble Carrions.    
\P 1661 PEPYS  \textit{Diary} 15 Sept., Pegg Kite‥will be‥a troublesome carrion to us executors.

\itembf{5.} Used of animals: sometimes app. in sense ‘noxious beast’, ‘vermin’; sometimes merely ‘poor, wretched, or worthless beast’. Obs.

\P 1477 EARL RIVERS  (Caxton) \textit{Dictes} 142 The euill creatures ben wors than serpentes, lyons or caraynes.    
\P 1562 J. HEYWOOD  \textit{Prov. \& Epigr.} (1867) 119 Daws ar carren.    
\P 1573 TUSSER  \textit{Husb.} xvi. (1878) 35 Let carren \& barren be shifted awaie, For best is the best, whatsoever you paie.    
\P 1634 W. WOOD  \textit{New Eng. Prosp.} i. vi, The beasts of offence be Squunckes, Ferrets, Foxes.    Ibid. i. viii, Having shewed you the most offensive carrions that belong to our Wildernesse.
\P a1639 W. WHATELY  \textit{Prototypes} i. xix. (1640) 227 They [dogs and monkeys] be paltry carrions.

\itembf{6.} fig. Anything vile or corrupt; †corrupt mass; ‘garbage’, ‘filth’.

\P 1524 S. FISH  \textit{Supplic. Begg.} 18 Declaring suche an horrible carayn of euyll ageinst the ministres of iniquite.    
\P 1597 \textit{1st Pt. Return Parnass.} v. i. 1455, I woulde prove it upon that carrion of thy witt.    
\P 1845 CARLYLE  \textit{Cromwell} (1873) I. 21 Flunkyism, falsity and other carrion ought to be buried!    
\P 1870 EMERSON  \textit{Soc. \& Sol.}, Courage Wks. (Bohn) III. 113 Melancholy sceptics with a taste for carrion, who batten on the hideous facts in history.    
\P 1879 FROUDE  \textit{Cæsar} xxiii. 402 note, Roman fashionable society hated Cæsar, and any carrion was welcome to them which would taint his reputation.

\itembf{B.} attrib. passing into adj.

\itembf{1. a.} Consisting of, or pertaining to, corrupting flesh. (Usually with some notion of contempt.)

\P a1535 MORE  \textit{De quat. Noviss.} Wks. 101 No man findeth fault, but carrieth his carien corse into ye quere, and‥burieth ye body boldly at the hie alter.    
\P 1583 STANYHURST  \textit{Æneis} iii. (Arb.) 77 A stincking Foule carrayne sauoure.
\P c1613 ROWLANDS  \textit{More Knaves} 30 Some carion beast, Whereon the Rauens and the crowes doe feast.    
\P 1860 PUSEY  \textit{Min. Proph.} 454 The carrion-remains should be entombed only in the bowels of vultures and dogs.

\itembf{b.} As an epithet of Death personified; also of Charon. Obs.

\P 1566 W. ADLINGTON  \textit{Apuleius} 62 Deliver to carraine Charon one of the halfepens, which thou bearest, for thy passage.    
\P 1587 \textit{Mirr.  Mag. Q. Cordila} xlvii. 4 By hir elbowe carian death for me did watch.    
\P 1576 \textit{Parad.  Daynty Dev.} (N.) Seeing no man then can death escape‥We ought not feare his carraine shape.    
\P 1596 SHAKES.  \textit{Merch. V.} ii. vii. 63 A carrion death, Within whose emptie eye there is a written scroule.

\itembf{2.} Applied in contempt to the living human body, as no better than carrion (cf. 3).

\P 1537 \textit{Surr.  Northampton Priory} in Prance \textit{Addit. Narr. Pop. Plot} (1679) 36 In continual ingurgitations and farcyngs of our carayne Bodies.    
\P 1563 \textit{Homilies}  ii. Excess Appar. (1859) 316 Why pamperest thou that carreyne flesh so hye?    
\P 1577 STANYHURST  \textit{Desc. Irel.} in \textit{Holinshed} VI. 14 By the imbalming of their carian soules with the sweet and sacred flowers of holie writ.    
\P 1606 SHAKES.  \textit{Tr. \& Cr.} iv. i. 71 For euery scruple Of her contaminated carrion weight.

\itembf{3. a.} Carrion-lean, skeleton-like. Obs. \textbf{b.} Rotten; vile, loathsome; expressing disgust.

\P 1565 HARDING  \textit{Confut. Apol.}, Ye will haue your spiritual Bankets so leane and Carrien.    
\P 1580 HOLLYBAND  \textit{Treas. Fr. Tong.}, Eslance, as chevaux eslancez, carren horses.    
\P 1645–6 Evelyn \textit{Diary} 28 Jan., My base, unlucky, stiffnecked trotting carrion mule.    
\P 1653 H. COGAN  \textit{Pinto's Trav.} xxii. §3. 79 Mounted on horses, or to say better, on lean carrion Tits that were nothing but skin and bone.    
\P 1826 in Cobbett \textit{Rur. Rides} (1885) II. 82 The foul, the stinking, the carrion baseness, of the fellows that call themselves ‘country gentlemen’.    
\P 1867 \textit{N. \& Q.}  Ser. iii. XI. 32/2 Then she called me all sorts o' carrion names.

\itembf{C.} Comb. \textbf{a.} attributive with sense ‘having to do with, feeding on carrion’, as carrion-bird, carrion-chafer, carrion-fly, carrion-hawk, carrion-kite, carrion-raven, carrion-vulture; b.C.b objective and instrumental, as carrion-feeder, carrion-nosing ppl. adj., carrion-strewn pa. pple.; c.C.c similative, as carrion-like adj. or adv., carrion-scented ppl. adj. Also carrion-beetle, any beetle of the family Silphidæ, which feed on carrion; carrion-flower, a name for the genus Stapelia, also for Smilax herbacea, from the scent of their blossoms; †carrion-lean a., lean as a wasting corpse or skeleton; fig. meagre, very deficient; †carrion-row, a place where inferior meat or offal was sold. Also carrion crow.

\P 1817 KIRBY  \& SPENCE \textit{Entomol.} II. xxi. 242 Those unclean feeders, the *carrion beetles (Silphæ, L.)‥are at the same time very fetid.    
\P 1959 E. F. LINSSEN  \textit{Beetles} I. 159 Burying beetles, carrion beetles, rove beetles, etc.

\P 1839 THIRLWALL  \textit{Greece} III. 137 Neither dogs, nor *carrion-birds, would touch them‥so long as the pestilence lasted.

\P 1816 KIRBY  \& Sp. \textit{Entomol.} (1828) II. xxiv. 386 The *carrion-chafers, and others of the lamellicorn beetles.

\P 1855 J. F. W.  JOHNSTON \textit{Chem. Com. Life} I. 332 The Stapelias are called *carrion-flowers because of the disagreeable putrid odours they exhale.    
\P 1852 THOREAU  \textit{Summer} (1884) 1/23 The Smilax herbacea, carrion flower, a rank green vine‥It smells exactly like a dead rat in the wall, and apparently attracts flies like carrion.

\P 1787 BEST  \textit{Angling} (ed. 2) 114 The Oak, Ask, Woodcock, *Carion or Down hill fly comes on about the sixteenth of May.    
\P 1796 WOLCOTT  (P. Pindar) \textit{Sat. Wks.} 1812 III. 395 Court-sycophants, the Carrion-flies.    
\P 1861 HULME tr. \textit{Moquin-Tandon} ii. iv. i. 241 Larvæ of the carrion fly.

\P 1581 T. HOWELL  \textit{Deuises} (1879) 234 Art thou so fond, with *carren kyte to haunt.

\P 1542 UDALL  \textit{Erasm. Apophth.} 245 b, Because it was so *caren leane.    
\P 1554 J. PROCTER tr. \textit{Vincentius} To Rdr., How owgle and carrion-lean ye are to se.    
\P 1581 J. BELL  \textit{Haddon's Answ. Osor.} 135 So carrion leane in the knowledge of Scriptures.    
\P 1602 W. FULBECKE  \textit{1st Pt. Parall.} 74 It is better to haue a declaration too copious then carion-leane.    
\P 1710  \textit{Brit. Apollo} III. 18. 2/1 He is so Carrion-lean.

\P 1620 VENNER  \textit{Via Recta} viii. 189 It maketh them *carran$\sim$like leane.

\P 1878 TENNYSON  \textit{Q. Mary} iv. iii. 171 The *carrion-nosing mongrel.

\P 1589 COOPER  \textit{Admon.} 140 As *carren Rauens flye‥to stinking carcasses.

\P 1728 SWIFT  \textit{Answ. Memorial} Wks. 1755 V. II. 173 The district in the several markets, called *carrion-row.

\P 1829 SCOTT  \textit{Anne of G.} ii, The huge *carrion vulture floated past him.
\end{myenumerate}


%%%%%%%%%%%%%%%%%%%%%%%%%%%%%%%%%
\myitem{castigate} v.

\noindent \phonetic{(ˈkæstɪgeɪt)}

\noindent [f. L. castīgāt- ppl. stem of castīgā-re to chastise, correct, reprove (f. castus pure, chaste) + -ate3. See chastise.]
\vspace{-0.3cm}

\begin{myenumerate}

\itembf{1.} trans. To chastise, correct, inflict corrective punishment on; to subdue by punishment or discipline, to chasten; now usually, to punish or rebuke severely.

\P 1607 SHAKES.  \textit{Timon} iv. iii. 240 If thou didst put this soure cold habit on To castigate thy pride, 'twere well.    
\P 1665 GLANVILL  \textit{Sceps. Sci.} 167 He‥that cannot castigate his passions.    
\P 1865 MOZLEY  \textit{Mirac.} vii. 291 It has only‥castigated and educated the belief, and not destroyed it.    
\P 1873 H. SPENCER  \textit{Stud. Sociol.} vii. 170 Daily we castigate the political idol with a hundred pens.    
\P 1878 S. COX  \textit{Salv. Mundi} vi. (ed. 3) 142 Discipline by which they should be castigated for their sins.

\itembf{2.} To correct, revise, and emend (a literary work).

\P 1666 EVELYN  \textit{Mem.} (1857) III. 190 Seneca's tragedies‥have‥been castigated abroad by several learned hands.
\P a1742 BENTLEY  \textit{Lett.} 237 He had adjusted and castigated the then Latin Vulgate to the best Greek exemplars.

\itembf{3.} transf. To chasten or subdue (in intensity).

\P 1653 H. MORE  \textit{Conject. Cabbal.} (1713) 174 Morning is‥a parcel of that full Day which was first created, and is castigated and mitigated by its conjunction with the dark Matter into a moderate Matutine Splendour.    
\P 1662 GLANVILL  \textit{Lux Orient.} xiv. (T.) Being so castigated, they are duly attempered to the more easy body of air again.    
\P 1669 W. SIMPSON  \textit{Hydrol. Chym.} 112 If the narcotick Sulphur was castigated.

\noindent
Hence \textbf{castigated} ppl. a., chastened.

\P 1728 YOUNG  \textit{Love Fame} v. (1757) 136 The modest look, the castigated grace.    
\P 1784 J. BARRY  \textit{Lect. Art} vi. (1848) 228 This happily castigated style of design.    
\P 1787 BURNS  \textit{Unco Guid} iv, When your castigated pulse Gies now and then a wallop.
\end{myenumerate}


%%%%%%%%%%%%%%%%%%%%%%%%%%%%%%%%
\myitem{casuistry} n.

\noindent \phonetic{(ˈkæzjuːɪstrɪ, ˈkæʒ(j)uː-)}

\noindent [f. casuist + -ry. App. at first contemptuous = ‘the casuist's trade’; cf. sophistry, Jesuitry, foolery. A term of more respectful application would prob. have been casuism: Fr. has la casuistique, as if ‘casuistics’.]

\begin{myenumerate}
\itembf{1.} The science, art, or reasoning of the casuist; that part of Ethics which resolves cases of conscience, applying the general rules of religion and morality to particular instances in which ‘circumstances alter cases’, or in which there appears to be a conflict of duties. Often (and perhaps originally) applied to a quibbling or evasive way of dealing with difficult cases of duty; sophistry.

\P 1725 POPE  \textit{Rape Lock} v. 122 Cages for gnats‥and tomes of casuistry.    
\P 1736 BOLINGBROKE  \textit{Patriot.} (1749) 170 Casuistry‥destroys, by distinctions and exceptions, all morality, and effaces the essential difference between right and wrong.    
\P 1836 PENNY  \textit{Cycl.} VI. 359 The science of casuistry‥has been termed not inaptly the ‘art of quibbling with God’.    
\P 1841 EMERSON  \textit{Lect. the Times} Wks. (Bohn) II. 254 The Temperance-question‥is a gymnastic training to the casuistry and conscience of the time.    
\P 1862 MILL  \textit{Utilit.} 37 Self-deception and dishonest casuistry.    
\P 1887 FOWLER  \textit{Princ. Morals} ii. vi. 247 Granted that duties may clash, or that general rules may be modified by special circumstances, it is surely most important to determine beforehand, as far as we can, what those circumstances are, and, in the case of clashing duties, which should yield to the other. Now this, and this alone, is the task which ‘Casuistry’ or the attempt to ‘resolve cases of conscience’ proposes to itself.

\itembf{2.} A register or record of (medical) cases.

\P 1883 J. W. LEGG  in \textit{Barthol. Hosp. Rep.} XIX. 202 Nor can I find any similar case in the casuistry of pemphigus as recorded in the year-books.
\end{myenumerate}





%%%%%%%%%%%%%%%%%%%%%%%%%%%%%%%%
\myitem{cataclysm} n.

\noindent \phonetic{(ˈkætəklɪz(ə)m)}

\noindent [a. F. cataclysme (16th c. in Littré), ad. Gr. κατακλυσµός deluge (also fig.), f. κατα-κλύζειν to deluge, f. κατά down + κλύζ-ειν to wash, dash as a wave.]
\vspace{-0.3cm}

\begin{myenumerate}

\itembf{1.} A great and general flood of water, a deluge; esp. the Noachian deluge, the Flood.
   In Geol. resorted to by some as a hypothesis to account for various phenomena; hence used vaguely for a sudden convulsion or alteration of physical conditions.

\P 1637 HEYWOOD  \textit{Roy. Ship} 3 More soules‥then perisht in the first Vniversall Cataclisme.    
\P 1660 R. COKE  \textit{Power \& Subj.} 91 Mankind sinned Malitiously, before God brought the general cataclysme upon them.    
\P 1833 LYELL  \textit{Princ. Geol.} III. 101 For the proofs of these general cataclysms we have searched in vain.    
\P 1878 H. M. STANLEY  \textit{Dark Cont.} II. ii. 52 The accumulated waters‥will sweep through the ancient gap with the force of a cataclysm.    
\P 1879 tr. \textit{Haeckel's Evol. Man} I. iv. 77 The hypothesis usually called the Theory of Cataclysms or Catastrophes.

\itembf{2.} fig.; esp. a political or social upheaval which sweeps away the old order of things.

\P 1633 \textit{True  Trojans} ii. 1 in Hazl. Dodsley XII. 468 Ready to pour down cataclysms of blood.    
\P 1633 T. ADAMS  \textit{Exp. 2 Peter} ii. 6 Heaven rained on them great cataclysms of flames.    
\P 1861  \textit{Sat. Rev.} 20 July 67 That the Indian army surgeons will be swept away in the general cataclysm.    
\P 1882 J. H. BLUNT  \textit{Ref. Ch. Eng.} II. 108 In the general upheaval of doctrine‥during the Reformation cataclysm.
\end{myenumerate}


%%%%%%%%%%%%%%%%%%%%%%%%%%%%%%%%%
\myitem{catharsis} Med.

\noindent \phonetic{(kəˈθɑːsɪs)}

\noindent [mod.L., a. Gr. κάθαρσις cleansing, purging, f. καθαίρειν to cleanse, purge, f. καθαρός clean.]
\vspace{-0.3cm}

\begin{myenumerate}

\itembf{a.} Purgation of the excrements of the body; esp. evacuation of the bowels.

\P 1803  \textit{Med. Jrnl.} IX. 418 Causing vomiting, catharsis, or diabetes.    
\P 1875 H. WOOD  \textit{Therap.} (1879) 449 The production of catharsis is the surest mode of relief in general dropsy.

\itembf{b.} The purification of the emotions by vicarious experience, esp. through the drama (in reference to Aristotle's Poetics 6). Also more widely.

[\P 1867 J. A. SYMONDS  \textit{Let.} 22 Aug. (1967) I. 751 The world desiderates now‥a trilogy, whereof the whole third part shall exhibit ‘the height, the space, the gloom, the glory’, of ultimate final and perfect κάθαρσις.]    
\P 1872 G. S. MORRIS tr. \textit{Ueberweg's Hist. Phil.} I. i. 179 Aristotle can not have meant‥to exclude from among the effect of the Tragedy, its effect as‥ethical discipline. With the ‘Catharsis’‥are‥joined‥the other effects of the same,—the latter effects flow from the ‘Catharsis’.    
\P 1897 COSTELLOE  \& MUIR tr. \textit{Zeller's Philos. Greeks} II. xv. 311/2 According to Aristotle there is a kind of music which produces a catharsis, although it possesses no ethical value‥—namely, exciting music.    
\P 1904 DOWDEN  \textit{Browning} 289 Balaustion, stricken at heart, yet feels that this tragedy of Athens brings the tragic katharsis.    
\P 1920 D. H. LAWRENCE  \textit{Touch \& Go} iii. i. 72 It's a cleansing process—like Aristotle's Katharsis. We shall hate ourselves clean at last, I suppose.    
\P 1924 L. COOPER  \textit{Aristotelian Theory Com.} 180 Aristotle‥would recognize some sort of catharsis, and the resultant pleasure, to be the proper end of comedy.    
\P 1924 W. B. SELBIE  \textit{Psychol. Relig.} 159 There may‥be cases where experiences of this kind produce a moral catharsis which has good results.    
\P 1959 \textit{Chamber's  Encycl.} I. 592/1 The word catharsis (purgation), in which he [sc. Aristotle] summed up the emotional effect of tragedy, has also received much fanciful interpretation; in reality it is a medical term, with no directly moral or spiritual implications.

\itembf{c.} Psychotherapy. The process of relieving an abnormal excitement by re-establishing the association of the emotion with the memory or idea of the event which was the first cause of it, and of eliminating it by abreaction.

\P 1909 A. A. BRILL in \textit{Freud's Sel. Papers Hysteria} 6 The German abreagiren‥has different shades of meaning, from defense reaction to emotional catharsis.    
\P 1951 J. C. FLUGEL  \textit{Hundred Years Psychol.} (ed. 2) viii. 280 The mere bringing back and discussing of memories‥which Freud and Breuer called subsequently ‘abreaction’ or ‘catharsis’.
\end{myenumerate}


%%%%%%%%%%%%%%%%%%%%%%%%%%%%%%%%%
\myitem{catholic} a. and n.

\noindent \phonetic{(ˈkæθəlɪk)}

\noindent [a. F. catholique (13th c. in Littré) ad. late L. catholic-us, a. Gr. καθολικός general, universal, f. καθόλου (i.e. καθ' ὅλου) on the whole, in general, as a whole, generally, universally, f. κατά concerning, in respect of, according to + ὅλος whole. (If immed. derived from L. or Gr., the Eng. word would, according to the regular analogy of words in -ic, have been accented \phonetic{caˈtholic}).]

\begin{myenumerate}
\itembf{A.} adj. In non-ecclesiastical use.

\itembf{1.} gen. Universal.

\P 1551 T. WILSON  \textit{Logike} 1 b, Catholike being a greeke word signifieth nothing in English but universall or common.    
\P 1613 R. C. \textit{Table  Alph.} (ed. 3) Catholicke, vniuersall or generall.    
\P 1660 N. INGELO  \textit{Bentiv. \& Ur.} (1682) 11, The Indisputable Commands of a Catholick Dictator in knowledge.    
\P 1885  \textit{Times} (weekly ed.) 11 Sept. 7/1 Science is truly catholic, and is bounded only by the universe.

\itembf{2.} In specific uses: a.A.I.2.a Universally prevalent: said e.g. of substances, actions, laws, principles, customs, conditions, etc. Obs.

\P 1561 T. NORTON  \textit{Calvin's Inst.} iii. 248 This is to be holden for a catholike principle.    
\P 1615 CROOKE  \textit{Body of Man} 418 It is a Catholicke principle, Euery thing is preserued and refreshed with his like.    
\P 1657 S. PURCHAS  \textit{Pol. Flying-Ins.} 95 This is a common, but no catholique custome [among bees] for I have often observed the contrary.    
\P 1660 SHARROCK  \textit{Vegetables} 79 The universal and catholick order of all bulbous plants, is‥that about St. James' tyde they be taken out of the ground.    
\P 1662 STILLINGFL.  \textit{Orig. Sacr.} iii. ii. §14 The Catholick Laws of nature which appear in the world.    1665–6 Phil. Trans. I. 192 All Bodies are made of one Catholick matter common to them all.    
\P 1675 EVELYN  \textit{Terra} (1729) 10 There is but one Catholic homogeneous, fluid matter.    
\P 1692 BENTLEY  \textit{Boyle Lect.} 112 This Catholick Principle of Gravitation.    
\P 1696 EDWARDS  \textit{Exist. \& Provid. God} i. 3 A great proof of the catholick degeneracy of this present age.

\itembf{b.} Universally applicable or efficient; spec. of medicines, remedies. Obs.

\P 1612 WOODALL  \textit{Surg. Mate Wks.} (1653) 43 It hath the prime place, for a Catholick medicine in exulcerations.    
\P 1621 BURTON  \textit{Anat. Mel.} ii. v. i. v. (1651) 393 There is no Catholike medicine to be had: that which helps one is pernitious to another.    
\P 1658 A. FOX  \textit{Wurtz' Surg.} iv. ii. 309 A Catholick Plaister, used for all wounds and stabs.    
\P 1671 SALMON  \textit{Syn. Med.} iii. xlix. 559 A noble Extract, and a catholick purge.    
\P 1691 RAY  \textit{Creation} i. (1704) 115 Fire‥which is the only Catholick Dissolvent.    
\P 1693 SLARE in \textit{Phil. Trans.} XVII. 906 Tho' Spirit of Wine be a very Catholic Menstruum.    
\P 1713  \textit{Lond. \& Country Brew.} iv. (1743) 261 [Water] is the only Catholick Nourishment of all Vegetables, Animals, and Minerals.    
\P 1752 HUME  \textit{Ess.} (1777) II. 11 Accurate and just reasoning is the only Catholic remedy.

\itembf{c.} More loosely: Common, prevalent. Obs.

\P 1607 DEKKER  \textit{Northw. Hoe} v. Wks. 1873 III.  74 What is more catholick i' the city than for husbands daily for to forgive the nightly sins of their bedfellows?    
\P 1631 MASSINGER  \textit{Emper. of East} iv. iv, The pox, sir‥Is the more catholic sickness.    
\P 1660 SHARROCK  \textit{Vegetables} 130 Hot beds are the most general and catholick help.

\itembf{d.} Entire, without exception. Obs.

\P 1664 EVELYN  \textit{Sylva} 19 Deep interring of Roots is amongst the Catholick Mistakes.    
\P 1671 DRYDEN  \textit{Even. Love} iv. i, Alon. And, how fares my Son-in-law that lives there? Mel. In Catholick Health, Sir.

\itembf{3.} In current use: \textbf{a.} Of universal human interest or use; touching the needs, interests, or sympathies of all men.

\P a1631 DONNE  \textit{Serm.} lxvi. (1640) So are there some‥Catholique, universal Psalmes, that apply themselves to all necessities.    
\P 1704 SWIFT  \textit{Mech. Operat. Spirit} (1711) 279 All my Writings‥for universal Nature, and Mankind in general. And of such Catholick Use I esteem this present Disquisition.    
\P 1838–9 HALLAM \textit{Hist. Lit.} iii. v. §4 Catholic poetry, by which I mean that which is good in all ages and countries.    
\P 1844 EMERSON  \textit{Lect. New Eng. Ref.} Wks. (Bohn) I. 264 A grand phalanx of the best of the human race, banded for some catholic object.    
\P 1867 FROUDE  \textit{Short Stud.} 363 What was of catholic rather than national interest.

\itembf{b.} Having sympathies with, or embracing, all: said of men, their feelings, tastes, etc.; also fig. of things. (Closely connected with 8.)

\P 1586 BRIGHT  \textit{Melanch.} iv. 16 The stomach becommeth the most Catholicke part in all the bodie, carying a more indifferent affection to what soever is receiued then anie part beside.    
\P 1620 J. PARKINSON  \textit{Paradisus} xxvi. 215 Such as are Catholicke obseruers of all natures store.    
\P 1817 COLERIDGE  \textit{Biog. Lit.} I. iv. 73 Others more catholic in their taste.    
\P 1833 LAMB  \textit{Elia, Books \& Read.}, I bless my stars for a taste so catholic, so unexcluding.    
\P 1851 CARLYLE  \textit{Sterling} i. iv. (1872) 31 Of these two Universities, Cambridge is decidedly the more catholic (not Roman catholic, but Human catholic).    
\P 1878 STEVENSON  \textit{Inland Voy.}, On these different manifestations, the sun poured its clear and catholic looks.    
\P 1879 TOURGEE  \textit{Fool's Err.} xxxviii. 271 A man of unusually broad and catholic feeling.

\itembf{4.} Catholic Epistle: a name originally given to the ‘general’ epistles of James, Peter, and Jude, and the first of John, as not being addressed to particular churches or persons. The second and third epistles of John are now conventionally included among the number.
   It is not certain that this was the original sense of ἐπιστολὴ καθολικὴ, since some early writers appear to use it in the sense ‘genuine and accepted’ (see canonical): but the attribute has been understood in the sense ‘encyclical’ or ‘general’ since the 10th or 11th c.

\P 1582 N. T. (RHEM.)  \textit{James} (heading) The Catholic Epistle of St. James the apostle.    
\P 1725 tr. \textit{Dupin's Eccl. Hist.} I. v. 69 The Encyclick, Circular, or Catholick Letters, were address'd to all Churches, or to all the Faithful.    
\P 1855 WESTCOTT  \textit{Canon N.T.} (1881) 395 It may be inferred that the seven Catholic Epistles were formed into a collection at the close of the third century.

\itembf{II.} In ecclesiastical use.

   The earlier history of this lies outside English, and may be found in such works as Smith's Dict. Christian Antiq. or in Lightfoot's Ignatius I. 398–400, 605–607; II. 310–312. Ἡ καθολικὴ ἐκκλησία ‘the catholic church’ or ‘church universal’, was first applied to the whole body of believers as distinguished from an individual congregation or ‘particular body of Christians’. But to the primary idea of extension ‘the ideas of doctrine and unity’ were super-added; and so the term came to connote the Church first as orthodox, in opposition to heretics, next as one historically, in opposition to schismatics. Out of this widest qualitative sense arose a variety of subordinate senses; it was applied to the faith the Church held, to particular communities or even individual members belonging to it, and especially in the East, to cathedrals as distinguished from parish churches, then later to parish churches as opposed to oratories or monastic chapels. After the separation of East and West ‘Catholic’ was assumed as its descriptive epithet by the Western or Latin Church, as ‘Orthodox’ was by the Eastern or Greek. At the Reformation the term ‘Catholic’ was claimed as its exclusive right by the body remaining under the Roman obedience, in opposition to the ‘Protestant’ or ‘Reformed’ National Churches. These, however, also retained the term, giving it, for the most part, a wider and more ideal or absolute sense, as the attribute of no single community, but only of the whole communion of the saved and saintly in all churches and ages. In England, it was claimed that the Church, even as Reformed, was the national branch of the ‘Catholic Church’ in its proper historical sense. As a consequence, in order to distinguish the unreformed Latin Church, its chosen epithet of ‘Catholic’ was further qualified by ‘Roman’; but see sense 7. On this analogy Anglo-Catholic has been used by some, since about 1835, of the Anglican Church.

\itembf{5.} Catholic Church, Church Catholic: the Church universal, the whole body of Christians.

\P 1559 \textit{Injunctions  by Queens Majestie} D iv, Ye shall praye for Christes holy Chatholique church, that is, for the whole congregation of Christian people, dispearsed throughout the whole worlde, and specially for the Church of England and Irelande.    
\P 1560–61 \textit{Scotch Conf. Faith} xvi, Whiche Kirk is Catholik, that is universall, becaus it conteanes the Elect of all aiges, all realmes, nationis, and tounges, be thai of the Jewis or be thai of the Gentiles, who have communioun and societie with God the Father, and with his Sone Christ Jesus.    
\P 1630 PRYNNE  \textit{Anti-Armin.} 129 There is a holy Catholicke Church, to wit, the whole company of Gods Elect.    
\P 1645 USSHER  \textit{Body Div.} (1647) 187 The Catholick Church, that is, God's whole or universall Assembly.    
\P 1651 BAXTER  \textit{Inf. Bapt.} 304, I hope this learned man doth not take the particular Romane Church, for the Catholick Church.    
\P 1685 KEN  \textit{Ch. Catech., ‘Holy Cath. Ch.’}    
\P 1839 J. YEOWELL  \textit{Anc. Brit.} Ch. xi. (1847) 110 As members of the church catholic.    Mod. In this sense many accept the article of the Creed, ‘I believe in the holy catholic church’.

\itembf{b.} Of or belonging to the church universal, universal Christian.

\P 1579 FULKE  \textit{Heskins' Parl.} 94 He can neuer prooue his reseruation to be catholike or vniversally allowed and practised of the Church.    
\P 1651 C. CARTWRIGHT  \textit{Cert. Relig.} i. 10 That Church whose Doctrine is most Catholick and universall must be the Catholick Church.    
\P 1657 CROMWELL  \textit{Sp.} 3 Apr., Such a Catholic interest of the people of God.    
\P 1777 FLETCHER \textit{Reconcil.} Wks. 1795 IV. 211  A great friend to a catholic gospel.    
\P 1807 KNOX  \& JEBB \textit{Corr.} I. 370 A catholic liturgy must be formed on a catholic plan; that is, from a harmony of those dispersed and vital truths, which in different ages, different countries, and different churches, were popularly, and effectually embodied, in established liturgies.    
\P 1882 FARRAR  \textit{Early Chr.} I. 250 Christianity in all Churches was, and ever must be, in its essence Catholic—one and indivisible.

\itembf{6. a.} As an epithet, applied to the Ancient Church, as it existed undivided, prior to the separation of East and West, and of a church or churches standing in historical continuity therewith, and claiming to be identical with it in doctrine, discipline, orders, and sacraments. (a) After the separation, assumed by the Western or Latin Church, and so commonly applied historically. (b) After the Reformation in the 16th c. claimed as its exclusive title by that part of the Western Church which remained under the Roman obedience (see 7); but (c) held by Anglicans not to be so limited, but to include the Church of England, as the proper continuation in England, alike of the Ancient and the Western Church.

(Whatever the application, the implied sense is ‘the Church or Churches which now truly represent the ancient undivided Church of Christendom’.)

\P 1532 MORE  \textit{Confut. Tindale} Wks. 690/1 The very name he sayth of catholike, yt is to sai vniuersal, gaue to ward ye getting of hys credence ye catholike church gret autoritye.
\P 1534 ABP. LEE in \textit{Lingard Hist. Eng.} (1855) V. i. 18/1 note, So that‥the unitie of the faiethe and of the Catholique Chyrche [be] saved.    
\P 1552 ABP. HAMILTON  \textit{Catech.} (1884) 47 Quhilk catholike kirk is trewly represented in all general counsellis.    
\P 1651 HOBBES Leviath. Wks. 1839 III.  517 The Christians of that time [before Constantine], except a few, in respect of whose paucity the rest were called the Catholic Church and others heretics.
\P c1670 JER. TAYLOR  \textit{Duty of Clergy} ii. 4 The Catholic Church hath been too much and too soon divided‥but in things simply necessary, God hath preserved us still unbroken: all nations and all ages recite the Creed‥and all Churches have been governed by Bishops.    
\P 1704 NELSON  \textit{Fest. \& Fasts} vii. (1739) 538 The ancientest Fathers of the Catholick Church.    
\P 1834 \textit{Tracts  for Times} No. 61, We [English Church] are a branch of the Church Catholic.    
\P 1854 HOOK  \textit{Ch. Dict.} s.v. Creed, There are three creeds recognized by the catholic church.    Ibid. s.v. Tradition, The great deference paid by the Church of England as a branch of the Catholic Church to tradition.    
\P 1866 LD. ROMILLY in \textit{Law Rep.} 3 Eq. 29 The Catholic Church of Christ, of which the Church of England is a branch.    
\P 1872 FREEMAN  \textit{Gen. Sketch} vi. 111 The people of the Oriental provinces‥putting forth or adopting doctrines which the Catholic Church, both of the Old and of the New Rome, looked on as heretical.

\itembf{b.} Hence, Of or belonging to this Church; of the true apostolic Church, orthodox: (a) Of belief, doctrine, etc.

\P c1500 \textit{Melusine}  (1888) 31 My byleue is as a Catholique byleue oughte for to be.  
\P a1556 CRANMER  \textit{Wks.} (1844) I. 9 An explication and assertion of the true catholic faith in the matter of the sacrament.    
\P 1549 \textit{Bk. Com.  Prayer}, Athan. Crede, And the Catholike faithe is this: That we worship one God in trinitie, and trinitie in unitie.    
\P 1634 HABINGTON  \textit{Castara} (Arb.) 112 The Catholique faith is the foundation on which he erects Religion.    
\P 1840 \textit{Tracts  for Times} No. 85 vi, The Catholic or Church system of doctrine and worship.    
\P 1854 HOOK  \textit{Ch. Dict.} s.v. Image worship, Protesting against Roman corruptions of the Catholic Faith.

\noindent
(b) Of persons: Holding the faith of this Church; rightly believing, orthodox. (This and sense a appear to be the earliest uses in English. The n. is in 1425.)

\P 1500 \textit{Melusine}  (1888) 32 A man very catholoque \& of good feith.    
\P 1531 ELYOT  \textit{Govt.} iii. xxiii, Wherein no good catholyke man wyll any thynge doute, though they be meruaylous.    
\P 1552 HULOET,  Catholyke or perfect Christian, orthodoxus.    
\P 1854 HOOK  \textit{Ch. Dict.} s.v., In ecclesiastical history‥a catholic Christian denotes an orthodox Christian.    
\P 1881 FREEMAN  \textit{Hist. Geog. Eur.} I. iv. 101 The lands ruled either by the Catholic Frank or by the Arian Goth.

\noindent
(c) Of the writers, fathers, or antiquity, of the ancient undivided church, or accepted by the orthodox historical church.

\P 1548 UDALL, etc. \textit{Erasm. Par.} Pref. 14 Whatsoeuer in any catholike wryter is conteyned.    
\P 1593 BILSON  \textit{Govt. Christ's} Ch. xi, What Presbytery the primitiue Churches and Catholike fathers did acknowledge.    
\P 1842 \textit{Tracts  for Times} No. 86 v. §3 What is popularity when it is opposed to Catholic Antiquity?

\noindent
(d) Of a particular body: Forming part of, or in communion with, this church. (Cf. Anglo-Catholic.)

\P 1833 CRUSE  \textit{Eusebius} vi. xliii. 265 One bishop in a catholic church.    
\P 1854 HOOK  \textit{Ch. Dict.} s.v. Lights, We of the Anglo-Catholic Church.    Ibid. s.v. Catholic, A Catholic Church means a branch of this one great society, as the Church of England is said to be a Catholic Church: the Catholic Church includes all the Churches in the world under their legitimate Bishops.

\itembf{7.} As applied (since the Reformation) to the Church of Rome (Ecclesia apostolica catholica Romana) = Roman Catholic, q.v. (Opposed to Protestant, Reformed, Evangelical, Lutheran, Calvinistic, etc.)

   Roman Catholic is the designation known to English law; but ‘Catholic’ is that in ordinary use on the continent of Europe, especially in the Latin countries; hence historians frequently contrast ‘Catholic’ and ‘Protestant’, especially in reference to the continent; and, in familiar non-controversial use, ‘Catholic’ is often said instead of Roman Catholic.

\P 1554 (MARCH)  \textit{Q. Mary's Injunct.} in Wilkins \textit{Concilia} (1737) IV. 90 To remove them, and place catholic men in their rooms.
\P a1555 J. BRADFORD in Foxe \textit{A. \& M.} (1583) 1647 This  Latine seruice is a playne marke of anti$\sim$christs Catholike Synagoge.    
\P 1563  \textit{Ibid.} 1844 The  Catholike prelates of the Popes band.    
\P 1588 ALLEN  \textit{Admon.} in Lingard \textit{Hist. Eng.} (1855) VI. 358 She [Q. Eliz.] hath abolished the Catholic religion.    
\P 1602 CAREW  \textit{Cornwall} 71 a, A matter practised‥as well by the reformed as Catholike Switzers.    
\P 1620 FR. HUNT  \textit{(title)}, Appeal to the King, proving that our Saviour was Author of the Catholic Roman Faith.    
\P 1622 RUSHW.  \textit{Hist. Coll.} (1659) I. 287 His Majesties Roman Catholick-Subjects.    
\P 1660 R. COKE  \textit{Power \& Subj.} 215 If the Pope would be Head of the Catholique Church, the King would be Head of the Church of England.    
\P 1790 BURKE  \textit{Fr. Rev.} Wks. V. 60 Whether‥the catholick heir [gave way] when the protestant was preferred.    
\P 1845 S. AUSTIN  \textit{Ranke's Hist. Ref.} II. 513 What was begun by the evangelical governments, was carried on in an analogous manner by the catholic.    
\P 1845 BRIGHT  \textit{Sp. Maynooth Grant} 16 Apr., A Protestant soldiery, who, at the beck and command of a Protestant priest, have butchered and killed a Catholic peasant.    
\P 1872 FREEMAN  \textit{Gen. Sketch} xiii. 252 That the government of each German state might set up which religion it pleased, Catholic or Protestant.    
\P 1873 MORLEY  \textit{Rousseau} I. 229 A Catholic country like France.

\itembf{b.} Catholic Seat: = apostolic See. Obs.

   In ancient times the καθολικοὶ θρόνοι or catholic sees, were those of Rome, Alexandria, Antioch, and Jerusalem.

\P 1563 FOXE  \textit{A. \& M.} (1583) 798 The proud, cruell, and bloudy rage of the Catholique Seat.

\itembf{c.} Catholic King, his Catholic Majesty: a title given to the kings of Spain.

   (In much earlier times the title belonged to the kings of France, Pipin being so called a.d. 767.)

\P 1555 EDEN  \textit{Decades W. Ind. To Rdr.} (Arb.) 50 By the moste catholyke \& puissaunt kynge Ferdinando.    Ibid. 288 Wheruppon I wente into Spayne to the Catholyke kynge.    
\P 1588 ALLEN  \textit{(title)}, Admonition to the Nobility and People of England‥by the high and mightie kinge Catholike of Spaine.    
\P 1627 SANDERSON  \textit{Serm.} I. 281 He that‥hath better title to the stile of most catholick king than any that ever yet bare it‥I mean the devil, the prince of this world.    
\P 1636 MASSINGER  \textit{Bashf. Lover} iv. i.    
\P 1704  \textit{Lond. Gaz.} No. 3987/3 To wait upon his Catholick Majesty.    
\P 1725 DE FOE  \textit{Voy. round W.} (1840) 280 Does not his Catholic majesty claim a title to the possession of it?

\itembf{d.} See also B.

\itembf{8.} Recognizing, or having sympathies with, all Christians; broadly charitable in religious matters. (Cf. 3 b. which differs only in not being restricted to things ecclesiastical or religious.)

\P 1658 BAXTER  in H. Rogers \textit{J. Howe} iii. (1863) 59 The Lord Protector is noted as a man of a catholic spirit, desirous of the unity and peace of all the servants of Christ.    
\P 1719 DE FOE  \textit{Crusoe} (1840) II. vii. 158 If such a temper was universal, we might be all Catholic Christians, whatever church or particular profession we joined to, or joined in.    
\P 1734 WATTS  \textit{Reliq. Juv.} (1789) 155 To see all the disciples of Christ grown up into such a catholic spirit, as to be ready to worship God their common Father‥in the same assembly.    
\P 1874 BLACKIE  \textit{Self-Cult.} 80 A spirit of deep and catholic piety.

\itembf{9.} transf. Orthodox (applied e.g. to orthodox Muslims). Obs.

\P 1613 PURCHAS  \textit{Pilgr.} vii. vii. 575 They are not all Catholike Mahumetans.    
\P 1625 \textit{Pilgrimes} vi. i. §3 By some they are accounted Catholique or true Mahumetans, and by others they are holden for heretiks.

\itembf{10.} Catholic (and) Apostolic Church: the religious body otherwise called Irvingites. (See quots. 1861, 1867.)

[\P 1837 \textit{Testimony  to Bps.}, etc. 32 That no section of the baptized bears the character of the one Holy Catholic Apostolic Church.]    
\P 1861 NORTON  \textit{Restor. Apostles and Proph. in Cath. Apostolic Ch.} 159 In assuming, as our only title and name, that of ‘the Catholic and Apostolic Church’—we arrogate to ourselves nothing, for we do not appropriate it in any exclusive sense.    
\P 1867 \textit{Address}  in Miller \textit{Irvingism} i. 5 Catholic and Apostolic Churches, a name which we have not assumed, and to which we have no exclusive right‥But it is the only name by which we can, without protest, suffer ourselves to be called.    
\P 1888 \textit{Whitaker's  Almanac, Relig. Sects}, Places‥certified to the Registrar-General on behalf of persons described as‥Catholic Apostolic Church.

\itembf{11.} Comb., as Catholic-minded adj.

\P 1879 \textit{Dublin  Rev.} Jan. 95 The learned, irresolute, yet pious and Catholic-minded men at the head of whom was Fisher's friend, Cuthbert Tunstal.    
\P 1964 P. F. ANSON  \textit{Bishops at Large} xi. 534 An alternative to Roman Catholicism to a catholic-minded people.

\itembf{B.} n.

\itembf{1.} A member of a church recognized or claiming to be ‘Catholic’ in sense A. 6; e.g. an orthodox member of the Church before the disruption of East and West, as opposed to an Arian or other ‘heretic’; of the Latin Church as opposed to the Greek or any separating sect or community (e.g. the Lollards); of a church or churches now taken to represent the primitive Church.

\P c1425 WYNTOUN  \textit{Cron.} ix. xxvi. 63 He was a constant Catholike All Lollard he hatyt and Heretike.    
\P 1594 HOOKER  \textit{Eccl. Pol.} iv. §5 Let the Church of Rome be what it will,‥hold them for Catholics, or hold them for Heretics, it is not a thing‥in this present question greatly material.    
\P 1597 J. JONES  \textit{Preserv. Bodie \& Soule} Ded., It is‥of the faithfull, Christian, and Catholike certainly beleeued.    
\P 1609 BIBLE  \textit{(Douay) Proemial Annot.}, Some of these bookes‥were sometimes doubted of by some Catholiques, and called Apochryphal.    
\P 1702 tr. Le Clerc's Prim. Fathers 241 An Edict bearing date the 27th of February (380)‥That those who would profess it should be called Catholics, and the others Hereticks.    
\P 1854 HOOK  \textit{Ch. Dict.} s.v., Let the member of the Church of England assert his right to the name of Catholic, since he is the only person in England who has a right to that name. The English Romanist is a Roman Schismatic, and not a Catholic.    
\P 1860 FROUDE  \textit{Hist. Eng.} VI. 39, I must again remind my readers of the distinction between Catholic and Papist. Three quarters of the English people were Catholics; that is, they were attached to the hereditary and traditionary doctrines of the Church.    
\P 1872 FREEMAN  \textit{Gen. Sketch} v. 102 He [Chlodwig] became‥not only a Christian but a Catholic‥all the other Teutonic Kings were Arians.

\itembf{2. a.} spec. A member of the Roman Church. English Catholic = English Roman Catholic.

\P 1570 B. GOOGE  \textit{Pop. Kingd.} iv. (1880) 60 Accounting here for Catholickes, themselves \& all their traine.    
\P 1581 \textit{(Title)} A Checke or Reproofe of M. Howlet‥with an answere to the Reasons why Catholikes (as they are called) refuse to goe to Church.    
\P 1584 in Foley Rec. \textit{Eng. Prov. S.J.} (1880) VI. 740 He said‥that all English Catholics were bound to pray for the King of Spain.    
\P 1588 ALLEN  \textit{Admon.} in Lingard \textit{Hist. Eng.} (1855) VI. 358/1 Not tolerable to the masters of her [Q. Eliz.] own sect, and to all Catholics in the world most ridiculous.    
\P 1602 BP. J. RIDER  \textit{(title)}, A caveat to Irish Catholicks.    
\P 1602 WARNER  \textit{Alb. Eng.} ix. xlix. (1612) 226 Euen Catholiques (that erred name doth please the Papists).    
\P 1611 BIBLE  \textit{Pref.} The Catholicks (meaning Popish Romanists).    
\P 1636 FEATLY  \textit{Clavis Myst.} xxxiv. 483 Other of the Pope his stoutest champions‥[say] we are sirnamed catholikes, therefore we are so.    
\P 1641 J. LOUTH in A. H. Mathew \textit{Convers. Sir T. Matthew} (1904) 176 The innocency and loyalty of English Catholics towards others.    
\P 1650 SIR E. NICHOLAS  in \textit{N. Papers} (1886) I. 180 That which has been proposed concerninge the Catholics.    
\P 1715 in Estcourt \& Payne Eng. Cath. Nonjurors of 1715 (1885)  8, I, Henry Englefield, do declare that I am, by the grace of God, an English Catholic.    
\P 1719 DE FOE  \textit{Crusoe} (1840) II. vi. 155, I am a Catholic of the Roman Church.    
\P 1800 C. BUTLER  \textit{Life Alban Butler} xvi, A person would deserve well of the English Catholics who should translate it into English.    
\P 1845 BRIGHT  \textit{Sp.} 16 Apr., The Irish Catholics would thank you infinitely more if you were to wipe out that foul blot.    
\P 1872 FREEMAN  \textit{Gen. Sketch} xiii. 254 The religious wars between the Catholics and Protestants within the country [France].    
\P 1876 GREEN  \textit{Short Hist.} vii. §4 The last hopes of the English Catholics were dispelled by the Queen's refusal to take part in the Council of Trent.    
\P 1889 J. O. PAYNE  \textit{(title)} Records of the English Catholics of 1715.

\itembf{b.} Old Catholic, a term introduced after the secession of John Henry Newman and others to distinguish members of Catholic families in England since the Reformation from Catholic immigrants and converts.

\P 1846 J. H. NEWMAN  \textit{Let.} 14 July in Gasquet Ld. Acton (1906) p. xiii, It will be one of your collisions with old Catholics.    
\P 1909 \textit{Dublin  Rev.} Jan. 56 The friction between converts and old Catholics‥was inevitable.    
\P 1918 L. STRACHEY  \textit{Emin. Victorians} i. v. 56 It seemed as if the harvest was to be gathered in by a crowd of converts, who were proclaiming on every side as something new and wonderful the truths which the Old Catholics‥had not only known, but for which they had suffered, for generations.    
\P 1962 V. A. MCCLELLAND  \textit{Cardinal Manning} i. 3 If one is to understand the opposition to Cardinal Manning and to the Oxford converts, one has to appreciate the feelings and position of the ‘Old Catholics’.

\itembf{3.} Defined or limited by a word prefixed, as †English Catholic, †Popish Catholic, Anglo-Catholic, Roman Catholic, q.v.
   (See a different use of English Catholics, in sense 2 quot. 1876.)

\P 1577 FULKE  \textit{(title)}, Two Treatises‥Answere of the Christian Protestant to the proud challenge of a Popish Catholicke.    
\P 1585 SIR W. HARBERT  \textit{(title)}, Letter to a Roman pretended Catholike.    
\P 1598 HAKLUYT  \textit{Voy.} I. 597 Many rebels against her maiestie and popish catholiques.    
\P 1837 J. H. NEWMAN  \textit{Par. Serm.} (1840) III. xiv, The Holy Church throughout all the world is broken into many fragments‥we are the English Catholics, abroad are the Roman Catholics‥elsewhere are the Greek Catholics, and so on.    
\P 1854 HOOK  \textit{Ch. Dict.} s.v. Protestant, We tell the Papist that with respect to him we are Protestant; we tell the Protestant Dissenter that in respect to him we are Catholics; and we may be called Protestant or Protesting Catholics, or as some of our writers describe us, Anglo-Catholics.

\itembf{b.} German Catholic, Old Catholic: names taken by religious parties who separated from the Roman Catholic communion in Germany, the former under Ronge in 1845 (reunited 1848), the latter after the Vatican Council in 1870–71. Old Catholic is also applied to members of other churches separated from Rome, and united by acceptance of the Declaration of Utrecht of 1889. 

\P 1871 \textit{Sunday  Mag.} Nov. 84/1 The Old Catholics have great hopes of support from the High Church party in England.    
\P 1871 \textit{Union  Rev.} 273 For German Catholics to succumb to the Vatican decrees, would be an act of moral suicide.    
\P 1931 W. TEMPLE  \textit{Thoughts on Probl. of Day} iv. 92 The Conference‥was concerned with advances towards union in two directions—on the one hand towards union with the Orthodox and the Old Catholic Churches, and on the other hand with the non-episcopal Churches.    
\P 1948 C. B. MOSS  \textit{Old Catholic Movement} i. 1 The Old Catholic Churches are a group of self-governing national churches, united by their acceptance of the Declaration of Utrecht (1889) as their dogmatic basis.    Ibid. xxviii. 348 Eight Dutch Old Catholic priests came to England to see the English Church for themselves.    
\P 1969 D. W. D.  SHAW tr. \textit{Heyer's Catholic Church} vi. 149 Political factors had produced an initial wave of interest in the ‘German Catholic’ movement.

\itembf{4.} = Catholicos. Obs.

\P 1612 BREREWOOD  \textit{Lang. \& Relig.} xxiv. 213 The Catholick of Armenia.    Ibid. 210 They acknowledge obedience‥to two Patriarchs of their own: whom they term Catholicks.    
\P 1735 JOHNSON tr. \textit{Lobo's Abyssinia} 307 Catholick like Patriarch is no more than an empty Title without the Power.

\itembf{C.} attrib. Of, relating to, affecting, or on the side of (Roman) Catholics. In Catholic Emancipation, etc. [In construction not distinct from the adj.]

\P 1791 J. MILNER  \textit{(title)}, A short Pamphlet on the Catholic Question.    
\P 1795 DUIGENAN  \textit{(title)}, Speech on the Catholic Bill in the Irish House of Commons.    
\P 1805 LD. HAWKESBURY  \textit{(title)}, Speech in the House of Lords, 10th of May on the Catholic Petition.    
\P 1809 SOUTHEY  \textit{Ess.} (1832) II. 301 For these people Catholic Emancipation can do nothing.    
\P 1878 SPENCER  \textit{Walpole Hist. Eng.} II. vii. 145 The anti-Catholic members of the Cabinet [in 1826] were as much opposed to their Catholic colleagues as to their regular opponents.    Ibid. note, Persons in favour of emancipation were classed as Catholic statesmen.
\end{myenumerate}


%%%%%%%%%%%%%%%%%%%%%%%%%%%%%%%%%
\myitem{caveat} n.

\noindent \phonetic{(ˈkeɪviːæt)}

\noindent [L. caveat let him beware, 3rd sing. pres. subj. of cavēre to beware.]
\vspace{-0.3cm}

\begin{myenumerate}

\itembf{1.} Law. \textbf{a.} A process in court (originally in ecclesiastical courts) to suspend proceedings; a notice given by some party to the proper officer not to take a certain step until the party giving the notice has been heard in opposition. Phrase, to enter or put in a caveat: also fig. see 2 b.

\P 1654 GATAKER  \textit{Disc. Apol.} 45 A Caveat they found entred in the Bishops Office, by a Gentleman, one of the Petti-Bag, who pretended a Title.    
\P 1656 BLOUNT  \textit{Glossogr.}, Caveat, used among the Proctors, when a person is dead, and a competition ariseth for the Executorship, or Administratorship, the party concerned enters a Caveat, to prevent or admonish others from intermedling.    
\P 1667 MARVELL  \textit{Corr.} cxiv. Wks. 1872–5 II. 273, I entered caveats both at Mr. Atturny's and Mr. Sollicitor's.    
\P 1726 AYLIFFE  \textit{Parerg.} 145 A Caveat in Law‥is an Intimation given to some Ordinary or Ecclesiastical Judge‥notifying to him that he ought to beware how he acts in such or such an Affair.    
\P 1818 CRUISE  \textit{Digest} V. 95    
\P 1884 LAW  \textit{Rep.} 9 Probate Div. 23 The‥defendant, one of the next of kin, entered a caveat.

\itembf{b.} caveat emptor [lit., let the purchaser beware], let the purchaser examine the article he is buying before the bargain is completed, so that in case of disappointment after purchase he may not blame the seller.

\P 1523 FITZHERBERT  \textit{Husb. f.} xxxvi, He [sc. the horse] is no chapmans ware yf he be wylde: but and he be tame and haue ben rydden vpon than caueat emptor be ware thou byer.    
\P 1629 T. ADAMS  \textit{Pol. Hunting} in \textit{Wks.} 118 We compell none to buy our Ware; Caueat emptor.    
\P 1809 H. MORE  \textit{Let.} 14 Aug. (1925) 139 Mr. C. in his last Review‥feels it is his duty to say, ‘Caveat Emptor’.    
\P 1902  \textit{Economic Jrnl.} XII. 12 Caveat emptor. It is the employer on whom the responsibility rests of testing the quality of the article he buys.    
\P 1950 T. H. MARSHALL  \textit{Citizenship \& Social Class} iv. 133 The principle of caveat emptor is at least plausible when you are buying a horse.

\itembf{2.} transf. \textbf{a.} A warning, admonition, caution.

\P 1557 RECORDE  \textit{Whetst.} Y iij b, A caueat, to be ware of to moche confidence.    
\P 1583 STANYHURST  \textit{Æneis} iii. (Arb.) 85 Such od caueats, as I to the frendlye can vtter.    
\P 1646 S. BOLTON  \textit{Arraignm, Err.} 50 A Caveat to you how you live.    
\P 1651 WITTIE tr. \textit{Primrose's Pop. Err.} iv. 248 Those Caveats, whereof Astrologers do every year warn the people.    
\P 1712 BUDGELL  \textit{Spect.} No. 365 ⁋1, I design this Paper as a Caveat to the Fair Sex.    
\P 1791 BOSWELL  \textit{Johnson} (1816) IV. 448 A caveat against ostentatious bounty and favour to negroes.    
\P 1855 H. SPENCER  \textit{Princ. Psychol.} (1872) I. v. iii. 531 With this caveat let us now pass‥to more complex cases.

\itembf{b.} to put in or enter a caveat (in senses 2 \& 3).

\P 1577 tr. \textit{Bullinger's Decades} (1592) 405 It pleased the goodnesse of God by giuing the law to put in a caueat‥for the tranquilitie of mankinde.    
\P 1600 HOLLAND  \textit{Livy} xxvi. xxiv. 602 They should put in a caveat, that he might have no libertie to warre upon the Ætolians.    
\P 1642 FULLER  \textit{Holy \& Prof. St.} i. xii. 37 She enters a silent caveat by a blush.    
\P 1755 YOUNG  \textit{Centaur} i. Wks. 1757 IV. 116  Putting in a caveat against the ridicule of infidels.    
\P 1875 E. WHITE  \textit{Life in Christ} ii. x. (1878) 108 To enter a caveat against a misconception.

\itembf{3.} A condition previously laid down; a proviso, reservation; = caution n. 2. Obs.

\P 1579 FULKE  \textit{Heskins' Parl.} 370 M. Heskins fombleth out the matter with a foolish caueat, that‥he suffreth not violence.    
\P 1648 GAGE  \textit{West Ind.} xxi. (1655) 196 Some were offered me for nothing, with this caveat, that‥I must, etc.

\itembf{4.} A precaution; = caution n. 5. Obs.

\P 1596 SPENSER  \textit{State Irel.} Wks. (1862) 539/1 The chiefest caveat and provision in the reformation of the North must be to keep out those Scottes.    
\P 1612 BRINSLEY  \textit{Lud. Lit.} 54 Let them vse this caueat especially; that they take but little at a time.    
\P 1643 J. BURROUGHES  \textit{Exp. Hosea} ix. (1652) 310 God laid in a caveat and provision for the encouragement of them.

\itembf{5.} U.S. Patent Laws. ‘A description of some invention, designed to be patented, lodged in the office before the patent right is taken out, operating as a bar to applications respecting the same invention, from any other quarter’ (Webster).

\P 1879 G. B. PRESCOTT  \textit{Sp. Telephone} 256 A caveat, describing this invention, was filed by Gray.
\end{myenumerate}

%%%%%%%%%%%%%%%%%%%%%%%%%%%%%%%%
\myitem{cavil} v.

\noindent \phonetic{(ˈkævɪl)}

\noindent [a. OF. cavill-er (14th c. in Godef.) to mock, jest, rail, ‘to cauill, wrangle, reason crossely, speake ouer thwartly’ (Cotgr.), ad. L. cavillāri (whence also It. cavillare, Sp. cavilar, Pg. cavillar), to practise jeering or mocking, satirize, jest, reason captiously, f. cavilla a jeering, scoffing, raillery.]
\vspace{-0.3cm}

\begin{myenumerate}

\itembf{1.} intr. ‘To raise captious and frivolous objections’ (J.); to object, dispute, or find fault unfairly or without good reason. Const. at, about (formerly also against, with, on).

\P 1548 UDALL, etc. \textit{Erasm. Par. Mark} ii. 19 b, Wheras ye can not thwarte and cauyll in the thynges you see doen before your iyes.    
\P 1564 \textit{Brief  Exam.} ***** iij b, Men dyd not cauill agaynst theyr whyte vestures.    
\P 1596 SHAKES.  \textit{1 Hen. IV}, iii. i. 140 But in the way of Bargaine‥Ile cauill on the ninth part of a hayre.    
\P 1597 MORLEY  \textit{Introd. Mus.} 28 Let no man cauil at my doing in that I have chaunged my opinion.    
\P 1635 SWAN  \textit{Spec. M.} i. §3 (1643) 14 After this manner, such mockers reasoned and cavilled with S. Peter.    
\P 1642 ROGERS  \textit{Naaman} 8 He‥who cavelled against the Prophet.    
\P 1750 WARBURTON  \textit{Lett. late Prelate} (1809) 61 Without finding anything considerable to cavil with you upon.    
\P 1798 MALTHUS  \textit{Popul.} (1878) 88 When the harvest is over they cavil about losses.
\P a1852 WEBSTER  \textit{Wks.} (1877) VI. 163 Those who do not value Christianity‥cavil about sects and schisms.    
\P 1871 ROSSETTI  \textit{Dante at Ver.} liii, To cavil in the weight of bread And to see purse-thieves gibbeted.    
\P 1884 SIR W. BRETT in \textit{Law Times Rep.} LI. 530/1 The rule exists, and I have not the smallest intention of cavilling at it.

\itembf{b.} with object-clause. Obs.

\P 1570 BILLINGSLEY  \textit{Euclid} i. ix. 19 He may cauill that the hed of the equilater triangle shall not fall betwene the two right lines.    
\P 1714 GAY  \textit{What d' ye call it} Pref., They cavil at it as a Comedy, that I had partly a View to Pastoral.

\itembf{2.} trans. To object to or find fault with captiously.

\P 1581 J. BELL  \textit{Haddon's Answ. Osor.} 232/2 This were perhappes not altogether from the purpose, that is cavilled.    
\P 1621 BP. R. MONTAGU  \textit{Diatribæ} 422 Nor can you cauill him for leauing out the word.    
\P 1667 MILTON  \textit{P.L.} x. 759 Wilt thou enjoy the good, Then cavil the conditions?    
\P 1750 WARBURTON  \textit{Wks.} (1811) VIII. 96 The testimony of Amm. Marcellinus, decisive as it is, hath been cavilled.    
\P 1875 H. E. MANNING  \textit{Mission H. Ghost} ix. 256 There are men whose intellectual pride cavils and perverts‥every truth of the revelation of God.

\itembf{b.} with away, out: To do away with, bring out, by cavilling.

\P 1642 MILTON  \textit{Apol. Smect.} (1851) 294 His seventh section labours to cavill out the flawes which were found in the Remonstrants logick.    
\P 1645 W. JENKYN  \textit{Serm.} 28 'Tis this which doth cavill away our peace and holinesse.    
\P 1863 LYTTON  \textit{Caxtoniana} I. 91 Nurse, cherish, never cavil away, the wholesome horror of Debt.

\itembf{3.} in sense of L. cavillāri. Obs.—0

\P 1570 LEVINS  \textit{Manip.} 126 Cauil, calumniari, cauillari.    
\P 1613 R. C. TABLE  \textit{Alph.} (ed. 3), Cauill, to iest, scoffe, or reason subtilly.    
\P 1616 in BULLOKAR.
\end{myenumerate}


%%%%%%%%%%%%%%%%%%%%%%%%%%%%%%%%%
%\myitem{cavil} n.

%\noindent \phonetic{(ˈkævɪl)}

%\noindent [f. the verb.]
%\vspace{-0.3cm}

%\begin{myenumerate}

%\itembf{1.} A captious, quibbling, or frivolous objection.

%\P 1570 LEVINS  \textit{Manip. 124 A cauill, calumnia.    
%\P 1581 J. BELL  \textit{Haddon's Answ. Osor. 336/2, I come now to the other part of your cavill, which is in all respectes as untrue and frivolous.    
%\P 1596 SHAKES.,  \textit{Tam. Shr. ii. i. 392 That's but a cauill.    
%\P 1656 HOBBES  \textit{Six Less. Wks.
%\P 1845 VII.  \textit{227 The ninth objection is an egregious cavil.    
%\P 1735 BERKELEY  \textit{Free-thinking in Math. §50 Whether there may not be fair objections as well as cavils.    
%\P 1850 GLADSTONE  \textit{Glean. V. xliv. 200 To meet this technical cavil on the wording of the Statutes.

%\itembf{2.} The raising of frivolous objections; cavilling.

%\P 1600 HOOKER  \textit{(J.), Wiser men consider how subject the best things have been unto cavil.    
%\P 1611 BIBLE  \textit{Pref. init., If there be any hole left for cauill to enter (and cauill, if it doe not finde a hole, will make one).    
%\P 1729 BUTLER Serm. Wks.
%\P 1874 II. PREF.  \textit{9 The first seems‥the least liable to cavil and dispute.    
%\P 1860 MOTLEY  \textit{Netherl. (1868) I. v. 144 His measures were sure to be the subject of perpetual cavil.    
%\P 1868 FREEMAN  \textit{Norm. Conq. (1876) II. viii. 183 There was no candidate whose claims were altogether without cavil.

%†\itembf{3.} [cf. L. cavilla.] A flout, gibe, jeer. Obs.

%\P 1615 CHAPMAN  \textit{Odyss. xxii. 235 Eumæus on his just infliction pass'd This pleasureable cavil.

%\itembf{4.} Comb., as cavil-proof adj.

%\P 1655 FULLER  \textit{Ch. Hist. iii. viii. §11 James‥ granted them a new Corporation Cavill-proof against all exceptions.



%\end{myenumerate}


%%%%%%%%%%%%%%%%%%%%%%%%%%%%%%%%%
\myitem{celibacy} n.

\noindent \phonetic{(ˈsɛlɪbəsɪ)}

\noindent [f. L. cælibātus in same sense, f. cælebs, cælib-em unmarried, single: see -acy 3. (Cælebs, and its noun of state cælibātus, are the only cognate words found in Latin).]


\noindent
The state of living unmarried.

\P 1663 \textit{ARON-BIMN.}  54 St. Paul's advice for cœlebacy, or single life.    
\P 1754 HUME  \textit{Hist. Eng.} ii, The celibacy of priests was introduced into the English System by Dunstan.    
\P 1791 BOSWELL  \textit{Johnson} (1831) I. xxiv. 387 Even ill assorted marriages were preferable to cheerless celibacy.    
\P 1796 H. HUNTER tr. \textit{St. Pierre's Stud. Nat.} (1799) III. 681 Celibacy may suit an individual, but never a corps.    
\P 1855 MILMAN  \textit{Lat. Chr.} (1864) II. iii. vii. 149 With Gregory celibacy was the perfection of human nature.


%%%%%%%%%%%%%%%%%%%%%%%%%%%%%%%%%
\myitem{champion} v.

\noindent \phonetic{(ˈtʃæmpɪən)}

\noindent [f. prec. n.]
\vspace{-0.3cm}

\begin{myenumerate}

\itembf{1.} To challenge to a contest; to bid defiance to. rare. Obs.

\P 1605 SHAKES.  \textit{Macb.} iii. i. 72 The Seedes of Banquo Kings. Rather then so, come Fate into the Lyst, And champion me to th'vtterance.    
\P 1821 BYRON  \textit{Juan} iv. xliii, She stood as one who champion'd human fears.

\itembf{2.} To fight for; to defend or protect as champion.

\P 1820 SCOTT  \textit{Ivanhoe} xxxix, Championed or unchampioned, thou diest by the stake and fagot.    
\P 1839–40 W. IRVING \textit{Wolfert's R.} (1855) 279 Who ever‥championed them [dames] more gallantly in the chivalrous tilts of the Vivarambla?

\itembf{3.} fig. To maintain the cause of, stand up for, uphold, support, back, defend, advocate.

\P 1844 H. ROGERS  \textit{Ess.} I. ii. 77 His nature‥prompted him to champion any cause in which justice had been outraged or innocence wronged.    
\P 1861 DICKENS  \textit{Lett.} (1880) II. 140 The idea must be championed, however much against hope.    
\P 1863 MRS. C. CLARKE  \textit{Shaks. Char.} xvi. 402 If a friend be in adversity, Gratiano will champion him with good words and deeds.

\itembf{4.} To make a champion of. rare.

\P 1886 SPURGEON  \textit{Treas. Dav. Ps.} cxlii. 7 They‥crowned him, and championed him.

\noindent
Hence \textbf{championing} ppl. a.

\P 1865 DICKENS  \textit{Mut. Fr.} iv. xi, The championing little wife.
\end{myenumerate}


%%%%%%%%%%%%%%%%%%%%%%%%%%%%%%%%%
\myitem{charlatan} n. and a.

\noindent \phonetic{(ˈʃɑːlətən, -tæn)}

\noindent [a. F. charlatan ‘a mountebanke, a cousening drug-seller, a pratling quack-salver, a tatler, babler’ (Cotgr.), ad. It. ciarlatano = ciarlatore babbler, patterer, mountebank, f. ciarlare to babble, patter, act the mountebank, f. ciarla, chat, prattle; cf. Sp., Pg. charlar, Wallachian charrar, ONF. charer (Diez) to prattle, babble. Cf. quack to gabble like a duck, talk like a Cheap Jack, puff patent medicines, act as a charlatan.]
\vspace{-0.3cm}

\begin{myenumerate}

\itembf{A.} n.

\itembf{1.} A mountebank or Cheap Jack who descants volubly to a crowd in the street; esp. an itinerant vendor of medicines who thus puffs his ‘science’ and drugs. (Now included under 2.)

[\P 1605 B. JONSON  \textit{Volpone} ii. ii, The Rabble of these ground Ciarlitani, that spred their Clokes on the Pavement.    
\P 1611 CORYAT  \textit{Crudities} Panegyr. Verses, Sometimes to hear the Ciarlatans.]    
\P 1618 D. BELCHIER  \textit{Hans Beer-pot} D j b, I think the Serieant is grown Mountebancke To cling by shifts, hey, passe, passe, Italian grown; a sharking Charlatan.    
\P 1646 SIR T. BROWNE  \textit{Pseud. Ep.} i. iii. 11 Saltimbancoes, Quacksalvers, and Charlatans, deceive them in lower degrees.    
\P 1678 BUTLER  \textit{Hud.} iii. ii. 971 For Chiarlatans can do no good, Vntil th' are mounted in a Crowd.    
\P 1771 MRS. HARRIS in \textit{Priv. Lett. 1st Ld. Malmesbury} I. 214 At the masquerade‥Mr. Banbury was a most excellent friseur, Lord Berkeley a charlatan.    [
[\P 1864 BURTON  \textit{Scot Abr.} I. iii. 145 He is called a charlatan, quack, and mountebank.]

\itembf{b.} One who puffs his wares; a puffer.

\P 1670 COTTON  \textit{Espernon} Pref., Though in the foregoing Paragraph, I have discover'd something of the Charlatan in the behalf of my Bookseller.

\itembf{2.} An empiric who pretends to possess wonderful secrets, esp. in the healing art; an empiric or impostor in medicine, a quack.

\P 1680 BUTLER  \textit{Rem.} (1759) II. 197 Charlatans make Diseases fit their Medicines, and not their Medicines Diseases.    
\P 1710 ADDISON  \textit{Tatler} No. 240 \cardo{⁋}3 Ordinary Quacks and Charlatans.    [
\P 1762 J. BROWN  \textit{Poetry \& Mus.} iii. 34 note, Charlatans, a Word with which we have none precisely correspondent in our Language: It signifies here, one who is a Pretender to Medecine by the Arts of Magic.]    
\P 1791 BURKE  \textit{Let. Memb. Nat. Assembly} Wks. 1842 I. 478  The nation is sick, very sick, by their medicines. But the charlatan tells them that what is passed cannot be helped.    
\P 1841 BREWSTER  \textit{Mart. Sc.} ii. iv. (1856) 153 The charlatans, whether they deal in moral or in physical wonders, form a race which is never extinct.    
\P 1860 TANNER  \textit{Pregnancy} i. 3.

\itembf{3.} An assuming empty pretender to knowledge or skill; a pretentious impostor.

\P 1809 \textit{Edin.  Rev.} Apr. 193 The Alexandrian sages [Proclus, etc.]‥were in fact the charlatans of antient philosophy.    
\P 1840 CARLYLE  \textit{Heroes} (1858) 268 A questionable step for me‥to say‥that Mahomet was a true Speaker at all, and not rather an ambitious charlatan.    
\P 1858 FROUDE  \textit{Hist. Eng.} III. xvi. 363 His [Cromwell's] true creed was a hatred of charlatans.    
\P 1872 GEO. ELIOT  \textit{Middlem.} v. xlv. 335 A charlatan in religion is sure to like other sorts of charlatans.

\itembf{B.} adj. Of or pertaining to a charlatan; empirical, quack.

\P 1671 \textit{True  Non-Conf.} 376 But the schareleton tricks of a pitiful impostor.    
\P 1852 GLADSTONE  \textit{Glean.} IV. ii. 141 Theatrical, not to say charlatan and mountebank, politics.    
\P 1862 SHIRLEY  \textit{Nugæ Crit.} xi. 472 Because I love freedom‥I hesitate to apply the charlatan quackeries which may fatally hurt all that is best and most living in English liberty.
\end{myenumerate}


%%%%%%%%%%%%%%%%%%%%%%%%%%%%%%%%%
\myitem{chary} a.

\noindent \phonetic{(ˈtʃɛərɪ)}

\noindent [OE. \phonetic{ceariᴁ} = OS. carag (in môdcarag), OHG. charag: —OTeut. type *karag-oz, f. karâ- sorrow, trouble, care. With the sense-development cf. careful.

   The palatalization of initial ca- in this word, while it remains guttural in
care, is thus accounted for: in the n. the original OE. type was nom. caru, gen.
*cære, whence ceare (cf. cæster, ceaster etc.); so app. the derivative
\phonetic{*cæriᴁ}, whence \phonetic{ceariᴁ}, with palatal ce- becoming ch-. But the n. retained guttural c in the nom. (even when by u- umlaut it was occasionally written cearu), so that no such form as chare is found in ME. As to sense 3 cf. chare a.]
\vspace{-0.3cm}

\begin{myenumerate}

\itembf{1.} Causing sorrow, grievous. Obs.

\P a1000 \textit{Doomsday}  67 (Gr.) Wæs Meotud on beam bunden fæste cearian clomme.

\itembf{2.} Feeling or showing sorrow; sorrowful, mournful. Obs.

\P a1000 \textit{Crist}  148 (Gr.) Hie bidon hwonne bearn Godes cwome to \phonetic{ceariᴁum}.
\P a1000 \textit{Soul's  Address} 162 (Gr.) Ne þurfon wyt beon cearie.
\P c1200 ORMIN 1274 For  turrtle ledeþþ chariȝ lif‥fra þatt hire make iss dæd.

\itembf{3.} Dear; precious, cherished. Obs.

\P ?a1400 \textit{Morte  Arth.} 2965 Ffore the charry childe so his chere chawngide, That the chillande watire one his chekes rynnyde!    
\P 1593 PEELE  \textit{Edw. I}, 200 And henceforth see you call it Charing-cross; For why, the chariest and the choicest queen, That ever did delight my royal eyes There dwells.
\P a1600 W. ELDERTON  in Farr \textit{S.P. Eliz. II.} 514 O God, what griefe is this thye charie church should want A bishoppe of so good a grace.    
\P 1610 HOLLAND  \textit{Camden's Brit.} i. 253 Things of charie price.    
\P 1820 SCOTT  \textit{Monast.} xxix, Fill the stirrup cup‥from a butt yet charier than that which he had pierced for the former stoup.

\itembf{4.} Careful, cautious, circumspect, wary.

\P 1542 UDALL tr. \textit{Erasm. Apoph.} 221 b, I am much more charie, that it may not be lost.    
\P 1566  \textit{Answ. Examination pretending to mayntayne Apparell, etc.} 148 Those prudent and chairie ouerseers which tythe mint and anice.    1625–8 tr. Camden's Hist. Eliz., I‥have not touched them but with a light and chary hand.    
\P 1857 SIR F. PALGRAVE  \textit{Norm. \& Eng.} II. 343 Yet in this concession, he was very chary.    
\P 1878 G. MACDONALD  \textit{Phantastes} II. xiii. 15 Enough to madden a chary lover.

\itembf{b.} Fastidious, shy, particular.

\P 1567 DRANT  \textit{Horace's Epist.} ii. ii. H iv, Whilste theye indite, and reade theire toyes, Moste chearie and most coy.    
\P 1592 GREENE  \textit{Ciceronis Amor.}, Man having swilled in this nectar of Love is so chary that he‥admitteth no partaker of her favours.    
\P 1602 SHAKES.  \textit{Ham.} i. iii. 36 The chariest Maid is Prodigall enough, If she vnmaske her beauty to the Moone.    
\P 1834 MUDIE  \textit{Brit. Birds} (1841) I. 114 Another [eagle]‥not quite so chary in its food as the former.

\itembf{c.} Const. in, of. Shy of, disinclined to.

\P 1579 G. HARVEY  \textit{Letter-bk.} (1884) 66 To be very chary and circumspect in opening himselfe.    
\P 1673 MARVELL  \textit{Reh. Transp.} ii. Wks. (1875) II. 253 Men ought to be chary of aspersing them [the clergy].    
\P 1828 SCOTT  \textit{F.M. Perth} vi, Chary of mixing in causeless strife.    
\P 1883 \textit{19th  Cent.} May 882 Crown authorities were very chary in putting it in force.    
\P 1884 \textit{Law  Times} 16 Feb. 278/1 Tradesmen chary of allowing vessels to leave port prior to payment.

\itembf{5.} Careful (in preservation of). Const. of, †over.

\P 1579 GOSSON  \textit{Sch. Abuse} (Arb.) 58 If you bee chary of your good name.    
\P 1598 GREENE  \textit{James IV} (1861) 219 With chary care I have recur'd the one.    
\P 1598 YONG  \textit{Diana} 390 Her father was so tender and charie ouer her, that few times he suffered her to be out of his sight.    
\P 1638 COWLEY  \textit{Love's Riddle} i. i, 'Faith, I am very Chary of my Health.
\P c1645 HOWELL  \textit{Lett.} (1650) I. 221 The curious sea-chest of glasses‥which I shall be very chary to keep as a monument of your love.    
\P 1754 RICHARDSON  \textit{Grandison} III. viii. 56 Be chary of them, and return them when perused.    
\P 1820 SCOTT  \textit{Monast.} xxiv, In reference to your safety and comfort, of which he desires us to be chary.

\itembf{6.} Careful not to waste or part with, frugal, sparing (of).

\P 1570 LEVINS  \textit{Manip.} 106 Cheyrye, parcus.    
\P 1592 GREENE  \textit{Disput.} 4 Hee that is most charie of his crownes abroad.    
\P 1756 C. LUCAS  \textit{Ess. Waters} I. 154 They drank nothing but water, of which they were very chary.    
\P 1826 SCOTT  \textit{Woodst.} iii. They were more chary of their royal presence.    
\P 1868 M. E. BRADDON  \textit{Dead Sea Fr. I.} ii. 20 He had much need to be careful of shillings, and chary even of pence.    
\P 1872 W. MINTO  \textit{Eng. Lit.} ii. vii. 478 He is rather chary than enthusiastic.    
\P 1874 SAYCE  \textit{Compar. Philol.} vii. 281 The primitive barbarian‥would have been extremely chary in his use of words.

\itembf{7.} Requiring care or careful handling. Obs.

\P 1581 MULCASTER  \textit{Positions} v. (1887) 28 The cheife and chariest point is, so to plie them all, as they may proceede voluntarily.

\itembf{8.} quasi-adv. Charily; carefully.

\P 1590 MARLOWE  \textit{Faust.} vi. 175 Thanks, Mephistophilis, for this sweet book, This will I keep as chary as my life.
\P a1600 W. ELDERTON  in Farr \textit{S.P. Eliz. II.} 513 And charie went to churche himself.
\P c1600 SHAKES.  \textit{Sonn.} xxii, Which I will keepe so chary, As tender nurse her babe.    
\P 1633 HEYWOOD  \textit{Eng. Trav.} iii. Wks. 1874 IV. 44 Let  men live as charie as they can.
\P a1845 HOOD  \textit{Mary's Ghost} v, You thought that I was buried deep, Quite decent like, and chary.
\end{myenumerate}


%%%%%%%%%%%%%%%%%%%%%%%%%%%%%%%%%
\myitem{chasm} n.

\noindent \phonetic{(ˈkæz(ə)m)}

\noindent [ad. L. chasma, a. Gr. χάσµα yawning hollow. The Gr.-L. form chasma was used for some time unchanged.]
\vspace{-0.3cm}

\begin{myenumerate}

\itembf{1.} A yawning or gaping, as of the sea, or of the earth in an earthquake. Obs.

\P 1596 C. FITZGEFFREY  \textit{Sir F. Drake} (1881) 31 Earth-gaping Chasma's, that mishap aboades.
\P a1619 M. FOTHERBY  \textit{Atheom.} ii. ii. §1 That gaping Chasma, and insatiable gulfe of the Soules appetite.    
\P 1652 FRENCH  \textit{Yorksh. Spa} ii. 31 Chasmes, and gapings of the Sea.    
\P 1656 S. H. \textit{Gold.  Law} 91 Earthquakes, Chasmaes, and Voragoes were at his command.    
\P 1655–60 STANLEY \textit{Hist. Philos.} (1701) 331/1 Earthquakes, Chasma's, and the like.

\itembf{2.} An alleged meteoric phenomenon, supposed to be a rending of the firmament or vault of heaven. [So in Latin.] Obs.

\P 1601 HOLLAND  \textit{Pliny} I. 17 The firmament also is seene to chinke and open, and this they name Chasma.    
\P 1686 GOAD  \textit{Celest. Bodies} i. i. 1 Halo's, Rainbows, Parelia, Paraselenæ, Chasms.    
\P 1741 SHORT in \textit{Phil. Trans.} XLI. 630 A list of all the Chasms or Burnings in the Heavens, recorded in our Annals.

\itembf{3.} A large and deep rent, cleft, or fissure in the surface of the earth or other cosmical body. In later times extended to a fissure or gap, not referred to the earth as a whole, e.g. in a mountain, rock, glacier, between two precipices, etc.

\P 1636 C. FITZGEFFREY  \textit{Bless. Birthd.} (1881) 147 Thus is th' Abyssus fild, the Chasma clos'd.    1622–62 Heylin Cosmogr. Introd. (1682) 23 The open chinks or Chasmaes of the Earth.    
\P 1695 WOODWARD  \textit{Nat. Hist. Earth} iii. §1. 134 This Effort‥in some Earthquakes‥tears the Earth, making Cracks or Chasmes in it some Miles in length.    
\P 1704 J. HARRIS  \textit{Lex. Techn.} s.v., The Water of this vast Abyss‥doth communicate with that of the Ocean by means of certain Holes, Hiatus's or Chasms, passing betwixt it and the Bottom of the Ocean.    
\P 1840 CARLYLE  \textit{Heroes} i. (1858) 196 Iceland‥with its‥horrid volcanic chasms.    
\P 1860 TYNDALL  \textit{Glac.} i. §7. 49 An arch of snow‥may span a chasm one hundred feet in depth.    
\P 1878 HUXLEY  \textit{Physiogr.} 135 The Colorado River‥flows‥at the bottom of a profound chasm.

\itembf{4.} A deep gap or breach in any structure; a wide crack, cleft, or fissure. Also fig.

\P 1626 W. SCLATER  \textit{Expos.} 2 Thess. (1629) 26 Heauen it selfe, and the great Chasma betwixt it and vs.    
\P 1672 WILKINS  \textit{Nat. Relig.} 107 So many chasmes or breaches must there be in the Divine Nature.    1756–7 tr. Keysler's Trav. (1760) III. 356 The amphitheatre of Verona‥has no holes or chasms in the wall.    
\P 1759 tr. Duhamel's Husb. i. v. (1762) 11 An infinite number of small chasms between them, into which the roots may glide.    
\P 1815 SCOTT  \textit{Guy M.} iv, This part of the castle‥exhibited a great chasm, through which Mannering could observe the sea.

\itembf{5.} fig. A break marking a divergence, or a wide and profound difference of character or position, a breach of relations, feelings, interests, etc.

\P 1641 R. BROOKE  \textit{Eng. Episc.} 99 Where then is that Chasma, that great Gulf of difference?    
\P 1660 H. MORE  \textit{Myst. Godl.} i. iv. 9 That great Chasma betwixt God and Matter will be as wide as before.    
\P 1845 S. AUSTIN tr. \textit{Ranke's Hist. Ref.} II. 203 The two hierarchies, the spiritual and the temporal‥were now separated by a deep and wide chasm.    
\P 1866 LIDDON  \textit{Bampt. Lect.} i. (1875) 25 If Christ be not truly man, the chasm which parted earth and heaven has not been bridged over.    
\P 1875 HAMERTON  \textit{Intell. Life} x. v. 390 A gulf‥almost like the chasm of death.

\itembf{6.} fig. A break or void affecting the continuity of anything, as of a chain of facts, a narrative, period of time, etc.; an intervening blank, hiatus, break, interval.

\P 1654 R. WHITLOCK  \textit{Zootomia} 216 Authors with many Plurima Desunts, many Chasmes and vacancys.    
\P 1677 HALE  \textit{Prim. Orig. Man.} 137 It is carried down from the beginning of Time‥without any chasma or interval.    
\P 1704 SWIFT  \textit{T. Tub} Author's Apol., In the author's original Copy there were not so many Chasms as appear in the book.    
\P 1712 ADDISON  \textit{Spect.} No. 519 \cardo{⁋}7 The whole chasm of nature, from a plant to a man, is filled up with divers kinds of creatures.    1762–71 H. Walpole Vertue's Anecd. Paint. (1786) I. 189 The fables with which our own writers have replenished the chasms in our history.    
\P 1843 CARLYLE  \textit{Past \& Pr.} (1858) 109 The chasm of Seven Centuries.    
\P 1869 J. MARTINEAU  \textit{Ess.} II. 52 There is an historical chasm manifest in their modes of thinking.

\itembf{7.} A vacant place affecting the completeness of anything; a void, blank, gap.

\P 1759 tr. \textit{Duhamel's Husb.} ii. (1762) 125 Some chasms occasioned by our not having kept the drill in a parallel direction.    
\P 1838 MACAULAY  \textit{Let.} in \textit{Trevelyan Life} (1876) II. 2 The chasm Tom's departure has made.    
\P 1855 \textit{Hist. Eng.} III. 580 Recruits were sent to fill the chasms which pestilence had made in the English ranks.
\end{myenumerate}


%%%%%%%%%%%%%%%%%%%%%%%%%%%%%%%%%
\myitem{chicanery} n.

\noindent \phonetic{(ʃɪˈkeɪnərɪ)}

\noindent [a. F. chicanerie, in Littré the earliest exemplified member of the group, implying however the existence of the vb. chicaner and n. chicaneur as its source: see -ery. Formerly more completely anglicized as chicanry.]
\vspace{-0.3cm}

\begin{myenumerate}

\itembf{1.} Legal trickery, pettifogging, abuse of legal forms; the use of subterfuge and trickery in debate or action; quibbling, sophistry, trickery.

\P a1613 OVERBURY  \textit{Observ. State France} (1856) 241 All this chiquanerey, as they call it, is brought into France from Rome.    
\P 1665 EVELYN  \textit{Lett. Sir P. Wyche} 20 June, We have hardly any words that do so fully expresse the French clinquant, naiveté‥chicaneries.
\P a1670 HACKET  \textit{Abp. Williams} ii. (1692) 151, I shall not advise this honourable House to use any chiquanery or pettiffoggery with this great representation of the kingdom.    
\P 1682 BURNET  \textit{Rights Princes} Pref. 57 To do it with all the Tricks and Chicanery possible.    
\P 1704 J. HARRIS  \textit{Lex. Techn.}, Chicanry, is a trickish and guileful Practice of the Law.    
\P 1708 OZELL  \textit{Boileau's Lutrin} v. (1730) 53 That foul Monster, void of Ears and Eyes, Call'd Chicanry.    
\P 1754 RICHARDSON  \textit{Grandison} (1781) IV. ii. 14 It was‥by the chicanery of the lawyers‥carried against him.    
\P 1827 HALLAM  \textit{Const. Hist.} II. xii. The period of lord Danby's administration‥was full of chicanery and dissimulation on the King's side.    
\P 1876 GREEN  \textit{Short Hist.} viii. §8. Forty days wasted in useless chicanery.

\itembf{b.} as a personal quality.

\P 1771 SMOLLETT  \textit{Humph. Cl. let.} 26 June, He carried home with him all the knavish chicanery of the lowest pettifogger.    
\P 1832 LANDER  \textit{Adv. Niger} III. xvi. 256 The artifice, chicanery and low cunning of a crafty and corrupt mind.

\itembf{2.} (with pl.) A dishonest artifice of law; a sophistry, quibble, subterfuge, trick.

\P 1688  \textit{Answ. Talon's Plea} 23 Pitifull Chicanneries and tricks of the Law.    
\P 1758 JORTIN  \textit{Erasm.} I. 103 These letters‥full of chicaneries about trifles.    
\P 1878 R. B. SMITH  \textit{Carthage} 227 Impatient of such chicaneries.
\end{myenumerate}


%%%%%%%%%%%%%%%%%%%%%%%%%%%%%%%%%
\myitem{chide} v.

\noindent \phonetic{(tʃaɪd)}

\noindent [OE. cíd-an wk. vb.: not known in the other Teutonic langs.
The original inflexions were: pa. tense OE. cídde, ME. chidd(e, chid, mod. chid; pa. pple. OE. cíded, cidd, cid, ME. chidd(e, chid, mod. chid; but in 5–6 chode, chidden formed on the analogy of the strong verbs (e.g. ride), came into partial use, and chidden at least is still common; chided is occasional in modern writers. (OE. and ME. contracted the 3rd pers. pres. indic. as cít, chit.)]
\vspace{-0.3cm}

\begin{myenumerate}

\itembf{1.} intr. To give loud or impassioned utterance to anger, displeasure,
disapprobation, reproof. \textbf{a.} To contend with loud and angry altercation; to brawl, wrangle. Obs.

\P c1000 ÆLFRIC  \textit{Exod.} xxi. 18 Gif men cidaþ.
\P c1050 \textit{Gloss.} in Wr.-Wülcker 347 Altercaretur, cidde.
\P c1205 LAY.  8149 Heo bigunnen to chiden.
\P c1250 \textit{Gen. \& Ex.}  2722 He saȝ chiden in ðe wey two egypcienis, modi \& strong.
\P c1340  \textit{Cursor M.} 6681 (Trin.) If two chide [earlier texts, flite] \& þat oon þe toþer smyte.
\P c1460 \textit{Towneley  Myst.} 115 We wille nawther‥Fyght nor chyte.    
\P 1483 \textit{Cath.  Angl.} 63/1 To chyde, litigare‥ubi, to flyte.    
\P 1552 \textit{Act  5 \& 6 Edw. VI}, c. 4 §1 Yf anye person‥shall‥by wordes onelye quarrell, chyde or brawle in any Churche or Churcheyarde.    
\P 1693 W. ROBERTSON  \textit{Phraseol. Gen.} 329 They did chide and brawl so long till they fell together by the ears.

\itembf{b.} To give loud and angry expression to dissatisfaction and displeasure; to scold. Obs.

\P 1175 \textit{Lamb.  Hom.} 113 Crist nalde flitan ne chidan.    
\P 1297 R. GLOUC.  (Rolls) 8024 He chydde \& made hym wroþ.    
\P 1340  \textit{Ayenb.} 67 Þe ilke þet ne dar ansuerye ne chide‥he beginþ to grochi betuene his teþ.    
\P 1377 LANGL.  \textit{P. Pl.} B. i. 191 Chewen heore charite and chiden after more.
\P c1386 CHAUCER  \textit{Can. Yeom. Prol. \& T.} 368 Whan that oure pot is broke‥Every man chyt.
\P c1440 \textit{York  Myst.} xxvi. 180 Þou chaterist like a churle þat can chyde.    
\P 1529 MORE  \textit{Comf. agst. Trib.} ii. Wks. 1187/2 Other folk‥had a good sporte to heare her chide.    17‥ Swift Lett. (1766) II. 293, I am confident you came chiding into the world, and will continue so while you are in it.

\itembf{c.} To scold by way of rebuke or reproof; in later usage, often merely, to utter rebuke.

\P 1393 LANGL.  \textit{P. Pl.} C. iv. 224 Ich cam noȝt to chiden.    
\P 1535 COVERDALE  \textit{Ps.} cii[i]. 9 He wil not allwaye be chydinge.    
\P 1660 MILTON  \textit{Sonn.} xiv, To serve therewith my Maker, and present My true account, lest he, returning, chide.    
\P 1764 GOLDSM.  \textit{Hermit} xxxvii, The wondering fair one turned to chide.
\P a1839 PRAED  \textit{Poems} (1864) I. 301 To smile on me, to speak to me, to flatter or to chide.

\itembf{d.} fig. Applied to sounds which suggest angry vehemence: as the yelping of hounds in ‘cry’, the querulous notes of quails, ‘brawling’ of a torrent, angry blast of the wind, etc.

\P 1594 \textit{2nd Rep. Faustus} xxii. in Thoms Prose Rom. (1858) III. 397 His javelin‥being denied entrance, for very anger, rent itself in forty pieces, and chid in the air.    
\P 1615 G. SANDYS  \textit{Trav.} 27 Partridges‥flie chiding about the vine$\sim$yards.    
\P 1620 MELTON  \textit{Astrolog.} 3 The lowdest storme that could ever chide.    
\P 1820 KEATS  \textit{Eve St. Agnes} iv, The silver snarling trumpets 'gan to chide.

\itembf{2.} Const. \textbf{a.} In OE. construed with dative of personal object, in sense ‘to rebuke’; later, with various preps., esp. at; hence by levelling of dat. and acc. the trans. sense 3. Obs.

\P c1000 \textit{Ags. Gosp. Mark} i. 25 Ða cydde se hælend him.
\P c1160 \textit{Hatton  G.} ibid., Þa kydde se hælend hym.    
\P 1393 GOWER  \textit{Conf.} I. 295 If‥thou at any time hast chid Toward thy love.    
\P 1588 SHAKES.  \textit{L.L.L.} iv. iii. 132 You chide at him, offending twice as much.    
\P 1591 \textit{Two Gent.} ii. i. 78 You chidde at Sir Protheus, for going vngarter'd.

\itembf{b.} with with: To complain aloud against (so later, to chide against); to quarrel or dispute angrily with; to have altercation with. Obs.

\P a1000 THORPE  \textit{Hom.} I. 96 (Bosw.) Cide he wið God.
\P c1175 \textit{Lamb.  Hom.} 103 Þe mon sorȝeð‥and chit þenne wið gode.
\P a1250 \textit{Owl  \& Night.} 287 Ne lust me wit the screwen chide.
\P a1300  \textit{Cursor M.} 12972 (Cott.) Yeitt can þat chinche wit godd to chide.   
\P 1382 WYCLIF  \textit{Judg.} xxi. 22 Whanne the faders of hem comen and aȝens ȝou bigynnen to pleyne and chiden.
\P a1450  \textit{Knt. de la Tour} (1868) 21 She‥chidde with hym afore alle the peple.    
\P 1513 DOUGLAS  \textit{Æneis} viii. Prol. 126 Churle, ga chat the and chyd with ane vther.    
\P 1535 COVERDALE  \textit{Gen.} xxxi. 36 And Iacob was wroth, and chode with Laban [so 1611].    
\P 1611 BIBLE  \textit{Ex.} xvii. 2 Why chide you with mee?    
\P 1693 W. ROBERTSON  \textit{Phraseol. Gen.} 329 To chide or quarrel with one.    
\P 1869 SPURGEON  \textit{J. Ploughm. Talk} 6 We have a stiff bit of soil to plough when we chide with sluggards.

\itembf{3. a.} trans. To address (a person) in terms of reproof or blame: in earlier use implying loud vehemence, to ‘scold’; in later use often little more than ‘reprove, rebuke’. (The main modern use, but now chiefly literary, and somewhat archaic).

   This comes down directly from the OE. const. with the dative, which may still be valid for early ME. examples. The later examples show modern instances of inflected forms.

\P c1230 \textit{Hali  Meid.} 31 Chit te \& cheopeð þe \& schent te schomeliche.
\P a1250 \textit{Owl \& Night.} 1329 AH ȝet  thu, fule thing, me chist.
\P c1340  \textit{Cursor M.} 13867 (Trin.) For iewes so had him chid.    
\P 1387 TREVISA  \textit{Higden} (Rolls) VII. 35 Þere Dunston was strongliche despised and i-ched.    
\P 1430 LYDG.  \textit{Chron. Troy} ii. xii, Ye shall heare anone how that he chit The quene Heleyne.    
\P 1557 \textit{K. Arthur}  (W. Copland) vii. vi, Euer she chode him and wolde not rest.    
\P 1596 SHAKES.  \textit{1 Hen. IV}, ii. iv. 410 Thou wilt be horrible chidde to morrow.    
\P 1629 J. COLE  \textit{Of Death} 32 Peevish children, who‥are but chidden in their first schoole.    
\P 1646 SIR R. MURRAY  in \textit{Hamilton Papers} (Camden 1880) 108 You encourage me‥when I should rather be chid for it.    
\P 1720 GAY  \textit{Poems} (1745) II. 64 The Priest‥First chid her, then her sins remitted.    
\P 1751 JOHNSON  \textit{Rambl.} No. 182 \cardo{⁋}5 Having chidden her for undutifulness.    
\P 1791 COWPER  \textit{Iliad} xvii. 520 He stroked them gently and as oft he chode.    
\P 1847 TENNYSON  \textit{Princ.} vi. 271 Kiss and be friends, like children being chid!    
\P 1848 A. JAMESON  \textit{Leg. Monast. Ord.} Introd. (1863) 40 The monks have been sorely chidden for [this].    
\P 1861 P. YOUNG  \textit{Daily Readings} II. 298 Our Lord‥chode them for their want of faith.    
\P 1865 MEREDITH  \textit{Rhoda Fleming} I. x. 164 The farmer chid her.    
\P 1870 BRYANT  \textit{Iliad} I. iv. 121 Atrides‥spake and chid them.    
\P 1879 BEERBOHM  \textit{Patagonia} vi. 97, I have never seen a child chided or remonstrated with.    
\P 1885 MRS. CAMPBELL  \textit{Praed Head Station} xxiii, Mrs. Clephane‥chided Jinks.    
\P 1897  \textit{Daily News} 15 Apr. 6/3 We‥notice with interest that Mr. Meredith, after vacillating in former editions between ‘chid’ and ‘chidded’, has now resolved that the past tense of ‘to chide’ is ‘chided’.    
\P 1925 C. S. DURRANT  \textit{Flem. Mystics \& Eng. Martyrs} i. x. 146 Margaret‥quietly chode her elder.

\itembf{b.} fig. and transf. To scold, rebuke, or find fault with (a thing, an action, etc.).

\P 1386 CHAUCER  \textit{Nun's Pr. T.} 531 The Friday for to chiden‥(For on a Fryday sothly slayn was he).    
\P 1590 SHAKES.  \textit{Mids. N.} iii. ii. 200 Wee haue chid the hasty footed time, For parting vs.    
\P 1606 \textit{Tr. \& Cr.} ii. iii. 221 The Rauen chides blacknesse.    
\P 1770 GOLDSM.  \textit{Des. Vill.} 150 He chid their wanderings, but relieved their pain.    
\P 1776 GIBBON  \textit{Decl. \& F.} I. xi. 303 The emperor‥chided the tardiness of the senate.    
\P 1860 CARD.  \textit{Wiseman Past. Lett.} 25 Mar. 20 Could that power have been reproved, chided, and even corrected‥by so dependent an authority?    
\P 1865 SWINBURNE  \textit{Poems \& Ball.}, Ilicet 137 Before their eyes all life stands chidden.

\itembf{c.} Said of hounds, brawling streams, etc.

\P 1590 SPENSER  \textit{F.Q.} i. i. 1 His angry steede did chide his foming bitt.    
\P 1596 SHAKES.  \textit{1 Hen. IV}, iii. i. 45 The Sea That chides the Bankes of England.    
\P 1697 DRYDEN  \textit{Virg. Eclog.} v. 132 Streams that‥the scarce cover'd Pebbles gently chide.    
\P 1810 SCOTT  \textit{Lady of L.} i. viii, The baffled dogs‥Chiding the rocks that yell'd again.

\itembf{4.} With adv. or advb. compl.: To drive, impel, or compel by chiding.

\P 1590 SHAKES.  \textit{Mids. N.} iii. ii. 312 He hath chid me hence.    
\P 1633 G. HERBERT  \textit{Temple}, Church Militant 105 He chid the Church away.    
\P 1634 MILTON  \textit{Comus} 258 Scylla‥chid her barking waves into attention.    
\P 1643 J. ANGIER  \textit{Lanc. Vall. Achor} 29 This seasonable check chode us to duty.    
\P 1738 WESLEY  \textit{Hymns}, ‘Triumphal Notes’ ii, Thy Word bids Winds and Waves be still, And chides them into Rest.    
\P 1836 EMERSON  \textit{Nature, Lit. Ethics} Wks. (Bohn) II. 219 Be neither chided nor flattered out of your position.
\end{myenumerate}


%%%%%%%%%%%%%%%%%%%%%%%%%%%%%%%%%
\myitem{chimera} chimæra, n.

\noindent \phonetic{(kɪˈmɪərə, kaɪ-)}

\noindent [ME. chimere, a. F. chimère, ad. L. chimæra, a. Gr. χίµαιρα she-goat
or monster, f. χίµαρ-ος he-goat. Since the 16th c. the earlier form from Fr. has
been supplanted by its Latin original. As chimere was certainly
(\phonetic{ˈtʃɪmɛr}), the two spoken forms are practically distinct words.]
\vspace{-0.3cm}

\begin{myenumerate}

\itembf{1. a.} A fabled fire-breathing monster of Greek mythology, with a lion's head, a goat's body, and a serpent's tail (or according to others with the heads of a lion, a goat, and a serpent), killed by Bellerophon.

\P 1382 WYCLIF  \textit{Bible Prol.} 31 Beestis clepid chymeres, that han a part of ech beest, and suche ben not, no but oonly in opynyoun.
\P c1430 LYDG.  \textit{Bochas} i. lv, The Chimere of Licy.
\P a1528 SKELTON \textit{P. Sparowe} 1334 BY Chemeras  flames.    
\P 1600 FAIRFAX  \textit{Tasso} viii. xviii, New Chimeres, Sphinges, or like monsters bred.    
\P 1613 HEYWOOD  \textit{Silver Age} i. i. Wks. 1874 III.  89 That monstrous beast of Cicily Cal'd the Chimera.    
\P 1667 MILTON  \textit{P.L.} ii. 628 All monstrous, all prodigious things‥worse Then fables yet have feign'd, or fear conceiv'd, Gorgons and Hydra's, and Chimera's dire.    
\P 1751 SMOLLETT  \textit{Per. Pic.} lxiv, A convocation of chimeras breathing fire and smoke.    
\P 1831 LANDOR  \textit{Siege Ancona} Wks. 1846 II. 584  The flames and coilings of the fell Chimæra.

\itembf{b.} Any fish of the family Chimæridæ; = rabbit-fish. (Cf. chimæroid a.)

\P 1804 HOLLOWAY  \& BRANCH \textit{Brit. Museum} III. 56 The Chimæra, or Chimæra Monstrosa, belongs to that class of fish which have close gills and cartilages instead of bones.    
\P 1808 E. DONOVAN  \textit{Nat. Hist. Brit. Fishes} V. Plate CXI, There are two species of the Chimæra genus, Monstrosa, and Callorhynchus; the latter of which is distinguished by the name of Southern Chimera and Elephant Fish.    
\P 1836 W. YARRELL  \textit{Hist. Brit. Fishes} II. 365 The Northern Chimæra is represented as a fish of singular appearance and beauty, a native of the northern seas only, where it seldom exceeds three feet in length.    
\P 1848 [See  RABBIT n.1 4].    
\P 1969 A. WHEELER  \textit{Fishes Brit. Isles} 111 The chimaeras are deep-water fishes, living on or below the edge of the continental shelf.

\itembf{2.} In Painting, Arch., etc. A grotesque monster, formed of the parts of various animals.

\P 1398 TREVISA  \textit{Barth. De P.R.} xix. xxxvii. (1495) 879 Somtyme they‥bryngyth to lesynges as he dooth that paynteth Chymera with thre heedes.]    
\P 1634 JACKSON  \textit{Creed} vii. xi, Chimeras, or painted devices which represent no visible creature.    
\P 1636 B. JONSON  \textit{Discov.}, He complains of their painting Chimaeras, by the vulgar unaptly called grotesque.    
\P 1711 ADDISON  \textit{Spect.} No. 83 \cardo{⁋}7 The third Artist‥had an excellent Hand at a Chimera.    
\P 1876 H. N. HUMPHREYS  \textit{Coin-Coll. Man.} vi. 66 The Chimæra enriching the helmet is the monster Scylla.

\itembf{3.} fig. with reference to the terrible character, the unreality, or the
incongruous composition of the fabled monster: \textbf{a.} A horrible and fear-inspiring phantasm, a bogy.

\P 1514 BARCLAY  \textit{Cyt. \& Uplondyshm.} (1847) 72 Against the Chimer here stoutly must he fight.    
\P 1601 CORNWALLYES  \textit{Ess.} xvii, Chimeræs, begotten betweene Feare, and Darknesse, which vanish with the Light.    
\P 1730 THOMSON \textit{Autumn} 1145 Full  of pale fancies and chimeras huge.    
\P 1856 FROUDE  \textit{Hist. Eng.} (1858) I. v. 429 The nation‥exorcised the chimæra with a few resolute words for ever.

\itembf{b.} An unreal creature of the imagination, a mere wild fancy; an unfounded conception. (The ordinary modern use.) See also bombinate.

\P 1587 GOLDING  \textit{De Mornay} xxv. 379 How could that Chymera haue come in any mans minde?
\P c1645 HOWELL  \textit{Lett.} I. i. iv, That golden myne is proved a meer Chymera, an imaginary airy myne.    
\P 1712 ARBUTHNOT  \textit{John Bull} ii. iii, Exploded chimera's, the perpetuum mobile‥philosopher's stone, etc..    
\P 1796 MORSE  \textit{Amer. Geog.} II. 18 The sea-snake, or serpent of the ocean, is no longer counted a chimera.    
\P 1835 SIR J. ROSS  \textit{N.W. Pass.} xv. 237 The ‘chimera of a north-west passage’, as it has been termed.

\itembf{c.} An incongruous union or medley.

\P 1832 G. DOWNES  \textit{Lett. Cont. Countries} I. 27 The exterior of the Church‥is a chimera in architecture, being Doric below, Corinthian above, and Ionic in the middle.

\itembf{d.} Biol. [ad. G. chimäre (H. Winkler 1907, in Ber. d. Deut. Bot. Ges. XXV. 574).] An organism (commonly a plant) in which tissues of genetically different constitution co-exist as a result of grafting, mutation, or some other process.

\P 1911 D. H. CAMPBELL in \textit{Amer. Naturalist} XLV. 44 Such monstrous forms, for which Winkler proposes the name ‘chimæra’, are not hybrids in any true sense of the word, but have arisen from buds in which there was a mere mechanical coalescence of tissue from the two parent forms at the junction of the stock and graft.    
\P 1926 J. S. HUXLEY  \textit{Ess.} in \textit{Pop. Sci.} xviii. 259 If the front half of one species be grafted on to the back half of another species, both continue to differentiate, and a chimaera or mosaic organism is produced.    
\P 1968 \textit{Nature}  9 Nov. 596 (heading) Mouse chimaeras obtained by the injection of cells into the blastocyst.    
\P 1969  \textit{New Scientist} 16 Jan. 133/1 Cytogeneticists have found human mosaic individuals, trisomics and chimeras.

\itembf{4.} attrib. and Comb.

\P 1619 BP. J. WILLIAMS  \textit{Serm. Apparell} (1620) 20 For a woman‥to come vnto a Church Chimæra-like‥halfe male and halfe female.    
\P 1761 F. SHERIDAN  \textit{S. Bidulph} III. 138 Our sex, said he, have not such chimæra notions.

\noindent
Hence \textbf{chimeraship} nonce-wd.

\P 1843 CARLYLE  \textit{Past \& Pr.} (1858) 170 His serene Chimeraship.
\end{myenumerate}


%%%%%%%%%%%%%%%%%%%%%%%%%%%%%%%%
\myitem{churlish} a.

\noindent \phonetic{(ˈtʃɜːlɪʃ)}

\noindent [OE. cierlisc, or (without umlaut) ceorlisc, f. ceorl churl + -isc, -ish. Cf. carlish.]
\vspace{-0.3cm}

\begin{myenumerate}

\itembf{1.} Of or relating to a churl; of the rank or position of a churl; pertaining to churls, rustic, common, vulgar, mean. Obs. (or arch.)

\P 1000 \textit{Laws  Ine} 18 in Thorpe I. 114 (Bosw.) Gif cierlisc [ciorlisc MS. H, cyrlisc B] mon \phonetic{betyᴁen} wære.
\P c1000 ÆLFRIC  \textit{Gloss.} in Wr.-Wülcker 153/33 Cibarius, ceorlisc hlaf.    
\P 1154 \textit{O.E.  Chron.} an. 893 Sæton feawa cirlisce men.    
\P 1382 WYCLIF  \textit{1 Chron.} xxvii. 26 To the churlische werk‥and to the erthe tilieris, that wrouȝten the erth.
\P c1386 CHAUCER  \textit{Miller's Prol.} 61 But tolde his cherlisch tale in his manere.
\P c1440  \textit{Promp. Parv.} 72 Cherlyche or charlysche, rusticalis.    
\P 1867 FREEMAN  \textit{Norm. Conq.} I. App. 727 Tradition asserts Godwine to have been a man of churlish birth.

\itembf{b.} Applied to churl's mustard: see churl 7 b.

\P 1597 GERARD  \textit{Herbal} i. xx. §7. 210 The seeds of these churlish kindes of treacle mustarde.

\itembf{2.} Intentionally boorish or rude in behaviour; hard, harsh, ‘brutal’, surly, ungracious.

\P 1386 CHAUCER  \textit{Frankl. T.} 787 Fro his lust yet were hym leuere abyde Than doon so heigh a cherlyssh [v.r. cherlyssh, cherliche, cherles, cheerlissch] wrecchednesse.
\P a1450 LE Morte Arth. 1078 So Churlysshe  of maners in feld ne hale Ne know I none.
\P c1530 LD. BERNERS  \textit{Arth. Lyt. Bryt.} (1814) 488 The dolphyn stepte forthe‥and said to the kynge: Thou foule olde churlysshe vilaine!    
\P 1600 SHAKES.  \textit{A.Y.L.} v. iv. 98 The Retort courteous‥the Quip-modest‥the reply Churlish.    
\P 1611 BIBLE  \textit{1 Sam.} xxv. 3 The man was churlish and euill in his doings. [Coverd., harde, and wicked in his doynges.]    
\P 1684 BUNYAN  \textit{Pilgr.} ii. 13 That which troubleth me most is my churlish carriages to him when he was under his distress.    
\P 1701 DE FOE  \textit{Trueborn Eng.} Pref., It cannot be denied but we are in many Cases, and particularly to Strangers, the churlishest People alive.    
\P 1865 LIVINGSTONE  \textit{Zambesi} xxv. 520 We found the people more churlish than usual.

\itembf{b.} transf. Of beasts, natural forces and agents: Violent, rough, etc. (Now only fig.)

\P 1477 PASTON  \textit{Lett.} 794. III. 186 So that he be not chorlissh at a spore, as plungyng.    
\P 1600 SHAKES.  \textit{A.Y.L.} ii. i. 7 The Icie phange And churlish chiding of the winters winde.    
\P 1633 P. FLETCHER  \textit{Pisc. Ecl.} ii. xiii, From thence he furrow'd many a churlish sea.    
\P 1671 J. WEBSTER  \textit{Metallogr.} xxvi. 318 It is a strong and chirlish vomit.    
\P 1678 CUDWORTH  \textit{Intell. Syst.} i. v. 689 Rude and churlish Blasts of wind.    
\P 1754 HUXHAM in \textit{Phil. Trans.} XLVIII. 857 It always proved a very churlish medicine. [Cf. churlous.]

\itembf{3.} Sordid, niggardly, stingy, grudging.
   [See note to churl n. 6.]

\P 1566 PAINTER  \textit{Pal. Pleas.} I. 99 As he liued a beastly and chorlish life euen so he required to haue his funerall done after that manner.    
\P 1600 SHAKES.  \textit{A.Y.L.} ii. iv. 80 My master is of churlish disposition, And little wreakes to finde the way to heauen By doing deeds of hospitalitie.    
\P 1682 BUNYAN  \textit{Holy War} 191 Nor was I ever so churlish as to keep the commendations of them from others.    
\P 1810 SCOTT  \textit{Lady of L.} ii. xxxv, Thy churlish courtesy‥Reserve.    
\P 1866 MRS. H. WOOD  \textit{St. Martin's Eve} ii. (1874) 12 He could not offer a churlish roof to his visitors.

\itembf{4.} Of soil: Unkindly, stiff, hard, and difficult to work, ill repaying the husbandman's toil. Formerly also of metal: Difficult to work, intractable. Also transf. of difficulties, obstacles, etc. (Now fig.)

\P 1577 B. GOOGE  \textit{Heresbach's Husb.} i. (1586) 22 In Sommer the ground is to hard and churlishe.    
\P 1596 SHAKES.  \textit{1 Hen. IV}, v. i. 16 Will you againe unknit This churlish knot of all-abhorred Warre.    
\P 1626 BACON  \textit{Sylva} §326 If there be Emission of spirit, the body of the Metal will be hard and Churlish.    
\P 1650 FULLER  \textit{Pisgah} ii. xii. 250 In assigning the west border of this Tribe, we meet with a churlish difficulty in the text.    
\P 1662 \textit{Worthies} (1840) I. 365 It is not churlish but good-natured metal.
\P a1722 LISLE  \textit{Husb.} (1752) 3 Harsh, churlish, obstinate clay.    
\P 1764 GOLDSM.  \textit{Trav.} 168 Where the black Swiss‥force a churlish soil for scanty bread.    
\P 1840 DICKENS  \textit{Barn. Rudge} xli, A churlish strong-box or a prison-door.

\itembf{5.} Comb., as churlish-throated.

\P 1631 DRAYTON  \textit{Wks.} III. 918 (Jodd.) The churlish-throated hounds then holding him at bay.
\end{myenumerate}


%%%%%%%%%%%%%%%%%%%%%%%%%%%%%%%%
\myitem{cipher} cypher n.

\noindent \phonetic{(ˈsaɪfə(r))}

\noindent [a. OF. cyfre, cyffre (mod.F. chiffre) = Sp. Pg. It. cifra, med.L. cifra, cifera, ciphra, f. Arab. çifr the arithmetical symbol ‘zero’ or ‘nought’ (written in Indian and Arabic numeration \phonetic{٠}), a subst. use of the adj. çifr ‘empty, void’, f. çafara to be empty. The Arabic was simply a translation of the Sanscrit name śūnya, literally ‘empty’.]
\vspace{-0.3cm}

\begin{myenumerate}

\itembf{1. a.} An arithmetical symbol or character (o) of no value by itself, but which increases or decreases the value of other figures according to its position. When placed after any figure or series of figures in a whole number it increases the value of that figure or series tenfold, and when placed before a figure in decimal fractions, it decreases its value in the same proportion.

\P 1399 LANGL.  \textit{Rich. Redeles} iv. 53 Than satte summe, as siphre doth in awgrym, That noteth a place, and no thing availith.
\P c1400 \textit{Test.  Love} ii. (1560) 286 b/1 Although a sipher in augrim have no might in signification of it selve, yet he yeveth power in signification to other.    
\P 1547 J. HARRISON  \textit{Exhort. Scottes} 229 Our presidentes‥doo serue but as Cyphers in Algorisme, to fill the place.
\P a1593 H. SMITH  \textit{Serm.} (1622) 310 You are‥like cyphers, which supply a place, but signifie nothing.    
\P 1611 SHAKES.  \textit{Wint. T.} i. ii. 6 Like a Cypher (Yet standing in rich place) I multiply With one we thanke you, many thousands moe, That goe before it.    
\P 1660 MILTON  \textit{Free Commw.} 429 Only like a great Cypher set to no purpose before a long row of other significant Figures.    
\P 1718 J. CHAMBERLAYNE  \textit{Relig. Philos.} (1730) I. xvi. §22 With 39 Noughts or Cyphers following.    
\P 1801-15 M. EDGEWORTH  \textit{Frank} (ed. 2) III. 143 It was said‥that all Cambridge scholars call the cipher aught and all Oxford scholars call it nought.    
\P 1827 HUTTON  \textit{Course Math.} I. 4 The first nine are called Significant Figures, as distinguished from the cipher, which is of itself quite insignificant.

\itembf{b.} The zero-point, or zero, of a thermometer. U.S.

\P 1796 MORSE  \textit{Amer. Geog.} I. 475 The range of the quick$\sim$silver‥is between the 24th degree below, and the 105th degree above cypher.    
\P 1815 D. DRAKE  \textit{Cincinnati} ii. 94 From nine years observations, at Cincinnati, it appears that the thermometer falls below cypher twice every winter.

\itembf{2.} fig. \itembf{a.} A person who fills a place, but is of no importance or worth, a nonentity, a ‘mere nothing’.

\P 1579 LYLY  \textit{Euphues} (Arb.) 46 If one be hard in conceiuing they pronounce him a dowlte‥if without speach, a Cipher.    
\P 1639 FULLER  \textit{Holy War} ii. v. (1840) 54 At this day the Roman emperor is a very cipher, without power or profit in Rome.    
\P 1770 LANGHORNE  \textit{Plutarch} (1879) I. 252/1 The tribunes' office, which has made ciphers of the consuls.    
\P 1844 H. H. WILSON  \textit{Brit. India} I. 259 The Raja was a cypher: the Dewan usurped the whole power.    
\P 1852 THACKERAY  \textit{Esmond} i. iii. (1876) 24 To the lady and lord rather—his lordship being little more than a cypher in the house.

\itembf{b.} of things.

\P 1603 SHAKES.  \textit{Meas. for M.} ii. ii. 39 Mine were the verie Cipher of a Function To fine the faults‥And let goe by the Actor.    
\P 1844 LD. BROUGHAM  \textit{Brit. Const.} viii. (1862) 105 The impotent estate being reduced to a cipher, is as if it had no existence.

\itembf{3.} In an extended sense, applied to all the Arabian numerals; a numeral figure; a number.

\P 1530 PALSGR. 684/2, I reken, I counte by cyfers of agrym.    
\P 1640 RECORDE, etc. \textit{Gr. Artes}, Of those ten [figures] one doth signifie nothing‥and is privately called a Cypher, though all the other sometime be likewise named.    
\P 1656 BLOUNT  \textit{Glossogr.}, Cipher, a figure or number.    
\P 1756 J. WARTON  \textit{Ess. Pope} (1782) I. §31. 185 It was Gerbert, who‥is said to have introduced into France, the Arabian and Indian cypher.    
\P 1858 CARLYLE  \textit{Fredk. Gt.} (1865) VII. xviii. i. 92, I remember to have seen ‘150 millions’ loosely given as the exaggerated cipher.    
\P 1875 RENOUF'S  \textit{Egypt. Gram.} 13 Numbers are almost always expressed by means of ciphers.

\itembf{4. a.} gen. A symbolic character, a hieroglyph.

\P 1533 ELYOT  \textit{Cast. Helthe} (1541) A iv, They wolde have deuysed a strange syphre or fourme of letters, wherin they wold have writen their science.    
\P 1555 FARDLE  \textit{Facions} i. iv. 40 Yeat ware not their Letters facioned to ioyne together in sillables like ours, but Ziphres, and shapes of men and of beastes.    
\P 1614 RALEIGH  \textit{Hist. World} (J.) In succeeding times this wisdom began to be written in ciphers and characters, and letters bearing the form of creatures.

\itembf{b.} An astrological sign or figure. Obs.

\P 1590 SPENSER  \textit{F.Q.} iii. ii. 45 May learned be by cyphers, or by Magicke might.    
\P 1664 BUTLER  \textit{Hud.} ii. iii. 988 He circles draws, and squares, With ciphers, astral characters.

\P 1841-44 EMERSON  \textit{Ess. Circles} Wks. (Bohn) I. 125 The eye‥is the highest emblem in the cipher of the world.

\itembf{5. a.} A secret or disguised manner of writing, whether by characters arbitrarily invented (app. the earlier method), or by an arbitrary use of letters or characters in other than their ordinary sense, by making single words stand for sentences or phrases, or by other conventional methods intelligible only to those possessing the key; a cryptograph. Also anything written in cipher, and the key to such a system.

\P 1528 GARDINER in  \textit{Pocock Rec. Ref.} I. No. 48. 92 We think not convenient to write them, but only in cipher.    
\P 1587 FLEMING  \textit{Cont. Holinshed} III. 1371/1 Letters betweene them were alwaies written in cipher.    
\P 1605 BACON  \textit{Adv. Learn.} ii. xvi. §6 The kinds of ciphers‥are many, according to the nature or rule of the infolding, wheel-ciphers, key-ciphers, doubles, etc.    
\P 1652 EVELYN  \textit{Mem.} (1857) I. 289, I had also addresses and cyphers, to correspond with his Majesty and Ministers abroad.    
\P 1748 HARTLEY  \textit{Observ. Man} i. i. 15 We admit the Key of a Cypher to be a true one, when it explains the Cypher completely.    
\P 1812 WELLINGTON in  \textit{Gurw. Disp.} IX. 235 We have deciphered the letter you sent and it goes back to you with the key of the cipher.    
\P 1839-57 ALISON  \textit{Hist. Europe} VIII. lii. §5. 293 Intercepting some of the correspondence in cipher.    
\P 1885 GORDON in  \textit{Standard} 24 Feb., Cypher letter‥which I cannot decypher, for Colonel Stewart took the cypher with him.

\itembf{b.} ciphers: Shorthand; = character 3 b.

\P 1541 ELYOT  \textit{Image Gou.} 28 Secretaries or clerkes‥in briefe notes or syphers made for that purpose, wrate euery woorde that by those counsaillours was spoken.
\P a1670 HACKET  \textit{Abp. Williams} i. 82 (D.) His speeches were much heeded, and taken by divers in ciphers.

\itembf{c.} fig.

\P 1674 CLARENDON  \textit{Surv. Leviath.} (1676) 12 To open the cipher of other mens thoughts.    
\P 1854 B. TAYLOR  \textit{Poems Orient}, L'Envoi, I found among the children of the Sun The cipher of my nature.

\itembf{6.} An intertexture of letters, esp. the initials of a name, engraved or stamped on plate, linen, etc.; a literal device, monogram; now esp. used of Turkish or Arabic names so expressed.

\P 1631 MASSINGER  \textit{Beleeve as You List} v. ii, Pull out the stone, and under it you shall finde My name, and cipher I then usde, ingraven.
\P a1672 WOOD  \textit{Life} (1848) 87 note, Above [the portrait] is his cypher.    
\P 1764 HARMER  \textit{Observ.} xix. x. 425 The Emir's flourish or cypher at the bottom, signifying, ‘The poor, the abject Mehemet, son of Turabeye’.    
\P 1824 J. JOHNSON  \textit{Typogr.} I. 348 At the end is Caxton's cypher on a white ground.    Mod. Turkish coins bearing no device except the Sultan's cipher.

\itembf{7.} The continuous sounding of any note upon an organ, owing to the imperfect closing of the pallet or valve without any pressure upon the corresponding key.

\P 1779 BURNEY  \textit{Infant Music.} in \textit{Phil. Trans.} LXIX. 198 He weakened the springs of two keys at once, which, by preventing the valves of the wind-chest from closing, occasioned a double cipher.    
\P 1884 W. S. ROCKSTRO  \textit{Mendelssohn} xii. 82 During the course of the Fantasia‥a long treble A began to sound on the swell‥We well remember whispering to Mr. Vincent Novello‥‘It must be a cypher’.

\itembf{8.} attrib. and in Comb., as cipher bishop (sense 2); cipher-letter, cipher-telegram, cipher-writing, etc. (sense 5); cipher-key, the key to writings in cipher; cipher officer, an officer in the military or diplomatic services responsible for the coding and decoding of ciphers; †cipher-tunnel, a false or mock chimney.

\P 1649 MILTON  \textit{Eikon.} Wks. (1738) I. 377 That foolish and self-undoing Declaration of twelve *Cypher Bishops.

\P 1872 TENNYSON  \textit{Gareth \& Lynette} 64 A red And *cipher face of rounded foolishness.

\P 1915 O. WILLIAMS  \textit{Let.} 23 Mar. in C. Mackenzie Gallipoli Mem. (1929) ii. 7 I'm *cipher officer on his Staff with the rank of Captain.    
\P 1948 \textit{Hansard} CDXLVIII. 1539 The  smooth running of an embassy abroad depends just as much on a happy, contented, well-paid staff of cipher officers‥as‥on the‥head of the Mission.

\P 1831 CARLYLE  \textit{Sart. Res.} (1858) 20 Laughter: the *cipher key, wherewith we decipher the whole man!

\P 1880  \textit{Brit. Post. Guide} 242 *Cypher telegrams are those containing series or groups of figures or letters having a secret meaning; or words not to be found in a standard dictionary.

\P 1655 FULLER  \textit{Ch. Hist.} v. iii. §46 The device of *Cypher Tunnels or mock-Chimneys meerly for uniformity of building.
\end{myenumerate}


%%%%%%%%%%%%%%%%%%%%%%%%%%%%%%%%%
\myitem{circuitous} a.

\noindent \phonetic{(səˈkjuːɪtəs)}

\noindent [ad. late L. circuitōs-us abounding in roundabout courses, f. circuitus circuit n.: see -ous.]

\noindent
Of the nature of a circuit, roundabout, indirect.

\P 1664 H. MOORE  \textit{Myst. Iniq.} 109 Any medium direct or circuitous.    
\P 1790 PALEY  \textit{Horæ Paul.} i. 4 Coincidences‥minute, circuitous, or oblique.    
\P 1796 MORSE  \textit{Amer. Geog.} I. 439 By this kind of circuitous commerce they subsisted and grew rich.    
\P 1800 COLQUHOUN  \textit{Comm. Thames} xi. 303 This ancient Court of Record is too circuitous in its procedure.    
\P 1845 WHATELY  \textit{Logic} in \textit{Encycl. Metr.} 219/1 An artificial and circuitous way of speaking.    
\P 1868 QUEEN VICTORIA  \textit{Life Highl.} 169 We had‥to take a somewhat circuitous route in order to avoid some bogs.

\itembf{2.} ? Circus-like. Obs. rare.

\P 1807 G. CHALMERS  \textit{Caledonia} I. i. ii. 92 There are other circuitous erections of stone.




%%%%%%%%%%%%%%%%%%%%%%%%%%%%%%%%%
\myitem{circumlocution} n.

\noindent \phonetic{(ˌsɜːkəmləʊˈkjuːʃən)}

\noindent [a. F. circonlocution, or ad. L. circumlocūtiōn-em, f. circum- + loqui to speak.]
\vspace{-0.3cm}

\begin{myenumerate}

\itembf{1.} Speaking in a roundabout or indirect way; the use of several words instead of one, or many instead of few. Formerly used of grammatical periphrasis; but now only of rhetorical.

\noindent
   Circumlocution Office: a satirical name applied, by Dickens, to Government Offices, on account of the circuitous formality by which they delay the giving of information, etc.

\P 1510 BARCLAY  \textit{Mirr. Good Mann.} (1570) F vj, When thou must in speche touche‥Such maners vnclenly, vse circumlocution.    
\P 1530 PALSGR. 112 Where we use circumlocution, the frenchemen have one onely worde.    
\P 1553 T. WILSON  \textit{Rhet.} 93 b, Circumlocution is a large description either to sette forth a thyng more gorgeouslie, or else to hyde it.    
\P 1595 A. DAY  \textit{Eng. Secretary} ii. (1625) 84 When by circumloquution anything is expressed, as when we say: The Prince of Peripateticks, for Aristotle.    
\P 1626 COCKERAM,  \textit{Circumlocution}, A speaking of many words when few may suffice: a long circumstance.    
\P 1713 ADDISON  \textit{Ct. Tariff}, He affirms everything roundly without any art or circumlocution.    
\P 1823 SCOTT  \textit{Peveril} xii, After much circumlocution, and many efforts to give an air of importance to what he had to communicate.    
\P 1855 DICKENS  \textit{Dorrit} i. x, The Circumlocution Office was (as everybody knows without being told) the most important Department under Government.    ibid., Whatever was required to be done, the Circumlocution Office was beforehand with all the public departments in the art of perceiving—How not to do it.

\itembf{b.} A phrase or sentence in which circumlocution is used; a roundabout expression.

\P 1533 TINDALE  \textit{Supper of Lord} 42 Going about the bush with this exposition and circumlocution.    
\P 1662 FULLER  \textit{Worthies} (1840) II. 452 In his pleadings‥he declined all circumlocutions.    
\P 1791 MACKINTOSH  \textit{Vind. Gall.} Wks. 1846 III.  83 The courtly circumlocution by which Mr. Burke designates the Bastille—‘the King's castle at Paris!’    
\P 1854 KINGSLEY  \textit{Lett.} (1878) I. 417 Courtesies and Circumlocutions are out of place, where the morals, health, lives of thousands are at stake.

\vspace{0.1cm} \noindent
So \textbf{circumlocutional, circumlocutionary}, adjs., pertaining to, or given to, circumlocution. circumlocutionist, one who employs circumlocution. circumlocutious a., given to circumlocution; whence circumlocutiousness.

\P 1865 DICKENS  \textit{Mut. Fr.} II. 308, I have found circumlocutional champions disposed to be warm with me.    
\P 1863 \textit{Scotsman}  16 Apr., An immense exercise of circumlocutionary skill.    
\P 1877 WALLACE  \textit{Russia} xxx. 500 The flowery circumlocutionary style of an Oriental scribe.    
\P 1846 WORCESTER  \textit{Circumlocutionist}, citing \textit{Gent. Mag.}    
\P 1855 DICKENS  \textit{Dorrit} i. xxxiv, This able circumlocutionist.    
\P 1827 R. HILL in Sidney \textit{Life} (1834) 213 O the dulness, the circumlocutiousness, the conceit, the tautology.
\end{myenumerate}


%%%%%%%%%%%%%%%%%%%%%%%%%%%%%%%%%
\myitem{circumspect} a.

\noindent \phonetic{(ˈsɜːkəmspɛkt)}

\noindent [a. F. circonspect, or ad. L. circumspect-us considerate, wary, cautious, circumspect, properly pa. pple. of circumspicĕre to look around, take heed, consider; hence of things, ‘well-considered’, transf. to persons ‘considerate, cautious’, etc.]
\vspace{-0.3cm}

\begin{myenumerate}

\itembf{1.} Of things or actions: Marked by circumspection, showing caution, well-considered, cautious.

\P 1422 LYDG.  \textit{Coronation Hen. VI}, in Ritson Anc. Songs 70 By circumspect advise.    
\P 1562 \textit{Act  5 Eliz.} c. 21 §1 If circumspect Remedy be not hereunto provided.    
\P 1709 STRYPE  \textit{Ann. Ref.} Ep. Ded. 1 Circumspect and holy labours.    
\P 1847 EMERSON  \textit{Poems}, Monadnoc Wks. (Bohn) I. 441 By circumspect ambition.

\itembf{2.} Of persons: Watchful on all sides, attentive to everything, cautious, heedful of all circumstances that may affect action or decision.

\P 1430 LYDG.  \textit{Chron. Troy} ii. xvi, Circumspect in all his gouernance.    
\P 1494 FABYAN vii. ccxlvi. 290 Which in all his faytes is so circumspecte.    
\P 1542 BOORDE  \textit{Dyetary} xxiii. (1870) 287 Sanguyne men‥must be cyrcumspect in eatynge of theyr meate.    
\P 1594 SHAKES.  \textit{Rich. III}, iv. ii. 31 High-reaching Buckingham growes circumspect.    
\P 1624 CAPT. SMITH  \textit{Virginia} iv. 147 This will make us more circumspect.    
\P 1728 NEWTON  \textit{Chronol. Amended} ii. 260 Herodotus was circumspect and faithful in his narrations.    
\P 1850 PRESCOTT  \textit{Peru} II. 31 The wild passes‥practicable‥for the sure and circumspect mule.    
\P 1881 BESANT  \& RICE \textit{Chapl. of Fleet} i. 38, I was to be circumspect in my behaviour.

\itembf{b.} with dependent sentence or clause. Obs.

\P 1573 G. HARVEY  \textit{Letter-bk.} (1884) 2 As circumspect to se to mi self.    
\P 1658 W. BURTON  \textit{Itin. Anton.} 172, I have‥been very scrupulous and circumspect what authorities I made use of.

\itembf{3.} Considered, respected. [late L. circumspectus.] Obs. rare.

\P 1579 TWYNE  \textit{Phisicke agst. Fortune} ii. xxxii. 209 a, Then wylt thou be the more circumspect, and the better knowne.
\end{myenumerate}


%%%%%%%%%%%%%%%%%%%%%%%%%%%%%%%%%
\myitem{clamorous} a.

\noindent \phonetic{(ˈklæmərəs)}

\noindent [Corresponds to med.L. clāmōrōs-us, and obs. F. clamoreux, f. L. clāmōrem clamour: see -ous.]

\noindent Characterized by clamour.
\vspace{-0.3cm}

\begin{myenumerate}
\itembf{1.} Of the nature of clamour; uttered with, or accompanied by, clamour or shouting; noisy.

\P 1526 PILGR.  \textit{Perf.} (W. de W. 1531) 92 b, Defendeth with hygh and clamorous wordes or speche his opinyon.    
\P 1596 SHAKES.  \textit{Tam. Shr.} iii. ii. 180 Hee‥kist her lips with such a clamorous smacke, that at the parting all the Church did eccho.    
\P 1667 MILTON  \textit{P.L.} x. 479 Chaos wilde‥fiercely oppos'd My journey strange, with clamorous uproare.    
\P 1712 ADDISON  \textit{Spect.} No. 440 \cardo{⁋}6 He still reasoned in a more clamorous and confused manner.    
\P 1828 D'ISRAELI  \textit{Chas.} I, II. i. 23 Loud and clamorous was the babble against the new soap.    
\P 1842 EMERSON  \textit{Transcendentalist} Wks. (Bohn) II. 291 They‥reject the clamorous nonsense of the hour.

\itembf{2.} Uttering loud and persistent cries or shouts; noisy, vociferous; loudly urgent. Said of persons and other agents, or instruments; and transf. of places where these are.

\P 1540-54 CROKE  \textit{Ps.} (1844) 19 Mercifull Lorde‥let ascende vp to thyne eare My wofull voyce, and clamorous.    
\P 1600 SHAKES.  \textit{A.Y.L.} iv. i. 152, I will bee‥more clamorous then a Parrat against raine.    
\P 1728 POPE  \textit{Dunc.} ii. 353 The clam'rous crowd is hush'd with mugs of Mum.    
\P 1810 SCOTT  \textit{Lady of L.} iii. i, Clamorous War-pipes yelled the gathering sound.    
\P 1858 W. JOHNSON  \textit{Ionica} 27 The zeal of those that miss the prize On clamorous river-banks.    
\P 1870 BRYANT  \textit{Iliad} I. ii. 45 Thersites only, clamorous of tongue, Kept brawling.

\itembf{3.} fig. That urgently claims attention, ‘crying’; importunate. (Often including actual noise.)

\P 1621-31 LAUD  \textit{Sev. Serm.} (1847) 98, I doubt our sins have been as clamorous upon God to heat His fire.    
\P 1691 T. H[ALE]  \textit{Acc. New Invent.} 44 Put an end to this clamorous Evil.    
\P 1712 ARBUTHNOT  \textit{John Bull} (1755) 13 Clamorous debts.    
\P 1836 J. GILBERT  \textit{Chr. Atonem.} i. (1852) 5 The age‥we may almost say, is clamorous for new works.
\end{myenumerate}


%%%%%%%%%%%%%%%%%%%%%%%%%%%%%%%%%
\myitem{clandestine} a. (n.)

\noindent \phonetic{(klænˈdɛstɪn)}

\noindent [ad. L. clandestīnus secret, hidden, clandestine, f. clam secretly, in private; cf. matutīnus. In French clandestin, -ine occurs in 16th c.]
\vspace{-0.3cm}

\begin{myenumerate}

\itembf{A.} adj. Secret, private, concealed; usually in bad sense, implying craft or deception; underhand, surreptitious.

\P 1566 LETHINGTON  \textit{To Cecil} in Burnet Records iii. No. 30 (R.) The vitiated and clandestine contract‥having no witness nor solemnization of Christian matrimony.    
\P 1658 MILTON  \textit{Lett. State} (1851) 400 A certain clandestine Hostility cover'd over with the name of Peace.    
\P 1698 W. CHILCOT  \textit{Evil Thoughts} ii. (1851) 18 The clandestine impurities of the hearts and souls of the whole world shall be revealed.    
\P 1754 ERSKINE  \textit{Princ. Sc. Law} (1809) 69 When the order of the church is observed, the marriage is called regular; when otherwise, clandestine. Clandestine marriage, though it be valid, has statutory penalties annexed to it.    
\P 1845 MCCULLOCH  \textit{Taxation} ii. x. (1852) 359 A powerful stimulus to clandestine distillation.    
\P 1860 W. COLLINS  \textit{Wom. White} iii. 472, I obtained access by clandestine means.

\itembf{B.} n. A clandestine or underhand proceeding.

\P 1656 S. H. GOLDEN  \textit{Law} 15 Such clandestines and ambushments attend continually for your surprisal.    Ibid. 87 Your Clandestines and Trecheries.
\end{myenumerate}


%%%%%%%%%%%%%%%%%%%%%%%%%%%%%%%%%
\myitem{claptrap} n.

\noindent \phonetic{(ˈklæptræp)}

\noindent [f. clap n.1 4 + trap n.]

\vspace{-0.3cm}

\begin{myenumerate}

\itembf{1.} (with pl.) A trick or device to catch applause; an expression designed to elicit applause.

\P 1727-31 BAILEY  II, A Clap Trap‥a trap to catch a clap by way of applause from the spectators at a play.    
\P 1788 DIBDIN  \textit{Musical Tour} lxiii. 161 Sentiments which, by the theatrical people, are known by the name of clap traps.    
\P 1799 SOUTHEY  \textit{Lett.} (1856) I. 67 There will be no clap-traps, nothing about ‘Britannia rule the Waves’.    
\P 1848 THACKERAY  \textit{Bk. Snobs} xx, Don't‥vent claptraps about your own virtue.

\itembf{2.} (without a or pl.) Language designed to catch applause; cheap showy sentiment. In modern use passing into sense ‘nonsense, rubbish’.

\P 1819 BYRON  \textit{Juan} ii. cxxiv, I hate‥that air Of clap-trap, which your recent poets prize.    
\P 1845 \textit{Punch}  Nov. 215/1 Dan‥fancies he covers his own astounding selfishness and indifference by this brutal clap$\sim$trap.    
\P 1880 DISRAELI  \textit{Endym.} lvii. 253 He disdained all cant and clap-trap.    
\P 1895  \textit{Daily News} 30 May 2/3 That is very eloquent but it is what I call vicious and wicked clap trap.    
\P 1915 A. HUXLEY  \textit{Let.} Nov. (1969) 86 How much better this book wd. have been had she made it a study of don-life in the 80's‥instead of the usual politico-Debrett clap-trap.    
\P 1955  \textit{Times} 26 Aug. 7/5 Cannot our educationists turn away from the pretentious claptrap put about during the past 20 years‥?    
\P 1966 \textit{Illustr.  London News} 30 July 28/2 The piece at one point turns to deplorable dramatic claptrap.

\itembf{3.} A mechanical contrivance for making a clapping noise to express applause, etc. Obs.

\P 1847 CRAIG,  Clap-trap‥a kind of clapper for making a noise in theatres.    
\P 1864 WEBSTER,  \textit{Clap-trap}, a contrivance for clapping in theaters.    
\P 1866 \textit{Cincinnati  Gaz.} in \textit{Public Opinion} 24 Feb., A street juggler‥sings some ditty to the sound of clap-traps which he swings or works in his hand.

\itembf{4.} attrib. (in senses 1, 2), passing into true adjectival use; = claptrappy.

\P 1815 \textit{Scribbleomania}  124 note, The Clap-Trap system which he has uniformly adopted during‥his theatrical career.    
\P 1842 G. S. FABER  \textit{Provinc. Lett.} (1844) II. 187 They triumphantly draw the clap-trap conclusion, that, etc.    
\P 1855 G. BRIMLEY  \textit{Ess. Tennyson} 74 Claptrap appeals to the war-feeling of the day.    
\P 1875 JOWETT  \textit{Plato} (ed. 2) II. 371 A regular clap-trap speaker.    
\P 1887 \textit{Spectator}  7 May 622/1 The subject is more or less clap-trap.

\vspace{0.1cm} \noindent
Hence claptrappery, claptrappish a., claptrappy a., -ily adv.; all nonce-wds.

\P 1820 COLERIDGE  \textit{Lett.} I. xi. 118 Her plebicolar Clap-Trapperies.    
\P 1880 \textit{Punch}  27 Dec. 306/2 Till ‘Goodwill’ sound verily, Cheerily, not claptrappily.    
\P 1809 SOUTHEY in  C. Southey \textit{Life} III. 205 Did I not tell you it [a passage in Kehama] was clap-trappish?    
\P 1865  \textit{Reader} 2 Dec. 636/2 The language being either claptrappish or vapid.    
\P 1873 \textit{Spectator}  4 Oct., Mr. Chamberlain's clap-trappy programme of a Free Church, a Free School, Free Labour, and Free Land.
\end{myenumerate}


%%%%%%%%%%%%%%%%%%%%%%%%%%%%%%%%%
\myitem{cleave} v.1

\noindent \phonetic{(kliːv)}

\noindent \phonetic{[Common Teut.: OE. clíofan, cléofan, pa. tense cléaf, pl. clufon, pa. pple. clofen, corresp. to OS. clioƀan (MDu. clieven, clûven, Du. klieven), OHG. chlioban (MHG., mod.G. klieben), ON. kljúfa (Sw. klyfva, Da. klöve), not recorded in Gothic:—OTeut. type *kleuƀ-, klauƀ—kluƀum, kluƀano-, corresp. to pre-Teutonic *gleubh-, in Gr. γλυϕ- ‘to cut with a knife, carve’, and perh. L. glūb- ‘to peel, flay’.
}

\noindent
The early ME. inflexion was cleoven (clēven), clêf (pl. cluven), cloven. Assimilation to the pa. pple. soon changed the plural of the pa. tense to cloven, clove, and by 14th c. clove was extended to the singular, where clêf, clêve, became obs. about 1500, making the later inflexion clēve, clōve, clōven. The pa. pple. had also the shortened form clove, which survives as a variant in poetry. A pa. tense clave occurs in northern writers in 14th c., passed into general use, and was very common down to c 1600; it survives as a Bible archaism. A weak inflexion cleaved came into use in 14th c.; and subsequently a form cleft; both are still used, cleft esp. in pa. pple., where it interchanges with cloven, with some differentiation in particular connexions, as ‘cleft stick’, ‘cloven foot’: see these words.

\noindent
From the 14th c. the inflexional forms of this verb have tended to run together with those of cleave v.2 ‘to stick’. Though the latter was originally clive, it had also the variants cleove, clēve, the latter of which at length prevailed; the two verbs having thus become indentical in the present stem were naturally confused in their other inflexions. The (originally northern) pa. tense clave, which appeared in both in 14th. c., is not normal in either; it was apparently analogical, taken over from one of the other classes of strong vbs. having a in the past, as from breke, brak(e, broken, speke, spak(e, spoken. (It would of course be possible to explain the pa. tense singular clove in the same way.) The weak pa. tense and pa. pple. cleaved were probably mainly taken over from cleave v.2, where they were original; but they might also arise independently in this verb. For the subsequent shortening of cleaved to cleft, there was the obvious precedent of leave, left, bereave, bereft, etc.]
\vspace{-0.3cm}

\begin{myenumerate}

%\itembf{A.} Forms.

%\itembf{1.} pres. stem. (α) 1 cléofan, 2-4 cleove-n, 2-7 cleve, cleue, (4-5 clefe, clewe,) 5-6 cleeue.

%\P 1000 GLOSS.  \textit{Prudent. (Record) 150 (Bosw.) Cleofan, scindere.    c 
%\P 1200 TRIN.  \textit{Coll. Hom. 61 He wile smite‥mid egge and cleuen.    c 
%\P 1300 K. ALIS.  \textit{7702 Many an hed wolde Y cleove.    
%\P 1340 HAMPOLE  \textit{Pr. Consc. 6736 Þair hertes sal nere clewe [v.r. clefe].    
%\P 1483 CATH.  \textit{Angl. 67 To cleve, scindere.    
%\P 1578 LYTE  \textit{Dodoens vi. lxxxii. 762 Which will soone riue, or cleeue asunder.    
%\P 1727 BRADLEY  \textit{Fam. Dict. s.v. Hoof hurt, The horn doth crack and cleve.

%(β) 6- cleave, (cleaue).

%\P 1530 PALSG.  \textit{486/2, I cleave a sonder.    
%\P 1558 WARDE  \textit{tr. Alexis' Secr. 112 b, It cleaveth soonest by the fyre.    
%\P 1610 W. FOLKINGHAM  \textit{Art Surv. i. viii. 17 That Earth, that‥doth clift and cleaue.    
%\P 1697 DRYDEN  \textit{Virg., Pref. to Past., The homely Employment of cleaving Blocks.

%(γ) 5 clyu-yn, 5-6 clyue (-ve), 6 clyffe, 6-7 cliue (-ve).

%\P 1440  \textit{Promp. Parv.} 82 Clyvyn or Parte a-sundyr a[s] men doone woode, findo.    
%\P 1558 PHAëR  \textit{Æneid v. (1573) N iiij, Along by heauen his arow driues‥therwith the skies he cliues.    
%\P 1570 LEVINS  \textit{Manip. 117 To clyffe, scindere.    
%\P 1575 TURBERV.  \textit{Venerie 135 Clyve the sides one from another.    
%\P 1651 RALEIGH'S  \textit{Appar. 90 To cliue and pierce the air.    
%\P 1686 WILDING in  \textit{Hearne Collect. (Oxf. Hist. Soc.) I. 264 For Wood \& cliving it‥4s. 10d.

%\itembf{2.} pa. tense. (α) sing. 1 cléaf, 2-3 clæf, 3 clef, 4-5 cleef, clefe, 5 cleve; rare pl. 4 clef.

%\P 1205 LAY. 2
%\P 1390 ENNE  \textit{cniht atwa [he] clæf.    a 
%\P 1300 FALL  \textit{\& Pass. in E.E.P. 70 (1862) 14 Hi‥clef is swet hert atwo.    c 
%\P 1330 R. BRUNNE  \textit{Chron. (1810) 326 Þe walle þorghout þei clef.    c 
%\P 1400 MANDEVILLE  \textit{viii. 86 The Roche cleef in two.    c 
%\P 1400  \textit{Destr. Troy} 7318 He clefe hym to þe coler.    c 
%\P 1440 GENERYDES  \textit{3035 He cleue a ij his hede.

%(β) pl. 1 clufon, 3 cluuen (= -ven), 4 clowen (= -ven), 4- cloue, clove; sing. 4-7 cloue, (4 clowe, 5 clofe, 6 cloaue), 4- clove (kləʊv).

%\P 1205 LAY.
%\P 1920 HIS  \textit{ban to-cluuen.    a 
%\P 1300  \textit{Cursor M.} 7810 (Gött.) His herte in tua i wat i clowe [v.r. claif, claue, cleef].    c 
%\P 1300 K. ALIS.  \textit{2765 This Thebes seyghen how men heom clowen.    c 
%\P 1430 SYR  \textit{Gener. (Roxb.) 5169 Clofe the heid in twoo.    
%\P 1490 CAXTON  \textit{Eneydos li. 144 Eneas‥cloue hym vnto the teeth.    
%\P 1535 COVERDALE  \textit{Ps. lxxviii. 15 He cloaue the hard rockes.    
%\P 1605 SHAKES.  \textit{Lear i. iv. 175 When thou clouest thy Crownes i'th' middle.    
%\P 1702 ROWE  \textit{Tamerl. v. i. 2203, I clove the Villain down.    
%\P 1883 LONGM.  \textit{Mag. July 286 Into‥the crowd porters clove their way with shouts.

%(γ) 4-7 claue, (4-6 claif(f, 5 claf, clafe, claffe), 6- clave (kleɪv).

%\P 1300  \textit{Cursor M.} 6262 (Cott.) It claue [v.r. clef, cleef, cleue], and gaue þam redi gat.    
%\P 1375 BARBOUR  \textit{Bruce v. 633 He the hede till the harnyss claiff [v.r. clafe].    c 
%\P 1430 SYR  \textit{Gener. (Roxb.) 4752 He clafe his shelde in twoo.    
%\P 1485 CAXTON  \textit{Chas. Gt. (1880) 169 Hyt claffe a marble stone.    
%\P 1470-85 MALORY  \textit{Arthur xviii. i. (1889) 689 He claf his helme.    c 
%\P 1530 LD. BERNERS  \textit{Arth. Lyt. Bryt. (1814) 305 He claue him to the sholders.    
%\P 1535 STEWART  \textit{Cron. Scot. II. 599 Richt to the schulderis doun he claif his heid.    
%\P 1563  \textit{Homilies} ii. Death \& Pass. Christ ii. (1859) 422 The stones clave a sunder.    
%\P 1611 BIBLE  \textit{Ps. lxxviii. 15 Hee claue the rockes in the wildernes.    
%\P 1755 WESLEY  \textit{Wks. (1872) II. 331 The earth also clave asunder.    
%\P 1885 BIBLE  \textit{(Rev.) Ps. lxxviii. 13 He clave the sea.

%(δ) 4-5 cleued(e, cleved(e, (5 -wed, -vyd, cleufit), 8-9 cleaved (kliːvd).

%\P 1300 K. ALIS.  \textit{2340 A fayr baroun, He cleved to the breste adoun.    Ibid. 3790 He clewyd his scheld.    c 
%\P 1400  \textit{Destr. Troy} 4034 fflodys‥into caues‥cleufit the erthe.    a 
%\P 1440 SIR EGLAM.  \textit{746 He clevyd hym by the rugge-bone.    
%\P 1762 FALCONER  \textit{Shipw. i. (R.) She cleav'd the wat'ry plain.    
%\P 1853 KANE  \textit{Grinnell Exp. xlvii. (1856) 444 As they cleaved the misty atmosphere.

%(ε) 6- cleft (klɛft).

%\P 1500 CHESTER  \textit{Pl. (1847) ii. 70 The roccke that never before clyfte Clave that men mighte knowe.    
%\P 1590 SPENSER  \textit{F.Q. i. ii. 19 It‥cleft his head.    
%\P 1671 MILTON  \textit{P.R. iii. 438 As the Red Sea and Jordan once he cleft.    
%\P 1752 YOUNG  \textit{Brothers i. i, I cleft yon Alpine rocks.    a 
%\P 1839 PRAED  \textit{Poems (1864) II. 406 She cleft it with her lover's brand.

%\itembf{3.} pa. pple. (α) 1-2 clofen, 2-7 clouen, 2- cloven (ˈkləʊv(ə)n).

%\P 1330 R. BRUNNE  \textit{Chron. Wace (Rolls) 4420 Lite failled þat he ne had Clouen þe hed.    c 
%\P 1420 LIBER  \textit{Cocorum (1862) 18 When þou hase‥Clovyn hom.    
%\P 1577 B. GOOGE  \textit{Heresbach's Husb. ii. (1586) 55 The stalke being tenderly cloven.    
%\P 1761 HUME  \textit{Hist. Eng. I. viii. 182 Having cloven his head with many blows.    
%\P 1860 TYNDALL  \textit{Glac. i. §1. 1 Cloven into thin plates.

%(β) 4-5 clove, yclove, 8- poet. clove (kləʊv).

%\P 1297 R. GLOUC.  \textit{(1724) 49 To haue y cloue hym al þat hed.    c 
%\P 1385 CHAUCER  \textit{L.G.W. 738 Tisbe, This wal‥Was cloue a two.    c 
%\P 1420 CHRON. Vilod.
%\P 1033 ÞE  \textit{armes‥from hurr' body y clove so was.    
%\P 1719 YOUNG  \textit{Revenge v. ii, Till I had clove thy crest.    c 
%\P 1800 K. WHITE  \textit{Gondol. lxiv, His head, half clove in two.

%(γ) 4-5 cleued, 7- cleaved (kliːvd). (Always used in Min. and Geol.)

%\P 1400 ST. ALEXIUS  \textit{(Laud 622) Whan þe whal was to⁓cleued.    
%\P 1432-50 tr. Higden (Rolls) I. 353 A rodde, cleuede in the hier parte of it.    
%\P 1694 NARBOROUGH  \textit{Acc. Sev. Late Voy. i. (1711) 51 Cleaved in his Quarters.    
%\P 1818 W. PHILLIPS  \textit{Outl. Min. \& Geol. (ed. 3) 16 The topaz can only be readily cleaved in one direction.    
%\P 1830 A. FONBLANQUE  \textit{Eng. under Seven Administr. (1837) II. 35 A force that would have cleaved an elephant in twain.

%(δ) 5- cleft.

%\P 1382 WYCLIF  \textit{Matt. xxvii. 51 Stoonys ben cleft [v.r. clouen,
%\P 1388 WEREN cloue].    
%\P 1398 TREVISA  \textit{Barth. De P.R. v. lx. (1495) 176 The synewe whyche is slytte and clouen‥for yf a veyne be slytte and clefte.    
%\P 1530 PALSG.  \textit{486/2 As wodde is clefte.    
%\P 1591 SHAKES.  \textit{Two Gent. v. iv. 103 How oft hast thou with periury cleft the roote?    
%\P 1755 WESLEY  \textit{Wks. (1872) II. 331 One part of the solid stone is cleft from the rest.    a 
%\P 1839 PRAED  \textit{Poems (1864) I. 352 His steel cap cleft in twain.

%¶cloved, pa. tense and pple.: ? error for cleved.

%\P 1200 AS HE WAS  \textit{clofy-d, styll he stod.    c 
%\P 1489 CAXTON  \textit{Sonnes of Aymon ii. 61 He cloued hym to the teeth.

%\itembf{B.} Signification.

\itembf{1.} trans. To part or divide by a cutting blow; to hew asunder; to split. Properly used of parting wood, or the like, ‘along the grain’, i.e. between its parallel fibres; hence, of dividing anything in the direction of its length, height, or depth; also, of dividing slate or crystals along their cleavage planes, and other things at their joints.

\P 1100 \textit{Gerefa} in  \textit{Anglia} IX. 261 In miclum \phonetic{ᴁefyrstum} timber cleofan.
\P c1300  \textit{Havelok} 917 Ful wel kan ich cleuen shides.    
\P 1481 CAXTON  \textit{Reynard} viii. (Arb.) 14 A grete oke whiche he had begonne to cleue.    
\P 1599 SHAKES.  \textit{Much Ado} ii. i. 261 She would haue made Hercules‥haue cleft his club to make the fire.    
\P 1611 BIBLE  \textit{Gen.} xxii. 3 Abraham‥claue the wood for the burnt offering.    
\P 1697 DRYDEN  \textit{Virg. Georg.} ii. 484 The Dog-star cleaves the thirsty Ground.    
\P 1705 OTWAY  \textit{Orphan} ii. iii. 516, I‥clove the Rebel to the Chine.    
\P 1722 SEWEL  \textit{Hist. Quakers} (1795) I. iii. 205 A butcher swore he would cleave her head.    
\P 1823 H. J. BROOKE  \textit{Introd. Crystallogr.} 44 If a cube of blende‥be cleaved in directions parallel to its diagonal planes.    
\P 1872 E. PEACOCK  \textit{Mabel Heron} iv. 55 The sections into which our society is cleft.

\itembf{b.} Often with asunder, in two, etc. to cleave down: to cut down.

\P c1205, etc. [See A 2 α].
\P 1300 \textit{K. Alis.}  2231 A-two [he] cleued his scheld.
\P c1320 \textit{Sir Beues}  4514 Man and hors he cleuede doun.
\P 1490 \textit{Adam  Bel \& Clym} C. 601 Cloudesly‥Claue the wand in to.    
\P 1590 SPENSER  \textit{F.Q.} i. xi. 39 The knotty sting Of his huge taile he quite in sunder cleft.    
\P 1603 SHAKES.  \textit{Meas. for M.} iii. i. 63 To cleaue a heart in twaine.    
\P 1796 MORSE  \textit{Amer. Geog.} I. 610 The mountain being cloven asunder.    
\P 1855 MACAULAY  \textit{Hist. Eng.} III. 361 He was cloven down while struggling in the press.

\itembf{c.} To pierce and penetrate (air, water, etc.). Also to cleave one's way through.

\P 1558 and 1651 [See  A 1 γ].    
\P 1671 MILTON  \textit{P.R.} iii. 433 At their passing cleave the Assyrian flood.    
\P 1704 POPE  \textit{Windsor For.} 188 The fierce eagle cleaves the liquid sky.    
\P 1791 COWPER  \textit{Iliad} ix. 447 Cleaving with my prows The waves of Hellespont.    
\P 1827 CARLYLE  \textit{Richter} Misc., Whose wailings have cleft the general ear.    
\P 1852 CONYBEARE  \& H. \textit{St. Paul} (1862) I. ix. 263 The vessel‥would soon cleave her way through the strait.    
\P 1877 L. MORRIS  \textit{Epic Hades} ii. 175 No sunbeam cleaves the twilight.

\itembf{d.} To intersect, penetrate, or fissure, in position.

\P 1808 J. BARLOW  \textit{Columb.} i. 247 Thine is the stream; it cleaves the well known coast.    
\P 1874 H. REYNOLDS  \textit{John Bapt.} iv. 232 Caverns which still cleave the limestone rocks.

\itembf{e.} Phrases. to cleave a hair: cf. ‘to split hairs’. to cleave the pin: (in archery) to hit the pin in the centre of the white of the butts (see pin); hence fig.

\P 1586 MARLOWE  \textit{1st Pt. Tamburl.} ii. iv, For kings are clouts that every man shoots at, Our crown the pin that thousands seek to cleave.    
\P 1592 SHAKES.,  \textit{Rom. \& Jul.} ii. iv. 15 The very pinne of his heart cleft with the blind Bowe-boyes but-shaft.
\P a1626 MIDDLETON  \textit{No Wit like Woman's} (N.), I'll cleave the black pin i' the midst of the white.    
\P 1655 FULLER  \textit{Ch. Hist.} iii. vi. §31 To cleave an hair betwixt the spiritual and temporal jurisdiction.    Ibid. ix. iii. §14 Mr. Fox came not up in all particulars to cleave the pin of Conformity (as refusing to subscribe) yet, etc.

\itembf{2.} To separate or sever by dividing or splitting.

\P 1300  \textit{Cursor M.} 27743 (Cott.) Man[s] aun wiit it fra him cleuisse.
\P c1330 R. BRUNNE  \textit{Chron.} (1810) 320 Fro þe body his heued‥did he cleue.    
\P 1420 [See A 3 β].    
\P 1575 [See A 1 γ].    
\P 1755 [See A 3 δ].    
\P 1857-8 SEARS \textit{Athan.} ix. 74 To cleave away our effete coverings.    
\P 1873 MRS. CHARLES in \textit{Sunday Mag.} June 625 The dreadful chasm cleaving us into separate existence was gone.

\itembf{3.} intr. (for refl.). To split or fall asunder.

\P 1225 \textit{Leg. Kath.} 2027 Hit bigon to claterin al \& to cleouen.
\P a1300  \textit{Cursor M.} 6251 Þou sal see it cleue in tua.    Ibid. 24419 Þe stanes claf.    
\P 1377 LANGL.  \textit{P. Pl.} B. xviii. 61 Þe wal wagged and clef.
\P c1440  \textit{Promp. Parv.} 82 Clyue, or ryue by the selfe, rimo, risco.    
\P 1575 TURBERV.  \textit{Falconrie} 302 The beake beginneth to ryve and clive from hir head.    
\P 1611 BIBLE  \textit{Numb.} xvi. 31 The ground claue asunder.
\P a1641 BP. R. MONTAGU  \textit{Acts \& Mon.} 220 The vaile of the Temple shall cleave in twaine.    
\P 1704 NEWTON  \textit{Opticks} (J.), It cleaves with a glossy polite substance.    
\P 1841 LANE  \textit{Arab. Nts.} I. 99 He struck the earth with his feet, and it clove asunder, and swallowed him.

\itembf{4.} intr. To cleave one's way, penetrate, pass.

\P 1655 \textit{Francion}  x. 18 Cleaving through the Presse, he did approach unto him, etc.    
\P 1805 WORDSW.  \textit{Prelude} iii. (1850) 68 Through the inferior throng I clove Of the plain Burghers.    
\P 1833 MARRYAT  \textit{P. Simple} xxix, As our swift frigate cleaved through the water.    
\P 1865 SWINBURNE  \textit{Poems \& Ball., Lament.} 74, I have cleft through the sea-straits narrow.
\end{myenumerate}


%%%%%%%%%%%%%%%%%%%%%%%%%%%%%%%%%
\myitem{cleave} v.2

\noindent \phonetic{(kliːv)}

\noindent 
\phonetic{
[OE. had two verbs; clífan str. (*cláf, pl. clifon, clifen), and clifian, cleofian weak (clifode, -od). (1) The former was a Com. Teut. strong vb., in OS. biklîƀan to adhere (MDu. clîƀan to cling, climb, Du. beklijven to adhere, stick), OHG. chlîban (MHG. rare, klîban) to adhere, stick, ON. klîfa to clamber, climb by clinging:—OTeut. *klîƀ-an, perhaps ultimately f. simpler root kli- to stick: cf. climb, clay, clam. Of this strong vb. OE. shows only a few examples of the present, its place being generally taken by (2) the derivative clifian, corresp. to OS. cliƀon (MDu. clēven, Du. kleven), OHG. chlebên (MHG. and G. kleben):—OTeut. *kliƀôjan, f. weak stem kliƀ- of the strong vb. This had in OE. the variants cliofian, cleofian (with o or u fracture of i; cf. lifian, leofian, to live, Sc. leeve), whence in ME. clive, and clēve, cleeve; the latter finally prevailed, and is now written cleave. Instead of the normal pa. tense and pple. clived, cleved, we find also from 14th c. clave, occas. clef, clof, clove, and in 17th c. cleft; in the pple. clave, clove, and cleft. At present cleave, cleaved, is the ordinary inflexion, but the influence of the Bible of 1611, in which clave is frequent (beside, and in the same sense as, cleaved), has made that an admissible form: clove, cleft are now left to cleave v.1
}

The final predominance of cleve rather than clive as the ME. form made the present stem identical in form with that of cleave v.1 to split. Hence their inflexional forms were naturally also confused, and to some extent blended or used indiscriminately. The pa. tense clave attached itself in the 14th c. to both; in this verb it corresponds to the original strong pa. tense *cláf, but does not appear to be continuous with it; it was prob. a new form due to analogy: see note to cleave v.1 The occasional pa. tense clef belongs properly to cleave v.1; as perhaps also clof, clove. (The occas. pa. pples. clave, clove, are from the pa. tense) The weak inflexion cleaved is of course proper to this verb, and prob. was transferred hence to cleave v.1 The shortened cleft found in both, appears to be due to the analogy of leave, left, bereave, -reft. To the same analogy is probably due the mod. spelling cleave in both verbs: this is not etymological, for both words had close e in ME., and would properly now be cleeve or clieve.]
%\vspace{-0.3cm}

\begin{myenumerate}

%\itembf{A.} Forms.

%\itembf{1.} pres. stem. (α) 1 clífan, clifian, 3-6 cliue(n, clyue(n, (4 clyuy), 6 clive (klɪv).

%\P 1000 in Thorpe Hom. II. 530 (Bosw.) Ðin tunge clifað to ðinum gomum.    c 
%\P 1250 GEN. \& EX.  \textit{372 And erðe freten wile he mai liuen, And atter [shall] on is tunge cliuen.    c 
%\P 1380 SIR Ferumb.
%\P 1901 ÞAT  \textit{al þy breyn scholde clyue al aboute ys fuste.    
%\P 1561 HOLLYBUSH  \textit{Hom. Apoth. 30 b, Festened or clyved upon the belly.    
%\P 1563 T. GALE  \textit{Antidot. ii. 8 They wyll‥cliue to the handes.

%(β) 1 clio-, cleofian, 3 cleou-, 4 cleuien, 4-6 cleue(n, 5 cleuy, clefe, cleeue, cleve, 6 cleeve.

%\P 1000 WHALE  \textit{73 (Gr.) Þa þe him on cleofiað.    c 
%\P 1205 LAY.
%\P 1960 ÞE  \textit{nome‥a summe stede cleouieð faste.    c 
%\P 1450 VOC. in  \textit{Wr.-Wülcker 562 Adhereo, to cleuy to.    
%\P 1483 CATH.  \textit{Angl. 67 Cleve to, herere.    
%\P 1552 ABP. HAMILTON  \textit{Catech. (1884) 36 Cleeve to him.    
%\P 1568 GRAFTON  \textit{Chron. Edw. IV, II. 699 To cleve to King Henry.    a 
%\P 1600 CHESTER  \textit{Pl. (1843-7) 214 To them‥Which cleeve to me allwaie.

%(γ) 6- cleave (cleaue).

%\P 1530 PALSGR. 486/2 My shyrte cleaveth to my backe.    
%\P 1561 T. NORTON  \textit{Calvin's Inst. iii. 211 The water stil cleaueth vpon them.    1581, 1635, etc. [see B. 2, 4].

%\itembf{2.} Past tense. (α) 1 clif-, cliof-, cleofede, 3-5 clivede, 3-6 clevede, 6- cleaved.

%\P 1000 AGS.  \textit{Gosp. Luke x. 11 Þæt dust þæt of eowre ceastre on urum fotum clifode [
%\P 1140 CLYOFEDE,
%\P 1160 CLEFEDE].    c 
%\P 1300  \textit{Havelok}
%\P 1300 AL THAT  \textit{euere in Denemark liueden On mine armes faste clyueden.    
%\P 1388 WYCLIF  \textit{Luke x. 11 The poudir that cleued [
%\P 1382 CLEUYDE]  \textit{to vs.    
%\P 1480 [SEE B 1].    
%\P 1568 GRAFTON  \textit{Chron. II. 533 He‥cleved to the Frenche king.    
%\P 1763 [SEE B 4].    
%\P 1855 TENNYSON  \textit{Maud iii. vi. iii, I cleaved to a cause that I felt to be pure and true.

%(β) 7 cleft.

%\P 1611 CHAPMAN  \textit{Iliad xvii. 359 The foes cleft one to other.    a 
%\P 1626 BP. ANDREWES  \textit{Serm. (1641) The core of corruption that cleft to our nature and to us.

%(γ) 4 claf, (claif), 4-7 claue, 7- clave (kleɪv).

%\P 1300  \textit{Cursor M.} 20745 His hend claf [Gött. clef, Fairf. cleued] to þat ber fast.    Ibid. 20954 A gast‥Þat in a maiden bodi claue [Gött. claif, Trin. clof].    
%\P 1611 BIBLE  \textit{Ruth i. 14 Ruth claue vnto her.    
%\P 1867 FREEMAN  \textit{Norm. Conq. (1876) I. ii. 60 Many of the Danes‥clave to their ancient worship.    
%\P 1887 HALL  \textit{Caine Son of Hagar II. ii. xiii. 43 His tongue clave to his mouth.

%(δ) 4 clef; (ε) 4 clof, 7-9 clove.

%\P 1300  \textit{Cursor M.} 20745 (Gött.) His hend clef to þe bere fast.    c 
%\P 1340  \textit{Ibid.} 20954 (Trin.) Þat in a maydenes body clof.    
%\P 1692 WASHINGTON  \textit{tr. Milton's Def. Pop. (1851) Pref. 10 You say, their tongues clove to the roof of their mouths‥I wish they had clove there to this day.    
%\P 1885 E. ARNOLD  \textit{Secr. Death 10 Bethink How those of old, the saints, clove to their word.

%\itembf{3.} pa. pple. (α) 1 clifod, cleofod, 3-6 cleued, 6 clyued, 6-9 cleaved.

%\P 1200 TRIN.  \textit{Coll. Hom. 73 Als hit cleued were.    
%\P 1535 COVERDALE  \textit{Job xxxi. 5 Yf I haue cleued vnto vanitie.    
%\P 1837 J. J. BLUNT  \textit{Plain Serm. Ser. iii. (1861) 256 That the Formularies of the Church‥should be cleaved unto.

%(β) 7 cleft.

%\P 1641 BROME  \textit{Joviall Crew iii. Wks.
%\P 1873 III.  \textit{411 Unlesse‥you have at least cleft or slept together.

%(γ) 7 clave, clove.

%\P 1642 ROGERS  \textit{Naaman 16 Had they clave to their duty.    
%\P 1692 [SEE  \textit{2 δ, clove].

%\itembf{B.} Signification.

\itembf{1.} To stick fast or adhere, as by a glutinous surface, to (†on, upon, in). (The perfect tenses were formerly formed with be.)

\P c897 K. ÆLFRED \textit{Gregory's Past.} xlvii. 361 His flæsces lima clifað ælc on oðrum.  
\P 1000 ÆLFRIC  \textit{Lev.} i. 8 Ealle þa \phonetic{þinᴁ} þe to þære lifre clifiaþ.
\P c1200 \textit{Trin.  Coll. Hom.} 73 Cleued bi mi tunge to mine cheken gif ich forgete þe ierusalem.
\P 1300 \textit{Fragm.  Pop. Sc.} (Wright) 229 Ren-forst‥cleveth in hegges al aboute.
\P c1430 \textit{Cookery  Bk.} 21 Ȝif it cleuey, let it boyle.    
\P 1480 CAXTON  \textit{Chron. Eng.} cci. 182 A drope of drye blode‥cleued on his hond.    
\P 1535 COVERDALE  \textit{Job} xxix. 10 Their tonges cleued [1611 cleaued]  to the rofe of their mouthes.    
\P 1561 HOLLYBUSH  \textit{Hom. Apoth.} 30 b, A pece of papir, the bignes of a groate, festened or clyued vpon the belly.    
\P 1592 GREENE in  \textit{Shaks. C. Praise} 2 Unto none of you‥sought those burres to cleaue.    
\P 1626 BACON  \textit{Sylva} §293 Water in small quantity cleaveth to any thing that is solid.    
\P 1867 M. E. HERBERT  \textit{Cradle L.} vi. 155 Huge masses of masonry, which seem to cleave to the bare rock.

\itembf{2.} fig. (Formerly said of attributes or adjuncts).

\P c888 K. ÆLFRED \textit{Boeth.} xvi. §3 Nu hi [wealth \& power] willaþ clifian [v.r. cliofian] on þæm wyrstan monnum.
\P 1325  \textit{E.E. Allit.} P. A. 1195 Bot ay wolde man of happe more hente Þen moȝten by ryȝt vpon hem clyuen.    
\P 1377 LANGL.  \textit{P. Pl.} B. xvii. 329 For kynde cleueth [v.r. clyueþ] on hym euere to contrarie þe soule.    
\P 1488 CAXTON  \textit{Chast. Goddes Chyld.} xxv. 73 The rote of his olde sinne cleuyth alway upon hym.    
\P 1581 R. GOADE in \textit{Confer.} ii. (1584) L iiij, It is no righteousnes cleauing in vs but in Christ.    
\P 1597 HOOKER  \textit{Eccl. Pol.} v. lxix. §2 The very opportunities which we ascribe to time cleave to the things themselves wherewith time is joined.    
\P 1711 ADDISON  \textit{Spect.} No. 68 \cardo{⁋}2 The Pains and Anguish which naturally cleave to our Existence in this World.    
\P 1790 PALEY  \textit{Horæ Paul.} (1849) 396 A peculiar word or phrase cleaving, as it were, to the memory.    
\P 1859 TENNYSON  \textit{Lancelot \& Elaine} 37 A horror lived about the tarn, and clave Like its own mists to all the mountain side.

\itembf{3.} In wider sense: To cling or hold fast to; to attach oneself (by grasping, etc.) to (†on, upon, in).

\P 1300 [See  A. 2 α].    
\P 1382 WYCLIF  \textit{Song of Sol.} viii. 5 What is she this‥faste cleuende vpon [v.r. to] hir leef? [Vulg. innixa super dilectum suum.]    
\P 1481 CAXTON  \textit{Myrr.} ii. vi. 76 Yf the culeuure clyue \& be on tholyfaunt.    
\P 1577 B. GOOGE  \textit{Heresbach's Husb.} iv. (1586) 185 The little Worme‥cleaving so to the Coame, as hee seemeth to be tied.

\itembf{4.} To adhere or cling to (a person, party, principle, practice, etc.); to remain attached, devoted, or faithful to. (= adhere v. 2, 3.)

\P 1330 R. BRUNNE  \textit{Chron.} (1810) 211, I trow on him gan cleue many riche present.    
\P 1377 LANGL.  \textit{P. Pl.} B. xi. 219, I conseille alle crystene cleue [v.r. clyue] nouȝte þer-on to sore.    
\P 1382 WYCLIF  \textit{Ephes.} v. 31 He schal clyue to his wyf.    
\P 1480 CAXTON  \textit{Chron. Eng.} ccxxvi. 233 In this tyme Englysshmen moche haunted and cleued to the wodenes and folye of the straungers.    
\P 1534 TINDALE  \textit{Rom.} xii. 9 Cleave [other 16th c. vv. cleaue] vnto that which is good.    
\P 1556 ABP. PARKER  \textit{Psalter} cix. 26 O helpe me Lorde‥to thee alone I clive.    
\P 1635 SWAN  \textit{Spec. M.} iii. §2. (1643) 48 To leave the literall sense‥and to cleave unto Allegories.    
\P 1763 WESLEY  \textit{Wks.} (1872) III. 140 My natural will ever cleaved to evil.    
\P 1777 BURKE  \textit{Addr. King} Wks. 1842 II. 403  We exhort you‥to cleave for ever to those principles.    
\P 1876 FREEMAN  \textit{Norm. Conq.} V. xxiii. 171 The mercenary soldiers‥clave to King Henry.

\itembf{5.} To remain steadfast, stand fast, abide, continue. Obs.

\P 1205 LAY.  9389 For nis nauere nan oðer gomen þat cleouieð alswa ueste.
\P c1250 \textit{Gen. \& Ex.}  2384 Al egipte in his wil cliueð.    
\P 1340 [See  CLEAVING ppl. a.2]    
\P 1594 HOOKER  \textit{Eccl. Pol.} iv. xi. (T.) The apostles did conform the Christians‥and made them cleave the better.

\itembf{6.} trans. To attach to. arch. rare.

\P 1958 T. H. WHITE  \textit{Once \& Future King} iii. xxviii. 460 He didna cleave importance tae it, but told the people for its worth.    
\P 1979 A. FRASER  \textit{King Charles} II ii. vii. 98 The real theme of the coronation—to cleave the Scottish people to their young King.
\end{myenumerate}


%%%%%%%%%%%%%%%%%%%%%%%%%%%%%%%%%
\myitem{clemency} n.

\noindent \phonetic{(ˈklɛmənsɪ)}

\noindent [ad. L. clēmentia, n. of state f. clēment-em clement: see -ency.]
\vspace{-0.3cm}

\begin{myenumerate}

\itembf{1.} Mildness or gentleness of temper, as shown in the exercise of authority or power; mercy, leniency.

\P 1553 \textit{Q. Mary's  Proclam.} in Strype \textit{Eccl. Mem.} III. App. v. 8 Her [the Queen's] great and aboundaunte clemencie.    
\P 1555 EDEN  \textit{Decades W.} Ind. iii. i. (Arb.) 141 To persuade hym of the clemencie of owre men.    
\P 1639 FULLER  \textit{Holy War} i. xvi. (1840) 27 A prince no less famous for his clemency than his conquests.    
\P 1716 ADDISON  \textit{Freeholder} No. 31, I have stated the true notion of clemency, mercy, compassion, good-nature, humanity, or whatever else it may be called, so far as is consistent with wisdom.    
\P 1827 HALLAM  \textit{Const. Hist.} (1876) III. xvi. 232 Clemency‥is the standing policy of constitutional governments, as severity is of despotism.    
\P 1869 LECKY  \textit{Europ. Mor.} I. xi. 199 Clemency is an act of judgment, but pity disturbs the judgment.

\itembf{b.} as a title. Obs. rare.

\P 1600 HOOKER  \textit{Eccl. Pol.} viii. vii. §4 May it please your clemencies to grant unto him the church of Tusculum.

\itembf{2.} Mildness of weather or climate; opposed to inclemency, severity.

\P 1667 E. CHAMBERLAYNE  \textit{St. Gt. Brit.} i. i. iv. (1743) 31 By reason of the clemency of the climate.    
\P 1750 JOHNSON  \textit{Rambler} No. 5 \cardo{⁋}8 The clemency of the weather.    
\P 1853 C. BRONTË  \textit{Villette} xv. (1876) 153 It rained still and blew; but with more clemency.
\end{myenumerate}


%%%%%%%%%%%%%%%%%%%%%%%%%%%%%%%%
\myitem{cloy} v.1

\noindent \phonetic{(klɔɪ)}

\noindent [Aphetic form of acloy, accloy; but it is possible that sense 1 directly represents OF. cloye-r, mod. clou-er to nail. Senses 5-8 appear to run together with those of clog v.]
\vspace{-0.3cm}

\begin{myenumerate}

\itembf{1.} trans. To nail, to fasten with a nail. Obs.

\P c1400  \textit{Beryn} 3464 Hym list to dryv in bet the nayll, til they wer fully Cloyid.

\itembf{2.} To prick (a horse) with a nail in shoeing; = accloy 1. Obs.

\P 1530 PALSGR. 487/2, I cloye a horse, I drive a nayle in to the quycke of his foote. Jencloue.‥ A smyth hath cloyed my horse.    
\P 1607 TOPSELL  \textit{Four-f. Beasts} (1673) 267 When a horse is shouldered‥or his hoof cloid with a nail.    
\P 1625 BACON  \textit{Apophth.} (R.), He would have made the worst farrier in the world; for he never shod horse but he cloyed him.    
\P 1726 \textit{Dict.  Rust.} (ed. 3) s.v., Cloyed or Accloyed, us'd by Farriers, when a Horse is pricked with a Nail in Shoeing.

\itembf{3.} To pierce as with a nail, to gore. rare.

\P 1590 SPENSER  \textit{F.Q.} iii. vi. 48 That foe‥of his [a wild boar], Which with his cruell tuske him deadly cloyd.

\itembf{4.} To spike (a gun), i.e. to render it useless by driving a spike or plug into the touch-hole. Obs.

\P 1577 HOLINSHED  \textit{Chron.} IV. 192 [They] stopped and cloied the touch holes of three peeces of the artillerie.    
\P 1603 KNOLLES  \textit{Hist. Turks} (1621) 801 They should‥cloy the great ordinance, that it might not afterwards stand the Turks in stead.    
\P 1617 MORYSON  \textit{Itin.} ii. ii. ii. 165 Hauing brought with them‥spykes, to cloy the Ordinance.    
\P 1669 STURMY  \textit{Mariner's Mag.} 19 Be sure that none of our Guns be cloy'd.    
\P 1711 \textit{Military \& Sea  Dict.} s.v. Nail, To Nail Cannon, or, as some call it, To Cloy‥but this is an antiquated Word.    
\P 1768 E. BUYS  \textit{Dict. Terms of Art} s.v. Cloyed, a Piece of Ordnance is said to be cloyed, when any Thing is got into the Touch-hole.

\itembf{5.} To stop up, block, obstruct, choke up (a passage, channel, etc.); to crowd or fill up. Obs.

\P 1548 W. PATTEN  \textit{Expedition Scotl.} in \textit{Arb. Garner} III. 86 These keepers had rammed up their outer doors, cloyed and stopped up their stairs within, etc.    
\P 1570 LAMBARDE  \textit{Peramb. Kent} (1826) 89 The fresh is not able to checke the salt water that cloyeth the chanell.    
\P 1581 MULCASTER  \textit{Positions} xxxvii. (1887) 165 Those professions and occupations, which be most cloyed vp with number.    
\P 1611 SPEED  \textit{Hist. Gt. Brit.} ix. xvi. (1632) 841 The Dukes purpose was to haue cloyed the harbour by sinking ships laden with stones, and such like choaking materials.    
\P 1636 BOLTON  \textit{Florus} 204 The Alps themselves heapt high with winter snowes, and so the wayes cloyed up.    
\P 1636 G. SANDYS  \textit{Paraphr. Div. Poems}, Lam. ii. (1648) 5 Thy Anger cloyes the Grave.

\itembf{6.} fig. To clog, obstruct, or impede (movement, activity, etc.); to weigh down, encumber. Obs.

\P 1564 BECON  \textit{Flower Godly Prayers} (1844) 18 That heavy bondage of the flesh, wherewith I am most grievously cloyed.    
\P 1567 TURBERV.  \textit{Poems, To Yng. Gentleman taking Wyfe} (R.), A bearing wyfe with brats will cloy thee sore.    
\P 1581 J. BELL  \textit{Haddon's Answ. Osor.} 137 Beyng clogged and fastened to this state of bondage (as it were cloyed in claye).    
\P 1665 GLANVILL  \textit{Sceps. Sci.} i. 3 The soul being not cloy'd by an unactive mass, as now.

\itembf{7.} To overload with food, so as to cause loathing; to surfeit or satiate (with over-feeding, or with richness, sweetness, or sameness of food).

\P 1530 PALSGR. 487/2, I cloye, I charge ones stomacke with to moche meate‥You have cloyed hym so moche that he is sicke nowe.    
\P 1586 COGAN  \textit{Haven Health} cliii. (1636) 148 The fat of flesh alone without leane is unwholesome, and cloyeth the stomach.    
\P 1593 SHAKES.  \textit{Rich. II}, i. iii. 296 Who can‥cloy the hungry edge of appetite by bare imagination of a Feast?    
\P 1621 BURTON  \textit{Anat. Mel.} ii. iii. iii. (1651) 323 They being alwayes accustomed to the same dishes‥are therefore cloyed.    
\P 1748 \textit{Anson's  Voy.} ii. xii. 266 Though this was a food that we had now been so long‥confined to‥yet we were far from being cloyed with it.    
\P 1857 DE QUINCEY  \textit{Goldsmith} Wks. VI. 197 To be cloyed perpetually is a worse fate than sometimes to stand within the vestibule of starvation.

\itembf{8.} fig. To satiate, surfeit, gratify beyond desire; to disgust, weary (with excess of anything).

\P 1576 GASCOIGNE  \textit{Compl. Philomene} (Arb.) 92 Both satisfied with deepe delight, And cloyde with al content.    
\P 1588 J. UDALL  \textit{Diotrephes} (Arb.) 17 Often preaching cloyeth the people.    
\P 1606 SHAKES.  \textit{Ant. \& Cl.} ii. ii. 241.    
\P 1624 CAPT. SMITH  \textit{Virginia} i. 17 But not to cloy you with particulars‥I refer you to the Authors owne writing.    
\P 1752 FIELDING  \textit{Amelia} iv. ii, Amelia's superiority to her whole sex, who could not cloy a gay young fellow by many years possession.    
\P 1819 BYRON  \textit{Juan} i. i, After cloying the gazettes with cant.

absol. \P 1639 FULLER  \textit{Holy War} v. xxvi. (1840) 288 These are enough to satisfy, more would cloy.    
\P 1748 HARTLEY  \textit{Observ. Man} i. ii. 227 The two frequent Recurrency of Concords cloys.    
\P 1829 H. NEELE  \textit{Lit. Rem.} 32 His [Pope's] sweetness cloys at last.

\itembf{b.} intr. (for refl.) To become satiated. rare.

\P 1721 RAMSAY  \textit{Tartana} 160 If Sol himself should shine thro' all the day, We cloy, and lose the pleasure of his ray.

\vspace{0.1cm} \noindent
To starve. (Some error.)

\P 1570 LEVINS  \textit{Manip.} 214/12 To cloy, fame consumere.
\end{myenumerate}


%%%%%%%%%%%%%%%%%%%%%%%%%%%%%%%%
\myitem{cogent} a.

\noindent \phonetic{(ˈkəʊdʒənt)}

\noindent [a. F. cogent (14th c. in Littré), ad. L. cōgent-em, pr. pple. of cōgĕre to drive together, compel, constrain, f. co- together + agĕre to drive.]
\vspace{-0.3cm}

\begin{myenumerate}

\itembf{1.} Constraining, impelling; powerful, forcible.

\P 1718 HICKES  \textit{J. Kettlewell} i. §17. 41 He was wont to do it in such an Obliging (and yet cogent) Way as‥to give no Offence.    
\P 1761 HUME  \textit{Hist. Eng.} II. xxix. 161 To these views of interest were added the motives, no less cogent, of passion and resentment.    
\P 1863 KINGLAKE  \textit{Crimea} (1877) II. i. 7 The French Emperor‥determined to insist in cogent terms.    
\P 1866 FERRIER  \textit{Grk. Philos.} I. ix. 199 Society's commands must be obeyed only in the second instance, because society is less real, less cogent than Nature.

\itembf{b.} esp. Having power to compel assent or belief; argumentatively forcible, convincing.

\P 1659 PEARSON  \textit{Creed} (1839) 135 Though the witness of John were thus cogent, yet the testimony of miracles was far more irrefragable.    
\P 1667 BOYLE  \textit{Orig. Formes \& Qual.}, To imploy such Arguments as I thought the clearest, and cogentest.    
\P 1690 LOCKE  \textit{Human Und.} i. iv, Undeniable cogent demonstrations.    
\P 1763 JOHNSON in  \textit{Boswell} an. 1781 (1847)  690/1 Sir, I have two very cogent reasons for not printing any list of subscribers.    
\P 1876 J. H. NEWMAN  \textit{Hist. Sk.} I. iv. ii. 382 The testimony of a number is more cogent than the testimony of two or three.

\itembf{c.} with dependent phr.

\P 1669 GALE  \textit{Crt. Gentiles} i. i. ii. 15 Conjectures, such as seem cogent to persuade us.    
\P 1836 PRICHARD  \textit{Phys. Hist. Mankind} (ed. 3) I. 374 Not so cogent of conviction as a positive argument would be.

\itembf{2.} Of persons: Employing force or compulsion, peremptory. Obs. rare.

\P 1672 MARVELL  \textit{Reh. Transp.} i. 89 All men are prone to be cogent and supercilious when they are in office.
\end{myenumerate}


%%%%%%%%%%%%%%%%%%%%%%%%%%%%%%%%%
\myitem{cognizant, -isant} a.

\noindent \phonetic{(ˈkɒgnɪzənt, ˈkɒnɪ-)}

\noindent [app. of modern introduction: not in Dictionaries of 18th c.; not in Todd's Johnson 1818, nor in Webster 1828; in Craig 1847. Thus, prob. formed anew, directly from cognizance, cognize; but it corresponds in form to OF. conisant, conusant pr. pple. Cf. cognoscent.]
\vspace{-0.3cm}

\begin{myenumerate}

\itembf{1.} Having cognizance or knowledge (see cognizance 2); aware (of).

\P 1820 SOUTHEY  \textit{Ode on Portrait of Bp. Heber}, If the Saints in bliss Be cognizant of aught that passeth here.    
\P 1832 AUSTIN  \textit{Jurispr.} (1879) I. xxv. 499 The party shall be presumed conusant of the law‥his ignorance shall not exempt him.    
\P 1879 CARPENTER  \textit{Ment. Phys.} i. ii. §82 The following circumstance, of which the writer is personally cognizant.

\itembf{b.} Philos. That knows or cognizes.

\P 1862 F. HALL  \textit{Hindu Philos. Syst.} 54 If this cognition were that which apprehends objects, the soul would be cognizant.

\itembf{2.} Law. Having cognizance or jurisdiction (see cognizance 3); competent to deal judicially with a cause, crime, etc.

\P 1847 in CRAIG.
\end{myenumerate}


%%%%%%%%%%%%%%%%%%%%%%%%%%%%%%%%%
\myitem{collation} n.

\noindent \phonetic{(kəˈleɪʃən)}

\noindent [a. OF. collation, -cion action of conferring, etc., ad. L. collātiōn-em, n. of action f. collāt- ppl. stem of confer-re to bring together: see confer, and -ation. This word has had many developments of meaning in med. Latin, French, and English; with us, it appears first as an ecclesiastical term, in sense 6.

   (In mod.F. collation is used in senses 3, 4; 8, 9; 10, 11. According to Littré in senses 8, 9, it is pronounced with one l only, whereas in the other senses both l's are heard; consequently he treats collation the repast as a distinct word (so far as modern use is concerned) from the other senses. In English, 8 and 9 are closely articulated to other senses.)]
\vspace{-0.3cm}

\begin{myenumerate}

\itembf{I.} Bringing together, comparison.

\itembf{1.} A bringing together or collection, esp. of money; a contribution. Obs.

\P 1382 WYCLIF  \textit{Rom.} xv. 26 To make sum collacioun [Vulg. collationem], or gedrynge of moneye.    
\P 1565 COOPER  \textit{Thesaurus, Symbolum,} a shotte: a collation.    
\P 1600 HOLLAND  \textit{Livy} v. xxv. 196 The collation and gathering of a small donative.    
\P 1725 tr. \textit{Dupin's Eccl. Hist. 17th c.} I. v. 67 They publish'd also in Sermons the Collations, that is, the Alms which they commonly collected every Sunday for the Poor.

\itembf{b.} Roman and Scotch Law. The throwing together of the possessions of several persons, in order to an equal division of the whole stock; hotch-pot; L. collatio bonorum.

\P 1828 WEBSTER,  \textit{Collation} 5 In Scots law, the right which an heir has of throwing the whole heritable and movable estates of the deceased into one mass, and sharing it equally with others who are of the same degree of kindred.    
\P 1886 J. MUIRHEAD  \textit{Encycl. Brit.} XX. 714 The application of the principle of collation to descendants generally, so that they were bound to throw into the mass of the succession before its partition every advance they had received from their parent in anticipation of their shares.

\itembf{c.} collation of seals (see quot.).

\P 1708-15 KERSEY  \textit{Collation of Seals} (in ancient Deeds), when one Seal was set on the Back of another, upon the same Ribbon, or Label. So 1721 in Bailey. 1848 in Wharton.

\itembf{2.} The action of bringing together and comparing; comparison.

\P 1374 CHAUCER  \textit{Boeth.} iv. iv. 125 Ellys he mot shewe þat þe colasioun of proposiciouns nis nat spedful to a necessarie conclusioun.    
\P 1398 TREVISA  \textit{Barth. De P.R.} ii. xviii. (1495) 43 An angel‥vnderstondyth and knowyth sodaynly wythout collacion of one thynge to a nother.    1570-6 Lambarde Peramb. Kent (1826) 98 That the truth may appeere, by collation of the divers reports.    
\P 1646 T. PHILIPOT  \textit{Poems} 43 A Collation between Death and Sleep.    
\P 1669 GALE  \textit{Crt. Gentiles} i. i. xi. 65 The Hebrew and Egyptian Language had some things commun; from the collation whereof, some light may arise.    
\P 1790 PALEY  \textit{Horæ Paul.} ii. §1 A close and attentive collation of the three writings.    1836-7 Sir W. Hamilton Metaph. xxxiv. (1859) II. 278 This‥necessarily supposes a comparison, a collation, between existence and non-existence.    
\P 1848 MILL  \textit{Pol. Econ.} I. 430.

\itembf{3.} esp. Textual comparison of different copies of a document; critical comparison of manuscripts or editions with a view to ascertain the correct text, or the perfect condition of a particular copy.

\P 1532 W. THYNNE  \textit{Chaucer's Wks.} Ded., The contrarietees and alteracions founde by collacion of the one [edition] with the other.    
\P 1568 in H. Campbell \textit{Love-lett. Mary Q. Scots App.} 52 The originals‥were duly conferred and compared‥with sundry other lettres‥in collation whereof no difference was found.    
\P 1717 ATTERBURY  \textit{Let. to Pope} 8 Nov., I return you your Milton, which, upon collation, I find to be revised and augmented in several places.    
\P 1768 JOHNSON  \textit{Pref. to Shaks. Wks.} IX. 292 By collation of copies, or sagacity of conjecture.    
\P 1868 FURNIVALL  \textit{Temp. Pref. Canterb. T.} (Chaucer Soc.) 5, The MS. was old and good enough to deserve collation for the next edition of Chaucer.

\itembf{b.} The recorded result of such comparison; a set of corrections or various readings obtained by comparing different copies.

\P 1699 BENTLEY  \textit{Phal.} Pref. Wks. 1836 I. 2  The collation, it seems, was sent defective to Oxon.    
\P 1758 JORTIN  \textit{Erasm.} I. 392 Erasmus desires Aldrige to get him a Collation of Seneca‥from a Manuscript of King's College.    
\P 1875 SCRIVENER  \textit{Lect. Grk. Test.} 54 Bentley's collation [of Codex A]‥is yet in manuscript at Trinity College, Cambridge.

\itembf{c.} Law. (See quot.)

\P 1727-51 CHAMBERS  \textit{Cycl.}, Collation, in common law, is the comparison, or presentation of a copy to its original, to see whether or no it be conformable: or the report, or act of the officer who made the comparison. A collated act is equivalent to an original; provided all the parties concerned were present at the collation.

\itembf{4.} Printing and Bookbinding. \textbf{a.} The action of collating the sheets or quires of a book or MS.

\itembf{b.} A description of a book or manuscript by its signatures or the number of its quires, and a statement of the sheets or leaves in each quire; also, a list of the various contents of a book and of the pages or parts of pages occupied by them.

\P 1834 LOWNDES  \textit{Bibliogr. Manual} Pref., He gives neither the collation nor prices of books.    
\P 1882 BLADES  \textit{Caxton} 131 In Caxton's books the collation of the sheets preceded the folding.    Ibid. 133 These indications‥enable us to decide, even where printed signatures are wanting, the true collation of a book.    Ibid. 173 The Game and Play of the Chess moralised‥Collation.—Eight 4ns and one 5n = 74 leaves.

\itembf{II.} Conference, discourse, refection, light repast.

\itembf{5.} A personal conferring together; consultation, conference, esp. of a private or informal sort.

\P 1382 WYCLIF  \textit{2 Macc.} xii. 43 Collacioun [Vulg. collatione], or spekinge to gidre.
\P c1386 CHAUCER  \textit{Clerk's T.} 269 Yit wol I‥That in my chambre, I and thou and sche Have a collacioun.    
\P 1474 CAXTON  \textit{Chesse} iii. v. G vj b, They ought not there to argue and dispute one agaynst another; but they ought to make good and symple colacion to geder.    
\P 1538 \textit{Songs  Costume} (Percy Soc.) 77 Quhen thay wald mak collatioun, With any lustie companyeoun.    
\P 1655 FULLER  \textit{Ch. Hist.} ii. ii. §90 Baronius and Binnius will in no case allow this for a council, only they call it a collation.    
\P 1666 EVELYN  \textit{Mem.} (1857) III. 176 Collation with our officers.

\itembf{b.} A discourse, sermon, or homily; a treatise, exposition. Obs.

\P 1417 J. FORESTER in \textit{Rymer Fœdera} (1710) IX. 434 Cardenal Comeracence‥had purposit‥to have y maad the ferste Collation to for the Kynge.    
\P 1494 FABYAN vii. 306 He made vnto them colacions or exortacions, \& toke for his anteteme, Haurietis aquas.    
\P 1525 LD. BERNERS  \textit{Froiss.} II. ci. [xcvii.] 295 The archebysshope of Canterbury sang the masse; and after masse ye bissoppe made a collacyon.    
\P 1526 \textit{Pilgr.  Perf.} (W. de W. 1531) 43 We shall fyrst declare by ordre thre thynges, and so procede in this poore collacyon or treatyse.    
\P 1555 \textit{Fardle  Facions} ii. xii. 273 The collacion‥made in the pulpite on Sondaies and haly daies.    
\P 1631 WEEVER  \textit{Anc. Fun.} Mon. 65 If any Priest came‥into the village, the inhabitants thereof would gather about him, and desire to haue some good lesson or collation made vnto them.    
\P 1655 FULLER  \textit{Hist. Camb.} 101 Bilney‥for the present gave them a Collation.

\itembf{6.} The title of the celebrated work of John Cassian, a.d. 410-420 Collationes Patrum in Scetica Eremo Commorantium, i.e. Conferences of (and with) the Egyptian Hermits.

[\P c540 \textit{Regula S. Benedicti} lxxiii, Nec non et Collationes Patrum et Instituta et Uita eorum, sed et Regula sancti patris nostri Basilii.]  
\P 1200 \textit{Winteney  Rule St. Benet} ibid., Oððe þa collatiuns, þæt Iohannes Cassianus awrat, \& þere haliȝere manna lif þe on Uitas Patrum is ȝeredd, \& þe regol ures haliȝes fader Basilies.    
\P 1340  \textit{Ayenb.} 155 Ase zayþ þe boc of collacions of holy uaderes.    
\P 1460-70  \textit{Bk. Quintessence} 18 As it is preued in vitas patrum, þat is to seye, in lyues \& colaciouns of fadris.
\P a1500 \textit{Orol.  Sap.} in \textit{Anglia} X. 357 Þe boke of lyfe of fadres \& her collacyons.    
\P 1532 MORE  \textit{Confut. Tindale} Wks. 516/2 Cassianus in the .xi. collacion the .xii. chapter.    
\P 1699 BURNET 39 \textit{Art.} xvii. (T.), No book was more read in the following ages than Cassian's Collations.    
\P 1885 \textit{Catholic  Dict.} s.v. Fast 341 St. Benedict‥requires his religious to assemble after supper and before compline and listen to ‘collations’—i.e. conferences (of Cassian), the lives of the fathers or other edifying books.

\itembf{b.} In OE., collationes, as above, was rendered \phonetic{þurhtoᴁenes} raca, þa \phonetic{þurhtoᴁenessa}, also simply race, recednesse, 
c1200 þA  raca, i.e. relations, narratives, discourses, and in ME. collation had the sense: Relation, account. Obs.

[c540 \textit{Regula S. Benedicti} xlii, Mox ut surrexerint a cena, sedeant omnes in unum, et legat unus collationes, vel vitas patrum, aut certe aliquid quod edificet audientes‥Accedant ad lectionem Collationum.
\P a1000 \textit{O.E.  Rule St. Benet} (Schröer) xlii, Ræde him mon þa raca oðþe lif þæra heahfædera.    Ibid. (Logemann) And ræde an \phonetic{þurhtoᴁenes} race oððe on ealdfædera lifa‥Hi gan to rædinge race oððe recednesse.
\P c1200 \textit{Winteney  Rule St. Benet}, ibid., And ræde an þa raca oððe lif þære heahfadera.]

\P 1430 PILGR.  \textit{Lyf Manhode} iii. xxxii. (1869) 153 It is wel‥myn entencioun þat þou make me þer of collacioun.

\itembf{7.} ‘The reading from the Collationes or lives of the Fathers, which St. Benedict (Regula xlii, see 6 b.) instituted in his monasteries before compline’ (Dict. Chr. Antiq.).

Whether the name actually originated in the Collationes Patrum read on these occasions does not appear certain. Already in Isidore, a 640, the name is simply collatio (Regula S. Isidori c. viii, ‘ad audiendum in Collatione Patrem‥ad collectam conveniant‥Sedentes autem omnes in Collatione tacebunt nisi,’ etc. Du Cange). By Smaragdus a 850, and Honorius of Autun (c 1300), the collatio is explained as being itself a conference of the monks upon the passage read, ‘aliis conferentibus interrogationes, conferunt alii congruas responsiones’. (See Du Cange.)

\P 1387 TREVISA  \textit{Higden} (Rolls) VI. 121 After þe nyȝt collacioun sche wook anon to þe day.    Ibid. VII. 373 He wolde be at þe colacioun of monkes, and made þe general confessioun wiþ oþere.    1450-
\P 1530 \textit{Myrr.  our Ladye} 165 Before Complyn ye haue a collacion, where ys redde some spyrytuall matter of gostly edyfycacion.    
\P 1482 \textit{Monk  of Evesham} vi. (Arb.) 26 The mene while‥hit range to the collacyon and the bretheren‥went thense.    
\P 1526 \textit{Pilgr.  Perf.} (W. de W. 1531) 65 Redynge in ye refectory, or in the chapyter hous at collacyon.    
\P 1536 R. BEERLEY in \textit{Four C. Eng. Lett.} 35 Monckes drynk an bowll after collacyon tell ten or xii. of the clock.

\itembf{8.} Extended to the light repast or refection taken by the members of a monastery at close of day, after the reading or conference mentioned in 7. (Many quotations combine senses 7 and 8.) Hence, in modern R.C. usage, A light repast made in lieu of supper on fasting days.

\P 1305 \textit{Land  Cokayne} 145 [The monks] Wendith meklich hom to drinke And geth to har collacione.    
\P 1582 MUNDAY  \textit{Eng. Rom. Life} in \textit{Harl. Misc.} II. 179 The time of studye expired, the bell calleth them from theyr chambers, downe into the Refectorium: Where euery one taketh a glasse of wine, and a quarter of a manchet, and so he maketh his collatione.    
\P 1725 tr. \textit{Dupin's Eccl. Hist. 17th c.} I. v. 84 This is that which is call'd Collation‥after the Conference they took Water or Wine, and a mouthful of Bread to support their Necessities.    
\P 1797 MRS. RADCLIFFE  \textit{Italian} xi, The lady-abbess, gave a collation to the padre abbate and such of the priests as had assisted at Vesper-service.    
\P 1885 \textit{Catholic  Dict.} s.v. Fast 342 The quantity permissible at collation has been gradually enlarged. St. Charles‥only allows a glass of wine with an ounce and a half of bread to be taken as a collation on the evening of fasting days.

\itembf{9.} Hence, in gen. use, A light meal or repast: one consisting of light viands or delicacies (e.g. fruit, sweets, and wine), or that has needed little preparation (often ‘a cold collation’). ‘A repast; a treat less than a feast’ (J.).

Originally applied to a repast between ordinary meals, and still retaining much of that character.

\P 1525 LD. BERNERS  \textit{Froiss.} II. xci. [lxxxvii.] 272 Than wyne and spyces were brought in, and so made collasyon.    
\P 1533 UDALL  \textit{Flowers} 75 (R.) Such bankettes are called collacions, a collatum, tu, that is of laiyng together every one his porcion.    
\P 1611 COTGR.,  \textit{Collation}‥also, a collation, rere$\sim$supper, or repast after supper.    
\P 1630 R. JOHNSON  \textit{Kingd. \& Commw.} 183 Very few which (besides their ordinary of dinner and supper) doe not Gouster, as they call it, and make collations, three or foure times the day.    
\P 1664 PEPYS  \textit{Diary} (1879) III. 4 Come to the Hope about one and there‥had a collacion of anchovies, gammon, etc.    
\P 1759 ROBERTSON  \textit{Hist. Scot.} I. vii. 536 A collation of wine and sweetmeats was prepared.    
\P 1771 SMOLLETT  \textit{Humph. Cl.} (1815) 111 Supping in different lodges on cold collations.    
\P 1775 JOHNSON  \textit{Western Isl.}, Buller of Buchan, Ladies come hither sometimes in the summer with collations [i.e. to picnic].    
\P 1882 SHORTHOUSE \textit{J. Inglesant} II. 205 A plentiful and delicate collation was spread‥with abundance of fruit and wine.

fig. \P 1652 A. ROSS  \textit{Hist. World} Pref. 13 Here they may have a short Collation after a long Feast.
\P a1661 FULLER  \textit{Worthies} iii. 96 May he be pleased to behold this my brief Description of Surrey, as a Running Collation to stay his Stomack, no set meal to satisfie his hunger.    
\P 1791 D'ISRAELI  \textit{Cur. Lit.}, Lit. Journ., The public‥now murmured at the want of that salt and acidity by which they had relished the fugitive collation.

\itembf{III.} Conferring, preferment to office, etc.

\itembf{10.} Conferring or bestowal (esp. of a dignity, prize, benefit, honorary degree). Obs. exc. as in 11.

\P 1579 FENTON  \textit{Guicciard.} ii. (1599) 90 Honoring in him by the collation of that dignitie, the vertue he shewed in the battell.    
\P 1642 JER. TAYLOR  \textit{Episc.} (1647) 47 In the collation of holy Orders.    
\P 1647 LILLY  \textit{Chr. Astrol.} xxxvii. 217 Mutuall reception or translation, or collation of light and nature betwixt them.    
\P 1660 BOND  \textit{Scut. Reg.} 88 The donation or collation of the power is from the Community.
\P a1677 BARROW  \textit{Serm.} I. viii. 95 In the collation, 'tis not in the gold or the silver‥in which the benefit consists, but the will and benevolent intention of him who bestows them.    
\P 1691 RAY  \textit{Creation} ii. (1704) 436 Neither are we to give Thanks alone for the first Collation of these Benefits.    
\P 1761 CHRON. in  \textit{Ann. Reg.} 128/1 The collation of the prize has been deferred.    
\P 1775 JOHNSON  \textit{Western Isl. Wks.} X. 332 The indiscriminate collation of degrees has justly taken away that respect which they originally claimed.

\itembf{11.} Eccl. \textbf{a.} The bestowal of a benefice or other preferment upon a clergyman. b.III.11.b (more usually) The appointment of a clergyman to a benefice; now, techn. Institution by the ordinary to a living which is in his own gift.

\P 1380 WYCLIF  \textit{Serm. Sel.} Wks. I. 305 It haþ fallen ofte tymes‥þat two men have grace at oo tyme of oo collacioun.    
\P 1421 HEN. V in \textit{Ellis Orig. Lett.} iii. 30 I. 71 Hit is wel oure entent whanne any sucche benefice voydeth of oure yifte yat ye make collacion to him yr of.    
\P 1611 SPEED  \textit{Hist. Gt. Brit.} ix. xiii. §88 They had enacted against all Collations of Bishoprickes and dignities by the Pope.    
\P 1625 BACON  \textit{Ess. Empire} (Arb.) 307 Where the Churchmen come in, and are elected, not by the Collation of the King, or particular Patrons, but by the People.    
\P 1641 \textit{Termes  de la Ley} 64 Collation is properly the bestowing of a Benefice by the Bishop, that hath it in his owne gift or patronage.    
\P 1765 BLACKSTONE  \textit{Comm.} I. 391 When the ordinary is also the patron, and confers the living, the presentation and institution are one and the same act, and are called a collation to a benefice.    
\P 1876 GRANT  \textit{Burgh Sch. Scotl.} i. i. 22 The earliest record of an actual collation by the chancellor of a master to a grammar school.

\itembf{c.} Right of institution.

\P 1480 \textit{Bury  Wills} (1850) 58 That‥the priour of the Monasterie of Bury‥shuld have the gyfte and collacion of the same.    
\P 1536  \textit{Act} 27 Hen. VIII, c. 42 §6 in Oxf. \& Camb. Enactm. 18 Any Parsonnage, Vicarage, Chauntrie or any other promocion spirituall‥being‥of the collacion or patronage of the said College.    
\P 1661 BRAMHALL  \textit{Just Vind.} iv. 79 And the Statute of provisors‥the King and his heirs shall have and enjoy for the time the collations to the Archbishopricks and other dignities elective.    
\P 1725 tr. \textit{Dupin's Eccl. Hist. 17th c.} I. ii. iii. 46 Pope Clement IV reserv'd to himself the Collation of all the vacant Benefices.

\itembf{d.} ? A certificate of recommendation to a benefice. Obs. [F. la provision du collateur.]

\P 1646 BP. MAXWELL  \textit{Burd. Issach.} in \textit{Phenix} (1708) II. 293 Before their Right could be compleated or perfected, they were to return to the King from the Superintendent a Collation or Certificate, That he was of that Ability to do good Service to the King and Church.
\end{myenumerate}


%%%%%%%%%%%%%%%%%%%%%%%%%%%%%%%%
\myitem{colloquy} n.

\noindent \phonetic{(ˈkɒləkwɪ)}

\noindent [ad. L. colloqui-um speaking together, conversation, conference, f. col- together + -loquium speaking, f. loqui to speak.]
\vspace{-0.3cm}

\begin{myenumerate}

\itembf{1.} A talking together; a conversation, dialogue. Also, a written dialogue, as Erasmus's Colloquies.

\P 1581 MULCASTER  \textit{Positions} xli. (1887) 238 All conferences, all both priuate and publike colloquies.    
\P 1660 R. BLOME  \textit{Fanat. Hist.} ii. 16 Frantick men that boasted of visions, and colloquies with God.    
\P 1755 JORTIN  \textit{Erasm.} I. 296 The Colloquies of Erasmus‥well deserve to be read.    
\P 1829 SOUTHEY  \textit{(title)}, Sir Thomas More: or Colloquies on the Progress and Prospects of Society.    
\P 1850 MRS. STOWE  \textit{Uncle Tom's C.} xxii. 222 The colloquy between Tom and Eva was interrupted by a hasty call from Miss Ophelia.    
\P 1885  \textit{Life Sir R. Christison} I. 168 Our host in the course of our colloquy, said, etc.

\itembf{b.} (without pl.) Converse, dialogue.

\P 1817 BYRON  \textit{Manfred} iii. i, Shunning‥All further colloquy.
\P a1839 PRAED  \textit{Poems} (1864) II. 36 When they chance to make In colloquy some small mistake.    
\P 1850 GROTE  \textit{Greece} ii. lxx. VI. 267 To invite the natives to amicable colloquy.

\itembf{2.} A meeting for conference.

\P 1563-87 FOXE  \textit{A. \& M.} (1596) 263/2 Cluniake, where was‥appointed a secret meeting or colloquie betweene the Pope and Lewis the French King.    
\P 1661 BRAMHALL  \textit{Just Vind.} ii. 22 Debated between the Catholick Bishops, and the schismatical Donatists at the Colloquie of Carthage.    
\P 1679  \textit{Trial of White \& Other Jesuits} 12 They adjourned into several Clubs or Colloquies, or what you please to call them.

\itembf{3.} Eccl. In the Reformed Genevan or Presbyterian Churches, a church court composed of the pastors and representative elders of the churches of a district, with judicial and legislative functions over these churches; = classis, presbytery.

\P 1672 P. NYE  \textit{Oath Suprem.} (1683) 54 There are Synods, Consistories, Colloquies, and other Ecclesiastical Courts.    
\P 1692 J. QUICK  \textit{Synodicon} xxxvii, In every Province the Churches shall be divided according to their numbers and conveniency of neighbour places into Colloquies or Classes.    
\P 1846 J. S. BURN  \textit{For. Prot. Refugees} 45 Charges against the moral character of this minister‥were entertained by the colloquy, which pronounced sentence in 1647.    
\P 1862 LATHAM in  \textit{Ansted Channel Isl.} iii. xv. (ed. 2) 367 The Curate of St. John's parish died, and the colloquy appointed to the vacant benefice.    
\P 1889 A. H. DRYSDALE  \textit{Hist. Presbyt. Eng.} i. 173 The Church Courts were the ‘Consistory’ and the ‘Colloquy’ or Presbytery meeting quarterly, and the Synod every two years in Jersey and Guernsey alternately. The Colloquies and Consistories were, as at Geneva, strict courts of morals, fitted in to the general civil jurisdiction.

\noindent
Hence colloquy v. intr., to hold colloquy.

\P 1868 HAWTHORNE  \textit{Amer. Note-bks.} (1879) II. 142 They colloquied at much length.
\end{myenumerate}


%%%%%%%%%%%%%%%%%%%%%%%%%%%%%%%%
\myitem{compendium} n.

\noindent \phonetic{(kəmˈpɛndɪəm)}

\noindent [a. L. compendium that which is weighed together, a sparing, saving, abbreviation, f. compend-ĕre to weigh together, f. com- + pendĕre to weigh.]
\vspace{-0.3cm}

\begin{myenumerate}

\itembf{1.} A short cut; ‘the near way’ (J.).

\P 1581 MULCASTER  \textit{Positions} xlii. (1887) 258 [He] may perhaps wish for some way without Grammer, and couet a Compendium.

\itembf{2. a.} An abridgement or condensation of a larger work or treatise, giving the sense and substance, within smaller compass.

\P 1589 NASHE  \textit{Pref. to Greene's Arcadia} (1616) 7 These men‥doe pound their capacitie in barren Compendiums.    
\P 1668 HALE  \textit{Pref. Rolle's Abridgm.} 5 There were an incredible number of‥Volumes of their Laws; whereupon that‥Prince‥reduced them into a better Compendium.    
\P 1793 T. BEDDOES  \textit{Math. Evid.} 79 The writers of compendiums of mathematics and natural philosophy.    
\P 1878 HUXLEY  \textit{Physiogr.} Pref. 6 Many highly valuable compendia of Physical Geography are extant.

fig. \P 1607 T. WALKINGTON  \textit{Opt. Glass} xv. (1664) 158 Others, having but the compendium of excellency, he alone had it in the greatest volumns.

\itembf{b.} An epitome, a summary, a brief.

\P 1608 MIDDLETON  \textit{Fam. Love} v. iii, You understand my case now? I do‥here's the compendium.    
\P 1619 DRAYTON  \textit{Legends} Pref., By way of Briefe or Compendium.    
\P 1713 \textit{Guardian}  No. 78 Indexes and dictionaries‥are the compendium of all knowledge.    
\P 1853 HERSCHEL  \textit{Pop. Lect. Sc.} iv. §30 (1873) 167 Admiral Fitzroy's interesting compendium of the state of the barometer, etc.

\itembf{c.} transf. and fig. A condensed representation, an embodiment in miniature; an abstract.

\P 1602 \textit{Return  fr. Parnass.} iii. iv. (Arb.) 44 Old Sir Raderick, that new printed compendum of all iniquity.    
\P 1634 SIR T. HERBERT  \textit{Trav.} 231 Great Brittaine, a Compendium of the World for varietie of Excellencies.    
\P 1766 STERNE  \textit{Serm.} v. 112 A case‥which may be looked upon as the compendium of all charity.    
\P 1863 GEO. ELIOT  \textit{Romola} i. vi, A compendium of extravagances and incongruities.

\itembf{d.} An abbreviation whereby two or more letters are expressed by a single character.

\P 1833 G. S. FABER  \textit{Recapit. Apost.} 88 In the construction of these compendia or‥contractions, the compendium ϛ was framed out of the two distinct cursive letters ς and τ.

\itembf{3.} Sparing or saving; economy of labour, space, etc. Obs.

\P 1638 WILKINS  \textit{New World} i. (1684) 29 Shewing a Compendium of Providence, that could make the same Body a World, and a Moon.    
\P 1651 CHARLETON  \textit{Ephes. \& Cimm. Matrons} ii. (1668) 71 Nor do we think that substraction a loss, but a Compendium.    
\P 1668 WILKINS  \textit{Real Char.} 372 Double Consonants‥for the Compendium of writing, are‥expressed by single Characters.    a 
\P 1734 NORTH  \textit{Lives} I. 248 The judges, for compendium of travel, took the first town‥capable of receiving them.    ― Exam. iii. x. (1740) 660 These Methods are used for Compendium.    
\P 1793 SMEATON  \textit{Edystone L.} §32 The manner‥is herein copied, on account of the compendium thereby suggested.    
\P 1812 WOODHOUSE  \textit{Astron.} xviii. 199 The sole object of this‥is compendium of calculation.

\itembf{4. a.} A box, etc., containing or comprising several different games.

\P 1899 \textit{(title)}  Guide to the compendium of games. Comprising rules for playing—backgammon, besique, chess, [etc.].    
\P 1960 R. C. BELL  \textit{Board \& Table Games} vii. 172 The antique compendium of games from western India.

\itembf{b.} A package of the stationery required for letter-writing.

\P 1923 H. A. MADDOX  \textit{Dict. Stationery} 20 Compendium, a line of stationery goods which was in considerable demand during the war for soldiers' use, comprising a pad of note, envelopes, and blotting.    
\P 1938 E. BOWEN  \textit{Death of Heart} ii. iii. 215 Portia bought a compendium—lightly ruled violet paper, purple lined envelopes.    
\P 1960 K. AMIS  \textit{Take a Girl like You} xiii. 158 He shut the compendium.
\end{myenumerate}


%%%%%%%%%%%%%%%%%%%%%%%%%%%%%%%%%
\myitem{complacent} a.

\noindent \phonetic{(kəmˈpleɪsənt)}

\noindent [ad. L. complacēnt-em pleasing, pr. pple. of complacēre: see above.]
\vspace{-0.3cm}

\begin{myenumerate}

\itembf{1.} Pleasing, pleasant, delightful. Obs. rare.

\P 1660 BURNEY  Κέρδ. Δῶρον (1661) 106 In the complacent moneth of May.    
\P 1772 MACKENZIE  \textit{Man of World} i. i, Her look was of that complacent sort which gains on the beholder.

\itembf{2.} spec. Feeling or showing pleasure or satisfaction, esp. in one's own condition or doings; self-satisfied.

\P 1767 JAGO  \textit{Edge Hill, Evening} iv. (R.), With complacent smile Thy social aspect courts the distant eye.
\P 1791 COWPER  \textit{Iliad} iv. 423 The monarch smiled Complacent.    
\P 1825 SOUTHEY  \textit{Paraguay} i. 25 The glorious savage‥vain of his array Look'd with complacent frown from side to side.    
\P 1841 L. HUNT  \textit{Seer} (1864) 52 Whenever Gibbon was going to say a good thing‥he announced it by a complacent tap on his snuff-box.    
\P 1875 GLADSTONE  \textit{Glean.} VI. xxxviii. 129 Multitudes‥will accede‥to this proposition‥but with a complacent conviction‥that it does not touch their case.

\itembf{3.} Disposed, or showing a disposition, to please; obliging in manner, complaisant. ? Obs.

\P 1790 BURKE  \textit{Fr. Rev.} Wks. V. 160 They look up with a sort of complacent awe and admiration to kings, who know how to keep firm in their seat.    
\P 1821 SCOTT  \textit{Kenilw.} xxii, The‥complacent flattery of Leicester.    
\P 1849 C. BRONTË  \textit{Shirley} vi. 62 Mr. Moore‥was‥a complacent listener to her talk.
\end{myenumerate}


%%%%%%%%%%%%%%%%%%%%%%%%%%%%%%%%%
\myitem{complaisant} a.

\noindent \phonetic{(ˈkɒmpleɪˌzɑːnt, -æ-, ˌkɒmpleɪˈzɑːnt, -æ-, now kəmˈpleɪzənt)}

\noindent [17th c. a. F. complaisant (16th c. in Littré), pr. pple. of complaire to acquiesce in order to please:—L. complacēre to be very pleasing to: cf. complacent, complease. In 17th c. it was sometimes assimilated in form to complease, pleasant, with stress on 2nd syllable; but a general recognition of its French nativity has preserved the Fr. spelling, with the main stress (c1891) varying between the 3rd and the 1st syllable. Walker c1800 has (\phonetic{kɒmpliːˈzænt}).]
\vspace{-0.3cm}

\begin{myenumerate}

\itembf{1.} Characterized by complaisance; disposed to please; obliging, politely agreeable, courteous. (Of persons, their actions, manners, etc.)

\P 1647 COWLEY  \textit{Mistr., Echo} (1669) 40 Complaisant Nymph [Echo], who do'est thus kindly share In griefs, whose cause thou do'st not know!    
\P 1651 CHARLETON  \textit{Ephes. \& Cimm. Matrons} (1668) 22 The most affable, compleasant, and chearfull creature in the world.    
\P 1664 SIR C. LYTTELTON in \textit{Hatton Corr.} (1878) 38 Feare not you will find mee as complizant.    
\P 1671 VILLIERS  (Dk. Buckhm.) \textit{Rehearsal} (1714) 55 That's very complaisant‥Mr. Bayes, to be of another Man's Opinion, before he knows what it is.
\P a1720 SHEFFIELD  (Dk. Buckhm.) \textit{Wks.} (1753) I. 14 Cautious the young, and complaisant the old.    
\P 1727 SWIFT  \textit{Gulliver} ii. iv. 131 The girl was complaisant enough to make the bearers stop.    
\P 1871 SMILES  \textit{Charac.} ix. (1876) 242 The French‥of even the humblest classes, are‥complaisant, cordial, and well-bred.

\itembf{b.} Disposed to comply with another's wishes; yielding, accommodating; compliant, facile.

\P 1676 G. ETHEREGE  \textit{Man of Mode} iv. i, I am sorry my face does not please you as it is, But I shall not be complaisant and change it.    
\P 1678 RYMER  \textit{Trag. Last Age} 69 Had [she] been formerly complaisant with him beyond discretion.    
\P 1839 JAMES  \textit{Louis XIV}, I. 246 Richelieu, not finding the clergy quite so complaisant as he could have desired.

\itembf{2.} Of things: Pleasant, agreeable. Obs. rare.

\P 1710 T. FULLER  \textit{Pharm. Extemp.} 293 An honest benign Medicine, yet its not very complaisant to the Palate.
\end{myenumerate}


%%%%%%%%%%%%%%%%%%%%%%%%%%%%%%%%%
\myitem{complement} v.

\noindent \phonetic{(kɒmplɪˈmɛnt)}

\noindent [f. prec.]
\vspace{-0.3cm}

\begin{myenumerate}

\itembf{I.} Extant sense.

\itembf{1.} trans. To make complete or perfect, to supply what is wanting; to form the complement to.

\P 1641 BAKER  \textit{Chron.} (1679) 38/1 He never stayed to complement the disaster.    
\P 1865  \textit{Reader} No. 143. 337/2 Information‥from other documents to complement these.    
\P 1875 STUBBS  \textit{Const. Hist.} I. ii. 36 The three principles‥complement and complicate each other's action.    
\P 1879 FARRAR  \textit{St. Paul II. App.} 614 Truths which complement but do not contradict each other.

\itembf{II.} Obsolete senses, afterwards expressed by compliment v.

\itembf{2.} intr. To employ ceremonies of formal courtesy, to exchange formal courtesies; to bow. Obs. (= compliment, sense 1.)

\P 1612 BEAUM. \& FL.  \textit{Coxcomb} i. ii. (1647) 24 Serv. Mistris there are 2 Gentlemen. Mar. Where? Serv. Complementing who should enter first.    
\P 1642 BP. REYNOLDS  \textit{Israel's Petit.} 3 Complementing with God, and then forsaking him.    
\P 1644 MILTON  \textit{Areop.} (Arb.) 40 Sometimes 5 Imprimaturs‥in the Piatza of one Title-page, complementing and ducking each to other with their shav'n reverences.    
\P 1658 SIR ASTON  COCKAIN \textit{Trappolin} iii. i, Complement with me no more than I complement with you.    
\P 1697 MOUNTFORT  \textit{Faustus} i. end, Here they Complement who shall go first.

\itembf{b.} So to complement it. Obs.

\P 1617 BP. ANDREWES  \textit{96 Sermons} (1661) 651 As if we could complement it with God, with face and phrases, as with men we do.    
\P 1624 D. CAWDREY  \textit{Humilitie Saints Liverie} 9 Thus shall you have a man‥complement it to the ground, lay his hands under your feet, etc.

\itembf{3.} trans. ‘To sooth with acts or expressions of respect; to flatter; to praise’: see compliment, sense 2. Obs.

\P 1649 FULLER  \textit{Just Man's Fun.} 11 Rabshakeh pretended a Commission from God‥and complements blasphemie.    
\P 1654 JER. TAYLOR  \textit{Real Pres.} 26 He cannot escape the Inquisition unlesse he complement the Church, and with a civility tell her that she knows better.    
\P 1661 A. MARVELL  \textit{Corresp. Lett.} 21 II. 55 Monsieur Du Plessis‥ is come ouer from them to complement his Majesty.    
\P 1700 SIR W. CALVERLEY  \textit{Note-bk.} (Surtees) 92 Sir John sent‥to complement them for their kindness.    
\P 1710  \textit{Life Bp. Stillingfleet} 84 Ready‥to strike with the Deists, to complement and cajole them.    
\P 1711 HEARNE  \textit{Collect.} III. 205 He complements me for my Ed. (most accurate Edition he calls it) of Leland's Itin.

\itembf{b.} to complement away, out of: see compliment v. 2 b.

\P 1640 NABBES  \textit{Bride} iii. ii, As if the enterteinment‥were not chargeable enough, but you must complement away wine and sweet meats.    
\P 1645 FULLER  \textit{Good Th. in Bad T. Hist. Appl.} vi. 101 Cæsar complemented his life away.    
\P 1665 \textit{Ch. Hist.} vi. iii. 308 King Henry his smiles complemented the former out of their Houses.    
\P 1697 COLLIER  \textit{Ess. Mor. Subj.} i. (1709) 231 Lest Church-Men should Complement away the Usefulness and Authority of their Calling; they would do well to decline superlative Observance.    
\P 1715 M. DAVIES  \textit{Athen. Brit.} i. 129 To shorten disputes‥and so complement them out of their Heresies.

\itembf{4.} to complement (a person) with (something): to present him with it as a mark of courtesy. Obs. (Now compliment, sense 4.)

\P 1697 W. DAMPIER  \textit{Voy.} (1698) I. xii. 328 He may be‥complemented‥with Tobacco and Betel-nut.    
\P 1732 in L'pool Munic. Rec. (1886) II. 92 That the Right Honole Hugh Lord Willoughby‥be complemented with his freedom.

\noindent
Hence COMPLEMENTING vbl. n. and ppl. a. = complimenting.

\P 1626 W. SCLATER  \textit{Expos.} 2 Thess. (1629) 74 All Complementings with Idolaters.    
\P 1649 MILTON  \textit{Eikon.} xx. (1851) 481 God, who stood neerer then hee for complementing minded, writ down those words.    
\P 1658 \textit{Whole  Duty Man} v. §22. 47 It's but a kind of formal complementing.    
\P 1704 J. BLAIR in \textit{W. Perry Hist. Coll. Amer. Col.} Ch. I. 94 They had refused to sign a complementing address.
\end{myenumerate}


%%%%%%%%%%%%%%%%%%%%%%%%%%%%%%%%%
\myitem{compliant} a. and n.

\noindent \phonetic{(kəmˈplaɪənt)}

\noindent [f. comply v. + -ant; after defiant, etc.]
\vspace{-0.3cm}

\begin{myenumerate}

\itembf{A.} adj.

\itembf{1.} Complying, disposed to comply; ‘civil, complaisant’ (J.); ready to yield to the wishes or desires of others.

\P 1642 LD. DIGBY in Clarendon \textit{Hist. Reb.} iv. (1843) 173/2 If after all‥he shall betake himself to the easiest and compliantest ways of accommodation.    
\P 1679 BURNET  \textit{Hist. Ref.} 71 The King did not doubt but the Pope would be compliant to his desires.    
\P 1828 SCOTT  \textit{F.M. Perth} vii, The rest will be compliant to the same resolution.    
\P 1870 DISRAELI  \textit{Lothair} xlii. 217, I do not like to be churlish when all are so amiable and compliant.    
\P 1874 GREEN  \textit{Short Hist.} iv. §2 (1882) 172 Their representatives‥proved far more compliant with the royal will than the barons.

\itembf{2.} Yielding to physical pressure, plaint. Obs.

\P 1667 MILTON  \textit{P.L.} iv. 3 Nectarine Fruits, which the compliant boughes Yeilded them.    
\P 1788 SMEATON  \textit{Quadrant} in \textit{Phil. Trans.} LXXIX. 6 The whole being slender and compliant, except in point of length.    
\P 1793 \textit{Edystone L.} §302 Wood wedges‥being more supple, elastic, and compliant than wedges of metal.

\itembf{B.} n. One who complies; a complier. Obs.

\P 1655 FULLER  \textit{Ch. Hist.} xi. VI. 314 It being a compliant with the papists, in a great part of their service, doth not a little confirm them in their superstition and idolatry.
\P a1661 \textit{Worthies} i. 331 His sturdy nature would not bow to Court-compliants.    
\P 1660 Z. CROFTON  \textit{Fast. St. Peter's Fetters} 37 Our Soft Covenanters, Speedy Complyants, and Temporizing Turn-Coats.
\end{myenumerate}


%%%%%%%%%%%%%%%%%%%%%%%%%%%%%%%%
\myitem{concomitant} a. and n.

\noindent \phonetic{(kənˈkɒmɪtənt)}

\noindent [ad. L. concomitānt-em, pr. pple. of concomitāri to accompany, go with: see concomitate.]
\vspace{-0.3cm}

\begin{myenumerate}

\itembf{A.} adj. Going together, accompanying, concurrent, attendant. Const. with (†of, †to).

\P 1607 TOPSELL  \textit{Serpents} (1653) 611 From the natural concomitant quality of heat, with exspiration, respiration, and inspiration.    
\P 1621 BURTON  \textit{Anat. Mel.} i. ii. ii. iv, Either concomitant, assisting, or sole causes‥of melancholy.    
\P 1651 CARTWRIGHT  \textit{Cert. Relig.} i. 166 That which was secret, yet was concomitant of that which was publike.    
\P 1711 STEELE  \textit{Spect.} No. 104 \cardo{⁋}1 So certainly is Decency concomitant to Virtue.    
\P 1799 KIRWAN  \textit{Geol. Ess.} 373 The concomitant lime⁓stone also contains marine petrifactions.    
\P 1856 MILL  \textit{Logic} I. 449 The law‥admits of corroboration by the Method of Concomitant Variations.    
\P 1864 BOWEN  \textit{Logic} x. (1870) 333 Every event has‥a crowd of concomitant circumstances.

\itembf{B.} n.

\itembf{1.} An attendant state, quality, circumstance, or thing; an accompaniment.

\P 1605 BACON  \textit{Adv. Learn.} i. viii. 42 Virgill did excellently‥couple the knowledge of causes, and the conquest of all fears, together as Concomitantia.]    
\P 1621 BURTON  \textit{Anat. Mel.} ii. iii. v, Death is not so terrible in it selfe, as the concomitants of it.    
\P 1682 NORRIS  \textit{Hierocles} 14 This reverence of an Oath is‥the constant attendant and concomitant of Piety.    
\P 1709 PRIOR  \textit{Paulo Purganti}, And for Tobacco (who could bear it?) Filthy Concomitant of Claret.    
\P 1750 JOHNSON  \textit{Rambl.} No. 79 \cardo{⁋}7 Suspicion is justly appointed the concomitant of guilt.    
\P 1846 PRESCOTT  \textit{Ferd. \& Is.} I. i. 96 Wealth with its usual concomitants, elegance and comfort.

\itembf{2.} A person that accompanies; a companion.

\P 1645 HOWELL  \textit{Lett.} I. i. xx, You are thus my concomitant through new places.    
\P 1651  \textit{Reliq. Wotton.} 81 [He] made him the chief concomitant of his heir apparant.    
\P 1698  \textit{Phil. Trans.} XX. 242 His Concomitants and Assistants in the Operations.    
\P 1794 SULLIVAN  \textit{View Nat.} II, I find this person often introduced as a concomitant of Psuche.

\itembf{3.} Math. (See quot.)

\P 1853 SYLVESTER in  \textit{Phil. Trans.} CXLIII. i. 543 Concomitant. Nomen generalissimum for a form invariantively connected with a given form or system of forms.    
\P 1859 SALMON  \textit{Higher Algebra} (1866) 104 Dr. Sylvester uses the name concomitant as a general word to include all functions whose relations to the quantic are unaltered by linear transformation, and he calls the functions now under consideration mixed concomitants.
\end{myenumerate}


%%%%%%%%%%%%%%%%%%%%%%%%%%%%%%%%
\myitem{concrete} a. and n.

\noindent \phonetic{(ˈkɒnkriːt)}

\noindent [ad. L. concrēt-us, pa. pple. of concrēscĕre to grow together: see concrescence. Cf. F. concret, -ète, 16th c. -ette. The stress has long been variable; \phonetic{conˈcrete}, the original mode, was given by Walker, and is used in verse by Lowell; \phonetic{ˈconcrete} was used by Chapman in 1611, and recognized by Johnson: the latter appears to be now the more frequent in the adj., and is universal in the n. B. 3.

   The frequent antithesis of concrete and discrete, appears to be influenced by a notion that the word represents L. concrētus, pa. pple. of concernĕre, in the same way as discrete is derived from L. discernĕre, discrētus.]
\vspace{-0.3cm}

\begin{myenumerate}

\itembf{A.} adj. (The earliest instances appear to be participial.)

\itembf{1. a.}  United or connected by growth; grown together. Obs.

\P 1471 RIPLEY  \textit{Comp. Alch.} in \textit{Ashm.} (1652) 112 For all the parts‥be Coessentiall and concrete.    
\P 1650 BULWER  \textit{Anthropomet.} x. (1653) 170 Men, that have monstrous Mouths, and some with concreate lips.

\itembf{b.} Continuous. In Acoustics applied to a sound or movement of the voice sliding continuously up or down; distinguished from discrete movement.

\P 1651 W. G. tr.  \textit{Cowel's Inst.} 60 The second manner of gaining, which‥is a discreet or distinct increase, or secretly a Concrete or continued. Whatsoever is born or comes from any sort of animalls under our Subiection or power are absolutely gained unto us.

\itembf{2.} Made up or compounded of various elements or ingredients; composite, compound. ? Obs.

\P 1536 LATIMER  \textit{2nd Serm. bef. Convoc.} i. 40 A thing concrete, heaped up and made of all kinds of mischief.    
\P 1850 W. IRVING  \textit{Goldsmith} v. 81 This concrete young gentleman, compounded of the pawn-broker, the pettifogger, and the West Indian heir.

\itembf{3.} Formed by union or cohesion of particles into a mass; congealed, coagulated, solidified; solid (as opposed to fluid). \textbf{a.} as pple.; \textit{b.} as adj.

\P 1533 ELYOT  \textit{Cast. Helthe} iv. (R.), Those same vapours‥be concrete or gathered into humour superfluous.    
\P 1567 J. MAPLET  \textit{Gr. Forest} Pref., Of the seconde sort is the Pumelse, concrete of froth.    
\P 1691 RAY  \textit{Creation} (1714) 323 Before it was concrete into a stone.

\P 1605 TIMME  \textit{Quersit.} i. xiii. 58 In all metalls and concrete bodies.    c 
\P 1611 CHAPMAN  \textit{Iliad} xi. (R.), Even to the concrete bloud That makes the liver.    
\P 1712 tr. \textit{Pomet's Hist. Drugs} I. 216 Scammony is a concrete resinous Juice.    
\P 1800 tr. \textit{Lagrange's Chem.} I. 74 One portion appears fluid and the other concrete.    
\P 1836 TODD  \textit{Cycl. Anat.} I. 51/2 Formed of blood scarcely concrete.    
\P 1854 HOOKER  \textit{Himal. Jrnls.} I. i. 16 The seeds too, yield a concrete oil.

\itembf{4. a.} Applied by the early logicians and grammarians to a quality viewed (as it is actually found) concreted or adherent to a substance, and so to the word expressing a quality so considered, viz. the adjective, in contradistinction to the quality as mentally abstracted or withdrawn from substance and expressed by an abstract noun: thus white (paper, hat, horse) is the concrete quality or quality in the concrete, whiteness, the abstract quality or quality in the abstract; seven (men, days, etc.) is a concrete number, as opposed to the number 7 in the abstract. concrete science (science 4 b).

   Afterwards concrete was extended also to substantives involving attributes, as fool, sage, hero, and has finally been applied by some grammarians to all substantives not abstract, i.e. all those denoting ‘things’ as distinguished from qualities, states, and actions. The logical and grammatical uses have thus tended to fall asunder and even to become contradictory; some writers on Logic therefore disuse the term concrete entirely: see quot. 1887. In this Dictionary, concr. is prefixed to those senses in which substantives originally abstract come to be used as names of ‘things’; e.g. crossing vbl. n., i.e. abstract n. of action, concr. a crossing in a street, on a railway, etc.
From an early period used as a quasi-n., a concrete (sc. term).

[\P 1581 J. BELL  \textit{Haddon's Answ. Osor.} 118 b, Turnyng awry, that is to say: From the Concreto to the Abstractum (to use here the termes of Sophistry).]

\P 1528 SKELTON  \textit{Bouge of Courte} (R.), A false abstracte cometh from a false concrete.    
\P 1594 BLUNDEVIL  \textit{Exerc.} i. xvi. (ed. 7) 41 Understand, that of numbers some are said to be abstract, and some concrete.    
\P 1614 SELDEN  \textit{Titles Hon.} 117 To expresse them by Abstracts from the Concret of their qualitie‥As Maiestie, Highnes, Grace.    
\P 1657 J. SMITH  \textit{Myst. Rhet.} A viij b, The concrete signifies the same form with those qualities which adhere to the subject: The concrete is the Adjective.    
\P 1690 LOCKE  \textit{Hum. Und.} iii. viii. §1 Our Simple ideas have all Abstract, as well as Concrete Names: the one whereof is (to speak the language of grammarians) a ‘substantive’, the other an ‘adjective’; as whiteness, white.    
\P 1725 WATTS  \textit{Logic} i. iv. §5 Concrete terms, while they express the quality, do also either express, or imply, or refer to some subject to which it belongs‥But these are not always noun adjectives‥a fool, a knave, a philosopher, and many other concretes are substantives.    
\P 1846 MILL  \textit{Logic} i. ii. §4 A concrete name is a name which stands for a thing; an abstract name is a name which stands for an attribute of a thing.    
\P 1851 MANSEL  \textit{Proleg. Log.} v. (1860) 144.    
\P 1854 H. SPENCER in  \textit{Brit. Q. Rev.} July 148 Let us observe how the relatively concrete science of geometrical astronomy, having been thus far helped forward by the development of geometry in general, reacted upon geometry, caused it also to advance, and was again assisted by it.    
\P 1864 BOWEN  \textit{Logic} iv. (1870) 88 The peculiar or proper appellation of a lower Concept or individual is called its concrete name.    
\P 1865 J. S. MILL  \textit{Comte} 33 The concrete sciences‥concern themselves only with the particular combinations of phaenomena which are found in existence.    
\P 1876 MASON  \textit{Eng. Gram.} §35 Abstract nouns are sometimes used in the concrete sense‥Thus nobility frequently means the whole body of persons of noble birth.    
\P 1876 JEVONS  \textit{Elem. Logic} (1880) 21 The reader should carefully observe that adjectives are concrete, not abstract.    
\P 1887 FOWLER  \textit{Deduct. Logic} i. i. (ed. 9) 15 Nothing has been said above of the common distinction between abstract and concrete terms‥I have availed myself of the expression ‘abstract term’, but avoided, as too wide to be of practical service, the contrasted expression ‘concrete term’. Concrete terms include what I have called attributives, as well as singular, collective, and common terms.

\itembf{b.} Philos. concrete universal [universal n. 1], the individual, when regarded as something maintaining its identity through qualitative change or diversity, or as a unity or system or class of separate but identical particulars. Also transf.

\P 1865 J. H. STIRLING  \textit{Secret of Hegel} p. xi, As Aristotle, with considerable assistance from Plato, made explicit the abstract Universal that was implicit in Socrates,—so Hegel‥made explicit the concrete Universal that was implicit in Kant.    
\P 1874 W. WALLACE tr. \textit{Hegel's Logic} ix. 267 The Judgment of Necessity‥contains‥in the predicate, partly the substance or nature of the subject, the concrete universal, the genus.    
\P 1876 F. H. BRADLEY  \textit{Eth. Stud.} v. 147 The good will‥is a concrete universal, because it not only is above but is within and throughout its details, and is so far only as they are.    
\P 1883 \textit{Princ. Logic} i. vi. 175 The concrete particular and the concrete universal both have reality, and they are different names for the individual.    
\P 1912 B. BOSANQUET  \textit{Princ. Individuality} ii. 38 A macrocosm constituted by microcosms, is the type of the concrete universal.    
\P 1920 M. T. COLLINS  \textit{Mod. Concept. Nat. Law} 95 A thing, a person, an act—anything—is only seen in its true nature when it is grasped as an organized unity, as a synthesis of the manifold. So far as it is a whole, it is a concrete universal.    
\P 1948 \textit{Poetry}  LXXIII. 159 Concrete universal, a concept, continuous in literary criticism, which implies the paradoxical union in a poem of the concrete, specific, and individual, together with the universal and general. The concrete universal persists among the New Critics.

\itembf{5.} Hence, generally, Combined with, or embodied in matter, actual practice, or a particular example; existing in a material form or as an actual reality, or pertaining to that which so exists. Opposed to abstract. (The ordinary current sense.)

   Absolutely, the concrete, that which is concrete; in the concrete, in the sphere of concrete reality, concretely.

\P 1648 MILTON  \textit{Tenure Kings} Wks. 1738 I. 314  These Apostles, whenever they give this Precept, express it in terms not concrete, but abstract, as Logicians are wont to speak.]    
\P 1656 HOBBES  \textit{Liberty, Necess., \& Ch.} (1841) 135 This‥is a metaphysical entity abstracted from the matter, which is better than non-entity‥But in the concrete it is far otherwise.    
\P 1710 BERKELEY  \textit{Princ. Hum. Knowl.} §97 Time, place, and motion, taken in particular or concrete.    
\P 1789 BURKE  \textit{Corr.} (1844) III. 114 It is with man in the concrete;—it is with common‥human actions, you are to be concerned.    
\P 1837 CARLYLE  \textit{Fr. Rev.} III. iii. i, But, quitting these somewhat abstract considerations, let History note the concrete reality which the streets of Paris exhibit.    
\P 1848 LOWELL  \textit{Fable for Critics}, ‘At slavery in the abstract my whole soul rebels, I am as strongly opposed to 't as any one else.’ ‘Ay, no doubt, but whenever I've happened to meet With a wrong or a crime, it is always concrete.’    
\P 1876 M. ARNOLD  \textit{Lit. \& Dogma} 234 note, The most concrete and unmetaphysical of languages.    
\P 1880 W. WALLACE  \textit{Epicureanism} 172 Their idea of this original matter was concrete and sensuous.

\itembf{6.} Made of concrete. [attrib. of B. 3.]

\itembf{7. a.} concrete music [tr. F. musique concrète]: a form of music constructed by the arrangement of various recorded sounds into a sequence. (Also with first word in French form concrète.)

\P 1953 \textit{Musical  Amer.} 15 Jan. 6/3 This method of basing a musical composition on fragmentary sounds, existing concretely, characterizes what Schaeffer has labeled concrete music.    
\P 1954 \textit{Gramophone  Record Rev.} Apr. 297 Concrete music is assembled rather than composed.    
\P 1954  \textit{Times Lit. Suppl.} 3 Dec. 778/4 The very latest thing‥Concrete Music, the term adopted for the French musique concrète, which is really synthetic electrophonics.    
\P 1958 \textit{Observer}  22 June 15/3 The music, an airborne plunking that deserves a less earthbound epithet than concrète, is by John Addison.

\itembf{b.} concrete poetry: a form of poetry in which the significance and the effect required depend to a larger degree than usual upon the physical shape or pattern of the printed material. Also ellipt. concrete. Hence concretist, concrete poem, concrete poet, etc.

  The term was coined independently and almost simultaneously in Brazil and Germany: in Brazil (poesia concreta) by the Noigandres group of poets; in Germany (die konkrete Dichtung) by Eugen Gomringer. The usage was formally adopted at a meeting in 1955 between the two originators. 

\P 1958 \textit{Pilot  Plan for Concrete Poetry} in M. E. Solt \textit{Concrete Poetry} (1970) 72 Concrete poem communicates its own structure: structure-content.‥ Concrete Poetry aims at the least common multiple of language.    
\P 1966 \textit{Isis}  16 Feb. 2/1 The Concrete poet tries to investigate language and the materials of which it is composed in a depth which he cannot achieve using conventional syntax.    Ibid. 9/1 The early ‘concretists’ were interested in setting words in isolation on the page.    Ibid. 9/2 His sensitivity led him to ‘concrete’ as a means of overcoming the deterioration language suffers through overexposure.    
\P 1966 \textit{Camb.  Rev.} 28 May 448/1 ‘Concrete’ poets ignore the traditional boundaries between word and image.    
\P 1967 S. BANN  \textit{Concrete Poetry} 17 He has recently contributed the pure Concrete ‘cube-poem’ to the Brighton Festival.    Ibid., The links between the early socially committed concrete poems and the ‘popcrete’ poems of Augusto de Campos.    Ibid. 24 His first contact with the Concrete movement, however, was with the Brazilians.    
\P 1968 \textit{Artes  Hispanicas} I. iii. 7/2 There is a fundamental requirement which the various kinds of concrete poetry meet: concentration upon the physical material from which the poem or text is made.

\itembf{B.} n.

\itembf{1.} quasi-n. a concrete, the concrete: see A. 4, 5.

\P 1528-1725 [See  A. 4].    
\P 1697 J. SERJEANT  \textit{Solid Philos.} 91 Entity is often us'd as a Concrete for the Thing it self.    
\P 1830 MACAULAY  \textit{Ess.}, Bunyan, Bunyan is almost the only writer who ever gave to the abstract the interest of the concrete.

\itembf{2.} gen. A concrete or concreted mass, a concretion, compound; a concrete substance. Also fig. (Obs. in lit. sense, exc. as in next.)

\P 1656 J. SERJEANT tr. \textit{T. White's Peripatet. Inst.} 361 The sun is a concrete of combustible matter.    
\P 1657 G. STARKEY  \textit{Helmont's Vind. Ep. to Rdr.}, The specifick excellency that is in any concrete of the whole vegetable family.    
\P 1706 PHILLIPS  (ed. Kersey) s.v., Antimony is a Natural Concrete, or a Mix'd Body compounded in the Bowels of the Earth; and Soap is a Factitious Concrete, or a Body mix'd together by Art.    
\P 1804 ABERNETHY  \textit{Surg. Observ.} 9 Thus an unorganized concrete becomes a living tumour.    a 
\P 1831 A. KNOX  \textit{Rem.} (1844) I. 63 That‥concrete of truth and error, of greatness and meanness‥the Roman Catholic Church.

\itembf{3.} spec. a.B.3.a A composition of stone chippings, sand, gravel, pebbles, etc., formed into a mass with cement; used for building under water, for foundations, pavements, walls, etc. armoured concrete = reinforced concrete. Often attrib. Also in comb. as concrete mixer (so -mixing); concrete paver; concrete-press, a machine for compressing concrete into blocks.

\P 1834  \textit{Lond. Archit. Mag.} I. 35 Making an artificial foundation of concrete (which has lately been done in many places).    
\P 1836 G. GODWIN in  \textit{Trans. Inst. Brit. Archit.} 12 The generic term concrete‥perhaps, can only date from that period when its use became general and frequent, probably not longer than 15 or 20 years ago.    
\P 1858 GLENNY  \textit{Gard. Every-day Bk.} 25/1 Paving with brick, tile, stone, or concrete.    
\P 1906 \textit{Concrete}  Mar. p. ii, Armoured Concrete Constructions.    
\P 1906 \textit{Westm.  Gaz.} 20 Sept. 9/3 An extensive installation of stone-breaking and concrete-mixing machinery is in full work.    
\P 1907  \textit{Daily Mail} 22 Oct., Armoured concrete, reinforced concrete, concrete-steel, or ferro-concrete.    
\P 1909 \textit{Cent.  Dict.} Suppl., Concrete-mixer, a machine for mixing cement, sand, crushed or broken stone, and water in varying proportions for making concrete.    
\P 1929 W. HEYLIGER  \textit{Builder of Dam} 33 A one-bag power concrete mixer.    
\P 1930 \textit{Engineering}  7 Mar. 324/1 The concrete-mixing plant is said to be the largest in Canada.    
\P 1954 \textit{Gloss.  Highway Engin. Terms (B.S.I.)} 49 Concrete paver, a concrete mixer capable of moving on crawler tracks or rails and provided with a boom and bucket for depositing the concrete in the required position in a pavement.

attrib. \P 1881 DARWIN  \textit{Form. Veg. Mould} 181 The junction of the concrete floor with the walls.

\textit{b.} Paving made of concrete.

\P 1911 E. FERBER  \textit{Dawn O'Hara} ii. 13 No tramping of restless feet on the concrete all through the long, noisy hours.
\end{myenumerate}


%%%%%%%%%%%%%%%%%%%%%%%%%%%%%%%%%
\myitem{conduit} n.

\noindent \phonetic{(ˈkʌndɪt, ˈkɒndɪt)}

\noindent [A particular application of the word conduct (OF. conduit, med.L. conductus in same sense), formerly having all the three type-forms conduit, condit (cundit), conduct; but, while in the other senses the Latin form conduct has prevailed, in this the French form conduit is retained, and the pronunciation descends from the ME. form condit or cundit.]
\vspace{-0.3cm}

\begin{myenumerate}
\itembf{1. a.} An artificial channel or pipe for the conveyance of water or other liquids; an aqueduct, a canal. (In Sc. in the form cundie commonly applied to a covered drain, not a tile drain.)

\noindent α \P 1340  \textit{Ayenb.} 91 Þise uif wytes byeþ ase uif condwys.    
\P 1382 WYCLIF  \textit{Ecclus.} xxiv. 41 As water kundute [1388 cundit].
\P 1385 CHAUCER  \textit{L.G.W.} 852 Tisbe, As water, whanne the conduyte broken ys.    
\P 1570 B. GOOGE  \textit{Pop. Kingd.} ii. 21 b, The Conduites runne, within continually.    
\P 1611 CORYAT  \textit{Crudities} 27 Conduits of lead, wherein the water shal be conueighed.    
\P 1704 ADDISON  \textit{Italy} (1733) 215 Conduits Pipes and Canals that were made to distribute the Waters.    
\P 1812  \textit{Act 52 Geo.} III, c. 141 §43 in Oxf. \& Camb. Enactm. 125 A certain Conduit called Hobsen's Conduit.    
\P 1833  \textit{Act 3-4 Will.} IV, c. 46 §116 The pipes or other conduits‥used for the conveyance of gas.    
\P 1864 A. MCKAY  \textit{Hist. Kilmarnock} (ed. 3) 274 Roads having side-drains and cross conduits.    
\P 1883 PARKES  \textit{Pract. Hygiene} (ed. 6) 25 Open conduits are liable to be contaminated by surface washings.

\noindent β \P 1382 WYCLIF  \textit{2 Sam.} ii. 24 Thei camen to the hil of the water kundit.    
\P 1382 \textit{1 Kings} xviii. 32 He beeldide vp an auter‥and he made a water cundid.    
\P 1387 TREVISA  \textit{Higden} (Rolls) I. 221 A greet condyt [aquæductum].
\P c1400 MANDEVILLE  v. (1839) 47 Þere is no water to drynke, but ȝif it come be condyt from Nyle [Roxb. vii. 24 in cundites fra the riuer].
\P c1400  \textit{Rom. Rose} 1414 Stremis  smale, that by devise Myrthe had done come through condise.    
\P 1432-50 tr. \textit{Higden} (Rolls) I. 181 Floode Danubius flowethe‥in condettes vnder the erthe.
\P c1450 \textit{Nominale} in  Wr.-Wülcker 733/40 Hic aqueductus, a cundyth undyr the erthe.    
\P 1541  \textit{Act} 33 Hen. VIII, c. 35 (heading) An acte concerning the condites at Gloucester.    
\P 1587 \textit{Bristol  Wills} (1886) 249 To the yerely Repayringe of the Cundyte of the said parishe.

\noindent γ \P 1491 WRIOTHESLEY  \textit{Chron.} (1875) I. 2 A conduict begun at Christ Churche.    
\P 1607 NORDEN  \textit{Surveyors Dial.} 85, I see the Conducts are made of earthen pipes, which I like farre better then them of Leade.    
\P 1642 PERKINS  \textit{Prof. Bk.} i. 49 A Pipe in the land to convey the water to my manour in a Conduct.

\itembf{b.} Electr. A tube or trough for receiving and protecting electric wires; a length or stretch of this. Also attrib., esp. in connection with the conduit system (see quot. 1940).

\P 1882 \textit{U.S.  Pat.} 266,916 My invention consists, first, in making an electric conduit, comprising an external casing, internal conductor pipes, and supporting diaphragms, of conducting material, so that any electric currents induced in the said pipes will be conducted‥directly to the ground.    
\P 1884 \textit{Cassell's  Fam. Mag.} Jan. 127/1 Conduits for holding electric wires laid along the streets.    
\P 1894  \textit{Daily News} 2 June 5/4 At Buda-Pesth, where the conduit electrical system is in such successful operation.    
\P 1894 \textit{Cassier's  Mag.} Sept. 385/1 A trial of the conduit on a commercial basis at Washington.    Ibid. 385/2 The open slot conduit with a continuous, bare trolley wire.    Ibid. 386/2 The contact or working conductors could readily be placed in a slotted conduit, or trough.    Ibid., The road at Blackpool, England,—an open conduit road.    Ibid. 387/1 The Love conduit system.    
\P 1896  \textit{Daily News} 17 Dec. 5/2 The electric power is conveyed from the conduit rail to the car by means of a small peculiarly-shaped conductor.    
\P 1899  \textit{Ibid.} 9 Jan. 3/6 New York will soon have 150 miles of conduit.    
\P 1903  \textit{Daily Chron.} 18 Nov. 3/5 A conduit line from Vauxhall Bridge to the Clapham-road.    
\P 1908 \textit{Installation  News} II. 47/2 Three parallel lengths of 3/4 in. Simplex conduit hung a few inches below the ceiling and seven feet apart.    
\P 1940  \textit{Chambers's Techn. Dict.} 189/1 Conduit box, a box adapted for connexion to the metal conduit used in electric wiring schemes.    Ibid., Conduit system, (1) a system of wiring‥in which the conductors are contained in a steel conduit; (2) a system of current collection used on some electric tramway systems.    
\P 1941 S. R. ROGET  \textit{Dict. Electr. Terms} (ed. 4) 69/1 Conduit Fittings, accessories such as conduit boxes, bends, tees, couplers, etc., for joining lengths of conduit tube for wiring.    
\P 1955 \textit{Oxf.  Jun. Encycl.} XI. 131/2 With one method of wiring, separate stranded copper wires with VIR insulation are used, the wires being placed inside black enamelled steel pipes, called ‘conduits’. The conduits are screwed together and joined to cast iron boxes containing the switches and connexions between the wires; the whole conduit system is then joined to earth.

\itembf{2. a.}  A structure from which water is distributed or made to issue; a fountain. Obs. or arch.

\noindent α \P 1430 LYDG.  \textit{Bochas} i. xiv. (1554) 30 a, Like a conduit gushed out the bloude.    
\P 1480 CAXTON  \textit{Chron. Eng.} clxi. 144 Oute of the conduyt of chepe ran whyte wyn and rede.    
\P 1568 GRAFTON  \textit{Chron.} II. 426 They newe buylded in the same place a fayre Conduyt, which at this day is called the Conduyt in Cornehyll.    
\P 1611 CORYAT  \textit{Crudities} 334 In the middle of the Court there is an exceeding pleasant Conduite that spowteth out water in three degrees one aboue another.    
\P 1774 WARTON  \textit{Hist. Eng. Poetry} III. xxvi. 154 On the conduit without Ludgate, where the arms and angels had been refreshed.    
\P 1871 ROSSETTI  \textit{Poems, Dante at Verona} xxviii, The conduits round the garden sing.

fig. \P 1645 HEYWOOD  \textit{Fort. by Land \& Sea} i. i, See you not these purple conduits run, Know you these wounds?

\noindent β \P 1400 \textit{Morte  Arth.} 201 Clarett and Creette, clergyally rennene, With condethes fulle curious alle of clene siluyre.
\P c1400 MANDEVILLE  xx. (1839) 217 Þei that ben of houshold, drynken at the condyt.
\P c1530 LD. BERNERS  \textit{Arth. Lyt. Bryt.} (1814) 139 At the foure corners of this bedde there were foure condytes‥out of the whiche there yssued so sweet an odour and so delectable.    
\P 1556 \textit{Chron.  Gr. Friars} (Camden) 27 At the condyd in Graschestret, the condet in Cornelle‥at the lyttyll condyd‥ronnynge wyne, rede claret and wythe.

\noindent γ \P 1533 \textit{Anne  Boleyn's Coronation} in Furniv. \textit{Ballads fr. MSS.} I. 393 At the conducte in Cornehill was exhibited a Pageaunte of the three Graces.    
\P 1538 LELAND  \textit{Itin.} II. 70 There is a Conduct in the Market Place.

\itembf{b.} ? A laver or large basin. Obs.

\P 1500 \textit{Will  of J. Ward} (Somerset Ho.), My grete lavatory of laton called a Condyte.    
\P 1592 R. D. tr.  \textit{Hypnerotomachia} 6 Great lauers, condites, and other infinite fragments of notable woorkmanship.

\itembf{3.} transf. Any natural channel, canal, or passage; †\itembf{a.} in the animal body (obs.); \itembf{b.} (19th c.) in geological or geographical formations; = canal 2, channel 6.

\noindent α \P 1340  \textit{Ayenb.} 202 Zuo þet o stream of tyeares yerne be þe condut of þe eȝen.    
\P 1483 CAXTON  \textit{De la Tour} L iij b, Wyn taken ouer mesure‥stoppeth the conduytes of the nose.    
\P 1561 HOLLYBUSH  \textit{Hom. Apoth.} 38 a, For thys drincke mollifieth it [the bladder] openeth the condute.    
\P 1578 LYTE  \textit{Dodoens} iv. lxxx. 544 It doth also stoppe the pores and conduites of the skinne.    
\P 1607 T. WALKINGTON  \textit{Opt. Glass} viii. (1664) 100 The Conduits of the Spirits, and the Arteries and Veins.    
\P 1774 GOLDSM.  \textit{Nat. Hist.} (1862) I. i. i. 269 The conduit that goes to the third stomach.    
\P 1830 R. KNOX  \textit{Béclard's Anat.} 88 The secretion of the fat‥is not performed in glands or in particular conduits.    
\P 1839 MURCHISON  \textit{Silur. Syst.} i. ix. 126 A subterranean conduit or eruptive channel by which the volcanic matter was protruded to the surface.    
\P 1862 DANA  \textit{Man. Geol.} 693.

\noindent β \P 1513 DOUGLAS  \textit{Æneis} xii. ix. 17 The stif swerd‥Persit his cost and breistis cundyt in hy.    
\P 1587 L. MASCALL  \textit{Govt. Cattle, Sheep} (1627) 249 In the condite of the teat.

\noindent γ \P 1536 BELLENDEN  \textit{Cron. Scot.} (1821) I. p. xlv, [The Sea-] hurcheon‥havand bot ane conduct to purge thair wambe and ressave thair meit.    
\P 1578 LYTE  \textit{Dodoens} i. xxxvii. 56 The juyce‥openeth the conductes of the nose.    
\P 1649 LOVELACE  \textit{Poems} 56 The sacred conduicts of her Wombe.

\itembf{4.} fig. The channel or medium by which anything (e.g. knowledge, influence, wealth, etc.) is conveyed; = canal 7, channel 8.

\noindent α \P 1540 COVERDALE  \textit{Fruitf. Lesson} i, Here are opened the conduits and well-pipes of life, the way of our health.
\P a1600 HOOKER  \textit{Eccl. Pol.} vi. iv. §15 Conduits of irremediable death to impenitent receivers.    
\P 1690 LOCKE  \textit{Hum. Und.} iii. xi. (1695) 290 Language being the great Conduit, whereby Men convey‥Knowledge, from one to another.    
\P 1737 WATERLAND  \textit{Eucharist} 290 Sacraments are‥his appointed Means or Conduits, in and by which He confers his Graces.    
\P 1818 HALLAM  \textit{Mid. Ages} (1841) I. iii. 303 These republics‥became the conduits through which the produce of the East flowed in.    
\P 1878 MORLEY  \textit{J. De Maistre Crit. Misc.} 99 Reaching people through those usual conduits of press and pulpits.

\noindent β \P 1651 JER. TAYLOR  \textit{Clerus Dom.} 53 The spirit‥running still in the first channels by ordinary conducts.    
\P 1670 \textit{Moral  State Eng.} 18 The addresses of the people to their Sovereign‥being convey'd through him as a conduct.

\itembf{5.} Arch. \textbf{a.} gen. A passage (obs.). \textbf{b.} spec. see quot. 1875.

\P 1624 WOTTON \textit{Archit.} in  \textit{Reliq. Wotton} (1672) 33 Doors, Windows, Stair-cases, Chimnies, or other Conducts.    
\P 1703 T. N. \textit{City  \& C. Purch.} 7.    
\P 1875 GWILT  \textit{Archit. Gloss.}, Conduit (Fr.), a long narrow walled passage underground, for secret communication between different apartments.

\itembf{6.} The leading (of water) by a channel. Obs.

\P 1555  \textit{Fardle Facions} Pref. 10 Thei deriued into cities‥the pure freshe waters‥by conduicte of pipes and troughes.

\itembf{7.} Mus. A short connecting passage, a codetta.

\P 1872 H. C. BANISTER  \textit{Music} §404 By a short passage --- Conduit‥it [the Motivo] is again returned to.    
\P 1880 OUSELEY in  \textit{Grove Dict. Mus.} I. 568/1. (See copula.)

\itembf{8.} Comb., as conduit-cock, conduit-like, conduit-water, adj. or adv.; conduit-head, a reservoir; = conduit 2; also fig.; †conduit-water, spring water; conduit-wise adv. Also conduit-pipe.

\P 1600 HEYWOOD  \textit{1st Pt. Edw. IV}, Wks. 1874 I. 10 We'le  take the tankards from the *conduit-cocks To fill with ipocras.

\P 1509 HAWES  \textit{Past. Pleas.} iv. iii, A fountayne‥A noble sprynge, a ryall *conduyte hede.    
\P 1607 DEKKER  \textit{Wh. Babylon} Wks. 1873 II. 244  Conduit-heads of treason.

\P 1580 SIDNEY  \textit{Arcadia} (1622) 141 Those saphir-coloured brookes Which *conduit-like with curious crookes, Sweet Ilands make.

\P 1545 T. RAYNALDE  \textit{Byrth Mankynde} (1564) 68 Holyoke sodden in *cunduite water.    
\P 1594 PLAT  \textit{Jewell-ho.} ii. 28 A glasse of conduit water.

\P 1611 CORYAT  \textit{Crudities} 9 A little chappell made *conduitwise.
\end{myenumerate}


%%%%%%%%%%%%%%%%%%%%%%%%%%%%%%%%
\myitem{congenital} a.

\noindent \phonetic{(kənˈdʒɛnɪtəl)}

\noindent [mod. f. L. congenit-us (see congenite) + -al1. So F. congénital, admitted into the 6th ed. of the Academy's Dictionary in 1835. The sense was formerly expressed by congenial, Fr. congénial.]
\vspace{-0.3cm}

\begin{myenumerate}

\itembf{a.} Existing or dating from one's birth, belonging to one from birth, born with one. \textbf{a.} techn. in Pathol. (as a congenital disease or congenital defect).

\P 1796 A. DUNCAN  \textit{Annals Med.} I. 20 Bronchocele‥is not often congenital.    
\P 1807 S. COOPER  \textit{First Lines Surg.} 387 Congenital hernia.    
\P 1856 SIR B. BRODIE  \textit{Psychol. Inq.} I. v. 181 The mind of an individual who labours under congenital blindness‥cannot fail to be imperfect.    
\P 1878 T. BRYANT  \textit{Pract. Surg.} I. 365 Ordinary congenital cataract.

\itembf{b.} in Bot.

\P 1862 DARWIN  \textit{Fertil. Orchids} vii. 315 The so-called congenital attachment of the pollinia by their caudicles.

\itembf{c.} in general use. Const. with.

\P 1848 KINGSLEY  \textit{Saint's Trag.} iv. i, The mind of God, revealed In laws, congenital with every kind And character of man.    
\P 1852 H. ROGERS  \textit{Ess.} I. vii. 374 Notions, coeval with the mind in date, congenital with its very faculties.    
\P 1852 BLACKIE  \textit{Stud. Lang.} 2 The living process of nature acting by congenital, divinely-implanted instinct.    
\P 1866 KINGSLEY  \textit{Lett.} (1878) II. 242 The congenital differences of character in individuals.    
\P 1879 M. ARNOLD  \textit{Mixed Ess.} 69 The French people, with its congenital sense for the power of social intercourse and manners.
\end{myenumerate}


%%%%%%%%%%%%%%%%%%%%%%%%%%%%%%%%
\myitem{consensus} n.

\noindent \phonetic{(kənˈsɛnsəs)}

\noindent [a. L. consensus agreement, accord, sympathy, common feeling, f. consens- ppl. stem of consentīre: see consent. Used in the physiological sense by Bausner, De consensu partium humani corporis, 1556, whence sense 1 in mod.F. and English.]
\vspace{-0.3cm}

\begin{myenumerate}

\itembf{1.} Phys. General agreement or concord of different parts or organs of the body in effecting a given purpose; sympathy. Hence transf. of the members or parts of any system of things.

\P 1854 G. BRIMLEY  \textit{Ess.}, Comte 320 In the universe‥he resolves to see only a vast consensus of forces.    
\P 1861 GOLDW.  SMITH \textit{Lect. Mod. Hist.} 24 There is a general connexion between the different parts of a nation's civilization; call it, if you will, a consensus, provided that the notion of a set of physical organs does not slip in with that term.    
\P 1870 H. SPENCER  \textit{Princ. Psychol.} I. ii. ix. 278 A mutually-dependent set of organs having a consensus of functions.

\itembf{2. a.} Agreement in opinion; the collective unanimous opinion of a number of persons.

\P 1861  \textit{Sat. Rev.} 21 Dec. 637 Bishop Colenso is‥decidedly against what seems to be the consensus of the Protestant missionaries.    
\P 1880 \textit{Athenæum}  10 Apr. 474/3 A consensus had actually been arrived at on the main features involved.

transf. \P 1884 H. A. HOLDEN  \textit{Plutarch's Themist.} 190 The consensus of [the MSS.] ABC leaves no room for doubt about a reading.

\itembf{b.} Also consensus of opinion, authority, testimony, etc.

\P 1858  \textit{Sat. Rev.} V. 287/1 Supported by a great consensus of very weighty evidence.    
\P 1874 H. R. REYNOLDS  \textit{John Bapt.} v. i. 289 Sustained by a great consensus of opinion.

\itembf{3.} attrib. and Comb.

\P 1966  \textit{New Statesman} 21 Oct. 583/3 The essence of consensus politics is directly related to consensus communications.    
\P 1967 \textit{Listener}  3 Aug. 136/2 Consensus journalism—with its millionaire proprietors and its multitudinous advertising departments—has been praised for being politically permissive.    
\P 1968 \textit{Peace  News} 10 May 10/2 Cicero‥is a more dubious case—an unsuccessful consensus-politician, if ever there was one.
\end{myenumerate}


%%%%%%%%%%%%%%%%%%%%%%%%%%%%%%%%
\myitem{consortium} n.

\noindent \phonetic{(kənˈsɔːʃɪəm, kənˈsɔːtɪəm)}

\noindent [L. consortium partnership, f. consors consort. Thence It. consorzio and OF. consorce.]
\vspace{-0.3cm}

\begin{myenumerate}

\itembf{1.} Partnership, association. Now more specifically, an association of business, banking, or manufacturing organizations.

\P 1829  \textit{Edin. Rev.} L. 89 If the consortium give pleasure to the shades of these good people, we must acquiesce in it.    
\P 1881 H. A. WEBSTER in \textit{Encycl. Brit.} XIII. 466/2 (Italy) The law [of 1874] united the six banks into a consorzio or union, bound, if required, to furnish to the national exchequer bank-notes to the value of 1,000,000,000 lire manufactured and renewed at their common expense; but by the law of 7th April 1881‥the consortium of the banks came to a close on the 30th June 1881, and the consortial notes actually current are formed into a direct national debt.    
\P 1930 \textit{Time \& Tide}  30 Aug. 1086 The  bankers have formed a consortium to help rationalize industry.    
\P 1936  \textit{Nature} 4 July 5/1 Preference was given for the execution of the work by consortia of landowners or public bodies.    
\P 1957  \textit{New Scientist} 12 Sept. 31/1 Spokesmen for two of the consortia that tendered for the first CEA nuclear power stations.    
\P 1961  \textit{Listener} 28 Dec. 1110/2 An interesting development‥is the formation of a Yorkshire consortium of local authorities. The chairmen of the housing committees of Sheffield, Hull, and Leeds have announced that they are going to co-ordinate their housing programmes.    
\P 1962 H. E. BEECHENO  \textit{Business Stud.} xii. 107 Recently there have been several cases of manufacturers with connected interests forming a consortium in order to get large overseas contracts for capital developments.    
\P 1963  \textit{Ann. Reg.} 1962 282  The formation of aid consortia for Turkey and Greece. The Turkish consortium was formed on 31 July.

\itembf{2.} Law. (The right of) association and fellowship between husband and wife.

  The action for loss of consortium was abolished by the Administration of Justice Act, 1982 (c. 53) § 2. 

\P 1658 H. GRIMSTON  tr. \textit{Second Pt. Rep. Sir G. Croke} 501 Trespass of Assault and Battery: for that the Defendant‥assaulted and beat the wife of the Plaintiff, per quod consortium uxoris suæ for three days amisit.    
\P 1768 BLACKSTONE  \textit{Comm.} III. viii. 140 The third injury is that of beating a man's wife.‥ If the‥husband is deprived for any time of the company and assistance of his wife, the law then gives him a‥remedy by an action upon the case for this ill-usage, per quod consortium amisit.]    
\P 1836 in  W. C. Curteis \textit{Rep. Cases in Doctors' Commons} (1840) I. 198 Mr. Sherwood would have a right to claim the consortium of his wife.    
\P 1861 \textit{Law  Times Rep.} V. 293/1 Consortium‥necessarily includes the idea of a union of two persons, each of whom is the consort of the other.    
\P 1932 \textit{Law  Rep. King's Bench Div.} II. 512 It seems‥clear that at the present day a husband has a right to the consortium of his wife, and the wife to the consortium of her husband.    
\P 1957 M. TURNER-SAMUELS  \textit{Law of Married Women} i. 11 It was held by the House of Lords‥that the right of a husband to damages for loss of consortium against a person who negligently injures his wife is an anomaly at the present day.    
\P 1971 R. A. PERCY  \textit{Charlesworth on Negligence} (ed. 5) iii. 72 The same principle would apply to an action by a husband suing for the loss of consortium of his wife, since he has been deprived of her services.

\itembf{3.} transf. and fig. Any association or collection.

\P 1964 E. HUXLEY  \textit{Back Street New Worlds} xii. 122 As you enter, you are engulfed in a consortium of odours in which dried and pickled fish predominate.    
\P 1975  \textit{New Yorker} 24 Nov. 58/2 The ice was cracked, if not broken, by the publication of such books as ‘The American Soldier’‥by a consortium of academics led by Samuel A. Stouffer, of Harvard.    
\P 1979 J. GRIMOND  \textit{Memoirs} viii. 128 The consortium of Majors present, of whom I was one, decided we had better ring up our superiors.
\end{myenumerate}


%%%%%%%%%%%%%%%%%%%%%%%%%%%%%%%%
\myitem{consummate} a.

\noindent \phonetic{(kənˈsʌmət, ˈkɒnsəmət)}

\noindent [ad. L. consummāt-us brought to the highest degree, perfect, complete, consummate, pa. pple. of consummāre (see next). As to pronunciation, see the vb.]
\vspace{-0.3cm}

\begin{myenumerate}

\itembf{A.} as pa. pple.

\itembf{1.} Completed, perfected, fully accomplished. Obsolescent.

\P 1471 RIPLEY  \textit{Comp. Alch.} i. in Ashm. (1652) 133 And alsoe thy Bace perfytly consummate.    
\P 1530 PALSG.  495/2 This worke that hath ben so longe in hande is nowe at the laste consommate.    
\P 1615 CHAPMAN  \textit{Odyss.} xiii. 284 Till righteous fate Upon the Wooers' wrongs were consummate.
\P a1626 BP. ANDREWES  \textit{Serm.} (1661) 9 a, Consummate it shall be, but not yet.    
\P 1752 YOUNG  \textit{Brothers} iii. i, Guilt, begun, must fly To guilt consummate, to be safe.    
\P 1767 BLACKSTONE  \textit{Comm.} II. 128 The husband by the birth of the child becomes tenant by the curtesy initiate‥but his estate is not consummate till the death of the wife.    
\P 1832 AUSTIN  \textit{Jurispr.} (1879) I. vi. 330 A fraction of a community already consummate or complete.

\itembf{2.} Of marriage: = consummated. Obs.

\P 1530 in Fiddes  \textit{Life Wolsey} (1726) ii. 171 The Matrymonie was consummate by that Act.    
\P 1599 SHAKES.  \textit{Much Ado} iii. ii. 2, I doe but stay till your marriage be consummate.    
\P 1649 BP. HALL  \textit{Cases Consc.} iv. v. 434 Not ratified onely, but consummate by carnal knowledge.    
\P 1765 BLACKSTONE  \textit{Comm.} I. 435 Marriages contracted‥in the face of the church, and consummate with bodily knowledge.

\itembf{B.} adj.

\itembf{1.} Summed up, finished; having in it finality.

\P 1430 tr.  \textit{T. à Kempis} 107 Holde a short and a consummate worde: Leve all \& þou shalt finde all; forsake couetynge and þou shalt finde rest.

\itembf{2.} Complete, perfect: \textbf{a.} of things. arch.

\P 1527 R. THORNE in  Hakluyt \textit{Voy.} (1589) 257 There lacke many thinges that a consummate Carde [= map] should haue.    
\P 1667 MILTON  \textit{P.L.} v. 481 Last the bright consummate floure Spirits odorous breathes.    
\P 1743 FIELDING  \textit{J. Wild} i. i, A perfect or consummate pattern of human excellence.    
\P 1868 M. PATTISON  \textit{Academ. Org.} v. 191 In Oxford‥degrees in arts were not final or consummate degrees, but steps on the road‥ to the doctor's degree.

\itembf{b.} of persons: Complete; accomplished, supremely qualified.

\P 1643 MILTON  \textit{Divorce} ii. iii. (1851) 69 What a consummat and most adorned Pandora was bestow'd upon Adam.    
\P 1725 POPE  \textit{Odyss.} iv. 283 Form'd by the care of that consummate sage.    
\P 1758 CHESTERFIELD  \textit{Lett.} IV. 126 The dignity and importance of a consummate Minister.    
\P 1789 BELSHAM  \textit{Ess.} I. xvi. 304 Those consummate generals, Condé, Turenne, and Luxemburg.    
\P 1848 MACAULAY  \textit{Hist. Eng.} II. 50 The consummate hypocrite.    
\P 1878 BROWNING  \textit{Poets Croisic} 67 Step thou forth Second consummate songster!

\itembf{3.} Perfect, of the highest degree or quality; supreme; utmost. Usually of qualities, or states, as consummate bliss, skill, wisdom, etc.

\P 1526  \textit{Pilgr. Perf.} (W. de W. 1531) 231 b, To knowe the god omnipotent is the consummate iustyce.    
\P 1644 MILTON  \textit{Areop.} 56 The most consummat act of his fidelity.    
\P 1695 WOODWARD  \textit{Nat. Hist. Earth} ii. (1723) 94 The most consummate and absolute Order and Beauty.    
\P 1704 HEARNE  \textit{Duct. Hist.} (1714) I. 406 A consummate skill in Arithmetic.    
\P 1725 WATTS  \textit{Logic} ii. v. §4 Consummate folly.    
\P 1805 WORDSW.  \textit{Prelude} iv. (1889) 259/1 That day consummate happiness was mine.    
\P 1855 MACAULAY  \textit{Hist. Eng.} IV. 271 Conducted with consummate ability.    
\P 1876 M. DAVIES  \textit{Unorth. Lond.} 371 It was a consummate sermon.    
\P 1880 BEACONSFIELD  \textit{Endym.} lxxiii. 340 Little dinners, consummate and select.

\itembf{4.} ? = consumed 2, consumpt. Obs.

\P 1684 tr.  \textit{Bonet's Merc. Compit.} viii. 298 Lixivia [in dropsy]‥are proper‥but not‥for such as are consummate, and make a red deep coloured urine.
\end{myenumerate}


%%%%%%%%%%%%%%%%%%%%%%%%%%%%%%%%
\myitem{consummate} v.

\noindent \phonetic{(ˈkɒnsəmeɪt, ˈkɒnsjʊ-, kənˈsʌmeɪt)}

\noindent [f. prec., or L. consummāt-, ppl. stem of consummāre to sum up, make up, complete, finish, f. con- altogether + summa sum, summus highest, utmost, supreme, extreme, etc. The ppl. adj. consummate was in earlier use than the vb., and after the latter came into use, continued for some time to be used as its pa. pple., until succeeded in this capacity by consummated. The pronunciation \phonetic{conˈsummate} is given in all the dictionaries until within the last few years, but \phonetic{ˈconsummate} is now prevalent: see contemplate. With this stress-pattern the second syll. is freq. (\phonetic{sjʊ}). In the adj. \phonetic{conˈsummate} is still usual, though \phonetic{ˈconsummate} is often said.]
\vspace{-0.3cm}

\begin{myenumerate}

\itembf{1.} trans. To bring to completion or full accomplishment; to accomplish, fulfil, complete, finish.

\P 1530 PALSG.  495/2, I consommate, I make a full ende of a thyng, je consumme.    
\P 1580 LYLY  \textit{Euphues} (Arb.) 450 [This] brought greater desire to them, to consumate them.    
\P 1595 SHAKES.  \textit{John v.} vii. 95 To consummate this businesse happily.    
\P 1610 \textit{Histrio-m.}  i. 214 The Sunne heere riseth in the East with us‥And so hee consummates his circled course In the Ecliptick line.    
\P 1632 tr.  \textit{Bruel's Praxis Med.} 399 This disease is consummated and brought to its full ripenes in 24 houres.    
\P 1692 RAY  \textit{Dissol. World} 25 God also consummated the Universe in six days.    
\P 1725 POPE  \textit{Odyss.} xx. 18 And let the Peers consummate the disgrace.    
\P 1798 SOUTHEY  \textit{Wife of Fergus} Poems II. 108 As if I knew not what must consummate My glory!    
\P 1835 BROWNING  \textit{Paracelsus} ii. 48 This done‥to perfect and consummate all‥I would supply all chasms with music.    
\P 1837 THIRLWALL  \textit{Greece} IV. xxx. 158 Lysander was eager to consummate his victory.

\itembf{b.} To make an end of, or put an end to, by doing away with. Obs.

\P 1634 SIR T. HERBERT  \textit{Trav.} 135 Arbela, where he [Darius] consummated life and monarchie.
\P a1649 CHAS. I \textit{Wks.} 292 What more speedy way was there to consummate those distractions then by a personal treaty.    
\P 1649 FULLER  \textit{Just Man's Fun.} 24 God would‥consummate this miserable world, put a period to the dark night.

\itembf{2.} To complete marriage by sexual intercourse.

\P 1540  \textit{Act 32 Hen. VIII}, c. 25 Your maieste‥maie‥contract and consummat matrimonie wyth any woman.    
\P 1709 STEELE  \textit{Tatler} No. 11 \cardo{⁋}5 Prince Nassau‥consummated on the 26th of the last Month his Marriage with the beauteous Princess of Hesse-Cassel.    
\P 1766 GOLDSM.  \textit{Vic. W.} xxxi, Her aunt‥had insisted that her nuptials with Mr. Thornhill should be consummated at her house.    
\P 1823 LINGARD  \textit{Hist. Eng.} VI. 202 That the marriage between Arthur and Catharine had been consummated.

\itembf{b.} absol.

\P 1748 H. WALPOLE  \textit{Corr.} (1837) I. 128 They consummated at her house.    
\P 1762 SCRAFTON  \textit{Indostan} (1770) 17 They are married in their infancy; and consummate at fourteen on the male side, and ten or eleven on the female.    
\P 1771 \textit{Contemplative  Man} I. 27 Her Highness was obliged to consummate at a lonely‥Cottage, to avoid being discovered.

\itembf{3.} To make perfect; to perfect. Obs.

\P 1535 GOODLY  \textit{Prymer} (1834) 165 After they are consummate in all kind of virtue.]    
\P 1582 N. T. (RHEM.)  \textit{Heb.} v. 9 Being consummated, he became, to all that obey him, the cause of eternal salvation.    
\P 1678 A. LOVELL tr. \textit{La Fontaine's Mil. Duties Cavalry} 79 Consummated in the experience of War.

\itembf{4.} intr. (for refl.) To fulfil or perfect itself.

\P 1839 BAILEY  \textit{Festus} (1848) p. xvi, From the first These things were fixed, and are and aye shall be Consummating.    
\P 1844 MRS. BROWNING  \textit{Vis. Poets}, Room‥for new hearts to come Consummating while they consume.
\end{myenumerate}


%%%%%%%%%%%%%%%%%%%%%%%%%%%%%%%%
\myitem{contentious} a.

\noindent \phonetic{(kənˈtɛnʃəs)}

\noindent [ad. F. contentieux:—L. contentiōsus given to contention, quarrelsome: see contention and -ous.]
\vspace{-0.3cm}

\begin{myenumerate}

\itembf{1.} Of persons or their dispositions: Given to contention; prone to strife or dispute; quarrelsome.

\P 1533 FRITH  \textit{Answ. More} (1829) 445 That you accept this worke with‥no contentious hart.    
\P 1611 BIBLE  \textit{Prov.} xxi. 19 It is better to dwell in the wildernesse, then with a contentious and an angry woman.    
\P 1682 BURNET  \textit{Rights Princes} i. 13 If two or three out of a contentious humour opposed it.    
\P 1732 BERKELEY  \textit{Alciphr.} v. §19 The most contentious, quarrelsome, disagreeing crew.    
\P 1853 MACAULAY  \textit{Biog. Atterbury} (1867) 14 His despotic and contentious temper.

\itembf{b.} transf.

\P 1605 SHAKES.  \textit{Lear} iii. iv. 6 Thou think'st 'tis much that this contentious storme Inuades vs to the skin.    
\P 1610 \textit{Temp.} ii. i. 118.    
\P 1695 BLACKMORE  \textit{Pr. Arth.} i. 455 She makes contentious Winds forget their Strife.

\itembf{c.} Bellicose, warlike. Obs.

\P 1535 COVERDALE  \textit{2 Sam.} xxi. 20 And there arose yet warre at Gath, where there was a contencious man which had sixe fyngers on his handes.    ― 2 Kings xix. 25 That contencious stronge cities mighte fall in to a waist heap of stones.

\itembf{2.} Characterized by or involving contention.

\P 1430 tr.  \textit{T. à Kempis} 119 To stryue wiþ contenciose wordes.    
\P 1535 JOYE  \textit{Apol. Tindale} 49 To wryte any maliciouse and contenciouse pistle agenst him.    
\P 1647 \textit{Proposals  of Army in Neal Hist. Purit.} III. 412 The present unequal, and troublesome, and contentious way of ministers' maintenance by Tithes.    
\P 1751 JOHNSON  \textit{Rambler} No. 142 \cardo{⁋}8 A contentious and spiteful vindication.    
\P 1875 GLADSTONE  \textit{Glean.} VI. liii. 170 Forbearing to raise contentious issues.

\itembf{3.} Law. Of or pertaining to differences between contending parties. contentious jurisdiction: right of jurisdiction in causes between contending parties.

\P 1483 CAXTON  \textit{Gold. Leg.} 427/1 Wel letterd, as it apperyd sythe, as wel in contempcious jugemente as gyuyng counceyll to the sowles upon the fayte of theyr conscyence.    
\P 1727-51 CHAMBERS  \textit{Cycl.} s.v., The Lords Chief Justices, judges, etc. have a contentious jurisdiction.    
\P 1768 BLACKSTONE  \textit{Comm.} III. 65 Such ecclesiastical courts, as have only what is called a voluntary and not a contentious jurisdiction.    
\P 1875 STUBBS  \textit{Const. Hist.} I. 233 In contentious suits it is difficult to draw the line between judicial decision and arbitration.
\end{myenumerate}


%%%%%%%%%%%%%%%%%%%%%%%%%%%%%%%%
\myitem{context} n.

\noindent \phonetic{(ˈkɒntɛkst)}

\noindent [ad. L. contextus (u-stem) connexion, f. ppl. stem of contexĕre to weave together, connect (see above). Cf. mod.F. contexte (in Cotgr.).]
\vspace{-0.3cm}

\begin{myenumerate}

\itembf{1.} The weaving together of words and sentences; construction of speech, literary composition. Obs.

\P 1432-50 tr.  \textit{Higden} (Rolls) I. 5 In the contexte historicalle [contextu historico] the rewle off lyvenge and forme of vertues moralle‥ȝiffe grete resplendence thro the diligence of croniclers.
\P c1645 HOWELL  \textit{Lett.} (1650) I. 459 Since these kings there is little difference in the context of [the French] speech, but only in the choice of words, and softness of pronounciation.

\itembf{2.} concr. The connected structure of a writing or composition; a continuous text or composition with parts duly connected. Obs.

\P 1526  \textit{Pilgr. Perf.} (W. de W. 1531) 181 Though the aungell in the contexte of his salutacyon, expressed not this name Maria.    
\P 1531 ELYOT  \textit{Gov.} iii. xxv, The bokes of the Euangelistes, vulgarely called the gospelles, which be one contexte of an historie.    
\P 1633 H. GARTHWAITE  \textit{(title)}, The Evangelical Harmonie, reducing the Four Evangelists into one Continued Context.    
\P 1641 MILTON  \textit{Ch. Govt.} Pref. (1851) 95 That book within whose sacred context all wisdome is infolded.

fig. \P 1635 QUARLES  \textit{Embl.} ii. vi, The skillful gloss of her reflection But paints the context of thy coarse complexion.

\itembf{3.} The connexion or coherence between the parts of a discourse. Obs.

\P 1613 R. C. TABLE  \textit{Alph.} (ed. 3), Context, the agreeing of the matter going before, with that which followeth.    
\P 1622 M. FOTHERBY  \textit{Atheom.} Pref. 20, I haue‥hindered not the context, and roundnesse of the speech.    
\P 1641 J. JACKSON  \textit{True Evang. T.} ii. 141 The context, or alliance that the text hath with the protext, or verse immediately foregoing.

\itembf{4. a.} concr. The whole structure of a connected passage regarded in its bearing upon any of the parts which constitute it; the parts which immediately precede or follow any particular passage or ‘text’ and determine its meaning. (Formerly circumstance q.v. 1 c, quots. 1549, 1579.)

\P 1568 FULKE  \textit{Answ. Chr. Protestant} (1577) 84 When the articles following are spoken in one context and phrase.    
\P 1583 \textit{Defence} (Parker Soc.) 561 The whole context is this: ‘Let no man say,’ etc.    
\P 1631 R. BYFIELD  \textit{Doctr. Sabb.} 24 If it bee meant of‥thou, that were absonant from the‥context.
\P c1680 BEVERIDGE  \textit{Serm.} (1729) II. 1 That we may understand these words aright, it will be necessary to take a short view of the context.    
\P 1709 BERKELEY  \textit{Th. Vision} §73 A word pronounced with certain circumstances, or in a certain context with other words.
\P a1714 SHARP  \textit{Wks.} VII. xv. (R.), To this I answer plainly according to all the light that the contexts afford in this matter.    
\P 1849 COBDEN  \textit{Speeches} 46, I wish honourable gentlemen would have the fairness to give the entire context of what I did say, and not pick out detached words.    
\P 1883 FROUDE  \textit{Short Stud.} IV. iii. 294 A paragraph‥unintelligible from want of context.

\itembf{b.} transf. and fig.

\P 1842 H. E. MANNING  \textit{Serm.} (1848) I. i. 9 We carry on with us from day to day the whole moral context of the day gone by.    
\P 1853 RUSKIN  \textit{Stones Ven.} II. vi, It is literally impossible, without consulting the context of the building, to say whether the cusps have been added for the sake of beauty or of strength.    
\P 1877 E. CAIRD  \textit{Philos. Kant} ii. v. 281 The position of facts in the context of experience.

\itembf{c.} in this context: in this connexion.

\P 1873 R. CONGREVE  \textit{Ess.}, etc. (1874) 480, I should avail myself of the words of one of our number—not used in this context, but suiting my present purpose.

\itembf{5.} = contexture. Obs.

\P 1707 E. WARD  \textit{Hud. Rediv.} (1715) I. xvii, Sooner penetrate a Board, Than by a Cut or Thrust divide The Context of the stubborn Hide.    
\P 1766 R. GRIFFITH  \textit{Lett. Henry \& Frances} III. 274 The Union of Soul and Body‥that mistic Context.

\itembf{6.} attrib. and Comb., as context-theory; context-bound, context-free, context-sensitive adjs.

\P 1965 \textit{Language}  XLI. 506 Further, synonymy must be *context-bound.

\P 1957 J. PASSMORE  100 Yrs. Philos. i. 16 All nouns and all adjectives‥are *context-free names.    
\P 1959 I. DE SOLA  Pool \textit{Trends in Content Analysis} vii. 219 Context-free measurement of symbolic forms which are instrumentally manipulated is apt to be misleading.

\P 1964 \textit{Language}  XL. 317 A prosodic feature is one involved in a *context-sensitive phonological rule.    
\P 1965 N. CHOMSKY  \textit{Aspects of Theory of Syntax} i. 61 The theory of context-sensitive phrase-structure grammar‥probably does not fail in weak generative capacity.

\P 1936 J. R. KANTOR  \textit{Objective Psychol. of Gram.} ix. 116 The *context theory. According to this theory, what a word means depends upon its connection in past experience with some other thing.
\end{myenumerate}


%%%%%%%%%%%%%%%%%%%%%%%%%%%%%%%%
\myitem{contiguous} a.

\noindent \phonetic{(kənˈtɪgjuːəs)}

\noindent [f. L. contigu-us (see contigue) + -ous.]

\vspace{-0.3cm}

\begin{myenumerate}

\itembf{1.} Touching, in actual contact, next in space; meeting at a common boundary, bordering, adjoining. Const to, formerly also with.

\P 1611 CORYAT  \textit{Crudities} 81 Two seuerall Castles built on a rocke which are so neare together that they are euen contiguous.    
\P 1626 BACON  \textit{Sylva} §865 Water, being contiguous with aire, cooleth it, but moisteneth it not.    
\P 1644 EVELYN  \textit{Diary} 21 Apr., This [island] is contiguous to ye towne by a stately stone bridge.    
\P 1722 J. MACKY  \textit{Journ. thro' Eng.} I. 177 London and Westminster‥are now by their Buildings become contiguous, and in a manner united.    
\P 1750 JOHNSON  \textit{Rambler} No. 34 \cardo{⁋}3 An heiress whose land lies contiguous to mine.    
\P 1842 W. GROVE  \textit{Corr. Phys. Forces} 49 The hydrogen‥unites with the oxygen of the contiguous molecule of water.    
\P 1874 S. COX  \textit{Pilgr. Ps.} iii. 51 Long rows of contiguous houses.

\itembf{b.} Math. contiguous angles: = adjacent angles.

\P 1727-51 CHAMBERS  \textit{Cycl.} s.v., Contiguous angles‥are such as have one leg common to each angle; otherwise called adjoining angles.

\itembf{2.} Next in time or order, immediately successive.

\P 1612-15 BP. HALL  \textit{Contempl., N.T.} iii. i, The favours of our benificent Saviour were at the least contiguous. No sooner hath hee raised the centurion's servant from his bed, then hee raises the widowe's son from his beere.    
\P 1748 HARTLEY  \textit{Observ. Man} ii. iv. 402 Two great Events will fall upon two contiguous Moments of Time.

\itembf{3.} Coadjacent in experience or thought.

\P 1770 BEATTIE  \textit{Ess. Truth} ii. ii. §3 (R.) The fancy is determined by habit to pass from the idea of fire to that of melted lead, on account of our having always perceived them contiguous and successive.

\itembf{4.} Continuous, with its parts in uninterrupted contact. Obs.

\P 1715 LEONI tr.  \textit{Palladio's Archit.} (1742) I. 51 Instead of Pilasters, there is a contiguous Wall.    
\P 1725 DE FOE  \textit{Voy. round World} ii. 47 The notion of the Hills being contiguous, like a wall that had no gates.

\itembf{5.} loosely. Neighbouring, situated in close proximity (though not in contact). †Of persons: Dwelling near.

\P 1710 PRIDEAUX  \textit{Orig. Tithes App.} 25 Those Parishes, within five miles distance, may be served by a Contiguous Minister.    
\P 1779 FORREST  \textit{Voy. N. Guinea} 149 The island of Goram is said to have thirteen mosques‥Contiguous is a small island called Salwak.
\P a1853 ROBERTSON  \textit{Serm.} Ser. iii. ii. (1872) I. 22 It [the spirit of the world] is found in a different form in contiguous towns.
\end{myenumerate}


%%%%%%%%%%%%%%%%%%%%%%%%%%%%%%%%
\myitem{contretemps} n.

\noindent \phonetic{(kɑːn.trə.tɑ̃)}

\noindent [F. contre-temps, -tems, bad or false time, motion out of time, inopportuneness, unexpected and untoward accident.]
\vspace{-0.3cm}

\begin{myenumerate}

\itembf{1.} Fencing. A pass or thrust which is made at a wrong or inopportune moment. Obs.

\P 1684 R. H. \textit{Sch.  Recreat.} 60 Counter Temps‥is when you Thrust without a good Opportunity, or when you Thrust, at the same time your Adversary does the like.    Ibid. 67 This preserves your Face from your Adversaries scattering or Counter-Temps Thrusts.    
\P 1694 SIR W. HOPE  \textit{Swordsman's Vade M.} 43 It is a fair Thrust, and cannot be called a Contre temps.    
\P 1725 in  \textit{New Cant. Dict.}

\itembf{2. a.} An inopportune occurrence; an untoward accident; an unexpected mishap or hitch.

\P 1802 M. EDGEWORTH  \textit{Manœuvring} i, I am more grieved than I can express‥by a cruel contre-temps.    
\P 1842 T. MARTIN  \textit{My Namesake} in \textit{Fraser's Mag.} Dec., I am used to these little contretems.    
\P 1872 J. L. SANFORD  \textit{Estimates Eng. Kings} 397 He [Charles II] regarded such contretemps as inevitable.

\itembf{b.} A disagreement or argument; a dispute.

\P 1961 \textit{Providence}  (Rhode Island) Jrnl. 4 July 24/3 There also came a brief contretemps with the sound mixers who made the mistake of being overheard during a quiet moment.    
\P 1977  \textit{Washington Post} 27 Dec. b7/3 There is his ongoing relationship with Beverly Switzler‥and a contretemps with another duck named Donald.    
\P 1983 M. EDWARDES  \textit{Back from Brink} ii. 26 The Zambian President had had a particularly unpleasant contretemps with the Rhodesians, and was about to put up tariff barriers across the Zambesi.    
\P 1984  \textit{New Yorker} 30 Jan. 69/2 Mondale and Glenn got into a new contretemps, this one over acid rain and environmental policy in general.

\itembf{3.} Dancing. A step danced on the unaccented portion of the beat; spec. in Ballet (see quots. 1952 and 1957). 

\P 1706 J. WEAVER tr. \textit{Feuillet's Orchesography} 45 Of Contre⁓temps, or compos'd Hops.    
\P 1728 J. ESSEX tr. \textit{Rameau's Dancing-Master} xxxvii. 97 The Contretems are those springing Steps which give a Life to Dancing by the different Manners of their Performance;‥To make one with the right Foot‥sink upon the Left, and rise upon it with a Spring; but at the same Time the right Leg‥moves forwards‥on the Toes, both Legs well extended; afterwards make another Step forwards‥which makes the Contretems compleat.    
\P 1830 R. BARTON tr. \textit{Blasis's Code of Terpsichore} vi. 488 Any dancer may be capable of executing a chassé, a pas de bourrée, a contre-tems, \&c.    
\P 1877  \textit{Encycl. Brit.} VI. 801/1 As may be seen from the technical language of dancing (assemblée, jetée‥contre-temps‥) it has undoubtedly been brought to greatest perfection in France.    
\P 1952 KERSLEY \& SINCLAIR  \textit{Dict. Ballet Terms} 38 Occasionally one sees the full and more difficult contretemps in which the dancer closes the left leg behind the right as both knees bend before springing out.    
\P 1957 G. B. L.  WILSON \textit{Dict. Ballet} 79 Contretemps, a step in which the dancer, with the left foot behind and pointed, jumps off the right foot bringing the left foot round in a small sweep to the front, replacing the right foot. The right foot moves out to the side and the dancer moves forward, repeating the step.

\noindent
Hence CONTRETEMPS (-temp) v. nonce-wd. Fencing. (a) trans. To make a contretemps at; (b) intr. to make contretemps.

\P 1684 R. H. \textit{Sch.  Recreat.} 72 If for all this your Adversary give a home-thrust, then you must Counter-temps him in the Face, and parry‥with your left Hand.    
\P 1694 SIR W. HOPE  \textit{Swordsman's Vade M.} 42 He can infallibly Contre-temps with the Ignorant as often as he pleaseth. An Ignorant Contre-temping an Artist‥The Artist that contre-tempeth the Ignorant.    Ibid. 61 An Artist may‥be Contre-tempsd or Resposted.
\end{myenumerate}


%%%%%%%%%%%%%%%%%%%%%%%%%%%%%%%%
\myitem{contrite} a. (and n.)

\noindent \phonetic{(ˈkɒntraɪt)}

\noindent [a. F. contrit (12th c.), ad. L. contrīt-us bruised, crushed, pa. pple. of conterĕre, f. con- together + terĕre to rub, triturate, bray, grind.

   The pronunciation long varied between the original \phonetic{conˈtrite and ˈcontrite}; the former was still recognized by Johnson and used by some 18th c. hymn-writers. J. has also \phonetic{conˈtriteness}; Browning has \phonetic{conˈtritely}; on the other hand \phonetic{ˈcontrite} is found in Piers Ploughman. Depending on this is the prosodic choice between hearts \phonetic{conˈtrite and ˈcontrite} hearts.]
\vspace{-0.3cm}

\begin{myenumerate}

\itembf{1.} lit. Bruised, crushed; worn or broken by rubbing. Obs. rare.

\P 1651 JER. TAYLOR  \textit{Serm. for Year} i. xxvii. 345 Though their strengths are no greater than a contrite reed or a strained arme.    
\P 1656 BLOUNT  \textit{Glossogr.}, Contrite, worn or bruised; but is most commonly used for penitent or sorrowful for misdeeds, remorseful.    
\P 1755 JOHNSON,  \textit{Contrite}, bruised; much worn.

\itembf{2.} fig. Crushed or broken in spirit by a sense of sin, and so brought to complete penitence.

\P 1340 HAMPOLE  \textit{Psalter} cxlvi. 3 Þat helis þe contryte of hert.    
\P 1377 LANGL.  \textit{P. Pl.} B. xiv. 89 If man be inliche contrit.
\P c1380 WYCLIF  \textit{Sel. Wks.} II. 400 To assoile men þat ben contrit.    
\P 1447 O. BOKENHAM  \textit{Seyntys} (Roxb.) 102 Ful contryht and cleen shrevyn also.
\P c1450 \textit{Castle  Hd. Life St. Cuthb.} 3783 He helyd þaim wer contrite in hert.    
\P 1526  \textit{Pilgr. Perf.} (W. de W. 1531) 140 b, Be contryte and sory for your fall.    
\P 1549 (Mar.)  \textit{Bk. Com. Prayer} 30 b, Create and make in vs newe and contrite heartes.    
\P 1667 MILTON  \textit{P.L.} x. 1091 With  our sighs‥sent from hearts contrite, in sign Of sorrow unfeign'd, and humiliation meek.
\P a1745 SWIFT  \textit{Beasts' Conf. to Priest}, The swine with contrite heart allow'd His shape and beauty made him proud.    
\P 1819 MONTGOMERY  \textit{Hymn}, ‘Prayer’ v, Prayer is the contrite sinner's voice Returning from his ways.    
\P 1856 R. A. VAUGHAN  \textit{Mystics} (1860) I. 194 No ecclesiastical absolution can help us unless we are contrite for our sin before God.

\itembf{b.} Of actions, etc.: Displaying, or arising from, contrition.

\P 1593 SHAKES. \textit{Lucr.} 1727 Her  contrite sighs unto the clouds bequeathed Her winged sprite.    
\P 1599 \textit{Hen. V}, iv. i. 313, I Richards body haue interred new, And on it haue bestowed‥contrite teares.    
\P 1829 SOUTHEY  \textit{All for Love} vii, He raised this contrite cry.    
\P 1868 E. EDWARDS  \textit{Raleigh} I. xiii. 257 In very contrite and earnest words.

\itembf{3.} Comb., as contrite-hearted.

\P 1611 CORYAT  \textit{Crudities} 422 A penitent and contrite-hearted Christian.    
\P 1871 FREEMAN  \textit{Hist. Ess. Ser.} i. iv. 106 Turned from notorious sinners into contrite-hearted penitents.

\itembf{B.} quasi-n. A contrite person, a penitent.

\P 1600 HOOKER  \textit{Eccl. Pol.} vi. vi. §13 Such contrites intend and desire absolution, though they have it not.
\end{myenumerate}


%%%%%%%%%%%%%%%%%%%%%%%%%%%%%%%%
\myitem{contumacious} a.

\noindent \phonetic{(kɒntjuːˈmeɪʃəs)}

\noindent [f. L. contumāci- (contumāx); see contumax and -acious.]

\vspace{-0.3cm}

\begin{myenumerate}

\itembf{1.} Contemning and obstinately resisting authority; stubbornly perverse, insubordinate, rebellious. (Of persons and their actions.)

\P 1603 KNOLLES  \textit{Hist. Turks} (1621) 997 Their Turcoman nation‥were grown verie contumatious.    
\P 1655 FULLER  \textit{Ch. Hist.} ii. ii. §81 His contumacious Company-keeping (contrary to his Confessours command) with an Excommunicated Count.    
\P 1772 \textit{Hist.  Rochester} 127 To reduce the contumacious monks to obedience.    
\P 1829 I. TAYLOR  \textit{Enthus.} x. 291 That spirit of contumacious scrupulosity which is the parent of schism.

\itembf{b.} Of diseases: Not readily yielding to treatment, stubborn. Obs.

\P 1605 TIMME  \textit{Quersit.} iii. 152 Contumacious sicknesses.    
\P 1684 tr.  \textit{Bonet's Merc. Compit.} viii. 263 In contumacious Diseases.

\itembf{2.} Law. Wilfully disobedient to the summons or order of a court.

\P 1600 HOOKER  \textit{Eccl. Pol.} vi. iv. §1 Contumacious persons which refuse to obey their sentence.    
\P 1726 AYLIFFE  \textit{Parerg.} 190 He is in Law said to be a contumacious Person, who, on his Appearance afterwards, departs the Court without leave.    
\P 1823 LINGARD  \textit{Hist. Eng.} VI. 202 On her refusal to appear in person or by her attorney, she was pronounced contumacious.    
\P 1859 HAWTHORNE  \textit{Fr. \& It. Jrnls.} II. 282 Contumacious prisoners were put to a dreadful torture.

\noindent
Hence \phonetic{contuˈmaciously adv., contuˈmaciousness}.

\P 1626 J. PORY in  Ellis \textit{Orig. Lett.} i. 333 III. 243 They contumaciously refused to go.    
\P 1654 CODRINGTON tr.  \textit{Hist. Ivstine} 219 Having their contumaciousness punish'd with a Pestilence.    
\P 1675 tr.  \textit{Machiavelli's Prince} (Rtldg. 1883) 286 The clients are contumaciously litigious.    
\P 1676 WISEMAN  \textit{Surgery} i. xxv. (R.), The difficulty and contumaciousness of cure [of elephantiasis].    
\P 1841 MACAULAY  \textit{W. Hastings Ess.} (1854) II. 645 Imposing a fine when that assistance was contumaciously withheld.    
\P 1887 \textit{Spectator} 28 May 723 Various delays in deciding upon his contumaciousness.
\end{myenumerate}


%%%%%%%%%%%%%%%%%%%%%%%%%%%%%%%%
\myitem{conundrum} n.

\noindent \phonetic{(kəˈnʌndrəm)}

\noindent [Origin lost: in 1645 (sense 3) referred to as an Oxford term; possibly originating in some university joke, or as a parody of some Latin term of the schools, which would agree with its unfixed form in 17-18th c. It is doubtful whether Nash's use (sense 1) is the original.]
\vspace{-0.3cm}

\begin{myenumerate}

\itembf{1.} Applied abusively to a person. (? Pedant, crotchet-monger, or ninny.) Obs.

\P 1596 NASHE  \textit{Saffron Walden} 158 So will I‥driue him [Gabriel Harvey] to confesse himselfe a Conundrum, who now thinks he hath learning inough to proue the saluation of Lucifer.

\itembf{2.} A whim, crotchet, maggot, conceit. Obs.

\P 1605 B. JONSON  \textit{Volpone} v. ii, I must ha' my crotchets! And my conundrums!    
\P 1623 MASSINGER  \textit{Bondman} ii. iii, (Tipsy man says) I begin To have strange conundrums in my head.    
\P 1651 BEDELL  \textit{Life Erasm.} in \textit{Fuller's Abel Rediv.} 61 These conimbrums, whether Reall or Nominall, went downe with Erasmus like chopt hay.    
\P 1687 A. BEHN  \textit{Lucky Chance} ii. ii, I hope he'll chain her up, the Gad Bee's in his Quonundrum.    a 
\P 1700 B. E. \textit{Dict.  Cant. Crew}, Conundrums, Whimms, Maggots, and such like.    
\P 1706 ESTCOURT  \textit{Fair Examp.} iv. i, You don't know her; she has more Conuncrums in her Head than a Fencer.    
\P 1719 D'URFEY  \textit{Pills} IV. 140 My Blood she advances, With Twenty Quadundrums, and Fifty Five Fancies.

\itembf{3.} A pun or word-play depending on similarity of sound in words of different meaning. Obs.

\P 1645 \textit{Kingdom's  Weekly Post} 16 Dec. 76 This is the man who would have his device alwayes in his sermons, which in Oxford they then called conundrums. For an instance‥Now all House is turned into an Alehouse, and a pair of dice is made a Paradice, was it thus in the days of Noah? Ah no!
\P a1704 T. BROWN  \textit{Praise Poverty} Wks. (1730) I. 94 Pun and conundrum pass with them for wit.    
\P 1707 E. WARD  \textit{Hud. Rediv.} (1715) I. x, Such frothy Quibbles and Cunnunders.    
\P 1711 ADDISON  \textit{Spect.} No. 61 \cardo{⁋}2 A Clinch, or a Conundrum.    
\P 1726 AMHERST  \textit{Terræ Fil.} xxxix. (1741) 204 Plain sense was esteem'd nonsense from the pulpit, which rung with ambiguities and double meanings; the poor sinner was mightily awaken'd to his duty by a pretty pun, and oftentimes owed his salvation to a quibble or a conundrum.    
\P 1731 BAILEY  (ed. 5), \textit{Conundrum}, a quaint humourous Expression, Word, or Sentence.    
\P 1755-73 JOHNSON,  \textit{Conundrum}, a low jest; a quibble; a mean conceit: a cant word.    
\P 1794 GODWIN  \textit{Cal. Williams} 47 Zounds! sir, do not think to put any of your conundrums upon me.

\itembf{4.} A riddle in the form of a question the answer to which involves a pun or play on words: called in 1769 conundrumical question. \textbf{b.} Any puzzling question or problem; an enigmatical statement. 

\P 1790 WOLCOTT  (P. Pindar) \textit{Elegy to Apollo} Wks. (1812) II. 278 The Riddle and Conundrum-mongers cry Pshaw!    1806-7 J. Beresford Miseries Hum. Life (1826) iii. xxxviii, Exhausting your faculties‥in vain endeavours to guess at a‥conundrum.    
\P 1824 BYRON  \textit{Juan} xv. xxi.    
\P 1845 DISRAELI  \textit{Sybil} (1863) 191 ‘You speak in conundrums’, said Morley; ‘I wish I could guess them’.    
\P 1886 FROUDE  \textit{Oceana} ii. 32 The stars‥will be after Adam's race has ceased to perplex itself with metaphysical conundrums.

\itembf{5.} A thing that one is puzzled to name, a ‘what-d'ye-call-it’. rare.

\P 1817 SCOTT  \textit{Let.} 8 June in Lockhart, We are attempting no castellated conundrums to rival those Lord Napier used to have executed in sugar.    
\P 1858 HOGG  \textit{Life Shelley} II. xii. 396 In her plain cap, plain kerchief, and plaited conundrums, by which the female Friends are distinguished.

\itembf{6.} Comb., as conundrum-game, conundrum-making, conundrum-monger (see prec. 4), conundrum-party.

\P 1716 M. DAVIES  \textit{Athen. Brit.} III. Dissert. 32 Mr. Wood‥makes a Conundrum-Game with poor Cornaro's Daughter Su.    
\P 1792 W. ROBERTS  \textit{Looker-on} (1794) I. No. 20. 271 Conundrum parties.    Ibid. No. 20. 281 Leger-de-main, conundrum-making, and punning.

\noindent
Hence, \phonetic{coˈnundrumed}, grown crotchety, slightly crazed; \phonetic{conunˈdrumical} a., whimsical, fantastic, crotchety; also, of the nature of a conundrum (sense 4); \phonetic{coˈnundrumize} v. intr., to make conundrums.

\P 1628 FORD  \textit{Lover's Mel.} ii. ii, Mel. Am I stark mad? Trol. No, no, you are but a little staring. There's difference between staring and stark mad. You are but whimsied yet; crotcheted, conundrumed, or so.    
\P 1743 \textit{London  Mag.} 36 Of all the conundrumical Inconsistencies, and incoherent Images that ever arose from a sick Stomach and a weak Head.    
\P 1769 \textit{Town \& Country  Mag.} 1 Sept. 462/2 Answers to Mr. Wags connundrumical questions.    
\P 1836  \textit{New Monthly Mag.} XLVIII. 420 The conundrumizing of the said Billy‥set everybody making conundrums.    
\P 1839 L. BLANCHARD  \textit{Ibid.} LVI. 519 It was from you that he had the joke first, while you were conundrumizing for want of thought.
\end{myenumerate}


%%%%%%%%%%%%%%%%%%%%%%%%%%%%%%%%%
\myitem{co-opt} v.

\noindent \phonetic{(kəʊˈɒpt)}

\noindent [ad. L. cooptāre, f. co(m) together + optāre to choose. In L. strictly ‘to choose as a colleague, friend, or member of one's tribe or family’; sometimes also ‘to elect into a body’, otherwise than by its members. Cf. the earlier uses of co-optate, co-optation.]
\vspace{-0.3cm}

\begin{myenumerate}

\itembf{1.} trans. To elect into a body by the votes of its existing members.

\P 1651 HOWELL  \textit{Venice} 158 The favour they did him to co-opt him into the body of their Nobility.    Ibid. 183 He sufferd himself to be coopted into the Colledg of Cardinalls.    
\P 1724 \textit{Reg.  Trin. Coll., Dublin} in Fraser \textit{Life Berkeley} iv. (1871) 101 Dr. Clayton was admitted and co-opted Senior Fellow.    
\P 1860 W. G. CLARK  \textit{Vac. Tour} 17 A body of bravoes‥who co-opt into their body those who, by strength of arm and skill in the use of the stiletto, may have shown themselves worthy of the distinction.    
\P 1862  \textit{Sat. Rev.} XIV. 217/1 The claim of the existing Residentiaries to coopt to a vacancy.    
\P 1875 STUBBS  \textit{Const. Hist.} III. xx. 418 These eight co-opted two more, and these ten two more.    
\P 1881  \textit{Nature} XXIII. 292 He was co-opted a Senior Fellow‥[and] made Vice-Provost.

\itembf{2.} To absorb into a larger (esp. political) group; to take over or adopt (an idea, etc.). U.S.

\P 1969  \textit{Atlantic Monthly} Oct. 18/1 A Republican Party based in the ‘Heartland’ (Midwest), West, and South can and should co-opt the Wallace vote.    
\P 1970  \textit{New Yorker} 16 May 34/3 All too often, mere approval of their social and political concern has, in the jargon, co-opted their causes and deadened them.    
\P 1982  \textit{N.Y. Times} 22 Apr. a6/3 The argument has been, co-opt the left before it's too late.    
\P 1986 B. FUSSELL  \textit{I hear Amer. Cooking} iv. xvii. 315 As English as apple pie, colonists must have said before America co-opted the dish for its own.

\vspace{0.1cm} \noindent
Hence \phonetic{co-ˈopted, co-ˈopting} ppl. adjs.

\P 1875 SYMONDS  \textit{Renaiss. Italy} I. iii. 149 The Grand Council‥as a co-opting body, tended to become a close aristocracy.    
\P 1881  \textit{Times} 17 May 4/1 The Convocation of Canterbury‥by means of members of their own body and co-opted scholars and divines‥have completed one portion of the work.    
\P 1887  \textit{Q. Rev.} Jan. 176 Coopted trustees.
\end{myenumerate}


%%%%%%%%%%%%%%%%%%%%%%%%%%%%%%%%%
\myitem{copious} a.

\noindent \phonetic{(ˈkəʊpɪəs)}

\noindent [ad. L. cōpiōs-us plentiful, f. cōpia plenty: cf. F. copieux (16th c. in Littré).]
\vspace{-0.3cm}

\begin{myenumerate}

\itembf{1.} Furnished plentifully with anything; having or yielding an abundant supply of; abounding in; Obs. exc. as in copious sources, where it passes into 3.

\P 1387 TREVISA  \textit{Higden} (Rolls) II. 17 (Mätz.) Þe erþe of that lond is copious of metal ore.    
\P 1398 \textit{Barth. De P.R.} xiii. vii. (1495) 444 Eufrates‥is moost copyous in gemmes and precyous stones.    
\P 1432-50 tr.  \textit{Higden} (Rolls) I. 287 A copious londe, and habundant in marchaundise.
\P a1533 FRITH  \textit{Bk. agst. Rastell} (1829) 218 He is more copious in labours, in stripes above measure.    
\P 1594 SHAKES.  \textit{Rich. III}, iv. iv. 135.    
\P 1596 DALRYMPLE tr.  \textit{Leslie's Hist. Scot.} (1885) 14 A certane toune copious in citizenis.    
\P 1632 LITHGOW  \textit{Trav.} iii. (1682) 106 It is indifferent copious of all things necessary for humane life.    
\P 1720 GAY  \textit{Poems} (1745) I. 172 Newgate's copious market.    
\P 1784 COWPER  \textit{Task} vi. 162 Copious of flow'rs the woodbine, pale and wan.    
\P 1838 PRESCOTT  \textit{Ferd. \& Is.} (1846) I. Introd. 53 More copious sources of knowledge.

\itembf{2.} In pregnant sense: \itembf{a.} Abounding in information; full of matter.

\P 1500 \textit{Orol. Sap.} in  \textit{Anglia} X. 327 Þei þat bene copiose and habundant in þe letterere science.    
\P 1561 T. HOBY tr. \textit{Castiglione's Courtyer} i. H iv, Those studyes shall make him copyous.    
\P 1630 PRYNNE  \textit{Anti-Armin.} 102 Our learned Diuinity Professors are full and copious in this point.    
\P 1652 NEEDHAM tr.  \textit{Selden's Mare Cl.} 41 Touching which particular both the Canonists and Civilians are very copious.    1716-8 Lady M. W. Montague Lett. I. xxxviii. 149 This copious subject has drawn me from my description of the exchange.    
\P 1775 JOHNSON  \textit{Let. Mrs. Thrale} 20 July, You have two or three of my letters to answer, and I hope you will be copious and distinct, and tell me a great deal of your mind.    
\P 1868 GLADSTONE  \textit{Juv. Mundi} i. (1869) 13 The Iliad and Odyssey give a picture of the age to which they refer, alike copious and animated, comprehensive and minute.

\itembf{b.} Having a plentiful command of language for the expression of ideas. Obs.

\P 1430 LYDG.  \textit{Chron. Troy} ii. xvi, And of wordes wonder copyous.    
\P 1589 PUTTENHAM  \textit{Eng. Poesie} ii. (Arb.) 94 It is a signe that such a maker is not copious in his owne language.    
\P 1672 MARVELL  \textit{Reh. Transp.} i. 50 Our author seems copious, but is indeed very poor of expression.

\itembf{c.} Profuse in speech; diffuse or exuberant in style or treatment.

\P 1430 LYDG.  \textit{Stans Puer} 74 in \textit{Babees Bk.} (1868) 28 Be not to copiose [v.r. copious] of langage.    
\P 1528 MORE  \textit{Dialogue} i. xxiii. Wks. 153 She will waxe copious and chop logicke.    
\P 1710 STEELE  \textit{Tatler} No. 244 \cardo{⁋}2 When you see a Fellow watch for Opportunities for being Copious.    
\P 1732 BERKELEY  \textit{Alciphr.} iii. §15 Declaimers of a copious vein.    
\P 1851 THACKERAY  \textit{Eng. Hum.} iii. (1858) 112 A copious Archdeacon, who has the command of immense papers, of sonorous language.

\itembf{d.} Of a language: Having a large vocabulary.

\P 1549 \textit{Compl.  Scot.} Prol. 17 Oure scottis tong is nocht sa copeus as is the lateen tong.    
\P 1651 HOBBES  \textit{Leviath.} iv. xlvi. 379 French, English, or any other copious language.    1772-7 Sir W. Jones Poems, Ess. i. 172 Their language is‥the most copious, perhaps, in the world.

\itembf{3.} Existing in rich abundance; plentiful; abundant. Now chiefly used with ns. expressing production or supply, or in reference to quantity produced; with names of material substances, it is obs. or arch., but is used of literary materials.

\P 1382 WYCLIF  \textit{Acts} xxii. 6 In the mydday‥a copious liȝt schon aboute me.    
\P 1387 TREVISA  \textit{Higden} (Rolls) I. 399 There lyme is copious and slattes for house.    
\P 1414 BRAMPTON  \textit{Penit. Ps.} cix. 41 Oure raumsoun is ful copyous, For thou art redy thi grace to sende.    
\P 1486  \textit{Bk. St. Albans, Her.} C j b, If the coloure of the poynt be more copiose or gretter in thos armys.    
\P 1609 BIBLE  (Douay) \textit{1 Macc.} ix. 35 To desire‥that they would lend him their provision which was copious.    
\P 1667 MILTON  \textit{P.L.} vii. 325 Rose as in Dance the stately Trees, and spred Their branches hung with copious Fruit.    
\P 1691 RAY  \textit{Creation} i. (1704) 67 Sea-water, containing a copious Salt.    
\P 1732 ARBUTHNOT  \textit{Rules of Diet} 287 The copious Use of Vinegar.    
\P 1762 FALCONER  \textit{Shipwr.} i. 158 The copious produce of her fertile plains.    
\P 1794 SULLIVAN  \textit{View Nat.} I. 212 The moisture‥is quickly condensed‥and falls down in copious dews.    
\P 1838 T. THOMSON  \textit{Chem. Org. Bodies} 714 Diacetate of lead throws down a copious white precipitate.    
\P 1845  \textit{Florist's Jrnl.} 94 Which‥induces a more copious display of flowers.    
\P 1854 H. MILLER  \textit{Sch. \& Schm.} vi. (1857) 98 A clear and copious spring comes bubbling out at its base.    
\P 1860 TROLLOPE  \textit{Framley P.} i. 3 Her hair which was copious.    
\P 1866 ROGERS  \textit{Agric. \& Prices} I. xix. 455 The evidence collected is exceedingly copious.

\itembf{b.} Multitudinous, numerous. Obs.

\P 1382 WYCLIF  \textit{1 Macc.} x. 1 Kyng Demetrie‥gadride an oost ful copiouse.    
\P 1432-50 tr.  \textit{Higden} (Rolls) I. 321 The peple of hit is copious, of semely stature.    
\P 1609 BIBLE  (Douay) \textit{1 Macc.} v. 6 A strong band, and a copious people.    
\P 1715-20 POPE  \textit{Iliad} i. 534 To heap the shores with copious death.    
\P 1741 BETTERTON  \textit{Eng. Stage} vi. 82 We shall‥conclude with the Actions of the Hands, more copious and various than all the other Parts of the Body.

\itembf{4.} as adv. = copiously.

\P 1791 COWPER  \textit{Iliad} xvii. 104 And from his wide wound bleeding copious still.    
\P 1808 J. BARLOW  \textit{Columb.} ii. 397 Buried gold drawn copious from the mine.
\end{myenumerate}


%%%%%%%%%%%%%%%%%%%%%%%%%%%%%%%%%
\myitem{corollary} n.

\noindent \phonetic{(kɒˈrɒlərɪ, ˈkɒrələrɪ)}

\noindent [ad. L. corollārium money paid for a chaplet or garland, gratuity, corollary, properly neut. of adj. corollārius belonging to a chaplet, f. corolla a little crown or chaplet. With senses 3 and 4 cf. Cotgr. ‘Corolaire, a Corollarie; a surplusage, ouerplus, addition to, vantage aboue measure’.]
\vspace{-0.3cm}

\begin{myenumerate}

\itembf{1.} In Geom., etc. A proposition appended to another which has been demonstrated, and following immediately from it without new proof; hence gen. an immediate inference, deduction, consequence.

\P 1374 CHAUCER  \textit{Boeth.} iii. x. 91 As þise geometriens whan þei han shewed her proposiciouns ben wont to bryngen in þinges þat þei clepen porismes‥ryȝt so wil I ȝeue þe here as a corolarie or a mede of coroune.
\P c1449 PECOCK  \textit{Repr.} i. v. 25 Of whiche‥folewith ferther this corelarie.    
\P 1551 RECORDE  \textit{Pathw. Knowl.} ii. liii, Of this Theoreme dothe there folowe an other‥whiche you maye calle‥a Corollary vnto this laste theoreme.    
\P 1563-87 FOXE  \textit{A. \& M.} (1596) 467/2 The corolary or effect of this conclusion is, that, etc.    
\P 1661 BRAMHALL  \textit{Just Vind.} vi. 110 Where that Author infers as a corollary from the former proposition, That no edict of a Soveraign Prince can justifie Schisme.    
\P 1722 WOLLASTON  \textit{Relig. Nat.} ix. 214 This is but a corollary from what goes before.    
\P 1832 LYTTON  \textit{Eugene A.} i. v, That is scarcely a fair corollary from my remark.    
\P 1870 JEVONS  \textit{Elem. Logic} xv. 135 [They] are in fact corollaries of the first six rules.    
\P 1874 HELPS  \textit{Soc. Press.} xvii. 239 There are corollaries to all axioms.

transf. \P 1828 HAWTHORNE  \textit{Fanshawe} vi, The lady of the house (and, as a corollary, her servant girl).

\itembf{b.} A thesis, theorem; = conclusion 6. Obs.

\P 1636 HEYLIN  \textit{Sabbath} 47 It is a Corollary or conclusion in Geographie, that, etc.    
\P 1800  \textit{Med. Jrnl.} III. 243 Dr. Pearson's Corollaries on the Cow-pox.    
\P 1821 BYRON  \textit{Sardan.} ii. i. 380 You have codes, And mysteries, and corollaries of Right and wrong.

\itembf{2.} transf. Something that follows in natural course; a practical consequence, result.

\P 1674  \textit{Govt. Tongue} (J.), Since we have considered the malignity of this sin‥it is but a natural corollary, that we enforce our vigilance against it.    
\P 1840 CARLYLE  \textit{Heroes} (1858) 305 The art of Writing, of which Printing is a simple, an inevitable‥corollary.    
\P 1884 S. E. DAWSON  \textit{Handbk. Canada} 29 This gigantic enterprise [the Canadian Pacific Railway] was a necessary corollary of the confederation of British America.

\itembf{3.} Something added to a speech or writing over and above what is usual or what was originally intended; an appendix; a finishing or crowning part, the conclusion. Obs.

\P 1603 HOLLAND \textit{Plutarch's Mor.} 1262 With  these verses as with Corollarie‥I will conclude this my discourse.    
\P 1644 BULWER  \textit{Chirol.} 11 A Corollarie of the Speaking motions‥of the Hand.    
\P 1649 EVELYN  \textit{Mem.} (1857) III. 36 There is published a declaration‥which, being now the corollary and ἐπιϕορὰ of what they have to say.    
\P 1676 WORLIDGE  \textit{Cyder} (1691) 200 A Corollary of the Names and Natures of most Fruits growing in England.    
\P 1717 PRIOR  \textit{Alma} ii. 122 Howe'er swift Alma's flight may vary (Take this by way of Corollary).

\itembf{4.} Something additional or beyond the ordinary measure; a surplus; a supernumerary. Obs.

[\P 1602 CAREW  \textit{Cornwall} 123 b, The other side is also ouer$\sim$looked by a great hill‥and for a Corollarium their Conduit water runneth thorow the Church-yard.]    
\P 1610 SHAKES.  \textit{Temp.} iv. i. 57 Now come my Ariell, bring a Corolary, Rather then want a Spirit.    
\P 1613 R. C. TABLE  \textit{Alph.} (ed. 3), Correllarie, ouerplus, that is more then measure.    
\P 1681 tr.  \textit{Willis' Rem. Med. Wks.} Voc., Corollary, addition, vantage, or overplus.
\end{myenumerate}


%%%%%%%%%%%%%%%%%%%%%%%%%%%%%%%%%
\myitem{corporeal} a. (n.)

\noindent \phonetic{(kɔːˈpɔəriːəl)}

\noindent [f. L. corpore-us of the nature of body, bodily, physical (f. corpus, corpor- body) + -al1: cf. corporeous.]
\vspace{-0.3cm}

\begin{myenumerate}

\itembf{A.} adj.

\itembf{1.} Of the nature of the animal body as opposed to the spirit; physical; bodily; mortal.

\P 1610 HEALEY  \textit{St. Aug. Citie of God} 706 Corporeall shall hee [Christ] sit; and thence extend His doome on soules.
\P a1661 FULLER  \textit{Worthies} (1840) III. 6 How inconsistent‥to couple a spiritual grace with matters of corporeal repast.    
\P 1709 STRYPE  \textit{Ann. Ref.} I. xxv. 281 Nor allowed of any manner of corporeal presence in the Sacrament.    
\P 1754 SHERLOCK  \textit{Disc.} (1759) I. vi. 202 It was universally agreed that all that was Corporeal of Man died.    
\P 1870 H. MACMILLAN  \textit{Bible Teach.} viii. 153 The corporeal frame of every human being‥is composed of the same mineral substances.

\itembf{2.} Of the nature of matter; material.

\P 1619 M. FOTHERBY  \textit{Atheom.} ii. xii. §1 (1622) 332 Of things corporeal, and incorporeall; of things liuing, and without life.    
\P 1660 BOYLE  \textit{New Exp. Phys.-Mech.} xvii. 119 Whether‥the exsuction of the Air do prove the place‥to be truly empty, that is, devoid of all Corporeal Substance.    
\P 1725 tr.  \textit{Dupin's Eccl. Hist.} 17th C. I. v. 164 He holds‥that‥the Devils and the Damn'd are punish'd by a Corporeal Fire.    
\P 1788 REID  \textit{Aristotle's Log.} i. §2. 7 Are genera and species corporeal or incorporeal?    
\P 1864 BOWEN  \textit{Logic} x. 334 Our conception of any corporeal thing must include‥those obvious qualities, such as shape, color, specific gravity, etc.    
\P 1875 JOWETT  \textit{Plato} (ed. 2) III. 533 That which is created is of necessity corporeal and visible and tangible.

\itembf{3.} Law. Tangible; consisting of material objects; esp. in corporeal hereditament: see quot. 1767.

\P 1670 HOBBES  \textit{Dial. Com. Laws} 45 Some Goods are Corporeal‥which may be handled, or seen; and some Incorporeal, as Priviledges, Liberties, Dignities, Offices.    
\P 1767 BLACKSTONE  \textit{Comm.} II. 17 Corporeal hereditaments consist wholly of substantial and permanent objects.    
\P 1844 WILLIAMS  \textit{Real Prop.} 11 A manor, which is corporeal property.    
\P 1880 MUIRHEAD tr.  \textit{Instit. Gaius} ii. §12 Corporeal [things] are those that are tangible, such as land, a slave, a garment, gold, silver, and other things innumerable.

\itembf{b.} Bodily; wherein the body is affected.

\P 1765 BLACKSTONE  \textit{Comm.} I. 271 Degrees of nobility‥by immediate grant from the crown: either expressed in writing, by writs or letters patent, as in the creations of peers and baronets; or by corporeal investiture, as in the creation of a simple knight.

\itembf{4.} Formerly used where corporal is now employed. Obs.

\P 1722 SEWEL  \textit{Hist. Quakers} (1795) I. Pref. 13 Death or any corporeal punishment.    
\P 1808  \textit{Med. Jrnl.} XIX. 1 Can a man really suffer corporeal pain, and have at the same time all the criteria, etc.?    
\P 1831 SIR W. HAMILTON  \textit{Discuss.} (1852) 408 He could enforce discipline by the infliction of corporeal punishment.

\itembf{B.} n. pl. [= corporeal things.] Things material.

\P 1647 H. MORE  \textit{Song of Soul} ii. ii. ii. vi, They [the senses] never knew ought but corporealls.    
\P 1678 CUDWORTH  \textit{Intell. Syst.} 779 We should think of Incorporeals, so as not to Confound their Natures with Corporeals.

\itembf{b.} Things pertaining to the human body. rare.

\P 1826  \textit{Blackw. Mag.} XX. 129/1 Of their mental powers, men‥form in general a pretty fair estimate, but they are often sadly out respecting corporeals.

\itembf{c.} Law. Corporeal possessions.

\P 1880 MUIRHEAD  \textit{Gaius} ii. §14 Nor does it affect our definition that there are corporeals included in an inheritance.
\end{myenumerate}


%%%%%%%%%%%%%%%%%%%%%%%%%%%%%%%%%
\myitem{correlate} v.

\noindent \phonetic{(kɒrɪˈleɪt)}

\noindent [f. cor- + relate: see correlate n.]
\vspace{-0.3cm}

\begin{myenumerate}

\itembf{1. a.} intr. To have a mutual relation; to stand in correlation, be correlative (with or to another).

\P 1742 FIELDING  \textit{J. Andrews} Pref., What Caricature is in painting, Burlesque is in writing; and, in the same manner the comic writer and painter correlate to each other.    
\P 1865 GROTE  \textit{Plato} I. xii. 421 The real alone is knowable, correlating with knowledge.
\P a1871 \textit{Eth. Fragm.} iv. (1876) 91 Ethical obligation correlates and is indissolubly conjoined with ethical right.

\itembf{b.} trans. To be correlative to. rare.

\P 1879 W. E. HEARN  \textit{Aryan Househ.} v. §3. 122 The right to the property correlated the duty to the Sacra.

\itembf{2.} To place in or bring into correlation; to establish or indicate the proper relation between.

\P 1849 MURCHISON  \textit{Siluria} vii. 134 Mr. Symonds was‥enabled to correlate these beds with their equivalents near Ludlow.    
\P 1881 J. GEIKIE in  \textit{Nature} 337 He correlates the interglacial beds of Mont Perrier with those of Dürnten.    
\P 1925 N. BOHR  \textit{Theory of Spectra} (ed. 2) 135 It has been possible to correlate each term with the occurrence of electron orbits of a given type.    
\P 1930  \textit{Economist} 18 Oct. 715/2 To prove by an historical statistical analysis that‥it is impossible to correlate from available evidence either high rates and low stock prices or low rates and high stock prices with any certainty.    
\P 1952 G. H. BOURNE  \textit{Cytol. \& Cell Physiol.} (ed. 2) vi. 273 Bennett was not able to correlate changes in the Golgi material with secretion in the cat adrenal.    
\P 1971  \textit{Nature} 15 Jan. 182/1 So the observed luminosity of the primary [star] can be correlated reliably with its original main-sequence mass.    
\P 1971  \textit{Daily Tel.} 12 Feb. 8/2 Are you wondering how many people spend their time‥in devising such idiot statistical measures, applying them, collating and correlating them?

\itembf{3.} pass. To have correlation, to be intimately or regularly connected or related (with, rarely to); spec. in Biol. of structures or characteristics in animals and plants (cf. correlation 3).

\P 1862 F. HALL  \textit{Hindu Philos. Syst.} 95 Transmuting relations into entities, and interposing these entities between things correlated.    
\P 1870 ROLLESTON  \textit{Anim. Life} Introd. 20 Parasitism‥is often found to be correlated with ‥ disappearance of structures.    
\P 1875 POSTE  \textit{Gaius} ii. Comm. (ed. 2) 160 Other rights‥have no determinate subject‥to which they are correlated.
\end{myenumerate}


%%%%%%%%%%%%%%%%%%%%%%%%%%%%%%%%%
\myitem{coruscate} v.

\noindent \phonetic{(ˈkɒrəskeɪt)}

\noindent [f. ppl. stem of L. coruscāre to vibrate, glitter, sparkle, gleam.]
\vspace{-0.3cm}

\begin{myenumerate}

\itembf{a.} intr. To give forth intermittent or vibratory flashes of light; to shine with a quivering light; to sparkle, glitter, flash.

\P 1705 [See CORUSCATING].    
\P 1808 J. BARLOW  \textit{Columb.} iii. 162 A sudden glare Coruscates wide.    
\P 1846 HAWTHORNE  \textit{Mosses, Mother Rigby's Pipe} ii, The star kept coruscating.    
\P 1883  \textit{Harper's Mag.} Jan. 186/2 The light was a brilliant green, coruscating from the centre‥in‥flashes of flame.

fig. \P 1851 CARLYLE  \textit{Sterling} ii. iii. (1872) 104 Like a swift dashing meteor he came into our circle; coruscated among us, for a day or two.    
\P 1880  \textit{Sat. Rev.} No. 1296. 262 The President will be chosen mainly for his power of coruscating.

\itembf{b.} with cognate object.

\P 1852 HAWTHORNE  \textit{Blithedale Rom.} xxii, Coruscating continually an unnatural light.
\end{myenumerate}


%%%%%%%%%%%%%%%%%%%%%%%%%%%%%%%%%
\myitem{cosset} n.

\noindent \phonetic{(ˈkɒsɪt)}

\noindent [Not found before the 16th c.: derivation uncertain.

   Prof. Skeat (Trans. Philol. Soc. 1889) has suggested that it is the same word as OE. cot-sǽta cot-sitter, dweller in a cot, cottar; cf. the Domesday forms, pl. coscez, cozets, cozez (z = ts). This is phonetically satisfactory, and the sense of ‘lamb dwelling in a cot’ or ‘kept by a cot-sǽta or cottar’ finds support in It. casiccio a tame lamb bred by hand, f. casa house; Ger. hauslamm house-lamb and ‘pet’, is analogous. Cf. also ‘Cotts, lambs brought up by hand, cades’, Marshall Rural Econ. E. Norfolk, 1787 (whence in Grose 1790). There is however a long gap between the coscez of Domesday and the cosset of 1579, during which no trace of the word in either sense has been found.] 

\vspace{-0.3cm}

\begin{myenumerate}

\itembf{1.} A lamb (colt, etc.) brought up by hand; a pet-lamb, cade-lamb. Also attrib. as cosset lamb.

\P 1579 SPENSER  \textit{Sheph. Cal.} Nov. 42, I shall thee give yond Cosset for thy payne.    
\P 1613 W. BROWNE  \textit{Sheph. Pipe} Wks. 1772 III.  39 The best cosset in my fold.    
\P 1626 BRETON  \textit{Fantastickes} Apr. (D.), The cosset lamb is learned to butt.    
\P 1674 RAY  \textit{S. \& E. C. Words} 62 A Cosset lambe or colt, \&c. i.e. a cade lamb, a lamb or colt brought up by the hand, Norf. Suff.    
\P 1749 W. ELLIS  \textit{Sheph. Guide} 77 A cossart-lamb in Hertfordshire is one left by its dam's dying by disease or hurt before it is capable of getting its own living; or is one that is taken from a ewe that brings two or three or four lambs at a yeaning, and is incapable of suckling and bringing them all up.    
\P 1883  \textit{Sat. Rev.} LVI. 109 The character of cosset lambs is notoriously bad; and‥the pet horse is, as a rule, a somewhat uncertain animal in stable.

\itembf{2.} Applied to persons, etc.: A pet of any kind; a petted, spoilt child.

\P 1596 NASHE  \textit{Saffron Walden} 143 Who but an ingrain cosset would keepe such a courting of a Curtezan.    
\P 1614 B. JONSON  \textit{Barth. Fair} i. i, I am for the cosset his charge.    
\P 1659 GAUDEN  \textit{Tears of Ch.} 595 Some are such Cossets and Tantanies that they congratulate their Oppressors and flatter their Destroyers.
\P a1700 B. E. \textit{Dict.  Cant. Crew}, Cosset, a Fondling Child.
\P a1825 FORBY  \textit{Voc. E. Anglia}, Cosset, a pet, something fondly caressed.
\end{myenumerate}



%%%%%%%%%%%%%%%%%%%%%%%%%%%%%%%%%
\myitem{cosset} v.

\noindent \phonetic{(ˈkɒsɪt)}

\noindent [f. prec. n. In literary use, chiefly of 19th c.]

\vspace{-0.3cm}

\begin{myenumerate}

\itembf{a.} trans. To treat as a cosset; to fondle, caress, pet, indulge, pamper.

\P 1659 GAUDEN  \textit{Tears of Ch.} 375 Episcopacy‥was even pampered and cosetted by so excessive a favour.
\P a1825 FORBY  \textit{Voc. E. Anglia}, Cosset, to fondle.    
\P 1857 SIR F. PALGRAVE  \textit{Norm. \& Eng.} II. 800 Henry, so cosseted during babyhood and boyhood by his grandmother.    
\P 1859 H. KINGSLEY  \textit{G. Hamlyn} xxvi. (D.), I have been cosseting this little beast up.    
\P 1860 EMERSON  \textit{Cond. Life} i. (1861) 7 Nature is no sentimentalist—does not cosset or pamper us.

\itembf{b.} intr. or absol.

\P 1871 B. TAYLOR  \textit{Faust} (1875) II. iii. 201 Probe and dally, cosset featly, Test your wanton sport completely.    
\P 1889 H. WEIR  \textit{Our Cats} 11 Another [cat] would cosset up close to a sitting hen.
\end{myenumerate}


%%%%%%%%%%%%%%%%%%%%%%%%%%%%%%%%%
\myitem{coterie} n.

\noindent \phonetic{(ˈkəʊtərɪ)}

\noindent [a. F. coterie ‘a company of people who live in familiarity, or who cabal in a common interest’ (Littré), orig. ‘a certain number of peasants united together to hold land from a lord’; ‘companie, societie, association of countrey people’ (Cotgr.), f. cotier = med.L. cotārius, coterius cottar, tenant of a cota or cot. Cf. F. cotterie ‘a base, ignoble, and seruile tenure, or tenement, not held in fee, and yeelding only rent, or if more, but cens or surcens at most’ (Cotgr.).

   By Walker and Smart stressed on the last syllable as French: the latter has the o short; whence the 18th c. cotterie, and its riming in Byron with lottery.]
%\vspace{-0.3cm}

\begin{myenumerate}

\itembf{1.} An organized association of persons for political, social, or other purposes; a club. Obs.

\P 1764  \textit{Univ. Museum} Jan. 6 A numerous and formidable society of persons of distinction, property, abilities, and influence in the nation, is now forming, and a large house of a deceased nobleman is hired for their assemblies, which society is to be called The cotery of revolutionists, or of anti-ministerialists, from the French word coterie, vulgarly called a club in English.    
\P 1766 D. BARRINGTON  \textit{Observ. Stat.} 249 note, The word cotterie, of which so much has been said of late.    
\P 1774 FOOTE  \textit{Cozeners} i. Wks. 1799 II. 146  My expences in‥subscription-money to most of the clubs and coteries.

\itembf{2.} A circle of persons associated together and distinguished from ‘outsiders’, a ‘set’: \itembf{a.} A select or exclusive circle in Society; the select ‘set’ who have the entrée to some house, as ‘the Holland House coterie’.
   ‘A friendly or fashionable association. It has of late years been considered as meaning a select party, or club, and sometimes of ladies only’ (Todd 1818).

\P 1738  \textit{Common Sense} I. 345 Beware of Select Cotteries, where, without an Engagement, a Lady passes but for an odd Body.    
\P 1768 STERNE  \textit{Sent. Journ.} (1778) II. 164, I was lifted directly into Madame de V***'s Coterie.    
\P 1779 F. BURNEY  \textit{Diary} Oct., You recollect what Mrs. Thrale said of him, among the rest of the Tunbridge coterie, last season.    
\P 1821 BYRON  \textit{Juan} iv. cix, Fame is but a lottery Drawn by the blue-coat misses of a coterie.    
\P 1828 J. W. CROKER in \textit{C. Papers} (1884) I. xiii. 400 Lady Holland was saying yesterday to her assembled coterie.    
\P 1880 V. LEE  \textit{Stud. Italy} iii. i. 68 A man‥belonging to the most brilliant coteries of the day.

\itembf{b.} A ‘set’ associated by certain exclusive interests, pursuits, or aims; a clique.

\P 1827 DE QUINCEY  \textit{Murder} Wks. III. 12 Catiline, Clodius and some of that coterie.    
\P 1830 CUNNINGHAM  \textit{Brit. Paint.} I. v. 207 A certain coterie, of men, skilful in the mystery of good painting.    
\P 1838-9 HALLAM \textit{Hist. Lit.} IV. vii. iv. §54. 329 Written for an exclusive coterie, not for the world.    
\P 1862 MERIVALE  \textit{Rom. Emp.} (1865) V. xlvi. 359 In vain had Tiberius chafed under the jeers of this licensed coterie.    
\P 1888 W. D. HAMILTON  \textit{Cal. State Papers, Domestic Ser.} 1644 Pref.  10 This religious element‥revived the bitter animosities of the old political parties, and caused the members [of Parliament] to group themselves into coteries.

\itembf{c.} A meeting or gathering of such a circle.

\P 1805 MOORE  \textit{To Lady H-} iv, Each night they held a coterie.    
\P 1849 E. E. NAPIER  \textit{Excurs. S. Africa} II. 347 We are so accustomed now to this style of fusillade, that all we do is to lie close, and continue our little coteries.

\itembf{d.} transf. and fig. Of animals, plants, etc.

\P 1869 GILLMORE  \textit{Reptiles \& Birds} 219 With the permission of the masters of the coterie they build their nests in the vacancies that occur in the squares.    
\P 1885 H. O. FORBES  \textit{Naturalist's Wand.} 85 The genus Pajus is an exceedingly handsome and attractive coterie of orchids.

\itembf{3.} attrib. and Comb., as coterie-speech. Also quasi-adj.

\P 1833 MILL  \textit{Lett.} (1910) I. 77 A paper which‥keeps aloof from all coterie influence.    
\P 1891  \textit{Pall Mall G.} 12 May 3/1 A coterie-speech—not to say a jargon—current only on the highest heights of culture.    
\P 1900 G. B. SHAW  \textit{Let.} 9 Feb. (1931) 375 This Stage Society‥is catching on in its little coterie-theatre way.    
\P 1933 P. GODFREY  \textit{Back-Stage} xiii. 165 Circulars designed to appeal to those who incline to coterie art and limited editions.    
\P 1962  \textit{Listener} 30 Aug. 327/2 The very exercise will remove accretions of coterie language and provincialism from serious writers who attempt it.

\vspace{0.1cm} \noindent
\phonetic{
Hence (chiefly nonce-wds.) ˈcoterie v., to associate in a coterie. coteˈriean a., of or pertaining to a coterie; n. a member of a coterie. ˈcoterieish a., savouring of a coterie. ˈcoterieism, the spirit or practice of coteries.
}

\P 1806 T. S. SURR  \textit{Winter in Lond.} (ed. 3) II. 156 If‥I can do otherwise than coterie with Neville and the Beauchamps.    
\P 1778 \textit{Learning  at a Loss} I. 67 Drest by Coteriean Laws.    
\P 1772  \textit{Poetry} in \textit{Ann. Reg.} 225 Ye Coterieans! who profess No business, but to dance and dress.    
\P 1841  \textit{Tait's Mag.} VIII. 590 [She] received an immense quantity of praise from the English press, courteous, cordial, and coterieish.    
\P 1825  \textit{New Monthly Mag.} XIII. 584 This spirit of coterieism is so prevalent.    
\P 1862 R. H. PATTERSON  \textit{Ess. Hist. \& Art} 517 The polished coterieism of Moore.
\end{myenumerate}


%%%%%%%%%%%%%%%%%%%%%%%%%%%%%%%%%
\myitem{craven} a. and n.

\noindent \phonetic{(ˈkreɪv(ə)n)}

\noindent [In early ME. crauant (rare), etymology obscure.

   Mr. Henry Nicol (Proc. Phil. Soc., Dec. 1879) suggested its identification with OF. cravanté, crevanté, crushed, overcome: see cravent v. But the total absence of the final é from the word, at a date when English still retained final e, makes a difficulty. Others have considered it a variant, in some way of creant (OF. creant, craant), which is a much more frequent word in the same sense in ME. The difficulty here is to account for the v (u), for which popular association with crave v. and its northern pa. pple. craved has been conjectured.]
%\vspace{-0.3cm}

\begin{myenumerate}

\itembf{A.} adj.

\itembf{1.} Vanquished, defeated; or, perh., confessing himself vanquished. Obs.

\P 1225 \textit{St. Marher.}  11 Ich am kempe ant he is crauant þet me wende to ouercumen.
\P a1225  \textit{Leg. Kath.} 133 Al ha icneowen ham crauant \& ourcumen, \& cweðen hire þe meistrie \& te menske al up.

\itembf{b.} to cry craven: to acknowledge oneself vanquished, to give up the contest, surrender. Also fig.

\P 1634 COKE  \textit{Inst.} iii. (1648) 221 If he become recreant, that is, a crying Coward or Craven he shall for his perjury lose liberam legem.    
\P 1639 FULLER  \textit{Holy War} iv. xi. (1840) 196 He had been visited with a desperate sickness, insomuch that all art cried craven, as unable to help him.    
\P 1768 BLACKSTONE  \textit{Comm.} III. 340 Or victory is obtained, if either champion proves recreant, that is, yields, and pronounces the horrible word of craven.    
\P 1805 SOUTHEY \textit{Madoc} in  \textit{W.} xv, I‥will make That slanderous wretch cry craven in the dust.    
\P 1869 FREEMAN  \textit{Norm. Conq.} (ed. 2) III. xv. 451 Neither King nor Duke was a man likely to cry craven.

\itembf{2.} That owns himself beaten or afraid of his opponent; cowardly, weak-hearted, abjectly pusillanimous.

\P 1400  \textit{Morte Arth.} 133 Haa! crauaunde knyghte! a cowarde þe semez!    
\P 1598 DRAYTON  \textit{Heroic. Epist.} v. 77 Those Beggers-Brats‥Ally the Kingdome to their cravand Brood.    
\P 1602 SHAKES.  \textit{Ham.} iv. iv. 40 Some craven scruple Of thinking too precisely on the event.    
\P 1656 TRAPP  \textit{Comm. 1 Cor.} xv. 55 Death is here out-braved, called craven to his face.    
\P 1808 SCOTT  \textit{Marm.} v. xii, The poor craven bridegroom said never a word.    
\P 1848 MACAULAY  \textit{Hist. Eng.} II 592 All other feelings had given place to a craven fear for his life.

\itembf{b.} Applied to a cock: see B 2. Obs.

\P 1579 LYLY  \textit{Euphues} (Arb.) 106 Though hee bee a cocke of the game, yet Euphues is content to be crauen and crye creake.    
\P 1609 BP. W. BARLOW  \textit{Answ. Nameless Cath.} 164 This Crauen Cocke, after a bout or two‥crowing a Conquest, being ready presently to Cry Creake.
\P c1622 FLETCHER  \textit{Love's Cure} ii. ii. Wks. (Rtldg.) II. 161/1 Oh, craven-chicken of a cock o' th' game!    
\P 1649 G. DANIEL  \textit{Trinarch., Hen. V}, xlix, Red Craven Cocks come in.

\itembf{B.} n.

\itembf{1.} A confessed or acknowledged coward.

\P 1581 J. BELL  \textit{Haddon's Answ. Osor.} 349 Monckes and Friers, and that whole generation of Cowled Cravines.    
\P 1599 SHAKES.  \textit{Hen. V}, iv. vii. 139 Hee is a Crauen and a Villaine else.    
\P 1610 ROWLANDS  \textit{Martin Mark-all} 53 In regard of manhood a meere crauant.    
\P 1795 SOUTHEY  \textit{Joan of Arc} x. 458 Fly, cravens! leave your aged chief.    
\P 1860 FROUDE  \textit{Hist. Eng.} VI. 73 He climbed to the highest round of the political ladder, to fall and perish like a craven.

\itembf{2.} A cock that ‘is not game’.

\P 1596 SHAKES.  \textit{Tam. Shr.} ii. i. 228 No Cocke of mine, you crow too like a crauen.    
\P 1611 SPEED  \textit{Hist. Gt. Brit.} ix. iv. 14 Whereto the Pope, (no Crauant to be dared on his owne dung-hill) as stoutly answered.    
\P 1826  \textit{Gentl. Mag.} Feb. 157/1 It is certainly a hard case that a fighting-cock should kill an unoffending craven.

\itembf{C.} Comb., as craven-hearted, craven-like adj. \& adv.

\P 1615 CROOKE  \textit{Body of Man} 245 All creatures whose Testicles are hidde within should be faynt and crauen-hearted.    
\P 1705 HICKERINGILL  \textit{Priest-Craft} Wks. (1716) III. 56 Not as Gentlemen and Scholars, but (Craven like) calling upon the Jailors, the Sumners, etc.    
\P 1836 WHITTIER  \textit{Song of the Free} i, Shrink we all craven-like, When the storm gathers?
\end{myenumerate}


%%%%%%%%%%%%%%%%%%%%%%%%%%%%%%%%%
\myitem{craw} n.

\noindent \phonetic{(krɔː)}

\noindent [ME. crawe, repr. an unrecorded OE. \phonetic{*craᴁa}, cogn. with OHG. chrago, MHG. krage, Du. kraag neck, throat; or else a later Norse krage, Da. krave in same sense. The limitation of sense in English is special to this language.]
\vspace{-0.3cm}

\begin{myenumerate}

\itembf{1.} The crop of birds or insects.

\P 1388 WYCLIF  \textit{2 Kings} vi. 25 The crawe of culueris. Margin, In Latyn it is seid of the drit of culuers; but drit is‥takun here‥for the throte, where cornes, etun of culueris, ben gaderid.
\P c1440  \textit{Promp. Parv.} 101 Craw, or crowpe of a byrde, or oþer fowlys, gabus, vesicula.    
\P 1552 HULOET,  Craye or gorge of a byrde, ingluuies.    
\P 1565-78 COOPER  \textit{Thesaurus}, Chelidonii‥Little stones in the crawe of a swallow.    
\P 1604 DRAYTON  \textit{Owle} 75 The Crane‥With Sand and Gravell burthening his Craw.    
\P 1774 HUNTER in  \textit{Phil. Trans.} LXIV. 313 Some birds, with gizzards, have a craw or crop also, which serves as a reservoir, and for softening the grain.    
\P 1855 LONGFELLOW  \textit{Hiaw.} viii. 209 Till their craws are full with feasting.    
\P 1855 THACKERAY  \textit{Newcomes} II. 35 Such an agitation of plumage, redness of craw, and anger of manner as a maternal hen shows.

\itembf{2.} transf. \textbf{a.} The stomach (of man or animals). humorous or derisive.

\P 1573 A. ANDERSON  \textit{Exp. Benedictus} 43 (T.) To gorge their craws with bibbing cheer.    
\P 1581 J. BELL  \textit{Haddon's Answ. Osor.} 320 b, Stuffing their crawes with most exquisite vyandes.    
\P 1791 WOLCOTT  (P. Pindar) \textit{Remonstrance} Wks. 1812. II. 449 They smite their hungry craws.    
\P 1822 BYRON  \textit{Juan} viii. xlix, As tigers combat with an empty craw.

\itembf{b.} to cast the craw: to vomit. Obs.

\P 1529 SKELTON  \textit{El. Rummyng} 489 Such a bedfellow Would make one cast his craw.

\itembf{3.} transf. The breast of a hill. Obs. rare.

\P 1658 CLEVELAND  \textit{May Day} ii, Phœbus tugging up Olympus craw.

\itembf{b.} Humorously applied to a cravat, falling over the chest in a broad fold of lace or muslin.
   See Fairholt s.v. Neckcloth.

\P 1787 ‘G.  GAMBADO’ \textit{Acad. Horsemen} (1809) 14 The creatures with monstrous craws.    
\P 1790  \textit{Poetry} in \textit{Ann. Reg.} 135 Now, at his word, th' obedient muslin swells, And beaux, with ‘Monstrous Craws,’ peep out at pouting belles.

\itembf{4.} Comb. †craw-bone, the ‘merry-thought’ of a bird, which lies over the craw; craw-thumper (slang), one who beats his breast (at confession); applied derisively to Roman Catholic devotees; so craw-thump v.

\P 1611 COTGR,  \textit{Bruchet}, the craw-bone, or merrie thought of a bird.    
\P 1785 WOLCOTT  (P. Pindar) \textit{Ode to R. A.'s} Wks. 1812 I. 93 We  are no Craw-thumpers, no Devotees.    1797-
\P 1802 G. COLMAN  \textit{Br. Grins, Knt. \& Friar} i. xxxv, Sir Thomas and the dame were in their pew Craw-thumping upon hassocks.    
\P 1873 \textit{Slang.  Dict.}, Craw thumper, a Roman Catholic. Compare Brisket-beater.
\end{myenumerate}


%%%%%%%%%%%%%%%%%%%%%%%%%%%%%%%%%
\myitem{credible} a.

\noindent \phonetic{(ˈkrɛdɪb(ə)l)}

\noindent [ad. L. crēdibilis worthy to be believed, f. crēd-ĕre to believe: see -ble. Also in 15-16th c. F. croidible, crédible.]
\vspace{-0.3cm}

\begin{myenumerate}

\itembf{1.} Capable of being believed; believable: \textbf{a.} of assertions.

\P 1374 CHAUCER  \textit{Boeth.} iv. iv. 124 Al be it so þat þis ne seme nat credible þing perauenture to somme folk.    
\P 1430 LYDG.  \textit{Chron. Troy} i. vi, The mortall harme‥That is well more then it is credible.    
\P 1594 [See CREDIBILITY].    
\P 1651 HOBBES  \textit{Leviath.} i. ii. 7 Than right reason makes that which they say, appear credible.    
\P 1798 FERRIAR  \textit{Varieties of Man} in \textit{Illustr. Sterne} 211 Who had the fate to be disbelieved in every credible assertion.    
\P 1883 FROUDE  \textit{Short Stud.} IV. i. xi. 142 When the falsehood ceased to be credible the system which was based upon it collapsed.

\itembf{b.} of matters of fact: with impersonal const.

\P 1526  \textit{Pilgr. Perf.} (W. de W. 1531) 165 b, And it is to suppose, \& credyble to byleue that, etc.    
\P 1563 FULKE  \textit{Meteors} (1640) 52 Some would make it seeme credible, that of vapours and Exhalations‥a calfe might be made in the clouds.    
\P 1653 H. COGAN  tr. \textit{Pinto's Trav.} xlix. 195 No news could be heard of her, which made it credible that she also suffered shipwrack.    
\P 1699 BURNETT  \textit{39 Art.} vi. (1700) 81 It is not all credible that an Imposture of this kind could have passed upon all the Christian Churches.

\itembf{c.} (See quot. 1963.) Cf. credibility b.

\P 1960  \textit{Times} 11 Feb. 11/6 As a guarantee of European nuclear retaliation against a nuclear attack a N.A.T.O. deterrent would be highly credible.    
\P 1963  \textit{Daily Tel.} 12 Jan. 13/8 ‘Credible’, in the language of nuclear strategy, does not mean ‘adequately frightful’. It means ‘such as an enemy will think likely to be used’.    
\P 1966 SCHWARZ \& HADIK  \textit{Strategic Terminology} 42 Credible first strike capability.    Ibid., The deterrent effect must also be credible to the allies who are to be protected by the threat.

\itembf{2.} Worthy of belief or confidence; trustworthy, reliable: †\itembf{a.} of information, evidence, etc. Obs.

\P 1393 GOWER  \textit{Conf.} III. 170 Among the kinges in the bible I finde a tale and is credible Of him.    
\P 1426 PASTON  \textit{Lett.} No. 7 I. 25, I herde‥no maner lykly ne credible evidence.    
\P 1513 MORE  \textit{Rich. III} Wks. 37/2 This haue I by credible informacion learned.    
\P 1601 SHAKES.  \textit{All's Well} i. ii. 4 So tis reported sir‥Nay tis most credible.    
\P 1632 LITHGOW  \textit{Trav.} iv. (1682) 139 It is holden to be so credible as if an Oracle had spoken it.

\itembf{b.} of persons. (Now somewhat arch., exc. in ‘credible witness’ or the like.)

\P 1478 SIR J. PASTON  \textit{Lett.} No. 814 III. 222 Any suche credyble man maye, iff he wyll, wytnesse ther-in with me.    
\P 1502 ARNOLDE  \textit{Chron.} (1811) 125 Promysing feithfully in the presence of credyble persones.    
\P 1550 CROWLEY \textit{Last Trump.} 1370 Though  the euidence be plaine, and the accusars credible.    
\P 1671 J. WEBSTER  \textit{Metallogr.} iii. 40 Observations from credible Authors.    
\P 1722 SEWEL  \textit{Hist. Quakers} (1795) I. Pref. 11 Which I noted down from the mouth of credible persons.    
\P 1875 JOWETT  \textit{Plato} (ed. 2) II. 473, I have been informed by a credible person that [etc.].

\itembf{3.} Ready, willing, or inclined to believe. Obs.

\P 1420  \textit{Chron. Vilod.} 1087 Þuse  men weren credeable of Seynt Edus godenasse.
\P c1440 LYDG. \textit{Secrees} 1060 Nat  lyghtly to be Credyble To Talys that make discencion.    
\P 1623 COCKERAM  ii. A iiij b, One too much Beleeuing, Credulous, Credible.    
\P 1675 TRAHERNE  \textit{Chr. Ethics} xv. 217 There is a fair way laid open to the credible of such objects attested and revealed with such circumstances.

\itembf{4.} Having or deserving credit or repute; of good repute, creditable, reputable. Obs.

\P 1631 MILTON \textit{Let}. in  \textit{Wks.} (ed. Birch 1738) I. 4 To which nothing is more helpful than the early entring into some credible Employment.    
\P 1647 LILLY  \textit{Chr. Astrol.} xxix. 191 He is in good estimation and lives in a credible way.    
\P 1712 ARBUTHNOT  \textit{John Bull} ii. iii, A good credible way of living.
\end{myenumerate}


%%%%%%%%%%%%%%%%%%%%%%%%%%%%%%%%%
\myitem{creditable} a.

\noindent \phonetic{(ˈkrɛdɪtəb(ə)l)}

\noindent [f. credit v. and n. + -able. (No corresp. Fr. word.)]
\vspace{-0.3cm}

\begin{myenumerate}

\itembf{1. a.}  Worthy to be believed; credible. Obs.

\P 1526 FRITH  \textit{Disput. Purgat.} 192 ‘Neither it is creditable’, (saith he) ‘that all which are cast into hell should straight$\sim$way go to heaven, therefore must we put a purgatory.’    
\P 1638 CHILLINGW.  \textit{Relig. Prot.} i. Pref. §43 Records farre more creditable then these.    
\P 1669 WOODHEAD  \textit{St. Teresa} i. Pref. (1671) a, Persons, sufficiently creditable, and perfectly informed.    
\P 1760 WINTHROP in  \textit{Phil. Trans.} LII. 8 The most distinct account I have had of it, was from a creditable person at Roxbury.    1807-8 W. Irving Salmag. xi. (1860) 252 A church-yard, which at least a hundred creditable persons would swear was haunted.

\itembf{b.} Comm. Worthy of receiving credit (commercially); having good credit. Obs.

\P 1776 ADAM  SMITH \textit{W.N.} I. ii. ii. 307 The creditable traders of any country.    
\P 1818 JAS.  MILL \textit{Brit. India} II. v. viii. 670 On receiving the security of creditable bankers for the balance which the Nabob owed to the Company.    
\P 1822 J. FLINT  \textit{Lett. fr. Amer.} 108 Banks that were creditable a few days ago, have refused to redeem their paper in specie.

\itembf{2. a.} That brings credit or honour; that does one credit; reputable. Often implying a slighter degree of praise or excellence: Respectable (see c).

\P 1659 \textit{Gentl.  Calling} (1696) 31 It is become a creditable thing, the badge and signature of a modern Wit, thus to be one of David's Fools, in saying, There is no God.    
\P 1691 HARTCLIFFE  \textit{Virtues} 89 Whatsoever is just, honest, and Creditable.    
\P 1828 SCOTT  \textit{F.M. Perth} xix, Did he not maintain an honest house‥and keep a creditable board?    
\P 1840 MACAULAY  \textit{Clive} 62 Clive made a creditable use of his riches.    
\P 1884  \textit{Law Rep.} 13 Q. Bench Div. 615 The father‥was not‥leading a creditable life.

\itembf{b.} That does credit to.

\P 1797 T. BEWICK  \textit{Brit. Birds} (1847) I. 231 Mr. Selby's splendid work on ornithology, so creditable to his zeal in the cause of Science.    
\P 1855 MACAULAY  \textit{Hist. Eng.} IV. 43 The places‥were filled in a manner creditable to the government.

\itembf{c.} Respectable, decent (a) in appearance or quality; (b) in social position or character. Obs.

\P 1688 MIEGE  \textit{Fr. Dict.} s.v., This suit of yours is a creditable Suit, Cet Habit est honnête.    
\P 1741 RICHARDSON  \textit{Pamela} II. 352 A creditable Silk for my dear Mother.    
\P 1765 GOLDSM.  \textit{Ess.} xxv. 224 This gentleman was born of creditable parents, who gave him a very good education.    
\P 1779 J. MOORE  \textit{View Soc. Fr.} II. xcv. 426 A Frenchman in a creditable way of life.    
\P 1825 MRS. CAMERON  \textit{Proper Spirit} in \textit{Houlston Tracts} I. ix. 7 To set a poor lad, like you, to teach creditable children.    
\P 1860 GEN. P. THOMPSON  \textit{Audi Alt.} III. cv. 14 It was once my fortune to serve with two Russian midshipmen; very creditable lads they were.

\itembf{3.} Capable of being ascribed to.

\P 1904 \textit{Rep.  Librarian Congress} 32 Many documents creditable to that period can be judged to be so and assigned to their proper group only by internal evidence.
\end{myenumerate}


%%%%%%%%%%%%%%%%%%%%%%%%%%%%%%%%%
\myitem{credulous} a.

\noindent \phonetic{(ˈkrɛdjʊləs)}

\noindent [f. L. crēdul-us (F. crédule) + -ous.]
\vspace{-0.3cm}

\begin{myenumerate}

\itembf{1.} Ready or disposed to believe. (Now rare exc. as in 2.)

\P 1579 G. HARVEY  \textit{Letter-bk.} (Camden) 86 Beinge over credulous to beleeve whatsoever is unadvisedly committid to writinge.    
\P 1596 SHAKES.  \textit{Tam. Shr.} iv. ii. 69 If he be credulous, and trust my tale.    
\P 1605 BP. HALL  \textit{Medit. \& Vows} ii. 15 Not a curious head, but a credulous and plaine heart is accepted with God.    
\P 1697 W. DAMPIER  \textit{Voy.} (1698) I. xiii. 364, I‥advised him not to be too credulous of the Generals promises. 
[\P 1839 LONGFELLOW  \textit{Flowers} xv, With childlike credulous affection.    
\P 1859 TENNYSON  \textit{Idylls, Geraint \& Enid} 1723 Like  simple noble natures, credulous Of what they long for, good in friend or foe.]

\itembf{2.} Over-ready to believe; apt to believe on weak or insufficient grounds.

\P 1576 FLEMING  \textit{Panopl. Epist.} 216 Bee not credulous‥and light of beleefe.    
\P 1604 SHAKES.  \textit{Oth.} iv. i. 46 Thus credulous Fooles are caught.    
\P 1687 T. BROWN  \textit{Saints in Uproar Wks.} 1730 I. 81 Seven  as arrant imposters as ever deluded the credulous world.    
\P 1791 COWPER \textit{Iliad} xvi. 1030 And  with vain words the credulous beguiled.
\P a1862 BUCKLE  \textit{Civiliz.} (1869) III. ii. 111 An ignorant and therefore a credulous age.    
\P 1876 J. H. NEWMAN  \textit{Hist. Sk.} I. iii. iv. 322 Well known to be of a credulous turn of mind.

\itembf{b.} transf. Of things, etc.: Characterized by or arising from credulity.

\P 1648 MILTON  \textit{Tenure Kings} Wks. 1738 I. 323  That credulous Peace which the French Protestants made with Charles the Ninth.    
\P 1769 ROBERTSON  \textit{Chas.} V, III. x. 190 The credulous superstition of the people.    
\P 1871 FARRAR  \textit{Witn. Hist.} ii. 57 Credulous exaggerations.

\itembf{c.} Believed too readily. Obs. rare.

\P 1625 BEAUMONT \& FL.  \textit{Faithf. Friends} iv. i, 'Twas he possessed me with your credulous death.
\end{myenumerate}


%%%%%%%%%%%%%%%%%%%%%%%%%%%%%%%%%
\myitem{crest-fallen} ppl. a.

\noindent \phonetic{(ˈkrɛstˌfɔːlən)}

\vspace{-0.3cm}

\begin{myenumerate}
\itembf{1.} With drooping crest; hence, cast down in confidence, spirits, or courage; humbled, abashed, disheartened, dispirited, dejected.

\P 1589 \textit{Pappe  w. Hatchet} D iv b, O how meager and leane hee lookt, so creast falne, that his combe hung downe to his bill.    
\P 1593 SHAKES.  \textit{2 Hen. VI}, iv. i. 59 Let it make thee Crest$\sim$falne, I, and alay this thy abortiue Pride.    
\P 1668 MARVELL  \textit{Corr.} cv. Wks. 1872-5 II. 264 He is here a kind of decrepit young gentleman and terribly crest-falln.    
\P 1860 THACKERAY  \textit{Four Georges} iii. (1876) 69 Slinking back into the club somewhat crestfallen after his beating.

\itembf{2.} Of a horse: see quot. 1725.

\P 1696  \textit{Lond. Gaz.} No. 3217/4 A grey Gelding‥black mane and tail, and a little Crest-fallen.    
\P 1725 BRADLEY  \textit{Fam. Dict.}, Crestfallen, a Distemper in Horses, when the Part on which the Main grows, which is the upper Part thereof, and call'd the Crest, hangs either to one Side or the other, and does not stand upright as it ought to do.

\vspace{0.1cm} \noindent \phonetic{
Hence ˈcrestˌfallenly adv., ˈcrestˌfallenness.
}

\P 1854 LYTTON  \textit{What will he} iv. i, That ineffable aspect of crestfallenness!    
\P 1880 R. BROUGHTON  \textit{Sec. Th.} I. i. ii. 28 The Squire is crestfallenly eying the shipwreck of his hopes.    
\P 1890 \textit{Alas!} II. xxiv. 125 A look of mortification and crestfallenness.
\end{myenumerate}


%%%%%%%%%%%%%%%%%%%%%%%%%%%%%%%%%
\myitem{culpable} a. (and n.)

\noindent \phonetic{(ˈkʌlpəb(ə)l)}

\noindent [ME. coupable, a. OF. coupable (cop-, coulpable, culpable, etc.) guilty:—L. culpābil-is blameworthy, f. culpa fault, blame. The OF. was regularly reduced to coupable in 13th c., but was frequently written culpable after L. in 14th c., coulpable in 16th c.; the latinized form has in Eng. been established both in spelling and pronunciation.]
\vspace{-0.3cm}

\begin{myenumerate}

\itembf{1.} Guilty, criminal; deserving punishment or condemnation. Obs. (or blended with sense 2.)

\P 1303 R. BRUNNE  \textit{Handl. Synne} 1331 Ȝyf  þou‥Fordost pore mannys sustynaunce Þat aftyrwarde he may nat lyve Þou art coupable.    
\P 1377 LANGL.  \textit{P. Pl.} B. xvii. 300 Any creature þat is coupable afor a kynges iustice.    
\P 1483 CAXTON  \textit{Cato} E j b, How be it that they ben gylty and culpable.    
\P 1573 BP. OF PETERBOROUGH  in Ellis \textit{Orig. Lett.} ii. 196 III. 35 If thei be able justelie‥to finde him culpable.    
\P 1661 BRAMHALL  \textit{Just Vind.} ii. 22 Meer Schisme‥a culpable rupture or breach of the Catholick communion.    
\P 1778 R. LOWTH  \textit{Isaiah Notes} (ed. 12) 343 The inflictor of the punishment may perhaps be as culpable as the sufferer.    
\P 1844 THIRLWALL  \textit{Greece} VIII. lxii. 151 He was considered at Thebes as culpable.

\itembf{b.} Const. of, †in (an offence, sin, wrong, etc.).

\P 1340 HAMPOLE  \textit{Psalter} xxxiv. 13 Þai wild haf made me culpabil of syn.
\P c1380 WYCLIF  \textit{Wks.} (1880) 312 We ben coupable in þis synne.    
\P 1428 SURTEES  \textit{Misc.} (1890) 8 He was gylty and coulpabyll of all ye trespasse.    
\P 1545 BRINKLOW  \textit{Compl.} iii. (1874) 14 What can the pore wyfe‥do witthall, being not culpable in the cryme?    
\P 1653 H. COGAN  tr. \textit{Pinto's Trav.} lvi. 220 They had found themselves culpable of gluttony.    
\P 1839 JAMES  \textit{Louis XIV}, I. 222 The greatest crime of which a man could render himself culpable.

\itembf{c.} culpable of (punishment, death, judgement, etc.): deserving, liable to. Also, culpable to be judged, etc. (see first quot.).

\P 1380 WYCLIF  \textit{Serm. Sel. Wks.} I. 16 Sich is coupable aȝens God to be jugid to helle.    Ibid., Þat man, as Crist seiþ, is coupable of þe fier of helle.
\P c1450 MIROUR  \textit{Saluacioun} 4570 He is of the deth coupable.    
\P 1557 N. T. (Genev.)  \textit{Matt.} v. 21 Whosoeuer killeth, shal be culpable of iudgement.    
\P 1612 T. TAYLOR  \textit{Comm. Titus} i. 7 Culpable of iudgement.    
\P 1612 W. SCLATER  \textit{Minister's Portion} 45 [Which] makes the offender culpable of death.

\itembf{2.} Deserving blame or censure, blameworthy.

[\P 1386 CHAUCER  \textit{Melib.} \cardo{⁋}575 Þe lawe saith þat he is coupable þat entremettith him or mellith him with such þing as aperteyneþ not vnto him.]    
\P 1613 R. C. TABLE  \textit{Alph.} (ed. 3), Culpable, blame-worthy, guiltie.    
\P 1651 HOBBES  \textit{Leviath.} i. viii. 33 What circumstances make an action laudable, or culpable.    
\P 1789 BELSHAM  \textit{Ess.} I. i. 7 Those inclinations‥they know to be highly culpable and unworthy.    
\P 1875 J. CURTIS  \textit{Hist. Eng.} 146 With great and culpable disregard to the public weal.

\itembf{b.} Artistically faulty or censurable. rare.

\P 1768 W. GILPIN  \textit{Ess. Prints} 2 It [a print] may have an agreeable effect as a whole, and yet be very culpable in its parts.    
\P 1851 [SEE culpableness].

\itembf{B.} n. A guilty person, a culprit. Obs. [So F. coupable.]

\P 1480 ROBT.  \textit{Devyll} 720 in Hazl. E.P.P. I. 247 Euery vnthryftye culpable.    
\P 1483 CAXTON  \textit{Gold. Leg.} 411/3 He punysshed the culpables.    
\P 1651 tr.  \textit{De las-Coveras' Hist. Don Fenise} 209 If he could discover the infamous culpable.
\P a1734 NORTH  \textit{Lives} (1808) II. 246 (D.) Those only who were the culpables.
\end{myenumerate}


%%%%%%%%%%%%%%%%%%%%%%%%%%%%%%%%%
\myitem{curmudgeon} n.

\noindent \phonetic{(kɜːˈmʌdʒən)}

\noindent [Derivation unknown: see below.]

\noindent
‘An avaricious churlish fellow; a miser, a niggard’ (J.).

\P 1577 STANYHURST  \textit{Descr. Irel.} 102/2 in Holinshed, Such a clownish Curmudgen.    
\P 1593 NASHE  \textit{Christ's T.} 85 b, Our English Cormogeons, they haue breasts, but giue no suck.    
\P 1604 T. WRIGHT  \textit{Passions} v. 289 Why do covetous cormogions distill the best substance of their braines to get riches.    
\P 1626 W. SCLATER  \textit{Exp. 2 Thess.} (1629) 270 Curre-megients, who scarcely know any other sentence of Scripture, yet‥haue this of Paul in their mouthes; worke for your liuing.    
\P 1656 EARL OF MONMOUTH  \textit{Advt. fr. Parnass.} 387 Certain greedy curmuggions, who value not the leaving of a good name behind them to posterity.    
\P 1705 HICKERINGILL  \textit{Priest-cr.} i. (1721) 8 If‥the rich Curmudgeon‥do not open his Purse wide.    
\P 1824 W. IRVING  \textit{T. Trav.} I. 254, I had a rich uncle‥a penurious accumulating curmudgeon.    
\P 1860 G. J. WHYTE-MELVILLE  \textit{Holmby House} 377 A thankless old curmudgeon.

   The occurrence in Holland's Livy, 1600, of cornmudgin (q.v.) has led to a suggestion that this was the original form, with the meaning ‘concealer or hoarder of corn’, mudgin being associated with ME. much-en, mich-en to pilfer, steal, or muchier, Norman form of OF. mucier, musser to conceal, hide away. But examination of the evidence shows that curmudgeon was in use a quarter of a century before Holland's date, and that cornmudgin is apparently merely a nonce-word of Holland's, a play upon corn and curmudgeon. The suggestion that the first syllable is cur, the dog, is perhaps worthy of note; but that of Dr. Johnson's ‘unknown correspondent’, cœur méchant for F. méchant cœur, ‘evil or malicious heart’, is noticeable only as an ingenious specimen of pre-scientific ‘etymology’, and as having been retailed by Ash in the form, ‘from the French cœur unknown, and mechant a correspondent’!


%%%%%%%%%%%%%%%%%%%%%%%%%%%%%%%%%
\myitem{cursory} a.

\noindent \phonetic{(ˈkɜːsərɪ)}

\noindent [ad. L. cursōri-us of or pertaining to a runner or a race, f. cursōr-em runner: in OF. corsoire, cursoire.]
\vspace{-0.3cm}

\begin{myenumerate}

\itembf{1.} Running or passing rapidly over a thing or subject, so as to take no note of details; hasty, hurried, passing.

\P 1601 DENT  \textit{Pathw. Heauen} 277 Cursory saying of a few praiers a little before death, auaileth not.    
\P 1661 J. STEPHENS  \textit{Procurations} 128, I had only a cursory view of it, and that by chance.    
\P 1766 GOLDSM.  \textit{Vic. W.} xviii, A traveller who stopped to take a cursory refreshment.    
\P 1857 KEBLE  \textit{Eucharist. Adorat.} 37 Obvious to the most cursory reader of the Gospel.    
\P 1866 ROGERS  \textit{Agric. \& Prices} I. iii. 60 A cursory inspection shews that these statements are untrustworthy.

\itembf{2.} Moving about, travelling. Obs. rare.

\P 1606 \textit{Proc.  agst. Garnet} F (T.), Father Cresswell, legier jesuit in Spain; father Baldwin, legier in Flaunders‥besides their cursorie men, as Gerrard, etc.    
\P 1610 ROWLANDS  \textit{Martin Mark-all} 24 Their houses are made cursary like our Coaches with foure wheeles that may be drawne from place to place.    
\P 1650 FULLER  \textit{Pisgah} ii. iv. ii. 21 Those Tribes dwelt in their Tents‥in a cursory condition, only grazing their Cattel during the season.

\itembf{3.} Entom. Adapted for running; = cursorious.

\itembf{4.} In mediæval universities: \textbf{a.} cursory lectures: lectures of a less formal and exhaustive character delivered, especially by bachelors, as additional to the ‘ordinary’ lectures of the authorized teachers in a faculty, and at hours not reserved for these prescribed lectures.

   [The name would appear to have been first given to the lectures delivered by bachelors as part of the cursus prescribed for the licence, but to have been afterwards extended to all ‘extraordinary’ lectures.]

\P 1841 G. PEACOCK  \textit{Stat. Univ. Camb.} p. xliv. note 1.    
\P 1894 RASHDALL  \textit{Med. Universities} vi. §4. 426 The ‘cursory’ lectures of Paris are the ‘extraordinary’ lectures of Bologna.    Ibid. 427 Vacation cursory lectures might be given at any hour.    Ibid. It is probable that the term ‘cursory’ came to suggest also the more rapid and less formal manner of going over a book usually adopted at these times.

\itembf{b.} cursory bachelor: (in modern writers) a bachelor who gave cursory lectures.
\end{myenumerate}


%%%%%%%%%%%%%%%%%%%%%%%%%%%%%%%%%
\myitem{curt} a.

\noindent \phonetic{(kɜːt)}

\noindent [ad. L. curt-us cut or broken short, mutilated, abridged, which became in late L. and Romanic the ordinary word for ‘short’: It., Sp. corto, Pr. cort, F. court.

   The Latin adj. was app. adopted at an early date in Ger., giving OS. and OFris. curt (MDu. cort, Du., MLG., and LG. kort, whence also mod.Icel. korta, Sw. and Da. kort), OHG. kurt, kurz (MHG. and mod.Ger. kurz), where the word has taken the place of an original Teut. *skurt-, in OHG. scurz, in OE. scort, sceort, short. But the latter was retained in English.]
%\vspace{-0.3cm}

\begin{myenumerate}

\itembf{1.} Short in linear dimension; shortened.

\P 1665 SIR T. HERBERT  \textit{Trav.} (1677) 295 In more temperate climes hair is curt.    
\P 1840 LYTTON  \textit{Pilgr. of Rhine} xix, Thy limbs are crooked and curt.    
\P 1862 MERIVALE  \textit{Rom. Emp.} (1865) III. xxviii. 297 Plancus‥enacted the part of the sea-god Glaucus in curt cerulean vestments.

\itembf{b.} of things immaterial, modes of action, etc.

\P 1664 H. MORE  \textit{Myst. Iniq.} 351 For which curt reckoning Grotius has no excuse.    
\P 1675 TRAHERNE  \textit{Chr. Ethics} xx. 318 That vertue so curt and narrow, which we thought to be infinite.
\P a1677 BARROW  \textit{Serm.} (1687) I. xviii. 258 The most curt and compendious way of bringing about dishonest or dishonourable designs.    
\P 1874 REYNOLDS  \textit{John Bapt.} ii. 89 An angelic Spirit makes a more curt and much easier use than we can do of the functions of matter in its most etherial form.

\itembf{2.} Of words, sentences, style, etc.: Concise, brief, condensed, terse; short to a fault.

\P 1630 B. JONSON  \textit{New Inn} iii. i, What's his name? Fly. Old Peck. Tip. Maestro de campo, Peck! his name is curt, A monosyllable, but commands the horse well.    
\P 1645 MILTON  \textit{Tetrach.} (1851) 177 The obscure and curt Ebraisms that follow.    
\P 1791 BOSWELL  \textit{Johnson} (1887) III. 274 He could put together only curt frittered fragments of his own.    
\P 1814 D'ISRAELI  \textit{Amen. Lit.} (1867) 132 Their Saxon-English is nearly monosyllabic, and their phraseology curt.    
\P 1866 ROGERS  \textit{Agric. \& Prices} I. iii. 61 The dry and curt language of a petition in parliament.

\itembf{b.} So brief as to be wanting in courtesy or suavity.

\P 1831 DISRAELI  \textit{Yng. Duke} v. vii. (L.), ‘Ah! I know what you are going to say’, observed the gentleman in a curt, gruffish voice, ‘It is all nonsense.’    
\P 1863 GEO. ELIOT  \textit{Romola} (1880) I. Introd. 9 He might have been a little less defiant and curt, though, to Lorenzo de' Medici.
\end{myenumerate}


%%%%%%%%%%%%%%%%%%%%%%%%%%%%%%%%%
\myitem{cynosure} n.

\noindent \phonetic{(ˈsɪnəʊ-, ˈsaɪnəʊsjə(r), -zjʊə(r))}

\noindent [a. F. cynosure (16th c.), ad. L. cynosūra, a. Gr. κυνόσουρα dog's tail, Ursa Minor.]
\vspace{-0.3cm}

\begin{myenumerate}

\itembf{1.} The northern constellation Ursa Minor, which contains in its tail the Pole-star; also applied to the Pole-star itself.

\P 1596 C. FITZGEFFREY  \textit{Sir F. Drake} (1881) 14 Cynosure, whose praise the sea-man sings.    
\P 1612 DAVIES  \textit{Why Ireland, etc.} (1787) 199 The circuit of the Cinosura about the pole.    
\P 1627 MAY  \textit{Lucan} iii. (1631) 239 These Ships‥the Cynosure Guides straight along the sea.    
\P 1792 D. LLOYD  \textit{Voy. Life} iv. 72 The stedfast Cynosure renown'd at sea.

\itembf{2.} fig. \textbf{a.} Something that serves for guidance or direction; a ‘guiding star’.

\P 1596 C. FITZGEFFREY  \textit{Sir F. Drake} (1881) 33 The Cynosura of the purest thought, Faire Helicé, by whom the heart is taught.    
\P 1649 BP. HALL  \textit{Cases Consc.} (1650) 9 For the guidance of our either caution or liberty‥the onely Cynosure is our Charity.    
\P 1691 WOOD  \textit{Ath. Oxon.} I. 18 He hath written, The Rudiments of Grammar‥the Cynosura for many of our best Grammarians.    
\P 1809 MRS. WEST  \textit{Mother} (1810) 225 Thy victor-flag Flames like a steady cynosure.

\itembf{b.} Something that attracts attention by its brilliancy or beauty; a centre of attraction, interest, or admiration.

[\P 1599 BROUGHTON'S  \textit{Lett.} viii. 26 You Cynosura and Lucifer of nations, the stupor and admiration of the world.]    
\P 1601 BP. W. BARLOW  \textit{Serm. Paules Crosse} 64 Himselfe‥the Cynosure of their affections.    
\P 1632 MILTON  \textit{L'Allegro} 77 Some beauty‥The Cynosure of neighbouring eyes.    
\P 1837 CARLYLE  \textit{Fr. Rev.} I. ii. i, The fair young Queen‥the cynosure of all eyes.    
\P 1870 DISRAELI  \textit{Lothair} lxxxiii. 445 Before another year elapses Rome will be the cynosure of the world.
\end{myenumerate}





\end{description}
