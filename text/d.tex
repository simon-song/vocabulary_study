\chapter*{D}
%\markboth{VOCABULARY STUDY}{}
\markright{OED: D}{}
\addcontentsline{toc}{chapter}{OED: D}%
%\chapter[Oxford English Dictionary]{Vocabulary Study of Oxford English Dictionary}

%\setitemize{nosep}  % set no itemsep for itemize lists
%\leftmargini=7mm  % controls the relative spacing of description list

%%%%%%%%%%%%%%%%%%%%%%%%%%%%%%%%%%%%%%%%%%%%%%%%%%%%%%%%%%%%%%%%%%
\noindent The list of words is from Schur, \textit{1000 Most Important Words.}

%\begin{description}
\begin{description}[wide, labelwidth=!, labelindent=0pt] % noindent


%%%%%%%%%%%%%%%%%%%%%%%%%%%%%%%%
\myitem{dalliance} n.

\noindent \phonetic{(ˈdælɪəns)}

\noindent [f. dally v. + -ance: prob. formed in OFr. or AngloFr., though not yet recorded.]

\vspace{-0.3cm}

\begin{myenumerate}

\itembf{1.} Talk, confabulation, converse, chat; usually of a light or familiar kind, but also used of serious conversation or discussion. Obs.

\P c1340 \textit{Gaw. \& Gr.  Knt.} 1012 Þurȝ her dere dalyaunce of her derne wordez.
\P c1440  \textit{Promp. Parv.} 112 Dalyaunce, confabulacio, collocucio, colloquium.    
\P 1447 O. BOKENHAM  \textit{Seyntys} (Roxb.) 162 Marthe fyrst met hym [Christ]‥And hadde wyth hym a long dalyaunce.    
\P 1496 \textit{Dives \& Paup.} (W. de W.) vi. xv. 259/1 Redynge \& dalyaunce of holy wryt \& of holy mennes lyues.

\itembf{2.} Sport, play (with a companion or companions); esp. amorous toying or caressing, flirtation; often, in bad sense, wanton toying.

\P c1385 CHAUCER  \textit{L.G.W.} Prol. 332 (Cambr. MS) For to han with ȝou sum dalyaunce.
\P c1386 \textit{Doctor's T.} 66 At festes, reueles, and at daunces, That ben occasiouns of daliaunces.
\P c1400 MANDEVILLE  (Roxb.) xxvi. 124 Þai schall‥ete and drinke and hafe dalyaunce with wymmen.
\P a1553 UDALL  \textit{Royster D.} iv. vi. (Arb.) 70 Dyd not I for the nonce‥Read his letter in a wrong sense for daliance?    
\P 1602 SHAKES.  \textit{Ham.} i. iii. 50 Whilst like a puft and recklesse Libertine Himselfe the Primrose path of dalliance treads.    
\P 1725 POPE  \textit{Odyss.} viii. 348 The lewd dalliance of the queen of love.    
\P 1742 FIELDING  \textit{J. Andrews} iii. vi, He, taking her by the hand, began a dalliance.    
\P 1820 SCOTT  \textit{Monast.} xxiv, Julian‥went on with his dalliance with his feathered favourite.    
\P 1860 MOTLEY  \textit{Netherl.} (1868) I. vi. 346 The Earl's courtship of Elizabeth was anything‥but a gentle dalliance.

\itembf{3.} Idle or frivolous action, trifling; playing or trifling with a matter.

\P 1548 BECON  \textit{Solace of Soul Catechism} (1844) 571 In health and prosperity Satan's assaults seem to be but trifles and things of dalliance.    
\P 1561 T. NORTON  \textit{Calvin's Inst.} iii. xii. §1 When they come into the sight of God, such dalliances must auoide, bicause there is‥no trifling strife aboute wordes.    
\P 1627 F. E. HIST.  \textit{Edw. II} (1680) 16 Divine Justice, who admits no dalliance with Oaths.    
\P 1641 \textit{Lett.} in  Sir J. Temple \textit{Irish Rebell.} ii. 47 Now there is no dalliance with them; who‥declare themselves against the State.    
\P 1814 WORDSW.  \textit{Excursion} i. Wks. (1888) 423/2 Men whose hearts Could hold vain dalliance with the misery Even of the dead.    
\P 1843 PRESCOTT  \textit{Mexico} (1850) I. 63 He continued to live in idle dalliance.

\itembf{4.} Waste of time in trifling, idle delay. Obs.

   The first quot. prob. does not belong here: see delayance.

[\P c1340  \textit{Cursor M.} 26134 (Fairf.), \& for-þink his lange daliaunce [Cott. delaiance] þat he for-drawen has his penance.]    1547–64 Bauldwin Mor. Philos. (Palfr.) v. vi, Death deadly woundeth without dread or daliance.    
\P 1590 SHAKES.  \textit{Com. Err.} iv. i. 59 My businesse cannot brooke this dalliance.
\end{myenumerate}


%%%%%%%%%%%%%%%%%%%%%%%%%%%%%%%%%
\myitem{daub} v.

\noindent \phonetic{(dɔːb)}

\noindent [a. OF. daube-r:—L. dealbāre to whiten over, whitewash, plaster, f. de- down, etc. + albāre to whiten, f. albus white. The word had in OF. the senses ‘clothe in white, clothe, furnish, white-wash, plaster’; in later F. ‘to beat, swinge, lamme’ (Cotgr.); cf. curry, anoint, etc. All the English uses appear to come through that of ‘plaster’.]
\vspace{-0.3cm}

\begin{myenumerate}

\itembf{1.} trans. In building, etc.: To coat or cover (a wall or building) with a layer of plaster, mortar, clay, or the like; to cover (laths or wattle) with a composition of clay or mud, and straw or hay, so as to form walls. (Cf. dab v. 8.)

\P 1325  \textit{E.E. Allit. P.} B. 313 Cleme hit [the ark] with clay comly with-inne, \& alle þe endentur dryuen daube withouten.    
\P 1382 WYCLIF  \textit{Lev.} xiv. 42 With other cley the hows to be dawbid.    
\P 1483 \textit{Cath.  Angl.} 102 Dobe, linere, illinere.    
\P 1489 CAXTON  \textit{Faytes of A.} ii. xxxiv. 145 Thys bastylle muste be aduironned with hirdels aboute and dawbed thykke with erthe and clay thereupon.    
\P 1515 BARCLAY  \textit{Egloges} iv. (1570) Civ/1 Of his shepecote dawbe the walles round about.    
\P 1530 PALSGR. 507/2 Daube up this wall a pace with plaster‥I daube with lome that is tempered with heare or strawe.    
\P 1605 SHAKES.  \textit{Lear} ii. ii. 71, I will tread this vnboulted villaine into mortar, and daube the wall of a Iakes with him.
\P c1710 C. FIENNES  \textit{Diary} (1888) 169 Little hutts and hovels the poor Live in Like Barnes‥daub'd with mud-wall.    
\P 1877 \textit{N.W. Linc. Gloss.} 243 Stud and mud walling, building without bricks or stones, with posts and wattles, or laths daubed over with road-mud.

absol. \P 1523 FITZHERB.  \textit{Surv.} 37 He shall bothe thacke \& daube at his owne cost and charge.    
\P 1642 ROGERS  \textit{Naaman} 534 He falls to dawbing with untempered mortar.

fig. 1612–5 Bp. Hall Contempl., O.T. xii. vi, He‥is faine to dawbe up a rotten peace with the basest conditions.

\itembf{2.} To plaster, close up, cover over, coat with some sticky or greasy substance, smear.

\P 1597–8 Bp. Hall \textit{Sat.} vi. i. (R.), Whose wrinkled furrows‥Are daubed full of Venice chalk.    
\P 1614 \textit{Recoll. Treat.} 174 Take away this clay from mine eyes, wherewith alas they are so dawbed up.    
\P 1658 A. FOX tr. \textit{Wurtz' Surg.} ii. xxviii. 190 She had been plaistered and dawbed with Salves a long time.    
\P 1719 DE FOE  \textit{Crusoe} (1840) II. xv. 309 We daubed him all over‥with tar.    
\P 1832 LANDER  \textit{Adv. Niger} II. viii. 26 The women daub their hair with red clay.

fig. \P 1784 COWPER  \textit{Task} v. 360, I would not be a king to be‥daubed with undiscerning praise.

\itembf{b.} To smear or lay on (a moist or sticky substance). Also fig.

\P 1646 FULLER  \textit{Wounded Consc.} (1841) 289 For comfort daubed on will not stick long upon it.    
\P 1750 E. SMITH  \textit{Compl. Housewife} 309 With a fine rag daub it often on the face and hands.

\itembf{c.} To bribe, 'grease', slang (Cf. quot. 1876 in  DAUB n. 2.)

\P 1700 B. E. \textit{Dict.  Cant. Crew}, Dawbing, bribing.    
\P 1785 GROSE  \textit{Dict. Vulg. Tongue}, The cull was scragged [hanged] because he could not dawb.

\itembf{3.} To coat or cover with adhering dirt; to soil, bedaub. Also fig.

\P 1450  \textit{Knt. de la Tour} (1868) 31 Her heles, the whiche is doubed with filthe.    
\P 1535 JOYE  \textit{Apol. Tindale} 50 Dawbing eche other with dirte and myer.    
\P 1651 C. CARTWRIGHT  \textit{Cert. Relig.} i. 5 Such‥verities, as would have adorned, and not dawb'd the Gospel.    
\P 1661 PEPYS  \textit{Diary} 30 Sept., Having been very much daubed with dirt, I got a coach and home.    
\P 1721 DE FOE  \textit{Mem. Cavalier} (1840) 197 The fall plunged me in a puddle‥and daubed me.    1768–74 Tucker Lt. Nat. (1852) II. 596 Filthy metal that one could not touch without daubing one's fingers.    
\P 1840 DICKENS  \textit{Old C. Shop} iii, To daub himself with ink up to the roots of his hair.    
\P 1881 BESANT \& RICE  \textit{Chapl. of Fleet} i. xi. (1883) 89 My name is too deeply daubed with the Fleet mud; it cannot be cleansed.

\itembf{4.} To soil (paper) with ink, or with bad or worthless writing. Obs.

\P 1589 \textit{Marprel. Epit.} (1843) 6 When men have a gift in writing, howe easie it is for them to daube paper.
\P a1618 BRADSHAW  \textit{Unreas. Separation} (1640) 81 In the proofe of the Assumption he daubs sixe pages.    
\P 1792 SOUTHEY  \textit{Lett.} (1856) I. 7 The latter loss, to one who daubs so much, is nothing.

\itembf{5.} In painting: To lay on (colours) in a crude or clumsy fashion; to paint coarsely and inartistically. Also absol.

\P 1630 [SEE DAUBED].    
\P 1642 FULLER  \textit{Holy \& Prof. St.} v. x. 394 A trovell will serve as well as a pencill to daub on such thick course colours.    
\P 1695 DRYDEN tr.  \textit{Du Fresnoy's Art of Painting} (L.), A lame, imperfect piece, rudely daubed over with too little reflection, and too much haste.    
\P 1796 BURKE  \textit{Regic. Peace} i. Wks. VIII. 147 The falsehood of the colours which [Walpole] suffered to be daubed over that measure.    
\P 1840 HOOD  \textit{Up the Rhine} Introd. 4 It had been so often painted, not to say daubed, already.    
\P 1867 TROLLOPE  \textit{Chron. Barset} II. li. 77 He leaned upon his stick, and daubed away briskly at the background.

\itembf{6.} To cover (the person or dress) with finery or ornaments in a coarse, tasteless manner; to bedizen. Obs. or dial.

\P 1592 GREENE \& LODGE  \textit{Looking Glass} Wks. (Rtldg.) 124/2 My wife's best gown‥how handsomely it was daubed with statute-lace.    
\P 1639 tr.  \textit{Du Bosq's Compl. Woman} ii. 32 They dawb their habits with gold lace.    
\P 1760 WESLEY  \textit{Wks.} (1872) III. 13 A person hugely daubed with gold.    
\P 1876  \textit{Whitby Gloss.} s.v., Daub'd out, fantastically dressed.

\itembf{7.} fig. To cover with a specious exterior; to whitewash, cloak, gloss. Obs.

\P 1543 BECON  \textit{Agst. Swearing} Early Wks. (1843) 375 Perjury cannot escape unpunished, be it never so secretly handled and craftily daubed.    
\P 1594 SHAKES.  \textit{Rich. III,} iii. v. 29 So smooth he dawb'd his Vice with shew of Vertue.    
\P 1678 YOUNG  \textit{Serm. at Whitehall} 29 Dec. 31 To dawb and palliate our faults, is but like keeping our selves in the dark.    
\P 1683 tr.  \textit{Erasmus' Moriæ Enc.} 114 They dawb over their oppression with a submissive flattering carriage.    
\P 1785 [SEE DAUBED].

\itembf{b.} absol. or intr. To put on a false show; to dissemble so as to give a favourable impression. \itembf{c.} To pay court with flattery. Obs. or dial.

\P 1605 SHAKES.  \textit{Lear} iv. i. 53 Poore Tom's a cold. I cannot daub it further.    
\P 1619 W. WHATELY  \textit{God's Husb.} ii. (1622) 52 What auailed it Ananias and Saphira, to dawbe and counterfeit?    
\P 1619 W. SCLATER  \textit{Exp. 1 Thess.} (1630) 288 With such idle distinctions doe they dawbe with conscience.    
\P 1650 BAXTER  \textit{Saints' R.} iii. xiii. (1662) 508 Do not daub with men, and hide from them their misery or danger.
\P a1716 SOUTH  (J.), Let every one, therefore, attend the sentence of his conscience; for, he may be sure, it will not daub, nor flatter.    
\P 1876 \textit{Whitby  Gloss.}, Daubing‥paying court for the sake of advantage.    
\P 1877 \textit{Holderness Gloss.}, Daub, to flatter, or besmear with false compliment, with the object of gaining some advantage.
\end{myenumerate}


%%%%%%%%%%%%%%%%%%%%%%%%%%%%%%%%%
\myitem{dauntless} a.

\noindent \phonetic{(ˈdɔːntlɪs)}

\noindent [f. daunt v. (hardly from the n.) + -less.]
\vspace{-0.3cm}

Not to be daunted; fearless, intrepid, bold, undaunted.

\P 1593 SHAKES.  \textit{3 Hen. VI}, iii. iii. 17 Let thy dauntlesse minde still ride in triumph, Ouer all mischance.    
\P 1667 MILTON  \textit{P.L.} i. 603 Browes Of dauntless courage.    
\P 1761 GRAY  \textit{Fatal Sisters} 41 Low the dauntless Earl is laid.    
\P 1817 SCOTT  \textit{(title)}, Harold the Dauntless.    
\P 1874 GREEN  \textit{Short Hist.} viii. §5. 514 Laud was as dauntless as ever.

Hence \textbf{dauntlessly} adv., \textbf{dauntlessness}.

\P 1813 SHELLEY  \textit{Q. Mab} vii. 196 Therefore I rose, and dauntlessly began My lonely‥pilgrimage.    
\P 1730–6 Bailey (folio), \textit{Dauntlesness}, a being without Fear or Discouragement.    
\P 1876 BANCROFT  \textit{Hist. U.S.} VI. xlviii. 292 Shelby‥among the dauntless singled out for dauntlessness.




%%%%%%%%%%%%%%%%%%%%%%%%%%%%%%%%%
\myitem{dearth} n.

\noindent \phonetic{(dɜːθ)}

\noindent [ME. derþe, not recorded in OE. (where the expected form would be díerðu, díerð, dýrð: cf. 14th c. dierþe in Ayenb.); but corresp. formally to ON. dýrð with sense ‘glory’, OS. diuriđa, OHG. tiurida, MHG. tiûrde, MG. tûrde glory, honour, value, costliness; abstr. n. f. WGer. diuri, OE. díere, déore, dear a.1: see -th1.

   The form derke in Gen. \& Exod. (bis) and Promp. Parv. seems to be a scribal error for derþe, derðe; but its repeated occurrence is remarkable.]

\vspace{-0.3cm}

\begin{myenumerate}
\itembf{1.} Glory, splendour. Obs. rare. [= ON. dyrð.]

\P 1325  \textit{E.E. Allit.} P. A. 99 Þe derþe þerof for to deuyse Nis no wyȝ worþe that tonge berez.

\itembf{2.} Dearness, costliness, high price. Obs.

   (This sense, though etymologically the source of those that follow, is not exemplified very early, and not frequent. In some of the following instances it is doubtful.)

\P 1480 CAXTON  \textit{Chron. Eng.} cii. 82 Ther felle grete derth and scarsyte of corne and other vytailles in that land.    
\P 1596 BP. BARLOW  \textit{Three Serm.} i. 5 Dearth is that, when all those things which belong to the life of man‥are rated at a high price.]    
\P 1632 in  Cramond \textit{Ann. Banff} (1891) I. 67 Compleining of‥the dearthe of the pryce thairof.    
\P 1644 R. BAILLIE  \textit{Lett. \& Jrnls.} (1841) II. 175, I cannot help the extraordinarie dearth: they say the great soume the author putts on his copie, is the cause of it.    
\P 1793 BENTHAM  \textit{Emanc. Colonies} Wks. 1843 IV. 413  When an article is dear, it is‥made so by freedom or by force. Dearth which is natural is a misfortune: dearth which is created is a grievance.

fig. \P 1602 SHAKES.  \textit{Ham.} v. ii. 123 His infusion of such dearth and rareness.

\itembf{3.} A condition in which food is scarce and dear; often, in earlier use, a time of scarcity with its accompanying privations, a famine; now mostly restricted to the condition, as in time of dearth.

\P c1250 \textit{Gen. \& Ex.}  2237 Wex derk [? derþe], ðis coren is gon.    Ibid. 2345.
\P a1300  \textit{Cursor M.} 4700 (Cott.) Sua bigan þe derth to grete.
\P c1400 MANDEVILLE  (Roxb.) vi. 20 If any derth com in þe cuntree [\textit{quant il fait chier temps}].
\P c1440  \textit{Promp. Parv.} 119 Derthe (P. or derke), cariscia.    
\P 1526 TINDALE  \textit{Luke} xv. 14 There rose a greate derth thorow out all that same londe.    
\P 1552  \textit{Bk. Com. Prayer, Litany}, In the tyme of dearth and famine.    
\P 1590 SPENSER  \textit{F.Q.} i. ii. 27 Dainty they say maketh derth.    
\P 1606 SHAKES.  \textit{Ant. \& Cl.} ii. vii 22 They know‥If dearth Or Foizon follow.    
\P 1625 BACON  \textit{Ess. Seditions} (Arb.) 403 The Causes and Motiues of Seditions are‥Dearths: Disbanded Souldiers.
\P a1687 PETTY  \textit{Pol. Arith.} (1690) 80 The same causes which make Dearth in one place do often cause plenty in another.    
\P 1781 GIBBON  \textit{Decl. \& F.} III. li. 217 The fertility of Egypt supplied the dearth of Arabia.    
\P 1841 W. SPALDING  \textit{Italy \& It. Isl.} I. 361 Augustus in a dearth, gave freedom to twenty thousand slaves.    
\P 1848 MILL  \textit{Pol. Econ.} (1857) II. iv. ii. 270 In modern times, therefore, there is only dearth, where there formerly would have been famine.

\itembf{b.} of (for) corn, victuals, etc.

\P c1400 MANDEVILLE  (Roxb.) vi. 23 Þer falles oft sithes grete derth of corne [chier temps].    
\P 1538 STARKEY  \textit{England} ii. i. 174 The darth of al such thyngys as for fode ys necessary.    
\P 1556  \textit{Chron. Gr. Friars} (Camden) 33 This yere [1527] was a gret derth in London for brede.    Ibid. 45 This yere was a gret derth for wode and colles.    
\P 1720 GAY  \textit{Poems} (1745) I. 139 At the dearth of coals the poor repine.    
\P 1721 SWIFT  \textit{Let. fr. Lady conc. Bank} Wks. (1841) II. 67 The South-Sea had occasioned such a dearth of money in the kingdom.

\itembf{4.} fig. and transf. Scarcity of anything, material or immaterial; scanty supply; practical deficiency, want or lack of a quality, etc.

\P 1340  \textit{Ayenb.} 256 Þe meste dierþe þet is aboute ham is of zoþnesse an of trewþe.
\P c1386 CHAUCER  \textit{Pars. T.} ⁋340 Precious clothyng is cowpable for the derthe of it.
\P c1477 CAXTON  \textit{Jason} 42 b, Ther is no grete derthe ne scarcete of women.    
\P 1596 DRAYTON  \textit{Legends} iv. 45 A time when never lesse the Dearth Of happie Wits.    
\P 1667 DRYDEN  \textit{Ess. Dram. Poesie} Wks. 1725 I. 55 That dearth of plot and narrowness of Imagination, which may be observed in all their Plays.    
\P 1671 C. HATTON in  \textit{Hatton Corr.} (1878) 60 The absence of ye Court occasions a great dirth of news here.    
\P 1754 RICHARDSON  \textit{Grandison} IV. xvii. 130 We live in an age in which there is a great dearth of good men.    
\P 1815 WORDSW.  \textit{White Doe} ii. 8 Her last companion in a dearth Of love.    
\P 1875 J. CURTIS  \textit{Hist. Eng.} 151 The great pestilence of 1349 led  to such a dearth of labourers.
\end{myenumerate}


%%%%%%%%%%%%%%%%%%%%%%%%%%%%%%%%%
\myitem{debacle} n.

\noindent \phonetic{(dɪˈbɑːk(ə)l)}

\noindent [a. F. débâcle, vbl. n. from débâcler to unbar, remove a bar, f. dé- = des- (see de- I. 6) + bâcler to bar.]
\vspace{-0.3cm}

\begin{myenumerate}

\itembf{1.} A breaking up of ice in a river; in Geol. a sudden deluge or violent rush of water, which breaks down opposing barriers, and carries before it blocks of stone and other debris.

\P 1802 PLAYFAIR  \textit{Illustr. Hutton. Th.} 402 Valleys are so particularly constructed as to carry with them a still stronger refutation of the existence of a debacle.    
\P 1823 W. BUCKLAND  \textit{Reliq. Diluv.} 158 They could have been transported by no other force than that of a tremendous deluge or debacle of water.    
\P 1893  \textit{Daily Tel.} 1 Feb., The debacle in the United States‥Telegrams state that the breaking up of the ice is being attended with great damage.

\itembf{2.} transf. and fig. A sudden breaking up or downfall; a confused rush or rout, a stampede.

\P 1848 THACKERAY  \textit{Van. Fair} xxxii, The Brunswickers were routed and had fled‥It was a general débâcle.    
\P 1887 \textit{Graphic}  15 Jan. 59/2 In the nightly débâcle [he] is often content to stand aside.
\end{myenumerate}


%%%%%%%%%%%%%%%%%%%%%%%%%%%%%%%%%
\myitem{debase} v.

\noindent \phonetic{(dɪˈbeɪs)}

\noindent [Formed in 16th c. from de- I. 1, 3 + base v.1: cf. ABASE.]
\vspace{-0.3cm}

\begin{myenumerate}

\itembf{1.} trans. To lower in position, rank, or dignity; to abase. Obs.

\P 1568 GRAFTON  \textit{Chron.} II. 69 The king hath debased himselfe ynough to the Bishop.    Ibid. II. 75 Debasyng himselfe with great humilitie and submission before the sayde two Cardinalles.    
\P 1593 SHAKES.  \textit{Rich. II,} iii. iii. 190 Faire Cousin, you debase your Princely Knee, To make the base Earth prowd with kissing it.    
\P 1610 HEALEY  \textit{St. Aug. Citie of God} iii. xvi. (1620) 121 Brutus debased Collatine and banished him the city.    
\P 1648 WILKINS  \textit{Math. Magick} i. i. 4 The ancient Philosophers‥refusing to debase the principles of that noble profession unto Mechanical experiments.    
\P 1671 MILTON  \textit{Samson} 999 God sent her to debase me.    
\P 1751 JOHNSON  \textit{Rambler} No. 187 ⁋4 A man [in Greenland] will not debase himself by work, which requires neither skill nor courage.    
\P 1827 POLLOK  \textit{Course T.} v, Debased in sackcloth, and forlorn in tears.

\itembf{2.} To lower in estimation; to decry, depreciate, vilify. Obs.

\P 1565 T. STAPLETON  \textit{Fortr. Faith} 62 The Manichee‥would so extol grace, and debace the nature of man.    
\P 1600 HOLLAND  \textit{Livy} ix. xxxvii. 341 Praising highly‥the Samnites warres, debasing the Tuscanes.    
\P 1704 J. BLAIR in  W. S. Perry \textit{Hist. Coll. Amer. Col.} Ch. I. 98, I have heard him often debase and vilify the Gentlemen of the Council, using to them the opprob[r]ious names of Rogue, Rascal [etc.].    
\P 1746 HERVEY  \textit{Medit.} (1818) 15 Why should we exalt ourselves or debase others?

\itembf{3.} To lower in quality, value, or character; to make base, degrade; to adulterate. \itembf{b.} spec. To lower the value of (coin) by the mixture of alloy or otherwise; to depreciate.

\P 1591 SPENSER  \textit{Tears of Muses, Urania} iii, Ignorance‥That mindes of men borne heavenlie doth debace.    
\P 1602 W. FULBECKE  \textit{1st Pt. Parall.} 54 Or els it may be changed in the value, as if a Floren, which was worth 4 li to be debased to 3 li.    
\P 1606 \textit{State  Trials, Gt. case of Impositions} (R.), That these staple commodities might not be debased.    
\P 1751 JOHNSON  \textit{Rambler} No. 168 ⁋4 Words which convey ideas of dignity‥are in time debased.    
\P 1789 \textit{Trans.  Soc. Encourag. Arts} I. 16 Much of the Zaffre brought to England is mixed with matters that debase its quality.    
\P 1879 FROUDE  \textit{Cæsar} xiii. 177 Laws against debasing the coin.
\end{myenumerate}


%%%%%%%%%%%%%%%%%%%%%%%%%%%%%%%%%
\myitem{debilitate} v.

\noindent \phonetic{(dɪˈbɪlɪteɪt)}

\noindent [f. L. dēbilitāt-, ppl. stem of dēbilitāre to weaken, f. dēbilis weak.]
\vspace{-0.3cm}

\begin{myenumerate}

\itembf{a.} trans. To render weak; to weaken, enfeeble.

\P 1533 ELYOT  \textit{Cast. Helthe} (1541) 46 a, Immoderate watch‥doth debilitate the powers animall.    
\P 1541 PAYNEL  \textit{Catiline} xlv. 71 To debylitate and cutte asunder theyr endeuoir and hope.
\P a1625 BEAUM. \& FL.  \textit{Faithful Friends} v. ii, If you think His youth or judgment‥Debilitate his person‥call him home.    
\P 1717 BULLOCK  \textit{Woman a Riddle} i. i. 8, I am totally debilitated of all power of elocution.    
\P 1715 LEONI  \textit{Palladio's Archit.} (1742) I. 57 The Sun shining‥would be apt to heat, debilitate, and spoil the Wine or other Liquors.    
\P 1829 I. TAYLOR  \textit{Enthus.} ix. 233 Whose moral sense had been debilitated.    
\P 1871 G. H. NAPHEYS  \textit{Prev. \& Cure Dis.} i. i. 45 A feeble constitution, which he further debilitated by a dissipated life.

\itembf{b.} Astrol. Cf. DEBILITY 4 b. Obs.

\P 1625 BEAUM. \& FL.  \textit{Bloody Bro.} iv. ii, Venus‥is‥clear debilitated five degrees Beneath her ordinary power.
\end{myenumerate}


%%%%%%%%%%%%%%%%%%%%%%%%%%%%%%%%%
\myitem{debunk} v. orig. U.S.

\noindent \phonetic{(dɪˈbʌŋk)}

\noindent [f. de- II. 2 + bunk n.4]
\vspace{-0.3cm}

trans. To remove the ‘nonsense’ or false sentiment from; to expose (false claims or pretensions); hence, to remove (a person) from his ‘pedestal’ or ‘pinnacle’. Also absol. Hence \textbf{debunker}, one who debunks; 
\textbf{debunking} vbl. n. and ppl. a.

\P 1923 W. E. WOODWARD  \textit{Bunk} i. 2 De-bunking means simply taking the bunk out of things.    Ibid., I'm a professional de-bunker.    Ibid. 4 To keep the United States thoroughly de-bunked would require the continual services of‥half a million persons.    Ibid., Just how do you go about your de-bunking operations?    Ibid. 6 Recently we de-bunked the head of a large financial institution.    
\P 1927  \textit{Daily Express} 21 Nov. 2/3 The Thucydidean school of what are known as ‘debunking’ historians.    
\P 1927  \textit{Brit. Weekly} 29 Dec. 327/2 The somewhat ruthless process which in America is called ‘debunking’—that is, pricking pretentious bubbles [etc.].    
\P 1930  \textit{Times} Lit. Suppl. 6 Mar. 174 The present fashion for ‘debunking’ great men.    Ibid. 13 Mar. 217 He is not indeed a ‘debunker’, but he is as far from being a blind hero-worshipper.    Ibid. 8 May 378 The aim of ‘debunking’ a reputation that has been swollen by the uncritical eulogies of contemporaries.  
\P 1934 \textit{Municipal  Engineering} 12 July 31/1 The London C.C. has decided to ‘debunk’ Waterloo Bridge, or, in other words, to take away the bunkum that has been attached to it.    
\P 1940  \textit{Illustr. Lond. News} CXCVI. 758/2 In fact, he is a reverent man, who enjoys ‘debunking’ the ‘debunkers’, if that word may be taken now as acceptable and established English.    
\P 1948  \textit{Sat. Rev.} 26 June 13/1 In dealing with military reputations, the author neither glorifies nor debunks.    
\P 1958  \textit{Spectator} 13 June 777/1 It is his duty‥to debunk the claims of the Fabians.    
\P 1960  \textit{Guardian} 10 Dec. 5/3 No cynic, but a debunker.



%%%%%%%%%%%%%%%%%%%%%%%%%%%%%%%%%
\myitem{decimate} v.

\noindent \phonetic{(ˈdɛsɪmeɪt)}

\noindent [f. L. decimā-re to take the tenth, f. decim-us tenth: see -ate3. Cf. F. décimer (16th c.).]
\vspace{-0.3cm}

\begin{myenumerate}

\itembf{1.} To exact a tenth or a tithe from; to tax to the amount of one-tenth. Obs. In Eng. Hist., see decimation 1.

\P 1656 in  BLOUNT \textit{Glossogr.}
\P 1657 MAJOR-GEN. DESBROWE  \textit{Sp. in Parlt.} 7 Jan., Not one man was decimated but who had acted or spoken against the present government.    
\P 1667 DRYDEN  \textit{Wild Gallant} ii. i, I have heard you are as poor as a decimated Cavalier.    
\P 1670 PENN  \textit{Lib. Consc. Debated} Wks. 1726 I. 447  The insatiable Appetites of a decimating Clergy.    
\P 1738 NEAL  \textit{Hist. Purit.} IV. 96 That all who had been in arms for the king‥should be decimated; that is pay a tenth part of their estates.
\P a1845 [see DECIMATED].

\itembf{2.} To divide into tenths, divide decimally. Obs.

\P 1749 SMETHURST in  \textit{Phil. Trans.} XLVI. 22 The Chinese‥are so happy as to have their Parts of an Integer in their Coins, \&c. decimated.

\itembf{3.} Milit. To select by lot and put to death one in every ten of (a body of soldiers guilty of mutiny or other crime): a practice in the ancient Roman army, sometimes followed in later times.

\P 1600 J. DYMMOK  \textit{Treat. Ireland} (1843) 42 All‥were by a martiall courte condemned to dye, which sentence was yet mittigated by the Lord Lieutenants mercy, by which they were onely decimated by lott.    
\P 1651  \textit{Reliq. Wotton.} 30 In Ireland‥he [Earl of Essex] decimated certain troops that ran away, renewing a peece of the Roman Discipline.    
\P 1720 OZELL  \textit{Vertot's Rom. Rep.} I. iii. 185 Appius decimated, that is, put every Tenth Man to death among the Soldiers.    
\P 1840 NAPIER  \textit{Penins. War} VI. xxii. v. 293 The soldiers could not be decimated until captured.    
\P 1855 MACAULAY  \textit{Hist. Eng.} IV. 577 Who is to determine whether it be or be not necessary‥to decimate a large body of mutineers?

\itembf{4.} transf. \textbf{a.} To kill, destroy, or remove one in every ten of. \textbf{b.} rhetorically or loosely. To destroy or remove a large proportion of; to subject to severe loss, slaughter, or mortality.

\P 1663 J. SPENCER  \textit{Prodigies} (1665) 385 The‥Lord‥sometimes decimates a multitude of offenders, and discovers in the personal sufferings of a few what all deserve.    
\P 1812 W. TAYLOR in  \textit{Monthly Rev.} LXXIX. 181 An expurgatory index, pointing out the papers which it would be fatiguing to peruse, and thus decimating the contents into legibility.    
\P 1848 C. BRONTË  \textit{Let.} in Mrs. Gaskell \textit{Life} 276 Typhus fever decimated the school periodically.    
\P 1875 LYELL  \textit{Princ. Geol.} II. iii. xlii. 466 The whole animal Creation has been decimated again and again.    
\P 1877 FIELD  \textit{Killarney to Golden Horn} 340 This conscription weighs very heavily on the Mussulmen‥who are thus decimated from year to year.    
\P 1883 L. OLIPHANT  \textit{Haifa} (1887) 76 Cholera‥was then decimating the country.

Hence \textbf{decimated}, \textbf{decimating} ppl. adjs.

\P 1661 MIDDLETON  \textit{Mayor of Q. Pref.}, Now whether this magistrate fear'd the decimating times.
\P 1667,1670 [SEE 1].
\P a1845 SYD. SMITH  \textit{Wks.} (1850) 688 The decimated person.
\end{myenumerate}


%%%%%%%%%%%%%%%%%%%%%%%%%%%%%%%%%
\myitem{déclassé} a. and n.

\noindent \phonetic{(deklase)}

\noindent [Fr., pa. pple. of déclasser declass v.]
\vspace{-0.3cm}

\textbf{A.} adj. Reduced or degraded from one's social class; having come down in the world. \textbf{B.} n. One who has been so reduced or degraded.

\P 1887 \textit{Fortn.  Rev.} Aug. 227 It is only the déclassé, the ne'er-do-well, or the really unfortunate, who has nothing to call his own.    
\P 1905  \textit{Spectator} 28 Jan. 144/2 Pamela‥quits the company of artists and actresses, declassés and divorcées.    
\P 1921 \textit{Glasgow  Herald} 3 Aug., The attempt by a body of declassés to form the policy of the entire working-class of this country.    
\P 1921  \textit{Times} Lit. Suppl. 29 Sept. 626/2 A girl of any family may, by force of circumstances, become déclassée.    
\P 1961 A. WILSON  \textit{Old Men at Zoo} ii. 97 Déclassé nations are very touchy.





%%%%%%%%%%%%%%%%%%%%%%%%%%%%%%%%%
\myitem{decorous} a.

\noindent \phonetic{(dəˈkɔərəs, ˈdɛkərəs)}

\noindent [In form ad. late L. decorōs-us elegant, beautiful (It. decoroso decorous, decent), f. decus, decor-: see decorate; but in sense corresp. to L. decōr-us becoming, seemly, fitting, proper, f. decor, decōr-em becomingness, f. decēre to become, befit. In harmony with this Johnson, Walker, and Smart 1849 pronounce 
\phonetic{deˈcōrous}. Bailey 1730 and Perry 1805 have \phonetic{ˈdecŏrous}; Craig 1847 and later dictionaries record both. The word is not very frequent colloquially.]
\vspace{-0.3cm}

\begin{myenumerate}

\itembf{1.} Seemly, suitable, appropriate. Obs.

\P 1664 H. MORE  \textit{Myst. Iniq.} 225 That decorous embellishment in the external Cortex of the Prophecy [is] punctually observed.    
\P 1680 \textit{Apocal. Apoc.} 75 So decorous is the representation.    
\P 1691 RAY  \textit{Creation} i. (1704) 57 It is not so decorous with respect to God, that he should immediately do all the meanest and triflingest things himself, without any inferiour or subordinate minister.

\itembf{2.} Characterized by decorum or outward conformity to the recognized standard of propriety and good taste in manners, behaviour, etc.

\P [1673 \textit{Rules  of Civility} 144 It is not decorous to look in the Glass, to comb, brush, or do any thing of that nature to ourselves, whilst the said person be in the Room.]    
\P 1792 V. KNOX  \textit{Serm.} ix. (R.), Individuals, who support a decorous character.    
\P 1795 BURKE  \textit{Corr.} (1844) IV. 291 Their language‥is cool, decorous, and conciliatory.    
\P 1821 BYRON  \textit{Vis. Judgm.} xcv, Some grumbling voice, Which now and then will make a slight inroad Upon decorous silence.    
\P 1858 HAWTHORNE  \textit{Fr. \& It. Jrnls.} I. 293 Washington, the most decorous and respectable personage that ever went ceremoniously through the realities of life.    
\P 1874 HELPS  \textit{Soc. Press.} iii. 40 In a great city everything has to be made outwardly decorous.

\itembf{b.} Of language: Exemplifying propriety of diction.

\P 1873 LOWELL  \textit{Among my Bks.} Ser. ii. 224 A treatise of permanent value for philosophic statement and decorous English.

\noindent Explained in the sense of L. \textit{decorōsus}.

\P 1727 BAILEY  vol. II, Decorous, Decorose, fair and lovely, beautiful, graceful, comely.
\end{myenumerate}


%%%%%%%%%%%%%%%%%%%%%%%%%%%%%%%%%
\myitem{decorum} n.

\noindent \phonetic{(dɪˈkɔərəm)}

\noindent [a. L. decōrum that which is seemly, propriety; subst. use of neuter sing. of decōr-us adj. seemly, fitting, proper. So mod.F. décorum (since 16th c.).]
\vspace{-0.3cm}

\begin{myenumerate}

\itembf{1.} That which is proper, suitable, seemly, befitting, becoming; fitness, propriety, congruity. \textbf{a.} esp. in dramatic, literary, or artistic composition: That which is proper to a personage, place, time, or subject in question, or to the nature, unity, or harmony of the composition; fitness, congruity, keeping. Obs.

\P 1568 R. ASCHAM  \textit{Scholem.} (Arb.) 139 Who soeuer hath bene diligent to read aduisedlie ouer, Terence, Seneca, Virgil, Horace‥he shall easelie perceiue, what is fitte and decorum in euerie one.    
\P 1576 FOXE  \textit{A. \& M.} 990/1, I‥lay all the wyte in maister More, the authour and contriuer of this Poeticall booke, for not kepyng Decorum personæ, as a perfect Poet should haue done.    Ibid., Some wyll thinke‥maister More to haue missed some part of his Decorum in makyng the euill spirite‥to be messenger betwene middle earth and Purgatory.    
\P 1621 BURTON  \textit{Anat. Mel.} ii. ii. vi. iv, If that Decorum of time and place‥be observed.    
\P 1644 MILTON \textit{Educ.} Wks. 1738 I. 140  What the Laws are of a true Epic Poem, what of a Dramatic, what of a Lyric, what Decorum is, which is the grand master-piece to observe.    
\P 1686 W. AGLIONBY  \textit{Painting Illust.} ii. 67 Simon Sanese began to understand the Decorum of Composition.    Ibid. iii. 119 The second part of Invention is Decorum; that is, that there be nothing Absurd nor Discordant in the Piece.    
\P 1704 HEARNE  \textit{Duct. Hist.} (1714) I. 132 Neither is a just Decorum always observ'd, for he sometimes makes Blockheads and Barbarians talk like Philosophers.    
\P 1756 J. WARTON  \textit{Ess. Pope} I. i. 5 Complaints‥[which] when uttered by the inhabitants of Greece, have a decorum and consistency, which they totally lose in the character of a British shepherd.

\itembf{b.} That which is proper to the character, position, rank, or dignity of a real person. arch.

\P 1589 PUTTENHAM  \textit{Eng. Poesie} iii. xxiv. (Arb.) 303 Our soueraign Lady (keeping alwaies the decorum of a Princely person) at her first comming to the crowne, etc.    
\P 1594 J. DICKENSON  \textit{Arisbas} (1878) 87 The minde of man degenerating from the decorum of humanitie becomes monstrous.    
\P 1606 SHAKES.  \textit{Ant. \& Cl.} v. ii. 17 Maiesty to keepe decorum, must No lesse begge then a Kingdome.    
\P 1683 CAVE  \textit{Ecclesiastici, Athanasius} 171 He was a Prince of a lofty Mind, careful to preserve the Decorum of State and Empire.
\P a1715 BURNET  \textit{Own Time} (1766) I. 130 He‥did not always observe the decorum of his post.    
\P 1848 MACAULAY  \textit{Hist. Eng.} I. 180 It was necessary to the decorum of her character that she should admonish her erring children.

\itembf{c.} That which is proper to the circumstances or requirements of the case: seemliness, propriety, fitness; = decency 1. arch.

\P 1586 T. B. \textit{La Primaud. Fr. Acad.} i. 171 A waie how to frame all things according to that which is decent or seemely, which the Latines call decorum.    
\P 1598 J. DICKENSON  \textit{Greene in Conc.} (1878) 147 She deemd it no decorum to blemish her yet-during pleasures with not auailing sorrow.    
\P 1677 GALE  \textit{Crt. Gentiles} II. iv. 19 Temperance formally consistes in giving al persons and things their just decorum and measure.    
\P 1809 MATHIAS in  \textit{Gray's Corr.} (1843) 16 There was a peculiar propriety and decorum in his manner of reading.    
\P 1858 TRENCH  \textit{Parables} (1860) 126 They argue that it is against the decorum of the Divine teaching, that, etc.

\itembf{2.} Qualities which result from sense 1: \textbf{a.} Beauty arising from fitness, or from absence of the incongruous; comeliness; grace; gracefulness.

\P 1613 R. C. \textit{Table  Alph.} (ed. 3), Decorum, comelinesse.    
\P 1618 DEKKER  \textit{Owles Almanacke}, A coloured cloute will set the stampe of decorum on a rotten partition.    
\P 1635 SWAN  \textit{Spec. M.} vii. §3 (1643) 320 To shew the due decorum and comely beauty of the worlds brave structure.    
\P 1729 G. SHELVOCKE  \textit{Artillery} v. 334 The Decorum and Gracefulness of any Pile, the making the whole Aspect of a Fabric so correct.

\itembf{b.} Orderly condition, orderliness. Obs.

\P 1610 HEALEY  \textit{St. Aug. Citie of God} xii. xxv. 442 Whose wisedome reacheth from end to end, ordering all in a delicate decorum.    Ibid. xxii. xxiv. 847 And brings the potentiall formes into such actuall decorum.    
\P 1684 T. BURNET  \textit{Th. Earth} i. 132 The first orders of things are more perfect and regular, and this decorum seems to be observ'd afterwards.

\itembf{c.} Orderly and grave array. Obs.

\P 1634 SIR T. HERBERT  \textit{Trav.} (1638) 238 In this Decorum they march slowly, and with great silence [at a funeral].

\itembf{3.} Propriety of behaviour; what is fitting or proper in behaviour or demeanour, what is in accordance with the standard of good breeding; the avoidance of anything unseemly or offensive in manner.

\P 1572 tr.  \textit{Buchanan's Detect. Mary M} iij a, To obserue decorum and comely conuenience in hir pairt‥sche counterfeiteth a mourning.
\P a1628 F. GREVILLE  \textit{Sidney} (1652) 93 She resolved to keep within the Decorum of her sex.    
\P 1668 DRYDEN  \textit{Evening's Love Epil.} 19 Where nothing must decorum shock.    
\P 1704 F. FULLER  \textit{Med. Gymn.} (1711) 143, I can't see any breach of Decorum, if a Lady‥should ride on Horse-back.    
\P 1791 MRS. RADCLIFFE  \textit{Rom. Forest} iii, The lady-abbess was a woman of rigid decorum and severe devotion.    
\P 1803  \textit{Med. Jrnl.} IX. 442 A spirit of levity and wrangling, wholly inconsistent with the grave decorum due to the investigation and decision of a philosophical subject.    
\P 1814 JANE AUSTEN  \textit{Mansf. Park} (1851) 81 My father‥would never wish his grown-up daughters to be acting plays. His sense of decorum is strict.    
\P 1866 G. MACDONALD  \textit{Ann. Q. Neighb.} xxvii. (1878) 475 If the mothers‥are shocked at the want of decorum in my friend Judy.

\itembf{4.} (with a and pl.) \textbf{a.} A fitting or appropriate act. Obs.

\P 1601 A. C. \textit{Answ. to Let. Jesuited Gent.} 114 (Stanf.) It had bin a decorum in them, to have shewd themselves thankful unto such kind office.    
\P 1692 DRYDEN  \textit{St. Evremont's Ess.} 372 The Laugh, the Speech, the Action, accompanied with Agreements and Decorums.    
\P 1717 BERKELEY  \textit{Tour Italy} 21 Jan. Wks. 1871 IV. 532  The tragedy of Caligula, where, amongst other decorums, Harlequin‥was very familiar with the Emperor himself.

\itembf{b.} An act or requirement of polite behaviour; a decorous observance; chiefly in pl., proprieties.

\P 1601 R. JOHNSON  \textit{Kingd. \& Commw.} (1603) 245 The Spanish nation‥using a certaine decorum (which they call an obeysance or‥a compliment or cerimonious curtesie).    
\P 1676 WYCHERLEY  \textit{Pl. Dealer} i. i, Tell not me‥of your Decorums, supercilious Forms, and slavish Ceremonies.    
\P 1706 ESTCOURT  \textit{Fair Examp.} i. i, My Lady Stately longs to see you, had paid you a Visit but for the Decorums: She expects the first from you.    
\P 1766 GOLDSM.  \textit{Vic. W.} xxx, No decorums could restrain the impatience of his blushing mistress to be forgiven.    
\P 1865 MERIVALE  \textit{Rom. Emp.} VIII. lxvi. 202 The dignity of his military character was hedged round by formalities and decorums.
\end{myenumerate}


%%%%%%%%%%%%%%%%%%%%%%%%%%%%%%%%%
\myitem{decry} v.

\noindent \phonetic{(dɪˈkraɪ)}

\noindent [a. F. décrier, in 14th c. descrier, f. des-, de- (see de- I. 6) + crier to cry. In Eng. the prefix appears always to have been taken in sense ‘down’: see de- I. 4.]
\vspace{-0.3cm}

\begin{myenumerate}

\itembf{1.} trans. To denounce, condemn, suppress, or depreciate by proclamation; = cry down (cry v. 17 a); chiefly said of foreign or obsolete coins; also to bring down the value (of any article) by the utterance or circulation of statements.

\P 1617 MORYSON  \textit{Itin.} i. iii. vi. 289 Having a singular Art to draw all forraine coynes when they want them, by raising the value, and in like sort to put them away, when they haue got abundance thereof, by decrying the value.    
\P 1633 T. STAFFORD  \textit{Pac. Hib.} iv. (1821) 267 The calling downe, and decrying of all other Moneys whatsoever.    
\P 1697 EVELYN  \textit{Numism.} vi. 204 Many others [medals of Elagabalus] decried and called in for his infamous life.    
\P 1710 WHITWORTH  \textit{Acc. Russia} (1758) 80 Next year‥the‥gold‥was left without refining, which utterly decried those Ducats.    
\P 1765 BLACKSTONE  \textit{Comm.} I. 278 The king may‥decry, or cry down, any coin of the kingdom, and make it no longer current.    
\P 1844  \textit{Act} 7–8 Vict. c. 24 §4 Spreading‥any false rumour, with intent to enhance or decry the price of any goods.

\itembf{2.} To cry out against; to disparage or condemn openly; to attack the credit or reputation of; = cry down (CRY 17 b).

\P 1641 J. JACKSON  \textit{True Evang. T.} i. 75 We goe‥to law one with another (which S. Paul so decryed).    
\P 1660 R. COKE  \textit{Justice Vind.} Pref. 1 All men‥have with one voice commended Virtue, and decried Vice.    
\P 1665 PEPYS  \textit{Diary} 27 Nov., The goldsmiths do decry the new Act.    
\P 1756 C. LUCAS  \textit{Ess. Waters} I. Pref., ‘Who is this’, says one, ‘that is come to decry our waters?’    
\P 1867 LEWES  \textit{Hist. Philos.} II. 105 He does not so much decry Aristotle, as the idolatry of Aristotle.    
\P 1872 YEATS  \textit{Growth Comm.} 371 The zeal with which the Church decried the taking of interest or usury.

\noindent Hence \textbf{decrying} vbl. n.

\P 1633 [SEE  1 above].    
\P 1637 \textit{State  Trials, John Hampden} (R.), There hath been a decrying by the people and they have petitioned in parliament against it.    
\P 1863 KINGLAKE  \textit{Crimea} (1876) I. vi. 84 A general decrying of arms.
\end{myenumerate}


%%%%%%%%%%%%%%%%%%%%%%%%%%%%%%%%%
\myitem{deduce} v.

\noindent \phonetic{(dɪˈdjuːs)}

\noindent [ad. L. dēdūc-ĕre to lead down, derive, in med.L. to infer logically, f. de- I. 1, 2 + dūcĕre to lead. Cf. deduct. In 16–17th c. there was frequent confusion of the forms of deduce and diduce, q.v.

   (The sense-development had already taken place in Latin, and does not agree with the chronological data in English.)]

\vspace{-0.3cm}

\begin{myenumerate}
\itembf{1.} lit. trans. \itembf{a.} To bring, convey; spec. (after Lat.), to lead forth or conduct (a colony). arch.

\P 1578 BANISTER  \textit{Hist. Man} v. 71 If any of the wayes deducyng choler, come vnto the bottome of the ventricle.    
\P 1612 SELDEN  \textit{Illustr. of Drayton} §17 (R.) Advising him he should hither deduce a colony.    
\P 1685 STILLINGFL.  \textit{Orig. Brit.} i. 5 The Romans began to deduce Colonies, to settle Magistrates and Jurisdictions here.    
\P 1822 T. TAYLOR  \textit{Apuleius} 340 Sagacious nature may from thence deduce it [the blood] through all the members.    
\P 1866 J. B. ROSE  \textit{Virgil's Georg.} 88 Still Ausonian colonists rehearse, Deduced from Troy, the incoherent verse.

\itembf{b.} To bring or draw (water, etc.) from. Obs.

\P 1602 W. FULBECKE  \textit{2nd Pt. Parall.} 54 By that meane he deduced water out of the earth.
\P c1630 RISDON  \textit{Surv. Devon} §107 (1810) 104 Conduits‥nourished with waters deduced from out of the fields.

\itembf{c.} To bring or draw down. Obs.

\P 1621 G. SANDYS  \textit{Ovid's Met.} xii. (1626) 244 Orions mother Mycale, eft-soone Could with her charmes deduce the strugling Moone.

\itembf{2.} fig. \itembf{a.} To lead, bring. Obs.

\P 1545 JOYE  \textit{Exp. Dan.} Ded. A. iv, Christ himself doth‥deduce us unto the readinge of thys boke.    
\P 1585 J. HILTON in  \textit{Fuller Ch. Hist.} ix. vi. §27 That‥we be‥made partakers of his Testament, and so deduced to the knowledge of his godly will.    
\P 1706 COLLIER  \textit{Refl. Ridic.} 25 He continually deduces the conversation to this topick.

\itembf{b.} Law. To bring before a tribunal.

\P 1612 BACON  \textit{Ess. Judicature} (Arb.) 458 Many times, the thing deduced to Iudgement, may bee meum et tuum [etc.].

\itembf{c.} To lead away, turn aside, divert.

\P 1541  \textit{Act 33 Hen. VIII}, c. 32 The vicar‥wolde deduce them from their said most accustomable parishe church of Whitegate, vnto his said church of Ouer.    
\P 1647 LILLY  \textit{Chr. Astrol.} clxvii. 720 The force of a Direction may continue many yeers, untill the Significator is deduced to another Promittor.

\itembf{d.} To bring down, convey by inheritance.

\P 1633 BP. HALL  \textit{Hard Texts} 483 If Abraham‥had this land given to him for his inheritance, how much more may wee, his seed, (to whom it is deduced)‥challenge a due interest in it.    
\P 1641 ‘SMECTYMNUUS’  \textit{Answ.} §6 (1653) 32 How this should have beene deduced to us in an uninterrupted Line, wee know not.

\itembf{3.} To draw or obtain from some source; to derive. Now somewhat rare.

\P 1596 H. CLAPHAM  \textit{Briefe Bible} i. 15 He, of Nothing, created Something‥whereout, Al other Creatures were to be diduced.    
\P 1634 SIR T. HERBERT  \textit{Trav.} (1638) 232 A ceremony diduced from the Romans.    
\P 1665  \textit{Ibid.} (1677) 181 Rivers that deduce their Springs near each other.    
\P 1790 COWPER  \textit{My Mother's Picture} 108 My boast is not, that I deduce my birth From loins enthron'd, and rulers of the earth.    
\P 1869 FARRAR  \textit{Fam. Speech} i. (1873) 20 The attempt to prove that all languages were deduced from the Hebrew.

\itembf{b.} intr. To be derived. rare. (Cf. to derive.)

\P 1866 J. B. ROSE tr. \textit{Ovid's Fasti} Notes 240 The former notion of a bird‥may deduce from the eastern word Gaph.    
\P 1889 COURTNEY  \textit{Mill} 20 The very first principles from which it deduces, are so little axiomatic that, etc.

\itembf{4.} trans. To trace the course of, trace out, go through in order (as in narrative or description); to bring down (a record) from or to a particular period. †Formerly, also, To conduct (a process), handle, treat, deal with (a matter).

\P 1528 GARDINER in  Pocock \textit{Rec. Ref.} I. l. 115 Considering how the process might be after the best sort deduced and handled.    c 
\P 1645 HOWELL  \textit{Lett.} vi. 61, I will deduce the business from the beginning.    
\P 1659 BP. WALTON  \textit{Consid. Considered} 259 These things are largely deduced and handled in the same Prolegomena.    
\P 1685 STILLINGFL.  \textit{Orig. Brit.} iii. 88 Having deduced the Succession of the British Churches down to‥the first Councel of Arles.    1728–46 Thomson Spring 577 Lend me your song, ye nightingales‥while I deduce, From the first note the hollow cuckoo sings, The symphony of Spring.    
\P 1776 GIBBON  \textit{Decl. \& Fall} I. 296 The general design of this work will not permit us‥to deduce the various fortunes of his private life.    
\P 1818 JAS. MILL  \textit{Brit. India} i. (1840) I. 2 To deduce to the present times a history of‥the British transactions, which have had an immediate relation to India.    
\P 1866 J. MARTINEAU  \textit{Ess.} I. 149 All the optical history‥is elaborately deduced.

\itembf{5.} To trace the derivation or descent of, to show or hold (a thing) to be derived from.

\P 1536 TINDALE  \textit{Wks.} 21 (R.) Deducyng the loue to God out of fayth, and the loue of a man's neighbour out of the loue of God.    
\P 1579 W. FULKE  \textit{Ref. Rastel} 715 They could not deduce the beginning from ye Apostles.    
\P 1658 USSHER  \textit{Annals} 593 They deduced themselves from the Athenians.    
\P 1676 HODGSON in  \textit{Phil. Trans.} XI. 766 Those‥who deduce the Scurvy from the use of Sugar.    
\P 1767 BLACKSTONE  \textit{Comm.} II. 114 He cannot deduce his descent wholly by heirs male.

\itembf{6.} To derive or draw as a conclusion from something already known or assumed; to derive by a process of reasoning or inference; to infer. (The chief current sense.)

\P 1529 MORE  \textit{Dyaloge} iii. Wks. 215/2 Ye case once graunted, ye deduce your conclusion very surelye.    
\P 1651 BAXTER  \textit{Inf. Bapt.} 87 It must be [known] rationally by deducing it from some premises.    
\P 1696 WHISTON  \textit{Th. Earth} ii. (1722) 184 The knowledge of Causes is deduc'd from their Effects.    
\P 1788 REID  \textit{Aristotle's Log.} iv. §4. 83 Rules‥deduced from the particular cases before determined.    
\P 1812 SIR H. DAVY  \textit{Chem. Philos.} p. viii, It was deduced from an indirect experiment.    
\P 1849 MURCHISON  \textit{Siluria} i. (1867) 2 This inference has been deduced from positive observation.    
\P 1885 C. LEUDESDORF  \textit{Cremona's Proj. Geom.} 277 From this we deduce a method for the construction.

\itembf{b.} Less commonly with obj. clause.

\P 1532 MORE  \textit{Confut. Tindale} Wks. 461/2 We deduce ther⁓upon that he wil not suffer his church fal into ye erronious belief of anie damnable vntrouthe.    
\P 1646 SIR T. BROWNE  \textit{Pseud. Ep.} v. vi. 243 That the custome of feasting upon beds was in use among the Hebrewes, many diduce from the 23. of Ezekiel.

\itembf{7.} To deduct, subtract. Obs.

\P 1614 BP. HALL  \textit{Recoll. Treat.} 514 The more we deduce, the fewer we leave.    
\P 1632 B. JONSON  \textit{Magn. Lady} ii. i, A matter of four hundred To be deduced upon the payment.    
\P 1662 STILLINGFL.  \textit{Orig. Sacr.} i. v. §3, 1117. which being deduced from 3940. the remainder is 2823.

\itembf{8.} To reduce (to a different form). Obs.

\P 1586 J. HOOKER  \textit{Girald. Irel.} in \textit{Holinshed} II. 10/1 By these meanes the whole land, which is now diuided into fiue prouinces or portions, maie be deduced and brought into one.    
\P 1654 GATAKER  \textit{Disc. Apol.} 36 After that my Morning Lecture was reduced, or deduced rather, to the ordinarie hour in most places.    
\P 1749 J. MILLAN  \textit{(title)}, Coins, Weights, and Measures, Ancient and Modern, of all Nations, deduced into English on above 100 Tables.

\noindent Hence \textbf{deducing} vbl. n., deduction.

\P 1530 PALSGR. 212/2 Deducyng, discours.    
\P 1532 MORE  \textit{Confut. Tindale} Wks. 461/2 Termes‥of drawyng oute \& deducinges and depending vpon scrypture.    
\P 1651 HOBBES  \textit{Leviath.} ii. xxv. 133 Consisting in a deducing of the benefit, or hurt that may arise, etc.    
\P 1827 WHATELY  \textit{Logic} (1837) 258 The deducing of an inference from those facts.
\end{myenumerate}

%%%%%%%%%%%%%%%%%%%%%%%%%%%%%%%%
\myitem{deem} v.

\noindent \phonetic{(diːm)}

\noindent [A Common Teut. derivative vb.; OE. d‹oeacu›man, déman + OFris. déma, OS. a-dômian (Du. doemen), OHG. tuomian, tuomen (MHG. tüemen), ON. d‹oeacu›ma (dæma), (Sw. döma, Da. dömme), Goth. dômjan:—OTeut. *dômjan. f. dômo-z, Goth. dôm-s, judgement, doom. Cf. deme n., doom v.]
\vspace{-0.3cm}

\begin{myenumerate}

\itembf{1.} intr. To give or pronounce judgement; to act as judge, sit in judgement; to give one's decision, sentence, or opinion; to arbitrate. Obs.

   In OE. construed with a dative of the person, ‘to pronounce judgement to, act as judge to’, equivalent to the trans. sense in 2.

\P c825 \textit{Vesp. Psalter} ii. 10 Alle ða ðe doemað eorðan.    
\P 971 \textit{Blickl. Hom.} 11 He cymeþ to demenne cwicum \& deadum.
\P c1000 \textit{Ags.  Gosp.} Matt. vii. 2 Witodlice ðam ylcan dome þe ᴁe demað, eow byð ᴁedemed.    Ibid. John viii. 15 Ge demað æfter flæsce, ic ne deme nanum men [
\P [c1160 \textit{Hatton  G.}, Ich ne deme nane men].    
\P a1300  \textit{Cursor M.} 17415 (Cott.), If yee þan rightwisli wil deme, Yeild vs ioseph þat yee suld yeme.    
\P 1393 GOWER  \textit{Conf.} I. 304 They..toke a juge therupon..And bede him demen in this cas.    
\P c1440 J. CAPGRAVE  \textit{St. Kath.} iii. 1464 She..spak and commaunded, bothe dempte and wrot.    
\P 1556 in  W. H. Turner \textit{Select. Rec. Oxford} 262 To arbytrate, deme, and judge betwixt the said Citie and..John Wayte.    
\P 1579 SPENSER  \textit{Sheph. Cal.} Aug. 137 Neuer dempt more right of beautye I weene The shepheard of Ida that iudged beauties Queene.

\itembf{2.} trans. To judge, sit in judgement on (a person or cause). Obs.

   The construction with a personal object takes, in Northumbrian and ME., the place of the OE. const. with dative in 1.

\P c950 \textit{Lindisf. Gosp.} Matt. vii. 2 In ðæm dome ᴁie doemes ᴁe biðon ᴁedoemed [Rushw. Gl. ᴁe beoþ doemde].    
\P Ibid. John viii. 15 Ic ne doemo æniᴁne monno. 
\P c1200  \textit{Trin. Coll. Hom.} 171 Ure drihten cumeð al middeneard to demen.    Ibid. 225 Þat sal deme þe quica and þe deade.    
\P a1300  \textit{Cursor M.} 2 1965 (COTT.),  In þe first he com dempt to be.    
\P 1382 WYCLIF  \textit{John} xvi. 11 The prince of this world is now demyd.    
\P 1483 CAXTON  \textit{Gold. Leg.} 59/2 Moyses satte \& juged \& demed the peple fro moryng vnto euenyng.    
\P 1596 SPENSER  \textit{F.Q.} iv. iii. 4 At th' one side sixe iudges were dispos'd, To view and deeme the deedes of armes that day.    
\P 1605 HEYWOOD  \textit{1st Pt. If you know not me} Wks. 1874 I. 203  Deeme her offences, if she haue offended, With all the lenity a sister can.    
\P 1609 SKENE  \textit{Reg. Maj.} 111 Thou Judge be ware, for as ye deme, ze sall be demed.

\itembf{b.} To rule (a people) as a judge. Obs.

\P a1300  \textit{Cursor M.} 7283 (Cott.), Fourti yeir dempt he israel.    
\P c1330 R. BRUNNE  \textit{Chron.} (1810) 280 Edward now he wille, þat Scotlond be wele ȝemed, And streitly in skille þorgh wise men demed.

\itembf{c.} To administer (law). arch.

\P 1393 LANGL.  \textit{P. Pl.} C. v. 175 By leel men and lyf-holy my lawe shal be demyd.    
\P 1718 BP. WILSON in  \textit{Keble Life} xii. (1863) 397 That..the 24 Keys may be called, according to the statute and constant practice to deem the law truly.    
\P 1887 HALL  \textit{Caine Deemster} viii. 54 The Deemster was a hard judge, and deemed the laws in rigour.

\itembf{d.} To decide (a quarrel). Obs.

\P 1494 FABYAN \textit{Chron.} v. cxxv. 105 To suffre his quarell to be demyd by dynt of swerde atwene them two.

\itembf{3.} To sentence, doom, condemn (to some penalty, to do or suffer something). Obs.

\P a1000 \textit{Elene}  500 (Gr.) Swa he..to cwale moniᴁe Cristes folces demde, to deaþe.    
\P c1175  \textit{Lamb. Hom.} 73 He wurð idemed to þolien wawe mid dovelen in helle.    
\P c1200  \textit{Trin. Coll. Hom.} 223 Þe sulle ben to deaðe idemd.    
\P a1300  \textit{Cursor M.} 15343 To-morn dai sal i be dempt On rode tre to hang.    
\P c1386 CHAUCER  \textit{Sompn. T.} 316 For which I deme the to deth certayn.    
\P 1426 AUDELAY  \textit{Poems} 12 Leve he is a lyere, his dedis thai done hym deme.    
\P 1529 RASTELL  \textit{Pastyme} (1811) 243 For whiche rebellyon they were there demyd to dethe.    
\P 1602 in  \textit{J. Mill Diary} (1889) 180 John Sinclair..is dempt to quyt his guddis.

\itembf{b.} fig. To pass (adverse) judgement upon; to condemn, censure. Obs.

\P a1300  \textit{Cursor M.} 28148 (Cott.) Oþer men dedis oft i demyd.    
\P 1488 CAXTON  \textit{Chast. Goddes Chyld.} 21 Many thynges they deme and blame.    
\P 1500-20 DUNBAR  \textit{Poems} xviii. 36 Wist thir folkis that vthir demis, How that thair sawis to vthir semis.    
\P 1555-86  \textit{Satir. Poems Reform.} xxxvii. 33 Do quhat ȝe dow, detractouris ay will deme ȝou.    
\P 1598 D. FERGUSON  \textit{Scot. Prov.}, Dame, deem warily; ye watna wha wytes yersell.

\itembf{4.} To decree, ordain, appoint; to decide, determine; to adjudicate or award (a thing to a person).

\P c900 tr. \textit{Bæda's Hist.} iv. xxix [xxviii.] (1891) 368 Ne wæs ða hweðre sona his halᴁunge ᴁedemed.   
\P a1000 \textit{Exeter  Bk.} vii. 16 Næfre God demeð þæt æniᴁ eft þæs earm ᴁeweorðe.    
\P c1175  \textit{Lamb. Hom.} 95 He demað stiðne dom þam forsune$\sim$ȝede.    
\P c1205 LAY. 460 He habbeð idemed Þat ich am duc ofer heom.    Ibid. 22116 He hæhte alle cnihtes demen rihte domes.    
\P a1300  \textit{Cursor M.} 2 1445 (COTT.)  Þe quen has biden us to deme To þe al þat to right es queme.    
\P c1386 CHAUCER  \textit{Doctor's T.} 199, I deme anoon this clerk his seruaunt haue.    
\P 1399  \textit{Rolls of Parlt.} III. 452/1 The Lordes..deme and ajuggen and decreen, that [etc.].    
\P c1400  \textit{Destr. Troy} 606 Whateuer ye deme me to do.    
\P 1464 \textit{Paston  Lett.} No. 493 II. 166 Fynes therefore dempt or to be dempt.    
\P 1483 CAXTON  \textit{Gold. Leg.} 72/2 In demyng of rightful domes.    
\P 1503-4  \textit{Act 19 Hen. VII}, c. 38 Preamb., It was enacted stablisshed ordeyned demed \& declared..that [etc.].    
\P 1568 GRAFTON  \textit{Chron.} II. 13 The Epistle, in the which Gregory..demed that the Church of Yorke and of London should be even Peres.    
\P a1605 MONTGOMERIE  \textit{Flyting} 373 Syne duelie they deemde, what death it sould die.

\itembf{b.} To decide (to do something). Obs.

\P c1340  \textit{Gaw. \& Gr. Knt.} 1089 Ȝe  han demed to do þe dede þat I bidde.

\itembf{5.} To form or express a judgement or estimate on; to judge, judge of, estimate. Obs.

\P a1225  \textit{Ancr. R.} 290 Euer bihold hire wurð þet he paide uor hire, and dem þerefter pris.    
\P c1325  \textit{E.E. Allit. P.} (A.) 312 To leue no tale be true to tryȝe, Bot þat hys one skyl may dem.    
\P 1388 WYCLIF  \textit{Matt.} xvi. 4 Thanne ȝe kunne deme the face of heuene, but ȝe moun not wite the tokenes of tymes.    
\P c1400  \textit{Rom. Rose} 2200 A cherle is demed by his dede.    
\P 1533 ELYOT  \textit{Cast. Helthe Proem} (1541) A iv b, I desyre men to deme well myne intente.    
\P 1596 SPENSER  \textit{Hymne Love} 168 Things hard gotten men more dearely deeme.

\itembf{b.} To judge between (things), to distinguish, discern. Obs.

\P 1530 PALSGR. 511/1 A blynde man can nat deme no coulours.    
\P 1581 RICH  \textit{Farewell} (1846) 67 He is not able to deeme white from blacke, good from badde, vertue from vice.    
\P 1596 SPENSER  \textit{F.Q.} v. i. 8 Thus she him taught In all the skill of deeming wrong and right.

\itembf{c.} intr. To judge of, to distinguish between.

\P 1340  \textit{Ayenb.} 82 Þet hi ne conne yknawe þane day uram þe nyȝt, ne deme betuene grat and smal.    
\P a1542 WYAT  \textit{Of Courtiers Life} 94 Nor Flaunders chere lettes not my syght to deme Of blacke and white.    
\P 1586 A. DAY  \textit{Eng. Secretary} i. (1625) 27 Here, by judging of our estate, thou maist accordingly deeme of our pleasures.    Ibid. ii. 111 Conversing among such as have discretion to deeme of a Gentleman.

\itembf{6.} To form the opinion, to be of opinion; to judge, conclude, think, consider, hold. (The ordinary current sense.) \itembf{a.} intr. or absol. (Now chiefly parenthetical.)

\P a800 \textit{Corpus Gloss.} 440 Censeo, doema.    
\P c900 tr. \textit{Bæda's Hist.} i. xvi. [xxvii.] (1890) 86 Þæs þe ic demo [ut arbitror]. 
\P c1000 ÆLFRIC  \textit{Gram.} xxvi. (Z.) 155 Censeo ic deme oððe ic asmeaᴁe.    
\P c1385 CHAUCER \textit{L.G.W.} 1244 (\textit{Dido}) And demede as hem liste.    
\P c1386  \textit{ Clerk's T.} 932 For sche is fairer, as thay demen alle, Than is Grisild.    
\P a1400 \textit{Relig.  Pieces fr. Thornton MS.} (1867) 20 To fele and with resone to deme.    
\P 1586 A. DAY  \textit{Eng. Secretary} ii. (1625) 15 He is not..here in the countrey, but as I deeme and you have enformed, about London.    
\P 1725 POPE  \textit{Odyss.} iii. 61 He too, I deem, implores the power divine.

\itembf{b.} with obj. and compl. (n., adj. or pple., or inf. phr.; formerly often with for, as).

\P a800 \textit{Corpus Gloss.} 440 Censeo, doema.    
\P c900 tr. \textit{Bæda's Hist.} i. xvi. [xxvii.] (1890) 86 Þæs þe ic demo [ut arbitror].    c 1000 Ælfric Gram. xxvi. (Z.) 155 Censeo ic deme oððe ic asmeaᴁe. 
\P c1205 LAY. 22140 Þene þe king demde for-lore.    
\P a1225  \textit{Ancr. R.} 120 Þet tu schalt demen þi suluen wod.    
\P a1300  \textit{Cursor M.} 26814 (Cott.) It mai nan him for buxum deme.    
\P 1340-70  \textit{Alex. \& Dind.} 218 Oure doctourus dere, demed for wise.    
\P c1400 \textit{Lanfranc's  Cirurg.} 102, I demede him for deed.    
\P c1450  \textit{St. Cuthbert} (Surtees) 5163 Þai demed it better all' to dye.    
\P 1548 HALL  \textit{Chron.} 191 b, What so ever jeoperdy or perill might bee construed or demed, to have insued.    
\P 1581 G. PETTIE  \textit{Guazzo's Civ. Conv.} i. (1586) 35 A vertue which you deeme yourselfe to have.    
\P 1628 DIGBY  \textit{Voy. Medit.} 51, I deemed it much my best and shortest way.    
\P 1681 P. RYCAUT  \textit{Critick} 201 He went to the House of the World, which was always deemed for a Deceiver.    
\P 1697 DRYDEN  \textit{Virg. Past.} i. 9 For never can I deem him less than God.    
\P 1754 J. SHEBBEARE  \textit{Matrimony} (1766) I. 45 Deemed as very unjust in Gaming.    
\P 1827 JARMAN  \textit{Powell's Devises} II. 293 A general permission..appears to have been deemed sufficient.    
\P 1852 C. M. YONGE  \textit{Cameos} I. xxxii. 277 Harold..deemed it time to repress these inroads.    
\P 1875 JOWETT  \textit{Plato} (ed. 2) V. 398 Works..which have been deemed to fulfil their design fairly.

\itembf{c.} with that and clause.

\P c1205 LAY. 24250 Men gunnen demen þat nes i nane londe burh nan swa hende.    
\P c1386 CHAUCER  \textit{Man of Law's T.} 940, I ought to deme..That in the salte see my wyf is deed.    
\P c1430 LYDG.  \textit{Bochas} i. ii. (1544) 5 a, Nembroth..Dempt..He transcended al other of noblesse.    
\P c1450  \textit{Merlin} 10 She demed that it was the enmy that so hadde hir begiled.    
\P 1597 HOOKER  \textit{Eccl. Pol.} v. i. (1611) 184 Wee may boldly deeme there is neither, where both are not.    
\P 1739 W. MELMOTH  \textit{Fitzosb. Lett.} (1763) 291 Nor dempt he, simple wight, no mortal may The blinded god..when he list, foresay.    
\P 1887 BOWEN  \textit{Virgil Æneid} ii. 371 (1889) 126 Deeming we come with forces allied.

\itembf{7.} intr. To judge or think (in a specified way) of a person or thing.

\P c1384 CHAUCER  \textit{H. Fame} ii. 88 Thow demest of thy selfe amys.    
\P c1400  \textit{Rom. Rose} 2198 Of hem noon other deme I can.    
\P c1440 \textit{Generydes}  4710 Wele I wote in hym ye demyd amys.    
\P 1581 SIDNEY  \textit{Apol. Poetrie} (Arb.) 24 Let vs see how the Greekes named it [Poetry], and howe they deemed of it.    
\P 1586 A. DAY  \textit{Eng. Secretary} i. (1625) 146, I shall..give you so good occasion to deeme well of me.    
\P 1667 MILTON  \textit{P.L.} viii. 599 Though higher of the genial Bed by far, And with mysterious reverence I deem.    
\P 1762 BLACKSTONE in  \textit{Gutch Coll. Cur.} II. 362 These capital mistakes..occasion'd the Editor..to deem with less reverence of this Roll.    
\P 1814 SCOTT  \textit{Wav.} lxi, Where the ties of affection were highly deemed of.    
\P 1860 J. P. KENNEDY  \textit{Horse Shoe} R. ix. 105, I cannot deem otherwise of them.

\itembf{8.} To think to do something, to expect, hope.

\P c1400  \textit{Apol. Loll.} 51 Symon Magus..was reprouid of Petre, for he demid to possede þe ȝeft of God bi money.    
\P 1819 BYRON  \textit{Juan} ii. clxxii, A creature meant To be her happiness, and whom she deem'd To render happy.

\itembf{9.} trans. To think of (something) as existent; to guess, suspect, surmise, imagine. Obs.

\P c1400  \textit{Destr. Troy} 528 Ne deme no dishonesty in your derfe hert, Þof I put me þus pertly my purpos to shewe.    
\P 1470-85 MALORY  \textit{Arthur} x. xxvi, As Kynge mark redde these letters, he demed treson by syr Tristram.    
\P 1586 A. DAY  \textit{Eng. Secretary} i. (1625) 114 Your imaginations doe already deeme the matter I must utter.    
\P 1598-9 \textit{Parismus}  i. (1661) 15 All the companie began to deeme that which afterward proued true.

\itembf{b.} intr. To think of, have a thought or idea of.

\P 1814 CARY  \textit{Dante} (Chandos) 302 The shining of a flambeau at his back Lit sudden ere he deem of its approach.    
\P 1818 BYRON  \textit{Ch. Har.} iv. cxxxvii, Something unearthly which they deem not of.

\itembf{10.} trans. To pronounce, proclaim, celebrate, announce, declare; to tell, say, utter. Also intr. with of. [An exclusively poetic sense, found already in OE., probably derived from sense 4. Cf. also ON. d‹oeacu›ma in poetry, to talk.]

\P a1000 \textit{Fat.  Apost.} (Gr.) 10 Þær hie dryhtnes æ deman sceoldon, reccan fore rincum.    
\P a1000 \textit{Guthlac}  (Gr.) 498 Þæt we æfæstra dæde demen, secᴁen dryhtne lof ealra þara bisena.    
\P c1205 LAY. 23059 Ælles ne cunne we demen.
\P [c1275 telle]  of Arðures deden.    
\P c1325  \textit{E.E. Allit. P.} C. 119 Dyngne Dauid..þat demed þis speche, In a psalme.    
\P c1330 R. BRUNNE  \textit{Chron. Wace} (Rolls) 154 Alle þer lymmes, how þai besemed, In his buke has Dares demed, Both of Troie \& of Grece.    
\P c1350  \textit{Will. Palerne} 151 Hire deth was neiȝ diȝt, to deme þe soþe.    
\P a1400-50  \textit{Alexander} 1231 Þan  he dryfes to þe duke, as demys [Dubl. MS. tellys] þe textis.    
\P a1547 SURREY  \textit{Aeneid} ii. 156 Then some gan deme to me The cruell wrek of him that framde the craft [crudele canebant artificis scelus].

\itembf{b.} with double obj. To celebrate as, style, call, name. poetic. Obs.

\P c1325  \textit{E.E. Allit. P.} B. 1020 Forþy  þe derk dede see hit is demed euer more.    
\P Ibid.1611 Baltazar..Þat now is demed Danyel of derne coninges.
\end{myenumerate}


%%%%%%%%%%%%%%%%%%%%%%%%%%%%%%%%%
%\myitem{deferential} a.1

%\noindent \phonetic{(dɛfəˈrɛnʃəl)}

%\noindent [f. deference (or its L. type *dēferentia) + -al1: cf. essence, essential, prudence, prudential, etc.]
%\vspace{-0.3cm}

%\begin{myenumerate}

%Characterized by deference; showing deference; respectful.

%\P 1822 SCOTT  \textit{Nigel xxii, If you seek deferential observance and attendance, I tell you at once you will not find them here.    
%\P 1838 DICKENS  \textit{Nich. Nick.} xvii, She was marvellously deferential to Madame Mantalini.    
%\P 1870 DISRAELI  \textit{Lothair xxviii, The Duke..could be soft and deferential to women.

%Hence deferentiˈality n., deference; defeˈrentially adv., in a deferential manner.

%\P 1880 CORNH.  \textit{Mag. Feb. 183 His master he recognises as such with respectful deferentiality.    
%\P a1846 GENTLEM. MAG. CITED in  \textit{Worcester for deferentially.    
%\P 1848 C. BRONTë  \textit{J. Eyre vii. (1873) 61 These ladies were deferentially received..and conducted to seats of honour.    
%\P 1865 DICKENS  \textit{Mut. Fr.} iii. i, Deferentially observant of his master's face.




%\end{myenumerate}


%%%%%%%%%%%%%%%%%%%%%%%%%%%%%%%%%
%\myitem{definitive} a. and n.

%\noindent \phonetic{(dɪˈfɪnɪtɪv)}

%\noindent [a. OF. definitif, diffinitif, -ive (12th c.), ad. L. dē-, diffīnītīv-us, f. ppl. stem of dēfīnīre: see definite.]
%\vspace{-0.3cm}

%\begin{myenumerate}

%\itembf{A.} adj. Having the function of defining, or of being definite.

%1. a.A.1.a Having the function of finally deciding or settling; decisive, determinative, conclusive, final: esp. in definitive sentence, and the like.

%\P c1386 CHAUCER  \textit{Doctor's T. 172 The Iuge answerd of þis in his absence I may not ȝiue diffinityf sentence.    
%\P 1474 CAXTON  \textit{Chesse iii. vi. H v b, The theef was..taken..and by sentence diffynytif was hanged.    
%\P 1523 LD. BERNERS  \textit{Froiss. I. xxiv. 35 It was the moneth of May folowyng, or [= ere] they had aunswere dyffinatyue.    
%\P 1583 STUBBES  \textit{Anat. Abus. ii. (1882) 106 Maye they as Capytall Iudges, geue definytiue sentence of lyfe and death vpon malefactors.    
%\P 1601 R. JOHNSON  \textit{Kingd. \& Commw. (1603) 57 Upon hearing of both parties, judgment definative is given, and may not be repealed.    
%\P 1688  \textit{Answ. Talon's Plea 3 Barely to say with a definitive Gravity, Here's a great abuse.    
%\P 1748 RICHARDSON  \textit{Clarissa} (1811) I. 11 Expecting a definitive answer.    
%\P 1763 WILKES  \textit{Corr. (1805) I. 84 The definitive treaty is now signed.    
%\P 1855 MACAULAY  \textit{Hist. Eng.} IV. 527 A jury had pronounced: the verdict was definitive.

%b.A.1.b transf. of persons. Obs.

%\P 1603 SHAKES.  \textit{Meas. for M.} v. i. 432 Neuer craue him, we are definitiue..Away with him to death.    
%\P 1639 FULLER  \textit{Holy War} iv. v. (1647) 176 Desiring rather to be scepticall then definitive in the causes of Gods judgements.    
%\P 1741 RICHARDSON  \textit{Pamela} (1824) I. 104, I will make you..my adviser in this matter, though not, perhaps, my definitive judge.

%c.A.1.c That settles or determines bounds or limits.

%\P 1860 J. P. KENNEDY  \textit{W. Wirt I. xiii. 164 [This] point of view should lead to a just and definitive limitation of the boundaries.

%\itembf{2.} Having the character of finality as a product; determinate, definite, fixed and final. Of an edition of a literary work, a textbook, etc.: authoritative; the most complete and authoritative to date. In Biol. opposed to formative or primitive, as definitive organs, definitive aorta.

%\P a1639 WOTTON  \textit{(J.), [It] being the very definitive sum of this art, to distribute usefully and gracefully a well chosen plot.    
%\P 1646 SIR T. BROWNE  \textit{Pseud. Ep.} i. vi, Other Authors write often dubiously, even in matters wherein is expected a strict and definitive truth.    
%\P 1821 J. Q. ADAMS in  \textit{C. Davies Metr. Syst. iii. (1871) 174 The temporary system established by the law of 1st August, 1793. The definitive system established by the law of 10th December, 1799.    
%\P 1865  \textit{Daily Tel.} 30 Oct. 4/4 Some days will probably elapse before we shall be able to announce a definitive result.    
%\P 1878 NEWCOMB  \textit{Pop. Astron. iii. v. 399 A definitive orbit of the comet.    
%\P 1882 SWINBURNE  \textit{Let. 27 Sept. in N. \& Q. (1965) CCX. 304/2 Dr. Grosart..is about to publish what the French would call a ‘definitive edition’ of Daniel.    
%\P 1887  \textit{Amer. Jrnl. Philol. VIII. 484 With the four volumes first mentioned the Goethe Society in Weimar begins the publication of the definitive edition of Goethe's works.    
%\P 1888 ROLLESTON \& JACKSON  \textit{Forms of Animal Life 803 The primitive ovum divides; one of the cells thus produced grows into the definitive ovum.    
%\P 1928 T. S. ELIOT in  \textit{E. Pound Sel. Poems p. vii, This book is, in my eyes, rather a convenient Introduction to Pound's work than a definitive edition.    
%\P 1949 ‘G.  \textit{Orwell’ Nineteen Eighty-Four i. iv. 44 Ampleforth..was engaged in producing garbled versions—definitive texts, they were called—of poems which had become ideologically offensive.    
%\P 1959  \textit{Spectator} 21 Aug. 235/1 That vague uneasiness one has come to feel in the presence of American ‘definitive’ biographies.

%\itembf{3.} Metaph. Having a definite position, but not occupying space: opposed to circumscriptive. Obs.

%\P 1624 SEE  \textit{definitively 2.]    
%\P 1657 HOBBES  \textit{Absurd Geom. Wks. VII. 385 Definitive or circumscriptive, and some other of your distinctions..are but snares.    
%\P 1665 GLANVILL  \textit{Sceps. Sci. xiii. 73 Who is it that retains not a great part of the imposture, by allowing them a definitive Ubi, which is still but Imagination?

%\itembf{4.} That makes or deals with definite statements.

%\P a1619 M. FOTHERBY  \textit{Atheom. ii. ix. §2 (1622) 296 Plutarch is more definitiue, and punctuall, in this point.    
%\P 1862 LIT.  \textit{Churchman VIII. 6/1 We should be glad to see more definitive teaching on the nature of Church Communion.

%\itembf{5.} That serves to define or state exactly what a thing is; that specifies the individual referred to; esp. in Gram. (Formerly used of the definite article, and of the finite verb.)

%\P 1731 BAILEY  \textit{vol. II, s.v. Article, Definitive Article, the article (the) so called, as fixing the sense of the word it is put before to one individual thing.    
%\P 1765 W. WARD  \textit{Gram. iv. iv. 164 Of the verb definitive.    
%\P 1800 W. TAYLOR in  \textit{Monthly Mag. VIII. 797 To preserve a name of sect, which ought to be simply definitive, from sliding into a term of reproach.    
%\P 1824 L. MURRAY  \textit{Eng. Gram. (ed. 5) I. 231 When a noun of multitude is preceded by a definitive word, which clearly limits the sense to an aggregate with an idea of unity, it requires a verb..in the singular number: as, ‘A company of troops was detached’.    
%\P 1854 ELLICOTT  \textit{Galat. 87 The..definitive force of the article.

%\itembf{6.} Concerned with the definition of form or outline. rare.

%\P 1815 W. TAYLOR in  \textit{Monthly Rev. LXXVI. 115 The lineless delicate contours of youth and bloom embarrass the definitive skill even of a Correggio.

%\itembf{B.} n. (the adj. used ellipt.)

%\itembf{1.} A definitive sentence, judgement, or pronouncement. Obs.

%\P 1595 W. HUBBOCKE  \textit{Apol. Infants Unbapt. 11 Is there no pardon from this general damnatorie sentence and cruell definitiue?    
%\P 1660 R. COKE  \textit{Power \& Subj. 134 Judgment is the definitive of him who by right commands, permits, or forbids a thing.    
%\P 1804 EUROP. MAG. in  \textit{Spirit Pub. Jrnls. (1805) VIII. 135 In spite of the Definitive, we shall have another battle of the books.

%\itembf{2.} Gram. A definitive word.

%\P 1751 HARRIS  \textit{Hermes (1841) 179 Definitives..are commonly called by grammarians, ‘articles,’ articuli, ἄρθρα. They are of two kinds, either those properly..so called, or else the pronominal articles, such as this, that, any, \&c.    
%\P 1786-98 H. TOOKE  \textit{Purley I. 20 About the time of Aristotle, when a fourth part of the speech was added,—the definitive, or article.    
%\P 1824 L. MURRAY  \textit{Eng. Gram. (ed. 5) I. 71 As articles are by their nature definitives..they cannot be united with such words as are..as definite as they may be; (the personal pronouns for instance).


%______________________________


%Additions 1993

%Add: [A.] [2.]A.2 b.A.2.b Philately. Of a postage stamp: belonging to or forming part of the standard issue of a country. Cf. *provisional a. 1 c.

%\P 1929 K. B. STILES  \textit{Stamps i. 8 Once more, provisionals appeared. These in turn were replaced by definitive stamps inscribed with the newly required values.    
%\P 1961 K. F. CHAPMAN  \textit{Commonwealth Stamp Collecting ii. 34 Both territories..have issued definitive stamps recognized by the Universal Postal Union for international use.    
%\P 1977 GLOBE \& MAIL  \textit{(Toronto) 23 Apr. 1/4 It is the first time the queen has not been on the definitive stamp.    
%\P 1986 SUNDAY  \textit{Express 21 Dec. 6/8 While 12p Christmas stamps were OK, 12p definitive stamps were not.

%[B.] \itembf{3.} Philately. A definitive postage stamp.

%\P 1929 K. B. STILES  \textit{Stamps i. 7 Stamps which are called definitives..are of permanent character—for use regularly until such time as the government issuing them shall decide to replace them with stamps of another design.    
%\P 1961 K. F. CHAPMAN  \textit{Commonwealth Stamp Collecting ii. 34 The low value definitives with naye paise surcharges began to appear in 1960.    
%\P 1986 STAMP  \textit{Mag. Feb. 70/4 The initial issue will be of definitives (5, 30, 60 and 150 cents) and ‘independence’ commemoratives.



%\end{myenumerate}


%%%%%%%%%%%%%%%%%%%%%%%%%%%%%%%%%
%\myitem{delectation} n.

%\noindent \phonetic{(diːlɛkˈteɪʃən)}

%\noindent [a. OF. delectation (12th c. in Hatzf.), also delitacion (Godef.), ad. L. dēlectātiōn-em, n. of action from dēlectāre to delight.]
%\vspace{-0.3cm}

%\begin{myenumerate}

%The action of delighting; delight, enjoyment, great pleasure.
%   Formerly in general use, and denoting all kinds of pleasure from sensual to spiritual; now (since c 1700) rarer, more or less affected or humorous, and restricted to the lighter kinds of pleasure.

%\P 13.. S. AUGUSTIN 730 in  \textit{Horstmann Altengl. Leg. 74 Þat luttel delectaciun Þat he feled in his etyng.    
%\P 1382 WYCLIF  \textit{2 Macc. ii. 26 Sothely we curiden..that it were delectacioun, or lykyng, of ynwitt to men willynge for to reede.    
%\P 1435 MISYN  \textit{Fire of Love v. 9 Wyckyd treuly þis warld lufe, settand þere-in þe lust of þere delectacyone.    
%\P 1526 TINDALE  \textit{2 Cor.} xii. 10 Therefore have I delectacion in infirmities.    
%\P 1570 DEE  \textit{Math. Pref. 32 To the glory of God, and to our honest delectation in earth.    
%\P 1620 VENNER  \textit{Via Recta iv. 75 It is pleasant to the pallat, and induceth..a smoothing delectation to the gullet.    
%\P a1711 KEN  \textit{Edmund Poet. Wks.
%\P 1721 II. 96 LIKING  \textit{shoots up unheeded to Delight, And Delectations soon Consent excite.    
%\P 1779-81 JOHNSON  \textit{L.P., Garth, ‘The Dispensary’..appears..to want something of poetical ardour and something of general delectation.    
%\P 1846 DICKENS  \textit{Cricket on Hearth i, Reproducing scraps of conversation for the delectation of the baby.    
%\P 1892  \textit{Times} 27 Dec. 7/1 A great many other entertainments were provided for the public delectation.

%\itembf{b.} transf. Something that delights; a delight.

%\P 1432-50 tr.  \textit{Higden (Rolls) I. 249 That the citesynnes scholde dispute of the commune profette yn tylle none: and not attende to eny other delectacion.    
%\P 1536 PRIMER  \textit{Hen. VIII, 149 Of mind Thou art the delectation, Of pure love the insuation.    
%\P 1576 FLEMING  \textit{Panopl. Epist. 63 If solitarinesse and living alone be your delectation.



%\end{myenumerate}


%%%%%%%%%%%%%%%%%%%%%%%%%%%%%%%%%
%\myitem{deleterious} a.

%\noindent \phonetic{(dɛlɪˈtɪərɪəs)}

%\noindent [f. mod.L. dēlētēri-us, a. Gr. δηλητήρι-ος noxious, hurtful, f. δηλήτηρ destroyer, f. δηλέ-εσθαι to hurt: see -ous.]
%\vspace{-0.3cm}

%\begin{myenumerate}

%Hurtful or injurious to life or health; noxious.

%\P 1643 SIR T. BROWNE  \textit{Relig. Med.} ii. §10 They were not deleterious to others onely, but to themselves also.    
%\P 1646  \textit{ Pseud. Ep. iii. vii. 119 Deleterious it may bee at some distance and destructive without a corporall contaction.    
%\P 1762 GOLDSM.  \textit{Cit. W. xci, In some places, those plants which are entirely poisonous at home lose their deleterious quality by being carried abroad.    
%\P 1821 BYRON  \textit{Juan} iv. lii, 'Tis pity wine should be so deleterious, For tea and coffee leave us much more serious.    
%\P 1869 PHILLIPS  \textit{Vesuv. viii. 213 This gas was well known to be deleterious.

%\itembf{b.} Mentally or morally injurious or harmful.

%\P 1823 BYRON  \textit{Juan} xiii. i, A jest at vice by virtue's called a crime, And critically held as deleterious.    
%\P 1860 EMERSON  \textit{Cond. Life, Power Wks. (Bohn) II. 335 Politics is a deleterious profession, like some poisonous handicrafts.

%Hence deleˈteriously adv., deleˈteriousness.

%\P 1812 SHELLEY  \textit{Let. 29 July (1964) I. 316, I have no doubts on the deleteriousness of classical education.    
%\P 1879  \textit{Cassell's Techn. Educ. IV. 359/1 The solution should not be deleteriously affected.    
%\P 1892 W. B. SCOTT  \textit{Autobiog. I. i. 15 David was..deleteriously influenced by studying these able but imperfect artists.



%\end{myenumerate}


%%%%%%%%%%%%%%%%%%%%%%%%%%%%%%%%%
%\myitem{demonic} a.

%\noindent \phonetic{(dɪˈmɒnɪk)}

%\noindent Also dæm-.
%\vspace{-0.3cm}

%\begin{myenumerate}

%[ad. L. dæmonic-us, a. Gr. δαιµονικ-ός of or pertaining to a demon, possessed by a demon, f. δαίµων, δαιµον-: see demon1 and -ic.]

%\itembf{1.} Of, belonging to, or of the nature of, a demon or evil spirit; demoniacal, devilish.

%\P 1662 EVELYN  \textit{Chalcogr. 68 Convulsive and even Demonic postures.    
%\P 1738 G. SMITH  \textit{Curious Relat. I. iv. 518 So many Demonick Delusions.    
%\P 1840 CARLYLE  \textit{Heroes} (1858) 197 ‘Jötuns,’ Giants, huge shaggy beings of a demonic character.    
%\P 1886  \textit{Q. Rev.} Oct. 53 The traditional demonic proposal, ‘I will be your servant here, and you shall be mine hereafter’.

%\itembf{2.} Of, relating to, or of the nature of, supernatural power or genius = Ger. dämonisch (Göthe): cf. demon1 1. (In this sense usually spelt dæmonic for distinction.)

%\P 1798 W. TAYLOR in  \textit{Monthly Rev. XXVI. 491 In his immature youth he had detected within himself a something dæmonic.    
%\P 1854 LOWELL  \textit{Cambridge 30 Yrs. Ago Pr. Wks.
%\P 1890 I. 87 SHALL  \textit{I take Brahmin Alcott's favorite word, and call him a Dæmonic man?    [
%\P 1874 SEE  \textit{demoniac 4.]    
%\P 1879 FITZGERALD  \textit{Lett. (1889) I. 447 There is enough to show the Dæmonic Dickens: as pure an instance of Genius as ever lived.    
%\P 1887 SAINTSBURY  \textit{Hist. Elizab. Lit. vii. (1890) 258 If they have not the dæmonic virtue of a few great dramatic poets, they have..plentiful substitutes for it.



%\end{myenumerate}


%%%%%%%%%%%%%%%%%%%%%%%%%%%%%%%%%
%\myitem{denigrate} v.

%\noindent \phonetic{(ˈdɛnɪgreɪt)}

%\noindent [f. ppl. stem of L. dēnigrāre to blacken, f. de- I. 3 + nigrāre to blacken, f. niger, nigr-, black; cf. F. dénigrer (14th c. in Hatzf.). Apparently disused in 18th c., and revived in 19th c.]
%\vspace{-0.3cm}

%\begin{myenumerate}

%\itembf{1.} trans. To blacken, make black or dark. lit. Now rare.

%\P 1623 COCKERAM,  \textit{Denigrate, to make blacke.    
%\P 1646 SIR T. BROWNE  \textit{Pseud. Ep.} vi. xii. 336 The fuliginous and denigrating humor.    
%\P 1657 TOMLINSON  \textit{Renou's Disp. 191 This Lotion will denigrate the hairs of hoary heads.    
%\P 1726 AYLIFFE  \textit{Parergon 231 Drunkenness..denigrates the Colour of the Body.    
%\P 1849 CARD.  \textit{Wiseman Ess. (1853) III. 603 How the north wind should always drive a down-draught, with its denigrating consequences, into the drawing-room.    
%\P 1857 J. RAINE  \textit{Mem. J. Hodgson I. 89 note, The..smoke of pits and manufactories, with..a..dash of denigrated fog from the river.

%\itembf{2.} fig. \itembf{a.} To blacken, sully, or stain (character or reputation); to blacken the reputation of (a person, etc.); to defame.

%\P 1526  \textit{Pilgr. Perf.} (W. de W. 1531) 93 To mynysshe, denygrate, or derke his good name or fame.    
%\P 1656 TRAPP  \textit{Comm. Mark i. 24 This he spake, not to honour Christ, but to denigrate him.    
%\P 1665 BOYLE  \textit{Occas. Refl. iii. v. (1845) 41 [They] do..so denigrate the Reputation of them that oppose them.    
%\P 1871 MORLEY  \textit{Voltaire (1886) 352 Napoleon..paying writers for years to denigrate the memory of Voltaire, whose very name he abhorred.    
%\P 1889 PLUMPTRE in  \textit{Antiquary Apr. 146/2 The character he is at such pains to denigrate.    
%\P 1952  \textit{Daily Herald 17 Nov. 4/1 Elements in this country which have always sought to denigrate the work of the United Nations.    
%\P 1971  \textit{Sci. Amer.} Sept. 237/2 They attempted to denigrate..our most crucial findings.

%\itembf{b.} To darken mentally, obscure. Obs. rare.

%\P 1583 STUBBES  \textit{Anat. Abus. (1877) 78 These..smells..do rather denigrate, darken, and obscure the spirit and sences.

%Hence ˈdenigrated ppl. a., ˈdenigrating ppl. a., deniˈgratory a.

%\P 1857 [see 1].    
%\P 1955  \textit{Times} 9 May 3/4 A revival of Shakespeare's Richard III, certainly the play most denigratory to the King.    
%\P 1967 COAST to Coast
%\P 1965-6 179  \textit{Here Miss Silver-and-Green made an insulting shrug of great beauty, and an exquisite denigratory hand movement.    
%\P 1969  \textit{Daily Tel.} 13 Nov. 8/3 This book..is provocative and controversial..and, intentionally or not, is denigratory.



%\end{myenumerate}


%%%%%%%%%%%%%%%%%%%%%%%%%%%%%%%%%
%\myitem{denizen} n. and a.

%\noindent \phonetic{(ˈdɛnɪzən)}

%\noindent [a. AF. deinzein, denzein, denszein = OF. deinzein, f. AF. deinz, denz, dens, mod.F. dans (:—L. dē intus) within + -ein:—L. -āneus: cf. foreign, forein, L. forāneus.]
%\vspace{-0.3cm}

%\begin{myenumerate}

%\itembf{A.} n.

%1. a.A.1.a A person who dwells within a country, as opposed to foreigners who dwell outside its limits. (In this, the original sense, including and mainly consisting of citizens.) Now rare in lit. sense.

%\P 14.. CHALMERLAIN  \textit{Ayr iii. (Sc. Stat. I), Alswel forreyns as deynseens [tam inhabitantes quam forinseci].    
%\P 1488-9  \textit{Act 4 Hen. VII, c. 23 Coin..conveied into Flaundres..as well by merchauntes straungers as by deynesins.    
%\P 1628 COKE  \textit{On Litt. 129 a, He that is born within the king's liegeance is called sometime a denizen, quasi deins nee, born within... But many times denizen is taken for an alien born that is infranchised or denizated by letters patent.    
%\P 1655 W. GURNALL CHR. in  \textit{Arm. i. 53 The Charter of London..is the birth$\sim$right of its own Denisions, not Strangers.    
%\P 1664 PENNSYLV.  \textit{Archives I. 25 All people shall continue free denizens and enjoy their lands.    
%\P 1734 tr.  \textit{Rollin's Anc. Hist. I. x. 388 To be a natural denizen of Athens it was necessary to be born of a father and mother both free and Athenians.    
%\P 1841 JAMES  \textit{Brigand i, The towns of that age and their laborious denizens.    
%\P 1847 LYTTON  \textit{Lucretia 374 The squalid, ill-favoured denizens, lounging before the doors.

%b.A.1.b transf. and fig. An inhabitant, indweller, occupant (of a place, region, etc.). Used of persons, animals, and plants: chiefly poetic or rhetorical.

%\P 1474 CAXTON  \textit{Chesse ii. iii. C iij, We be not deynseyns in the world but straungers, nor we ben not born in the world for to dwelle and abyde alwey therin, but for to goo and passe thrugh hit.    
%\P a1711 KEN  \textit{Hymns Evang. Poet. Wks.
%\P 1721 I. 11 BLESS'D  \textit{Denizon of Light [an angel].    
%\P 1712-4 POPE  \textit{Rape Lock ii. 55 He summons strait his Denizens of air.    
%\P 1816 SCOTT  \textit{Antiq. viii, Winged denizens of the crag.    
%\P 1860 MAURY  \textit{Phys. Geog. Sea xix. §806 As if the old denizens of the forest had been felled with an axe.

%2. a.A.2.a By restriction: One who lives habitually in a country but is not a native-born citizen; a foreigner admitted to residence and certain rights in a country; in the law of Great Britain, an alien admitted to citizenship by royal letters patent, but incapable of inheriting, or holding any public office.

%\P 1467 in  \textit{Eng. Gilds (1870) 391 Eny citizen or denysen.    Ibid. 393 Yf eny citezen denesyn or foreyn departe out of the seid cite.]    
%\P 1576 FLEMING  \textit{Panopl. Epist. 151 Cæsar had made many that came from Gallia transalpina, free denizens in Rome.    
%\P 1667 E. CHAMBERLAYNE  \textit{St. Gt. Brit. i. (1684) 81 The King by his Prerogative hath Power to Enfranchise an Alien, and make him a Denison.    
%\P 1719 W. WOOD  \textit{Surv. Trade 135 In our Colonies..all Foreigners may be made Denizons for an inconsiderable Charge.    
%\P 1765 BLACKSTONE  \textit{Comm.} I. 374 A Denizen is an alien born, but who has obtained ex donatione regis letters patent to make him an English subject.    
%\P 1830 D'ISRAELI  \textit{Chas. I, III. vi. 94 Charles seemed ambitious of making English denizens of every man of genius in Europe.    
%\P 1873 DIXON  \textit{Two Queens I. iii. iii. 133 Carmeliano, who had become a denizen, was his Latin secretary.

%b.A.2.b fig. One admitted to, or made free of, the privileges of a particular society or fellowship; one who, though not a native, is at home in any region.

%\P 1548 UDALL,  \textit{etc. Erasm. Par. Matt. v. 36 For they be made denisens in heauen.    
%\P a1653 GOUGE  \textit{Comm. Heb. xi. 21 iii. (1655) 88 Naturalized by Iacob, and made free Denisons of the Church.    
%\P 1857 H. REED  \textit{Lect. Eng. Poets II. xiv. 185 He was a denizen of ocean and of lake, of Alpine regions, and of Greek and Italian plains.

%c.A.2.c Used of things: e.g. of foreign words naturalized in a language, etc. In Nat. Hist., A plant or animal believed to have been originally introduced by human agency into a country or district, but which now maintains itself there as if native, without the direct aid of man; cf. colonist 2.

%\P 1578 LYTE  \textit{Dodoens v. lviii. 623 Tarragon..was allowed a Denizon in England long before the time of Ruelius writing.    
%\P a1626 BP. ANDREWES  \textit{Serm. vi. (1661) 148 The word Hypocrite is neither English nor Latin, but as a Denison.    
%\P 1878 HOOKER  \textit{Stud. Flora Pref. 7 To the doubtfully indigenous species I have added Watson's opinion as to whether they are ‘colonists’ or ‘denizens’.    
%\P a1895 MOD.  \textit{Melilotus officinalis is widely diffused in Great Britain, but is probably only a denizen.    
%\P 1933 SHORTER  \textit{Oxf. Eng. Dict. p. vii, Denizens are borrowings from foreign languages which have acquired full English citizenship, aliens are words that retain their foreign appearance and to some extent their foreign sound.    
%\P 1934 S.P.E.  \textit{Tract xlii. 35 Most words when first borrowed are aliens, but if they survive they are gradually accommodated to the language which borrows them and become denizens.

%\itembf{B.} adj. or attrib.

%\P 1483  \textit{Act 1 Rich. III, c. 9 §1 All merchauntes of the nacion of Italie..not made deinseyn.    
%\P 1509-10  \textit{Act 1 Hen. VIII c. 20 §1 Merchaundises of every merchaunt denyseyn and alien.    
%\P 1580 HOLLYBAND  \textit{Treas. Fr. Tong, Hobeine..the right which the prince hath vpon the goods of a stranger, not Denizen.    
%\P 1613 SIR H. FINCH  \textit{Law (1636) 41 The wife is of the same condition with her husband. Franck if he be free, Denison if he be an Englishman, though she were a nief before, or an alien borne.    
%\P 1766 ENTICK  \textit{London IV. 377 This house was..accounted a priory alien till the year 1380, when Richard II..made it denizen.



%\end{myenumerate}


%%%%%%%%%%%%%%%%%%%%%%%%%%%%%%%%%
%\myitem{deprecate} v.

%\noindent \phonetic{(ˈdɛprɪkeɪt)}

%\noindent [f. L. dēprecāt-, ppl. stem of dēprecārī to pray (a thing) away, to ward off by praying, pray against, f. de- I. 2 + precārī to pray.]
%\vspace{-0.3cm}

%\begin{myenumerate}

%\itembf{1.} trans. To pray against (evil); to pray for deliverance from; to seek to avert by prayer. arch.

%\P 1628 EARLE  \textit{Microcosm., Meddling Man (Arb.) 89 Wise men still deprecate these mens kindnesses.    
%\P 1631 GOUGE  \textit{God's Arrows ii. §3. 135 The judgements which Salomon..earnestly deprecateth and prayeth against.    
%\P 1633 BP. HALL  \textit{Medit. (1851) 153, I cannot deprecate thy rebuke: my sins call for correction: but I deprecate thine anger.    
%\P 1778 R. LOWTH  \textit{Transl. Isaiah xlvii. 11 Evil shall come upon thee, which thou shalt not know how to deprecate.    
%\P 1833 H. MARTINEAU  \textit{Three Ages ii. 47 While the rest of the nation were at church, deprecating God's judgements.

%\itembf{2.} intr. To pray (against). Obs. rare.

%\P 1652 GAULE  \textit{Magastrom. 37 Where we are to deprecate..against dangers of waters, let us commemorate the saving of Noah in the flood.

%\itembf{3.} trans. To plead earnestly against; to express an earnest wish against (a proceeding); to express earnest disapproval of (a course, plan, purpose, etc.).

%\P 1641 J. SHUTE  \textit{Sarah \& Hagar (1649) 133 Saint Paul undertaketh..that he shall return and deprecate his fault.    
%\P 1646 SIR T. BROWNE  \textit{Pseud. Ep.} vii. xix. 385 Other accounts..whose verities not onely, but whose relations honest minds doe deprecate.    
%\P 1659 BP. WALTON  \textit{Consid. Considered v. §2 Cappellus..no where that I know affirms this, but rather deprecates is as a calumny.    
%\P 1742 FIELDING  \textit{J. Andrews} iv. vi, I believe..he'd behave so that nobody should deprecate what I had done.    
%\P 1808  \textit{Med. Jrnl.} XIX. 389, I cannot help deprecating the conduct of the other two anatomists.    
%\P 1875 OUSELEY  \textit{Mus. Form xiii. 60 Such a method of proceeding is greatly to be deprecated.    
%\P 1882  \textit{Times} 5 Dec. 7 To deprecate panic is an excellent counsel in itself.

%\itembf{4.} To make prayer or supplication to, to beseech (a person). Obs.

%\P 1624 F. WHITE  \textit{Repl. Fisher Pref. 10 You haue libertie to deprecate his Gratious Maiestie to forget things past.    
%\P 1715-20 POPE  \textit{Iliad} ix. 236 Much he advised them all, Ulysses most, To deprecate the chief, and save the host.    
%\P 1758 JOHNSON  \textit{Idler No. 11 ⁋7 To deprecate the clouds lest sorrow should overwhelm us, is the cowardice of idleness.    
%\P 1822 T. TAYLOR  \textit{Apuleius 75 But the most iniquitous woman, falling at his knees, deprecated him as follows: Why, O my sone I beseech you, do you give [etc.].

%\itembf{b.} absol. To make supplication. Obs.

%\P 1625 DONNE  \textit{Serm.} 24 Feb. (1626) 8 He falls vpon his face..and laments, and deprecates on their behalfe.

%\itembf{5.} To call down by prayer, invoke (evil). Obs.

%\P 1746 W. HORSLEY  \textit{Fool (1748) I. No. 16. 114 Deprecating on unhappy Criminals, under Sentence of Death, all the Mischief they can think of.    
%\P a1790 FRANKLIN  \textit{Autobiog. 442 Upon the heads of these very mischievous men they deprecated no vengeance.

%Hence ˈdeprecated ppl. a., ˈdeprecating vbl. n.

%\P 1768 C. SHAW  \textit{Monody vii. 61 Why..strike this deprecated blow?    
%\P 1839  \textit{Times} 11 July in Spirit Metropol. Conserv. Press (1840) I. 158 To persist in such a deprecated and odious innovation.


%______________________________


%Additions 1993

%Add: [3.] \itembf{b.} More generally, to express disapproval of (a person, quality, etc.); to disparage or belittle. (Sometimes confused with depreciate.) Cf. self-deprecation, etc. s.v. self-.
%   Widely regarded as incorrect, though found in the work of established writers.

%\P 1897  \textit{Daily News} 8 Jan. 6/3 It looks rather an attempt to deprecate distinguished commanders of the Commonwealth to please Restoration Royalists.    
%\P 1927 V. WOOLF  \textit{To Lighthouse 73 He was disposed to slur that comfort over, to deprecate it.    
%\P 1960 C. S. LEWIS  \textit{Stud. in Words i. 18 We tell our pupils that deprecate does not mean depreciate or that immorality does not mean simply lechery because these words are beginning to mean just those things.    
%\P 1965 M. FRAYN  \textit{Tin Men xv. 80 Trying to shrink into himself, as if to deprecate..his authority and to become as other men.



%\end{myenumerate}


%%%%%%%%%%%%%%%%%%%%%%%%%%%%%%%%%
%\myitem{deracinate} v.

%\noindent \phonetic{(dɪˈræsɪneɪt)}

%\noindent [f. F. déracine-r (in OF. desr-), f. dé-, des-, L. dis- + racine root; see -ate3 7.]
%\vspace{-0.3cm}

%\begin{myenumerate}

%trans. To pluck or tear up by the roots; to uproot, eradicate, exterminate. lit. and fig.

%\P 1599 SHAKES.  \textit{Hen. V,} v. ii. 47 The Culter rusts, That should deracinate such Sauagery.    
%\P 1606  \textit{ Tr. \& Cr. i. iii. 99.    
%\P 1659 B. HARRIS  \textit{Parival's Iron Age 27 But neither Arms, nor Victories..[were] able to deracinate or root out this Doctrine.    
%\P 1788  \textit{Lond. Mag. 477 To deracinate and annihilate the whole system of moral, historical and revealed asseverations.    
%\P 1883 STEVENSON  \textit{Silverado Sq. (1886) 80 Disembowelling mountains and deracinating pines!

%\itembf{b.} transf.

%\P 1843 E. JONES  \textit{Poems, Sens. \& Event 167 Chill every river into stagnancy, Deracinate the fruitful earth of growth.

%Hence deraciˈnation, eradication, extirpation.

%\P c1800 tr.  \textit{Sonnini's Trav. I. 227 (L.) Nothing can resist an extreme desire to appear beautiful. The women submit to a painful operation—to a violent and total deracination.



%\end{myenumerate}


%%%%%%%%%%%%%%%%%%%%%%%%%%%%%%%%%
%\myitem{derelict} a. and n.

%\noindent \phonetic{(ˈdɛrɪlɪkt)}

%\noindent [ad. L. dērelict-us, pa. pple. of dērelinquĕre to forsake wholly, abandon, f. de- I. 3 + relinquĕre to leave, forsake.]
%\vspace{-0.3cm}

%\begin{myenumerate}

%\itembf{A.} adj.

%\itembf{1.} Forsaken, abandoned, left by the possessor or guardian; esp. of a vessel abandoned at sea; transf. said of land left dry by the recession of the sea.

%\P 1649 JER. TAYLOR  \textit{Gt. Exemp. i. i. ⁋10 The affections which these exposed and derelict children bear to their mothers.    
%\P 1700 LUTTRELL  \textit{Brief Rel. (1857) IV. 640 A tryal before the barons of the exchequer..about derelict lands left by the sea in Yorkshire.    
%\P 1848 HALLAM  \textit{Mid. Ages i. Notes iii. (1855) I. 106 Gaul, like Britain..had become almost a sort of derelict possession, to be seized by the occupant.    
%\P 1888  \textit{Times} 21 Aug. 9/3 Massowah, which, having been abandoned and left derelict by Egypt..was seized by Italy as a res nullius.

%\P 1774 BURKE  \textit{Amer. Tax. Wks. (1842) I. 171 They easily prevailed, so as to seize upon the vacant, unoccupied, and derelict minds of his friends.

%\itembf{2.} Guilty of dereliction of duty; unfaithful, delinquent (U.S.). Hence derelictness.

%\P 1864  \textit{Daily Tel.} 13 Sept., Probably you will think that United States Commissioner Newton was very ‘derelict’ in his duty.    
%\P 1888 THE  \textit{Voice (N.Y.) 4 Oct., The derelictness of many officials in Kansas.

%\itembf{B.} n.

%\itembf{1.} A piece of property abandoned by the owner or guardian; esp. a vessel abandoned at sea.

%\P 1670  \textit{Lond. Gaz. No. 534/1 A small Virginia ship laden with Tobacco, which they seised as a Derelict, pretending the men had forsaken the ship.    
%\P 1727-51 CHAMBERS  \textit{Cycl.,} Derelicts, in the civil law, are such goods as are wilfully thrown away, or relinquished by the owner.    
%\P 1838 DE QUINCEY  \textit{Mod. Greece Wks. XIV. 320 Often..plague..would absolutely depopulate a region..In such cases, mere strangers would oftentimes enter upon the lands as a derelict.    
%\P 1877 W. THOMSON  \textit{Cruise Challenger iv. 61 On the morning of March 23rd we steamed in search of the derelict.

%b.B.1.b A person abandoned or forsaken.

%\P 1728 SAVAGE  \textit{Bastard Pref., I was a Derelict from my cradle.    
%\P 1873 BROWNING  \textit{Red Cotton Night-Cap Country 258 To try conclusions with my helplessness,—To pounce on, misuse me, your derelict, Helped by advantage that bereavement lends?

%\itembf{2.} One guilty of dereliction of duty (U.S.). Cf. A. 2.

%\P 1888 THE  \textit{Voice (N.Y.) 3 Jan., The Republicans renominated and triumphantly re-elected the derelicts.



%\end{myenumerate}


%%%%%%%%%%%%%%%%%%%%%%%%%%%%%%%%%
%\myitem{deride} v.

%\noindent \phonetic{(dɪˈraɪd)}

%\noindent [ad. L. dērīdē-re to laugh to scorn, scoff at, f. de- I. 4 + L. rīdēre to laugh. Cf. OF. derire and rare derider (Godef.).]
%\vspace{-0.3cm}

%\begin{myenumerate}

%\itembf{1.} trans. To laugh at in contempt or scorn; to laugh to scorn: to make sport of, mock.

%\P 1530 [see DERIDING below].    
%\P 1545 JOYE  \textit{Exp. Dan. iii. 44 In al tymes haue the tyrants derided the godly while they paciently waited for Gods helpe.    
%\P 1581 G. PETTIE  \textit{Guazzo's Civ. Conv. i. (1586) 30 b, Mockers and flouters, who..deride everie man.    
%\P 1611 BIBLE  \textit{Luke xxiii. 35 And the rulers also..derided him.    
%\P 1621 BURTON  \textit{Anat. Mel.} iii. iv. i. i. (1652) 633, I knowe not whether they are more to be pitied or derided.    
%\P 1667 MILTON  \textit{P.L.} xi. 817 Of them derided, but of God observ'd The one just Man alive.    
%\P 1763 J. BROWN  \textit{Poetry \& Mus. v. 75 A Bagpipe (an Instrument which an Englishman derides).    
%\P 1781 GIBBON  \textit{Decl. \& F.} II. xxviii. 99 He justly derides the absurd reverence for antiquity.    
%\P 1853 J. H. NEWMAN  \textit{Hist. Sk. (1873) II. ii. vii. 272 Doctrines which, as an orator, he does not scruple to deride.

%\itembf{2.} intr. To laugh contemptuously or scornfully.

%\P 1619 H. HUTTON  \textit{Follies Anat. (Percy Soc.) 43 The hang$\sim$man..Began to scoffe, and thus deriding said.    
%\P 1663 WOOD  \textit{Life (Oxf. Hist. Soc.) I. 466 A club..where many pretended witts would meet and deride at others.    
%\P 1675 TRAHERNE  \textit{Chr. Ethics App. 562 When they deride at our profession.

%Hence deˈrided ppl. a., deˈriding vbl. n. and ppl. a.; deˈrider, one who derides, a mocker; deˈridingly adv., in a deriding way, with derision.

%\P 1530 PALSGR. 213/2 Deridyng, laughyng to skorne, derision.    
%\P 1543 NECESS.  \textit{Doctr. H iij, A dissembler or rather a deryder of penance.    
%\P 1563-87 FOXE  \textit{A. \& M. (1596) 635 (R.) In the same epistle [he] deridinglie commendeth them.    
%\P 1594 HOOKER  \textit{Eccl. Pol.} iv. i. §1 Prophane and deriding adversaries.    
%\P 1672  \textit{Life \& Death J. Alleine vi. (1837) 71 Deriding and menacing language.    
%\P 1680-90  \textit{Temple Ess. Heroic Virtue Wks.
%\P 1731 I. 221  \textit{Their decayed and derided Idolatry.    
%\P 1695 WOODWARD  \textit{Nat. Hist. Earth ii. (1723) 116 His indiscreet..Derideing..of his Father.    
%\P 1792 F. BURNEY  \textit{Diary Jan., ‘What do you mean by going home?’ cried she, somewhat deridingly.    
%\P 1845 LD. CAMPBELL  \textit{Chancellors (1857) IV. lxxiv. 8 He deridingly called the swan on his badge, ‘a goose’.    
%\P 1857 HUGHES  \textit{Tom Brown i. iii. (1871) 63 [He] smote his young derider on the nose.



%\end{myenumerate}


%%%%%%%%%%%%%%%%%%%%%%%%%%%%%%%%%
%\myitem{derisive} a.

%\noindent \phonetic{(dɪˈraɪsɪv)}

%\noindent [f. L. dērīs-, ppl. stem of dērīdēre to deride + -ive. Cf. OF. derrisif, -ive.]
%\vspace{-0.3cm}

%\begin{myenumerate}

%\itembf{a.} Characterized by derision; scoffing, mocking.

%\P a1662 GAUDEN  \textit{Sacrament 98 (L.) His derisive purple stained..with blood.    
%\P 1725 POPE  \textit{Odyss.} ii. 364 Derisive taunts were spread from guest to guest.    
%\P 1871 H. AINSWORTH  \textit{Tower Hill i. ii, ‘Soh! you are come!’ he exclaimed, in a deep, derisive tone.    
%\P a1897 MOD.  \textit{Newspr. Rept. of Parlt. The statement of the hon. member was received with derisive cheers [i.e. Hear! hear! uttered in derisive tones].

%\itembf{b.} That causes derision, ridiculous.

%\P 1896  \textit{Westm. Gaz.} 25 Feb. 2/1 In thirteen years he has brought a paper costing money to keep it going and with a derisive circulation to the front rank of the world's journalism.    
%\P 1923  \textit{Daily Mail 15 May 8 Germany has provided only a derisive amount to make good that cruel injury.

%Hence deˈrisively adv., in a mocking manner, with derision; deˈrisiveness.

%\P 1665 SIR T. HERBERT  \textit{Trav. (1677) 220 That hyperbole..which derisively term[s] Cairo and Damascus villages.    Ibid. 243 (R.) The Persians [were] thence called Magussæi derisively by other ethnicks.    
%\P 1838 DICKENS  \textit{Nich. Nick.} xlv, ‘Never you mind’, retorted that gentleman, tapping his nose derisively.    
%\P 1847 CRAIG,  \textit{Derisiveness, the state of being derisive.



%\end{myenumerate}


%%%%%%%%%%%%%%%%%%%%%%%%%%%%%%%%%
%\myitem{derogatory} a. and n.

%\noindent \phonetic{(dɪˈrɒgətərɪ)}

%\noindent [ad. L. dērogātōri-us, f. dērogātor: see prec. and -ory. Cf. F. dérogatoire (1341 in Hatzf.).]
%\vspace{-0.3cm}

%\begin{myenumerate}

%\itembf{A.} adj.

%\itembf{1.} Having the character of derogating, of taking away or detracting from authority, rights, or standing, of impairing in force or effect. Const. to, from (of).

%\P 1502-3 PLUMPTON  \textit{Corr. 174 Not intending to have his grant derogatorie unto justice.    
%\P 1638 CHILLINGW.  \textit{Relig. Prot. i. vi. §4. 326 If you conceive such a prayer derogatory from the perfection of your faith.    
%\P 1637-50 ROW  \textit{Hist. Kirk (1842) 501 That none be chosen, or no course be taken derogatory thereto.    
%\P 1651 HOBBES  \textit{Govt. \& Soc. xiv. §12. 221 Provided there be nothing contain'd in the Law..derogatory from his supreme power.    
%\P 1730 SWIFT  \textit{Drapier's Lett. ii. Rep. Comm. Whiteh., A just..exercise of your..royal prerogative, in no manner derogatory or invasive of any liberties.    
%\P 1788 V. KNOX  \textit{Winter Even. II. iv. x. 60 An opinion derogatory from the value of life.    
%\P 1825 SCOTT  \textit{Talism. xx, Incidents mortifying to his pride, and derogatory from his authority.    
%\P 1863 H. COX  \textit{Instit. i. vi. 34 This Act was annulled as derogatory to the King's just rights.

%\itembf{2.} Having the effect of lowering in honour or estimation; depreciatory, disparaging, disrespectful, lowering.

%\P 1563-87 FOXE  \textit{A. \& M. (1596) 1/2 The 2nd [was] derogatorie to kings and emperors.    
%\P 1592 NASHE  \textit{P. Penilesse (ed. 2) 13 a, All holy Writ warrants that delight, so it be not derogatory to any part of Gods owne worship.    
%\P 1776 SIR J. REYNOLDS  \textit{Disc. vii. (1876) 48 Who probably would think it derogatory to their character, to be supposed to borrow.    
%\P 1838-9 HALLAM  \textit{Hist. Lit. III. iv. iii. §34. 151 It would be..derogatory to a man of the slightest claim to polite letters, were he unacquainted with the essays of Bacon.    
%\P 1839 JAMES  \textit{Louis XIV, I. 292 Conduct..derogatory to his rank.    
%\P 1849 DICKENS  \textit{Dav. Copp. (C.D. ed.) 181 To have imposed any derogatory work upon him.    
%\P 1860 FARRAR  \textit{Orig. Lang. (1865) 40 What plans are consonant to, and what are derogatory of God's..Infinite Wisdom.

%\itembf{3.} derogatory clause: a clause in a legal document, a will, deed, etc., by which the right of subsequently altering or cancelling it is abrogated, and the validity of a later document, doing this, is made dependent on the correct repetition of the clause and its formal revocation. Obs.

%\P 1528 in  \textit{Strype Eccl. Mem. I. App. xxx. 89 As doth appear by composition made..and also confirmed by Boniface the IV..with clauses derogatory.    
%\P 1590 SWINBURNE  \textit{Testaments 266 What maner of reuocation is to be made in the second testament, that it may suffice to reuoke the former testament, wherein is a clause derogatorie of the will of the testator.    
%\P a1626 BACON  \textit{Max. \& Uses Com. Law xix. (1636) 70 A derogatory clause is good to disable any latter act, except you revoke the same clause before you proceed to establish any later disposition or declaration.

%\itembf{B.} n. Obs. rare—0.

%\P 1611 COTGR.,  \textit{Derogatoire, a derogatorie, or act of derogation.



%\end{myenumerate}


%%%%%%%%%%%%%%%%%%%%%%%%%%%%%%%%%
%\myitem{descry} v.1

%\noindent \phonetic{(dɪˈskraɪ)}

%\noindent [app. a. OF. descrier to cry, publish, decry, f. des-, dé-, L. dis- + crier to cry.
%\vspace{-0.3cm}

%\begin{myenumerate}
%   The sense-development is not altogether clear; it was perhaps in some respect influenced by the reduction of descrive to descry (see next), and consequent confusion of the two words: cf. descrive v. 4, also describe v. 7. In several instances it is difficult to say to which of the verbs the word belongs: thus
%\P c1300 K. ALIS.  \textit{138 For astronomye and nygremauncye No couthe ther non so muche discryghe.]

%\itembf{I.} To cry out, declare, make known, bewray.

%\itembf{1.} trans. To cry out, proclaim, announce, as a herald. Obs. rare.
%\P 1377 in  \textit{descrive v. 4.]

%\P a1440 SIR Eglam.
%\P 1178 HAROWDES  \textit{of armes than they wente, For to dyscrye thys turnayment In eche londys ȝende.

%\itembf{2.} To announce, declare; to make known, disclose, reveal: a.I.2.a of persons. b.I.2.b of things. Obs.

%\P c1460 TOWNELEY  \textit{Myst. (Surtees) 203 My name to you wille I descry.    
%\P 1549-62 STERNHOLD \& H.  \textit{Ps. xxv. 3 Thy right waies unto me, Lord, descrye.    
%\P 1621 BURTON  \textit{Anat. Mel.} i. ii. i. i, At length Jupiter descried himself, and Hercules yielded.    
%\P 1655-60 STANLEY  \textit{Hist. Philos. (1701) 290/2 Diogenes, thou..Who to content the ready way To following Ages didst descry.

%\P c1430 FREEMASONRY  \textit{323 Hyt [the seventhe poynt] dyscryeth wel opunly, Thou schal not by thy maystres wyf ly.    
%\P 1590 SPENSER  \textit{F.Q.} i. x. 34 Whose sober lookes her wisedome well descride.    
%\P a1592 H. SMITH  \textit{Wks. (1867) II. 200 This light..doth not only descry itself, but all other things round about it.    
%\P 1635 COWLEY  \textit{Davideis iv. 231 A thoughtful Eye That more of Care than Passion did descry.    
%\P 1639 DRUMMOND  \textit{of Hawthornden Fam. Epistles Wks. (1711) 140 His cheeks scarce with a small down descrying his sex.

%c.I.2.c With a sense of injurious revelation: To disclose what is to be kept secret; to betray, bewray; to lead to the discovery of. Obs.

%\P c1340  \textit{Cursor M.} 7136 (Trin.) Þat was a greet folye hir lordes [i.e. Samson's] counsel to discrye.    ?
%\P c1475 SQR.  \textit{lowe Degre 110 Thy counsayl shall i never dyscry.    
%\P 1596 NASHE  \textit{Saffron Walden 131 That he be not descride by his alleadging of Authors.    
%\P 1606 HOLLAND  \textit{Sueton. 90 Hee had like to have descried them [his parents] with his wrawling.    
%\P 1614 BP. HALL  \textit{Recoll. Treat. 509 In notorious burglaries, oft-times there is..a weapon left behinde, which descrieth the authors.    
%\P 1670 MILTON  \textit{Hist. Eng. 11, His purple robe he [Alectus] had thrown aside lest it should descry him.

%\itembf{II.} To cry out against, cry down, decry.

%\itembf{3.} To shout a war-cry upon, challenge to fight; = ascry v. 1 b.

%\P c1400 ROWLAND \& O.  \textit{273 No kyng in Cristyante Dare..discrye hym ther with steven.    
%\P 1480 CAXTON  \textit{Chron. Eng. cxcvii. 175 The gentil knyghtes fledden and the vileyns egrely hem discryed and grad an high ‘yelde yow traytours!’

%\itembf{4.} To denounce, disparage; = decry v. 2. Obs.

%\P c1400  \textit{York Manual (Surtees) p. xvi, We curse and descry..all thos that thys illys hase done.    
%\P 1677 GILPIN  \textit{Dæmonol. (1867) 407 They contemn and descry those, as ignorant of divine mysteries.

%\itembf{5.} To cry down, depreciate (coin); = decry.

%\P 1638 SIR R. COTTON  \textit{Abstr. Rec. Tower 23 The descrying of the Coyne.

%\itembf{III.} To get sight of, discover, examine.

%\itembf{6.} To catch sight of, esp. from a distance, as the scout or watchman who is ready to announce the enemy's approach; to espy.

%\P c1340  \textit{Gaw. \& Gr. Knt. 81 Þe comlokest [lady] to discrye.    
%\P c1430 SIR TRYAM.  \textit{1053, Xii fosters dyscryed hym then, That were kepars of that fee.    
%\P 1569 T. STOCKER tr.  \textit{Diod. Sic. iii. viii. 114 He might descry a mightie and terrible Nauie..sayling towards the citie.    
%\P 1605 PLAY STUCLEY in  \textit{Simpson Sch. Shaks. (1878) 190 The English sentinels do keep good watch; If they descry us all our labour's lost.    
%\P 1791 COWPER  \textit{Iliad} iii. 38 In some woodland height descrying A serpent huge.    
%\P 1868 QUEEN VICTORIA  \textit{Life Highl. 39 To meet Albert, whom I descried coming towards us.    
%\P 1877 BLACK  \textit{Green Past. xxxiii. (1878) 267 At intervals we descried a maple.

%\itembf{7.} To discover by observation; to find out, detect; to perceive, observe, see.

%\P c1430 SYR.  \textit{Tryam. 783 Hors and man felle downe..And sone he was dyscryed.    
%\P 1581 J. BELL  \textit{Haddon's Answ. Osor. 491 b, There is no man..that will not easily descry..want of Judgement..in you.    
%\P 1659 HAMMOND  \textit{On Ps. xxxiv. Paraphr. 181 Being by them descryed to be David.    
%\P 1667 MILTON  \textit{P.L.} i. 290 To descry new Lands, Rivers or Mountains in her spotty Globe.    
%\P 1797 SOUTHEY  \textit{Ballad K. Charlemain 1 All but the Monarch could plainly descry From whence came her white and her red.    
%\P 1812 J. WILSON  \textit{Isle of Palms ii. 582 He can descry That she is not afraid.    
%\P 1862 LD. BROUGHAM  \textit{Brit. Const. xvi. 249 The bounds which separated that school from Romanism were very difficult to descry.

%\P 1670 NARBOROUGH JRNL. in  \textit{Acc. Sev. Late Voy. (1711) 33, I could not see any sign of People..but still Hills and Vallies as far as we could descry.

%b.III.7.b intr. To discern, discriminate. Obs. rare.

%\P 1633 P. FLETCHER  \textit{Purple Isl. viii. viii. 108 Pure Essence, who hast made a stone descrie 'Twixt natures hid.

%\itembf{8.} trans. To investigate, spy out, explore. Obs.

%\P 1596 DRAYTON  \textit{Legends iii. 175 He had iudicially descryde The cause.    
%\P 1611 BIBLE  \textit{Judg. i. 23 The house of Ioseph sent to descry Bethel.    
%\P 1742 SHENSTONE  \textit{Schoolmistress 145 Right well she knew each temper to descry.



%\end{myenumerate}


%%%%%%%%%%%%%%%%%%%%%%%%%%%%%%%%%
%\myitem{desecrate} v.

%\noindent \phonetic{(ˈdɛsɪkreɪt)}

%\noindent [f. de- II. 1 + stem of con-secrate. In L. dēsecrāre or dēsacrāre meant to consecrate, dedicate. OF. had des-sacrer (des- = L. dis-) still in Cotgr. (1611) ‘to profane, violate, unhallow’, = It. dissacrare ‘to unconsecrate, unhallow’ (Florio); these may have suggested the formation of the English word.]
%\vspace{-0.3cm}

%\begin{myenumerate}

%trans. To take away its consecrated or sacred character from (anything); to treat as not sacred or hallowed; to profane.

%\P a1677 BARROW Serm. Wks.
%\P 1687 I. XV.  \textit{213 If we do venture to swear..upon any slight or vain..occasion, we then desecrate Swearing, and are guilty of profaning a most sacred Ordinance. [Not in Phillips, Cocker, Kersey.]    
%\P 1675 [see DESECRATING ppl. a.].    
%\P 1721 BAILEY,  \textit{Desecrate, to defile or unhallow.    
%\P 1741 MIDDLETON  \textit{Cicero I. vi. 416 What Licinia had dedicated..could not be considered as sacred: so that the Senate injoined the Prætor to see it desecrated and to efface whatever had been inscribed upon it.    
%\P 1776 HORNE  \textit{On Ps. lxxiv. (R.) When the soul sinks under a temptation, the dwelling-place of God's name is desecrated to the ground.    
%\P 1837 J. H. NEWMAN  \textit{Par. Serm. (ed. 2) III. xxi. 333 More plausibly even might we desecrate Sunday.    
%\P 1860 PUSEY  \textit{Min. Proph. 204 The..vessels of the Temple..were desecrated by being employed in idol-worship.

%\itembf{b.} To divert from a sacred to a profane purpose; to dedicate or devote to something evil.

%\P 1825  \textit{Blackw. Mag.} XVIII. 156 With a libation of unmixed water..did he devote us to the infernal gods—or..desecrate us to the Furies.    
%\P 1849 SIR J. STEPHEN  \textit{Eccl. Biog. (1850) I. 312 Particular spots..were desecrated to Satan.    
%\P 1860 PUSEY  \textit{Min. Proph. 76 Desecrating to false worship the place which had been consecrated by the revelation of the true God.

%c.c To dismiss or degrade from holy orders. arch.

%\P 1674 BLOUNT  \textit{Glossogr., Desecrate, to discharge of his orders, to degrade.    
%\P 1676 in  \textit{Coles.    
%\P c1800 W. TOOKE  \textit{Russia (W.), The [Russian] clergy can not suffer corporal punishment without being previously desecrated.



%\end{myenumerate}


%%%%%%%%%%%%%%%%%%%%%%%%%%%%%%%%%
%\myitem{desiccate} v.

%\noindent \phonetic{(dɪˈsɪkeɪt, ˈdɛsɪkeɪt)}

%\noindent [f. L. dēsiccāt-, ppl. stem of dēsiccāre to dry completely, dry up, f. de- I. 3 + siccāre to dry, siccus dry.
%\vspace{-0.3cm}

%\begin{myenumerate}
%   (For changing stress see note to contemplate: deˈsiccate is the only pronunciation in Dicts. down to 1864, and in Ogilvie 1882, Cassell 1883.)]

%\itembf{1.} trans. To make quite dry; to deprive thoroughly of moisture; to dry, dry up. Also fig.
%   In U.S. applied to the thorough drying of articles of food for preservation.

%\P 1575 TURBERV.  \textit{Faulconrie 261 They doe mollifie, and desiccate the wounde or disease.    
%\P 1626 BACON  \textit{Sylva} §727 Wine helpeth to digest and desiccate the moisture.    
%\P 1657 TOMLINSON  \textit{Renou's Disp. 181 This..will desiccate an ulcer.    
%\P 1808 J. BARLOW  \textit{Columb. iv. 426 No..courtly art [shall] Damp the bold thought or desiccate the heart.    
%\P 1832 I. TAYLOR  \textit{Saturday Even. (1834) 297 Atheism in all its forms desiccates the affections.    
%\P 1839 BAILEY  \textit{Festus Proem, Though we should by art Bring earth to gas and desiccate the sea.    
%\P 1883 PROCTOR in  \textit{Knowl. 3 Aug. 74/1 The shock was of sufficient intensity to..partially desiccate the muscular tissues.

%\itembf{2.} intr. To become dry. rare.

%\P 1679 RYCAUT  \textit{Grk. Church 277 Bodies of such whom they have Canonized for Saints to continue unconsumed, and..to dry and desiccate like the Mummies in Egypt.

%Hence desiccating vbl. n. and ppl. a.

%\P 1651 tr.  \textit{Bacon's Life \& Death 7 They speak much of the Elementary Quality of Siccity or Drienesse; and of things Desiccating.    
%\P 1866 J. MARTINEAU  \textit{Ess. I. 388 The very things which this desiccating rationalism flung off.    
%\P 1871 B. STEWART  \textit{Heat §63 The..air was..thoroughly dried by being passed through a desiccating apparatus.    
%\P 1893 ATHENæUM  \textit{1 Apr. 402/2 That desiccating of the Anglo-Saxon in North America which Humboldt and others have commented upon.



%\end{myenumerate}


%%%%%%%%%%%%%%%%%%%%%%%%%%%%%%%%%
%\myitem{desuetude} n.

%\noindent \phonetic{(ˈdɛswɪtjuːd)}

%\noindent [a. F. désuétude (1596 in Hatzf.), ad. L. dēsuētūdo disuse, f. dēsuētus, pa. pple. of dēsuēscĕre to disuse, become unaccustomed, f. de- 6 + suēscĕre to be accustomed, to be wont.]
%\vspace{-0.3cm}

%\begin{myenumerate}

%\itembf{1.} A discontinuance of the use or practice (of anything); disuse; protracted cessation from.

%\P 1623 COCKERAM,  \textit{Desuetude, lacke of vse.    
%\P 1629 tr.  \textit{Herodian (1635) 131 A generall lazinesse and desuetude of Martiall Exercises.    
%\P 1652-62 HEYLIN  \textit{Cosmogr., To Rdr., My desuetude from those younger studies.    
%\P 1661 BOYLE  \textit{Style of Script. (1675) 139 By a desuetude and neglect of it.    
%\P 1677 HALE  \textit{Prim. Orig. Man. ii. iv. 160 Desuetude from their former Civility and Knowledge.    
%\P 1706 J. SERGEANT  \textit{Account of Chapter (1853) Pref. xv, By a desuetude of acting, expire, and be buried in oblivion.

%\itembf{b.} The passing into a state of disuse.

%\P 1821 LAMB  \textit{Elia Ser. i. New Year's Eve, The gradual desuetude of old observances.

%\itembf{2.} The condition or state into which anything falls when one ceases to use or practise it; the state of disuse.

%\P 1637-50 ROW  \textit{Hist. Kirk (1842) 14 To revive acts buried and brought in [= into] desuetude by Prelats.    
%\P 1678 R. BARCLAY  \textit{Apol. Quakers x. §22. 315 The weighty Truths of God were neglected, and, as it were, went into Desuetude.    
%\P 1703  \textit{Lond. Gaz. No. 3914/4 Reviving such [Laws] as are in desuetude.    
%\P 1820 SCOTT  \textit{Monast.} i, The same mode of cultivation is not yet entirely in desuetude in some distant parts of North Britain.    
%\P 1826  \textit{Q. Rev.} XXXIV. 6 This beautiful work..fell (as the Scots lawyers express it) into desuetude.    
%\P 1874 GREEN  \textit{Short Hist. iv. §2. 168 The exercise of rights which had practically passed into desuetude.



%\end{myenumerate}


%%%%%%%%%%%%%%%%%%%%%%%%%%%%%%%%%
%\myitem{desultory} a. (n.)

%\noindent \phonetic{(ˈdɛsəltərɪ)}

%\noindent [ad. L. dēsultōri-us of or belonging to a vaulter, superficial, desultory, f. dēsultor: see desultor.]
%\vspace{-0.3cm}

%\begin{myenumerate}

%\itembf{A.} adj.

%\itembf{1.} Skipping about, jumping or flitting from one thing to another; irregularly shifting, devious; wavering, unsteady. lit. and fig.

%\P 1581 MULCASTER  \textit{Positions xxxix. (1887) 220 Not resting vpon any one thing, but desultorie ouer all.    
%\P 1594 BP. ANDREWES  \textit{Serm. II. 68 ‘Winter brooks’ as Job termeth flitting desultory Christians.    
%\P 1655 FULLER  \textit{Ch. Hist.} iii. ii. §31 The Crown, since the Conquest, never observed a regular, but an uncertain and desultory motion.    
%\P 1699 BENTLEY  \textit{Phal. 86 Persons of a light and desultory temper, that skip about, and are blown with every wind, as Grass$\sim$hoppers are.    
%\P 1699 BURNET  \textit{39 Art. xx. (1700) 195 All men ought to avoid the Imputations of a desultory Levity.    
%\P 1748 J. MASON  \textit{Elocut. 19 To cure an uneven, desultory Voice..do not begin your Periods..in too high or too low a Key.    
%\P 1754 EELES in  \textit{Phil. Trans. XLIX. 132 That desultory motion, by which it flies off from an electrified body.    
%\P 1784 H. ELLIOTT in  \textit{Dk. of Leeds's Pol. Mem. (1884) 259 There is also a peculiar desultory motion in His Royal Highnesses eye.    
%\P 1789 G. WHITE  \textit{Selborne xv. (1853) 63, I shot at it but it was so desultory that I missed my aim.    
%\P 1825 SOUTHEY  \textit{Paraguay Proem., Ceasing here from desultory flight.

%\itembf{2.} Pursuing a disconnected and irregular course of action; unmethodical.

%\P 1740 WARBURTON  \textit{Let. 2 Feb. (R.), This makes my reading wild and desultory.    
%\P 1773 BURKE  \textit{Corr.} (1844) I. 427 Writing..not in a desultory and occasional manner, but systematically.    
%\P 1779 F. BURNEY  \textit{Diary 14 June, She is a very desultory reader.    
%\P 1827 HARE  \textit{Guesses (1859) 146 Desultory reading is indeed very mischievous, by fostering habits of loose, discontinuous thought.    
%\P 1855 MILMAN  \textit{Lat. Chr. (1864) IV. vii. i. 3 A desultory and intermitting warfare.    
%\P 1872 GEO. ELIOT  \textit{Middlem. xxix. (1873) 104 Guests whose desultory vivacity makes their presence a fatigue.    
%\P 1876 STUBBS  \textit{Med. \& Mod. Hist. ii. 41 The temptation to desultory research must in every case be very great, and desultory research, however it may amuse or benefit the investigator, seldom adds much to the real stock of human knowledge.

%b.A.2.b Of a single thing: Coming disconnectedly; random.

%\P a1704 R. L'ESTRANGE  \textit{(J.), 'Tis not for a desultory thought to attone for a lewd course of life.    
%\P 1822 HAZLITT  \textit{Table-t. Ser. ii. vi. (1869) 131 He no sooner meditates some desultory project, than [etc.].

%c.A.2.c Irregular and disconnected in form or appearance; motley. rare.

%\P 1842 ALISON  \textit{Hist. Europe (1849-50) XIII. lxxxviii. §42. 148 They..shuddered when they gazed on the long and desultory array of Cossacks..sweeping by.    
%\P 1866 HOWELLS  \textit{Venet. Life ii. 19 A beggar in picturesque and desultory costume.

%\itembf{B.} n. A horse trained for the ‘desultor’ in a circus. Obs. rare—1.

%\P 1653 URQUHART  \textit{Rabelais i. xxiii, These horses were called desultories.



%\end{myenumerate}


%%%%%%%%%%%%%%%%%%%%%%%%%%%%%%%%%
%\myitem{detrimental} a. and n.

%\noindent \phonetic{(dɛtrɪˈmɛntəl)}

%\noindent [f. detriment n. + -al1.]
%\vspace{-0.3cm}

%\begin{myenumerate}

%\itembf{A.} adj. Causing loss or damage; harmful, injurious, hurtful.

%\P 1656 BLOUNT  \textit{Glossogr., Detrimental, hurtful, dangerous, full of loss.    
%\P a1661 FULLER  \textit{Worthies} (1840) I. 281 A gift indeed..loaded with no detrimental conditions.    
%\P 1719 W. WOOD  \textit{Surv. Trade 84 That the Trade..is most detrimental to the Nation.    
%\P 1801  \textit{Med. Jrnl.} V. 1 Particularly detrimental to the constitution.    
%\P 1872 YEATS  \textit{Growth Comm. 271 Their admission was detrimental to French industry.    
%\P 1875 JOWETT  \textit{Plato} (ed. 2) IV. 53 Paradoxes..which [are]..detrimental to the true course of thought.

%\itembf{B.} n. A person or thing that is prejudicial; in Society slang, a younger brother of the heir of an estate; an ineligible suitor.

%\P 1831 WESTM.  \textit{Rev. XIV. 424 The eldest son is pursued by..damsels, while the younger are termed ‘detrimentals’..and avoided by ‘mothers and daughters’ as more dangerous company than the plague.    
%\P 1832 MARRYAT  \textit{N. Forster xxv, These detrimentals (as they have named themselves) may be provided for.    
%\P 1854 LADY  \textit{Lytton Behind the Scenes I. ii. iii. 188 There were also plenty of detrimentals, such as younger brothers, unpaid red tapeists, heiress-seekers, and political connection-hunters.    
%\P 1870 C. F. GORDON-CUMMING in  \textit{Gd. Words 137/1 The sisters of the wife being considered detrimentals, are placed in Buddhist convents.    
%\P 1886 HOUSEH.  \textit{Words 13 Mar. 400 (Farmer) A detrimental, in genteel slang, is a lover, who, owing to his poverty is ineligible as a husband; or one who professes to pay attentions to a lady without serious intention of marriage, and thereby discourages the intentions of others.    
%\P 1893 MRS. C. PRAED  \textit{Outlaw \& Lawmaker II. 80 Mrs. Valliant..thought that the detrimentals kept off desirable suitors.

%Hence detrimenˈtality, detriˈmentalness.

%\P 1727 BAILEY  \textit{vol. II, Detrimentalness, prejudicialness.    
%\P 1873  \textit{Daily News} 5 Aug., When you are hinting to your fair daughter the detrimentality of Charlie Fraser..who has his subaltern's pay and about 50l. a year thrown in.



%\end{myenumerate}


%%%%%%%%%%%%%%%%%%%%%%%%%%%%%%%%%
%\myitem{detritus} Physiogr.

%\noindent \phonetic{(dɪˈtraɪtəs)}

%\noindent [a. L. dētrītus (u-stem) rubbing away.
%\vspace{-0.3cm}

%\begin{myenumerate}
%\P 1780 BY HATZ.-DARM. EARLIER in  \textit{the century, according to the Dict. de Trévoux, the more correct détritum was used in F.]

%\itembf{1.} Wearing away or down by detrition, disintegration, decomposition. Obs.

%\P 1795 HUTTON  \textit{Theory of Earth (1797) I. 115 Such materials as might come from the detritus of granite.    Ibid. 206, I have nowhere said that all the soil of this earth is made from the decomposition or detritus of these stony substances.    
%\P 1802 PLAYFAIR  \textit{Illustr. Hutton. Th. Wks.
%\P 1822 I. 63 THE  \textit{effects of waste and detritus.    Ibid. 113 Proofs of a detritus which nothing can resist.    Ibid. 123 The waste and detritus to which all things are subject.

%\itembf{2.} Matter produced by the detrition or wearing away of exposed surfaces, especially the gravel, sand, clay, or other material eroded and washed away by aqueous agency; a mass or formation of this nature.

%\P 1802 PLAYFAIR  \textit{Illustr. Hutton. Th. Wks.
%\P 1822 I. 409  \textit{The quantity of detritus brought down by the rivers.    Ibid. 425 The distance to which the detritus from the land is confessedly carried.    
%\P 1802  \textit{ in Edin. Rev. I. 207 When the detritus of the land is delivered by the rivers into the sea.    
%\P 1823 W. BUCKLAND  \textit{Reliq. Diluv. 26 Deposits of diluvial detritus, like the surface gravel beds of England.    
%\P 1832 H. T. DE  \textit{la Beche Geol. Man. (ed. 2) 210 The whole is evidently a detritus of the Alpine rocks, and in it organic remains are by no means common.    
%\P 1851 MAYNE  \textit{Reid Scalp Hunt. xli, We entered the cañon, and galloped over the detritus.    
%\P 1862 DANA  \textit{Man. Geol. 643 The fine earthy material deposited by streams or their sediment, is called silt or detritus.    
%\P 1876 PAGE  \textit{Adv. Text-bk. Geol. xix. 389 That broad valley..covered to an immense depth with an angular detritus.

%\itembf{3.} transf. and fig. Waste or disintegrated material of any kind; debris.

%\P 1834 J. FORBES  \textit{Laennec's Dis. Chest (ed. 4) 189 The walls of this abscess had..no surface, the pus being observed gradually to pass into a purulent detritus, and this into a firmer tissue.    
%\P 1849 H. ROGERS  \textit{Ess. II. vi. 306 The loose detritus of thought, washed down to us through long ages.    
%\P 1851 SIR F. PALGRAVE  \textit{Norm. \& Eng. I. 701 The detritus of languages covering the Northern Gauls.    
%\P 1876 tr.  \textit{Wagner's Gen. Pathol. 192 The red blood-corpuscles and fibrinous detritus..are reabsorbed.

%\itembf{b.} An accumulation of debris of any sort.

%\P 1851 LAYARD  \textit{Pop. Acc. Discov. Nineveh vii. 134 We found ourselves at the foot of an almost perpendicular detritus of loose stones.    
%\P 1866 R. CHAMBERS  \textit{Ess. Ser. i. 185 There is a detritus of ruin in every corner, composed of broken toys, sofa-pillows, foot-stools.


%______________________________

%Additions 1997

%Add: \itembf{4.} spec. in Ecol. Non-living organic material, esp. as a source of nourishment. Freq. attrib., esp. in detritus-feeding.

%\P 1925 O. D. HUNT in  \textit{Jrnl. Marine Biol. Assoc. XIII. 567 Those which feed by selecting from the surrounding water the suspended micro-organisms and detritus,..for want of a better term, may be termed Suspension-feeders.    
%\P 1949  \textit{New Biol. VI. 17 The appearance of reeds..leads to large increases in the numbers of algæ and in the amount of organic detritus.    
%\P 1959  \textit{Ibid.} XXIX. 99 Others have used tubs containing water in which algae and small herbivorous and detritus-feeding animals succeeded one another.    
%\P 1984 A. C. \&  \textit{A. Duxbury Introd. World's Oceans xv. 481 (caption) The sea cucumbers feed on detritus suspended in the water.    
%\P 1990  \textit{Compl. Angler's Guide Spring 6/1 Nymphs mostly live in or among the silt and bottom detritus.



%\end{myenumerate}


%%%%%%%%%%%%%%%%%%%%%%%%%%%%%%%%%
%\myitem{dexterous} dextrous a.

%\noindent \phonetic{(ˈdɛkstərəs, ˈdɛkstrəs)}

%\noindent [f. L. dexter, dextr- right, handy, dexterous, dextra the right hand + -ous. If an analogous word had been formed in L., it would have been dextrōsus; hence dextrous (cf. sinistrous) is the more regular form; but dexterous appears to prevail in 19th c. prose.]
%\vspace{-0.3cm}

%\begin{myenumerate}

%\itembf{1.} Situated on the right side or right-hand; right, as opposed to left; = dexter 1, dextral.

%\P 1646 SIR T. BROWNE  \textit{Pseud. Ep.} iv. v. 190 The dextrous and sinistrous parts of the body.    
%\P 1678 CUDWORTH  \textit{Intell. Syst. 221 The Contrarieties and Conjugations of things, such as..Dextrous and Sinistrous, Eaven and Odd, and the like.

%\itembf{2.} Handy, convenient, suitable, fitting. Obs.

%\P 1605 BACON  \textit{Adv. Learn.} ii. xv. §2 The Art..is barren, that is, not dexterous to be applyed to the serious vse of businesse and occasions.

%\itembf{3.} Deft or nimble of hand, neat-handed; hence skilful in the use of the limbs and in bodily movements generally.

%\P 1635-56 COWLEY  \textit{Davideis iv. 353 So swift, so strong, so dextrous none beside.    
%\P 1650 FULLER  \textit{Pisgah i. 423 Though skilfull in the Mathematicall..so dexterous in the manual part.    
%\P 1697 DRYDEN  \textit{Virg. Georg. iii. 570 The dext'rous Huntsman wounds not these afar.    
%\P 1776 GIBBON  \textit{Decl. \& F.} I. xviii. 483 He was a dextrous archer.    
%\P 1801 SOUTHEY  \textit{Thalaba iii. xviii, With dexterous fingers.    
%\P 1818 JAS. MILL  \textit{Brit. India II. iv. i. 13 The flagellants in India are said to be so dextrous, as to kill a man with a few strokes of the chawbuck.

%\itembf{4.} Having mental adroitness or skill; skilful or expert in contrivance or management; clever.

%\P 1622 MABBE tr.  \textit{Aleman's Guzman d'Alf. ii. * * iv a, As dextrous in Letters as disciplin'd in Armes.    
%\P 1642 FULLER  \textit{Holy \& Prof. St. iv. ix. 281 Generally the most dexterous in spirituall matters are left-handed in temporall businesse.    
%\P 1672 MARVELL  \textit{Reh. Transp. i. 194 A dexterous Scholastical Disputant.    
%\P a1720 SHEFFIELD  \textit{(Dk. Buckhm.) Wks. (1753) II. 25 To which, that dextrous Minister replied something haughtily.    
%\P 1838 THIRLWALL  \textit{Greece} IV. 433 A dexterous politician of Lysander's school.    
%\P a1843 SOUTHEY  \textit{Doctor clxxiv. (1862) 457 She was devout in religion, decorous in conduct..dextrous in business.    
%\P 1850 A. JAMESON  \textit{Leg. Monast. Ord. (1863) 333 Dexterous in the management of temporal affairs.

%\itembf{b.} In a bad sense: ‘Clever’, crafty, cunning.

%\P 1701 tr.  \textit{Le Clerc's Prim. Fathers (1702) 154 Eusebius..was a dextrous Person which made no scruple to subscribe to Terms which he did not like.    
%\P a1715 BURNET  \textit{Own Time (1823) I. 332 Ward..was a very dexterous man if not too dexterous; for his sincerity was much questioned.

%\itembf{5.} Of things: Done with or characterized by dexterity; skilful, clever.

%\P a1625 BEAUM. \& FL.  \textit{Bloody Brother iv. ii, He..cuts through the elements for us..In a fine dextrous line.    
%\P 1627-77 FELTHAM  \textit{Resolves i. lxxxviii. 136 A dexterous Art shows cunning and industry; rather than judgment and ingenuity.    
%\P 1748  \textit{Anson's Voy.} ii. xiv. 287 Trained to the dexterous use of their fire-arms.    
%\P 1808 SYD. SMITH  \textit{Wks. (1859) I. 115/1 An uninterrupted series of dexterous conduct.

%\itembf{6.} Using the right hand in preference to the left; right-handed.

%   In mod. Dicts.



%\end{myenumerate}


%%%%%%%%%%%%%%%%%%%%%%%%%%%%%%%%%
%\myitem{diaphanous} a.

%\noindent \phonetic{(daɪˈæfənəs)}

%\noindent [f. med.L. diaphan-us (see diaphane) + -ous. The form diaphaneous more closely represented the Gr.: cf. diaphaneity.]
%\vspace{-0.3cm}

%\begin{myenumerate}

%Permitting the free passage of light and vision; perfectly transparent; pellucid.

%\P 1614 RALEIGH  \textit{Hist. World i. i. §7 Aristotle calleth light a quality inherent, or cleauing to a Diaphanous body.    
%\P 1633 T. ADAMS  \textit{Exp. 2 Peter ii. 4 In hell there shall be nothing diaphanous, perspicuous, clear.    
%\P c1645 HOWELL  \textit{Lett.} I. i. xxix, To transmute Dust and Sand to such a diaphanous pellucid dainty body as you see a Crystal-Glasse is.    
%\P 1669 W. SIMPSON  \textit{Hydrol. Chym. 10 The diaphaneous texture of the particles in the vitrioline solution.    
%\P 1680 BOYLE  \textit{Scept. Chem. v. 326 The one substance is Opacous, and the other somewhat Diaphanous.    
%\P 1794 MARTYN  \textit{Rousseau's Bot. xxxii. 500 The fructifications are in a diaphanous membrane.    
%\P 1833 PENNY  \textit{Cycl. I. 450/2 The crystals of the amethyst vary from diaphanous to translucent.    
%\P 1868 DUNCAN  \textit{Insect World ii. 59 The wings are whitish, not diaphanous.    
%\P 1895 THE  \textit{Lady 31 Jan. 133 With this was worn a diaphanous white picture hat caught up with..white ribbons.

%Hence diˈaphanously adv., in a diaphanous manner, transparently; diˈaphanousness, diaphanous quality, transparency.

%\P 1683 E. HOOKER  \textit{Pref. Epist. Pordage's Mystic Div., Most Diaphanously, perspicuously, no less clearly..than the Sun Beams upon a Wall of Crystall.    
%\P 1710 T. FULLER  \textit{Pharm. Extemp. 220 As here order'd 'twill be diaphanously clear.    
%\P 1727 BAILEY  \textit{vol. II, Diaphaneity, Diaphanousness, the property of a diaphanous Body.    
%\P 1969 A. GLYN  \textit{Dragon Variation i. 4 A pot-bellied shah was reclining on cushions..while a diaphanously veiled maiden knelt before him, offering a cup.



%\end{myenumerate}


%%%%%%%%%%%%%%%%%%%%%%%%%%%%%%%%%
%\myitem{diatribe} n.

%\noindent \phonetic{(ˈdaɪətraɪb)}

%\noindent [a. F. diatribe (15th c. in Hatz.-Darm.), ad. L. diatriba a learned discussion, a school, a. Gr. διατριβή a wearing away (of time), employment, study, and (in Plato) discourse, f. διατρίβ-ειν to rub through or away. The senses in F. and Eng. exactly correspond.]
%\vspace{-0.3cm}

%\begin{myenumerate}

%\itembf{1.} A discourse, disquisition, critical dissertation. arch.

%\P 1581 J. BELL  \textit{Haddon's Answ. Osor. 246 b, I heare the sounde of an Argument from the Popish Diatriba.    
%\P 1643 R. BAILLIE  \textit{Lett. \& Jrnls. (1841) II. 65 Some parergetick Diatribes of that matter.    
%\P 1672 MEDE'S  \textit{Wks. Gen. Pref. A, That excellent Diatriba upon S. Mark i. 15.    
%\P 1683  \textit{Lond. Gaz. No. 1820/4 The constant Communicant; a Diatribe, proving that Constancy in receiving the Lords Supper is the indispensable Duty of every Christian.    
%\P 1703 J. QUICK  \textit{Dec. Wife's Sister Lett., Possibly this poor Diatribe may contribute something thereunto.    
%\P 1816 KIRBY \& SP.  \textit{Entomol. (1828) II. xxiv. 397, I shall conclude this diatribe upon the noises of insects.    
%\P 1875 LOWELL  \textit{Spenser Prose Wks.
%\P 1890 IV. 273  \textit{A diatribe on the subject of descriptive poetry.

%\itembf{2.} In modern use: A dissertation or discourse directed against some person or work; a bitter and violent criticism; an invective.

%\P 1804 SCOTT LET. ELLIS in  \textit{Lockhart Life xiii, One must always regret so very serious a consequence of a diatribe.    
%\P 1830 CUNNINGHAM  \textit{Brit. Paint. II. 132 On the appearance of this bitter diatribe in 1797.    
%\P 1850 KINGSLEY  \textit{Alt. Locke xxviii, A rambling, bitter diatribe on the wrongs and sufferings of the labourers.    
%\P 1854 THACKERAY  \textit{Newcomes II. 293 Breaking out into fierce diatribes.    
%\P 1877 MORLEY  \textit{Carlyle Crit. Misc. Ser. i. (1878) 201 The famous diatribe against Jesuitism in the Latter-Day Pamphlets.

%Hence ˈdiatribe v. intr., to utter a diatribe; to inveigh bitterly.

%\P 1893 NATIONAL  \textit{Observer 6 May 630/1 Why diatribe against the tradesmen of Liskeard?



%\end{myenumerate}


%%%%%%%%%%%%%%%%%%%%%%%%%%%%%%%%%
%\myitem{dichotomy} n.

%\noindent \phonetic{(daɪˈkɒtəmɪ)}

%\noindent [ad. Gr. διχοτοµία a cutting in two, f. διχότοµ-ος (see dichotomous): cf. F. dichotomie (1754 in Hatz.-Darm.).]
%\vspace{-0.3cm}

%\begin{myenumerate}

%\itembf{1.} Division of a whole into two parts. \itembf{a.} spec. in Logic, etc.: Division of a class or genus into two lower mutually exclusive classes or genera; binary classification.

%\P 1610 HEALEY  \textit{St. Aug. Citie of God 303 This Trichotomy..doth not contradict the other Dichotomy that includeth all in action and contemplation.    
%\P 1725 WATTS  \textit{Logic} i. vi. §8 Some.. have disturbed the Order of Nature..by an Affectation of Dichotomies, Trichotomies, Sevens, Twelves, \&c. Let the Nature of the Subject, considered together with the Design which you have in view, always determine the Number of Parts into which you divide it.    
%\P 1864 BOWEN  \textit{Logic iv. 97 Convenience often requires what Logicians call division by dichotomy, in which a Genus is divided into two Species having Contradictory Marks.    
%\P 1877 E. CAIRD  \textit{Philos. Kant ii. vi. 302 The whole sphere of reality may be divided in relation to any predicate..in what is called dichotomy by contradiction, e.g. that ‘everything must either be red or not red’.

%\itembf{b.} gen. Division into two. Something divided into two or resulting from such a division; something paradoxical or ambivalent.

%\P 1636 FEATLY  \textit{Clavis Myst. xxi. 277 Whose day after a ramisticall dichotomy being divided into forenoone and afternoone.    
%\P 1668 WILKINS  \textit{Real Char. ii. vii. §3. 190 The way of Dichotomy or Bipartition being the most natural and easie kind of Division.    
%\P 1868  \textit{Contemp. Rev.} Apr. 598 Popular theology is rather founded on the dichotomy of man into body and soul, than on the Christian trichotomy of body, soul, and spirit.    
%\P 1942  \textit{N.Y. Times} 28 Nov. 12/4 ‘An absolute dichotomy between science and reason on the one hand and faith and poetry on the other.’ In other and simpler days people spoke of the conflict between science and religion, or the clash between the two, or the wide gulf between the two.    
%\P 1950  \textit{Listener} 28 Sept. 430/1 The questions of thinking in one language and writing in another, the doubtful dichotomy of east and west.    
%\P 1957 ‘J.  \textit{Wyndham’ Midwich Cuckoos xiii. 105 By a dichotomy familiar to us all, a woman requires her own baby to be perfectly normal, and at the same time superior to all other babies.    
%\P 1966  \textit{Listener} 3 Mar. 323/2 Their uncritical use of the ‘Communist’ versus ‘free world’ dichotomy.

%\itembf{2.} Astron. That phase of the moon (or of an inferior planet) at which exactly half the disk appears illuminated; the ‘half-moon’.

%\P 1686 GOAD  \textit{Celest. Bodies i. xv. 81 This Quadrate or Quartile in its Dichotomy, as the Greeks call it.    
%\P 1797  \textit{Encycl. Brit.} II. 419/1 Aristarchus..gave a method of determining the distance of the sun by the moon's dichotomy.    
%\P 1878 NEWCOMB  \textit{Pop. Astron. 551 Dichotomy, the aspect of a planet when half illuminated.

%\itembf{3.} Bot., Zool., etc. A form of branching in which each successive axis divides into two; repeated bifurcation: see dichotomous 2.

%\P 1707 SLOANE  \textit{Jamaica I. 264 From the middle of the leaves rise one or two stalks..always divided into two, or observing a Dichotomy.    
%\P 1835 KIRBY  \textit{Hab. \& Inst. Anim. II. xiii. 13 The last [Encrinus] seems to differ..in the dichotomies and length of the arms.    
%\P 1880 GRAY  \textit{Struct. Bot. iii. §3. 47 note, Dichotomy or forking, the division of an apex into two.    
%\P 1882 VINES  \textit{Sachs' Bot. 169 Dichotomy..never produces structures..dissimilar to the producing structure; the divisions of a root produced by dichotomy are both roots, those of a leaf-bearing shoot both leaf-bearing shoots..dichotomy hence always falls under the conception of branching in the..narrower sense.    Ibid. 464.



%\end{myenumerate}


%%%%%%%%%%%%%%%%%%%%%%%%%%%%%%%%%
%\myitem{didactic} a. and n.

%\noindent \phonetic{(dɪˈdæktɪk)}

%\noindent [mod. ad. Gr. διδακτικ-ός apt at teaching, f. διδάσκειν to teach. Cf. F. didactique (1554 in Hatz.-Darm.)]
%\vspace{-0.3cm}

%\begin{myenumerate}

%\itembf{A.} adj. Having the character or manner of a teacher or instructor; characterized by giving instruction; having the giving of instruction as its aim or object; instructive, preceptive.

%\P 1658 R. FRANCK  \textit{North. Mem. (1821) 54 Must I be didactick to initiate this art?    
%\P 1661 WORTHINGTON  \textit{To Hartlib xvi. (T.), Finding in himself a great promptness in such didactic work.    
%\P 1756 J. WARTON  \textit{Ess. Pope (1782) I. iii. 101 A poem of that species, for which our author's genius was particularly turned, the didactic and the moral.    
%\P 1824 DIBDIN  \textit{Libr. Comp. 682 The dullest of all possible didactic and moral poetry.    
%\P 1830 MACKINTOSH  \textit{Eth. Philos. Wks.
%\P 1846 I. 59 A  \textit{permanent foundation of his [Hobbes'] fame remains in his admirable style, which seems to be the very perfection of didactic language.    
%\P 1878 R. B. SMITH  \textit{Carthage 130 Polybius..is too didactic—seldom adorning a tale but always ready to point a moral.    
%\P 1878 R. W. DALE  \textit{Lect. Preach. viii. (ed. 2) 226, I do not mean that sermons addressed to Christian people should be simply didactic.

%\P 1754 A. MURPHY  \textit{Gray's-Inn Jrnl. No. 90 ⁋6 Both [Eloquence and Poetry]..have occasionally strengthened themselves with Insertions of the Didactic.

%\itembf{B.} n.

%\itembf{1.} A didactic author or treatise. Obs.

%\P 1644 MILTON  \textit{Educ. Wks. (1847) 98/2 To search what many modern Januas and Didactics..have projected, my inclination leads me not.    
%\P 1835 SOUTHEY  \textit{Doctor III. 162 Acknowledged in the oldest didactics upon this subject.

%\itembf{2.} pl. didactics [see -ics]: The science or art of teaching.

%\P 1846 WORCESTER  \textit{cites Biblical Repos.    
%\P 1856 MRS. BROWNING  \textit{Aur. Leigh i. Poems
%\P 1890 VI. 38 DIDACTICS,  \textit{driven Against the heels of what the master said.    
%\P 1860 EMERSON  \textit{Cond. Life, Consid. Wks. (Bohn) II. 412 Life is rather a subject of wonder, than of didactics.    
%\P 1881 J. G. FITCH  \textit{Lect. Teach. ii. 36 The art of teaching, or Didactics as we may for convenience call it, falls under two heads.



%\end{myenumerate}


%%%%%%%%%%%%%%%%%%%%%%%%%%%%%%%%%
%\myitem{diffident} a.

%\noindent \phonetic{(ˈdɪfɪdənt)}

%\noindent [ad. L. diffīdent-em, pr. pple. of diffīdĕre to mistrust; see diffide, and -ent. (The opposite of confident.)]
%\vspace{-0.3cm}

%\begin{myenumerate}

%\itembf{1.} Wanting confidence or trust (in); distrustful, mistrustful (of).

%\P 1598 FLORIO,  \textit{Diffidénte, mistrustful, diffident.    
%\P a1618 RALEIGH  \textit{Mohomet (1637) 207 In the constancie of his people he was somewhat diffident.    
%\P a1631 DONNE  \textit{Serm.} xii. 114 A fainting and a diffident Spirit.    
%\P 1667 MILTON  \textit{P.L.} viii. 562 Be not diffident Of Wisdom, she deserts thee not, if thou Dismiss her not, when most thou needst her nigh.    
%\P 1691 RAY  \textit{Creation i. (1704) 159, I am somewhat diffident of the truth of those Stories.    
%\P 1734 WATTS  \textit{Reliq. Juv. (1789) 131 A feeble man and diffident had need to pray daily, Lord, lead us not into temptation.    
%\P 1802 H. MARTIN  \textit{Helen of Glenross III. 330 Had I been more diffident in its effects, I had not trusted..to it.    
%\P 1873 SYMONDS  \textit{Grk. Poets v. 141 The English are not musicians, and are diffident in general of the artist class.

%\itembf{2.} Wanting in self-confidence; distrustful of oneself; not confident in disposition; timid, shy, modest, bashful. (The usual current sense.)

%\P 1648 EIKON  \textit{Bas. xi. (1824) 88, I am not so diffident of My selfe, as brutishly to submit to any men's dictates.]    
%\P 1713 ADDISON  \textit{Cato ii. i, Let us appear nor rash nor diffident.    
%\P 1785 F. BURNEY  \textit{Lett. 3 Jan., He [Dr. Johnson] never attacked the unassuming, nor meant to terrify the diffident.    
%\P 1835 W. IRVING  \textit{Newstead Abbey Crayon Misc. (1863) 362 She was shy and diffident.    
%\P 1882 B. M. CROKER  \textit{Proper Pride I. ii. 42 She little knew that the apparently diffident young man was the life and soul of his mess.



%\end{myenumerate}


%%%%%%%%%%%%%%%%%%%%%%%%%%%%%%%%%
%\myitem{digress} v.

%\noindent \phonetic{(dɪˈgrɛs, daɪ-)}

%\noindent [f. L. dīgress- ppl. stem of dīgredī to go aside, depart, f. di-, dis- 1 + gradī to step, walk, go.]
%\vspace{-0.3cm}

%\begin{myenumerate}

%\itembf{1.} intr. To go aside or depart from the course or track; to diverge, deviate, swerve.

%\P 1552 HULOET,  \textit{Digresse or go a little out of the pathe, digredior.    
%\P 1582 N. LICHEFIELD tr.  \textit{Castanheda Conq. E. Ind. 65 b, It was not vnpossible but that they might somewhat digresse from their right course.    
%\P 1603 DEKKER  \textit{Grissil (Shaks. Soc.) 22, I must disgress from this bias, and leave you.    
%\P 1649 ALCORAN  \textit{86 God..punisheth them that digresse from the right path.    
%\P 1750 JOHNSON  \textit{Rambler} No. 25 ⁋11 Frighted from digressing into new tracts of learning.    
%\P 1825 LAMB  \textit{Elia Ser. ii. Superannuated man, I find myself in Bond Street..I digress into Soho, to explore a bookstall.

%\itembf{b.} Astron. Cf. digression 3. Obs.

%\P 1601 HOLLAND  \textit{Pliny I. 12 Shee (Venus) beginnes to digresse in latitude and to diminish her motion from the morn rising: but to be retrograde, and withall to digresse in altitude from the euening station.

%\itembf{2.} fig. To depart or deviate (from a course, mode of action, rule, standard, etc.); to diverge. Obs.

%\P 1571 GOLDING  \textit{Calvin on Ps. lxxi. 16 As the other translation agreeth very well, I would not digresse from it.    
%\P 1592 SHAKES.  \textit{Rom. \& Jul.} iii. iii. 127 Thy Noble shape, is but a forme of waxe, Digressing from the Valour of a man.    
%\P 1603 HOLLAND  \textit{Plutarch's Mor. 25 Digresse good sir from such lewd songs.    
%\P 1611 USSHER in  \textit{Gutch Coll. Cur. I. 39 The subjects rebelled, and digressed from their allegiance.

%\itembf{3.} To diverge from the right path, to transgress. Obs.

%\P 1541-93 [see DIGRESSING below].    
%\P 1640 G. WATTS tr.  \textit{Bacon's Adv. Learn. vii. iii. (R.), So man, while he aspired to be like God in knowledge, digressed and fell.

%\itembf{b.} trans. To transgress. Obs.

%\P 1592 W. WYRLEY  \textit{Armorie 56 Faire points of honor I would not disgresse.

%\itembf{4.} intr. To deviate from the subject in discourse or writing. (Now the most frequent sense.)

%\P 1530 PALSGR. 516/1, I dygresse from my mater and talke of a thyng that nothynge belongeth therunto.    
%\P 1555 EDEN  \textit{Decades} 8 To returne to the matter from which we haue digressed.    
%\P 1597 MORLEY  \textit{Introd. Mus. 74 Let vs come againe to our example from which wee haue much disgressed.    
%\P 1682 BURNET  \textit{Rights Princes viii. 292, I shall not digress to give any account of these.    
%\P 1727 SWIFT  \textit{Modest Proposal, I have too long digressed, and therefore shall return to my subject.    
%\P 1752 JOHNSON  \textit{Rambler} No. 200 ⁋10 While we were conversing upon such subjects..he frequently digressed into directions to the servant.    
%\P 1813 W. TAYLOR in  \textit{Ann. Rev. I. 374 Mr. P. digresses on the subject of parliamentary reform.    
%\P 1869 FARRAR  \textit{Fam. Speech iii. (1873) 99, I will not here digress into the interesting question as to the origin of writing.

%Hence diˈgressing vbl. n. and ppl. a., diˈgressingly adv.

%\P 1529 MORE  \textit{Comf. agst. Trib. ii. Wks. 1200/1 Were it properly perteining to ye present matter, or sumwhat disgressing therfro.    
%\P 1541  \textit{Act 33 Hen. VIII in Bolton Stat. Irel. (1621) 218 Albeit that upon any disloyaltie or disgressing contrary to the duety of a subject.    
%\P 1593 SHAKES.  \textit{Rich. II,} v. iii. 66 This deadly blot, in thy digressing sonne.    
%\P 1864  \textit{Q. Rev.} CXVI. 168 The sarcophagus on which appears the incident we have thus digressingly analysed.



%\end{myenumerate}


%%%%%%%%%%%%%%%%%%%%%%%%%%%%%%%%%
%\myitem{dilettante} n.

%\noindent \phonetic{(dɪlɪˈtæntɪ, It. diletˈtante)}

%\noindent [It. dilettante ‘a lover of music or painting’, f. dilettare:—L. dēlectāre to delight: see delect, etc. So mod.F. dilettante, 1878 in Dict. Acad.]
%\vspace{-0.3cm}

%\begin{myenumerate}

%\itembf{1.} A lover of the fine arts; originally, one who cultivates them for the love of them rather than professionally, and so = amateur as opposed to professional; but in later use generally applied more or less depreciatively to one who interests himself in an art or science merely as a pastime and without serious aim or study (‘a mere dilettante’).

%\P 1733-4 [‘THE  \textit{Society of Dilettanti’ was founded].    
%\P 1748 CHESTERFIELD  \textit{Lett. ii. xl, You are likely to hear of it as a virtuoso; and if so, I should be glad to profit of it, as an humble dillettante.    
%\P 1769 (TITLE),  \textit{Ionian Antiquities, By the Society of Dilettanti.    
%\P 1770 FOOTE  \textit{Lame Lover i. i, Frederick is a bit of Macaroni and adores the soft Italian termination in a... Yes, a delitanti all over.    
%\P 1775 F. BURNEY  \textit{Diary 21 Nov., A female dilettante of great fame and reputation..as a singer.    
%\P 1789 BURNEY  \textit{Hist. Mus. III. ii. 161 Personages whose [musical] talents are celebrated whether they are regarded as professors or Diletanti.    
%\P 1801 W. TAYLOR in  \textit{Monthly Mag. XII. 576 Religious dilettanti, of every sex and age, reinforce the industry of the regular priesthood.    
%\P 1802  \textit{Edin. Rev.} I. 165 Dilettanti who have pushed themselves into high places in the scientific world.    
%\P 1826 BARONESS BUNSEN in  \textit{Hare Life II. vii. 265 It would be difficult to find a dilettante who understood the art of managing it [a parlour organ].    
%\P 1831 CARLYLE  \textit{Sart. Res.} i. x, Thou hitherto art a Dilettante and sandblind Pedant.    
%\P 1840 MACAULAY  \textit{Ess., Clive (1854) 534/2 The Dilettante sneered at their want of taste. The Maccaroni black$\sim$balled them as vulgar fellows.    
%\P 1879 FROUDE  \textit{Cæsar ii. 17 [The Romans] cared for art as dilettanti; but no schools either of sculpture or painting were formed among themselves.    
%\P 1886 RUSKIN  \textit{Præterita I. 271 Rogers was a mere dilettante, who felt no difference between landing where Tell leaped ashore, or standing where ‘St. Preux has stood’.

%\itembf{b.} with of: a lover, one who is fond of. Obs.

%\P 1783 HAMILTON in  \textit{Phil. Trans. LXXIII. 189 Those who are professed dilettanti of miracles.

%\itembf{2.} attrib. \itembf{a.} In apposition, as dilettante musician, etc. = amateur.

%\P 1774 ‘J.  \textit{Collier’ Mus. Trav. (1775) 4 That great Dilettante performer on the harp.    
%\P 1789 F. BURNEY  \textit{Lett. 27 Oct., A Dilettante purchaser may yet be found.    
%\P 1806-7 J. BERESFORD  \textit{Miseries Hum. Life (1826) xv. iii, You are almost entirely reduced to Dilletanti Musicians.    
%\P 1816 T. L. PEACOCK  \textit{Headlong Hall iii, Sir Patrick O'Prism, a dilettante painter of high renown.    
%\P 1821 CRAIG  \textit{Lect. Drawing v. 252 Suited for the dilettante artist.    
%\P 1871 MORLEY  \textit{Voltaire (1886) 57 The dilettante believer is indeed not a strong spirit, but the weakest.

%\itembf{b.} Of, pertaining to, or characteristic of a dilettante (in the shades of meaning the word has passed through).

%\P 1753 SMOLLETT  \textit{Ct. Fathom xxxii, He sometimes held forth upon painting, like a member of the Dilettanti club.    
%\P 1774 ‘J.  \textit{Collier’ Mus. Trav. (1775) 58 He ordered his servant to bring in his Dilettante ring and wig.    
%\P 1794 MATHIAS  \textit{Purs. Lit. (1798) 386 The dilettante spirit which too frequently prevails in Dr. Warton's comments.    
%\P 1840 CARLYLE  \textit{Heroes} vi. (1891) 198 To us it is no dilettante work, no sleek officiality; it is sheer rough death and earnest.    
%\P a1847 MRS. SHERWOOD  \textit{Lady of Manor II. xiii. 151, I will have a dilletante play, or concert, or some such thing, got up.    
%\P 1868 M. PATTISON  \textit{Academ. Org. v. 148 A dilettante fastidiousness, an aimless inertia.

%Hence dileˈttante v., dileˈttantize v., to play the dilettante (also to dilettante it); dileˈttanting ppl. a.; dileˈttantedom, the world of dilettanti; dileˈttanteship, the condition of a dilettante.

%\P 1835 JAMES  \textit{Gipsy v, In the elegant charlatanism of dilettanteship.    
%\P 1837  \textit{Blackw. Mag.} XLII. 515 To go on dilettanteing it in the grossness of the moral atmosphere of the Continental cities.    
%\P 1843  \textit{Tait's Mag.} X. 346 Shooting partridges and dilettantizing at legislation.    
%\P 1887  \textit{Pall Mall G. 1 Jan. 5/2 The favourite actress of dilettantedom.    
%\P 1890  \textit{Spectator} 11 Oct. 495 The Shakespeare temptation remains as strong as ever with the dilettanting world.



%\end{myenumerate}


%%%%%%%%%%%%%%%%%%%%%%%%%%%%%%%%%
%\myitem{diminution} n.

%\noindent \phonetic{(dɪmɪˈnjuːʃən)}

%\noindent [a. AF. diminuciun (a 1300), F. diminution = Pr. diminutio, Sp. diminucion, Pg. diminuição, It. diminuzione, ad. L. dīminūtiōn-em later spelling of dēminūtiōn-em, n. of action from dēminuĕre to lessen. Classical L. analogies would give the form deminution: see diminish, diminue.]
%\vspace{-0.3cm}

%\begin{myenumerate}

%1. \itembf{a.} The action of diminishing or making less; the process of diminishing or becoming less; reduction in magnitude or degree; lessening, decrease.

%\P c1374 CHAUCER Troylus iii.
%\P 1286 (1335)  \textit{To encrece or maken dyminucioun Of my langage.    
%\P 1495  \textit{Act 11 Hen. VII, c. 2 §6 Dymynucion of punysshment..shalbe had for women greate with child.    
%\P 1594 HOOKER  \textit{Eccl. Pol.} iii. xi. (1611) 120 Change by addition or diminution.    
%\P 1617 MORYSON  \textit{Itin. ii. iii. i. 213 The remainder can hardly beare such deminution, as all Armies are subiect vnto.    
%\P 1682 BURNET  \textit{Rights Princes viii. 315 Rather than consent to the least diminution of that Right.    
%\P 1691 T. H[ALE]  \textit{Acc. New Invent. p. cvii, Enlargements or Diminutions of Wharfs or Banks.    
%\P 1712 ADDISON  \textit{Spect.} No. 517 ⁋1 A copy of his letter, without any alteration or diminution.    
%\P 1857 WHEWELL  \textit{Hist. Induct. Sc. II. 175 The Diminution of the Obliquity of the Ecliptic.

%\itembf{b.} Apparent lessening, as by distance. ? Obs.

%\P 1611 SHAKES.  \textit{Cymb. i. iii. 18 To looke vpon him, till the diminution Of space, had pointed him sharpe as my Needle.    
%\P 1667 MILTON  \textit{P.L.} vii. 369 From human sight So farr remote, with diminution seen.

%\itembf{2.} \itembf{a.} Representation of something as less than it is; extenuation. \itembf{b.} as a Rhet. figure. Obs.

%\P 1303 R. BRUNNE  \textit{Handl. Synne 12416 Ȝyt þer ys an enchesun Ys kallede ‘dymynucyun’, On englys hyt ys to mene To make þy synne lytyl to seme.    
%\P 1586 A. DAY  \textit{Eng. Secretary ii. (1625) 93 Example.. for diminution, might be this..these I must confesse are injuries to some, but unto me they are trifles.    
%\P 1659 O. WALKER  \textit{Oratory 75 Gradation is by Oratours most-what observed, and the weightiest word said last: or, in diminutions, the contrary.

%\itembf{3.} Lessening of honour or reputation; derogation, depreciation, belittling. Obs.

%\P 1586 A. DAY  \textit{Eng. Secretary i. (1625) 9 What approbations, diminutions, insinuations.    
%\P 1599  \textit{Life Sir T. More in Wordsw. Eccl. Biog. (1853) II. 181 Under pardon of those saints..for I intend not the diminution of their glorious deaths.    
%\P 1646 FULLER  \textit{Wounded Consc. (1841) 351 A diminution to the majesty of God.    
%\P 1648 EIKON  \textit{Bas. 49, I shall not much regard the worlds opinion or diminution of me.    
%\P 1712 STEELE  \textit{Spect. No. 468 ⁋4 Thinking nothing a Diminution to me, but what argues a Depravity of my Will.    
%\P a1734 NORTH  \textit{Lives} (1826) II. 176 All that appeared..of diminution to the reputation..which his Lordship..had acquired.

%\itembf{4.} Partial deprivation, curtailment, abatement.

%\P 1548 HALL  \textit{Chron., Hen. V, 70 b, That we suffre harme or diminicion in person, estate, worship, or goodes.    
%\P 1661 BRAMHALL  \textit{Just Vind. iv. 78 Untill it came to sentence of death, or diminution of member.    
%\P 1675 BAXTER  \textit{Cath. Theol. ii. i. 20 Had this been any injury or diminution to the rest?

%\itembf{5.} Mus. \itembf{a.} The repetition of a subject (in contrapuntal writing) in notes of half or a quarter the length of the original: opp. to augmentation. \itembf{b.} (quot. 1614) The condition of being diminished (of an interval): see diminished 4 (obs. rare).

%\P 1597 MORLEY  \textit{Introd. Mus. 24 Diminution is a certaine lessening or decreasing of the essential value of the notes and rests.    
%\P 1609 DOULAND  \textit{Ornith. Microl. 48 Diminution..is the varying of Notes of the first quantity..or it is a certain cutting off of the measure.    
%\P 1614 T. RAVENSCROFT  \textit{(title), A briefe Discourse of the true but neglected Vse of characterizing the Degrees by their perfection, imperfection and diminution, in measurable Musicke.    
%\P 1869 OUSELEY  \textit{Counterp. xv. 104 [In] imitation by diminution..the consequent substitutes notes of smaller value for those proposed by the antecedent.

%\itembf{6.} Her. With earlier authors: The defacing of part of an escutcheon. By later writers said to be = difference.

%\P 1610 J. GUILLIM  \textit{Heraldry i. viii. (1660) 43 Diminution is a blemishing or defacing of some particular point..of the Escocheon, by reason of the imposition of some stain and colour thereupon.    
%\P 1787 PORNY  \textit{Her. Gloss., Diminution, word sometimes used instead of Difference.    
%\P 1830 ROBSON  \textit{Brit. Herald III. Gloss., Diminution of Arms, an expression sometimes used..instead of differences, or, as the French call them, brisures..from the Latin diminutiones, lessenings, as showing a family to be less than the chief.

%\itembf{7.} Gram. The formation of a diminutive word from a primitive. Obs. rare.

%\P a1637 B. JONSON  \textit{Eng. Gram. xi, The common affection of nouns is diminution..The diminution of substantives hath these four divers terminations: El..Et..Ock..Ing..Diminution of adjectives is in this one end, ish.

%\itembf{8.} Law. An omission in the record of a case sent up by an inferior court to a superior, in proceedings for reversal of judgement.

%\P 1610 COKE  \textit{Bk. of Entries 242 a/2 (marg.) Le def. alledge diminution en le Here. fac. seisinam.    Ibid. 251 b/1 (marg.) Diminution alledge per le def. en les proclamations.    
%\P 1626 SIR W. JONES  \textit{Reports, Weever v. Fulton 2 Car. 1 (1675) 140 Car apres in nullo est Erratum plede, neque le Plaintiff neque le Defendant poient alledge diminution, car per le joinder ils allowe recorde.]    
%\P 1657 GRIMSTON tr.  \textit{Croke's Repts. (1683) ii. 597, Johns v. Bowen, 18 Jas. I, After the Record certified, the plaintiff in the Writ of Error alledges Diminution for want of an Original, which was certified and entered.    
%\P 1708 TERMES  \textit{de la Ley 248, Diminution, is when the Plaintiff or Defendant in a Writ of Error alledges..that part of the Record remains in the Inferiour Court not certifyed, and prays that it be certifyed by Certiorari.    
%\P 1848 in  \textit{Wharton Law Lex.

%\itembf{9.} Arch. The gradual decrease in diameter of the shaft of a column, etc.; the tapering of a column or other part of a building; also, the amount of this tapering in the whole length.

%\P 1706 PHILLIPS  \textit{(ed. Kersey), Diminution..in Architecture, the lessening of a Pillar by little and little from the Base to the Top.    
%\P 1726 LEONI  \textit{Alberti's Archit. II. 20/1 The diameter of the lower diminution.    
%\P 1727-51 CHAMBERS  \textit{Cycl.} s.v., The Gothic architects..observe neither diminution nor swelling; their columns are perfectly cylindrical.    
%\P 1766 ENTICK  \textit{London IV. 356 [The] turret..ends with a fine diminution.    
%\P 1842-76 GWILT  \textit{Archit. iii. i. 809 The diminution or tapering form given to a column..sometimes commences from the foot of the shaft, sometimes from a quarter or one third of its height.    Ibid. 814 Vitruvius in this order [the Tuscan] forms the columns six diameters high, and makes their diminution one quarter of the diameter.

%\itembf{10.} Cytology. [a. F. diminution (V. Herla 1895, in Arch. de Biol. XIII. 485).] The loss or expulsion, during the embryogenesis of certain organisms, of some chromosome material from the nuclei of cells that go to form somatic tissue.

%\P 1925 E. B. WILSON  \textit{Cell (ed. 3) iv. 326 In these cases..the process of diminution is somehow connected with the segregation of germ-cells from somatic cells.    
%\P 1942  \textit{Nature} 17 Jan. 67/2 In Sorghum we know that the chromosomes which undergo ‘diminution’ are in fact dispensable not only in parts of the plant but also in parts of the species.    
%\P 1965 C. D. DARLINGTON  \textit{Cytology ii. iii. 658 Coordinated reactions of centromeres, heterochromatin and cytoplasm are no doubt responsible for diminution.



%\end{myenumerate}


%%%%%%%%%%%%%%%%%%%%%%%%%%%%%%%%%
%\myitem{disaffection} n.

%\noindent \phonetic{(dɪsəˈfɛkʃən)}

%\noindent [f. dis- 9 + affection; or n. of action f. disaffect v.1 and v.2, after affection.]
%\vspace{-0.3cm}

%\begin{myenumerate}

%\itembf{1.} Absence or alienation of affection or kindly feeling; dislike, hostility: see affection 6.

%\P 1640 SANDERSON  \textit{Serm. II. 145 Chastening is..far from being any argument of the father's dis-affection.    
%\P 1643 MILTON  \textit{Divorce} ii. vii. (1851) 78 Not to root up our naturall affections and disaffections.    
%\P 1655 FULLER  \textit{Ch. Hist.} x. iii. §6 His disaffection to the discipline established in England.    
%\P 1706-7 FARQUHAR  \textit{Beaux Strat. iii. iii, What Evidence can prove the unaccountable Disaffections of Wedlock?    
%\P 1879 STEVENSON  \textit{Trav. Cevennes 87 Modestine..seemed to have a disaffection for monasteries.

%\itembf{2.} spec. Political alienation or discontent; a spirit of disloyalty to the government or existing authority: see disaffected 1.

%\P 1605 B. JONSON  \textit{Volpone ii. i, Nor any dis-affection to the state Where I was bred.    
%\P 1683  \textit{Brit. Spec. 218 To take away all Occasions of Disaffection to the Anointed of the Lord.    
%\P 1697 W. DAMPIER  \textit{Voy. I. 371 The whole Crew were at this time under a general disaffection, and full of very different Projects.    
%\P 1751 JOHNSON  \textit{Rambler} No. 204 ⁋2 Thou hast reconciled disaffection, thou hast suppressed rebellion.    
%\P 1808 SYD. SMITH  \textit{Wks. (1867) I. 115 A very probable cause of disaffection in the troops.    
%\P 1874 GREEN  \textit{Short Hist. 556 The popular disaffection told even on the Council of State.

%\itembf{3.} The condition of being evilly affected physically; physical disorder or indisposition. Obs.

%\P 1654 GAYTON  \textit{Pleas. Notes iii. xi. 144 Forc'd to fly to Physick, for cure of the disaffection.    
%\P 1676 WISEMAN  \textit{(J.), The disease took its original merely from the disaffection of the part, and not from the peccancy of the humours.    
%\P 1688 BOYLE  \textit{Final Causes Nat. Things, Vitiated Sight 260 This woman..had a disaffection of sight very uncommon.    
%\P 1741  \textit{Compl. Fam.-Piece i. i. 78 If the Patient be subject to..any Swelling, Heat, or Disaffection in the Eyelids.



%\end{myenumerate}


%%%%%%%%%%%%%%%%%%%%%%%%%%%%%%%%%
%\myitem{disconsolate} a. (n.)

%\noindent \phonetic{(dɪsˈkɒnsələt)}

%\noindent [a. med.L. disconsōlāt-us comfortless (Du Cange), f. dis-, dis- 4 + L. consōlātus: see consolate ppl. a. Cf. 16th c. F. desconsolé, It. sconsolato, Sp. desconsolado.]
%\vspace{-0.3cm}

%\begin{myenumerate}

%\itembf{1.} Destitute of consolation or comfort; unhappy, comfortless; inconsolable, forlorn.

%\P 1429  \textit{Pol. Poems (Rolls) II. 145 Rewe on the poore and folk desconsolate.    
%\P 1494 FABYAN Chron. v. cxl. 127 Thou mother to wretchis and other disconsolate.    
%\P 1594 SPENSER  \textit{Amoretti lxxxviii, So I alone, now left disconsolate, Mourne to my selfe the absence of my love.    
%\P 1663 PEPYS  \textit{Diary} 19 Oct., The King..is most fondly disconsolate for her, and weeps by her.    
%\P a1704 T. BROWN  \textit{Two Oxf. Scholars Wks.
%\P 1730 I. 7  \textit{A poor disconsolate widow.    
%\P 1709 STEELE  \textit{Tatler} No. 23 ⁋2 The Disconsolate soon pitched upon a very agreeable Successor.    
%\P 1863 LONGFELLOW  \textit{Wayside Inn i. Falc. Ser Fed. xix, She..passed out at the gate With footstep slow and soul disconsolate.    
%\P 1864 TENNYSON  \textit{En. Ard. 678 On the nigh-naked tree the robin piped Disconsolate.

%\itembf{2.} Of places or things: Causing or manifesting discomfort; dismal, cheerless, gloomy.

%\P c1374 CHAUCER  \textit{Troylus v. 542 O paleys desolat!.. O paleys empti and disconsolat!    
%\P 1655-62 W. GURNALL CHR. in  \textit{Arm. (1669) 256/2 When the Christians affairs are most disconsolate, he may soon meet with a happy change.    
%\P 1691 RAY  \textit{Creation (1714) 66 The disconsolate Darkness of our Winter Nights.    
%\P 1720 DE FOE  \textit{Capt. Singleton ix. (1840) 156 It was..a desolate, disconsolate wilderness.    
%\P 1855 MACAULAY  \textit{Hist. Eng.} III. 665 The island..to French courtiers was a disconsolate place of banishment.

%\itembf{B.} as n. A disconsolate person.

%\P 1781 S. J. PRATT  \textit{Emma Corbett III. 14 Raymond, our poor disconsolate, the mutual joy of our hearts.



%\end{myenumerate}


%%%%%%%%%%%%%%%%%%%%%%%%%%%%%%%%%
%\myitem{discursive} a. (n.)

%\noindent \phonetic{(dɪˈskɜːsɪv)}

%\noindent [f. L. discurs- ppl. stem of discurrĕre (see discursion) + -ive.]
%\vspace{-0.3cm}

%\begin{myenumerate}

%\itembf{1.} Running hither and thither; passing irregularly from one locality to another. rare in lit. sense.

%\P 1626 BACON  \textit{Sylva} §745 Whatsoeuer moueth Attention.. stilleth the Naturall and discursiue Motion of the Spirits.    
%\P 1834 WEST  \textit{Ind. Sketch Bk. II. 240 Misgivings, that Our road..might prove somewhat more discursive.    Ibid. 282 The regularity of the streets..prevented the breezes being so discursive as..among the unconnected dwellings.

%\itembf{2.} fig. Passing rapidly or irregularly from one subject to another; rambling, digressive; extending over or dealing with a wide range of subjects.

%\P 1599 MARSTON  \textit{Sco. Villanie iii. xi. 231 Boundlesse discursiue apprehension Giving it wings.    
%\P 1665 HOOKE  \textit{Microgr. Pref. G., Men are generally rather taken with the plausible and discursive, then the real and the solid part of Philosophy.    
%\P 1791 BOSWELL Johnson an.
%\P 1774 (1816)  \textit{II. 296 Such a discursive Exercise of his mind.    
%\P 1827 CARLYLE  \textit{Richter Misc. Ess.
%\P 1872 I. 8  \textit{The name Novelist..would ill describe so vast and discursive a genius.    
%\P 1850 TENNYSON  \textit{In Mem. cix, Heart-affluence in discursive talk From household fountains never dry.    
%\P 1867 FREEMAN  \textit{Norm. Conq. (1876) I. iv. 149 A most vivid, though very discursive and garrulous, history of the time.

%\itembf{3.} Passing from premisses to conclusions; proceeding by reasoning or argument; ratiocinative. (Cf. discourse v. 2.) Often opp. to intuitive.

%\P 1608 D. T. ESS.  \textit{Pol. \& Mor. 117 Ignorance..depriveth Reason of her discursive facultie.    
%\P a1652 J. SMITH  \textit{Sel. Disc. v. 137 We cannot attain to science but by a discursive deduction of one thing from another.    
%\P 1667 MILTON  \textit{P.L.} v. 488 Whence the soule Reason receives, and reason is her being, Discursive, or Intuitive; discourse Is oftest yours, the latter most is ours.    
%\P 1817 COLERIDGE  \textit{Biog. Lit. I. x. 161 Philosophy has hitherto been discursive: while Geometry is always and essentially intuitive.    
%\P 1836-7 SIR W. HAMILTON  \textit{Metaph. (1877) II. xx. 14 The Elaborative or Discursive Faculty..has only one operation, it only compares.    
%\P 1874 L. STEPHEN HOURS in  \textit{Library (1892) II. i. 15 Johnson..is always a man of intuitions rather than of discursive intellect.

%\itembf{B.} as n. A subject of ‘discourse’ or reasoning (as distinguished from a subject of perception). Obs. rare.

%\P 1677 HALE  \textit{Prim. Orig. Man. iv. viii, 364 Sometimes..the very subjectum discursus is imperceptible to Sense..such are also the discursives of moral good and evil, just, unjust, which are no more perceptible to Sense than Colour is to the Ear.



%\end{myenumerate}


%%%%%%%%%%%%%%%%%%%%%%%%%%%%%%%%%
%\myitem{disdain} v.

%\noindent \phonetic{(dɪsˈdeɪn)}

%\noindent [ME., a. OF. desdeignier, -deigner (3rd s. pres. -deigne), in later F. dédaigner, = Pr. desdegnar, Cat. desdenyar, Sp. dedeñar, Pg. desdenhar, It. disdegnare (sdegnare); a Common Romanic vb. representing, with des- for L. dē- (see de- 6), L. dēdignāre (collateral form of dēdignārī) to reject as unworthy, disdain. f. de- 6 + dignāre, -ārī to think or treat as worthy; cf. deign.]
%\vspace{-0.3cm}

%\begin{myenumerate}

%\itembf{1.} trans. To think unworthy of oneself, or of one's notice; to regard or treat with contempt; to despise, scorn. \itembf{a.} with simple obj.

%\P c1386 CHAUCER  \textit{Clerk's T. 42 (Ellesm. MS.) Lat youre eres nat my voys desdeyne [other MSS. disdeyne].    
%\P 1483  \textit{Cath. Angl.} 93/1 To Desden (Deden A.), dedignari, detrahere, detractare; vbi. to disspise.

%\P c1386 [see α and β].    
%\P 1509 HAWES  \textit{Past. Pleas. xvi. lvii, I fere to sore I shal disdayned be.    
%\P 1573 G. HARVEY  \textit{Letter-bk. (Camden) 4 He laid against me..that I did disdain everi mans cumpani.    
%\P 1613 PURCHAS  \textit{Pilgrimage v. xvii. 459 Whose proud top would disdaine climing.    
%\P 1754 EDWARDS  \textit{Freed. Will iv. iv. 217 Some seem to disdain the Distinction that we make between natural and moral Necessity.    
%\P 1821 SHELLEY  \textit{Prometh. Unb. i. 52 If they disdained not such a prostrate slave.    
%\P 1858 LYTTON  \textit{What will he do? i. x, I disdain your sneer.

%\itembf{b.} with inf. or gerund. To think it beneath one, to scorn (to do or doing something).

%\P c1380  \textit{Sir Ferumb.} 2179 Ys herte was so gret, þat he dedeynede to clepe, ‘oundo’; bot ran to wiþ is fet.

%\P 1393 GOWER  \textit{Conf.} III. 227 If..a king..Desdaineth for to done hem grace.

%\P 1489 CAXTON  \textit{Faytes of A. i. xv. 43 They dysdayne to obeye to theyre capytayne.    
%\P a1533 LD. BERNERS  \textit{Huon xxiv. 70 They dysdayne to speke to me.    
%\P 1611 BIBLE  \textit{Transl. Pref. 11 Neither did we disdaine to reuise that which we had done.    
%\P 1769 GOLDSM.  \textit{Roman Hist. (1786) I. 397 This..was the title the Roman general disdained granting him.    
%\P 1786 W. THOMSON  \textit{Watson's Philip III (1839) 357 [They] disdained to follow this example of submission.    
%\P 1868 E. EDWARDS  \textit{Raleigh I. xx. 455 Grey..had disdained to beg his life.

%\itembf{c.} To think (a thing) unworthy of (something). (Cf. deign v. 2.)

%\P 1646 J. HALL  \textit{Horæ Vac. 23 Nature disdeigned it a Roome.

%d.1.d To think (anything) unworthy of.

%\P 1591 SPENSER  \textit{Ruins of Time Ded., God hath disdeigned the world of that most noble Spirit.

%\itembf{2.} To be indignant, angry, or offended at. Obs.

%\P 1494 FABYAN Chron. ii. xlviii. 32 The kynge disdeynynge this demeanure of Andragius, after dyuers monycions..gatheryd his knyghtes and made warre vpon Andragius.    
%\P 1632 LITHGOW  \textit{Trav. Prol. B, To shun Ingratitude, which I disdaine as Hell.    
%\P 1633 T. STAFFORD  \textit{Pac. Hib. vi. (1821) 84 His answer was much disdained.    
%\P 1695 LD. PRESTON  \textit{Boeth. iii. 106 Hence..we often so much disdain their being conferr'd upon undeserving Men.

%\itembf{b.} with subord. clause: To be indignant that.

%\P 1548 HALL  \textit{Chron., Rich. III, 45 The kyng of Scottes disdeignynge that the stronge castell of Dumbarre should remayne in thenglish mennes handes.    
%\P 1587 TURBERV.  \textit{Trag. T. (1837) 128 Who highly did disdaine That such..abuse his honour should distaine.    
%\P 1602 MARSTON  \textit{Ant. \& Mel. ii. Wks.
%\P 1856 I. 27,  \textit{I have nineteene mistresses alreadie, and I not much disdeigne that thou shold'st make up the ful score.    
%\P 1796 W. TAYLOR in  \textit{Monthly Mag. I. 14 Disdaining that the enemies of Christ should abound in wealth.

%\itembf{3.} intr. To be moved with indignation, be indignant, take offence. Const. at (rarely against, of, on). Obs.

%\P 1382 WYCLIF  \textit{Job xxxii. 3 But aȝen the thre frendis of hym he dedeynede, forthi that thei hadden not founde a resounable answere.     Matt. xxi. 15 The princis of prestis and scribis..dedeyneden, and seiden to hym, Heerist thou what these seyen?    
%\P a1400 RELIG.  \textit{Pieces fr. Thornton MS. 90 Þat deuyls lymme, dedeyned at þi dede.

%\P 14.. EPIPH. in  \textit{Tundale's Vis. 108 Of whos cumyng though thou dysdeyne Hyt may not pleynly help.    
%\P 1526 TINDALE  \textit{Matt.} xx. 24 They disdayned at the two brethren.     John vii. 23 Disdayne ye at me, because I made a man every whit whoale?    
%\P c1563 CAVENDISH  \textit{Ld. Seymour iv., in Wolsey, etc. (1825) II. 105 To disdayn ayenst natures newe estate.    
%\P 1636 B. JONSON  \textit{Discov. ad fin., Ajax, deprived of Achilles' armour..disdains; and growing impatient of the injury, rageth, and runs mad.    
%\P 1634 SIR T. HERBERT  \textit{Trav. 150 Cheese and Butter is among them, but such as squemish English stomacks will disdaine at.

%\itembf{4.} trans. To move to indignation or scorn; to offend, anger, displease. Obs.

%\P a1470 TIPTOFT  \textit{Caesar x. (1530) 12 Induciomarus was sore displeased and dysdayned at thys doynge.    
%\P 1627 VOX  \textit{Piscis A v b, It shall nothing disdaine you; for it is no new thing, but even that which you have continually looked for.    
%\P 1650 HOWELL  \textit{Giraffi's Rev. Naples 18 The people..being much disdain'd that the Vice-Roy had scap'd.    1790-
%\P 1817 COMBE  \textit{Devil upon Two Sticks in Lond. I. 251 Fashionable amusements delight him not, and even elegant vice disdains him.

%\itembf{b.} impers. it disdains me: it moves my indignation, offends me.

%\P c1440  \textit{York Myst.} v. 11 Me thoght þat he The kynde of vs tane myght, And þer-at dideyned me.



%\end{myenumerate}


%%%%%%%%%%%%%%%%%%%%%%%%%%%%%%%%%
%\myitem{disingenuous} a.

%\noindent \phonetic{(dɪsɪnˈdʒɛnjuːəs)}

%\noindent [dis- 10.]
%\vspace{-0.3cm}

%\begin{myenumerate}

%The opposite of ingenuous; lacking in candour or frankness, insincere, morally fraudulent. (Said of persons and their actions.)

%\P 1655 [see DISINGENIOUS].    
%\P 1657 BURTONS'S  \textit{Diary (1828) II. 291 It will be disingenuous to think that his Highness and the Council should be under an oath, and your members free.    
%\P 1673 LADY'S  \textit{Call. i. v. ⁋3. 32 Of such disingenuous addresses, 'tis easy to read the event.    
%\P 1718 FREETHINKER  \textit{No. 67. ⁋9 A Disingenuous Speaker is most effectually refuted without Passion.    
%\P 1827 HALLAM  \textit{Const. Hist. (1876) I. ii. 98 Cranmer..had recourse to the disingenuous shift of a protest.    
%\P 1875 HELPS  \textit{Ess., Advice 46 It is a disingenuous thing to ask for advice, when you mean assistance.

%Hence disinˈgenuously adv., in a disingenuous manner, not openly or candidly, meanly, unfairly.

%\P 1661 H. NEWCOME  \textit{Diary (1849) 26 So disingenuously..I have carryed toward my God.    
%\P 1678 [see DISINGENIOUS].    
%\P 1748 RICHARDSON  \textit{Clarissa} (1811) I. xxxix. 289 Although I had most disingenuously declared otherwise to my mother.    
%\P 1836 J. GILBERT  \textit{Chr. Atonem. viii. (1852) 232 We should deem it to be disingenuously evasive.



%\end{myenumerate}


%%%%%%%%%%%%%%%%%%%%%%%%%%%%%%%%%
%\myitem{disparate} a. and n.

%\noindent \phonetic{(ˈdɪspərət)}

%\noindent [orig. ad. L. disparāt-us separated, divided, pa. pple. of disparāre, f. dis- 1 + parāre to make ready, prepare, provide, contrive, etc.; but in use, app. often associated with L. dispar unequal, unlike, different.]
%\vspace{-0.3cm}

%\begin{myenumerate}

%\itembf{A.} adj.

%\itembf{1.} Essentially different or diverse in kind; dissimilar, unlike, distinct. In Logic, used of things or concepts having no obvious common ground or genus in which they are correlated. Hence distinguished from contrary, since contrary things are at least correlated in pairs, e.g. good and bad. Also distinguished from disjunct, since disjunct concepts may all be reduced to a common kind.
%   Disparātus appears first in Cicero De Inv. Rhet. 28. 42, applied to the mere separation expressed by sapere, non sapere, or A is not B, as against the opposition of hot and cold, life and death; it is used by Boethius, De Syll. Hyp. (ed. Bas.) 608, to denote things which are only different, without any conflict of contrariety (tantum diversa, nulla contrarietate pugnantia). It reappears in 14-15th c. with the school of Occam, e.g. in Rud. Strodus and Paulus Venetus, and is retained in modern transformations of the scholastic logic. According to Ueberweg Logic §53, disparate conceptions are those which do not fall within the extent of the same higher, or at least of the same next higher conception. (Prof. W. Wallace.)

%\P 1608 BP. J. KING  \textit{Serm. 5 Nov. 5 Two disperate species and sorts of men.    
%\P 1633 AMES  \textit{Agst. Cerem. ii. 243 Can men give manifold disparate senses to one and the same Ceremonies?    
%\P 1642 FULLER  \textit{Holy \& Prof. St. iv. vii. 273 Not onely disparate, but even opposite terms.    
%\P 1684 T. BURNET  \textit{Th. Earth i. 302 As remote in their nature..as any two disparate things we can propose or conceive; number and colour.    
%\P 1748 HARTLEY  \textit{Observ. Man i. iii. 296 The Terms must be disparate, opposite, or the same.    
%\P 1781 BENTHAM  \textit{Wks. (1843) X. 92 A personage of a nature very disparate to the former.    
%\P 1837-8 SIR W. HAMILTON  \textit{Logic xii. (1860) I. 224 Notions co-ordinated in the whole of comprehension, are, in respect of the discriminating characters, different without any similarity. They are thus, pro tanto, absolutely different; and, accordingly, in propriety are called Disparate Notions, (notiones disparatæ). On the other hand, notions co-ordinated in the quantity or whole of extension..are only relatively different (or diverse); and, in logical language, are properly called Disjunct or Discrete Notions.    
%\P 1865 GROTE  \textit{Plato} I. vi. 249 Other creeds, disparate or discordant.    
%\P 1883 F. HARRISON in  \textit{Pall Mall G. 3 Nov. 1/2 The questions are so utterly disparate as not to be reducible to the same argument.

%b.A.1.b (See quot.)

%\P 1867 L. H. ATWATER  \textit{Elem. Logic ii. §11. 69 Any one of given Co-ordinate Species, is called, in relation to any one part of a higher or lower Co-ordinate Division under the Summum Genus, Disparate. Thus..lion, as compared to fish, Shetland pony, or bull-dog, is Disparate.

%c.A.1.c (See quot.)

%\P 1883 SYD. SOC.  \textit{Lex., Disparate points, two points upon the two retinæ which, when a ray of light falls upon them, do not produce similar impressions. Used by Fachner in opposition to corresponding points.

%\itembf{2.} Unequal, on a disparity.

%\P 1764 T. PHILLIPS  \textit{Life Pole (1767) I. 6 Which at very disparate years united these two persons.    
%\P a1834 LAMB  \textit{Misc. Wks. (1871) 449 Between ages so very disparate.    
%\P 1879 FARRAR  \textit{St. Paul I. 416 Paul proceeds to narrate the acknowledgment of the Three that his authority was in no sense disparate with theirs.

%\itembf{B.} n. Chiefly pl. Disparate things, words, or concepts; things so unlike that they cannot be compared with each other.

%\P 1586 BRIGHT  \textit{Melanch. xii. 59 Contrarie faculties or such as we call desparates in logicke.    
%\P 1588 FRAUNCE  \textit{Lawiers Log. i. x. 47 Disparates are sundry opposites whereof one is equally and in like manner opposed unto many.    
%\P 1623 COCKERAM,  \textit{Disparates, words which are differing one from another, but not contrarie, as heat and cold are contraries, but heat and moisture disparates.    
%\P 1654 JER. TAYLOR  \textit{Real Pres. 109 It is the style of both the Testaments to speak in signs and representments, where one disparate speaks of another; as it does here: the body of Christ, of the bread.    
%\P 1682 R. BURTHOGGE  \textit{An Arg. (1684) 154 Disparates are distinct, and are not opposites.    
%\P 1722 WOLLASTON  \textit{Relig. Nat. v. 71 If they are supposed to be only different, not opposite, then if they differ as disparates, there must be some genus above them.    
%\P 1849 GROTE  \textit{Greece} ii. lxviii. (1862) VI. 180 Blending together disparates or inconsistencies.



%\end{myenumerate}


%%%%%%%%%%%%%%%%%%%%%%%%%%%%%%%%%
%\myitem{dissemble} v.1

%\noindent \phonetic{(dɪˈsɛmb(ə)l)}

%\noindent [app. a later form of dissimule v., through the intermediate stages dissimill, dissimble, influenced perh. by resemble. (There is no corresponding form in F.: cf. the next two words.)]
%\vspace{-0.3cm}

%\begin{myenumerate}

%\itembf{1.} trans. To alter or disguise the semblance of (one's character, a feeling, design, or action) so as to conceal, or deceive as to, its real nature; to give a false or feigned semblance to; to cloak or disguise by a feigned appearance.

%\P 1513 MORE  \textit{Rich. III, Wks. 65 Some..not able to dissemble their sorrow, were fayne at his backe to turne their face to the wall.    
%\P 1552  \textit{Bk. Com. Prayer, Morn. Pr., That we shoulde not dissemble nor cloke them [our sins] before the face of Almighty God.    
%\P 1665 MANLEY  \textit{Grotius' Low C. Warres 715 Among the Bodies..was found a Woman, who had dissembled her Sex, both in courage and a military Habit.    
%\P 1709 TATLER  \textit{No. 32 ⁋4 With an Air of great Distance, mixed with a certain Indifference, by which he could dissemble Dissimulation.    
%\P 1781 GIBBON  \textit{Decl. \& F.} II. xlvi. 723 He dissembled his perfidious designs.    
%\P 1850 PRESCOTT  \textit{Peru II. 20 He was well pleased with the embassy, and dissembled his consciousness of its real purpose.    
%\P 1860 EMERSON  \textit{Cond. Life, Behaviour Wks. (Bohn) II. 385 How many furtive inclinations avowed by the eye, though dissembled by the lips!

%\itembf{2.} To disguise. Obs.

%\P 1508 DUNBAR  \textit{Tua mariit Wemen 254, I wes dissymblit suttelly in a sanctis liknes.    
%\P 1529 MORE  \textit{Dyaloge iv. Wks. 283/1 Though he dissembled himselfe to bee a Lutherane whyle he was here, yete as sone as he gate him hence, he gate him to Luther strayght.    
%\P 1601 SHAKES.  \textit{Twel. N.} iv. ii. 4 Ile put it on, and I will dissemble my selfe in't; and I would I were the first that euer dissembled in such a gowne.    
%\P 1665 J. SPENCER  \textit{Vulg. Prophecies 21 Their deformity appeared through the finest colors he could dissemble it with.    
%\P 1697 DRYDEN  \textit{Æneid xii. 340 Dissembling her immortal form, she [Juturna] took Camertus meen.

%\itembf{3.} To pretend not to see or notice; to pass over, neglect, ignore.

%\P c1500 [see DISSEMBLING vbl. n.].    
%\P c1555 HARPSFIELD  \textit{Divorce Hen. VIII (1878) 233, I will not urge..the Pope's..authority..I will dissemble that excellency.    
%\P 1568 GRAFTON  \textit{Chron. II. 823 Wherfore he determined to dissemble [Hall dissimule] the matter as though he knew nothing.    
%\P 1579 LYLY  \textit{Euphues} (Arb.) 150 Some lyght faults lette them dissemble, as though they knew them not, and seeing them let them not seeme to see them.    
%\P 1692 RAY  \textit{Dissol. World iii. viii. (1732) 395, I must not dissemble a great Difficulty.    
%\P 1701 WALLIS 24 SEPT. in  \textit{Pepys Mem., It hath been too late to dissemble my being an old man.    
%\P 1703 ROWE  \textit{Ulysses i. i. 75 Learn to dissemble Wrongs.    
%\P 1761 HUME  \textit{Hist. Eng.} II. xlii. 451 Philip..seemed to dissemble the daily insults and injuries which he received from the English.

%\itembf{b.} with clause: To shut one's eyes to the fact.

%\P 1554 RIDLEY  \textit{Lord's Supper Wks. 41 It is neither to be denied, nor dissembled that..there be diuerse points wherein men..canne not agree.    
%\P 1611 BIBLE  \textit{Transl. Pref. 11 It cannot be dissembled, that..it hath pleased God [etc.].    
%\P 1692 RAY  \textit{Dissol. World ii. ii. (1732) 107, I must not dissemble or deny, that in the Summer-time the Vapours do ascend.    
%\P a1831 A. KNOX  \textit{Rem. (1844) I. 54 It cannot be dissembled, that..the House of Commons seems to feel no other principle than that of vulgar policy.    
%\P 1871 MORLEY  \textit{Voltaire (1886) 8 No attempt is made in these pages to dissemble in how much he was condemnable.

%\itembf{c.} intr. const. with.

%\P a1533 FRITH  \textit{Wks. (1573) 51 These holy doctours..thought it not best..to condemne all thinges indifferently: but to suffer and dissemble wyth the lesse.

%\itembf{4.} absol. or intr. To conceal one's intentions, opinions, etc. under a feigned guise; ‘to use false professions, to play the hypocrite’ (J.).

%\P 1523 LD. BERNERS  \textit{Froiss. I. clxxx. 216 Therfore the duke dissembled for the pleasur of the prouost.    
%\P 1535 COVERDALE  \textit{1 Macc. xi. 53 He dyssembled in all that euer he spake.    
%\P 1596 SHAKES.  \textit{Tam. Shr.} ii. i. 9 Tel Whom thou lou'st best: see thou dissemble not.    
%\P 1671 MILTON  \textit{P.R.} i. 467 The subtle fiend..Dissembled, and this answer smooth return'd.    
%\P 1713 ADDISON  \textit{Cato i. ii, I must dissemble, And speak a language foreign to my heart.    
%\P 1852 LONGFELLOW  \textit{Warden Cinque Ports xi, He did not pause to parley nor dissemble.

%\itembf{b.} const. with: To use dissimulation with.

%\P 1586 A. DAY  \textit{Eng. Secretary i. (1625) 142, I dissemble not with you..for you shall finde it and prove it to be true.    
%\P 1667 POOLE  \textit{Dial. betw. Protest. \& Papist (1735) 83, I will not dissemble with you, they do not.    
%\P 1718 FREETHINKER  \textit{No. 75 ⁋3 He who dissembles with, or betrays, one Man, would betray every Man.    
%\P 1829 SOUTHEY  \textit{All for Love vi, Dissemble not with me thus.

%\itembf{5.} trans. To put on a feigned or false appearance of; to feign, pretend, simulate. Obs.

%\P 1538 STARKEY  \textit{England} i. iii. 91 Men may dyssembyl and fayne grete pouerty, where as non ys.    
%\P 1581 J. BELL  \textit{Haddon's Answ. Osor. 467 You were not your selfe ignoraunt, albeit you dissembled the contrary.    
%\P 1660 F. BROOKE tr.  \textit{Le Blanc's Trav. 304 This Creature..that can dissemble death so naturally.    
%\P 1709 STEELE  \textit{Tatler} No. 83 ⁋2 I'm lost if you don't dissemble a little Love for me.    
%\P 1791 BOSWELL Johnson an.
%\P 1752 TO SUPPOSE  \textit{that Johnson's fondness for her was dissembled.

%\itembf{b.} with inf. or clause. Obs.

%\P 1654 R. CODRINGTON tr.  \textit{Hist. Ivstine 60 The King dissembled that his Coat of Mayl was not fit for him.    
%\P 1813 T. BUSBY tr.  \textit{Lucretius iv. 913 Fancy..Lost friends, past joys, dissembleth to restore.

%\itembf{c.} To feign or pretend (some one) to be something. Also with ellipsis of the inf., or of both object and inf. Obs.

%\P 1634 FORD  \textit{P. Warbeck i. i, Charles of France..Dissembled him the lawful heir of England.    
%\P 1655 FULLER  \textit{Ch. Hist.} iii. vii. §19 John Scott dissembled himself an English-man.    
%\P 1660 F. BROOKE tr.  \textit{Le Blanc's Trav. 176 Esteemed a Jew though he dissembled the Christian.    Ibid. 246 Moores who dissembled Christians.

%d.5.d fig. To simulate by imitation. Obs.

%\P 1697 DRYDEN  \textit{Æneid viii. 880, The gold dissembl'd well their yellow hair.



%\end{myenumerate}


%%%%%%%%%%%%%%%%%%%%%%%%%%%%%%%%%
%\myitem{dissertation} n.

%\noindent \phonetic{(dɪsəˈteɪʃən)}

%\noindent [ad. L. dissertātiōn-em discourse, disquisition, n. of action f. dissertāre to dissertate.]
%\vspace{-0.3cm}

%\begin{myenumerate}

%\itembf{1.} Discussion, debate. Obs.

%\P 1611 SPEED  \textit{Hist. Gt. Brit. ix. xxii. (R.) As in a certaine dissertation had once with Master Cheeke it appeared.    
%\P 1623 COCKERAM,  \textit{Dissertation, a disputing on things.    
%\P 1677 GALE  \textit{Crt. Gentiles iii. 27 Paul mentions some who had turned aside..to unprofitable dissertation or disputation.    
%\P 1709 STRYPE  \textit{Ann. Ref. I. xi. 137 [They] altogether refused..to engage in further dissertation with them.

%\itembf{2.} A spoken or written discourse upon or treatment of a subject, in which it is discussed at length; a treatise, sermon, or the like; = discourse n. 5.

%\P 1651 HOBBES  \textit{Govt. \& Soc. Title-p., A Dissertation concerning Man in his severall habitudes and respects.    
%\P 1683 DRYDEN  \textit{Life Plutarch 60 Observing this, I made a pause in my dissertation.    
%\P 1728 POPE  \textit{Dunc.} iii. Notes, He compos'd three dissertations a week on all subjects.    
%\P 1762-71 H. WALPOLE  \textit{Vertue's Anecd. Paint. (1786) I. 238 Vermander dedicated to Ketel a dissertation on the statues of the ancients.    
%\P 1841 D'ISRAELI  \textit{Amen. Lit. (1867) 476 Warton has expressly written a dissertation on that subject.    
%\P 1879 GLADSTONE  \textit{Glean.} V. i. 77 The sermon is a dissertation, and does violence to nature in the effort to be like a speech.

%Hence disserˈtational a., belonging to or of the nature of a dissertation; disserˈtationist, one who makes a dissertation.

%\P 1844 DE QUINCEY  \textit{Logic of Political Economy 36 This remark was levelled by the dissertationist..(I believe) at Ricardo.    
%\P 1846 WORCESTER  \textit{cites Ch. Observ. for Dissertational.    
%\P 1865  \textit{Reader No. 113. 234/2 Dissertational, poetic, and rhetorical plays.    
%\P 1866  \textit{Spectator} 20 Oct. 1162/2 The dissertational language of so dry a piece of theoretic definition as the creed called the Athanasian.


%______________________________


%Additions 1993

%Add: [2.] \itembf{b.} spec. An extended scholarly essay, usu. based upon original research, submitted for a degree or other academic qualification.

%\P 1873 CATAL.  \textit{Officers \& Students
%\P 1873-74 (COLUMBIA  \textit{College, N.Y.) 114 Graduates of the School who pursue for one year a course of study prescribed by the Faculty, and present an acceptable dissertation embodying their results, receive the degree of Doctor of Philosophy.    
%\P 1930 J. C. ALMACK  \textit{Research \& Thesis Writing i. 11 Like thesis, dissertation has come to connote research, but it is a more pedantic term, and usually is reserved strictly for the thesis for the doctorate.    
%\P 1964 CAL.  \textit{Univ. Newcastle upon Tyne
%\P 1964-65 327  \textit{The Final Honours Examination shall consist of nine papers, or eight papers and a dissertation.    
%\P 1972 A. J. AYER  \textit{Russell i. 14 Russell obtained his Fellowship at Trinity with a dissertation on the Foundations of Geometry.    
%\P 1984 D. CUPITT  \textit{Sea of Faith v. 142 Marx duly obtained his doctorate, moved to Bonn to work with Bauer, and began to write a qualifying dissertation.



%\end{myenumerate}


%%%%%%%%%%%%%%%%%%%%%%%%%%%%%%%%%
%\myitem{dissimulate} v.

%\noindent \phonetic{(dɪˈsɪmjʊleɪt)}

%\noindent [f. L. dissimulāt- ppl. stem of dissimulāre: see dissimule.
%\vspace{-0.3cm}

%\begin{myenumerate}
%   Rare bef. the end of 18th c.; not in J., T., nor Webster 1828.]

%\itembf{1.} trans. To pretend not to see, leave unnoticed, pass over, neglect. Obs. rare.

%\P a1533 LD. BERNERS  \textit{Gold. Bk. M. Aurel. ix. (R.) That al thyng be forgiuen to theim that be olde and broken, and to theim that be yonge and lusty to dissimulate for a time, and nothyng to be forgiuen to very yong children.

%\itembf{2.} To conceal or disguise under a feigned appearance; to dissemble.

%\P 1610 BP. CARLETON  \textit{Jurisd. 204 Frederick..being taken prisoner when he would haue dissimulated his estate, he was knowne by his picture.    
%\P 1872 GEO. ELIOT  \textit{Middlem. iii, Public feeling required the meagreness of nature to be dissimulated by tall barricades of frizzed curls and bows.    
%\P 1882 STEVENSON  \textit{New. Arab. Nts. (1884) 127 If ever..he described some experience personal to himself, it was so aptly dissimulated as to pass unnoticed with the rest.

%\itembf{b.} intr. To practise dissimulation, to dissemble.

%\P 1796 MRS. HOWELL  \textit{Anzoletta Zadoski I. 152 He could not so far dissimulate as to promise his concurrence.    
%\P 1847 LYTTON  \textit{Lucretia ii, All weakness is prone to dissimulate.

%\itembf{3.} Electr. To conceal the presence of (electricity) by neutralizing it; cf. disguise v. 8.

%\P 1838 FARADAY  \textit{Exp. Res. Electr. §
%\P 1684 THE  \textit{terms free charge and dissimulated Electricity convey therefore erroneous notions if they are meant to imply any difference as to the mode or kind of action.    Ibid. The one [charge] is not more free or more dissimulated than the other.    
%\P 1870 J. T. SPRAGUE in  \textit{Eng. Mech. 11 Feb. 519/3 The negative electricity..neutralises the positive..which is thus bound or dissimulated.

%Hence diˈssimulated ppl. a.; diˈssimulating vbl. n. and ppl. a.

%\P 1794 MISS GUNNING  \textit{Packet I. 56 The mask..was torn from..the dissimulating Mrs. Johnson.    
%\P 1838 DISSIMULATED  \textit{electricity [see 3 above].    
%\P 1843 BROWNING BLOT in  \textit{Scutcheon i. iii, Some fierce leprous spot Will mar the brow's dissimulating.    
%\P 1874 MIVART EVOLUTION in  \textit{Contemp. Rev. Oct. 773 The long dissimulated Atheism of Mill is now avowed.



%\end{myenumerate}


%%%%%%%%%%%%%%%%%%%%%%%%%%%%%%%%%
%\myitem{dissolute} a. (n.)

%\noindent \phonetic{(ˈdɪsəl(j)uːt)}

%\noindent [ad. L. dissolūtus loose, disconnected, pa. pple. of dissolvĕre to loosen, disunite, dissolve; cf. F. dissolu.
%\vspace{-0.3cm}

%\begin{myenumerate}
%   The appearance of the senses in Eng. does not correspond with their original development in Latin.]

%\itembf{1.} Having their connexion or union dissolved; disconnected, disjoined, disunited. Obs.

%\P 1541 R. COPLAND  \textit{Guydon's Quest. Chirurg. C j, Nature..wyl nat leue them [membres sparmatyf] thus dyssolute, reioyneth and knytteth them the best that she may.    
%\P 1578 BANISTER  \textit{Hist. Man i. 3 It were requisite, that the..bones should neither be dissolute and unioyned, nor yet altogether whole, and continuall.    
%\P 1651 HOBBES  \textit{Leviath.} iii. xlii. 278 The part excommunicated is no longer a Church, but a dissolute number of individuall persons.    
%\P 1651  \textit{ Govt. \& Soc. vii. §10. 107 It is no longer a Court, or one Person, but a dissolute multitude without any supreme power.

%\itembf{2.} Relaxed, enfeebled, weak; wanting consistence or firmness of texture or temperament. Obs.

%\P c1450 tr.  \textit{De Imitatione iii. xlv, But I be holpen of þe \& inwardly enformed, I am made all leuke \& dissolute.    
%\P 1577 HANMER  \textit{Anc. Eccl. Hist. (1619) 188 You loose hands, and dissolute knees, ye shall be strengthened.    
%\P 1607 TOPSELL  \textit{Four-f. Beasts (1658) 345 The flesh of the Alzabo..is of a slender and dissolute substance.    
%\P 1684 tr.  \textit{Bonet's Merc. Compit. iv. 120 This lax and dissolute consistency [of the blood]..makes it apt to dissolve into Serum.    
%\P 1816 COLERIDGE  \textit{Statesm. Man. 354 Vital warmth..relaxing the rigid, consolidating the dissolute, and giving cohesion to that which is about to sink down.

%\itembf{3.} Having the energies, attention, etc. relaxed; wanting firmness, strictness, or assiduity; loose, lax, slack, careless, negligent, remiss. Obs.

%\P 1382 WYCLIF  \textit{Prov. xix. 15 Slouthe sendeth in slep; and a dissolut [
%\P 1388 NEGLIGENT]  \textit{soule shal hungre.    
%\P c1430 LYDG.  \textit{Minor P. (1840) 245 (Mätz.) Now passyng besy, now dissolut, now ydil.    
%\P 1574 WHITGIFT  \textit{Def. Aunsw. iii. Wks.
%\P 1851 I. 330  \textit{Neither the law was then cruel, neither yet the gospel is now dissolute for the greatness of forgiveness.    
%\P 1589 HAKLUYT  \textit{Voy. 188 Through meere dissolute negligence shee [a ship] perished on a sand.    
%\P 1597 HOOKER  \textit{Eccl. Pol.} v. lxxii. §18 To temper the minde, lest contrarie affection comming in place should make it too profuse and dissolute.    
%\P 1619 W. SCLATER  \textit{Exp. 1 Thess. (1630) 459 Alas, how cold..are our affections often? How dissolute our practice? How dull our memory?

%\itembf{4.} Unrestrained in behaviour or deportment; not subject to proper restraint; loose, wanton. (In quot. 1620, Wasteful, lavish.) Obs. (exc. as involved in 5).

%\P c1460 STANS  \textit{Puer 20 (MS. Harl. 2251) in Babees Bk. 26 With dissolute [MS. Lamb. wantowne] laughters do thow non offence To-fore thy souerayn.    
%\P 1526  \textit{Pilgr. Perf.} (W. de W. 1531) 99 b, What cause hast yu to be so dissolute \& mery?    
%\P 1616 SURFL. \& MARKH.  \textit{Country Farme 117 This cattell is foolish and dissolute, easie to stray abroad hither and thither, contrarie vnto sheepe, which keepe together.    
%\P 1620 SHELTON  \textit{Don Quixote ii. iv, A great deal of Goods..of all which the young man remained a dissolute Lord.    
%\P 1652 NEEDHAM tr.  \textit{Selden's Mare Cl. 45 A rude sort of men, without Laws, without Government, free and dissolute [liberum alque solutum].    
%\P 1713 BERKELEY  \textit{Guardian No. 3 ⁋1 It is a certain Characteristick of a dissolute and ungoverned mind to rail or speak disrespectfully of them.

%\itembf{b.} Careless or lawless in style. Now rare.

%\P 1566 T. STAPLETON  \textit{Ret. Untr. Jewel Epist., Your maner of writing is..so Dissolut Loose and Negligent.    
%\P 1619 W. SCLATER  \textit{Exp. 1 Thess. (1630) 559 Either hee is too profound, or too plaine..too dissolute, or too exact.    
%\P 1718 PRIOR  \textit{Solomon Pref., Heroic with continued rhyme..was found too dissolute and wild.    
%\P 1771 H. WALPOLE  \textit{Vertue's Anecd. Paint. IV. i. (R.) A loose, and, if I may use the word, a dissolute kind of painting.    
%\P 1851 RUSKIN  \textit{Stones Ven.} (1874) I. xvii. 184 The dissolute dulness of English Flamboyant.

%\itembf{5.} That has thrown off the restraints of morality and virtue; lax in morals, loose-living; licentious, profligate, debauched. (Of persons, their actions, etc.) The current sense.

%\P 1513 BRADSHAW  \textit{St. Werburge i. 28 Dyssolute man folowyng sensualyte.    
%\P 1548 HALL  \textit{Chron., Rich. III (an. 2) 32 b, A woman geven to carnall affection, and dissolute livinge.    
%\P 1598 SHAKES.  \textit{Merry W. iii. iii. 204 Wee will yet haue more trickes with Falstaffe: his dissolute disease will scarse obey this medicine.    
%\P 1671 MILTON  \textit{P.R.} ii. 150 Belial, the dissolutest Spirit that fell, The sensualest, and, after Asmodai, The fleshliest Incubus.    
%\P 1729 BUTLER Serm. Wks.
%\P 1874 II. 15 THE  \textit{many untimely deaths occasioned by a dissolute course of life.    
%\P 1874 GREEN  \textit{Short Hist. vi. §i. 267 The nobles were as lawless and dissolute at home as they were greedy and cruel abroad.

%\itembf{B.} n. A dissolute person, a profligate. rare.

%\P 1608 DAY  \textit{Hum. out of Br. iv. iii, Did your euer conuerse with a more straunger dissolute?    
%\P 1824 LANDOR  \textit{Wks. (1846) I. 177/2 Half the dissolutes in the parish.    
%\P 1838 SOUTHEY  \textit{Poet's Pilgrim. ii. iii. x. note, The homely but scriptural appellation..has been delicately softened down..Helen Maria Williams names her [Ch. of Rome] the Dissolute of Babylon.

%¶There are many instances of dissolute for desolate (dissolate), mostly scribal or typographical errors, sometimes perh. owing to actual confusion.

%\P 1509 HAWES  \textit{Past. Pleas. xxxvi. i, A place of dissolute darkenes.    
%\P 1612 BREREWOOD  \textit{Lang. \& Relig. x. 83 Greece..more dissolute then any region of Europe subject to the Turk.    
%\P 1834 T. C. CROKER  \textit{Fairy Leg. S. Irel. 135 I got ashore, somehow or other..upon a dissolute island.



%\end{myenumerate}


%%%%%%%%%%%%%%%%%%%%%%%%%%%%%%%%%
%\myitem{distraught} ppl. a. arch.

%\noindent \phonetic{(dɪˈstrɔːt)}

%\noindent [modification of distract ppl. a., L. distract-us.
%\vspace{-0.3cm}

%\begin{myenumerate}
%   Not of ordinary phonetic origin, but due app. to association with other pa. pples. in -ght, as caught, taught, bought, brought, sought, thought, wrought. Perh. more immediately influenced by straught, pa. pple. of stretch; as the latter had also the form streight, straight, it may be that distraught = distreight = distrait.]

%\itembf{1.} Mentally distracted, by being drawn or driven in diverse directions or by conflicting emotions; deeply agitated or troubled; = distracted 4.

%\P 1393 GOWER  \textit{Conf.} I. 218 Wherof his herte is so distraught.    Ibid. 279 Many a good felawe Hath be destraught by sodein chaunce.    
%\P c1491 CHAST.  \textit{Goddes Chyld. xxvii. 79 Some ben so ferforth distraught..that whan they come ayen to hemself it is clene fro her mynde where they left.    
%\P 1591 SPENSER  \textit{Ruines of Time 578, I in minde remained..Distraught twixt feare and pitie.    
%\P 1608-11 BP. HALL  \textit{Medit. \& Vowes i. §92 The worldling standes amazed and distraught with the evill.    
%\P 1610 G. FLETCHER  \textit{Christ's Tri. (1632) 44 With present fear, and future grief distraught.    
%\P 1848 LYTTON  \textit{Harold i. i, Her mind is somewhat distraught with her misfortunes.    
%\P 1877 L. MORRIS  \textit{Epic Hades i. 17, I lay awake Distraught with warring thoughts.

%\itembf{2.} Driven to madness; mentally deranged; crazy: = distracted 5.

%\P 1592 SHAKES.  \textit{Rom. \& Jul.} iv. iii. 49.    
%\P 1594  \textit{ Rich. III, iii. v. 4 And then againe begin, and stop againe, As if thou were distraught, and mad with terror.    
%\P 1598 STOW  \textit{Surv. (1842) 167/2 One house, wherein sometime were distraught and lunatic people.    
%\P 1652 GAULE  \textit{Magastrom. 90 Fools, madmen, melancholy, fanatic, distraught.    
%\P 1828 SCOTT  \textit{F.M. Perth} xix, ‘Are ye distraught, lassie?’ shouted Dorothy.    
%\P 1886 HALL  \textit{Caine Son of Hagar iii. v, Hugh Ritson rushed here and there like a man distraught.

%\itembf{b.} Const. of, in (wits, senses, etc.). Obs. (In senses 1 and 2.)

%\P 1556 AURELIO \& ISAB.  \textit{(1608) F, Folkes distraghte of wisdome.    
%\P 1583 T. WATSON  \textit{Centurie of Loue lxxxix. (Arb.) 125 Loue is distraught of witte, and hath no end.    
%\P 1653 H. COGAN tr.  \textit{Pinto's Trav. viii. 23 Like a man distraught of his wits I cast myself at the feet of the Elephant.    
%\P 1657 HOWELL  \textit{Londinop. 66 In this place [Bethlem] people that be distraught in their wits.

%\itembf{3.} lit. Pulled asunder, drawn in different directions. (Spenserian use.) Obs.

%\P 1596 SPENSER  \textit{F.Q.} iv. vii. 31 [An arrow] in his nape arriving, through it thrild His greedy throte, therewith in two distraught.    Ibid. v. v. 2 A Camis..Trayled with ribbands diversly distraught.    
%\P 1604 R. CAWDREY  \textit{Table Alph., Distraught, drawne into diuers parts.    
%\P 1642 H. MORE  \textit{Song of Soul ii. ii. ii. x, By distrought distension.

%\itembf{4.} As pa. pple. of distract, or distraught v.

%\P 1581 G. PETTIE  \textit{Guazzo's Civ. Conv. i. (1586) 40 b, [They] have bene distraught of their right understanding.    
%\P 1625 K. LONG tr.  \textit{Barclay's Argenis ii. xxi. 139 What fury..hath distraught you of your wits?    
%\P 1816 SOUTHEY  \textit{Lay of Laureate Epil. 2 Have fanatic dreams distraught his sense?



%\end{myenumerate}


%%%%%%%%%%%%%%%%%%%%%%%%%%%%%%%%%
%\myitem{doff} v.

%\noindent \phonetic{(dɒf)}

%\noindent [Coalesced form of do off: see do v. 47. Cf. also daff v.2
%\vspace{-0.3cm}

%\begin{myenumerate}
%   In ordinary colloquial use in north of England (not in Scotl.). Elsewhere, since 16th c., a literary word with an archaic flavour. Ray noted it as a northern provincialism; Johnson, as ‘in all its senses obsolete, and scarcely used except by rustics’. In 19th c., from the time of Scott, very frequent in literary use.]

%\itembf{1.} trans. To put off or take off from the body (clothing, or anything worn or borne); to take off or ‘raise’ (the head-gear) by way of a salutation or token of respect.

%\P c1350  \textit{Will. Palerne 2342 Dof blive þis bere skyn.    
%\P c1400 MANDEVILLE (Roxb.) xxv. 120 He doffez his hatte.    
%\P 1401  \textit{Pol. Poems (Rolls) II. 107 The sacred host..to whiche we knele and doffe our hodes.    
%\P 1483  \textit{Cath. Angl.} 103/1 To Doffe, exuere.    
%\P 1595 SHAKES.  \textit{John} iii. i. 128 Thou weare a Lyons hide! doff it for shame.    
%\P 1596 SPENSER  \textit{F.Q.} vi. ix. 36 Calidore..doffing his bright armes, himselfe addrest In shepheards weed.    
%\P 1621 G. SANDYS  \textit{Ovid's Met. xiii. (1626) 259 Then made him d'off those weeds.    
%\P 1714 GAY  \textit{Sheph. Week iv. 21 Upon a rising Bank I sat adown, Then doff'd my Shoe.    
%\P 1768 BEATTIE  \textit{Minstr. i. xxxv, The little warriors doff the targe and spear.    
%\P 1808 SCOTT  \textit{Marm.} vi. xi, Doffed his furred gown, and sable hood.    
%\P 1859 TENNYSON Enid
%\P 1444 THE..EARL..CAST  \textit{his lance aside, And doff'd his helm.

%\itembf{b.} Const. off; also intr. with with. Obs. rare.

%\P a1400  \textit{Morte Arth.}
%\P 1023 ÞOW  \textit{doffe of thy clothes, And knele in thy kyrtylle.    
%\P 1643 [see DOFFING vbl. n.].    
%\P 1764 FOOTE  \textit{Mayor of G. ii. Wks.
%\P 1799 I. 186  \textit{If you will doff with your boots, and box a couple of bouts.

%\itembf{c.} absol. To raise one's hat (to a person). rare.

%\P 1674 N. FAIRFAX  \textit{Bulk \& Selv. To Rdr., To look full on a Great man standing in my way, and not to vouchsafe him worth Doffing to.    
%\P 1833 TENNYSON  \textit{Goose 19 The grave churchwarden doff'd, The parson smirk'd and nodded.

%\itembf{2.} refl. To undress oneself, put off one's clothes. Also fig. Now only dial.

%\P 1697 DE LA PRYME  \textit{Diary (Surtees) 150 The quaker doffs him stark naked, and takeing a burning candle in his hand he goes to the church.    [
%\P 1838 J. SCHOLES  \textit{Lanc. Witches in Harland L. Lyrics (1865) 133 ‘Hie thi whoam an' doff thi.’]

%\itembf{3.} transf. and fig. To put off as a dress or covering; to throw off, lay aside; hence (in wider sense), to do away with, get rid of (anything associated with oneself).  Also with off (obs.).

%\P 1592 SHAKES.  \textit{Rom. \& Jul.} ii. ii. 47.    
%\P 1599 B. JONSON  \textit{Ev. Man out of Hum. v. v, He..oftentimes d'offeth his owne nature and puts on theirs.    
%\P 1605 SHAKES.  \textit{Macb.} iv. iii. 188 Your eye..would create Soldiours, make our women fight, To doffe their dire distresses.    
%\P 1628 EARLE  \textit{Microcosm., Vp-start Countrey Knt. (Arb.) 38 He ha's doft off the name of a Clowne.    
%\P 1854-6 PATMORE ANGEL in  \textit{Ho. i. ii. x. (1879) 237 Love..doffed at last his heavenly state.    
%\P 1867 BP. FORBES  \textit{Exp. 39 Art. ii. (1881) 29 The Word is said to have donned human nature, never more to doff it.

%\itembf{4.} To put (any one) off (with an excuse, etc.); to turn aside: cf. daff v.2 2. Obs.

%\P 1622 SHAKS.  \textit{Oth. iv. ii. 176 (Qo. 1) Euery day thou dofftst [Fol. i. dafts] me with some deuise, Iago.    
%\P a1637 B. JONSON  \textit{Sad Sheph. i. ii, They..strew tods' hairs, or with their tails do sweep The dewy grass, to do'ff the simpler sheep.    
%\P 1658-9  \textit{Burton's Diary (1828) IV. 67 They doffed us off as long as they could, and then locked up their doors.

%\itembf{5.} Textile Manuf. \itembf{a.} To strip off the slivers of wool, cotton, etc., from the carding-cylinders. \itembf{b.} To remove the bobbins or spindles when full to make room for empty ones. See doffer.

%\P 1825 [see DOFFING vbl. n. b].    
%\P 1851  \textit{Art Jrnl. Catal. Gt. Exhib. p. iv **/2 This..instrument doffs the cotton in a fine transparent fleece.    
%\P 1864 R. A. ARNOLD  \textit{Cotton Fam. 33 Spinners..have, in technical language..to ‘doff the cops’; in other words..to remove and relieve the spindles of the spun yarn.    
%\P 1879  \textit{Cassell's Techn. Educ. IV. 356/2.



%\end{myenumerate}


%%%%%%%%%%%%%%%%%%%%%%%%%%%%%%%%%
%\myitem{dogged} a. (adv.)

%\noindent \phonetic{(ˈdɒgɪd)}

%\noindent [f. dog n.1 + -ed2: cf. crabbed, which appears to be of about the same age.]
%\vspace{-0.3cm}

%\begin{myenumerate}

%\itembf{A.} adj.

%\itembf{1.} gen. a.A.1.a Like a dog; having the character, or some characteristic, of a dog. b.A.1.b Of or pertaining to a dog or dogs, canine. dogged appetite, hunger: = canine appetite, bulimy (obs.). (Now rare in gen. sense.)

%\P c1440  \textit{Promp. Parv.} 125/2 Doggyd, caninus.    
%\P 1589 PASQUIL'S  \textit{Ret. 12 This dogged generation, that is euer barking against the Moone.    
%\P 1595 SHAKES.  \textit{John} iv. iii. 149 Now for the bare-pickt bone of Maiesty, Doth dogged warre bristle his angry crest, And snarleth in the gentle eyes of peace.    
%\P 1608 HIERON  \textit{2nd Pt. Def. Reas. Refus. Subscript. 121 That hunger which Phisitions cal the dogged appetite.    
%\P 1658 J. JONES  \textit{Ovid's Ibis 594 Dianas guard the Tragic poet slew, So be thou torn by a watchful dogged crew.    
%\P 1740 PINEDA  \textit{Sp. Dict. s.v. R, This Letter..They call..dogged, because it sounds like the Noise a Dog makes when he growls.

%\itembf{2.} Having the bad qualities of a dog; currish. a.A.2.a Ill-conditioned, malicious, crabbed, spiteful, perverse; cruel. (Of persons, their actions, etc.) Obs.

%\P a1307  \textit{Pol. Songs (Camden) 199 The fals wolf stode behind; He was doggid and ek felle.    
%\P c1400  \textit{Destr. Troy} 10379 Of so dogget a dede.    
%\P c1440  \textit{Promp. Parv.} 125/2 Doggyde, malycyowse, maliciosus, perversus, bilosus.    
%\P 1540 MORYSINE  \textit{Vives' Introd. Wysd. H viij b, It is a token of a dogged harte, to rejoyce in an other mans mysfortune.    
%\P 1663 BUTLER  \textit{Hud. i. i. 632 Fortune unto them turn'd dogged. For they a sad Adventure met.    
%\P 1684 ROXB.  \textit{Ball. (1895) VIII. 40 This dogged answer cut this poor soul to the heart.

%b.A.2.b transf. Of things: Awkward, ‘crabbed’, difficult to deal with. Obs.

%\P 1634 SIR T. HERBERT  \textit{Trav. 66 The most craggie, steepe, and dogged Hils in Persia.    
%\P 1677 A. YARRANTON  \textit{Eng. Improv. 147 The Spanish [Iron] works tough, churlish and dogged.

%c.A.2.c Ill-tempered, surly; sullen, morose. Now with some mixture of sense 3: Having an air of sullen obstinacy.

%\P c1400  \textit{Rom. Rose} 4028 If Bialacoil be sweete and free, Dogged and felle thou shuldist be.    
%\P 1593 NASHE  \textit{Christ's T. 55 There is vaine-glory in..being Diogenicall and dogged.    
%\P 1667 PEPYS  \textit{Diary} (1879) IV. 424 My wife in a dogged humour for my not dining at home.    
%\P 1757 J. RUTTY  \textit{Diary 5 Feb. in Boswell Johnson, Very dogged or snappish.    
%\P 1852 MRS. STOWE  \textit{Uncle Tom's C. xli, Legree..looked in with a dogged air of affected carelessness, and turned away.

%\itembf{3.} Having the persistency or tenacity characteristic of various breeds of dogs; obstinate, stubborn; pertinacious. (The current use.) Esp. in colloq. phr. it's dogged as does it: persistency and tenacity win in the end.

%\P 1779 JOHNSON 1 APR. in  \textit{Boswell, [He commended one of the Dukes of Devonshire for] ‘a dogged veracity’.    
%\P 1818 SCOTT  \textit{Rob Roy} xxx, An air of stupid impenetrability, which might arise either from conscious innocence or from dogged resolution.    
%\P 1855 PRESCOTT  \textit{Philip II,} I. ii. viii. 229 The dogged tenacity with which he clung to his purposes.    
%\P 1863 KINGSLEY  \textit{Water Bab. vii. (1878) 323 He was such a little dogged, hard, gnarly, foursquare brick of an English boy.    
%\P 1864 M. B. CHESNUT  \textit{Diary 6 Aug. (1949) 429 ‘It's dogged as does it,’ says Isabella.    
%\P 1867 TROLLOPE  \textit{Chron. Barset lxi, There ain't nowt a man can't bear if he'll only be dogged... It's dogged as does it.    
%\P 1874 BLACKIE  \textit{Self-Cult. 20 In this domain nothing is denied to a dogged pertinacity.    
%\P 1896  \textit{Daily News} 27 June 8/1 All his own writing seems to have been done in about three hours a day. ‘It's dogged as does it,’ he has been wont to explain.    
%\P 1942 N. MARSH  \textit{Death \& Dancing Footman x. 195 ‘If we stick..they can damn' well produce a farm animal to lug us out...’ ‘It's dogged as does it,’ said Chloris.

%\itembf{4.} Comb., as dogged-sprighted a., having a ‘dogged’ or malicious spirit (obs.).

%\P 1600 ROWLANDS  \textit{Let. Humours Blood vii. 84 Enuie's the fourth: a Deuill, dogged sprighted.

%\itembf{B.} as adv. ‘As a dog’; very, extremely. colloq. or slang. (Cf. dog n.1 19 d.)

%\P 1819  \textit{Sporting Mag.} IV. 272 He [a horse] was dogged ‘rusty’ when your man passed our house.    
%\P 1847-78 HALLIWELL,  \textit{Dogged, very; excessive. Var. dial.



%\end{myenumerate}


%%%%%%%%%%%%%%%%%%%%%%%%%%%%%%%%%
%\myitem{doggerel} doggrel a. and n.

%\noindent \phonetic{(ˈdɒgərəl, ˈdɒgrəl)}

%\noindent [Origin unknown; but cf. dog n.1 19 e.]
%\vspace{-0.3cm}

%\begin{myenumerate}

%\itembf{A.} adj. An epithet applied to comic or burlesque verse, usually of irregular rhythm; or to mean, trivial, or undignified verse.

%\P c1386 CHAUCER  \textit{Melib. Prol. 7 Now swich a Rym the deuel I biteche This may wel be Rym dogerel quod he.    
%\P 1494 FABYAN Chron. vii. 294 For thoughe I shulde all day tell Or chat with my ryme dogerell.    
%\P 1526 SKELTON  \textit{Magnyf. 413 In bastarde ryme after the doggrell gyse.    
%\P 1589 PUTTENHAM  \textit{Eng. Poesie ii. iv. (Arb.) 89 A rymer that will be tyed to no rules at all..such maner of Poesie is called in our vulgar, ryme dogrell.    
%\P 1630 J. TAYLOR  \textit{(Water P.) Dogge of Warre Wks. ii. 226/1 In doggrell Rimes my Lines are writ As for a Dogge I thought it fit.    
%\P 1711 ADDISON  \textit{Spect.} No. 60 ⁋11 The double Rhymes, which are used in Doggerel Poetry.    
%\P 1789 BELSHAM  \textit{Ess. I. xii. 233 The vile doggrel translation of Hobbes.    
%\P 1868 STANLEY  \textit{Westm. Abb. v. 397 The doggrel epitaphs which were hung over the royal tombs.

%b.A.b transf. Bastard, burlesque.

%\P 1550 BALE  \textit{Apol. 93 (R.) The diuinite doggerell of that dronken papist Johan Eckius.    
%\P 1873 G. C. DAVIES  \textit{Mount. \& Mere xix. 177 A doggrel form of prayer.

%\itembf{B.} n. Doggerel verse; burlesque poetry of irregular rhythm; bad or trivial verse.

%\P 1630 TINCKER  \textit{of Turvey Ep. Ded. 5 Clownes [have here] plaine dunstable dogrell, for them to laugh at.    
%\P 1710 ADDISON  \textit{Whig Exam. No. 1 ⁋14 He has a happy talent at doggrel.    
%\P 1880 L. STEPHEN  \textit{Pope iii. 71 Chapman..sins..by constantly indulging in sheer doggerel.

%b.B.b A piece of doggerel; a doggerel poem.

%\P 1857 O. A. BROWNSON  \textit{Convert Wks. V. 120 The electioneering campaign of 1840, carried on by doggerels [etc.].    
%\P 1892 ANNE  \textit{Ritchie Rec. Tennyson, etc. iii. vii. 216 A doggerel always had a curious fascination for him [Browning].

%Hence ˈdogg(e)rel v., -ize v., intr. to compose doggerel; trans. to turn into doggerel; ˈdogg(e)reler, -ist, -izer, a writer of doggerel; ˈdogg(e)relism, a doggerel manner of writing.

%\P 1680 R. L'ESTRANGE  \textit{Answ. Litter Libels 9 His Ranging of them Together is a kinde of a Doggrilism.    
%\P 1732  \textit{Gentl. Instructed (ed. 10) 43 (D.) Were I disposed to doggrel it, I would only gloss upon that text.    
%\P 1817  \textit{Monthly Mag. XLIII. 421 The Scotch doggerelist.    
%\P 1821  \textit{Blackw. Mag.} X. 388 The Atys, which..Mr. Lambe has so cruelly doggrelized.    
%\P 1822  \textit{Ibid.} XI. 363 These dabbling doggrelers.    
%\P 1832 SOUTHEY  \textit{Lett. (1856) IV. 259 Some true doggrelizers.    
%\P 1850 READE  \textit{Chr. Johnstone vi. (1853) 65 He had been doggrelling when he ought to have been daubing.



%\end{myenumerate}


%%%%%%%%%%%%%%%%%%%%%%%%%%%%%%%%%
%\myitem{dogmatic} a. and n.

%\noindent \phonetic{(dɒgˈmætɪk)}

%\noindent [ad. L. dogmatic-us (Ausonius), a. Gr. δογµατικός, f. δόγµα, δογµατ- dogma: cf. F. dogmatique (16th c.).]
%\vspace{-0.3cm}

%\begin{myenumerate}

%\itembf{A.} adj.

%\itembf{1.} Pertaining to the setting forth or laying down of opinion; didactic. rare.

%\P 1678 GALE  \textit{Crt. Gentiles III. Pref., To render our Discourse the lesse offensive, we have cast it into a thetic and dogmatic method, rather than agonistic and polemic.    
%\P 1875 JOWETT  \textit{Plato} (ed. 2) V. 5 He is no longer interrogative but dogmatic.

%\itembf{2.} Of, pertaining to, or of the nature of, dogma or dogmas; characterized by or consisting in dogma; doctrinal.

%\P 1706 PHILLIPS  \textit{(ed. Kersey), Dogmatical or Dogmatick, relating to a Dogma, instructive.    
%\P 1727-38 GAY  \textit{Fables ii. xiv. (R.), Dogmatick jargon learnt by heart.    
%\P 1841 W. SPALDING  \textit{Italy \& It. Isl. II. 28 The rest of his compositions are versified treatises of dogmatic theology.    
%\P 1859 MILL  \textit{Liberty ii. (1865) 15 A..Christian in all but the dogmatic sense of the word.    
%\P 1883 FROUDE  \textit{Short Stud. IV. v. 350 No inclination to substitute dogmatic Protestantism for dogmatic Catholicism.

%\itembf{3.} Proceeding upon a priori principles accepted as true, instead of being founded upon experience or induction, as dogmatic philosophy, dogmatic medicine.

%\P 1696 PHILLIPS  \textit{(ed. 5), Dogmatick Philosophy, is that which [ed.
%\P 1706 BEING  \textit{grounded upon sound Principles] positively assures a thing, and is opposed to Sceptic.    
%\P 1823 CRABB  \textit{Technol. Dict., Dogmatic sect (Med.), an ancient sect of physicians, at the head of which is placed Hippocrates.    
%\P 1864 BOWEN  \textit{Logic x. 330 The foundations of all philosophy, whether dogmatic, critical, or sceptical.

%\itembf{4.} Of persons, their writings, etc.: Asserting or imposing dogmas or opinions, in an authoritative, imperious, or arrogant manner.

%\P 1681 tr.  \textit{Willis' Rem. Med. Wks. Vocab., Dogmatic, stiff in opinion.    
%\P 1712 ADDISON  \textit{Spect.} No. 253 ⁋7 Those criticks who write in a positive dogmatick way.    
%\P 1814 D'ISRAELI  \textit{Quarrels Auth. (1867) 458 He wrote against dogmas with a spirit perfectly dogmatic.    
%\P 1868 M. PATTISON  \textit{Academ. Org. v. 306 Not by dogmatic delivery of truths, but by scientific training in the method of enquiry.    
%\P 1873 HELPS  \textit{Anim. \& Mast. viii. (1875) 200 One is afraid of being dogmatic about it, and of being dogmatically wrong.

%b.A.4.b Of assured opinion, convinced. Obs. rare.

%\P 1678 CUDWORTH  \textit{Intell. Syst. 434 (R.) From sundry other places of his writings, it sufficiently appears, that he [Cicero] was a dogmatick and hearty theist.

%\itembf{B.} n.

%\itembf{1.} A philosopher of the dogmatic school; = dogmatist 3. Obs.

%\P a1631 DONNE  \textit{Paradoxes (1652) 22 The Skeptike..was more contentious then..the Dogmatick.    
%\P 1650 HOBBES  \textit{De Corp. Pol. 165 All these Opinions are maintained in the Books of the Dogmaticks, and divers of them taught in Publick Chaires.    
%\P 1702 tr.  \textit{Le Clerc's Prim. Fathers 57 A Suspension [of judgment] suited not with the Dogmaticks, who can hardly confess that they know not all things.

%b.B.1.b A dogmatic physician; see quot. 1883. Obs.

%\P 1605 TIMME  \textit{Quersit. Pref. 5 Among Physitians there are Empericks, Dogmaticks, Methodici, or Abbreuiators, and Paracelsians.    
%\P 1771 T. PERCIVAL  \textit{Med. \& Exp. Ess. (1778) I. 41 (heading) The Dogmatic; or Rationalist.    
%\P 1883 SYD. SOC.  \textit{Lex., Dogmatics, an ancient sect of physicians, so called because they endeavoured to discover, by reasoning, the essence and the occult causes of diseases.

%\itembf{2.} A dogmatic person. Obs.

%\P 1640 HOBBES  \textit{Hum. Nat. xiii. §4 The fault lieth altogether in the dogmatics, that is to say, those that are imperfectly learned, and with passion press to have their opinions pass every where for truth.

%\itembf{3.} Chiefly in pl. form dogmatics: A system of dogma; spec. dogmatic theology.

%\P 1845 GEO. ELIOT in  \textit{Life (1885) 137 ‘Dogmatik’ is the idea, I believe—i.e. positive theology. Is it allowable to say dogmatics, think you?    
%\P 1857 M. PATTISON  \textit{Ess. (1889) II. 222 The Reformation dogmatic rests on..the exclusive sufficiency of Scripture.    
%\P 1858  \textit{Lond. Rev. Oct. 220 To expound the polemical dogmatics of the Reformation.    
%\P 1893 FAIRBAIRN CHRIST in  \textit{Mod. Theol. i. i. i. 29 note, The book ‘De Theologicis Dogmatibus’, published at Paris 1644-50..the first attempt at a scientific history of dogmata, and..notable as suggesting to modern theology the term Dogmatics.    
%\P 1894 MITCHELL tr.  \textit{Harnack's Hist. Dogma i. 28 Dogmatic is a positive science which has to take its material from history.

%Hence dogˈmaticism, dogmatic quality.

%\P 1880 FAIRBAIRN  \textit{Stud. Life Christ ix. (1881) 156 The dogmaticism he subtly concealed.



%\end{myenumerate}


%%%%%%%%%%%%%%%%%%%%%%%%%%%%%%%%%
%\myitem{doleful} a.1 (and n.)

%\noindent \phonetic{(ˈdəʊlfʊl)}

%\noindent [f. dole n.2 + -ful. In ME. found with the variant forms of dole n.2; but doleful has been the standard form since 16th c.]
%\vspace{-0.3cm}

%\begin{myenumerate}

%\itembf{A.} adj. Full of or attended with dole or grief; sorrowful.

%\itembf{1.} Fraught with, accompanied by, or causing grief, sorrow, etc.; distressful, gloomy, dreary, dismal.

%\P c1275 LAY. 6902 Ac hit was a deolful þing: Þat he ne moste leng beo king.    
%\P 1297 R. GLOUC.  \textit{(1724) 237 Þat was a deluol cas.    
%\P a1300  \textit{Cursor M.} 7182 (Gött.) To doleful [v.rr. deleful, deolful] dede þai suld him bring.    
%\P c1420 ANTURS  \textit{of Arth. xiii, Lo! hou dilful dethe hase thi Dame dyȝte!    
%\P c1435 TORR.  \textit{Portugal 521 Torrent toke a dulful wey, Downe in a depe valey.    
%\P c1440  \textit{York Myst.} xxvi. 99 Lord, who schall do þat doulfull dede?    
%\P 1500-20 DUNBAR  \textit{Poems} lxxxi. 23 Scho playit sangis duilfull to heir.    
%\P 1565 T. RANDOLPH in  \textit{Ellis Orig. Lett. Ser. i. II. 202 The deulfull daye of the buriall of her howsbande.    
%\P 1568 TILNEY  \textit{Disc. Mariage D vj, The doolefull place, where he lay.    
%\P 1624 CAPT. SMITH  \textit{Virginia iii. ii. 49 The most dolefullest noyse he ever heard.    
%\P 1667 MILTON  \textit{P.L.} i. 65 Regions of sorrow, doleful shades.    
%\P 1725 POPE  \textit{Odyss.} xxiii. 349 In the doleful mansions he survey'd His royal mother.    
%\P 1847 EMERSON  \textit{Repr. Men, Shaks. Wks. (Bohn) I. 354 Here is..a string of doleful tragedies, merry Italian tales, and Spanish voyages.

%\itembf{2.} Of persons, their state, etc.: Full of pain, grief, or suffering; sorrowful, sad.

%\P c1430 LYDG.  \textit{Thebes iii. (R.) Amphiorax they carry Set in his chaire with a doleful hert.    
%\P a1555 BRADFORD in  \textit{Coverdale Lett. Mart. (1564) 307 For the doulefull bodies of Gods people to reste in.    
%\P 1590 SPENSER  \textit{F.Q.} i. vi. 9 There find the virgin, doolfull, desolate.    
%\P 1647 COWLEY  \textit{Mistress, Heart fled again iii, The doleful Ariadne so, On the wide shore forsaken stood.    
%\P 1829 LYTTON  \textit{Devereux ii. ii, Never presume to look doleful again.

%\itembf{3.} Expressing grief, mourning, or suffering.

%\P c1275 LAY. 1
%\P 1997 HIS  \textit{heorte ne mihte beo sori for þane deolfulle cri.    
%\P 1340 HAMPOLE  \textit{Pr. Consc. 6877 Þai sal duleful crying and sorow here.    
%\P 1393 GOWER  \textit{Conf.} III. 291 In dolfull clothes they hem clothe.    
%\P 1660 F. BROOKE tr.  \textit{Le Blanc's Trav. 104 In signe of mourning: Women..are cloathed in white, the doleful colour there.    
%\P 1797 MRS. RADCLIFFE  \textit{Italian iii. (1824) 550 She would..look up..with such a doleful expression.    
%\P 1865 KINGSLEY  \textit{Herew. xiii, He went to his business with a doleful face.

%\itembf{B.} n. (pl.) A doleful state. colloq. (Cf. dismals.)

%\P 1822 E. NATHAN  \textit{Langreath II. 309 You have enough of the dolefuls at Langreath.    
%\P 1882 M. E. BRADDON  \textit{Mt. Royal II. viii. 149 We shall be in the dolefuls all the year.



%\end{myenumerate}


%%%%%%%%%%%%%%%%%%%%%%%%%%%%%%%%%
%\myitem{dolt} n.

%\noindent \phonetic{(dəʊlt)}

%\noindent [Found with its derivatives from middle of 16th c.; perh. earlier in dialect use. App. related to OE. dol, ME. dol, doll, dull, and to dold, stupid, inert of intellect or faculty. For the -t, cf. ME. dult in sense of dulled: see dull v.]
%\vspace{-0.3cm}

%\begin{myenumerate}

%\itembf{1.} A dull, stupid fellow; a blockhead, numskull.

%\P 1543 [IMPLIED in  \textit{doltish].    
%\P 1551 ROBINSON tr.  \textit{More's Utop. (Arb.) 39 Thies wysefooles and verye archedoltes.    
%\P a1553 UDALL  \textit{Royster D. iii. ii. (Arb.) 42 A very dolt and loute.    
%\P 1604 SHAKES.  \textit{Oth.} v. ii. 163 Oh Gull, oh dolt, As ignorant as durt.    
%\P 1658 CLEVELAND  \textit{Rustic Rampant Wks. (1687) 417 Not only these Doults, these Sots.    
%\P 1725 SWIFT  \textit{Wood the Ironmonger 32 Wood's adulterate copper, Which..we like dolts Mistook at first for thunderbolts.    
%\P 1847 DISRAELI  \textit{Tancred v. i, The prerogative of dolts and dullards.

%2. \itembf{a.} attrib. or as adj. Doltish, stupid, senseless, foolish. \itembf{b.} Comb., as dolt-head, (a) a dolt, blockhead; (b) a stupid head (quot. 1711).

%\P 1679 DRYDEN  \textit{Troil. \& Cress. ii. iii, Dolt-heads, asses, And beasts of burden.    
%\P 1711 E. WARD  \textit{Quix. I. 414 As soon as each had bolted From out his Straw, and scratch'd his Dolthead.    
%\P 1828 SOUTHEY  \textit{To A. Cunningham Poems III. 311 The dolt image is not worth its clay.    
%\P 1852 R. KNOX  \textit{Gt. Artists \& Anat. 57 North Germany, the land of schnapps, and insolence, and dolt stupidity.

%Hence ˈdoltage, ˈdoltry, the condition of a dolt; ˈdoltify v. trans., to make a dolt of.

%\P 1559 J. AYLMER  \textit{Harbor. Faithf. Subj. G iij b, Women..doltefied with the dregges of the Deuils dounge hill.    
%\P 1581 MULCASTER  \textit{Positions xxxix. (1887) 205 Where I see nobilitie betraid to donghillrie, and learning to doultrie.    
%\P 1593 NASHE  \textit{Four Lett. Confut. G j b, I have usually seene uncircumsised doltage have the porch of his Panims pilfries very hugely pestred with praises.



%\end{myenumerate}


%%%%%%%%%%%%%%%%%%%%%%%%%%%%%%%%%
%\myitem{don} v.1 arch.

%\noindent \phonetic{(dɒn)}

%\noindent [contracted from do on: see do v. 48.
%\vspace{-0.3cm}

%\begin{myenumerate}
%\P 1650 RETAINED in  \textit{popular use only in north. dial.; as a literary archaism it has become very frequent in 19th c.]

%\itembf{1.} trans. To put on (clothing, anything worn, etc.). The opposite of doff.

%\P 1567 TURBERV.  \textit{Ovid's Ep. 109 b, Do'n hornes And Bacchus thou shalt be.    
%\P 1602 SHAKES.  \textit{Ham.} iv. v. 52 Then vp he rose, \& don'd his clothes.    
%\P 1613-16 W. BROWNE  \textit{Brit. Past. ii. iv. (R.), In Autumne..when stately forests d'on their yellow coates.    
%\P 1621 QUARLES  \textit{Argalus \& P. (1678) 84 Up Argalus, and d'on thy Nuptial weeds.    
%\P a1764 LLOYD  \textit{Henriade (R.), Mars had donn'd his coat of mail.    
%\P 1828 SCOTT  \textit{F.M. Perth} vi, My experience has been in donning steel gauntlets on mailed knights.    
%\P 1861 T. A. TROLLOPE  \textit{La Beata II. xii. 61 To shut up his studio, and don his best coat.    
%\P 1879 DIXON  \textit{Windsor I. iii. 23 She donned the garment of a nun.

%\itembf{2.} transf. To dress (a person) in a garment; refl. to dress oneself. Chiefly north. dial.

%\P 1801 R. ANDERSON  \textit{Cumberld. Ball. 17 Sae doff thy clogs, and don thysel.    
%\P 1845 E. BRONTë  \textit{Wuthering Heights xix, Joseph was donned in his Sunday garments.

%Hence ˈdonning vbl. n.

%\P 1847 EMERSON  \textit{Poems (1857) 161 Too much of donning and doffing.    
%\P 1888 ELWORTHY  \textit{W. Somerset Word-bk., Donnings, Sunday clothes, also finery.



%\end{myenumerate}


%%%%%%%%%%%%%%%%%%%%%%%%%%%%%%%%%
%\myitem{dormant} a. and n.

%\noindent \phonetic{(ˈdɔːmənt)}

%\noindent [a. OF. dormant (12th c. in Hatz.-Darm.), pr. pple. of dormir:—L. dormīre to sleep.]
%\vspace{-0.3cm}

%\begin{myenumerate}

%\itembf{A.} adj.

%\itembf{1.} Sleeping, lying asleep or as asleep; hence, fig. intellectually asleep; with the faculties not awake; inactive as in sleep.

%\P 1623 COCKERAM,  \textit{Dormant, sleeping.    
%\P 1640 G. WATTS tr.  \textit{Bacon's Adv. Learn. Pref. 16 If we have bin too credulous, or too dormant.    
%\P 1681 GREW  \textit{Musæum (J.), His prey, for which he lies, as it were, dormant, till it swims within his reach.    
%\P 1726 ADV.  \textit{Capt. R. Boyle 285 That he only lay dormant to meditate some Mischief to me.    
%\P 1858 HAWTHORNE  \textit{Fr. \& It. Jrnls. I. 132 Some Romans were lying dormant in the sun.    
%\P 1869 FARRAR  \textit{Fam. Speech iii. (1873) 104 The hitherto dormant members of the Aryan family.

%b.A.1.b Of animals: With animation suspended.

%\P 1772 FORSTER in  \textit{Phil. Trans. LXII. 378 It lies dormant the greater part of the winter.

%c.A.1.c Of plants: With development suspended.

%\P 1863 BERKELEY  \textit{Brit. Mosses ii. 5 In dry weather they [Mosses] are often completely dormant.    
%\P 1882 VINES  \textit{Sachs' Bot. 640 The numerous dormant buds of woody plants may long remain buried and yet retain their vitality.    
%\P 1883 SYD. SOC.  \textit{Lex., Dormant bud, a bud which remains, it may be for years, undeveloped on a plant stem.

%d.A.1.d Her. Represented in a sleeping or recumbent attitude; with the head resting on the paws.

%\P c1500  \textit{Sc. Poem Heraldry 130 in Q. Eliz. Acad. etc. 98 xv maneris of lionys in armys..the viij dormand.    
%\P 1646 SIR T. BROWNE  \textit{Pseud. Ep.} v. x. 248 Yet were it not probably a Lyon Rampant..but rather couchant or dormant.    
%\P 1766 ENTICK  \textit{London IV. 82 At his foot a cupid dormant.    
%\P 1851 R. R. MADDEN  \textit{Shrines \& Sepulchres II. 37, I would rather call the ancient figures dormant.

%\itembf{2.} In a state of rest or inactivity; quiescent; not in motion, action, or operation; ‘slumbering’, in abeyance.

%\P 1601 HOLLAND  \textit{Pliny II. 597 This riuer runneth but slowly, and seemeth a dead or dormant water.    
%\P 1639 EARL OF BARRYMORE in  \textit{Lismore Papers Ser. ii. (1888) IV. 39 Your lordshipps directions..must lye dormant by me.    
%\P 1708 SWIFT  \textit{Abolit. Chr. Wks.
%\P 1755 II. I. 85  \textit{What if there be an old dormant statute or two against him, are they not now obsolete to a degree?    
%\P 1731  \textit{ Pulteney Ibid. IV. i. 166 Thy dormant ducal patent.    
%\P 1766 FORDYCE  \textit{Serm. Yng. Wom. (1767) I. vi. 257 It is possible for original talents to lie dormant.    
%\P 1792 N. CHIPMAN  \textit{Amer. Law Rep. (1871) 21 Plaintiffs who have since revived a dormant claim.    
%\P 1806 GAZETTEER  \textit{Scot. (ed. 2) 390 Newark..formerly gave title of Baron to the family of Leslie, now dormant.    
%\P 1878 HUXLEY  \textit{Physiogr. 203 Many volcanoes..are merely dormant.

%b.A.2.b dormant commission, dormant credit, dormant warrant, dormant writing, etc., one drawn out in blank to be filled up with a name or particulars, when required to be used; dormant partner, a ‘sleeping’ partner, who takes no part in the working of a concern.

%\P 1551 HOUSEH. ACC. ELIZ. in  \textit{Camden Misc. 34 Paid..unto James Russell, by warrante dormaunte..xx. s.    
%\P c1614 CORNWALLIS in  \textit{Gutch Coll. Cur. I. 148 The warrant dormant, which all Leiger Ambassadors have, to propound and discourse of all things, which they think may tend to the encreasing of amity.    
%\P 1662 MARVELL  \textit{Corr.} xxxv. Wks.
%\P 1872-5 II. 80 THAT  \textit{you would send us up a dormant credit for an hundred pound.    
%\P 1679-88 SECR.  \textit{Serv. Money Chas. \& Jas. (Camden) 101 For charge of passing a dormant privy seale, 12li 8s, and of dormant l'res patents, 30li 2s 2d.    
%\P 1714 SWIFT  \textit{Pres. St. Affairs Wks.
%\P 1755 II. I. 221  \textit{A power was given of chusing dormant viceroys.    
%\P 1716 ADDISON  \textit{Freeholder} 36 (Seager) He likewise signed a dormant commission for another to be his high admiral.    
%\P 1845 STEPHEN  \textit{Comm. Laws Eng. (1874) II. 102 Partners thus unknown to the public are said to be dormant.

%c.A.2.c Mechanics.
%   dormant-bolt, ‘a concealed bolt working in a mortise in a door, and usually operated by a key; sometimes by turning a knob’; dormant-lock, ‘a lock having a bolt that will not close of itself’ (Knight Dict. Mech.).

%\itembf{3.} Fixed, stationary. dormant tree = B. 1.

%\P c1440  \textit{Promp. Parv.} 127/2 Dormawnte tre..trabes.    
%\P 1703 T. N. CITY  \textit{\& C. Purchaser 128 Dormant tree. In Architecture is a great Beam lying cross a House, otherwise call'd a Summer.    
%\P 1793 SMEATON  \textit{Edystone L. §238 The dormant wedge or that with the point upward, being held in the hand, while the drift wedge or that with its point downward, was driven with a hammer.    
%\P 1798 TERM  \textit{Rep. VII. 599 To the sleepers or dormant timbers they affixed railways or waggon ways.    
%\P 1876 GWILT  \textit{Archit. Gloss., Dormant-tree or Summer.

%b.A.3.b dormant table, a table fixed to the floor, or forming a fixed piece of furniture. arch.

%\P c1386 CHAUCER  \textit{Prol. 353 His table dormant in his halle alway Stood redy couered al the longe day.    
%\P 1430 LYDG.  \textit{Chron. Troy ii. xi, Eke in the hall..On eche partye was a dormaunt table.    [
%\P 1448 INV. T. MORTON in  \textit{Test. Ebor. III. 108 De ij mensis vocatis dormoundes.]    
%\P 1610 B. JONSON  \textit{Alch. v. v, Were not the pounds told out..vpon the table dormant.    
%\P 1767 BLACKSTONE  \textit{Comm.} II. xxviii. 428 Whatever is strongly affixed to the freehold or inheritance..as marble chimney-pieces, pumps, old fixed or dormant tables, benches, and the like.    
%\P 1851 TURNER  \textit{Dom. Archit. I. ii. 54.

%\P a1635 NAUNTON  \textit{Fragm. Reg. (Arb.) 24 She held a dormant Table in her own Princely breast.

%\itembf{4.} Causing or producing sleep. Obs. rare.

%\P 1654 tr.  \textit{Scudery's Curia Pol. 66 The effects of Dormant and Narcotique remedies.

%\itembf{5.} dormant window, also dormant = dormer 2.

%\P 1651 CLEVELAND  \textit{Senses' Fest. ii, Old Dormant Windows must confess Her Beams.    
%\P 1727-51 CHAMBERS  \textit{Cycl.,} Dormer or Dormant, in architecture, denotes a window made in the roof of an house.    
%\P 1804  \textit{Ann. Reg. 829 A dormant must break out in the roof.    
%\P 1823 J. F. COOPER  \textit{Pioneer x, The dormant windows in the roof.

%\itembf{B.} n.

%\itembf{1.} A fixed horizontal beam; a sleeper; a summer. More fully dormant tree (see A. 3). Obs.

%\P 1453 PASTON  \textit{Lett. No. 185 I. 250 Sir Thomas Howes hath purveyed iiij. dormants for the drawte chamer, and the malthouse, and the browere.    
%\P 1582 WILLS \& INV.  \textit{N.C. (Surtees 1860) 46 In the hay barne..Certaine sawen baulkes, viz., ix dormonds and j sile 10s.    
%\P 1587 HARRISON  \textit{England ii. xii. (1877) i. 233 Summers (or dormants).    
%\P 1665 VESTRY  \textit{Bks. (Surtees) 201, 2 clasps of iron for fastning the great dormond in the church, 6 s.

%b.B.1.b The part between the opening and the top of a doorway; the tympanum. Obs. rare.

%\P 1723 CHAMBERS tr.  \textit{Le Clerc's Treat. Archit. I. 102 Coach-Gates..have a Dormant (i.e. the upper part of the Gate that does not open), which Dormant, where the Gate is arch'd, commences from the Spring of the Arch.

%\itembf{2.} = dormer window: see A. 5.

%\itembf{3.} A dish which remains on the table throughout a repast; a centre-piece which is not removed.

%\P 1845 J. BREGION  \textit{Pract. Cook 25 (Stanf.) A centre ornament, whether it be a dormant, a plateau..or a candelabra.



%\end{myenumerate}


%%%%%%%%%%%%%%%%%%%%%%%%%%%%%%%%%
%\myitem{dossier} n.

%\noindent \phonetic{(ˈdɒsɪə(r), ˈdɒsɪeɪ, ˈdɒsjeɪ)}

%\noindent [a. F. dossier, in sense ‘bundle of papers’, which from their bulging are likened to a back (dos): see dosser1.]
%\vspace{-0.3cm}

%\begin{myenumerate}

%A bundle of papers or documents referring to some matter; esp. a bundle of papers or information about a person.

%\P 1880  \textit{Contemp. Rev.} 992 The dossiers of the electioneering agent.    
%\P 1884  \textit{Pall Mall Gaz. 13 June 11/2 In neatly-docketed cabinets round his office stood the dossiers of all the criminals with whom he has had anything to do for the past eight years.    
%\P 1885  \textit{Spectator} 8 Aug. 1040/2 A part of the Great Hastings dossier, the case against Sir Elijah Impey.    
%\P 1912 H. BELLOC  \textit{Servile State ix. 176 A series of dossiers by which the record of each workman can be established.    
%\P 1920 ‘SAPPER’  \textit{Bulldog Drummond xii. 300 Here's his dossier..‘Ditchling, Charles. Good speaker; clever; unscrupulous. Requires big money; worth it. Drinks.’    
%\P 1939 M. SPRING  \textit{Rice Working-class Wives ii. 25 Questionnaires filled in by women of a better..position... Such dossiers would have served as ‘controls’.    
%\P 1955 BULL.  \textit{Atomic Sci. Apr. 129/3 A file check of government dossiers.    
%\P 1967 A. S. NEILL  \textit{Talking of Summerhill x. 125, I guessed they had rung up our Home Office to ask for my dossier.



%\end{myenumerate}


%%%%%%%%%%%%%%%%%%%%%%%%%%%%%%%%%
%\myitem{doughty} a.

%\noindent \phonetic{(ˈdaʊtɪ)}

%\noindent [The original OE. form was dyhtiᴁ, corresp. to OHG. *tuhtîg, MHG. tühtec, Ger. tüchtig, MDu. and MLG. duchtich, from an OTeut. n. *duhti-z, MHG. tuht ability, capacity, from dugan: see dow v.1 (If this had come down, its mod.Eng. repr. would be dighty.) OE. dohtiᴁ was a later formation, of which the vowel is difficult to explain, unless perh. by assimilation to dohte, pa. tense of duᴁan. It came down in the ME. doȝti, dohty, dowghty, Sc. dochtie, douchtie, to the mod. spelling doughty, of which the expected pronunciation would be (ˈdɔːtɪ): cf. bought, wrought, daughter. Beside it, ME. had duhtiȝ, duȝti, duhti, 16th c. Sc. duchtie; and also from 14th c., dowtie, douty, erroneously spelt (by assimilation to another word of same sound) doubty; whence evidently the current spoken word (ˈdaʊtɪ). The phonology presents many points of difficulty.]
%\vspace{-0.3cm}

%\begin{myenumerate}

%\itembf{1.} Able, capable, worthy, virtuous; valiant, brave, stout, formidable: now with an archaic flavour, and often humorous. \itembf{a.} of persons.

%\P 1030 ABINGDON  \textit{Chron., Hacun se dohtiᴁa eorl.    
%\P c1200 ORMIN 113 Zacariȝe..haffde an duhhtiȝ wif..Elysabæþ ȝehatenn.    
%\P 1297 R. GLOUC.  \textit{(1724) 592 Edward, that doughty knyght.    
%\P a1300  \textit{Cursor M.} 3555 (Cott.) Sir Ysaac þat dughti [Gött. dohuti] man.    
%\P c1314 GUY Warw. (A.)
%\P 1480 A  \textit{duhtti kniȝt and no coward.    
%\P 1375 BARBOUR  \textit{Bruce ii. 166 For all his eldris war douchty.    
%\P c1380  \textit{Sir Ferumb.} 423 Doȝty men \& wiȝt.    
%\P c1420 AVOW.  \textit{Arth. xiv, Did as a duȝty knyȝte.    
%\P c1440  \textit{York Myst.} xxxviii. 163 Sir knyghtis, þat are in dedis dowty.    
%\P 1480 CAXTON  \textit{Chron. Eng. lxxiii. 55 Kyng Arthur was..bolde and doubty of body.    
%\P 1535 STEWART  \textit{Cron. Scot. (1858) I. 42 Lord and knycht..And mony other richt duchtie and conding.    
%\P 1600 HOLLAND  \textit{Livy} xxiv. xlvi. 541 Certaine Tribunes and marshals, valourous and doubtie good men.    
%\P 1609  \textit{ Amm. Marcell. xiv. ix. 19 A doutie warrior.    
%\P 1655 FULLER  \textit{Ch. Hist.} iii. vi. §50 All the Scotish Nobility (Doughty Douglas alone excepted).    
%\P 1795 SOUTHEY  \textit{Joan of Arc v. 126 The doughty Paladins of France.    
%\P 1814 D'ISRAELI  \textit{Quarrels Auth. (1867) 263 The doughty critic was at once silenced.    
%\P 1847 LEWES  \textit{Hist. Philos. (1867) II. 98 Oxford called upon her doughty men to brighten up their arms.    
%\P 1848 DICKENS  \textit{Dombey} (C.D. ed.) 115 Nor did he ever again face the doughty Mrs. Pipchin.

%\itembf{b.} of actions, and other things.

%\P 1287 (Z.)  \textit{Sweord ecᴁum dyhtiᴁ.    
%\P a1000 CæDMON'S Genesis
%\P 1993 SWEORD  \textit{ecᴁum dihtiᴁ.]    
%\P a1225  \textit{Leg. Kath. 782 Of mine bileaue, beo ha duhti oðer dusi, naue þu nawt to donne.    
%\P a1300  \textit{Cursor M.} 2112 (Cott.) Mani contre þarin es And dughti cites mare and lesse.    
%\P 1393 LANGL.  \textit{P. Pl.} C. viii. 141 Of thyne douhtieste dedes.    
%\P 1535 STEWART  \textit{Cron. Scot. II. 510 Of his duchtie Deidis and Justice done.    
%\P 1568 T. HOWELL  \textit{Arb. Amitie (1879) 81 Nor men deserue the crowne, and doubtie diademe.    
%\P 1590 SPENSER  \textit{F.Q.} i. v. 1 How that doughtie turnament With greatest honour he atchieven might.    
%\P a1667 JER. TAYLOR  \textit{Serm. for Year (1678) Suppl. 185 In this doughty cause they think it fit to fight and die.    
%\P 1733 CHEYNE  \textit{Eng. Malady iii. iv. (1734) 302 Another doughty Objection against a Vegetable Diet, I have heard.    
%\P 1829 SCOTT  \textit{Jrnl. 28 Apr., After this doughty resolution, I went doggedly to work.

%\itembf{2.} absol. = Man or men of valour. Obs.

%\P c1420 ANTURS  \textit{of Arth. i, Bothe the kyng and the qwene And other doȝti by-dene.    
%\P c1475  \textit{Rauf Coilȝear 590 Thair wald na douchtie this day for Iornay be dicht.    
%\P 1800 A. CARLYLE  \textit{Autobiog. 140, I..was going up the field to tell this when my doughty arrived.

%\itembf{3.} Comb., as doughty-handed adj.

%\P 1606 SHAKES.  \textit{Ant. \& Cl.} iv. viii. 5 Doughty handed are you.



%\end{myenumerate}


%%%%%%%%%%%%%%%%%%%%%%%%%%%%%%%%%
%\myitem{dour} a. orig. Sc.

%\noindent \phonetic{(duːr)}

%\noindent [ad. L. dūr-us, or F. dur hard (cf. dure).
%\vspace{-0.3cm}

%\begin{myenumerate}
%   Derivation from French is unlikely on account of the vowel, since F. u gives in Sc. not ū but ü (or ö). An early (11th or 12th c.) adoption of L. dūr-us, would suit phonetically; of this however we have no evidence.]

%\itembf{1.} Hard, severe, bold, stern, fierce, hardy.

%\P 1375 BARBOUR  \textit{Bruce x. 170 [He] wes dour \& stout.    
%\P c1425 WYNTOUN  \textit{Cron.} viii. xvi. 103 Dyntis dowre ware sene.    
%\P 1513 DOUGLAS  \textit{Æneis ii. vi. [v.] 23 The dour Vlixes als, and Athamas.    
%\P 1533 BELLENDEN  \textit{Livy ii. (1822) 166 Thir legatis wes gevin ane doure answere be Marcius.    
%\P 1596 DALRYMPLE tr.  \textit{Leslie's Hist. Scot. iv. 249 He led a dour and hard lyfe.    
%\P 1794 BURNS  \textit{Winter Night i, Biting Boreas, fell and doure.    
%\P 1848 LYTTON  \textit{Harold vi. i, Tostig is a man..dour and haughty.    
%\P 1891 ATKINSON  \textit{Moorland Par. 261 The dour, merciless intensity of a northern moorland..storm.

%\itembf{2.} Hard to move, stubborn, obstinate, sullen.

%\P c1470 HENRY  \textit{Wallace} iv. 187 Malancoly he was of complexioun..dour in his contenance.    
%\P 1513 DOUGLAS  \textit{Æneis xiii. vi. 106 All our prayeris..Mycht nowder bow that dowr mannis mynd.    
%\P 1572  \textit{Satir. Poems Reform. xxxviii. 76 Our men are dour men.    
%\P 1816 SCOTT  \textit{Old Mort.} viii, ‘He's that dour, ye might tear him to pieces, and..ne'er get a word out o' him.’    
%\P 1854 MRS. GASKELL  \textit{North \& S. xvii, Thornton is as dour as a door-nail; an obstinate chap.

%Hence ˈdourly adv., with hard sternness, stubbornly, obstinately; ˈdourness, hardness of disposition, obstinacy, sullenness.

%\P c1375  \textit{Sc. Leg. Saints, Jacobus minor 337 Thai..in to durnes ay abad.    
%\P c1475  \textit{Rauf Coilȝear 918 To ding thame doun dourly that euer war in my way.    
%\P 1596 DALRYMPLE tr.  \textit{Leslie's Hist. Scot. v. 281 And fercely had fochtne thame, and dourlie dantount.    
%\P 1871 C. GIBBON  \textit{Lack of Gold iv, ‘Give me those letters, father’, she said dourly.    
%\P 1882  \textit{Sat. Rev.} No. 1411. 629 Scotchmen..have the same caution..courage, and ‘dourness’ [as Yorkshiremen].



%\end{myenumerate}


%%%%%%%%%%%%%%%%%%%%%%%%%%%%%%%%%
%\myitem{doyen} n.

%\noindent \phonetic{(dwaj‹opetilde›)}

%\noindent [F. doyen:—L. decān-us dean. In sense 1 from OF.; in sense 2 anew from mod.French.]
%\vspace{-0.3cm}

%\begin{myenumerate}

%\itembf{1.} A leader or commander of ten. Obs.

%\P 1422 tr.  \textit{Secreta Secret., Priv. Priv. (E.E.T.S.) 214 Euery ledere [had] ten doiens, and..euery doiens ten men.

%\itembf{2.} The senior member of a body. = dean1 10. Cf. doyenne2.

%\P 1670 COTTON  \textit{Espernon ii. v. 242 This was he..that was afterwards Doyen to the Council of State.    
%\P 1883  \textit{Pall Mall G. 12 Nov. 3/2 A member of the Royal Danish Academy of Arts, of which he died the doyen.    
%\P 1886  \textit{Ibid.} 23 Sept. 3 The doyen of the Russian press.



%\end{myenumerate}


%%%%%%%%%%%%%%%%%%%%%%%%%%%%%%%%%
%\myitem{Draˈconian} a.

%\noindent \phonetic{[f. as Draconic + -ian.]}

%\noindent = Draconic 1, 2.
%\vspace{-0.3cm}

%\begin{myenumerate}

%\P 1876 C. M. DAVIES  \textit{Unorth. Lond. 97 The Swedenborgian rubrics are not so Draconian.    
%\P 1877 D. M. WALLACE  \textit{Russia xiii. 206 Refraining from all Draconian legislation.    
%\P 1880  \textit{Daily Tel.} 10 Nov., In the course of one of these draconian performances..the mummer's tail came off.

%Hence Draˈconianism.

%\P 1819 GIFFORD in  \textit{Smiles J. Murray I. 404, I never much admired the vaunt of Draconianism, ‘And all this I dare do, because I dare’.



%\end{myenumerate}


%%%%%%%%%%%%%%%%%%%%%%%%%%%%%%%%%
%\myitem{Draconic} a.

%\noindent \phonetic{(drəˈkɒnɪk)}

%\noindent [f. L. draco, -ōnem, ad. Gr. δράκων dragon, also f. the Greek personal name, Δράκων, Draco: see -ic.]
%\vspace{-0.3cm}

%\begin{myenumerate}

%\itembf{1.} Of, pertaining to, or characteristic of Draco, archon at Athens in 621 b.c., or the severe code of laws said to have been established by him; rigorous, harsh, severe, cruel.

%\P 1708 MOTTEUX  \textit{Rabelais v. xi. (1737) 43 Any Law so rigorous and Draconic.    
%\P 1872 YEATS  \textit{Growth Comm. 35 Their criminal code, which was Draconic in severity.

%\itembf{2.} Pertaining to, or of the nature of, a dragon.

%\P 1680 H. MORE  \textit{Apocal. Apoc. 118 ‘The great Dragon was cast out’..This..signified the destruction of the Empire as Draconick and Idolatrous.    
%\P 1791 tr.  \textit{Swedenborg's Apoc. Rev. xiv. §655 To whom the draconic spirit addressed the same words.    
%\P 1820 SCOTT  \textit{Abbot xv, ‘Marry come up—are you there with your bears?’ muttered the dragon, with a draconic silliness.

%\itembf{3.} Astron. = dracontic.
%   (Sometimes erroneously explained as ‘Relating to the constellation Draco’.)

%\P 1876 G. CHAMBERS  \textit{Astron. ii. i. 174 This is termed a ‘nodical revolution of the Moon.’ note. Sometimes the Draconic Period.



%\end{myenumerate}


%%%%%%%%%%%%%%%%%%%%%%%%%%%%%%%%%
%\myitem{droll} a.

%\noindent \phonetic{[f. F. drôle: see prec. n.]}

%\noindent 1.1 Intentionally facetious, amusing, comical, funny. droll painting, caricature; d. painter, caricaturist.
%\vspace{-0.3cm}

%\begin{myenumerate}

%\P 1623 JAS. I in  \textit{Four C. Eng. Lett. 45, I heartily thank thee for thy kind droll letter.    
%\P 1756-82 J. WARTON  \textit{Ess. Pope (ed. 4) I. ii. 51 Landschape-painting..being even preferred to single portraits, to pieces of still-life, to droll-figures.    
%\P 1762-71 H. WALPOLE  \textit{Vertue's Anecd. Paint. (1786) III. 45 Daniel Boon, Of the same country, a droll painter.    
%\P 1789 BELSHAM  \textit{Ess. I. x. 202 The droll inventions of Hogarth.    
%\P 1858 LYTTON  \textit{What will He do i. xii, He was a droll and joyous humourist.    
%\P 1861 WRIGHT  \textit{Ess. Archæol. II. xxiii. 230 Everybody has a perception of what is droll and ludicrous.

%\itembf{2.} Unintentionally amusing; queer, quaint, odd, strange, ‘funny’.

%\P 1753 W. MELMOTH  \textit{Cicero iv. ix. (R.) Imitating the droll figures those gallant youths exhibited.    
%\P 1790 BURNS  \textit{Tam O'Shanter 159 Wither'd beldams, auld and droll.    
%\P 1822 SCOTT LET. in  \textit{Taylor \& Raine Mem. Surtees (1852) 164, I have built a droll sort of house here..a pretty, though somewhat fantastical residence.    
%\P a1876 G. DAWSON  \textit{Biog. Lect. (1886) 94 Charles the Second certainly was the drollest idol ever nation set up.

%Hence ˈdrollity, the quality of being droll; concr. a droll thing; ˈdrollness.

%\P 1639 DAVENANT  \textit{Salmacida Spolia Dram. Wks.
%\P 1872 II. 317  \textit{Four Grotesques or drollities.    
%\P 1823 F. CLISSOLD  \textit{Ascent Mt. Blanc (1825) 10 Excited, as he said, by the drollness of the scene.    
%\P 1885 LIBRARY  \textit{Mag. (N.Y.) July 4 The ground-cuckoo is an embodiment of drollness and absurdity.



%\end{myenumerate}


%%%%%%%%%%%%%%%%%%%%%%%%%%%%%%%%%
%\myitem{dudgeon} n.1

%\noindent \phonetic{(ˈdʌdʒən)}

%\noindent [Occurs as digeon in AF.: the form of the word suggests a French origin; but no corresp. word has been found in continental French.]
%\vspace{-0.3cm}

%\begin{myenumerate}

%\itembf{1.} A kind of wood used by turners, esp. for handles of knives, daggers, etc. Obs.
%\P 1597 =  \textit{boxwood. The same sense has been attributed to dudgin in the following quot. from Holland's Pliny, where however the Latin is obscure, and the English a very rude rendering of it.)

%\P 1601 HOLLAND  \textit{Pliny xvi. xvi, Now for the Box tree, the wood thereof is in as great request as the very best: seldom hath it any grain crisped damask-wise, and neuer but about the root, the which is dudgin and ful of work. For otherwise the grain runneth streight and euen without any wauing. [Pliny: In primis vero materies honorata buxo est raro crispanti nec nisi radice, de cetero lenis quies est materiæ silentio quodam et duritie ac pallore commendabilis, in ipsa vero arbore topiario opere.]

%\P 1380 ORDINANCE  \textit{for Cutlers, Lond., in Lett. Bk. H. lf. cxviii, Qe nulles manches darbre forsqe digeo‹nmac› soyent colourez. [tr. in Riley Mem. London (1868) 439 No handle of wood, except dogeon.]    
%\P 1439  \textit{Test. Ebor. (Surtees) III. 96 De j dagger, cum manubrio de dogeon.    
%\P 1443  \textit{Ibid.} II. 88 Unum par cultellorum cum manubrio de dugion.    
%\P c1440  \textit{Promp. Parv.} 436/2 Ronnyn, as dojoun, or masere, or oþer lyke.    
%\P 1502 ARNOLDE  \textit{Chron. (1811) 245 All my stuf beyng in my [Cutler's] shoppe, that is to saye, yuery, dogeon, horn, mapyll.    
%\P 1535 in  \textit{Maddison Linc. Wills (1888) 11 A pare of beads of dogeon.    1550-
%\P 1600 CUSTOMS  \textit{Duties (B.M. Add. MS. 25097) Dogen, the c peces containing vxx xs.    
%\P 1562 TURNER  \textit{Herbal ii. 71 b, The wilde ashe..can scarsly be knowen from dudgyon and I thynke that the moste parte of dogion is the root of the wilde ashe.    
%\P 1597 GERARDE Herbal (1633)
%\P 1225 (L.)  \textit{Turners and cutlers..doe call this woode [box woode] dudgeon, wherewith they make dudgeon-hefted daggers.    
%\P 1660  \textit{Act 12 Chas. II, c. 4 Sched., Dudgeon the hundred peeces cont. five score, j. li.

%\itembf{2.} The hilt of a dagger, made of this wood: cf. dudgeon-haft in 4. Obs.

%\P 1605 SHAKES.  \textit{Macb.} ii. i. 46, I see..on thy Blade, and Dudgeon, Gouts of Blood.

%\itembf{3.} Hence dudgeon-dagger, and in later use dudgeon: A dagger with a hilt made of ‘dudgeon’; also, a butcher's steel. arch.

%\P 1581 J. BELL  \textit{Haddon's Answ. Osor. 10 b, Upon the whiche when you rushe with your doodgean daggar eloquence.    
%\P 1590 GREEN  \textit{Wks. (1882) VIII. 199 Loose in the haft like a dudgin dagger.    
%\P a1687 COTTON  \textit{Poet. Wks. (1765) 83 With Dudgeon Dagger at his Back.    
%\P 1826 SCOTT  \textit{Woodst. vii, Bid me give him three inches of my dudgeon-dagger.

%\P 1638  \textit{Brome Antipodes v. v. Wks.
%\P 1873 III.  \textit{328 Take your dudgeon, Sir, I ha done you simple service.    
%\P 1663 BUTLER  \textit{Hud. i. i. 379 It was a serviceable Dudgeon, Either for fighting or for drudging.    Ibid. ii. 769 That Wight With gauntlet blue and Bases white And round blunt Dudgeon [some later edd. truncheon].    
%\P 1837 CARLYLE  \textit{Fr. Rev.} II. iii. v, And still the dudgeon sticks from his left lapelle.    
%\P 1882 SHORTHOUSE  \textit{J. Inglesant (ed. 2) II. xix. 372.

%\itembf{4.} attrib. and Comb., as dudgeon-knife; dudgeon-dagger: see 3; dudgeon-haft, the hilt of a dagger, made of ‘dudgeon’; hence dudgeon-hafted a. (arch.); dudgeon-tree = 1.

%\P 1559  \textit{Will of J. Gryffyn (Somerset Ho.), My dagger wt the *dudgen hafte gilte.    
%\P 1611 COTGR.,  \textit{Dague a roëlles, a Scottish dagger; or Dudgeon haft dagger.    
%\P a1612 HARINGTON  \textit{Epigr. iv. 11 A gilded blade hath oft a dudgen haft.    
%\P 1634-5 BRERETON  \textit{Trav. (1844) 108 [I] bought in Edinburgh..a dudgeon-hafted dagger, and knives, gilt.    
%\P 1816 SCOTT  \textit{Old Mort.} xxxvi, I'll dash your teeth out with my dudgeon-haft!

%\P 1841 BORROW  \textit{Zincali (1872) 213 I'd straight unsheath my *dudgeon knife And cut his weasand through.    
%\P 1861 THORNBURY  \textit{True as Steel (1863) III. 20 Cutting out the heavy lead window frame with a short heavy dudgeon-knife.

%\P 1551 ABERDEEN  \textit{Reg. V. 21 (Jam.) Certane *dugeon tre coft be him.

%\P 1602 DEKKER Satirom. Wks.
%\P 1873 I. 195,  \textit{I am too well rancht..to bee stab'd With his *dudgion wit.



%\end{myenumerate}


%%%%%%%%%%%%%%%%%%%%%%%%%%%%%%%%%
%\myitem{duplicity} n.

%\noindent \phonetic{(djuːˈplɪsɪtɪ)}

%\noindent [a. F. duplicité (13th c.), ad. L. duplicitāt-em, n. of quality f. duplex, duplic-em: see duplex.]
%\vspace{-0.3cm}

%\begin{myenumerate}

%\itembf{1.} The quality of being ‘double’ in action or conduct (see double a. 5); the character or practice of acting in two ways at different times, or openly and secretly; deceitfulness, double-dealing. (The earliest and still the most usual sense.)

%\P c1430 LYDG.  \textit{Min. Poems 165 (Mätz.) In symulacioune is false duplicite.    
%\P 1503 HAWES  \textit{Examp. Virt. v. (Arb.) 19 Wo worth the man full of duplycyte.    
%\P 1597 J. PAYNE  \textit{Royal Exch. 14 Suche ys the choyce that these make of duplicitie and hypocrisie.    
%\P 1650 BULWER  \textit{Anthropomet. 143 Whether this Duplicity of Tongue be in them Lusus Naturæ, or a meer Device of Art.    
%\P 1771 JUNIUS  \textit{Lett. lii. 267 I am astonished he does not see through your Duplicity.    
%\P 1828 D'ISRAELI  \textit{Chas. I, I. vi. 206 We have here complete evidence of the duplicity of the King's conduct.

%\itembf{2.} lit. The state or quality of being numerically or physically double or twofold: doubleness.

%\P 1589 PUTTENHAM  \textit{Eng. Poesie iii. xviii. (Arb.) 205 Because of the darkenes and duplicitie of his sence.    
%\P 1688 BOYLE  \textit{Final Causes Nat. iv. 163 Nature has furnished men with double parts..where that duplicity may be highly useful.    
%\P 1764 REID  \textit{Inquiry vi. §13. Wks. I. 165/2 We as invariably see two objects unite into one, and, in appearance, lose their duplicity.    
%\P 1863 C. PRITCHARD in  \textit{Smith's Dict. Bible III.
%\P 1375 THE  \textit{duplicity of the two stars must have been apparent.    
%\P 1867-77 G. F. CHAMBERS  \textit{Astron. viii. 769 The duplicity of Saturn's ring.    
%\P 1892 MIVART  \textit{Ess. \& Crit. I. 403 Due to non-appreciation of our duplicity in unity.

%\itembf{3.} Law. The pleading of two (or more) matters in one plea; double pleading.

%\P 1628 COKE  \textit{On Litt. 304 The Plea that containes duplicity or multiplicity of distinct matter to one and the same thing..is not allowable in Law.]    
%\P 1848 WHARTON  \textit{Law Lex., Duplicity. See Double Pleading.

%Hence duˈplicitous a., showing duplicity, deceitful.

%\P 1961 in  \textit{Webster.    
%\P 1966  \textit{New Statesman 11 Mar. 350/3 Peggy Mount, as the duplicitous washerwoman,..subdues her comic extravagance.    
%\P 1969 G. LEFF  \textit{Hist. \& Soc. Theory ii. 44 Whether John was contrite or merely duplicitous in acceding to the barons' demands in
%\P 1215 IS IRRELEVANT  \textit{to the meaning of Magna Carta.



%\end{myenumerate}












\end{description}

